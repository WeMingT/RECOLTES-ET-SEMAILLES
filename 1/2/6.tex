\section{Point de vue et vision}

Mais je reviens à ma propre personne et à mon oeuvre.

Si j'ai excellé dans l'art du mathématicien, c'est moins par l'habileté et la persévérance à résoudre des problèmes légués par mes devanciers, que par cette propension naturelle en moi qui me pousse à voir des questions, visiblement cruciales, que personne n'avait vues, ou à dégager les "bonnes notions" qui manquaient (sans que personne souvent ne s'en soit rendu compte, avant que la notion nouvelle ne soit apparue),ainsi que les "bons énoncés" auxquels personne n'avait songé. Bien souvent, notions et énoncés s'agencent de façon si parfaite, qu'il ne peut y avoir aucun doute dans mon esprit qu'ils ne soient corrects (à des retouches près, tout au plus) - et souvent alors, quand il ne s'agit d'un "travail sur pièces" destiné à publication, je me dispense d'aller plus loin, et de prendre le temps de mettre au point une démonstration qui bien souvent, une fois l'énoncé et son contexte bien vus, ne peut plus guère être qu'une question de "métier", pour ne pas dire de routine. Les choses qui sollicitent l'attention sont innombrables, et il est impossible de suivre jusqu'au bout l'appel de chacune ! Cela n'empêche que les propositions et théorèmes démontrés en bonne et due forme, dans mon oeuvre écrite et publiée, se chiffrent par milliers, et je crois pouvoir dire qu'à très peu d'exceptions près, ils sont tous entrés dans le patrimoine commun des choses communément admises comme "connues" et couramment utilisées un peu partout en mathématique.

Mais plus encore que vers la découverte de questions, de notions et d'énoncés nouveaux, c'est vers celle de points de vue féconds, me conduisant constamment à introduire, et à développer peu ou prou, des thèmes entièrement nouveaux, que me porte mon génie particulier. C'est là, il me semble, ce que j'ai apporté de plus essentiel à la mathématique de mon temps. A vrai dire, ces innombrables questions, notions, énoncés dont je viens de parler, ne prennent pour moi un sens qu'à la lumière d'un tel "point de vue" - vu pour mieux dire, ils en naissent spontanément, avec la force de l'évidence ; à la même façon qu'une lumière (même diffuse) qui surgit dans la nuit noire, semble faire naître du néant ces contours plus ou moins flous ou nets qu'elle nous révèle soudain. Sans cette lumière qui les unit dans un faisceau commun, les dix ou cent ou mille questions, notions, énoncés apparaîtraient comme un monceau hétéroclite et amorphe de "gadgets mentaux", isolés les uns des autres - et non comme les parties d'un Tout qui, pour rester peut-être invisible, se dérobant encore dans les replis de la nuit, n'en est pas moins clairement pressenti.

Le point de vue fécond est celui qui nous révèle, comme autant de parties vivantes d'un même Tout qui les englobe et leur donne un sens, ces questions brûlantes que nul ne sentait, et (comme en réponse peut-être à ces questions) ces notions tellement naturelles que personne pourtant n'avait songé à dégager, et ces énoncés enfin qui semblent couler de source, et que personne certes ne risquait de poser, aussi longtemps que les questions qui les ont suscités, et les notions qui permettent de les formuler, n'étaient pas apparues encore. Plus encore que ce qu'on appelle les "théorèmes-clef" en mathématique, ce sont les points de vue féconds qui sont, dans notre art \footnote{Il n'en est sûrement pas ainsi dans "notre art" seulement, mais (il me semble) dans tout travail de découverte, tout au moins quand celui-ci se situe au niveau de la connaissance intellectuelle.}, les plus puissants outils de découverte - ou plutôt, ce ne sont pas des outils, mais ce sont les yeux même du chercheur qui, passionnément, veut connaître la nature des choses mathématiques.

Ainsi, le point de vue fécond n'est autre que cet "oeil" qui à la fois nous fait découvrir, et nous fait reconnaître l'unité dans la multiplicité de ce qui est découvert. Et cette unité est véritablement la vie même et le souffle qui relie et anime ces choses multiples.

Mais comme son nom même le suggère, un "point de vue" en lui-même reste parcellaire. Il nous révèle un des aspects d'un paysage ou d'un panorama, parmi une multiplicité d'autres également valables, également "réels". C'est dans la mesure où se conjuguent les points de vue complémentaires d'une même réalité, où se multiplient nos "yeux", que le regard pénètre plus avant dans la connaissance des choses. Plus la réalité que nous désirons connaître est riche et complexe, et plus aussi il est important de disposer de plusieurs "yeux" \footnote{Tout point de vue amène à développer un langage qui l'exprime et qui lui est propre. Avoir plusieurs "yeux" ou plusieurs "points de vue" pour appréhender une situation, revient aussi (en mathématique tout au moins) à disposer de plusieurs langages différents pour la cerner.} pour l'appréhender dans toute son ampleur et dans toute sa finesse.

Et il arrive, parfois, qu'un faisceau de points de vue convergents sur un même et vaste paysage, par la vertu de cela en nous apte à saisir l' Un à travers le multiple, donne corps à une chose nouvelle ; à une chose qui dépasse chacune des perspectives partielles, de la même façon qu'un être vivant dépasse chacun de ses membres et de ses organes. Cette chose nouvelle, on peut l'appeler une vision. La vision unit les points de vue déjà connus qui l'incarnent, et elle nous en révèle d'autres jusque là ignorés, tout comme le point de vue fécond fait découvrir et appréhender comme partie d'un même Tout, une multiplicité de questions, de notions et d'énoncés nouveaux.

Pour le dire autrement : la vision est aux poins de vue dont elle paraît issue et qu'elle unit, comme la claire et chaude lumière du jour est aux différentes composantes du spectre solaire. Une vision vaste et profonde est comme une source inépuisable, faite pour inspirer et pour éclairer le travail non seulement de celui en qui elle est née un jour et qui s'est fait son serviteur, mais celui de générations, fascinés peut-être (comme il le fut lui-même) par ces lointaines limites qu'elle nous fait entrevoir...



