\section{Les topos - ou le lit à deux places}

Le point de vue et le langage des faisceaux introduit par Leray nous a amené à regarder les "espaces" et "variétés" en tous genres dans une lumière nouvelle. Ils ne touchaient pas, pourtant, à la notion même d'espace, se contentant de nous faire appréhender plus finement, avec des yeux nouveaux, ces traditionnels "espaces", déjà familiers à tous. Or, il s'est avéré que cette notion d'espace est inadéquate pour rendre compte des "invariants topologiques" les plus essentiels qui expriment la "forme" des variétés algébriques "abstraites" (comme celles auxquelles s'appliquent les conjectures de Weil), voire celle des "schémas" généraux (généralisant les anciennes variétés). Pour les "épousailles" attendues, "au nombre et de la grandeur", c'était comme un lit décidément étriqué, où l'un seulement des futurs conjoints (à savoir, l'épousée) pouvait à la rigueur trouver à se nicher tant bien que mal, mais jamais des deux à la fois! Le "principe nouveau" qui restait à trouver, pour consommer les épousailles promises par des fées propices, ce n'était autre aussi que ce "lit" spacieux qui manquait aux futurs époux, sans que personne jusque là s'en soit seulement aperçu...

Ce "lit à deux places" est apparu (comme par un coup de baguette magique...) avec l'idée du topos. Cette idée englobe, dans une intuition topologique commune, aussi bien les traditionnels espaces (topologiques), incarnant le monde de la grandeur continue, que les (soi-disant) "espaces" (ou "variétés") des géomètres algébristes abstraits impénitents, ainsi que d'innombrables autres types de structures, qui jusque là avaient semblé rivées irrémédiablement au "monde arithmétique" des agrégats "discontinus" ou "discrets".

C'est le point de vue des faisceaux qui a été le guide silencieux et sûr, la clef efficace (et nullement secrète), me menant sans atermoiements ni détours vers la chambre nuptiale au vaste lit conjugal. Un lit si vaste en effet (telle une vaste et paisible rivière très profonde...), que
\begin{quote}
    ``tous les chevaux du roi\\
    y pourraient boire ensemble...''
    \end{quote}
- comme nous le dit un vieil air que sûrement tu as dû chanter toi aussi, ou du moins l'entendre chanter. Et celui qui a été le premier à le chanter a mieux senti la beauté secrète et la force paisible du topos, qu'aucun de mes savants élèves et amis d'antan...

La clef a été la même, tant dans l'approche initiale et provisoire (via la notion très commode, mais non intrinsèque du "site"), que dans celle du topos. C'est l'idée du topos que je voudrais essayer à présent de décrire.

Considérons l'ensemble formé de tous les faisceaux sur un espace (topologique) donné, ou, si on veut, cet arsenal prodigieux formé de tous ces "mètres" servant à l'arpenter \footnote{(A l'intention du mathématicien) A vrai dire, il s'agit ici des faisceaux d'ensembles, et non des faisceaux abéliens, introduits par Leray comme coeffi cients les plus généraux pour former des "groupes de cohomologie". Je crois d'ailleurs être le premier à avoir travaillé systématiquement avec les faisceaux d'ensembles (à partir de 1955, dans mon article "A général theory of fi bre spaces with structure sheaf" à l'Université de Kansas).}. Nous considérons cet "ensemble" ou "arsenal" comme muni de sa structure la plus évidente, laquelle y apparaît, si on peut dire, "à vue de nez"; à savoir, une structure dite de "catégorie". (Que le lecteur non mathématicien ne se trouble pas, de ne pas connaître le sens technique de ce terme. Il n'en aura nul besoin pour la suite.) C'est cette sorte de "superstructure d'arpentage", appelée "catégorie des faisceaux" (sur l'espace envisagé), qui sera dorénavant considérée comme "incarnant" ce qui est le plus essentiel à l'espace. C'est bien là chose licite (pour le "bon sens mathématique"), car il se trouve qu'on peut "reconstituer" de toutes pièces un espace topologique \footnote{(A l'intention du mathématicien) A strictement parler, ceci n'est vrai que pour des espaces dits "sobres". Ceux-ci comprennent cependant la quasi-totalité des espaces qu'on rencontre communément, et notamment tous les espaces "séparés" chers aux analystes.} en termes de cette "catégorie de faisceaux" (ou de cet arsenal d'arpentage) associée. (De le vérifier est un simple exercice - une fois la question posée, certes...) Il n'en faut pas plus pour être assuré que (s'il nous convient pour une raison ou une autre) nous pouvons désormais "oublier" l'espace initial, pour ne plus retenir et ne nous servir que de la "catégorie" (ou de l' "arsenal") associée, laquelle sera considérée comme l'incarnation la plus adéquate de la "structure topologique" (ou "spatiale") qu'il s'agit d'exprimer.

Comme si souvent en mathématique, nous avons réussi ici (grâce à l'idée cruciale de "faisceau", ou de "mètre cohomologique") à exprimer une certaine notion (celle d' "espace" en l'occurrence) en termes d'une autre (celle de "catégorie"). A chaque fois, la découverte d'une telle traduction d'une notion (exprimant un certain type de situations) en termes d'une autre (correspondant à un autre type de situations), enrichit notre compréhension et de l'une et de l'autre notion, par la confluence inattendue des intuitions spécifiques qui se rapportent soit à l'une, soit à l'autre. Ainsi, une situation de nature "topologique" (incarnée par un espace donné) se trouve ici traduite par une situation de nature "algébrique" (incarnée par une "catégorie"); ou, si on veut, le "continu" incarné par l'espace, se trouve "traduit" ou "exprimé" par la structure de catégorie, de nature "algébrique" (et jusque là perçue comme étant de nature essentiellement "discontinue" ou "discrète").

Mais ici, il y a plus. La première de ces notions, celle d'espace, nous était apparue comme une notion en quelque sorte "maximale" - une notion si générale déjà, qu'on imagine mal comment en trouver encore une extension qui reste "raisonnable". Par contre, il se trouve que de l'autre côté du miroir \footnote{Le "miroir" dont il est question ici, comme dans Alice au pays des merveilles, est celui qui donne comme "image" d'un espace, placé devant lui, la "catégorie" associée, considérée comme une sorte de "double" de l'espace, "de l'autre côté du miroir"...}, ces "catégories" (ou "arsenaux") sur lesquels on tombe, en partant d'espaces topologiques, sont de nature très particulière. Elles jouissent en effet d'un ensemble de propriétés fortement typées \footnote{(A l'intention du mathématicien) Il s'agit ici surtout de propriétés que j'ai introduites en théorie des catégories sous le nom de "propriétés d'exactitude" (en même temps que la notion catégorique moderne de "limites" inductives et projectives générales). Voir "Sur quelques points d'algèbre homologique", Tohoku math, journal, 1957 (p. 119-221).}, qui les font s'apparenter à des sortes de "pastiches" de la plus simple imaginable d'entre elles - celle qu'on obtient en partant d'un espace réduit à un seul point. Ceci dit, un "espace nouveau style (ou topos), généralisant les espaces topologiques traditionnels, sera décrit tout simplement comme une "catégorie" qui, sans provenir forcément d'un espace ordinaire, possède néanmoins toutes ces bonnes propriétés (explicitement désignées une fois pour toutes, bien sûr) d'une telle "catégorie de faisceaux".

\begin{center}
    * \quad * \\
    *
    \end{center}

Voici donc l'idée nouvelle. Son apparition peut être vue comme une conséquence de cette observation, quasiment enfantine à vrai dire, que ce qui compte vraiment dans un espace topologique, ce ne sont nullement ses "points" ou ses sous-ensembles de points \footnote{Ainsi, on peut construire des topos très "gros", qui n'ont qu'un seul "point", ou même pas de "points" du tout !}, et les relations de proximité etc entre ceux-ci, mais que ce sont les faisceaux sur cet espace, et la catégorie qu'ils forment. Je n'ai fait, en somme, que mener vers sa conséquence ultime l'idée initiale de Leray - et ceci fait, franchir le pas.

Comme l'idée même des faisceaux (due à Leray), ou celle des schémas, comme toute "grande idée" qui vient bousculer une vision invétérée des choses, celle des topos a de quoi déconcerter par son caractère de naturel, d' "évidence", par sa simplicité (à la limite, dirait-on, du naïf ou du simpliste, voire du "bébête" par cette qualité particulière qui nous fait nous écrier si souvent : "Oh, ce n'est que ça! ", d'un ton mi-déçu, mi-envieux ; avec en plus, peut-être, ce sous entendu du "farfelu", du "pas sérieux", qu'on réserve souvent à tout ce qui déroute par un excès de simplicité imprévue. A ce qui vient nous rappeler, peut-être, les jours depuis longtemps enfouis et reniés de notre enfance...

