\section{L'importance d'être seul}

Quand j'ai finalement pris contact avec le monde mathématique à Paris, un ou deux ans plus tard, j'ai fini par y apprendre, entre beaucoup d'autres choses, que le travail que j'avais fait dans mon coin avec les moyens du bord, était (à peu de choses près) ce qui était bien connu de "tout le monde", sous le nom de théorie de la mesure et de l'intégrale de Lebesgue Aux yeux des deux ou trois saines à qui j'ai parlé de ce travail (voire même, montré un manuscrit), c'était un peu comme si j'avais simplement perdu mon temps, à refaire du "déjà connu". Je ne me rappelle pas avoir été déçu, d'ailleurs. A ce moment-là, l'idée de recueillir un "crédit", ou ne serait-ce qu'une approbation ou simplement l'intérêt d'autrui, pour le travail que je faisais, devait être encore étrangère à mon esprit. Sans compter que mon énergie était bien assez accaparée à me familiariser avec un milieu complètement différent, et surtout, à apprendre ce qui était considéré à Paris comme le B.A.BA du mathématicien \footnote{Je fais un court récit de cette époque de transition un peu rude, dans la première partie de Récoltes et Semailles (ReS I), dans la section "L'étranger bienvenu" (n° 9).}.

Pourtant, en repensant maintenant à ces trois années, je me rends compte qu'elles n'étaient nullement gaspillées. Sans même le savoir, j'ai appris alors dans la solitude ce qui fait l'essentiel du métier de mathématicien - ce qu'aucun maître ne peut véritablement enseigner. Sans avoir eu jamais à me le dire, sans avoir eu à rencontrer quelqu'un avec qui partager ma soif de comprendre, je savais pourtant, "par mes tripes" je dirais, que j'étais un mathématicien : quelqu'un qui "fait" des maths, au plein sens du terme - comme on "fait" l'amour. La mathématique était devenue pour moi une maîtresse toujours accueillante à mon désir. Ces années de solitude ont posé le fondement d'une confiance qui n'a jamais été ébranlée - ni par la découverte (débarquant à Paris à l'âge de vingt ans) de toute l'étendue de mon ignorance et de l'immensité de ce qu'il me fallait apprendre ; ni (plus de vingt ans plus tard) par les épisodes mouvementés de mon départ sans retour du monde mathématique ; ni, en ces dernières années, par les épisodes souvent assez dingues d'un certain "Enterrement". (anticipé et sans bavures) de ma personne et de mon oeuvre, orchestré par mes plus proches compagnons d'antan...

Pour le dire autrement : j'ai appris, en ces années cruciales, à être seul\footnote{Cette formulation est quelque peu impropre. Je n'ai jamais eu à "apprendre à être seul", pour la simple raison que je n'ai jamais désappris, au cours de mon enfance, cette capacité innée qui était en moi à ma naissance, comme elle est en chacun. Mais ces trois ans de travail solitaire, où j'ai pu donner ma mesure à moi-même, suivant les critères d'exigence spontanée qui étaient les miens, ont confirmé et reposé en moi, dans ma relation cette fois au travail mathématique, une assise de confiance et de tranquille assurance, qui ne devait rien aux consensus et aux modes qui font loi. J'ai occasion d'y faire allusion à nouveau dans la note "Racines et solitude" (ReS IV, n$^{\circ}$ 171$_3$, notamment p. 1080).}. J'entends par là : aborder par mes propres lumières les choses que je veux connaître, plutôt que de me fier aux idées et aux consensus, exprimés ou tacites, qui me viendraient d'un groupe plus ou moins étendu dont je me sentirais un membre, ou qui pour toute autre raison serait investi pour moi d'autorité. Des consensus muets m'avaient dit, au lycée comme à l'université, qu'il n'y avait pas lieu de se poser de question sur la notion même de "volume", présentée comme "bien connue", "évidente", "sans problème". J'avais passé outre, comme chose allant de soi - tout comme Lebesgue, quelques décennies plus tôt, avait dû passer outre. C'est dans cet acte de "passer outre", d'être soi-même en somme et non pas simplement l'expression des consensus qui font loi, de ne pas rester enfermé à l'intérieur du cercle impératif qu'ils nous fixent - c'est avant tout dans cet acte solitaire que se trouve "la création". Tout le reste vient par surcroît.

Par la suite, j'ai eu l'occasion, dans ce monde des mathématiciens qui m'accueillait, de rencontrer bien des gens, aussi bien des aînés que des jeunes gens plus ou moins de mon âge, qui visiblement étaient beaucoup plus brillants, beaucoup plus "doués" que moi. Je les admirais pour la facilité avec laquelle ils apprenaient, comme en se jouant, des notions nouvelles, et jonglaient avec comme s'ils les connaissaient depuis leur berceau - alors que je me sentais lourd et pataud, me frayant un chemin péniblement, comme une taupe, à travers une montagne informe de choses qu'il était important (m'assurait-on) que j'apprenne, et dont je me sentais incapable de saisir les tenants et les aboutissants. En fait, je n'avais rien de l'étudiant brillant, passant haut la main les concours prestigieux, assimilant en un tournemain des programmes prohibitifs.

La plupart de mes camarades plus brillants sont d'ailleurs devenus des mathématiciens compétents et réputés. Pourtant, avec le recul de trente ou trente-cinq ans, je vois qu'ils n'ont pas laissé sur la mathématique de notre temps une empreinte vraiment profonde. Ils ont fait des choses, des belles choses parfois, dans un contexte déjà tout fait, auquel ils n'auraient pas songé à toucher. Ils sont restés prisonniers sans le savoir de ces cercles invisibles et impérieux, qui délimitent un Univers dans un milieu et à une époque donnée. Pour les franchir, il aurait fallu qu'ils retrouvent en eux cette capacité qui était leur à leur naissance, tout comme elle était mienne : la capacité d'être seul.

Le petit enfant, lui, n'a aucune difficulté à être seul. Il est solitaire par nature, même si la compagnie occasionnelle ne lui déplaît pas et qu'il sait réclamer la totosse de maman, quand c'est l'heure de boire. Et il sait bien, sans avoir eu à se le dire, que la totosse est pour lui, et qu'il sait boire. Mais souvent, nous avons perdu le contact avec cet enfant en nous. Et constamment nous passons à côté du meilleur, sans daigner le voir...

Si dans Récoltes et Semailles je m'adresse à quelqu'un d'autre encore qu'à moi-même, ce n'est pas à un "public". Je m’y adresse à toi qui me lis comme à une personne, et à une personne seule. C'est à celui en toi qui sait être seul, à l'enfant, que je voudrais parler, et à personne d'autre. Il est loin souvent l'enfant, je le sais bien. Il en a vu de toutes les couleurs et depuis belle lurette. Il s'est planqué Dieu sait où, et c'est pas facile, souvent, d'arriver jusqu'à lui. On jurerait qu'il est mort depuis toujours, qu'il n'a jamais existé plutôt et pourtant, je suis sûr qu'il est là quelque part, et bien en vie.

Et je sais aussi quel est le signe que je suis entendu. C'est quand, au delà de toutes les différences de culture et de destin, ce que je dis de ma personne et de ma vie trouve en toi écho et résonance ; quand tu y retrouves aussi ta propre vie, ta propre expérience de toi-même, sous un jour peut-être auquel tu n'avais pas accordé attention jusque là. Il ne s'agit pas d'une "identification", à quelque chose ou à quelqu'un d'éloigné de toi. Mais peut-être, un peu, que tu redécouvres ta propre vie, ce qui est le plus proche de toi, à travers la redécouverte que je fais de la mienne, au fil des pages dans Récoltes et Semailles et jusque dans ces pages que je suis en train d'écrire aujourd'hui même.



