\section{Les motifs - ou le coeur dans le coeur}

Le thème du topos est issu de celui des schémas, l'année même où sont apparus les schémas - mais en étendue il dépasse largement le thème-mère. C'est le thème du topos, et non celui des schémas, qui est ce "lit", ou cette "rivière profonde", où viennent s'épouser la géométrie et l'algèbre, la topologie et l'arithmétique, la logique mathématique et la théorie des catégories, le monde du continu et celui des structures "discontinues" ou "discrètes". Si le thème des schémas est comme le coeur de la géométrie nouvelle, le thème du topos en est l'enveloppe, ou la demeure. Il est ce que j'ai conçu de plus vaste, pour saisir avec finesse, par un même langage riche en résonances géométriques, une "essence" commune à des situations des plus éloignées les unes des autres, provenant de telle région ou de telle autre du vaste univers des choses mathématiques.

Ce thème du topos est très loin pourtant d'avoir connu la fortune de celui des schémas. Je m'exprime à ce sujet en diverses occasions dans Récoltes et Semailles, et ce n'est pas le lieu ici de m'attarder sur les vicissitudes étranges qui ont frappé cette notion. Deux des maîtres-thèmes de la géométrie nouvelle sont pourtant issus de celui du topos, deux "théories cohomologiques" complémentaires, conçues l'une et l'autre aux fins de fournir une approche vers les conjectures de Weil : le thème étale (ou " $\ell$-adique"), et le thème cristallin. Le premier s'est concrétisé entre mes mains en l'outil cohomologique $\ell$-adique, qui dès à présent apparaît comme un des plus puissants outils mathématiques du siècle. Quant au thème cristallin, réduit après mon départ à une existence quasi-occulte, il a finalement été exhumé (sous la pression des besoins) en juin 1981, sous les feux de la rampe et sous un nom d'emprunt, dans des circonstances plus étranges encore que celles autour des topos.

L'outil cohomologique $\ell$-adique a été, comme prévu, l'outil essentiel pour établir les conjectures de Weil. J'en ai démontré moi-même un bon paquet, et le dernier pas a été accompli avec maestria, trois ans après mon départ, par Pierre Deligne, le plus brillant de mes élèves "cohomologistes".

J'avais d'ailleurs dégagé, vers l'année 1968, une version plus forte et surtout, plus "géométrique" des conjectures de Weil. Celles-ci restaient "entachées" (si on peut dire !) d'un aspect "arithmétique" apparemment irréductible, alors pourtant que l'esprit même de ces conjectures est d'exprimer et de saisir "l'arithmétique" (ou "le discret") par la médiation du "géométrique" (ou du "continu") \footnote{(A l'intention du mathématicien) Les conjectures de Weil sont subordonnées à des hypothèses de nature "arithmétique", du fait notamment que les variétés envisagées doivent être définies sur un corps fini. Du point de vue du formalisme cohomologique, cela conduit à donner une place à part à l'endomorphisme de Frobenius associé à une telle situation. Dans mon approche, les propriétés cruciales (type "théorème de l'index généralisé") concernent les correspondances algébriques quelconques, et ne font aucune hypothèse de nature arithmétique sur un corps de base préalablement donné.}. En ce sens, la version des conjectures que j'avais dégagée me paraît plus "fidèle" que celle de Weil lui-même à la "philosophie de Weil" - à cette philosophie non écrite et rarement dite, qui a été peut-être la principale motivation tacite dans l'extraordinaire essor de la géométrie au cours des quatre décennies écoulées \footnote{Il y a eu cependant, après mon départ en 1970, un mouvement de réaction très nette, lequel s'est concrétisé par une situation de stagnation relative, que j'ai occasion plus d'une fois d'évoquer dans les lignes de Récoltes et Semailles.}. Ma reformulation a consisté, pour l'essentiel, à dégager une sorte de "quintessence" de ce qui devait rester valable, dans le cadre des variétés algébriques dites "abstraites", de la classique "théorie de Hodge", valable pour les variétés algébriques "ordinaires" \footnote{"Ordinaire" signifie ici : "définie sur le corps des complexes". La théorie de Hodge (dite "des intégrales harmoniques") était la plus puissante des théories cohomologiques connues dans le contexte des variétés algébriques complexes.}. J'ai appelé "conjectures standard" (pour les cycles algébriques) cette nouvelle version, entièrement géométrique, des fameuses conjectures.

Dans mon esprit, c'était là un nouveau pas, après le développement de l'outil cohomologique $\ell$-adique, en direction de ces conjectures. Mais en même temps et surtout, c'était aussi un des principes d'approche possibles vers ce qui m'apparaît encore comme le thème le plus profond que j'aie introduit en mathématique \footnote{C'est le thème le plus profond, tout au moins dans la période "publique" de mon activité de mathématicien, entre 1950 et 1969, c'est-à-dire jusqu'au moment de mon départ de la scène mathématique. Je considère le thème de la géométrie algébrique anabélienne et de la théorie de Galois-Teichmüller, développé à partir de 1977, comme étant d'une profondeur comparable.} : celui des motifs (lui-même né du "thème cohomologique $\ell$-adique"). Ce thème est comme le coeur ou l'âme, la partie la plus cachée, la mieux dérobée au regard, du thème schématique, qui lui-même est au coeur de la vision nouvelle. Et les quelques phénomènes-clef dégagés dans les conjectures standard \footnote{(A l'intention du lecteur géomètre algébriste) Il y a lieu, éventuellement, de reformuler ces conjectures. Pour des commentaires plus circonstanciés, voir "Le tour des chantiers" (ReS IV note n ${ }^{\circ}$ 178, p. 1215-1216) et la note de b. de p. p 769 dans "Conviction et connaissance" (ReS III, note $\mathrm{n}^{\circ} 162$).} peuvent être vus comme formant une sorte de quintessence ultime du thème motivique, comme le "souffle" vital de ce thème subtil entre tous, de ce "coeur dans le coeur" de la géométrie nouvelle.

Voici en gros de quoi il s'agit. Nous avons vu, pour un nombre premier $p$ donné, l'importance (en vue notamment des conjectures de Weil) de savoir construire des "théories cohomologiques" pour les "variétés (algébriques) de caractéristique $p$ ". Or, le fameux "outil cohomologique $\ell$-adique" fournit justement une telle théorie, et même une infinité de théories cohomologiques différentes, à savoir une associée à tout nombre premier différent de la caractéristique $p$. Il y a là encore visiblement, une "théorie qui manque", qui correspondrait au cas d'un $\ell$ qui serait égal à $p$. Pour y pourvoir, j'ai imaginé tout exprès une autre théorie cohomologique encore à laquelle il a été déjà fait allusion tantôt), dite "cohomologie cristalline". D'ailleurs, dans le cas important où $p$ est infini, on dispose de trois autres théories cohomologiques encore \footnote{(A l’intention du lecteur mathématicien) Ces théories correspondent respectivement à la cohomologie de Betti (définie par voie transcendante, à l'aide d'un plongement du corps de base dans le corps des complexes), à la cohomologie de Hodge (définie par Serre) et à la cohomologie de De Rham (définie par moi), ces deux dernières remontant déjà aux années cinquante (et celle de Betti, au siècle dernier).} - et rien ne prouve qu'on ne sera conduit, tôt ou tard, à introduire encore de nouvelles théories cohomologiques, ayant des propriétés formelles toutes analogues. Contrairement à ce qui se passait en topologie ordinaire, on se trouve donc placé là devant une abondance déconcertante de théories cohomologiques différentes. On avait l'impression très nette qu'en un sens qui restait d'abord assez flou, toutes ces théories devaient "revenir au même", qu'elles "donnaient les mêmes résultats" \footnote{(A l'intention du lecteur mathématicien) Par exemple, si $f$ est un endomorphisme de la variété algébrique $X$, induisant un endomorphisme de l'espace de cohomologie $H^{i}(X)$, le "polynôme caractéristique" de ce dernier devait être à coefficients entiers, ne dépendant pas de la théorie cohomologique particulière choisie (par exemple : $\ell$-adique, pour $\ell$ variable). Itou pour des correspondances algébriques générales, quand $X$ est supposée propre et lisse. La triste vérité (et qui donne une idée de l'état de lamentable abandon de la théorie cohomologique des variétés algébriques en caractéristique $p>0$, depuis mon départ), c'est que la chose n'est toujours pas démontrée à l'heure actuelle, même dans le cas particulier où $X$ est une surface projective et lisse et $i=2$. En fait, à ma connaissance, personne après mon départ n'a encore daigné s'intéresser à cette question cruciale, typique de celles qui apparaissent comme subordonnées aux conjectures standard. Le décret de la mode, c'est que le seul endomorphisme digne d'attention est l'endomorphisme de Frobenius (lequel a pu être traité à part par Deligne, par les moyens du bord...).
}. C’est pour parvenir à exprimer cette intuition de "parenté" entre théories cohomologiques différentes, que j'ai dégagé la notion de "motif" associé à une variété algébrique. Par ce terme, j'entends suggérer qu'il s'agit du "motif commun" (ou de la "raison commune") sous-jacent à cette multitude d'invariants cohomologiques différents associés à la variété, à l'aide de la multitude des toutes les théories cohomologiques possibles a priori. Ces différentes théories cohomologiques seraient comme autant de développements thématiques différents, chacun dans le "tempo", dans la "clef" et dans le "mode" ("majeur" ou "mineur") qui lui est propre, d'un même "motif de base" (appelé "théorie cohomologique motivique"), lequel serait en même temps la plus fondamentale, ou la plus "fine", de toutes ces "incarnations" thématiques différentes (c'est-à-dire, de toutes ces théories cohomologiques possibles). Ainsi, le motif associé à une variété algébrique constituerait l'invariant cohomologique "ultime", "par excellence", dont tous les autres (associés aux différentes théories cohomologiques possibles) se déduiraient, comme autant d' "incarnations" musicales, ou de "réalisations" différentes. Toutes les propriétés essentielles de "la cohomologie" de la variété se "liraient" (ou s' "entendraient") déjà sur le motif correspondant, de sorte que les propriétés et structures familières sur les invariants cohomologiques particularisés ( $\ell$-adique ou cristallins, par exemple), seraient simplement le fidèle reflet des propriétés et structures internes au motif \footnote{(A l'intention du lecteur mathématicien) Une autre façon de voir la catégorie des motifs sur un corps $k$, c'est de la visualiser comme une sorte de "catégorie abélienne enveloppante" de la catégorie des schémas séparés de type fini sur $k$. Le motif associé à un tel schéma $X$ (ou "cohomologie motivique de $X$ ", que je note $H_{\text {mot}}^{*}(X)$ ) apparaît ainsi comme une sorte de "avatar" abélianisé de $X$. La chose cruciale ici, c'est que, tout comme une variété algébrique $X$ est susceptible de "variation continue" (sa classe d'isomorphie dépend donc de "paramètres" continus, ou "modules"), le motif associé à $X$, ou plus généralement, un motif "variable", est lui aussi susceptible de variation continue. C'est là un aspect de la cohomologie motivique, qui est en contraste frappant avec ce qui se passe pour tous les invariants cohomologiques classiques, y compris les invariants $\ell$-adique, à la seule exception de la cohomologie de Hodge des variétés algébriques complexes.

Ceci donne une idée à quel point la "cohomologie motivique" est un invariant plus fin, cernant de façon beaucoup plus
serrée la "forme arithmétique" (et plus hasardeuse cette expression soit-elle) d'une variété $X$, que les invariants purement topologiques.
Dans ma vision des motifs, ceux-ci constituent une sorte de "cordon" très caché et très délicat, reliant les propriétés algéro-
géométriques d'une variété $X$, aux propriétés de nature "arithmétique" incarnées par son motif. Ce dernier peut être
considéré comme un objet de nature "géométrique" dans son esprit même, mais où les propriétés "arithmétiques" subordonnées
à la géométrie ne servent, pour ainsi dire, "mises à nu".

Ainsi, le motif m'apparaît comme le plus profond "invariant de la forme" qu'on ait su associer jusqu'à présent à une variété
algébrique, mis à part son "groupe fondamental motivique" $\pi_1$ et ce l'autre invariant récemment pour moi comme les "ombres"
d'un "type d'homotopie motivique" qui resterait à décrire (et sur lequel je dis quelques mots en passant dans la note "Le tour
des chantiers - ou outils et vision" (ReS IV, n° 178, voir chapitre 5 Motifs), et notamment page 1214). C'est ce dernier objet
qui me semble devoir être l'incarnation la plus parfaite de l'élusive intuition de "forme arithmétique" (ou "motivique") d'une
variété algébrique quelconque.}.

C'est là, exprimé dans le langage non technique d'une métaphore musicale, la quintessence d'une idée d'une simplicité enfantine encore, délicate et audacieuse à la fois. J'ai développé cette idée, en marge des tâches de fondements que je considérais plus urgentes, sous le nom de "théorie des motifs" ou de "philosophie (ou "yoga") des motifs", tout au long des années 1963-69. C'est une théorie d'une richesse structurale fascinante, dont une grande partie est restée encore conjecturale \footnote{J'ai expliqué ma vision des motifs à qui voulait l'entendre, tout au long de ces années, sans prendre la peine de rien publier à ce sujet noir sur blanc (ne manquant pas d'autres tâches au service de tous). Cela a permis plus tard à certains de mes élèves de piller plus à l'aise, sous l'oeil attendri de l'ensemble de mes anciens amis, bien au courant de la situation. (Voir note de b. de p. qui suit.)}.

Je m'exprime à diverses reprises dans Récoltes et Semailles au sujet de ce "yoga des motifs", qui me tient particulièrement à coeur. Ce n'est pas le lieu de revenir ici sur ce que j'en dis ailleurs. Qu'il me suffise de dire que les "conjectures standard" découlent le plus naturellement du monde de ce yoga des motifs. En même temps elles fournissent un principe d'approche pour une des constructions en forme possibles de la notion de motif.

Ces conjectures m'apparaissent, et m'apparaissent aujourd'hui encore, comme l'une des deux questions les plus fondamentales qui se posent en géométrie algébrique. Ni cette question, ni l'autre question toute aussi cruciale (celle dite de la "résolution des singularités") n'est encore résolue à l'heure actuelle. Mais alors que la deuxième de ces questions apparaît, aujourd'hui comme il y a cent ans, comme une question prestigieuse et redoutable, celle que j'ai eu l'honneur de dégager s'est vue classer par les péremptoires décrets de la mode (dès les années qui ont suivi mon départ de la scène mathématique, et tout comme le thème motivique lui-même \footnote{En fait, ce thème a été exhumé en 1982 (un an après le thème cristallin), sous son nom d'origine cette fois (et sous une forme étriquée, dans le seul cas d'un corps de base de caractéristique nulle), sans que le nom de l'ouvrier ne soit prononcé. C'est là un exemple parmi un nombre d'autres, d'une notion ou d'un thème enterré aux lendemains de mon départ comme des fantasmagories grothendieckiennes, pour être exhumés l'un après l'autre par certains de mes élèves au cours des dix ou quinze années suivantes, avec une fi erté modeste et (est-il besoin encore de le préciser) sans mention de l'ouvrier...}) comme aimable fumisterie grothendieckienne. Mais encore une fois j'anticipe...

