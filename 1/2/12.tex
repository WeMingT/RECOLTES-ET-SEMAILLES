\section{La topologie - ou l'arpentage des brumes}

L'idée novatrice du "schéma", nous venons de le voir, est celle qui permet de relier entre elles les différentes "géométries" associées aux différents nombres premiers (ou différentes "caractéristiques"). Ces géométries, pourtant, restaient encore chacune de nature essentiellement "discrète" ou "discontinue", en contraste avec la géométrie traditionnelle léguée par les siècles passés (et remontant à Euclide). Les nouvelles idées introduites par Zariski et par Serre restituaient dans une certaine mesure, pour ces géométries, une "dimension" de continuité, héritée aussitôt par la "géométrie schématique" qui venait d'apparaître, aux fins de les unir. Mais pour ce qui était des "fantastiques conjectures" (de Weil), on était très loin du compte. Ces "topologies de Zariski" étaient, de ce point de vue, à tel point grossières, que c'était quasiment comme si on en était resté encore au stade des "agrégats discrets". Ce qui manquait, visiblement, était quelque principe nouveau, qui permette de relier ces objets géométriques (ou "variétés", ou "schémas") aux "espaces" (topologiques) habituels, ou "bon teint" ; ceux, disons, dont les "points" apparaissent comme nettement séparés les uns des autres, alors que dans les espaces-sans-foi-ni-loi introduits par Zariski, les points ont une fâcheuse tendance à s'agglutiner les uns aux autres...

C'était l'apparition d'un tel "principe nouveau" décidément, et rien de moins, qui pouvait faire se consommer ces "épousailles du nombre et de la grandeur" ou de la "géométrie du discontinu" avec celle du "continu", dont un premier pressentiment se dégageait des conjectures de Weil.

La notion d' "espace" est sans doute une des plus anciennes en mathématique. Elle est si fondamentale dans notre appréhension "géométrique" du monde, qu'elle est restée plus ou moins tacite pendant plus de deux millénaires. C'est au cours du siècle écoulé seulement que cette notion a fini, progressivement, par se détacher de l'emprise tyrannique de la perception immédiate (d'un seul et même "espace" qui nous entoure), et de sa théorisation traditionnelle ("euclidienne"), pour acquérir son autonomie et sa dynamique propres. De nos jours, elle fait partie des quelques notions les plus universellement et les plus couramment utilisées en mathématique, familière sans doute à tout mathématicien sans exception. Notion protiforme d'ailleurs s'il en fut, aux cents et mille visages, selon le type de structures qu'on incorpore à ces espaces, depuis les plus riches de toutes (telles les vénérables structures "euclidiennes", ou les structures "affines" et "projectives", ou encore les structures "algébriques" des "variétés" de même nom, qui les généralisent et qui assouplissent) jusqu'aux plus dépouillées : celles où tout élément d'information "quantitatif" quel qu'il soit semble disparu sans retour, et où ne subsistent plus que la quintessence qualitative de la notion de "proximité" ou de celle de "limite" \footnote{Parlant de la notion de "limite", c'est surtout à celle de "passage à la limite" que je pense ici, plutôt qu'à celle (plus familière au non mathématicien) de "frontière".}, et la version la plus élusive de l'intuition de la forme (dite "topologique"). La plus dépouillée de toutes parmi ces notions, celle qui jusqu'à présent, au cours du demi-siècle écoulé, avait tenu lieu d'une sorte de vaste giron conceptuel commun pour englober toutes les autres, était celle d'espace topologique. L'étude de ces espaces constitue l'une des branches les plus fascinantes, les plus vivaces de la géométrie : la topologie.

Si élusif que puisse paraître de prime abord cette structure "de qualité pure" incarnée par un "espace" (dit, "topologique"), en l'absence de toute donnée de nature quantitative (telle la distance entre deux points, notamment) qui nous permette de nous raccrocher à quelque intuition familière de "grandeur" ou de "petitesse", on est pourtant arrivé, au cours du siècle écoulé, à cerner finement ces espaces dans les mailles serrées et souples d'un langage soigneusement "taille sur pièces". Mieux encore, on a inventé et fabrique de toutes pièces des sortes de "mètres" ou de "toises" pour servir tout de même, envers et contre tout, à attacher des sortes de "mesures" (appelées "invariants topologiques") à ces "espaces" tentaculaires qui semblaient se dérober, telles des brumes insaisissables, à toute tentative de mensuration. Il est vrai que la plupart de ces invariants, et les plus essentiels, sont de nature plus subtile qu'un simple "nombre" ou une "grandeur" - ce sont plutôt eux-mêmes des structures mathématiques plus ou moins délicates, attachées (à l'aide de constructions plus ou moins sophistiquées) à l'espace envisagé. L'un des plus anciens et des plus cruciaux de ces invariants, introduits déjà au siècle dernier (par le mathématicien italien Betti), est formé des différents "groupes" (ou "espaces") dits de "cohomologie", associés à l'espace \footnote{A vrai dire, les invariants introduits par Betti étaient les invariants d'homologie. La cohomologie en constitue une version plus ou moins équivalente, "duale", introduite beaucoup plus tard. Cet aspect a acquis une prééminence sur l'aspect initial, "homologique ", surtout (sans doute) à la suite de l'introduction, par Jean Leray, du point de vue des faisceaux, dont il est question plus bas. Au point de vue technique, on peut dire qu'une grande partie de mon oeuvre de géomètre a consisté à dégager, et à développer plus ou moins loin, les théories cohomologiques qui manquaient, pour les espaces et variétés en tous genres et surtout, pour les "variétés algébriques" et les schémas. Chemin faisant, j'ai été amené aussi à réinterpréter les invariants homologiques traditionnels en termes cohomologiques, et par là-même, à les faire voir dans un jour entièrement nouveau.

Il y a de nombreux autres "invariants topologiques" qui ont été introduits par les topologues, pour cerner tel type de propriétés ou tel autre des espaces topologiques. A part la "dimension" d'un espace, et les invariants (co)homologiques, les premiers autres invariants sont les "groupes d'homotopie". J'en ai introduit un autre en 1957, le groupe (dit "de Grothendieck") $K(X)$, qui a connu aussitôt une grande fortune, et dont l'importance (tant en topologie qu'en arithmétique) ne cesse de se confirmer.

Une foule de nouveaux invariants, de nature plus subtile que les invariants actuellement connus et utilisés, mais que je sens fondamentaux, sont prévus dans mon programme de "topologie modérée" (dont une esquisse très sommaire se trouve dans l' "Esquisse d'un Programme", à paraître dans le volume 4 des Réflexions). Ce programme est basé sur la notion de "théorie modérée" ou "d'espace modéré", qui constitue, un peu comme celle de topos, une (deuxième) "métamorphose de la notion d'espace". Elle est bien plus évidente (me semble-t-il) et moins profonde que cette dernière. Je prévois que ses retombées immédiates sur la topologie "proprement dite" vont être pourtant nettement plus percutantes, et qu'elle va transformer de fond en comble le "métier" de topologue géomètre, par une transformation profonde du contexte conceptuel dans lequel il travaille. (Comme cela a été le cas aussi en géométrie algébrique avec l'introduction du point de vue des schémas.) J'ai d'ailleurs envoyé mon "Esquisse" à plusieurs de mes anciens amis et illustres topologues, mais il ne semble pas qu'elle ait eu le don d'en intéresser aucun.}.Ce sont eux qui interviennent (surtout entre les lignes", il est vrai) dans les conjectures de Weil, qui en font la "raison d'être" profonde et qui (pour moi du moins, "mis dans le bain" par les explications de Serre) leur donnent tout leur sens. Mais la possibilité d'associer de tels invariants aux variétés algébriques "abstraites" qui interviennent dans ces conjectures, de façon à répondre aux desiderata très précis exigés pour les besoins de cette cause-là - c'était là un simple espoir. Je doute qu'en dehors de Serre et de moi-même, personne d'autre (pas même, et surtout, André Weil lui-même ! \footnote{Chose paradoxale, Weil avait un "bloc" tenace, apparemment viscéral, contre le formalisme cohomologique - alors que ce sont en grande partie ses célèbres conjectures qui ont inspiré le développement des grandes théories cohomologiques en géométrie algébrique, à partir des années 1955 (avec Serre donnant le coup d'envoi, avec son article fondamental FAC, déjà mentionné dans une précédente note de bas de page).

Il me semble que ce "bloc" fait partie, chez Weil, d'une aversion générale contre tous les "gros fourbis", contre tout ce qui s'apparente à un formalisme (quand celui-ci ne peut se résumer en quelques pages), ou à une "construction" tant soit peu imbriquée. Il n'avait rien du "bâtisseur", certes, et c'est visiblement à son corps défendant qu'il s'est vu contraint, au cours des années trente, à développer les premiers fondements de géométrie algébrique "abstraits" qui (vu ces dispositions) se sont révélés un véritable "lit de Procruste" pour l'usager.

Je ne sais s'il m'en a voulu d'être allé au delà, et de m'être investi à construire les vastes demeures, qui ont permis aux rêves d'un Kronecker et au sien de s'incarner en un langage et en des outils délicats et efficaces. Toujours est-il qu'à aucun moment il ne m'a fait un mot de commentaire au sujet du travail dans lequel il me voyait engagé, ou de celui qui était déjà fait. Je n'ai pas non plus eu d'écho à Récoltes et Semailles, que je lui avais envoyé il y a plus de trois mois, avec une dédicace chaleureuse de ma main.}) n’y croyait vraiment...

Peu de temps avant, notre conception de ces invariants de cohomologie s'était d'ailleurs vue enrichir et renouveler profondément par les travaux de Jean Leray (poursuivis en captivité en Allemagne, pendant la guerre, dans la première moitié des années quarante). L'idée novatrice essentielle était celle de faisceau (abélien) sur un espace, auxquels Leray associe une suite de "groupes de cohomologie" correspondants (dits "à coefficients dans ce faisceau"). C'était comme si le bon vieux "mètre cohomologique" standard dont on disposait jusqu'à présent pour "arpenter" un espace, s'était soudain vu multiplier en une multitude inimaginablement grande de nouveaux "mètres" de toutes les tailles, formes et substances imaginables, chacun intimement adapté à l'espace en question, et dont chacun nous livre à son sujet des informations d'une précision parfaite, et qu'il est seul à pouvoir nous donner. C'était là l'idée maîtresse dans une transformation profonde dans notre approche des espaces en tous genres, et sûrement une des idées les plus cruciales apparues au cours de ce siècle. Grâce surtout aux travaux ultérieurs de Jean-Pierre Serre, les idées de Leray ont eu comme premiers fruits, au cours de la décennie déjà suivant leur apparition, un redémarrage impressionnant dans la théorie des espaces topologiques (et notamment, de leurs invariants dits "d'homotopie", intimement liés à la cohomologie), et un autre redémarrage, non moins capital, de la géométrie algébrique dite "abstraite" (avec l'article fondamental "FAC" de Serre, paru en 1955). Mes propres travaux en géométrie, à partir de 1955, se placent en continuité avec ces travaux de Serre, et par là même, avec les idées novatrices de Leray.




