\section{A la découverte de la Mère - ou les deux versants}

A vrai dire, mes réflexions sur les conjectures de Weil elles-mêmes en vue de les établir, sont restées sporadiques. Le panorama qui avait commencé à s'ouvrir devant moi et que je m'efforçais de scruter et de capter, dépassait de très loin en ampleur et en profondeur les hypothétiques besoins d'une démonstration, et même tout ce que ces fameuses conjectures avaient pu d'abord faire entrevoir. Avec l'apparition du thème schématique et de celui des topos, c'est un monde nouveau et insoupçonné qui s'était ouvert soudain. "Les conjectures" y occupaient une place centrale, certes, un peu comme le ferait la capitale d'un vaste empire ou continent, aux provinces innombrables, mais dont la plupart n'ont que des rapports des plus lointains avec ce lieu brillant et prestigieux. Sans avoir eu à me le dire jamais, je me savais le serviteur désormais d'une grande tâche : explorer ce monde immense et inconnu, appréhender ses contours jusqu'aux frontières les plus lointaines ; et aussi, parcourir en tous sens et inventorier avec un soin tenace et méthodique les provinces les plus proches et les plus accessibles, et en dresser des cartes d'une fidélité et d'une précision scrupuleuse, où le moindre hameau et la moindre chaumière auraient leur place...

C'est ce dernier travail surtout qui absorbait le plus gros de mon énergie - un patient et vaste travail de fondements que j'étais le seul à voir clairement et, surtout, à "sentir par les tripes". C'est lui qui a pris, et de loin, la plus grosse part de mon temps, entre 1958 (l'année où sont apparus, coup sur coup, le thème schématique et celui des topos) et 1970 (l'année de mon départ de la scène mathématique).

Souvent d'ailleurs je rongeais mon frein d'être retenu ainsi, comme par un poids tenace et collant, avec ces interminables tâches qui (une fois vu l'essentiel) s'apparentaient plus pour moi à "de l'intendance", qu'à une lancée dans l'inconnu. Constamment je devais retenir cette pulsion de m'élancer de l'avant - celle  du pionnier ou de l'explorateur, parti à la découverte et à l'exploration de mondes inconnus et sans nom, m'appelant sans cesse pour que je les connaisse et les nomme. Cette pulsion-là, et l'énergie que j'y investissais (comme à la dérobée, quasiment !), étaient constamment à la portion congrue.

Pourtant, je savais bien au fond que c'était cette énergie-là, dérobée (pour ainsi dire) à celle que je devais à mes "tâches", qui était de l'essence la plus rare et la plus déliée - que la "création" dans mon travail de mathématicien, c'était avant tout là qu'elle se plaçait : dans cette attention intense pour appréhender, dans les replis obscurs, informes et moites d'une chaude et inépuisable matrice nourricière, les premières traces de forme et de contours de ce qui n'était pas né encore et qui semblait m'appeler, pour prendre forme et s'incarner et naître... Dans le travail de découverte, cette attention intense, cette sollicitude ardente sont une force essentielle, tout comme la chaleur du soleil pour l'obscure gestation des semences enfouies dans la terre nourricière, et pour leur humble et miraculeuse éclosion à la lumière du jour.

Dans mon travail de mathématicien, je vois à l'œuvre surtout ces deux forces ou pulsions, également profondes, de nature (me semble-t-il) différentes. Pour évoquer l'une et l'autre, j'ai utilisé l'image du bâtisseur, et celle du pionnier ou de l'explorateur. Mises côte-à-côte, l'une et l'autre me frappent soudain comme vraiment très "yang", très "masculines", voire "macho"! Elles ont la résonance altière du mythe, ou celle des "grandes occasions". Sûrement elles sont inspirées par les vestiges, en moi, de mon ancienne vision "héroïque" du travail créateur, la vision super-yang. Telles quelles, elles donnent une vision fortement teintée, pour ne pas dire figée, "au garde à vous", d'une réalité bien plus fluide, plus humble, plus "simple" - d'une réalité vivante.

Dans cette mâle pulsion du "bâtisseur", qui semble sans cesse me pousser vers de nouveaux chantiers, je discerne bien pourtant, en même temps, celle du casanier : de celui profondément attaché à "la" maison. Avant toute autre chose, c'est "sa" maison, celle des "proches" - le lieu d'une intime entité vivante dont il se sent faire partie. Ensuite seulement, et à mesure que s'élargit le cercle de ce qui est ressenti comme "proche", est-elle aussi une "maison pour tous". Et dans cette pulsion de "faire des maisons" (comme on "ferait" l'amour...) il y a aussi et avant tout une tendresse. Il y a la pulsion du contact avec ces matériaux qu'on façonne un à un, avec un soin amoureux, et qu'on ne connaît vraiment que par ce contact aimant. Et, une fois montés les murs et posés les poutres et le toit, il y a la satisfaction profonde à installer une pièce après l'autre, et à voir peu à peu s'instaurer, parmi ces salles, ces chambres et ces réduits l'ordre harmonieux de la maison vivante - belle, accueillante, bonne pour y vivre. Car la maison, avant tout et secrètement en chacun de nous, c'est aussi la mère - ce qui nous entoure et nous abrite, à la fois refuge et réconfort ; et peut-être (plus profondément encore, et alors même que nous serions en train de la construire de toutes pièces) c'est cela aussi dont nous sommes nous-mêmes issus, ce qui nous a abrité et nourri, en ces temps à jamais oubliés d'avant notre naissance... C'est aussi le Giron.

Et l'image apparue spontanément tantôt, pour aller au delà de l'appellation prestigieuse de "pionnier", et pour cerner la réalité plus cachée qu'elle recouvrait, était elle aussi dépouillée de tout accent "héroïque". Là encore, c'était l'image archétype du maternel qui est apparue - celle de la "matrice" nourricière et de ses informes et obscurs labeurs...

Ces deux pulsions qui m'apparaissaient comme "de nature différente" sont finalement plus proches que je ne l'aurais pensé. L'une et l'autre sont dans la nature d'une "pulsion de contact", nous portant à la rencontre de "la Mère" : de Celle qui incarne et ce qui est proche, "connu", et ce qui est "inconnu". M'abandonner à l'une ou l'autre pulsion, c'est "retrouver la Mère". C'est renouveler le contact à la fois au proche, au "plus ou moins connu", et au "lointain", à ce qui est "inconnu" mais en même temps pressenti, sur le point de se faire connaître.

La différence ici est de tonalité, de dosage, non de nature. Quand je "bâtis des maisons", c'est le "connu" qui domine, et quand "j'explore", c'est l'inconnu. Ces deux "modes" de découverte, ou pour mieux dire, ces deux aspects d'un même processus ou d'un même travail, sont indissolublement liés. Ils sont essentiels l'un et l'autre, et complémentaires. Dans mon travail mathématique, je discerne un mouvement de va-et-vient constant entre ces deux modes d'approche, ou plutôt, entre les moments (ou les périodes) où l'un prédomine, et ceux où prédomine l'autre \footnote{Ce que je dis ici sur le travail mathématique est vrai également pour le travail de "méditation" (dont il sera question un peu partout dans Récoltes et Semailles). Il n'y a guère de doute pour moi que c'est là une chose qui apparaît dans tout travail de découverte, y compris dans celui de l'artiste (écrivain ou poète, disons). Les deux "versants" que je décris ici peuvent être vus également comme étant, l'un celui de l'expression et de ses exigences "techniques", l'autre celui de la réception (de perceptions et d'impressions de toutes sortes), devenant inspiration par l'effet d'une attention intense. L'un et l'autre sont présents en tout moment du travail, et il y a ce mouvement constant de "va-et-vient" entre les "temps" où l'un prédomine, et ceux où prédomine l'autre.}. Mais il est clair aussi qu’en chaque moment, et l’un et l’autre mode est présent. Quand je construis, aménage, ou que je déblaie, nettoie, ordonne, c'est le "mode" ou le "versant" "yang", ou "masculin" du travail qui donne le ton. Quand j'explore à tâtons l'insaisissable, l'informe, ce qui est sans nom, je suis le versant "yin", ou "féminin" de mon être.

Il n'est pas question pour moi de vouloir minimiser ou renier l'un ou l'autre versant de ma nature, essentiels l'un et l'autre - le "masculin" qui construit et qui engendre, et le "féminin" qui conçoit, et qui abrite les lentes et obscures gestations. Je "suis" l'un et l'autre - "yang" et "yin", "homme" et "femme". Mais je sais aussi que l'essence la plus délicate, la plus déliée dans les processus créateurs se trouve du côté du versant "yin", "féminin" - le versant humble, obscur, et souvent de piètre apparence.

C'est ce versant-là du travail qui, depuis toujours je crois, a exercé sur moi la fascination la plus puissante. Les consensus en vigueur m'encourageaient pourtant à investir le plus clair de mon énergie dans l'autre versant, dans celui qui s'incarne et s'affirme dans des "produits" tangibles, pour ne pas dire finis et achevés des produits aux contours bien tranchés, attestant de leur réalité avec l'évidence de la pierre taillée...

Je vois bien, avec le recul, comment ces consensus ont pesé sur moi, et aussi comment j'ai "accusé le poids" - en souplesse! La partie "conception" ou "exploration" de mon travail était maintenue à la portion congrue jusqu'au moment encore de mon départ, soit. Et pourtant, dans ce coup d'œil rétrospectif sur ce que fut mon œuvre de mathématicien, il ressort avec une évidence saisissante que ce qui fait l'essence et la puissance de cette œuvre, c'est bien ce versant de nos jours négligé, quand il n'est objet de dérision ou d'un condescendant dédain : celui des "idées", voire celui du "rêve", nullement celui des "résultats". Essayant dans ces pages de cerner ce que j'ai apporté de plus essentiel à la mathématique de mon temps, par un regard qui embrasse une forêt, plutôt que de s'attarder sur des arbres - j'ai vu, non un palmarès de "grands théorèmes", mais un vivant éventail d'idées fécondes \footnote{Ce n'est pas que ce qu'on peut appeler les "grands théorèmes" manquent dans mon œuvre, y compris des théorèmes qui résolvent des questions posées par d'autres que moi, que personne avant moi n'avait su résoudre. (J'en passe en revue certains dans la note de b. de p. (***) page 554, de la note "La mer qui monte..." (ReS III, n ${ }^{\circ}$ 122).) Mais, comme je l'ai souligné déjà dès les débuts de cette "promenade" (dans l'étape "Points de vue et vision", $\mathrm{n}^{\circ} 6$ ), ces théorèmes ne prennent pour moi tout leur sens que par le contexte nourricier d'un grand thème, initié par une de ces "idées fécondes". Leur démonstration dès lors découle, comme de source et sans effort, de la nature même, de la "profondeur" du thème qui les porte - comme les vagues du fleuve semblent naître en douceur de la profondeur même de ses eaux, sans rupture et sans effort. Je m'exprime dans un sens tout analogue, mais avec d'autres images, dans la note déjà citée "La mer qui monte...".}, venant concourir toutes à une même et vaste vision.








