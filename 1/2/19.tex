\phantomsection
\section*{Epilogue : les Cercles invisibles}
\addcontentsline{toc}{chapter}{Epilogue : les Cercles invisibles}

\section{La mort est mon berceau (ou trois marmots pour un moribond)}

Jusqu'à l'apparition du point de vue des topos, vers la fin des années cinquante, l'évolution de la notion d'espace m'apparaît comme une évolution essentiellement "continue". Elle paraît se poursuivre sans heurts ni sauts, à partir de la théorisation euclidienne de l'espace qui nous entoure, et de la géométrie léguée par les grecs, s'attachant à l'étude de certaines "figures" (droites, plans, cercles, triangles etc.) vivant dans cet espace. Certes, des changements profonds ont eu lieu dans la façon dont le mathématicien ou le "philosophe de la nature" concevait "l'espace" \footnote{Mon propos initial, en écrivant l' Epilogue, avait été d'inclure une esquisse très sommaire de certains de ces "changements profonds", et faire apparaître cette "continuité essentielle" que j'y vois. J'y ai renoncé, pour ne pas allonger outre mesure cette Promenade, déjà bien plus longue que prévu! Je pense y revenir dans les Commentaires Historiques prévus dans le volume 4 des "Réflexions", à l'intention cette fois d'un lecteur mathématicien (ce qui change totalement la tâche d'exposition).}. Mais ces changements me semblent tous dans la nature d'une "continuité" essentielle - ils n'ont jamais placé le mathématicien, attaché (comme tout un chacun) aux images mentales familières, devant un dépaysement soudain. C'étaient comme les changements, profonds peut-être mais progressifs, qui se font au fil des ans dans un être que nous aurions connu déjà enfant, et dont nous aurions suivi l'évolution depuis ses premiers pas jusqu'à son âge adulte et sa pleine maturité. Des changements imperceptibles en certaines longues périodes de calme plat, et tumultueux peut-être en d'autres. Mais même dans les périodes de croissance ou de mûrissement les plus intenses, et alors même que nous l'aurions perdu de vue pendant des mois, voire des années, à aucun moment il ne pouvait pourtant y avoir le moindre doute, la moindre hésitation : c'est bien lui encore, un être bien connu et familier, que nous retrouvions, fût-ce avec des traits changés.

Je crois pouvoir dire, d'ailleurs, que vers le milieu de ce siècle, cet être familier avait déjà beaucoup vieilli - tel un homme qui se serait finalement épuisé et usé, dépassé par un afflux de tâches nouvelles auxquelles il n'était nullement préparé. Peut-être même était-il déjà mort de sa belle mort, sans que personne ne se soucie d'en prendre note et d'en faire le constat. "Tout le monde" faisait bien mine encore de s'affairer dans la maison d'un vivant, que c'en était quasiment comme s'il était encore bel et bien vivant en effet.

Or doncques, jugez de l'effet fâcheux, pour les habitués de la maison, quand à la place du vénérable vieillard figé, droit et raide dans son fauteuil, on voit s'ébattre soudain un gamin vigoureux, pas plus haut que trois pommes, et qui prétend en passant, sans rire et comme chose qui irait de soi, que Monsieur Espace (et vous pouvez même désormais laisser tomber le "Monsieur", à votre aise...) c'est lui ! Si encore il avait l'air au moins d'avoir les traits de famille, un enfant naturel peut-être qui sait... mais pas du tout! A vue de nez, rien qui rappelle le vieux Père Espace qu'on avait si bien connu (ou cru connaître...), et dont on était bien sûr, en tous cas (et c'était bien là la moindre des choses...) qu'il était éternel...

C'est ça, la fameuse "mutation de la notion d'espace". C'est ça que j'ai dû "voir", comme chose d'évidence, dès les débuts des années soixante au moins, sans avoir jamais eu l'occasion de me le formuler avant ce moment même où j'écris ces lignes. Et je vois soudain avec une clarté nouvelle, par la seule vertu de cette évocation imagée et de la nuée d'association qu'elle suscite aussitôt : la notion traditionnelle d' "espace", tout comme celle étroitement apparentée de "variété" (en tous genres, et notamment celle de "variété algébrique"), avait pris, vers le moment où je suis venu dans les parages, un tel coup de vieux déjà, que c'était bien comme si elles étaient mortes... \footnote{Cette affi rmation (qui semblera péremptoire à certains) est à prendre avec un "grain de sel". Elle n'est ni plus, ni moins valable que celle (que je reprends à mon compte plus bas) que le "modèle newtonien" de la mécanique (terrestre ou céleste) était "moribond" au début de ce siècle, quand Einstein est venu à la rescousse. C'est un fait qu'encore aujourd'hui, dans la plupart des situations "courantes" en physique, le modèle newtonien est parfaitement adéquat, et ce serait de la folie (vue la marge d'erreur admise dans les mesures faites) d'aller chercher des modèles relativistes. De même, dans de nombreuses situations en mathématique, les anciennes notions familières d' "espace" et de "variété" restent parfaitement adéquates, sans aller chercher des éléments nilpotents, des topos ou des "structures modérées". Mais dans l'un et l'autre cas, pour un nombre croissant de contextes intervenant dans une recherche de pointe, les anciens cadres conceptuels sont devenus inaptes à exprimer les situations même les plus "courantes".} Et je pourrais dire que c'est avec l'apparition coup sur coup du point de vue des schémas (et de sa progéniture \footnote{(A l'intention du mathématicien) Dans cette "progéniture", je compte notamment les schémas formels, les "multiplicités" en tous genres (et notamment, les multiplicités schématiques, ou formelles), enfin les espaces dits "rigide-analytiques" (introduits par Tate, en suivant un "maître d'œuvre" fourni par moi, inspiré par la notion nouvelle de topos, en même temps que par celle de schéma formel). Cette liste n'est d'ailleurs nullement exhaustive...}, plus dix mille pages de fondements à la clef), puis de celui des topos, qu'une situation de crise-qui-ne-dit-pas-son-nom s'est trouvée finalement dénouée.

Dans l'image de tantôt, ce n'est pas d'un gamin d'ailleurs qu'il faudrait parler, comme produit d'une mutation soudaine, mais de deux. Deux gamins, de plus, qui ont entre eux un "air de famille" irrécusable, même s'ils ne ressemblent guère au défunt vieillard. Et encore, en y regardant de près, on pourrait dire que le bambin Schémas ferait comme un "chaînon de parenté" entre feu Père Espace (alias Variétés-en-tous-genres) et le bambin Topos \footnote{Il y aurait lieu d'ailleurs, à ces deux bambins, d'en ajouter encore un troisième plus jeune, apparu en des temps moins cléments : c'est le marmot Espace modéré. Comme je l'ai signalé ailleurs, il n'a pas eu droit à un certificat de naissance, et c'est dans l'illégalité totale que je l'ai néanmoins inclus au nombre des douze "maître-thèmes" que j'ai eu l'honneur d'introduire en mathématique.}.

