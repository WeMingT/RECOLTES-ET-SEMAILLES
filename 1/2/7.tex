\section{La "grande idée" - ou les arbres et la forêt}

La période dite "productive" de mon activité mathématique, c'est-à-dire celle attestée par des publications en bonne et due forme, s'étend entre 1950 et 1969, donc sur vingt ans. Et pendant vingt-cinq ans, entre 1945 (quand j'avais dix-sept ans) et 1969 (quand j'allais sur les quarante-deux), j'ai investi pratiquement la totalité de mon énergie dans la recherche mathématique. Investissement démesuré, certes. Je l'ai payé par une longue stagnation spirituelle, par un "épaississement" progressif, que j'aurai plus d'une fois l'occasion d'évoquer dans les pages de Récoltes et Semailles. Pourtant, à l'intérieur du champ limité d'une activité purement intellectuelle, et par l'éclosion et la maturation d'une vision restreinte au monde des seules choses mathématique, c'étaient des années de créativité intense.

Pendant cette longue période de ma vie, la quasi-totalité de mon temps et de mon énergie était consacré à ce qu'on appelle du "travail sur pièces" : au minutieux travail de façonnage, d'assemblage et de rodage, requis pour la construction de toutes pièces des maisons qu'une voix (ou un démon...) intérieur m'enjoignait de bâtir, selon un maître d'oeuvre qu'elle me soufflait au fur et à mesure que le travail avançait. Pris par les tâches de "métier" : celles tour à tour de tailleur de pierre, de maçon, de charpentier, voire de plombier, de menuisier et d'ébéniste - rarement ai-je pris le loisir de noter noir sur blanc, ne fût-ce qu'à grands traits, le maître-plan invisible à tous (comme il est apparu plus tard...) sauf à moi, qui au cours des jours, ces mois et des années guidait ma main avec une sûreté de somnambule \footnote{L'image du "somnambule" m'a été inspirée par le titre du remarquable livre de Koestler "Les somnambules" (Calman Lévy), présentant un "Essai sur l'histoire des conceptions de l'Univers", depuis les origines de la pensée scientifi que jusqu'à Newton. Un des aspects de cette histoire qui a frappé Koestler et qu'il met en évidence, c'est à quel point, souvent, le cheminement d'un certain point dans notre connaissance du monde, à quelque autre point qui (logiquement et avec le recul) semble tout proche, passe par les détours parfois les plus acadabrants, qui semblent défi er la saine raison ; et comment pourtant, à travers ces mille détours qui semblent devoir les fourvoyer à jamais, et avec une "sûreté de somnambule", les hommes partis à la recherche des "clefs" de l'Univers tombent, comme malgré eux et sans même s'en rendre compte souvent, sur d'autres "clefs" qu'ils étaient loin de prévoir, et qui se révèlent pourtant être "les bonnes".

Par ce que j'ai pu observer autour de moi, au niveau de la découverte mathématique, ces faramineux détours dans le cheminement de la découverte sont le fait de certains chercheurs de grand format, mais nullement de tous. Cela pourrait être dû au fait que depuis deux ou trois siècles, la recherche dans les sciences de la nature, et plus encore en mathématique, se trouve dégagée des présupposés religieux ou métaphysiques impératifs relatifs à une culture et à une époque données, lesquels ont été des freins particulièrement puissants au déployement (pour le meilleur et pour le pire) d'une compréhension "scientique" de l'Univers. Il est vrai pourtant que certaines idées et des notions les plus fondamentales et les plus évidentes en mathématique (comme celles de déplacement, de groupe, le nombre zéro, le calcul littéral, les coordonnées d'un point dans l'espace, la notion d'ensemble, ou celle de "forme" topologique, sans même parler des nombres négatifs et des nombres complexes) ont mis des millénaires avant de faire leur apparition. Ce sont là autant de signes éloquents de ce "bloc" invétéré, profondément implanté dans la psyché, contre la conception d'idées totalement nouvelles, même dans les cas où celles-ci sont d'une simplicité enfantine et semblent s'imposer d'elles-mêmes avec la force de l'évidence, pendant des générations, voire, pendant des millénaires...

Pour en revenir à mon propre travail, j'ai l'impression que dans celui-ci les "foirages" (plus nombreux peut-être que chez la plupart de mes collègues) se bornent exclusivement à des points de détail, généralement vite repérés par mes propres soins. Ce sont de simples "accidents de parcours", de nature purement "locale" et sans incidence sérieuse sur la validité des intuitions essentielles concernant la situation examinée. Par contre, au niveau des idées et des grandes intuitions directrices, il me semble que mon oeuvre est exempte de tout "raté", si incroyable que cela puisse paraître. C’est cette sûreté jamais en défaut pour appréhender à chaque moment, sinon les aboutissements ultimes d'une démarche (lesquels restent le plus souvent cachés au regard), mais du moins les directions les plus fertiles qui s'offrent pour me mener droit vers les choses essentielles - c'est cette sûreté-là qui avait fait resurgir en moi l'image de Koestler du "somnambule".}.Il faut dire que le travail sur pièces, dans lequel j’aimais à mettre un soin amoureux, n'était nullement fait pour me déplaire. De plus, le mode d'expression mathématique qui était professé et pratiqué par mes aînés donnait prééminence (à dire le moins) à l'aspect technique du travail, et n'encourageait guère les "digressions" qui se seraient attardées sur les "motivations"; voire, celles qui auraient fait mine de faire surgir des brumes quelque image ou vision peut-être inspirante, mais qui, faute de s'être incarnée encore en des constructions tangibles en bois, en pierre ou en ciment pur et dur, s'apparentait plus à des lambeaux de rêve, qu'au travail de l'artisan, appliqué et consciencieux.

Au niveau quantitatif, mon travail pendant ces années de productivité intense s'est concrétisé surtout par quelques douze mille pages de publications, sous forme d'articles, de monographies ou de séminaires \footnote{A partir des années 1960, une partie de ces publications a été écrite avec la collaboration de collègues (surtout J. Dieudonné) et d'élèves.}, et par des centaines, si ce n'est des milliers, de notions nouvelles, qui sont entrées dans le patrimoine commun, avec les noms même que je leur avais donné quand je les avais dégagées \footnote{Les plus importantes parmi ces notions sont passées en revue dans l'Esquisse Thématique, et dans le Commentaire Histoire qui l'accompagne, lesquels seront inclus dans le volume 4 des Réflexions. Certains des noms m'ont été suggérés par des amis ou des élèves, tels le terme "morphisme lisse" (J.Dieudonné) ou la panoplie "site, champ, gerbe, lien", développée dans la thèse de Jean Giraud.}. Dans l'histoire des mathématiques, je crois bien être celui qui a introduit dans notre science le plus grand nombre de notions nouvelles, et en même temps, celui qui a été amené, par cela même, à inventer le plus grand nombre de noms nouveaux, pour exprimer ces notions avec délicatesse, et de façon aussi suggestive que je le pouvais.

Ces indications toutes "quantitatives" ne fournissent, certes, qu'une appréhension plus que grossière de mon oeuvre, passant à côté de ce qui véritablement en fait l'âme, la vie et la vigueur. Comme je l'écrivais tantôt, ce que j'ai apporté de meilleur dans la mathématique, ce sont les "points de vue" nouveaux que j'ai su entrevoir d'abord, et ensuite dégager patiemment et développer peu ou prou. Comme les notions dont je viens de parler, ces nouveaux points de vue, s'introduisant dans une vaste multiplicité de situations très différentes, sont eux-mêmes quasiment innombrables.

Il est pourtant des points de vue qui sont plus vastes que d'autres, et qui à eux seuls suscitent et englobent une multitude de points de vue partiels, dans une multitude de situations particulières différentes. Un tel point de vue peut être appelé aussi, à juste titre, une "grande idée". Par la fécondité qui est sienne, une telle idée donne naissance à une grouillante progéniture, d’idées qui toutes héritent de sa fécondité, mais dont la plupart (sinon toutes) sont de portée moins vaste que l'idée-mère.

Quant à exprimer une grande idée, "la dire" donc, c'est là, le plus souvent, une chose presque aussi délicate que sa conception même et sa lente gestation dans celui qui l'a conçue - ou pour mieux dire, ce laborieux travail de gestation et de formation n'est autre justement que celui qui "exprime" l'idée : le travail qui consiste à la dégager patiemment, jour après jour, des voiles de brumes qui l'entourent à sa naissance, pour arriver peu à peu à lui donner forme tangible, en un tableau qui s'enrichit, s'affermit et s'affine au fil des semaines, des mois et des années. Nommer simplement l'idée, par quelque formule frappante, ou par des mots-clef plus ou moins techniques, peut être affaire de quelques lignes, voire de quelques pages - mais rares seront ceux qui, sans déjà bien la connaître, sauront entendre ce "nom" et y reconnaître un visage. Et quand l'idée est arrivée en pleine maturité, cent pages peut-être suffiront à l'exprimer, à la pleine satisfaction de l'ouvrier en qui elle était née - comme il se peut aussi que dix mille pages, longuement travaillées et pesées, n'y suffiront pas \footnote{Au moment de quitter la scène mathématique en 1970, l'ensemble de mes publications (dont bon nombre en collaboration) sur le thème central des schémas, devait se monter à quelques dix mille pages. Cela ne représentait pourtant qu'une partie modeste du programme de vaste envergure que je voyais devant moi, concernant les schémas. Ce programme a été abandonné sine die dès mon départ, et ceci malgré le fait qu'à très peu de choses près, tout ce qui avait été développé et publié déjà pour être mis à la disposition de tous, est entré d'emblée dans le patrimoine commun des notions et des résultats communément utilisés comme "bien connus".
La partie de mon programme sur le thème schématique et sur ses prolongements et ramifi cations, que j'avais accomplie au moment de mon départ, représente à lui seul le plus vaste travail de fondements jamais accompli dans l'histoire de la mathématique, et sûrement un des plus vastes aussi dans l'histoire des Sciences.}.

Et dans l'un comme l'autre cas, parmi ceux qui, pour la faire leur, ont pris connaissance du travail qui enfin présente l'idée en plein essor, telle une spacieuse futaie qui aurait poussé là sur une lande déserte - il y a fort à parier que nombreux seront ceux qui verront bien tous ces arbres vigoureux et sveltes et qui en auront l'usage (qui pour y grimper, qui pour en tirer poutres et planches, et tel autre encore pour faire flamber les feux dans sa cheminette...), Mais rares seront ceux qui auront su voir la forêt...





