\section{L'unique - ou le don de solitude}

Cette brève excursion chez les "voisins d'en face", les physiciens, pourra servir de point de repère pour un lecteur qui (comme la plupart des gens) ignore tout du monde des mathématiciens, mais qui a sûrement entendu causer d'Einstein et de sa fameuse "quatrième dimension", voire même, de mécanique quantique. Après tout, même si ce n'était pas prévu par les inventeurs que leurs découvertes se concrétiseraient en des Hiroshima, et plus tard en des surenchères atomiques tant militaires que (soi-disant) "pacifiques", le fait est que la découverte en physique a un impact tangible et quasi-immédiat sur le monde des hommes en général. L'impact de la découverte mathématique, et surtout en mathématiques dites "pures" (c'est à dire, sans motivation en vue d' "applications") est moins direct, et sûrement plus délicat à cerner. Je n'ai pas eu connaissance, par exemple, que mes contributions à la mathématique aient "servi" à quoi que ce soit, pour construire le moindre engin disons. Je n'y ai aucun mérite qu'il en soit ainsi, c'est sûr, mais ça n'empêche que ça me rassure. Dès qu'il y a des applications, on peut être sûr que c'est les militaires (et après eux, la police) qui sont les premiers à s'en emparer - et pour ce qui est de l'industrie (même celle dite "pacifique"), ce n'est pas toujours tellement mieux...

Pour ma propre gouverne certes, ou pour celle d'un lecteur mathématicien, il s'imposerait plutôt d'essayer de situer mon œuvre par des "points de repère" dans l'histoire de la mathématique elle-même, plutôt que d'aller chercher des analogies ailleurs. J'y ai pensé ces derniers jours, dans la limite de ma connaissance assez vague de l'histoire en question \footnote{Depuis que je suis gosse déjà, je n'ai jamais trop accroché à l'histoire (ni à la géographie d'ailleurs). (Dans la cinquième partie de Récoltes et Semailles (écrite seulement en partie), j'ai l'occasion "en passant" de détecter ce qui me semble la raison profonde de ce "bloc" partiel contre l'histoire - un bloc qui est en train de se résorber, je crois, au cours de ces dernières années.) L'enseignement mathématique reçu par mes aînés, dans le "cercle bourbachique", n'a pas été d'ailleurs pour arranger les choses - les références historiques occasionnelles y ont été plus que rares.}. Au cours de la "Promenade" déjà, j'avais eu l'occasion d'évoquer une "lignée" de mathématiciens, d'un tempérament en lequel je me reconnais : Galois, Riemann, Hilbert. Si j'étais mieux au courant de l'histoire de mon art, il y a des chances que je trouverais à prolonger cette lignée plus loin dans le passé, ou à y intercaler peut-être quelques autres noms que je ne connais guère que par ouï-dire. La chose qui m'a frappé, c'est que je ne me rappelle pas avoir eu connaissance, ne fût-ce que par allusion par des amis ou collègues mieux versés en histoire que moi, d'un mathématicien à part moi qui ait apporté une multiplicité d'idées novatrices, non pas plus ou moins disjointes les unes des autres, mais comme parties d'une vaste vision unificatrice (comme cela a été le cas pour Newton et pour Einstein en physique et en cosmologie, et pour Darwin et pour Pasteur en biologie). J'ai eu connaissance seulement de deux "moments" dans l'histoire de la mathématique, où soit née une vision nouvelle de vaste envergure. L'un de ces moments est celui de la naissance de la mathématique, en tant que science au sens où nous l'entendons aujourd'hui, il y a 2500 ans, dans la Grèce antique. L'autre est, avant tout, celui de la naissance du calcul infinitésimal et intégral, au dix-septième siècle, époque marquée par les noms de Newton, Leibnitz, Descartes et d'autres. Pour autant que je sache, la vision née en l'un ou en l'autre moment a été l'œuvre non d'un seul, mais l'œuvre collective d'une époque.

Bien sûr, entre l'époque de Pythagore et d'Euclide et le début du dix-septième, la mathématique avait eu le temps de changer de visage, et de même entre celle du "Calcul des infiniments petits" créé par les mathématiciens du dix-septième siècle, et le milieu du présent dix-neuvième. Mais pour autant que je sache, les changements profonds qui sont intervenus pendant ces deux périodes, l'une de plus de deux mille ans et l'autre de trois siècles, ne se sont jamais concrétisés ou condensés en une vision nouvelle s'exprimant dans une œuvre donnée \footnote{Des heures après avoir écrit ces lignes, j'ai été frappé que je n'aie pas songé ici à la vaste synthèse des mathématiques contemporaines que s'efforce de présenter le traité (collectif) de M. Bourbaki. (Il sera encore abondamment question du groupe Bourbaki dans la première partie de Récoltes et Semailles.) Cela tient, il me semble, à deux raisons.

D'une part, cette synthèse se borne à une sorte de "mise en ordre" d'un vaste ensemble d'idées et de résultats déjà connus, sans y apporter d'idée novatrice de son crû. Si idée nouvelle il y a, ce serait celle d'une définition mathématique précise de la notion de "structure", qui s'est révélée un fil conducteur précieux à travers tout le traité. Mais cette idée me semble s'assimiler plutôt à celle d'un lexicographe intelligent et imaginatif, qu'à un élément de renouveau d'une langue, donnant une appréhension renouvelée de la réalité (ici, de celle des choses mathématiques).

D'autre part, dès les années cinquante, l'idée de structure s'est vue dépasser par les événements, avec l'afflux soudain des méthodes "catégoriques" dans certaines des parties les plus dynamiques de la mathématique, telle la topologie ou la géométrie algébrique. (Ainsi, la notion de "topos" refuse d'entrer dans le "sac bourbachique" des structures, décidément étroit aux entournures !) En se décidant, en pleine connaissance de cause, certes, à ne pas s'engager dans cette "galère", Bourbaki a par là-même renoncé à son ambition initiale, qui était de fournir les fondements et le langage de base pour l'ensemble de la mathématique contemporaine.

Il a, par contre, fixé un langage et, en même temps, un certain style d'écriture et d'approche de la mathématique. Ce style était à l'origine le reflet (très partiel) d'un certain esprit, vivant et direct héritage de Hilbert. Au cours des années cinquante et soixante, ce style a fini par s'imposer - pour le meilleur et (surtout) pour le pire. Depuis une vingtaine d'années, il a fini par devenir un rigide "canon" d'une "rigueur" de pure façade, dont l'esprit qui l'animait jadis semble disparu sans retour.}, et pourtant d'une façon similaire à ce qui a eu lieu en physique et en cosmologie avec les grandes synthèses de Newton, puis d'Einstein, en deux moments cruciaux de leur histoire.

Il semblerait bien qu'en tant que serviteur d'une vaste vision unificatrice née en moi, je sois "unique en mon genre" dans l'histoire de la mathématique de l'origine à nos jours. Désolé d'avoir l'air de vouloir me singulariser plus qu'il ne paraît permis ! À mon propre soulagement, je crois pourtant discerner une sorte de frère potentiel (et providentiel !). J'ai déjà eu tantôt l'occasion de l'évoquer, comme le premier dans la lignée de mes "frères de tempérament" : c'est Évariste Galois. Dans sa courte et fulgurante vie, je crois discerner l'amorce d'une grande vision - celle justement des "épousailles du nombre et de la grandeur", dans une vision géométrique nouvelle. J'évoque ailleurs dans Récoltes et Semailles \footnote{Voir "L'héritage de Galois" (ReS I, section 7).} comment, il y a deux ans, est apparu en moi cette intuition soudaine : que dans le travail mathématique qui à ce moment exerçait sur moi la fascination la plus puissante, j'étais en train de "reprendre l'héritage de Galois". Cette intuition, rarement évoquée depuis, a pourtant eu le temps de mûrir en silence. La réflexion rétrospective sur mon œuvre que je poursuis depuis trois semaines y aura sûrement encore contribué. La filiation la plus directe que je crois reconnaître à présent avec un mathématicien du passé, est bien celle qui me relie à Évariste Galois. A tort ou à raison, il me semble que cette vision que j'ai développée pendant quinze années de ma vie, et qui a continué encore à mûrir en moi et à s'enrichir pendant les seize années écoulées depuis mon départ de la scène mathématique - que cette vision est aussi celle que Galois n'aurait pu s'empêcher de développer \footnote{Je suis persuadé d'ailleurs qu'un Galois serait allé bien plus loin encore que je n'ai été. D'une part à cause de ses dons tout à fait exceptionnels (que je n'ai pas reçus en partage, quant à moi). D'autre part parce qu'il est probable qu'il n'aurait pas, comme moi, laissé se distraire la majeure part de son énergie, pour d'interminables tâches de mise en forme minutieuse, au fur et à mesure, de ce qui est déjà plus ou moins acquis...}, s'il s'était trouvé dans \footnote{Évariste Galois (1811-1832) est mort dans un duel, à l'âge de vingt-et-un ans. Il y a, je crois, plusieurs biographies de lui. J'ai lu comme jeune homme une biographie romancée, écrite par le physicien Infeld, qui m'avait beaucoup frappée à l'époque.} es parages à ma place, et sans qu'une mort précoce ne vienne brutalement couper court un magnifique élan.

Il y a une autre raison encore, sûrement, qui contribue à me donner ce sentiment d'une "parenté essentielle" - d'une parenté qui ne se réduit pas au seul "tempérament mathématique", ni aux aspects marquants d'une œuvre. Entre sa vie et la mienne, je sens aussi une parenté de destins. Certes, Galois est mort stupidement, à l'âge de vingt-et-un ans, alors que je vais, moi, sur mes soixante ans, et bien décidé à faire de vieux os. Cela n'empêche pourtant qu'Évariste Galois est resté de son vivant, tout comme moi un siècle et demi plus tard, un "marginal" dans le monde mathématique officiel. Dans le cas de Galois, il pourrait sembler à un regard superficiel que cette marginalité était "accidentelle", qu'il n'avait tout simplement pas eu le temps encore de "s'imposer" par ses idées novatrices et par ses travaux. Dans mon cas, ma marginalité, pendant les trois premières années de ma vie de mathématicien, était due à mon ignorance (délibérée peut-être...) de l'existence même d'un monde des mathématiciens, auquel j'aurais à me confronter; et depuis mon départ de la scène mathématique, il y a seize ans, elle est la conséquence d'un choix délibéré. C'est ce choix, sûrement, qui a provoqué en représailles une "volonté collective sans failles" d'effacer de la mathématique toute trace de mon nom, et avec lui la vision aussi dont je m'étais fait le serviteur.

Mais au delà de ces différences accidentelles, je crois discerner à cette "marginalité" une cause commune, que je sens essentielle. Cette cause, je ne la vois pas dans des circonstances historiques, ni dans des particularités de "tempérament" ou de "caractère" (lesquels sont sans doute aussi différents de lui à moi qu'ils peuvent l'être d'une personne à une autre), et encore moins certes au niveau des "dons" (visiblement prodigieux chez Galois, et comparativement modestes chez moi). S'il y a bien une "parenté essentielle", je la vois à un niveau bien plus humble, bien plus élémentaire.

J'ai senti une telle parenté en quelques rares occasions dans ma vie. C'est par elle aussi que je me sens "proche" d'un autre mathématicien encore, et qui fut mon aîné : Claude Chevalley \footnote{Je parle de Claude Chevalley ici et là dans Récoltes et Semailles, et plus particulièrement dans la section "Rencontre avec Claude Chevalley - ou liberté et bons sentiments" (ReS I section 11), et dans la note "Un adieu à Claude Chevalley" (ReS III, note n ${ }^{\circ}$ 100).}. Le lien que je veux dire est celui d'une certaine "naïveté", ou d'une "innocence", dont j'ai eu occasion de parler. Elle s'exprime par une propension (souvent peu appréciée par l'entourage) à regarder les choses par ses propres yeux, plutôt qu'à travers des lunettes brevetées, gracieusement offertes par quelque groupe humain plus ou moins vaste, investi d'autorité pour une raison ou une autre.

Cette "propension", ou cette attitude intérieure, n'est pas le privilège d'une maturité, mais bien celui de l'enfance. C'est un don reçu en naissant, en même temps que la vie - un don humble et redoutable. Un don souvent enfoui profond, que certains ont su conserver tant soit peu, ou retrouver peut-être...

On peut l'appeler aussi le don de solitude.







