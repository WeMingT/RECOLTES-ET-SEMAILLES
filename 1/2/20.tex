\section{Coup d’oeil chez les voisins d’en face}

La situation me semble très proche de celle qui s'est présentée au début de ce siècle, avec l'apparition de la théorie de la relativité d'Einstein. Il y avait un cul-de-sac conceptuel, plus flagrant encore, se concrétisant par une contradiction soudaine, laquelle semblait irrésoluble. Comme de juste, l'idée nouvelle qui allait remettre de l'ordre dans le chaos était une idée d'une simplicité enfantine. La chose remarquable (et conforme à un scénario des plus répétitifs...), c'est que parmi tous ces gens brillants, éminents, prestigieux qui étaient sur les dents soudain, pour essayer de "sauver les meubles", personne n'y ait songé, à cette idée. Il fallait que ce soit un jeune homme inconnu, frais émoulu (si ça se trouve) des bancs des amphithéâtres estudiantins, qui vienne (un peu embarrassé peut-être de sa propre audace...) expliquer à ses illustres aînés ce qu'il fallait faire pour "sauver les phénomènes" : il y avait qu'à plus séparer l'espace du temps \footnote{C'est un peu court, bien sûr, comme description de l'idée d'Einstein. Au niveau technique, il fallait mettre en évidence quelle structure mettre sur le nouvel espace-temps (c'était pourtant déjà "en l'air", avec la théorie de Maxwell et les idées de Lorenz). Le pas essentiel ici était non de nature technique, mais bien "philosophique" : se rendre compte que la notion de simultanéité pour des événements éloignés n'avait aucune réalité expérimentale. C'est ça, la "constatation enfantine", le "mais l'Empereur est nu !", qui a fait franchir ce fameux "cercle impérieux et invisible qui limite un Univers"...}. Techniquement, tout était réuni alors pour que cette idée éclose et soit accueillie. Et c'est à l'honneur des aînés d'Einstein, qu'ils aient su en effet accueillir l'idée nouvelle, sans trop morigéner. C'est là un signe que c'était encore une grande époque...

Du point de vue mathématique, l'idée nouvelle d'Einstein était banale. Du point de vue de notre conception de l'espace physique par contre, c'était une mutation profonde, et un "dépaysement" soudain. La première mutation du genre, depuis le modèle mathématique de l'espace physique dégagé par Euclide il y avait 2400 ans, et repris tel quel pour les besoins de la mécanique par tous les physiciens et astronomes depuis l'antiquité (y inclus Newton), pour décrire les phénomènes mécaniques terrestres et stellaires.

Cette idée initiale d'Einstein s'est par la suite beaucoup approfondie, s'incarnant en un modèle mathématique plus subtil, plus riche et plus souple, en s'aidant du riche arsenal des notions mathématiques déjà existantes \footnote{Il s'agit surtout de la notion de "variété riemanienne", et du calcul tensoriel sur une telle variété.}. Avec la "théorie de la relativité généralisée", cette idée s'élargit en une vaste vision du monde physique, embrassant dans un même regard le monde subatomique de l'infiniment petit, le système solaire, la voie lactée et les galaxies lointaines, et le cheminement des ondes électromagnétiques dans un espace-temps courbé en chaque point par la matière qui s'y trouve \footnote{Un des traits les plus frappants qui distingue ce modèle du modèle euclidien (ou newtonien) de l'espace et du temps, et aussi du tout premier modèle d'Einstein ("relativité restreinte"), c'est que la forme topologique globale de l'espace-temps reste indéterminée, au lieu d'être prescrite impérativement par la nature même du modèle. La question de savoir quelle est cette forme globale, me paraît (en tant que mathématicien) l'une des plus fascinantes de la cosmologie.}. C'est là la deuxième et la dernière fois dans l'histoire de la cosmologie et de la physique (à la suite de la première grande synthèse de Newton il y a trois siècles), qu'est apparue une vaste vision unificatrice, dans le langage d'un modèle mathématique, de l'ensemble des phénomènes physiques dans l'Univers.

Cette vision einsteinienne de l'Univers physique a d'ailleurs été débordée à son tour par les événements. "L'ensemble des phénomènes physiques" dont il s'agit de rendre compte a eu le temps de s'étoffer, depuis les débuts du siècle ! Il est apparu une multitude de théories physiques, pour rendre compte chacune, avec plus ou moins de succès, d'un paquet limité de faits, dans l'immense capharnaüm de tous les "faits observés". Et on attend toujours le gamin audacieux, qui trouvera en jouant la nouvelle clef (s'il en est une...), le "modèlegâteau" rêvé, qui veuille bien "marcher" pour sauver tous les phénomènes à la fois... \footnote{On a appelé "théorie unitaire" une telle théorie hypothétique, qui arriverait à "unifier" et à concilier la multitude de théories partielles dont il a été question. J'ai le sentiment que la réflexion fondamentale qui attend d'être entreprise, aura à se placer sur deux niveaux différents.

$1^{\circ}$) Une réflexion de nature "philosophique", sur la notion même de "modèle mathématique" pour une portion de la réalité. Depuis les succès de la théorie newtonienne, c'est devenu un axiome tacite du physicien qu'il existe un modèle mathématique (voire même, un modèle unique, ou "le" modèle) pour exprimer la réalité physique de façon parfaite, sans "décollement" ni bavure. Ce consensus, qui fait loi depuis plus de deux siècles, est comme une sorte de vestige fossile de la vivante vision d'un Pythagore que "Tout est nombre". Peut-être est-ce là le nouveau "cercle invisible", qui a remplacé les anciens cercles métaphysiques pour limiter l'Univers du physicien (alors que la race des "philosophes de la nature" semble définitivement éteinte, supplantée haut-la-main par celle des ordinateurs...). Pour peu qu'on veuille bien s'y arrêter ne fût-ce qu'un instant, il est bien clair pourtant que la validité ce consensus-là n'a rien d'évident. Il y a même des raisons philosophiques très sérieuses, qui conduisent à le mettre en doute a priori, ou du moins, à prévoir à sa validité des limites très strictes. Ce serait le moment ou jamais de soumettre cet axiome à une critique serrée, et peut-être même, de "démontrer", au delà de tout doute possible, qu'il n'est pas fondé : qu'il n'existe pas de modèle mathématique rigoureux unique, rendant compte de l'ensemble des phénomènes dits "physiques" répertoriés jusqu'à présent.

Une fois cernée de façon satisfaisante la notion même de "modèle mathématique", et celle de la "validité" d'un tel modèle (dans la limite de telles "marges d'erreur" admises dans les mesures faites), la question d'une "théorie unitaire" ou tout au moins celle d'un "modèle optimum" (en un sens à préciser) se trouvera enfin clairement posée. En même temps, on aura sans doute une idée plus claire aussi du degré d'arbitraire qui est attaché (par nécessité, peut-être) au choix d'un tel modèle.

$2^{\circ}$) C'est après une telle réflexion seulement, il me semble, que la question "technique" de dégager un modèle explicite, plus satisfaisant que ses devanciers, prend tout son sens. Ce serait le moment alors, peut-être, de se dégager d'un deuxième axiome tacite du physicien, remontant à l'antiquité, lui, et profondément ancré dans notre mode de perception même de l'espace : c'est celui de la nature continue de l'espace et du temps (ou de l'espace-temps), du "lieu" donc où se déroulent les "phénomènes physiques".

Il doit y avoir déjà quinze ou vingt ans, en feuilletant le modeste volume constituant l'œuvre complète de Riemann, j'avais été frappé par une remarque de lui "en passant". Il y fait observer qu'il se pourrait bien que la structure ultime de l'espace soit "discrète", et que les représentations "continues" que nous nous en faisons constituent peut-être une simplification (excessive peut-être, à la longue...) d'une réalité plus complexe; que pour l'esprit humain, "le continu" était plus aisé à saisir que "le discontinu", et qu'il nous sert, par suite, comme un "approximation" pour appréhender le discontinu. C'est là une remarque d'une pénétration surprenante dans la bouche d'un mathématicien, à un moment où le modèle euclidien de l'espace physique n'avait jamais encore été mis en cause ; au sens strictement logique, c'est plutôt le discontinu qui, traditionnellement, a servi comme mode d'approche technique vers le continu.

Les développements en mathématique des dernières décennies ont d'ailleurs montré une symbiose bien plus intime entre structures continues et discontinues, qu'on ne l'imaginait encore dans la première moitié de ce siècle. Toujours est-il que de trouver un modèle "satisfaisant" (ou, au besoin, un ensemble de tels modèles, se "raccordant" de façon aussi satisfaisante que possible...), que celui-ci soit "continu", "discret" ou de nature "mixte" - un tel travail mettra en jeu sûrement une grande imagination conceptuelle, et un flair consommé pour appréhender et mettre à jour des structures mathématiques de type nouveau. Ce genre d'imagination ou de "flair" me semble chose rare, non seulement parmi les physiciens (où Einstein et Schrödinger semblent avoir été parmi les rares exceptions), mais même parmi les mathématiciens (et là je parle en pleine connaissance de cause).

Pour résumer, je prévois que le renouvellement attendu (s'il doit encore venir...) viendra plutôt d'un mathématicien dans l'âme, bien informé des grands problèmes de la physique, que d'un physicien. Mais surtout, il y faudra un homme ayant "l'ouverture philosophique" pour saisir le nœud du problème. Celui-ci n'est nullement de nature technique, mais bien un problème fondamental de "philosophie de la nature".}.

La comparaison entre ma contribution à la mathématique de mon temps, et celle d'Einstein à la physique, s'est imposée à moi pour deux raisons : l'une et l'autre œuvre s'accomplit à la faveur d'une mutation de la conception que nous avons de "l'espace" (au sens mathématique dans un cas, au sens physique dans l'autre) ; et l'une et l'autre prend la forme d'une vision unificatrice, embrassant une vaste multitude de phénomènes et de situations qui jusque là apparaissaient comme séparés les uns des autres. Je vois là une parenté d'esprit évidente entre son œuvre \footnote{Je ne prétends nullement être familier de l'œuvre d'Einstein. En fait, je n'ai lu aucun de ses travaux, et ne connais ses idées que par ouï-dire et très approximativement. J'ai pourtant l'impression de discerner "la forêt", même si je n'ai jamais eu à faire l'effort de scruter aucun de ses arbres...} et la mienne.

Cette parenté ne me semble nullement contredite par une différence de "substance" évidente. Comme je l'ai déjà laissé entendre tantôt, la mutation einsteinienne concerne la notion d'espace physique, alors qu'Einstein puise dans l'arsenal des notions mathématiques déjà connues, sans avoir jamais besoin de l'élargir, voire de le bouleverser. Sa contribution a consisté à dégager, parmi les structures mathématiques connues de son temps, celles qui étaient le mieux aptes à \footnote{Pour des commentaires sur le qualificatif "moribond", voir une précédente note de bas de page (note page 55).} servir de "modèles" au monde des phénomènes physiques, en lieu et place du modèle moribond légué par ses devanciers. En ce sens, son œuvre a bien été celle d'un physicien, et au delà, celle d'un "philosophe de la nature", au sens où l'entendaient Newton et ses contemporains. Cette dimension "philosophique" est absente de mon œuvre mathématique, où je n'ai jamais été amené à me poser de question sur les relations éventuelles entre les constructions conceptuelles "idéales", s'effectuant dans l'Univers des choses mathématiques, et les phénomènes qui ont lieu dans l'Univers physique (voire même, les événements vécus se déroulant dans la psyché). Mon œuvre a été celle d'un mathématicien, se détournant délibérément de la question des "applications" (aux autres sciences), ou des "motivations" et des racines psychiques de mon travail. D'un mathématicien, en plus, porté par son génie très particulier à élargir sans cesse l'arsenal des notions à la base même de son art. C'est ainsi que j'ai été amené, sans même m'en apercevoir et comme en jouant, à bouleverser la notion la plus fondamentale de toutes pour le géomètre : celle d'espace (et celle de "variété"), c'est à dire notre conception du "lieu" même où vivent les êtres géométriques.

La nouvelle notion d'espace (comme une sorte d' "espace généralisé", mais où les points qui sont censés former l' "espace" ont plus ou moins disparu) ne ressemble en rien, dans sa substance, à la notion apportée par Einstein en physique (nullement déroutante, elle, pour le mathématicien). La comparaison s'impose par contre avec la mécanique quantique découverte par Schrödinger \footnote{Je crois comprendre (par des échos qui me sont revenus de divers côtés) qu'on considère généralement qu'il y a eu en ce siècle trois "révolutions" ou grands bouleversements en physique : la théorie d'Einstein, la découverte de la radio-activité par les Curie, et l'introduction de la mécanique quantique par Schrödinger.}. Dans cette mécanique nouvelle, le "point matériel" traditionnel disparaît, pour être remplacé par une sorte de "nuage probabiliste", plus ou moins dense d'une région de l'espace ambiant à l'autre, suivant la "probabilité" pour que le point se trouve dans cette région. On sent bien, dans cette optique nouvelle, une "mutation" plus profonde encore dans nos façons de concevoir les phénomènes mécaniques, que dans celle incarnée par le modèle d'Einstein - une mutation qui ne consiste pas à remplacer simplement un modèle mathématique un peu étroit aux entournures, par un autre similaire mais taillé plus large ou mieux ajusté. Cette fois, le modèle nouveau ressemble si peu aux bons vieux modèles traditionnels, que même le mathématicien grand spécialiste de mécanique a dû se sentir dépaysé soudain, voire perdu (ou outré...). Passer de la mécanique de Newton à celle d'Einstein doit être un peu, pour le mathématicien, comme de passer du bon vieux dialecte provençal à l'argot parisien dernier cri. Par contre, passer à la mécanique quantique, j'imagine, c'est passer du français au chinois.

Et ces "nuages probabilistes", remplaçant les rassurantes particules matérielles d'antan, me rappellent étrangement les élusifs "voisinages ouverts" qui peuplent les topos, tels des fantômes évanescents, pour entourer des "points" imaginaires, auxquels continue à se raccrocher encore envers et contre tous une imagination récalcitrante...

