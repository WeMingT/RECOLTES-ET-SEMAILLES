\section{Mutation de la notion d'espace - ou le souffle et la foi}

La notion de schéma constitue un vaste élargissement de la notion de "variété algébrique", et à ce titre elle a renouvelé de fond en comble la géométrie algébrique léguée par mes devanciers. Celle de topos constitue une extension insoupçonnée, pour mieux dire, une métamorphose de la notion d'espace. Par là, elle porte la promesse d'un renouvellement semblable de la topologie, et au delà de celle-ci, de la géométrie. Dès à présent d'ailleurs, elle a joué un rôle crucial dans l'essor de la géométrie nouvelle (surtout à travers les thèmes cohomologiques $\ell$-adique et cristallin qui en sont issus, et à travers eux, dans la démonstration des conjectures de Weil). Comme sa soeur aînée (et quasi-jumelle), elle possède les deux caractères complémentaires essentiels pour toute généralisation fertile, que voici.

Primo, la nouvelle notion n'est pas trop vaste, en ce sens que dans les nouveaux "espaces" (appelés plutôt "topos", pour ne pas indisposer des oreilles délicates \footnote{Le nom "topos" a été choisi (en association avec celui de "topologie", ou "topologique") pour suggérer qu'il s'agit de "l'objet par excellence" auquel s'applique l'intuition topologique. Par le riche nuage d'images mentales que ce nom suscite, il faut le considérer comme étant plus ou moins l'équivalent du terme "espace" (topologique), avec simplement une insistance plus grande sur la spécificité "topologique" de la notion. (Ainsi, il y a des "espaces vectoriels", mais pas de "topos vectoriels" jusqu'à nouvel ordre !) Il s'impose de garder les deux expressions conjointement, chacune avec sa spécificité propre.}), les intuitions et les constructions "géométriques" les plus essentielles \footnote{Parmi ces "constructions", il y a notamment celle de tous les "invariants topologiques" familiers, y compris les invariants cohomologiques. Pour ces derniers, j’avais fait tout ce qu’il fallait dans l’article déjà cité ("Tohoku" 1955), pour pouvoir leur donner un sens pour tout "topos".}, familières pour les bons vieux espaces d'antan, peuvent se transposer de façon plus ou moins évidente. Autrement dit, on dispose pour les nouveaux objets de toute la riche gamme des images et associations mentales, des notions et de certaines au moins de techniques, qui précédemment restaient restreintes aux objets ancien style.

Et secundo, la nouvelle notion est en même temps assez vaste pour englober une foule de situations qui, jusque là, n'étaient pas considérées comme donnant lieu à des intuitions de nature "topologico-géométrique" - aux intuitions, justement, qu'on avait réservées par le passé aux seuls espaces topologiques ordinaires (et pour cause...).

La chose cruciale ici, dans l'optique des conjectures de Weil, c'est que la nouvelle notion est assez vaste en effet, pour nous permettre d'associer à tout "schéma" un tel "espace généralisé" ou "topos" (appelé le "topos étale" au schéma envisagé). Certains "invariants cohomologiques" de ce topos (tout ce qu'il y a de "bébêtes" !) semblaient alors avoir une bonne chance de fournir "ce dont on avait besoin" pour donner tout leur sens à ces conjectures, et (qui sait !) de fournir peut-être les moyens de les démontrer.

C'est dans ces pages que je suis en train d'écrire que, pour la première fois dans ma vie de mathématicien, je prends le loisir d'évoquer (ne serait-ce qu'à moi-même) l'ensemble des maître-thèmes et des grandes idées directrices dans mon oeuvre mathématique. Cela m'amène à mieux apprécier la place et la portée de chacun de ces thèmes, et des "points de vue" qu'ils incarnent, dans la grande vision géométrique qui les unit et dont ils sont issus. C'est par ce travail que sont apparues en pleine lumière les deux idées novatrices névralgiques dans le crémier et puissant essor de la géométrie nouvelle : l'idée des schémas, et celle des topos.

C'est la deuxième de ces idées, celle des topos, qui à présent m'apparaît comme la plus profonde des deux. Si d'aventure, vers la fin des années cinquante, je n'avais pas retroussé mes manches, pour développer obstinément jour après jour, tout au long de douze longues années, un "outil schématique" d'une délicatesse et d'une puissance parfaites - il me semblerait quasiment impensable pourtant que dans les dix ou vingt ans déjà qui ont suivi, d'autres que moi auraient pu à la longue s'empêcher d'introduire à la fin des fins (fut-ce à leur corps défendant) la notion qui visiblement s'imposait, et de dresser tant bien que mal tout au moins quelques vétustés baraquements en "préfab", à défaut des spacieuses et confortables demeures que j’ai eu à coeur d'assembler pierre par pierre et de monter de mes mains. Par contre, je ne vois personne d'autre sur la scène mathématique, au cours des trois décennies écoulées, qui aurait pu avoir cette naïveté, ou cette innocence, de faire (à ma place) cet autre pas crucial entre tous, introduisant l'idée si enfantine des topos (ou ne serait-ce que celle des "sites"). Et, à supposer même cette idée-là déjà gracieusement fournie, et avec elle la timide promesse qu'elle semblait receler - je ne vois personne d'autre, que ce soit parmi mes amis d'antan ou parmi mes élèves, qui aurait eu le souffle, et surtout la foi, pour mener à terme cette humble idée \footnote{(A l’intention du lecteur mathématicien.) Quand je parle de "mener à terme cette humble idée", il s'agit de l'idée de la cohomologie étale comme approche vers les conjectures de Weil. C'est inspiré par ce propos que j'avais découvert la notion de site en 1958, et que cette notion (ou la notion très voisine de topos), et le formalisme cohomologique étale, ont été développé entre 1962 et 1966 sous mon impulsion (avec l'assistance de quelques collaborateurs dont il sera question en lieu).

Quand je parle de "souffle" et de "foi", il s'agit là des qualités de nature "non-technique", et qui ici m'apparaissent bien comme les qualités essentielles. A un autre niveau, je pourrais y ajouter aussi ce que j'appellerais le "fuir cohomologique", c'est-à-dire le genre de fuir qui s'était développé en moi pour l'édification des théories cohomologiques. J'avais cru le communiquer à mes élèves cohomologistes. Avec un recul de dix-sept ans après mon départ du monde mathématique, je constate qu'il ne s'est conservé en aucun d'eux.} (si dérisoire en apparence, alors que le but semblait infiniment lointain...) : depuis ses premiers débuts balbutiants, jusqu'à la pleine maturité de la "maîtrise de la cohomologie étale", en quoi elle a fini par s'incarner entre mes mains, au cours des années qui ont suivi.




