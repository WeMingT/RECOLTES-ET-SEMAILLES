\section{Forme et structure - ou la voie des choses}

Sans l'avoir prévu, cet "avant-propos" a fini, de fil en aiguille, par devenir une sorte de présentation en règle de mon oeuvre, à l'intention (surtout) du lecteur non mathématicien. Trop engagé déjà pour pouvoir encore reculer, il ne me reste plus qu'à terminer "les présentations"! Je voudrais essayer tant bien que mal de dire au moins quelques mots sur la substance de ces mirifiques "grandes idées" (ou de ces "maître-thèmes") que j'ai fait miroiter dans les pages précédentes, et sur la nature de cette fameuse "vision" en quoi ces idées maîtresses sont censées venir confluer. Faute de pouvoir faire appel à un langage tant soit peu technique, je ne pourrai sans doute que faire passer une image d'un flou extrême (si tant est que quelque chose veuille bien "passer" en effet. . . \footnote{Que cette image doive rester "foue" n'empêche nullement que cette image ne soit fi dèle, et qu'elle ne restitue bel et bien quelque chose de l'essence de ce qui est regardé (en l'occurrence, mon oeuvre). Inversement, une image a beau être nette, elle peut fort bien être distordue, et de plus, n'inclure que l'accessoire et manquer entièrement l'essentiel. Aussi, si tu "accroches" à ce que je vois à dire sur mon oeuvre (et sûrement alors quelque chose de l'image en moi "passera" bel et bien), tu pourras te flutter d'avoir mieux saisi ce qui fait l'essentiel dans mon oeuvre, qu'aucun peut-être de mes savants collègues!}).

Traditionnellement, on distingue trois types de "qualités" ou d' "aspects" des choses de l' Univers, qui soient objet de la réflexion mathématique : ce sont le nombre \footnote{Il est entendu ici qu'il s'agit des "nombres" dits "entiers naturels" $0,1,2,3$ etc, ou (à la rigueur) des nombres (tels les nombres fractionnaires) qui s'expriment à l'aide de ceux-ci par des opérations de nature élémentaire. Ces nombres ne se prêtent pas, comme les "nombres réels", à mesurer une grandeur susceptible de variation continue, telle la distance entre deux points variables sur une droite, dans un plan ou dans l'espace.}, la grandeur, et la forme. On peut aussi les appeler l'aspect "arithmétique", l'aspect "métrique" (ou "analytique"), et l'aspect "géométrique" des choses. Dans la plupart des situations étudiées dans la mathématique, ces trois aspects sont présents simultanément et en interaction étroite. Cependant, le plus souvent, il y a une prédominance bien marquée de l'un des trois. Il me semble que chez la plupart des mathématiciens, il est assez clair (pour ceux qui les connaissent, ou qui sont au courant de leur oeuvre) quel est leur tempérament de base, s'ils sont "arithméticiens", "analystes", ou "géomètres" - et ceci, alors même qu'ils auraient beaucoup de cordes à leur violon, et qu'ils auraient travaillé dans tous les registres et diapasons imaginables.

Mes premières et solitaires réflexions, sur la théorie de la mesure et de l'intégration, se placent sans ambiguïté possible dans la rubrique "grandeur", ou "analyse". Et il en est de même du premier des nouveaux thèmes que j'ai introduits en mathématique (lequel m'apparaît de dimensions moins vastes que les onze autres). Que je sois entré dans la mathématique par le "biais" de l'analyse m'apparaît comme dû, non pas à mon tempérament particulier, mais à ce qu'on peut appeler une "circonstance fortuite" : c'est que la lacune la plus énorme, pour mon esprit épris de généralité et de rigueur, dans l'enseignement qui m'était proposé au lycée comme à l'université, se trouvait concerner l'aspect "métrique" ou "analytique" des choses.

L'année 1955 marque un tournant crucial dans mon travail mathématique : celui du passage de l' "analyse" à la "géométrie". Je me rappelle encore de cette impression saisissante (toute subjective certes), comme si je quittais des steppes arides et revêches, pour me retrouver soudain dans une sorte de "pays promis" aux richesses luxuriantes, se multipliant à l'infini partout où il plaît à la main de se poser, pour cueillir ou pour fouiller... Et cette impression de richesse accablante, au delà de toute mesure \footnote{J'ai utilisé l'association de mots "accablant, au delà de toute mesure", pour rendre tant bien que mal l'expression en allemand "überwältigend", et son équivalent en anglais "overwhelming". Dans la phrase précédente, l'expression (inadéquate) "impression saisissante" est à comprendre aussi avec cette nuance-là : quand les impressions et sentiments suscités en nous par la confrontation à une splendeur, à une grandeur ou à une beauté hors du commun, nous submergent soudain, au point que toute velléité d'exprimer ce que nous ressentons semble comme anéantie d'avance.}, n'a fait que se confirmer et s'approfondir au cours des ans, jusqu'à aujourd'hui même.

C'est dire que s'il y a une chose en mathématique qui (depuis toujours sans doute) me fascine plus que toute autre, ce n'est ni "le nombre", ni "la grandeur", mais toujours la forme. Et parmi les mille-et-un visages que choisit la forme pour se révéler à nous, celui qui m'a fasciné plus que tout autre et continue à me fasciner, c'est la structure cachée dans les choses mathématiques.

La structure d'une chose n'est nullement une chose que nous puissions "inventer". Nous pouvons seulement la mettre à jour patiemment, humblement en faire connaissance, la "découvrir". S'il y a inventivité dans ce travail, et s'il nous arrive de faire oeuvre de forgeron ou d'infatigable bâtisseur, ce n'est nullement pour "façonner", ou pour "bâtir", des "structures". Celles-ci ne nous ont nullement attendues pour être, et pour être exactement ce qu'elles sont! Mais c'est pour exprimer, le plus fidèlement que nous le pouvons, ces choses que nous sommes en train de découvrir et de sonder, et cette structure réticente à se livrer, que nous essayons à tâtons, et par un langage encore balbutiant peut-être, à cerner. Ainsi sommes-nous amenés à constamment "inventer" le langage apte à exprimer de plus en plus finement la structure intime de la chose mathématique, et à "construire" à l'aide de ce langage, au fur et à mesure et de toutes pièces, les "théories" qui sont censées rendre compte de ce qui a été appréhendé et vu. Il y a là un mouvement de va-et-vient continuel, ininterrompu, entre l'appréhension des choses, et l'expression de ce qui est appréhendé, par un langage qui s'affine et se re-crée au fil du travail, sous la constante pression du besoin immédiat.

Comme le lecteur l'aura sans doute deviné, ces "théories", "construites de toutes pièces", ne sont autres aussi que ces "belles maisons" dont il a été question précédemment : celles dont nous héritons de nos devanciers et celles que nous sommes amenés à bâtir de nos propres mains, à l'appel et à l'écoute des choses. Et si j'ai parlé tantôt de l' "inventivité" (ou de l'imagination) du bâtisseur ou du forgeron, il me faudrait ajouter que ce qui en fait l'âme et le nerf secret, ce n'est nullement la superbe de celui qui dit : "je veux ceci, et pas cela !" et qui se complaît à décider à sa guise ; tel un piètre architecte qui aurait ses plans tout prêts en tête, avant d'avoir vu et senti un terrain, et d'en avoir sondé les possibilités et les exigences. Ce qui fait la qualité de l'inventivité et de l'imagination du chercheur, c'est la qualité de son attention, à l'écoute de la voix des choses. Car les choses de l' Univers ne se lassent jamais de parler d'elles-mêmes et de se révéler, à celui qui se soucie d'entendre. Et la maison la plus belle, "celle en laquelle apparaît l'amour de l'ouvrier, n'est pas celle qui est plus grande ou plus haute que d'autres. La belle maison est celle qui reflète fidèlement la structure et la beauté cachées des choses.


