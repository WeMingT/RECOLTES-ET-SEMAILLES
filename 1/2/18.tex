\section{L'enfant et la Mère}

Quand cet "avant-propos" a commencé à tourner à la promenade à travers mon œuvre de mathématicien, avec mon petit topo sur les "héritiers" (bon teint) et sur les "bâtisseurs" (incorrigibles), a commencé aussi à apparaître un nom pour cet avant-propos manqué : ce serait "L'enfant et le bâtisseur". Au cours des jours suivants, il devenait de plus en plus clair que "l'enfant" et "le bâtisseur" étaient un seul et même personnage. Ce nom est donc devenu, plus simplement, "L'enfant bâtisseur". Un nom, ma foi, qui ne manquait pas d'allure, et tout fait pour me plaire!

Mais voilà que la réflexion fait apparaître que cet altier "bâtisseur", ou (plus modestement) l'enfant-qui-joue-à-faire-des-maisons, ce n'était qu'un des visages du fameux enfant-qui-joue, lequel en avait deux. Il y a aussi l'enfant-qui-aime-à-explorer-les-choses, à aller fouiner et s'enfouir dans les sables ou dans les vases boueuses et sans nom, les endroits les plus impossibles et les plus saugrenus... Pour donner le change sans doute (ne serait-ce qu'à moi-même...), j'ai commencé par l'introduire sous le nom flamboyant de "pionnier", suivi de celui, plus terre-à-terre mais encore auréolé de prestige, d' "explorateur". C'était à se demander, entre le "bâtisseur" et le "pionnier-explorateur", lequel était le plus mâle, le plus alléchant des deux ! Pile ou face?

Et puis, en y regardant d'un peu plus près, voilà notre intrépide "pionnier" qui se trouve finalement être une fille (qu'il m'avait plu d'habiller en garçon) - une sœur des mares, de la pluie, des bruines et de la nuit, silencieuse et quasiment invisible à force de s'effacer dans l'ombre - celle que toujours on oublie (quand on ne fait mine de se gausser d'elle...). Et j'ai bien trouvé moyen moi aussi, pendant des jours et des jours, de l'oublier - de l'oublier doublement, pourrais-je dire : je n'avais voulu voir d'abord que le garçon (celui qui joue à faire des maisons...) - et même quand je n'ai pu m'empêcher, à force, de voir quand même l'autre, je l'ai vue encore en garçon, elle aussi...

Pour ce qui est du beau nom pour ma promenade, du coup il ne tient plus du tout. C'est un nom tout-enyang, tout "macho", un nom-qui-boite. Pour le faire tenir pas de guingois, il faudrait y faire figurer l'autre également. Mais, chose étrange, "l'autre" n'a pas vraiment de nom. Le seul qui colle tant soit peu, c'est "explorateur", mais c'est encore un nom de garçon, rien à faire. La langue ici est une garce, elle nous piège sans même qu'on s'en rende compte, visiblement de mèche avec des préjugés ancestraux.

On pourrait s'en tirer peut-être avec "L'enfant-qui-bâtit et l'enfant-qui-explore". En laissant non-dit que l'un est "garçon" et l'autre est "fille", et que c'est un seul et même enfant garçon-fille qui, en bâtissant explore, et en explorant, bâtit... Mais hier, en plus du double versant yin-yang de ce qui contemple et explore, et de ce qui nomme et construit, était apparu un autre aspect encore des choses.

L' Univers, le Monde, voire le Cosmos, sont choses étrangères au fond et très lointaines. Elles ne nous concernent pas vraiment. Ce n'est pas vers eux qu'au plus profond de nous-mêmes nous porte la pulsion de connaissance. Ce qui nous attire, c'est leur Incarnation tangible et immédiate, la plus proche, la plus "charnelle", chargée en résonances profondes et riche en mystère - Celle qui se confond avec les origines de notre être de chair, comme avec celles de notre espèce - et Celle aussi qui de tout temps nous attend, silencieuse et prête à nous accueillir, "à l'autre bout du chemin". C'est d'elle, la Mère, de Celle qui nous a enfanté comme elle a enfanté le Monde, que sourd la pulsion et que s'élancent les chemins du désir - et c'est à Sa rencontre qu'ils nous portent, vers Elle qu'ils s'élancent, pour retourner sans cesse et s'abîmer en Elle.

Ainsi, au détour du chemin d'une "promenade" imprévue, je retrouve à l'improviste une parabole qui me fût familière, et que j'avais un peu oubliée - la parabole de l'enfant et la Mère. On peut la voir comme une parabole pour "La Vie, à la quête d'elle-même". Ou, au niveau plus humble de l'existence individuelle, une parabole pour "l'être, à la quête des choses".

C'est une parabole, et c'est aussi l'expression d'une expérience ancestrale, profondément implantée dans la psyché - le plus puissant parmi les symboles originels qui nourrissent les couches créatrices profondes. Je crois y reconnaître, exprimé dans le langage immémorial des images archétypes, le souffle même du pouvoir créateur en l'homme, animant sa chair et son esprit, dans ses manifestations les plus humbles et les plus éphémères, comme les plus éclatantes et les plus durables.

Ce "souffle", tout comme l'image charnelle qui l'incarne, est la chose au monde la plus humble. C'est aussi la chose la plus fragile, et la plus ignorée de tous et la plus méprisée...

Et l'histoire des vicissitudes de ce souffle-là au cours de ton existence n'est autre que ton aventure, l' "aventure de connaissance" dans ta vie. La parabole sans paroles qui l'exprime est celle de l'enfant et la Mère.

Tu es l'enfant, issu de la Mère, abrité en Elle, nourri de Sa puissance. Et l'enfant s'élance de la Mère, la Toute-proche, la Bien-connue - à la rencontre de la Mère, l' Illimitée, à jamais Inconnue et pleine de mystère...

\hfill Fin de la "Promenade à travers une œuvre"

