\section{La magie des choses}

Quand j’étais gosse, j’aimais bien aller à l’école. On avait le même maître pour nous enseigner à lire et à écrire, le calcul, le chant (il jouait d’un petit violon pour nous accompagner), ou les hommes préhistoriques et la découverte du feu. Je ne me rappelle pas qu’on se soit jamais ennuyé à l’école, à ce moment. Il avait la magie des nombres, et celle des mots, des signes et des sons. Celle de la rime aussi, dans les chansons ou dans les petits poèmes. Il semblait y avoir dans la rime un mystère au delà des mots. Il en a été ainsi, jusqu’au jour ou quelqu’un m’a expliqué qu’il y avait un "truc" tout simple : que la rime, c’est tout simplement quand on fait se terminer par la même syllabe deux mouvements parlés consécutifs, qui du coup, comme par enchantement, deviennent des vers. C’était une révélation ! À la maison, où je trouvais du répondant autour de moi, pendant des semaines et des mois, je m’amusais à faire des vers. À un moment, je ne parlais plus qu’en rimes. Ça m’a passé, heureusement. Mais même aujourd’hui à l’occasion, il m’arrive encore de faire des poèmes - mais sans plus guère aller chercher la rime, si elle ne vient d’elle-même.

À un autre moment un copain plus âgé, qui allait déjà au lycée, m’a appris les nombres négatifs. C’était un autre jeu bien amusant, mais plus vite épuisé. Et il y avait les mots croisés - je passais des jours et des semaines à en fabriquer, de plus en plus imbriqués. Dans ce jeu se combinait la magie de la forme, et celle des signes et des mots. Mais cette passion-là m’a quitté, sans apparemment laisser de traces.

Au lycée, en Allemagne d’abord la première année, puis en France, j’étais bon élève, sans être pour autant "l’élève brillant". Je m’intéressais sans compter dans ce qui m’intéressait le plus, et avait tendance à négliger ce qui m’intéressait moins, sans trop me soucier de l’appréciation du "prof" concerné. La première année de lycée en France, en 1940, j’étais interne avec ma mère au camp de concentration, à Rieucros près de Mende. C’était la guerre, et on était des étrangers - des "indésirables", comme on disait. Mais l’administration du camp fermait un œil pour les gosses du camp, tout indésirables qu’il soient. On entrait et sortait un peu comme on voulait. J’étais le plus âgé, et le seul à aller au lycée, à quatre ou cinq kilomètres de là, qu’il neige ou qu’il vente, avec des chaussures de fortune qui toujours prenaient l’eau.

Je me rappelle encore la première "composition de maths", où le prof m’a collé une mauvaise note, pour la démonstration d’un des "trois cas d’égalité des triangles". Ma démonstration n’était pas celle du bouquin, qu’il suivait religieusement. Pourtant, je savais pertinemment que ma démonstration n’était ni plus ni moins convaincante que celle qui était dans le livre et dont je suivais l’esprit, à coups des sempiternels "on fait glisser telle figure de telle façon sur telle autre" traditionnels. Visiblement, cet homme qui m’enseignait ne se sentait pas capable de juger par ses propres lumières (ici, la validité d’un raisonnement). Il fallait qu’il se reporte à une autorité, celle d’un livre en l’occurrence. Ça devait m’avoir frappé, ces dispositions, pour que je me souvienne encore de ce petit incident. Par la suite et jusqu’à aujourd’hui encore, j’ai eu ample occasion pourtant de voir que de telles dispositions ne sont nullement l’exception, mais la règle quasi universelle. Il y aurait beaucoup à dire à ce sujet - un sujet que j’effleure plus d’une fois sous une forme ou sous une autre, dans Récoltes et Semailles. Mais aujourd’hui encore, que je le veuille ou non, je me sens décontenancé, chaque fois que je m’y trouve à nouveau confronté...

Les dernières années de la guerre, alors que ma mère restait internée au camp, j’étais dans une maison d’enfants du "Secours Suisse", pour enfants réfugiés, au Chambon sur Lignon. On était juifs la plupart, et quand on était averti (par la police locale) qu’il y aurait des rafles de la Gestapo, on allait se cacher dans les bois pour une nuit ou deux, par petits groupes de deux ou trois, sans trop nous rendre compte qu’il y allait bel et bien de notre peau. La région était bourrée de juifs cachés en pays cévenol, et beaucoup ont survécu grâce à la solidarité de la population locale.

Ce qui me frappait surtout au "Collège Cévenol" (où j’étais élève), c’était à quel point mes camarades s’intéressaient peu à ce qu’ils apprenaient. Quant à moi, je dévorais les livres de classe en début d’année scolaire, pensant que cette fois, on allait enfin apprendre des choses vraiment intéressantes ; et le reste de l’année j’employais mon temps du mieux que je pouvais, pendant que le programme prévu était débité inexorablement, à longueur de trimestres. On avait pourtant des profs sympa comme tout. Le prof d’histoire naturelle, Monsieur Friedel, était d’une qualité humaine et intellectuelle remarquable. Mais, incapable de "sévir", il se faisait chahuter à mort, au point que vers la fin de l’année, il devenait impossible de suivre encore, sa voix impuissante couvertes par le tohu-bohu général. C’est pour ça, si ça se trouve, que je ne suis pas devenu biologiste !

Je passais pas mal de mon temps, même pendant les leçons (chut...), à faire des problèmes de maths. Bienôt ceux qui se trouvaient dans le livre ne me suffisaient plus. Peut-être parce qu’ils avaient tendance, à force, à ressembler un peu trop les uns aux autres : mais surtout, je crois, parce qu’ils tombaient un peu trop du ciel, comme ça à la queue-leu-leu, sans dire d’où ils venaient ni où ils allaient. C’étaient les problèmes du livre, et pas mes problèmes. Pourtant, les questions vraiment naturelles ne manquaient pas. Ainsi, quand les longueurs $a$, $b$, $c$ des trois côtés d’un triangle sont connues, ce triangle est connu (abstraction faite de sa position), donc il doit y avoir une "formule" explicite pour exprimer, par exemple, l’aire du triangle comme fonction de $a$, $b$, $c$. Pareil pour un tétraèdre dont on connaît la longueur des six arêtes : quel est le volume ? Ce coup-là je crois que j’ai du peiner, mais j’ai dû finir par y arriver, à force. De toutes façons, quand une chose me "tenait", je ne comptais pas les heures ni les jours que j’y passais, quitte à oublier tout le reste ! (Et il en est ainsi encore maintenant...)

Ce qui me satisfaisait le moins, dans nos livres de maths, c’était l’absence de toute définition sérieuse de la notion de longueur (d’une courbe), d’aire (d’une surface), de volume (d’un solide). Je me suis promis de combler cette lacune, dès que j’en aurais le loisir. J’y ai passé le plus clair de mon énergie entre 1945 et 1948, alors que j’étais étudiant à l’Université de Montpellier. Les cours à la Fac n’étaient pas faits pour me satisfaire. Sans me l’être jamais dit en clair, je devais avoir l’impression que les profs se bornaient à répéter leurs livres, tout comme mon premier prof de maths au lycée de Mende. Aussi je ne mettais les pieds à la Fac que de loin en loin, pour me tenir au courant du sempiternel "programme". Les livres y suffisaient bien, au dit programme, mais il était bien clair aussi qu’ils ne répondaient nullement aux questions que je me posais. A vrai dire, ils ne les voyaient même pas, pas plus que mes livres de lycée ne les voyaient. Du moment qu’ils donnaient des recettes de calcul à tout venant, pour des longueurs, des aires et des volumes, à coups d’intégrales simples, doubles, triples (les dimensions supérieures à trois restant prudemment éludées...), la question d’en donner une définition intrinsèque ne semblait pas se poser, pas plus pour mes professeurs que pour les auteurs des manuels.

D’après l’expérience limitée qui était mienne alors, il pouvait bien sembler que j’étais le seul être au monde doué d’une curiosité pour les questions mathématiques. Telle était en tous cas ma conviction inexprimée, pendant ces années passées dans une solitude intellectuelle complète, et qui ne me pesait pas \footnote{Entre 1945 et 1948, je vivais avec ma mère dans un petit hameau à une dizaine de kilomètres de Montpellier, Maurargues (par Vendargues), perdu au milieu des vignes. (Mon père avait disparu à Auschwitz, en 1942.) On vivait chichement sur ma maigre bourse d’étudiant. Pour arriver à joindre les deux bouts, je faisais les vendanges chaque année, et après les vendanges, du vin de grappillage, que j’arrivais à couler tant bien que mal (en contrevenant, paraît-il, de la législation en vigueur...). De plus il y avait un jardin qui, sans avoir à le travailler jamais, nous fournissait en abondance figues, épipinards et même (vers la fi n) des tomates, plantées par un voisin complaisant au beau milieu d’une mer de splendides pavots. C’était la belle vie - mais parfois juste aux entournures, quand il s’agissait de remplacer une monture de lunettes, ou une paire de souliers usés jusqu’à la corde. Heureusement que pour ma mère, affligée et malade à la suite de son long séjour dans les camps, on avait droit à l’assistance médicale gratuite. Jamais on ne serait arrivés à payer un médecin...}A vrai dire, je crois que je n’ai jamais songé, pendant ce temps, à approfondir la question si oui ou non j’étais bien la seule personne au monde susceptible de s’intéresser à ce que je faisais. Mon énergie était suffisamment absorbée à tenir la gageure que je m’étais proposé : 

Il n’y avait aucun doute en moi que je ne pourrai manquer d’y arriver, de trouver le fin mot des choses, pour peu seulement que je me donne la peine de les scruter, en mettant noir sur blanc ce qu’elles me disaient, au fur et à mesure. L’intuition du volume, disons, était irrécusable. Elle ne pouvait qu’être le reflet d’une réalité élusive pour le moment, mais parfaitement tangible. C’est cette réalité qu’il s’agissait de saisir, tout simplement - un peu, peut-être, comme cette réalité magique de "la rime" avait été saisie, "comprise" un jour.

En m'y mettant, à l'âge de dix-sept ans et frais émoulu du lycée, je croyais que ce serait l'affaire de quelques semaines. Je sus resté dessus pendant trois ans. J'ai trouvé même moyen, à force, de louper un examen, en fin de deuxième année de Fac - celui de trigonométrie sphérique (dans l'option "astronomie approfondie", sic), à cause d'une erreur idiote de calcul numérique. (Je n'ai jamais été bien fort en calcul, il faut dire, une fois sorti du lycée...) C'est pour ça que j'ai dû rester encore une troisième année à Montpellier pour y terminer ma licence, au lieu d'aller à Paris tout de suite - le seul endroit, m'assurait-on, où j'aurais l'occasion de rencontrer les gens au courant de ce qui était considéré comme important, en maths. Mon informateur, Monsieur Soula, m'assurait aussi que les derniers problèmes qui s'étaient encore posés en maths avaient été résolus, il y avait vingt ou trente ans, par un dénommé Lebesgue. Il aurait développé justement (drôle de coïncidence, décidément!) une théorie de la mesure et de l'intégration, laquelle mettait un point final à la mathématique.

Monsieur Soula, mon prof de "calcul diff", était un homme bienveillant et bien disposé à mon égard. Je ne crois pas qu'il m'ait convaincu pour autant. Il devait déjà y avoir en moi la présence que la mathématique est une chose illimitée en étendue et en profondeur. La mer a-t-elle un "point final" ? Toujours est-il qu'à aucun moment je n'ai été effleuré par la pensée d'aller dénicher le livre de ce Lebesgue dont Monsieur Soula m'avait parlé, et qu'il n'a pas dû non plus jamais tenir entre les mains. Dans mon esprit, il n'y avait rien de commun entre ce que pouvait contenir un livre, et le travail que je faisais, à ma façon, pour satisfaire ma curiosité sur telles choses qui m'avaient intrigué.



