\section{Le tableau de moeurs}

En parlant de mon passé de mathématicien, et par la suite en découvrant (comme à mon corps défendant) les péripéties et les arcanes du gigantesque Enterrement de mon oeuvre, j'ai été amené, sans l'avoir cherché, à faire le tableau d'un certain milieu et d'une certaine époque - d'une époque marquée par la décomposition de certaines des valeurs qui donnaient un sens au travail des hommes. C'est l'aspect "tableau de moeurs", brossé autour d'un "fait divers" sans doute unique dans les annales de "la Science". Ce que j'ai dit précédemment dit assez clairement, je pense, que tu ne trouveras pas dans Récoltes et Semailles un "dossier" concernant une certaine "affaire" peu ordinaire, histoire de te mettre au courant vite fait. Tel ami pourtant à la recherche du dossier, est passé yeux fermés et sans rien voir, à côté de presque tout ce qui fait la substance et la chair de Récoltes et Semailles.

Comme je l'explique de façon beaucoup plus circonstanciée dans la Lettre, "l'enquête" (ou le "tableau de moeurs") se poursuit surtout au cours des parties II et IV, "L' Enterrement (1) - ou la robe de l' Empereur de Chine" et "L' Enterrement (3) - ou les Quatre Opérations". Au fil des pages, j'y tire au jour obstinément, l'un après l'autre, une multitude de faits juteux (à dire le moins), que j'essaye tant bien que mal de "caser" au fur et à mesure. Petit à petit, ces faits s'assemblent dans un tableau d'ensemble qui progressivement sort des brumes, en des couleurs de plus en plus vives, avec des contours de plus en plus nets. Dans ces notes au jour le jour, les "faits bruts" qui viennent d'apparaître se mélangent inextricablement à des réminiscences personnelles, et à des commentaires et des réflexions de nature psychologique, philosophique, voire même (occasionnellement) mathématique. C'est comme ça et je n'y puis rien !

À partir du travail que j'ai fait, qui m'a tenu en haleine pendant plus d'une année, constituer un dossier, en style "conclusions d'enquête", devrait représenter un travail supplémentaire de l'ordre de quelques heures ou de quelques jours, selon la curiosité et l'exigence du lecteur intéressé. J'ai bien essayé à un moment de le constituer, le fameux dossier. C'était quand j'ai commencé à écrire une note qui devait s'appeler "Les Quatre Opérations"\footnote{La note prévue à finir par éclater en la partie IV (de même nom "Les quatre opérations") de Récoltes et Semailles, comprenant dans les 70 notes s'étendant sur bien quatre cent pages.}. Et puis non, il y a rien eu à faire. J'y arrivais pas! Ce n'est pas là mon style d'expression, décidément, et sur mes vieux jours moins que jamais. Et j'estime à présent, avec Récoltes et Semailles, en avoir assez fait pour le bénéfice de la "communauté mathématique", pour laisser sans remords à d'autres que moi (s'il s'en trouve parmi mes collègues qui se sentiraient concernés) le soin de constituer le "dossier" qui s'impose.


