\section{Les héritiers et le bâtisseur}

Il est temps que je dise quelques mots ici sur mon oeuvre mathématique, qui a pris dans ma vie et y garde (à ma propre surprise) une place importante. Plus d'une fois dans Récoltes et Semailles je reviens sur cette oeuvre - parfois d'une façon clairement intelligible à chacun, et en d'autres moments en des termes tant soit peu techniques\footnote{Il y a également ici et là, en plus d'aperçus mathématiques sur mon oeuvre passée, des passages contenant aussi des développements mathématiques nouveaux. Le plus long est "Les cinq photos (cristaux et $\mathscr{D}$-Modules)" dans ReS IV, note n$^{\circ}$ 171 (ix).}. Ces derniers passages vont en grande partie passer "par dessus la tête" non seulement du "profane", mais même du collègue mathématicien qui ne serait plus ou moins "dans le coup" des maths dont il y est question. Tu peux bien sûr sauter sans plus les passages qui te paraîtront de nature un peu trop "calée". Comme tu peux aussi les parcourir, et saisir peut-être au passage un reflet de la "mystérieuse beauté" (comme m'écrivait un ami non mathématicien) du monde des choses mathématique, surgissant comme autant d' "étranges îlots inaccessibles" dans les vastes eaux mouvantes de la réflexion...

La plupart des mathématiciens, je l'ai dit tantôt, sont portés à se cantonner dans un cadre conceptuel, dans un "Univers" fixé une bonne fois pour toutes - celui, essentiellement, qu'ils ont trouvé "tout fait" au moment où ils ont fait leurs études. Ils sont comme les héritiers d'une grande et belle maison toute installée, avec ses salles de séjour et ses cuisines et ses ateliers, et sa batterie de cuisine et un outillage à tout venant, avec lequel il y a, ma foi, de quoi cuisiner et bricoler. Comment cette maison s'est construite progressivement, au cours des générations, et comment et pourquoi ont été conçus et façonnés tels outils (et pas d'autres...), pourquoi les pièces sont agencées et aménagées de telle façon ici, et de telle autre là - voilà autant de questions que ces héritiers ne songeraient pas à se demander jamais. C'est ça "l' Univers", le "donné" dans lequel il faut vivre, un point c'est tout ! Quelque chose qui paraît grand (et on est loin, le plus souvent, d'avoir fait le tour de toutes ses pièces), mais familier en même temps, et surtout : immuable. Quand ils s'affairent, c'est pour entretenir et embellir un patrimoine : réparer un meuble bancal, crépir une façade, affûter un outil, voire même parfois, pour les plus entreprenants, fabriquer à l'atelier, de toutes pièces, un meuble nouveau. Et il arrive, quand ils s'y mettent tout entier, que le meuble soit de toute beauté, et que la maison toute entière en paraisse embellie.

Plus rarement encore, l'un d'eux songera à apporter quelque modification à un des outils de la réserve, ou même, sous la pression répétée et insistante des besoins, d'en imaginer et d'en fabriquer un nouveau. Ce faisant, c'est tout juste s'il ne se confondra pas en excuses, pour ce qu'il ressent comme une sorte d'enfreinte à la piété due à la tradition familiale, qu'il a l'impression de bousculer par une innovation insolite.

Dans la plupart des pièces de la maison, les fenêtres et les volets sont soigneusement clos - de peur sans doute que ne s'y engouffre un vent qui viendrait d'ailleurs. Et quand les beaux meubles nouveaux, l'un ici et l'autre là, sans compter la progéniture, commencent à encombrer des pièces devenues étroites et à envahir jusqu'aux couloirs, aucun de ces héritiers-là ne voudra se rendre compte que son Univers familier et douillet commence à se faire un peu étroit aux entournures. Plutôt que de se résoudre à un tel constat, les uns et les autres préféreront se faufiler et se coincer tant bien que mal, qui entre un buffet Louis XV et un fauteuil à bascule en rotin, qui entre un marmot morveux et un sarcophage égyptien, et tel autre enfin, en désespoir de cause, escaladera de son mieux un monceau hétéroclite et croulant de chaises et de bancs...

Le petit tableau que je viens de brosser n'est pas spécial au monde des mathématiciens. Il illustre des conditionnements invétérés et immémoriaux, qu'on rencontre dans tous les milieux et dans toutes les sphères de l'activité humaine, et ceci (pour autant que je sache) dans toutes les sociétés et à toutes les époques. J'ai eu occasion déjà d'y faire allusion, et je ne prétends nullement en être exempt moi-même. Comme le montrera mon témoignage, c'est le contraire qui est vrai. Il se trouve seulement qu'au niveau relativement limité d'une activité créatrice intellectuelle, j'ai été assez peu touché\footnote{J'en vois la cause principale dans un certain climat propice qui a entouré mon enfance jusqu'à l'âge de cinq ans. Voir à ce sujet la note "L'innocence" (ReS III, n$^{\circ}$ 107).} par ce conditionnement-là, qu'on pourrait appeler la "cécité culturelle" - l'incapacité de voir (et de se mouvoir) en dehors de l' "Univers" fixé par la culture environnante.

Je me sens faire partie, quant à moi, de la lignée des mathématiciens dont la vocation spontanée et la joie est de construire sans cesse des maisons nouvelles\footnote{Cette image archétype de la "maison" à construire, fait surface et se trouve formulée pour la première fois dans la note "Yin le Serviteur, et les nouveaux maîtres" (ReS III, n$^{\circ}$ 135).}. Chemin faisant, ils ne peuvent s'empêcher d'inventer aussi et de façonner au fur et à mesure tous les outils, ustensiles, meubles et instruments requis, tant pour construire la maison depuis les fondations jusqu'au faîte, que pour pourvoir en abondance les futures cuisines et les futurs ateliers, et installer la maison pour y vivre et y être à l'aise. Pourtant, une fois tout posé jusqu'au dernier chêneau et au dernier tabouret, c'est rare que l'ouvrier s'attarde longuement dans ces lieux, où chaque pierre et chaque chevron porte la trace de la main qui l'a travaillé et posé. Sa place n'est pas dans la quiétude des univers tout faits, si accueillants et si harmonieux soient-ils - qu'ils aient été agencés par ses propres mains, ou par ceux de ses devanciers. D'autres tâches déjà l'appelant sur de nouveaux chantiers, sous la poussée impérieuse de besoins qu'il est peut-être le seul à sentir clairement, ou (plus souvent encore) en devançant des besoins qu'il est le seul à pressentir. Sa place est au grand air. Il est l'ami du vent et ne craint point d'être seul à la tâche, pendant des mois et des années et, s'il le faut, pendant une vie entière, s'il ne vient à la rescousse une relève bienvenue. Il n'a que deux mains comme tout le monde, c'est sûr - mais deux mains qui à chaque moment devinent ce qu'elles ont à faire, qui ne répugnent ni aux plus grosses besognes, ni aux plus délicates, et qui jamais ne se lassent de faire et de refaire connaissance de ces choses innombrables qui les appellent sans cesse à les connaître. Deux mains c'est peu, peut-être, car le Monde est infini. Jamais elles ne l'épuiseront! Et pourtant, deux mains, c'est beaucoup...

Moi qui ne suis pas fort en histoire, si je devais donner des noms de mathématiciens dans cette lignée-là, il me vient spontanément ceux de Galois et de Riemann (au siècle dernier) et celui, de Hilbert (au début du présent siècle). Si j'en cherche un représentant parmi les aînés qui m'ont accueilli à mes débuts dans le monde mathématique\footnote{Je parle de ces débuts dans la section "L'étranger bienvenu" (ReS I, n$^{\circ}$ 9).}, c'est le nom de Jean Leray qui me vient avant tout autre, alors que mes contacts avec lui sont pourtant restés des plus épisodiques \footnote{Cela n'empêche que j'ai été (à la suite de H. Cartan et J.P. Serre) un des principaux utilisateurs et promoteurs d'une des grandes notions novatrices introduites par Leray, celle de faisceau, laquelle a été un des outils essentiels à travers toute mon oeuvre de géomètre. C'est elle aussi qui m'a fourni la clef pour l'élargissement de la notion d'espace (topologique) en celle de topos, dont il sera question plus bas.
Leray diffère d'ailleurs du portrait que j'ai tracé du "bâtisseur", me semble-t-il, en ceci qu'il ne semble pas être porté à "construire des maisons depuis les fondations jusqu'au faîte". Plutôt, il n'a pu s'empêcher d'amorcer des vastes fondations, en des lieux auxquels personne n'aurait songé, tout en laissant à d'autres le soin de les terminer et de bâtir dessus, et, une fois la maison construite, de s'installer dans les lieux (ne fût-ce que pour un temps)...}.

Je viens là d'esquisser à grands traits deux portraits : celui du mathématicien "casanier" qui se contente d'entretenir et d'embellir un héritage, et celui du bâtisseur-pionnier \footnote{Je viens, subrepticement et "par la bande", d'accoler là deux qualificatifs aux mâles résonances (celui de "bâtisseur" et celui de "pionnier"), lesquels expriment pourtant des aspects bien différents de la pulsion de découverte, et de nature plus délicate que ces noms ne sauraient l'évoquer. C'est ce qui va apparaître dans la suite de cette promenade-réflexion, dans l'étape "A la découverte de la Mère - ou les deux versants" ( $\mathrm{n}^{\circ} 17$ ).}, qui ne peut s'empêcher de franchir sans cesse ces "cercles invisibles et impérieux" qui délimitent un Univers \footnote{Du même coup d'ailleurs, et sans l'avoir voulu, il assigne à cet Univers ancien (sinon pour lui-même, du moins pour ses congénères moins mobiles que lui) des limites nouvelles, en de nouveaux cercles plus vastes certes, mais tout aussi invisibles et tout aussi impérieux que le furent ceux qu'ils ont remplacés.}. On peut les appeler aussi, par des noms un peu à l'emporte-pièce mais suggestifs, les "conservateurs" et les "novateurs". L'un et l'autre ont leur raison d'être et leur rôle à jouer, dans une même aventure collective se poursuivant au cours des générations, des siècles et des millénaires. Dans une période d'épanouissement d'une science ou d'un art, il n'y a entre ces deux tempéraments opposition ni antagonisme \footnote{Tel a été le cas notamment dans le monde mathématique, pendant la période (1948-1969) dont j'ai été un témoin direct, alors que je faisais moi-même partie de ce monde. Après mon départ en 1970, il semble y avoir eu une sorte de réaction de vaste envergure, une sorte de "consensus de dédain" pour les "idées" en général, et plus particulièrement, pour les grandes idées novatrices que j'avais introduites.}. Ils sont différents et ils se complètent mutuellement, comme se complètent la pâte et le levain.

Entre ces deux types extrêmes (mais nullement opposés par nature), on trouve bien sûr tout un éventail de tempéraments intermédiaires. Tel "casanier" qui ne songerait à quitter une demeure familière, et encore moins à aller se coltiner le travail d'aller en construire une autre Dieu sait où, n'hésitera pas pourtant, lorsque décidément ça commence à se faire étroit, à mettre la main à la truelle pour aménager une cave ou un grenier, surélever un étage, voire même, au besoin, adjoindre aux murs quelque nouvelle dépendance aux modestes proportions \footnote{La plupart de mes "aînés" (dont il est question p. ex. dans "Une dette bienvenue", Introduction, 10) correspondent à ce tempérament intermédiaire. J'ai pensé notamment à Henri Cartan, Claude Chevalley, André Weil, Jean-Pierre Serre, Laurent Schwartz. Sauf peut-être Weil, ils ont d'ailleurs tous accordé un "oeil de sympathie", sans "inquiétude ni réprobation secrètes", aux aventures solitaires dans lesquelles ils me voyaient m'embarquer.}. Sans être bâtisseur dans l'âme, souvent pourtant il regarde avec un oeil de sympathie, ou tout au moins sans inquiétude ni réprobation secrètes, tel autre qui avait partagé avec lui le même logis, et que voilà trimer à rassembler poutres et pierres dans quelque cambrousse impossible, avec les airs d'un qui y verrait déjà un palais...



