\section{L'aventure intérieure - ou mythe et témoignage}

Avant toute chose, Récoltes et Semailles est une réflexion sur moi-même et sur ma vie. Par là-même, c'est aussi un témoignage, et ceci de deux façons. C'est un témoignage sur mon passé, sur lequel porte le poids principal de la réflexion. Mais en même temps c'est aussi un témoignage sur le présent le plus immédiat - sur le moment même où j'écris, et où naissent les pages de Récoltes et Semailles au fil des heures, des nuits et des jours. Ces pages sont les fidèles témoins d'une longue méditation sur ma vie, telle qu'elle s'est poursuivie réellement (et se poursuit encore en ce moment même...).

Ces pages n'ont pas de prétention littéraire. Elles constituent un document sur moi-même. Je ne me suis permis d'y toucher (pour des retouches stylistiques occasionnelles, notamment) qu'à l'intérieur de limites très étroites\footnote{Ainsi, les rectifications éventuelles d'erreurs (matérielles, ou de perspective, etc) ne sont pas l'occasion de retouches du premier jet, mais se font dans des notes de bas de page, ou lors d'un "retour" ultérieur sur la situation examinée.}. S'il a une prétention, c'est celle seulement d'être vrai. Et c'est beaucoup.

Ce document, par ailleurs, n'a rien d'une "autobiographie". Tu n'y apprendras ni ma date de naissance (qui n'aurait guère d'intérêt que pour dresser une carte astrologique), ni les noms de ma mère et de mon père ou ce qu'ils faisaient dans la vie, ni les noms de celle qui fut mon épouse et d'autres femmes qui ont été importantes dans ma vie, ou ceux des enfants qui sont nés de ces amours, et ce que les uns et les autres ont fait de leur vie. Ce n'est pas que ces choses n'aient été importantes dans ma vie, et ne gardent une importance encore maintenant. Mais telle que cette réflexion sur moi-même s'est engagée et poursuivie, à aucun moment je ne me suis senti incité à m'engager tant soit peu dans une description de ces choses que je frôle ici et là, et encore moins, à aligner consciencieusement des noms et des chiffres. À aucun moment, il ne m'aurait semblé que cela pouvait ajouter quoi que ce soit au propos que je poursuivais en ce moment-là. (Alors que dans les quelques pages qui précèdent, j'ai été amené, comme malgré moi, à inclure peut-être plus de détails matériels sur ma vie que dans les mille pages qui vont suivre...)

Et si tu me demandes quel est donc ce "propos" que je poursuis à longueur de mille pages, je répondrai : c'est de faire le récit, et par là-même la découverte, de l'aventure intérieure qu'a été et qu'est ma vie. Ce récit-témoignage d'une aventure se poursuit en même temps sur les deux niveaux dont je viens de parler. Il y a l'exploration d'une aventure dans le passé, de ses racines et de son origine jusque dans mon enfance. Et il y a la continuation et le renouvellement de cette "même" aventure, au fil des instants et des jours alors que j'écris Récoltes et Semailles, en réponse spontanée à une interpellation violente me venant du monde extérieur\footnote{Pour des précisions au sujet de cette "interpellation violente", voir "Lettre", notamment sections 3 à 8.}.

Les faits extérieurs viennent alimenter la réflexion, dans la mesure seulement où ils suscitent et provoquent un rebondissement de l'aventure intérieure, ou contribuent à l'éclairer. Et l'enterrement et le pillage de mon oeuvre mathématique, dont il sera longuement question, a été une telle provocation. Elle a suscité en moi la levée en masse de réactions égotiques puissantes, et en même temps m'a révélé les liens profonds et ignorés qui continuent à me relier à l'oeuvre issue de moi.

Il est vrai que le fait que je fasse partie des "forts en maths" n'est pas forcément une raison (et encore moins une bonne raison) pour t'intéresser à mon aventure" particulière - ni le fait que j'aie eu des ennuis avec mes collègues, après avoir changé de milieu et de style de vie. Il ne manque d'ailleurs pas de collègue ni même d'amis, qui trouvent du plus grand ridicule d'étaler en public comme ils disent) ses "états d'âme". Ce qui compte, ce sont les "résultats". L' "âme", elle, c'est-à-dire cela en nous qui vit la "production" de ces résultats", ou aussi ses retombées de toutes sortes (tant dans la vie du "producteur", que dans celle de ses semblables), est objet de mésestime, voire d'une dérision ouvertement affichée. Cette attitude se veut expression d'une "modestie", j'y vois le signe d'une fuite, et un étrange dérèglement, promu par l'air même que nous respirons. Il est sûr que je n'écris pas pour celui frappé par cette sorte de mépris larvé de lui-même, qui lui fait dédaigner ce que j'ai de meilleur à lui offrir. Un mépris pour ce qui véritablement fait sa propre vie, et pour ce qui fait la mienne : les mouvements superficiels et profonds, grossiers ou subtils qui animent la psyché, cette "âme" justement qui vit l'expérience et qui y réagit, qui se fige ou qui s'épanouit, qui se replie ou qui apprend...

Le récit d'une aventure intérieure ne peut être fait que par celui qui la vit, et par nul autre. Mais alors même que le récit ne serait destiné qu'à soi-même, il est rare qu'il ne glisse dans l'ornière de la construction d'un mythe, dont le narrateur serait le héros. Un tel mythe naît, non de l'imagination créatrice d'un peuple et d'une culture, mais de la vanité de celui qui n'ose assumer une humble réalité, et qui se plaît à lui substituer une construction, œuvre de son esprit. Mais un récit vrai (s'il s'en trouve), d'une aventure telle qu'elle fut vécue vraiment, est chose de prix. Et ceci, non par un prestige qui (à tort ou à raison) entourerait le narrateur, mais par le seul fait d'exister, avec sa qualité de vérité. Un tel témoignage est précieux, qu'il vienne d'un homme de notoriété voire illustre, ou d'un petit employé sans avenir et chargé de famille, ou d'un criminel de droit commun.

Si un tel récit à une vertu pour autrui, c'est avant tout de le reconfronter à lui-même, à travers ce témoignage sans fard de l'expérience d'un autre. Ou aussi (pour le dire autrement) d'effacer peut-être en lui (et ne serait-ce que l'espace du temps que dure une lecture) ce mépris en lequel il tient sa propre aventure, et cette "âme" qui en est le passager et le capitaine...


