\section{Tous les chevaux du roi...}

Oui, la rivière est profonde, et vastes et paisibles sont les eaux de mon enfance, dans un royaume que j'ai cru quitter il y a longtemps. Tous les chevaux du roi y pourraient boire ensemble à l'aise et tout leur soûl, sans les épuiser! Elles viennent des glaciers, ardentes comme ces neiges lointaines, et elles ont la douceur de la glaise des plaines. Je viens de parler d'un de ces chevaux, qu'un enfant avait amené boire et qui a bu son content, longuement. Et j'en ai vu un autre venant boire un moment, sur les traces du même gamin si ça se trouve - mais là ça n'a pas traîné. Quelqu'un a dû le chasser. Et c'est tout, autant dire. Je vois pourtant des troupeaux innombrables de chevaux assoiffés qui errent dans la plaine - et pas plus tard que ce matin même leurs hennissements m'ont tiré du lit, à une heure indue, moi qui vais sur mes soixante ans et qui aime la tranquillité. Il n'y a rien eu à faire, il a fallu que je me lève. Ça me fait peine de les voir, à l'état de rosses efflanquées, alors que la bonne eau pourtant ne manque pas, ni les verts pâturages. Mais on dirait qu'un sortilège malveillant a été jeté sur cette contrée que j'avais connue accueillante, et condamné l'accès à ces eaux généreuses. Ou peut-être est-ce un coup monté par les maquignons du pays, pour faire tomber les prix qui sait? Ou c'est un pays peut-être où il n'y a plus d'enfants pour mener boire les chevaux, et où les chevaux ont soif, faute d'un gamin qui retrouve le chemin qui mène à la rivière...


