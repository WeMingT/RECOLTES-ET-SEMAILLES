\section{Amende honorable - ou l'esprit du temps (2)}

Dans la Lettre, je me suis suffisamment expliqué, je pense, sur l'esprit dans lequel j'ai écrit Récoltes et Semailles, pour qu'il soit bien clair que je ne prétends nullement y faire œuvre d'historien. Il s'agit d'un témoignage de bonne foi, concernant un vécu de première main, et d'une réflexion sur ce vécu. Témoignage et réflexion sont à la disposition de tous, y compris de l'historien, qui pourra l'utiliser comme un matériau parmi d'autres. C'est à lui qu'il appartient alors de soumettre ce matériau à une analyse critique, conforme aux canons de rigueur de son art.

Il convient, bien sûr, de distinguer entre les faits au sens restreint (les "faits bruts" ou "faits matériels"), et l' "évaluation" ou "interprétation" de ces faits, qui leur donne un sens, lequel n'est pas le même, pour un observateur (ou un coacteur) et pour un autre. Grosso-modo, on peut dire que l'aspect "témoignage" de Récoltes et Semailles concerne les faits, et que son aspect "réflexion" concerne leur interprétation, c'est à dire mon travail pour leur donner un sens. Parmi les "faits" formant le témoignage, je range également les "faits psychiques", et notamment les sentiments, associations et images de toutes sortes dont mon témoignage est le reflet, que ceux-ci aient lieu dans un passé plus ou moins reculé, ou au moment même de l'écriture.

Pour les faits que je décris ou dont je fais état dans Récoltes et Semailles, je distingue trois sortes de sources. Il y a les faits que me restitue le souvenir, plus ou moins précis ou plus ou moins flou d'une occasion à l'autre, et parfois déformé. À leur sujet, je puis me porter garant pour des dispositions de vérité au moment où j'écris, mais nullement de l'absence de toute erreur. Au contraire, j'ai eu l'occasion d'en relever un certain nombre, erreurs de détail que je signale en leur lieu par des notes de bas de page ultérieures. Il y a, d'autre part, les documents écrits, notamment des lettres et surtout des publications scientifiques en bonne et due forme, auxquelles je réfère à l'occasion avec toute la précision souhaitable. Il y a, enfin, le témoignage de tierces personnes. Parfois il vient en complément à mes propres souvenirs, me permettant de les raviver, de les préciser et, parfois, de les corriger. Dans certaines rares occasions (sur lesquelles je vais revenir tantôt), ce témoignage m'apporte des informations entièrement nouvelles par rapport à celles qui m'étaient déjà connues. Quand il m'arrive de me faire l'écho d'un tel témoignage, cela ne signifie pas que j'aie eu la possibilité d'en vérifier l'exactitude et le bien-fondé sur toute la ligne, mais simplement qu'il s'est inséré de façon suffisamment plausible dans le riche tissu de faits qui m'étaient connus de première main, pour entraîner ma conviction (à tort ou à raison...) que ce témoignage correspondait bien, pour l'essentiel, à la vérité.

Pour un lecteur attentif, je pense qu'il n'y aura aucune difficulté, à aucun moment, à faire "la part des choses" entre le compte rendu des faits et l'interprétation de ceux-ci, et (dans le premier cas) à discerner, parmi les trois sources que je viens de décrire, laquelle entre en jeu.

\begin{center}
    * \quad * \\
    *
    \end{center}

Quand j’ai fait allusion à l'instant au témoignage d'une tierce personne, dont je me suis fait l'écho sans avoir pu "en vérifier le bien-fondé sur toute la ligne", il s'agit de celui de Zoghman Mebkhout, au sujet de la vaste opération d'escamotage autour de son œuvre. Parmi les "faits matériels" dont je fais état dans Récoltes et Semailles, les seuls qui soient à présent sujet à controverse ou qui, selon mon propre jugement à présent, demandent rectification, sont certains des faits attestés par le seul témoignage de Mebkhout. Pour terminer ce post-scriptum, je tiens à présenter ici des commentaires critiques au sujet de la version de l' "affaire Mebkhout" présentée dans le tirage provisoire de Récoltes et Semailles. Des commentaires et des rectifications plus circonstanciés seront inclus, chacun et chacune en son lieu, dans l'édition imprimée (constituant le texte définitif de Récoltes et Semailles).

La "version Mebkhout" dont j'ai voulu me faire l'interprète, me semble consister pour l'essentiel en les deux thèses que voici :

\begin{enumerate}
    \item Entre 1972 et 1979, Mebkhout aurait été seul \footnote{Exception faite du théorème de constructibilité de Kashiwara de 1975, dont l'importance dans la théorie n'est nullement contestée. Mais selon la version de Mebkhout, ce serait là la seule et unique contribution de Kashiwara à la théorie en train de naître. Cette version (inexacte) était corroborée par l'absence d'autres publications de Kashiwara, où il aurait fait au moins allusion à certaines des idées maîtresses.}, dans l'indifférence générale et en s'inspirant de mon œuvre, à développer la "philosophie des $\mathscr{D}$-Modules", en tant que nouvelle théorie des "coefficients cohomologiques" en mon sens.
    \item  Il y aurait eu un consensus unanime, tant en France qu'au niveau international, pour escamoter son nom et son rôle dans cette théorie nouvelle, une fois que sa portée a commencé à être reconnue.
\end{enumerate}

Cette version était fortement documentée, d'une part par les publications de Mebkhout, tout à fait convaincantes, d'autre part par de nombreuses publications d'autres auteurs (et notamment, par celle des Actes du Colloque de Luminy de juin 1981), où le propos délibéré d'escamotage ne peut faire aucun doute. Enfin, les détails plus circonstanciés que Mebkhout m'a fournis ultérieurement (et dont je me fais l'écho dans la partie "L'Enterrement (3) - ou les Quatre Opérations"), sans être directement vérifiables, concordaient cependant entièrement avec une certaine ambiance générale, dont la réalité ne pouvait plus faire pour moi aucun doute.

Je viens d'avoir connaissance de plusieurs faits nouveaux \footnote{Je suis reconnaissant à Pierre Schapira et à Christian Houzel pour avoir bien voulu attirer mon attention sur ces faits, et sur le caractère tendancieux de ma présentation du différend Mebkhout-Kashiwara.}, qui montrent qu'il y a lieu de nuancer fortement le point $1^{\circ}$ ) ci-dessus. L'isolement dans lequel Mebkhout se trouvait \footnote{Cet isolement provenait avant tout de l'indifférence de mes ex-élèves pour les idées et les travaux de Mebkhout, qui faisait mine obstinément de s'inspirer d'un "ancêtre" voué à l'oubli par un consensus unanime...} était bel et bien réel, mais c'était un isolement relatif. Il y a eu en France les travaux de J.P. Ramis dans le même sujet (travaux dont Mebkhout ne m'a soufflé mot), et surtout, il apparaît que certaines idées importantes développées et menées à terme par Mebkhout, et dont il s'attribue la paternité, pourraient être dues à Kashiwara \footnote{La plus importante de ces idées est celle de la "correspondance" (pour utiliser le jargon nouveau style) dite "de Riemann-Hilbert" pour les $\mathscr{D}$-Modules. La conjecture pertinente a été prouvée par Mebkhout, et également (selon ce que m'affirme Schapira) par Kashiwara (alors que Mebkhout m'assurait que sa démonstration était la seule publiée). La question de la priorité pour la démonstration reste pour moi nébuleuse, et je renonce à passer le restant de mes jours à la tirer au clair...

Quant à l’énoncé-soeur en termes de $\mathscr{D}^{\infty}$ Modules, il ne semble pas y avoir le moindre doute que la paternité pour l'idée et pour la démonstration appartient bien à Mebkhout.}.

Du coup cela rend invraisemblable ou douteux certains des épisodes du différend Kashiwara-Mebkhout, tels qu'ils sont rapportés dans la version Mebkhout dont je me suis fait le (trop) fidèle interprète.

Il est hors de doute qu'au niveau du "travail sur pièces", comme aussi par la conception de certaines des idées qu'il a su mener à bonne fin, Mebkhout a été un des principaux pionniers de la nouvelle théorie des $\mathscr{D}$-Modules, peut-être même le principal pionnier; le seul en tous cas qui se soit investi corps et âme dans cette tâche-là, dont la portée véritable lui échappait encore, comme elle échappait à tous. Et il est vrai aussi que l'opération d'escamotage qui a eu lieu autour de cette œuvre, opération culminant avec le Colloque de Luminy, reste pour moi une des grandes disgrâces du siècle dans le monde mathématique. Mais il serait faux de prétendre (comme je l'ai fait de bonne foi) que Mebkhout ait été le seul à la tâche. Par contre, il a été le seul à avoir l'honnêteté et le courage de dire clairement l'importance de mes idées et de mon œuvre dans ses travaux et dans l'éclosion de la théorie nouvelle.

Ce n'est pas le lieu, dans ce post-scriptum, d'entrer dans plus de détails sur cette affaire - je le ferai en son lieu, y compris par des commentaires de nature à éclairer le contexte psychologique de la "version Mebkhout". Si le "contentieux Mebkhout-Kashiwara" revêt pour moi un intérêt, c'est dans la mesure seulement où il éclaire l'ambiance générale d'une époque. Et pour moi, jusque dans ses déformations même et par les forces qui ont joué pour les faire surgir, la "version Mebkhout" apparaît elle aussi, parmi d'autres matériaux moins contestables que j'apporte au "dossier d'une époque", un "signe des temps" éloquent.

Il me reste à faire amende honorable pour la légèreté, en présentant du différend Mebkhout-Kashiwara un tableau qui ne tenait compte que du témoignage et des documents fournis par Mebkhout, et ceci, comme si cette version ne pouvait faire l'objet d'aucun doute. Cette version présentait une tierce personne sous un jour ridicule, voire odieux, raison de plus pour faire preuve de prudence. Pour ma légèreté et pour ce manque de saine prudence, je présente bien volontiers ici à M. Kashiwara mes excuses les plus sincères.







