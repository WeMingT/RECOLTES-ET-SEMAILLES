\section{Quatre vagues dans un mouvement}

Comme de juste, cette lettre ne ressemble pas du tout à ce que j'avais prévu en m'y mettant. Je pensais surtout y faire un petit "topo" sur l'Enterrement : voilà ce qui s'est passé dans les grandes lignes, tu me croiras ou pas (moi-même j'ai eu du mal à le croire...), mais c'est bien ça pourtant, indubitable, même, que cela te plaise ou non, publications noir sur blanc tel périodique ou tel livre, telle date telle page, il n'y a qu'à regarder - d'ailleurs tout est dévissé par le menu dans Récoltes et Semailles ; voir "Quatre Opérations" telles notes - à prendre ou à laisser ! Et si tu préfères t'abstenir de me lire, d'autres s'en chargeront bien à ta place...

Finalement il n'y a rien eu de tout ça - et pourtant cette lettre en est déjà au cap des trente pages, alors que j'en prévoyais cinq ou six en tout et pour tout. Sans même que j’aie fait exprès, ce sont les choses essentielles que j'ai été amené à te dire, au fil des pages, alors que ce "sac" que j'avais été si impatient de vider (là bien en évidence pour le coup, aux premières pages !) il n'est toujours pas déballé ! Ça ne me chatouille même plus dans les doigts, l'envie s'est dissipée en chemin. J'ai compris que ce n'était pas ici le lieu...

A vrai dire, la partie IV de Récoltes et Semailles (et la plus longue de toutes), ayant nom "L'Enterrement (3)" ou "Les Quatre Opérations", est issue d'une "note" prévue initialement comme "un petit topo" justement, pour résumer dans les grandes lignes ce que m'avait révélé l'enquête-à-surprise (et en coup de vent) de l'année dernière, poursuivie dans la partie II ("L'Enterrement (1)", ou "La robe de l'Empereur de Chine"). Je pensais qu'il y en aurait pour une "note" de cinq ou dix pages, pas plus. Finalement, de fil en aiguille, cela a fait repartir l'enquête, il y en a eu pour près de quatre cents pages - près du double de la partie dont j'étais censé faire un résumé ou tirer un bilan ! Ça fait donc qu'il manque toujours le petit topo en question, alors que dans les six cents pages de Récoltes et Semailles sont consacrées à l'enquête sur l'Enterrement. C'est un peu idiot, c'est vrai. Mais il sera toujours temps de le rajouter dans une troisième partie à l'Introduction (qui n'en est plus à dix ou vingt pages près), avant de confier mes notes à un imprimeur.

Les cinq parties de Récoltes et Semailles (dont la dernière n'est pas terminée encore, et ne le sera sans doute pas avant quelques mois) représentent une alternance de (trois) vagues-"méditation" et de (deux) vagues"enquête". Il y a là comme un reflet, en raccourci, de ma vie de ces dernières neuf années, qui a consisté en une alternance, elle aussi, de "vagues" surgies des deux passions qui aujourd'hui dominent ma vie, la passion de la méditation et la passion mathématique. Et à vrai dire, les deux parties (ou "vagues") de Récoltes et Semailles que je viens de qualifier du nom à l'emporte-pièce ``enquête'', sont celles justement qui sont surgies directement de mon enracinement dans mon passé de mathématicien, nues par la passion mathématique en moi et par les attachements égotiques qui se sont enracinées en elle.

La première vague, ``Fatuité et Renouvellement'', est une première rencontre avec mon passé de mathématicien, débouchant sur une méditation sur mon présent, dont je viens de découvrir l'enracinement dans ce passé. Sans que cela ait été le moins du monde prémédité, certes, cette partie pose le ``ton de base'' pour toute la suite de Récoltes et Semailles, elle est comme une préparation intérieure, providentielle et indispensable, pour assumer la découverte de l'``Enterrement dans toute sa splendeur'' qui la suit de près, au cours de la deuxième vague, L' Enterrement (1) - ou la robe de l' Empereur de Chine''. Plus qu'une ``enquête'', à vrai dire, c'est bien là l'histoire de cette découverte au jour le jour, de son impact sur mon être, de mes efforts pour faire face à ce qui me dégringolait ainsi dessus sans crier gare, pour arriver à situer l'inconcevable en termes de mon vécu, de ce qui a fini par me devenir familier, le rendre intelligible tant bien que mal. Ce mouvement débouche sur un premier aboutissement provisoire, dans la note ``Le Fossoyeur - ou la Congrégation toute entière'' (n° 97), premier essai pour discerner une explication et un sens dans quelque chose qui, depuis des années déjà et maintenant de façon plus aiguë que jamais, prenait les allures d'un redoutable défi au bon sens !

Ce même deuxième mouvement débouche également sur un ``épisode maladie''\footnote{Cet épisode fait l'objet de deux notes ``L'incident - ou le corps et l'esprit et ``Le piège - ou facilité et épuisement'' (n° s 98, 99), ouvrant le ``Cortège XI'' nommé ``Le défunt (toujours pas décédé)''.}, me contraignant à un repos absolu et mettant fin pendant plus de trois mois à toute activité intellectuelle. C'était à un moment où je me croyais à nouveau sur le point d'avoir mené à terme Récoltes et Semailles (à des dernières tâches ``d'intendance'' près\ldots). En reprennant une activité normale, vers la fin septembre l'an dernier, et m'apprêtant à mettre enfin la dernière main à mes notes restées en détresse, je croyais toujours en avoir pour deux ou trois notes terminales à ajouter, y compris une au sujet de ``l'incident-santé'' par lequel je venais de passer. En fait, de semaine en semaine et de mois en mois, c'est mille pages encore qui sont venues - plus du double de ce qui était déjà écrit - et cette fois, il est bien clair que je n'ai toujours pas terminé\footnote{Toujours pas terminé'' - ne serait-ce que parce qu'il doit encore venir une partie V, qui n'est pas terminée au moment d'écrire ces lignes.} ! En fait, cette longue interruption, pendant laquelle j'avais perdu pratiquement le contact avec une substance qui était tout ce qu'il y a de chaude (et même brûlante) au moment de la quitter, a pratiquement forcé à revenir sur cette substance avec des yeux nouveaux, si je ne voulais me borner à ``boucler'' bêtement la fin dernière d'un ``programme'' avec lequel j'avais perdu un contact vivant.

C'est ainsi que naît la troisième vague dans le vaste mouvement qu'est Récoltes et Semailles - une longue ``vague-méditation'' sur le thème du yin et du yang, les versants ``ombre'' et ``lumière'' dans la dynamique des choses et dans l'existence humaine. Issue du désir d'une compréhension plus approfondie des forces profondes à l'œuvre dans l' Enterrement, cette méditation acquiert pourtant dès le début une autonomie et une unité propres, et se porte d'emblée vers ce qui est le plus universel, comme aussi vers ce qui est le plus intimement personnel. C'est au cours de cette méditation que je découvre cette chose (évidente à vrai dire, pour peu qu'on se pose la question), que dans ma démarche spontanée à la découverte des choses, que ce soit en mathématique ou ailleurs, le ``ton de base'' est ``yin'', ``féminin''; et aussi si surprenant que certainement à ce qui se passe le plus souvent, je suis resté fidèle à cette nature originelle en moi\footnote{Cette ``fidélité à ma nature originelle'' n'a nullement été totale d'ailleurs. Pendant longtemps, elle s'est bornée à mon travail mathématique, alors que partout ailleurs et notamment dans mes relations à autrui, je suivais le mouvement général en valorisant et donnant primauté aux traits en moi ressentis comme ``virils'', et en réprimant les traits ``féminins''. Il en est question de façon assez circonstanciée dans le groupe de notes ``Histoire d'une vie : un cycle en trois mouvements'' (n° 107-110), qui ouvre pratiquement la Clef du Yin et du Yang.}, sans jamais l'infléchir ou la corriger pour l'adapter à ce qui se passe le plus souvent, je suis resté fidèle à cette nature originelle en moi, sans jamais l'infléchir ou la corriger pour l'adapter à ce qui se passe le plus souvent, je suis resté fidèle à cette nature originelle en moi, sans jamais l'infléchir ou la corriger pour l'adapter à ce qui se passe le plus souvent, je suis resté fidèle à cette nature originelle en moi, sans jamais l'infléchir ou la corriger pour me conformer aux valeurs dominantes en honneur dans les milieux environnants. Cette découverte m'apparaît d'abord comme une simple curiosité. C'est peu à peu seulement qu'il se révèle pourtant comme une clef essentielle pour une compréhension de l'Enterrement. De plus - et c'est là une chose qui me paraît de plus grande portée encore - je vois maintenant très clairement et sans résidu du moindre doute ceci : que si, avec des dons intellectuels nullement exceptionnels, j'ai pu néanmoins constamment donner ma pleine mesure dans mon travail mathématique, et produire une œuvre et enfanter une vision vastes, puissantes et fécondes, ce n'est à rien d'autre qu'à cette fidélité que je le dois, à cette absence de tout souci de me conformer à des normes, grâce à quoi je m'abandonne avec une totale confiance à la pulsion de connaissance originelle, sans la tailler ni ne l'amputer en rien de ce qui fait sa force et sa finesse et sa nature indivise.

Ce n'est pourtant pas la créativité et ses sources qui se trouvent au centre de l'attention dans cette méditation "L'Enterrement (2) - ou la Clef du Yin et du Yang", mais c'est bien plutôt "le conflit", l'état de blocage de la créativité, ou de dispersion de l'énergie créatrice par l'affrontement, dans la psyché, de forces antagonistes (le plus souvent occultes). Les aspects de violence, de violence (en apparence) "gratuite", "pour le plaisir", m'avaient déconcerté plus d'une fois dans l'Enterrement, et ont fait resurgir une foule de situations vécues similaires. L'expérience de cette violence a été dans ma vie comme "le noyau dur, irréductible, de l'expérience du conflit". Jamais encore je ne m'étais confronté au mystère redoutable de l'existence même et de l'universalité de cette violence dans l'existence humaine en général, et dans la mienne en particulier. C'est ce mystère qui est au centre de l'attention, tout au long de la deuxième moitié (le versant "yin", ou "déclin") de la méditation sur le yin et le yang. C'est au cours de cette partie de la méditation que se dégage progressivement une vision plus profonde du sens de l'Enterrement, et des forces qui s'y expriment. C'est aussi la partie de Récoltes et Semailles qui a été la plus féconde, il me semble, au niveau de la connaissance de moi-même, en me mettant en contact avec des questions et des situations névralgiques, et en me faisant sentir justement ce caractère "névralgique", qui jusqu'à l'an dernier encore était resté éludé.

Une fois au bout de cette interminable "digression" sur le yin et le yang, je restais toujours, à peu de choses près, avec mes "deux ou trois notes" à écrire encore (plus une ou deux autres encore, tout au plus, dont l'une avait déjà son nom tout trouvé "Les quatre opérations"...), pour en avoir terminé avec Récoltes et Semailles. On connaît la suite : ces "quelques dernières notes" ont fini par faire la partie la plus longue de Récoltes et Semailles, de près de cinq cents pages. C'est donc là la "quatrième vague" du mouvement. C'est aussi la troisième et dernière partie de l'Enterrement, et je lui ai donné le nom "Les Quatre Opérations", lequel est aussi celui du groupe de notes ("Les quatre opérations (sur une dépouille)") qui constitue le cœur de ce quatrième souffle de la réflexion. C'est là, dans Récoltes et Semailles, la partie "enquête" au sens le plus strict du terme - avec ce grain de sel, pourtant, que cette enquête ne se borne pas au pur aspect "technique", à l'aspect "détective" en somme, mais que la réflexion y est mue avant tout, comme partout ailleurs dans Récoltes et Semailles, par le désir de connaître et de comprendre. Le ton y est plus "musclé" certes que dans la première partie de l'Enterrement, où j'en étais encore, un peu, à me frotter les yeux et à me demander si j'étais en train de rêver ou quoi ! Cela n'empêche que les faits mis à jour au fil des pages viennent souvent à point nommé, pour illustrer sur le vif beaucoup de choses qui avaient été seulement effleurées en passant ici ou là, en s'incarner dans des exemples précis et frappants. C'est dans cette partie aussi que les digressions mathématiques prennent une place importante, stimulée par un contact renouvelé (par les nécessités de l'enquête) avec une substance que pendant quinze ans j'avais perdue de vue. Il y a également, à l'autre bout du spectre, des récits sur le vif des mésaventures de mon ami Zoghman Mebkhout (à qui cette partie-là est dédiée), aux mains d'une "mafia" de haut vol et sans scrupules, dont il n'avait aucunement rêvé en s'embarquant dans le sujet (passionnant certes, et d'anodine apparence) de la cohomologie des variétés en tous genres. Pour un fil conducteur succinct à travers le dédale intriqué des notes, sous-notes, sous-sous-notes... de toute cette partie "enquête", je te renvoie à la table des matières (notes 167' à $176_{7}$), et à la première des notes du paquet, "Le détective - ou la vie en rose" ( $\mathrm{n}^{\circ} 167^{\prime}$ ). Je signale cependant que cette note, datée du 22 avril, a été ensuite un peu "dépassée par les événements", puisque, de rebondissements en rebondissements, cette enquête que je croyais alors (pratiquement) menée à terme, a continué à brin de zinc pendant deux mois encore.

Ce quatrième souffle s'est prolongé sur plus de quatre mois d'affilée, depuis la mi-février jusqu'à vers la fin juin. C'est dans cette partie de la réflexion surtout, par un "travail sur pièces" méticuleux et obstiné, que s'établit peu à peu au fil des jours et des pages, un contact concret, tangible, avec la réalité de l'Enterrement ; que j'arrive à me "familiariser" avec lui, en somme, tant soit peu, nonobstant les réactions viscérales de refus qu'il avait suscitées (et qu'il continue à susciter) en moi, faisant obstacle à une véritable prise de connaissance. Cette longue réflexion prend son départ avec une rétrospective sur la visite de Deligne (dont il a été question déjà dans cette lettre), et elle s'achève avec la réflexion "de dernière minute" sur ma relation à Serre et sur le rôle de Serre dans l'Enterrement \footnote{Dans les parties c, d, e, de la note "L'album de famille" ( $\mathrm{n}^{\circ} 173$ ), dont la dernière est datée du 18 juin (il y a exactement dix jours). Il y a une seule note ou portion de note dont la date soit ultérieure (savoir, "Cinq thèses pour un massacre - ou la piété filiale", $\mathrm{n}^{\circ}$ $176_{7}$, datée du lendemain le 19 juin). Tu noteras que dans cette quatrième partie de Récoltes et Semailles, ou "partie enquête", contrairement à ce qui a lieu pour les autres, les notes se suivent souvent dans un ordre logique plutôt que chronologique. Ainsi, les deux dernières notes de l'Enterrement (formant le "De Profundis" final) sont datées du 7 avril, deux mois et demi avant la note que je viens de citer. Je signale quand même qu'en dehors de la partie "enquête" proprement dite de l'Enterrement (3) (notes $\mathrm{n}^{\circ} \mathrm{s} 167^{\prime}-176_{7}$), formant le "cinquième temps" de la cérémonie Funèbre (dont la Clef du Yin et du Yang est le deuxième), les notes se suivent dans l'ordre où elles ont été écrites, à de rares exceptions près.}. C'était d'avoir tacitement mis Serre "hors de cause", en faveur de ce "tabou" dont j'ai déjà parlé, qui me semble maintenant la lacune la plus sérieuse peut-être qui restait dans ma compréhension de l'Enterrement, jusqu'au mois dernier encore - et c'est cette réflexion "de dernière minute" qui du coup m'apparaît comme la chose la plus importante que m'ait apportée ce "quatrième souffle" de Récoltes et Semailles, pour une appréhension moins ténue, plus étoffée de l'Enterrement et des forces qui s'y expriment.

