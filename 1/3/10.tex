\section{Quatre vagues dans un mouvement}

Comme de juste, cette lettre ne ressemble pas du tout à ce que j'avais prévu en m'y mettant. Je pensais surtout y faire un petit "topo" sur l'Enterrement : voilà ce qui s'est passé dans les grandes lignes, tu me croiras ou pas (moi-même j'ai eu du mal à le croire...), mais c'est bien ça pourtant, indubitable, même, que cela te plaise ou non, publications noir sur blanc tel périodique ou tel livre, telle date telle page, il n'y a qu'à regarder - d'ailleurs tout est dévissé par le menu dans Récoltes et Semailles ; voir "Quatre Opérations" telles notes - à prendre ou à laisser ! Et si tu préfères t'abstenir de me lire, d'autres s'en chargeront bien à ta place...

Finalement il n'y a rien eu de tout ça - et pourtant cette lettre en est déjà au cap des trente pages, alors que j'en prévoyais cinq ou six en tout et pour tout. Sans même que j’aie fait exprès, ce sont les choses essentielles que j'ai été amené à te dire, au fil des pages, alors que ce "sac" que j'avais été si impatient de vider (là bien en évidence pour le coup, aux premières pages !) il n'est toujours pas déballé ! Ça ne me chatouille même plus dans les doigts, l'envie s'est dissipée en chemin. J'ai compris que ce n'était pas ici le lieu...

A vrai dire, la partie IV de Récoltes et Semailles (et la plus longue de toutes), ayant nom "L'Enterrement (3)" ou "Les Quatre Opérations", est issue d'une "note" prévue initialement comme "un petit topo" justement, pour résumer dans les grandes lignes ce que m'avait révélé l'enquête-à-surprise (et en coup de vent) de l'année dernière, poursuivie dans la partie II ("L'Enterrement (1)", ou "La robe de l'Empereur de Chine"). Je pensais qu'il y en aurait pour une "note" de cinq ou dix pages, pas plus. Finalement, de fil en aiguille, cela a fait repartir l'enquête, il y en a eu pour près de quatre cents pages - près du double de la partie dont j'étais censé faire un résumé ou tirer un bilan ! Ça fait donc qu'il manque toujours le petit topo en question, alors que dans les six cents pages de Récoltes et Semailles sont consacrées à l'enquête sur l'Enterrement. C'est un peu idiot, c'est vrai. Mais il sera toujours temps de le rajouter dans une troisième partie à l'Introduction (qui n'en est plus à dix ou vingt pages près), avant de confier mes notes à un imprimeur.

Les cinq parties de Récoltes et Semailles (dont la dernière n'est pas terminée encore, et ne le sera sans doute pas avant quelques mois) représentent une alternance de (trois) vagues-"méditation" et de (deux) vagues"enquête". Il y a là comme un reflet, en raccourci, de ma vie de ces dernières neuf années, qui a consisté en une alternance, elle aussi, de "vagues" surgies des deux passions qui aujourd'hui dominent ma vie, la passion de la méditation et la passion mathématique. Et à vrai dire, les deux parties (ou "vagues") de Récoltes et
