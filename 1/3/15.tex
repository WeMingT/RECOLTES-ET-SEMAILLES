\section{La gangrène - ou l'esprit du temps (1)}

Cette "opération motifs" n'est qu'une parmi quatre "grandes opérations" de la même eau, et parmi une nuée d'autres de moindre envergure et dans le même esprit. Ce n'est nullement la plus "grosse" des mystifications collectives qui viennent étoffer mon "tableau de mœurs" d'une époque, ni surtout la plus inique. Elle a consisté à piller seulement le troupeau du riche, à la faveur de son absence (ou de son décès...), et non point à venir (dans l'indifférence générale) étrangler pour le plaisir et sous ses yeux, la brebis du pauvre. Et jusque dans le langage mathématique entré dès à présent dans l'usage courant, des noms d'anodine apparence de livres, de notions ou d'énoncés cités à tout moment, sont par eux-mêmes déjà une mystification ou une imposture \footnote{Je pense ici, surtout, au sigle insolite "SGA $4 \frac{1}{2}$ " (c'est utile les nombres fractionnaires !), qui est une double imposture à lui tout seul (et un des sigles les plus cités dans la littérature mathématique contemporaine), et aux noms "dualité de Verdier" ou "dual de Verdier", "conjecture de Deligne-Grothendieck", ou enfin "catégories tannakiennes" (où Tannaka, pour le coup, n'est pas en cause, car il n'a jamais été consulté...). Il en sera question de façon plus circonstanciée en son lieu.}, et témoignent à leur façon de la disgrâce d'une époque.

Si je crois avoir jamais fait œuvre utile pour la "communauté mathématique", c'est d'avoir porté à la pleine lumière du jour un certain nombre de faits peu glorieux, qui faisandaient dans l'ombre. Le genre de faits, sûrement, que tout le monde côtoie tous les jours ou peu s'en faut, de près ou de loin. Combien en est-il parmi eux qui ont pris le loisir de s'arrêter ne fût-ce qu'un instant, pour humer l'air et pour regarder?

Celui qui s'est lui-même trouvé en butte à la morgue des uns et à la malhonnêteté des autres (ou des mêmes), peut-être se flattait-il que c'était là une malchance toute spéciale, à lui dévolue. Confrontant son expérience à mon témoignage, peut-être sentira-t-il que cette "malchance" est aussi un nom qu'il a donné à un esprit du temps, lequel pèse sur lui comme il pèse sur tous. Et (qui sait !) peut-être cela l'incitera-t-il à s'impliquer dans un débat, qui le concerne tout autant qu'il me concerne.

Mais si ce "linge sale" que "j'étais sur la place publique" ne suscite autre chose que le ricanement sans joie des uns et l'embarras poli des autres, dans l'indifférence de tous, une situation qui était trouble sera devenue très claire. (Pour celui du moins qui se soucie encore de se servir de ses yeux.) Les consensus traditionnels de la bonne foi et de la décence \footnote{Quand je parle de ces "consensus de bonne foi et de décence", je n'entends pas dire qu'ils n'étaient jamais transgressés. Mais alors même qu'ils étaient transgressés, c'était bien de "transgressions" qu'il s'agissait, et les consensus eux-mêmes n'en restaient pas moins acceptés.}, dans la relation entre mathématiciens et dans celle du mathématicien à son art, seraient désormais choses du passé, "dépassées". Sans que quelque association internationale de mathématiciens ait encore à le proclamer solennellement, ce serait pourtant chose entendue désormais et quasiment officielle : à présent, tous les coups sont permis, sans plus aucune réserve ni limitation, pour la "confrérie par cooptation" de ceux qui disposent du pouvoir dans le monde mathématique. Tous les magouillages d'idées pour mener par le bout du nez le lecteur apathique qui ne demande qu'à croire, tous les trafics de paternité, et les citations-bidon entre compères et le silence pour ceux voués au silence, et les copinages et les falsifications de toutes sortes et jusqu'au plagiat le plus grossier au vu et su de tous - oui et amen à tout, avec la bénédiction, par la parole ou par le silence (quand ce n'est avec la participation active et empressée), de tous les "grands noms" et de tous les grands et petits patrons sur la place publique mathématique. Oui et amen au "nouveau style" qui y fait fureur! Ce qui fut un art, le voilà devenu, par assentiment (quasiment) unanime, la foire à l'embrouille et à l'empoigne, sous l'œil paterne des chefs.

Il fût un temps où l'exercice du pouvoir, dans le monde des mathématiciens, était limité par des consensus unanimes et intangibles, expression d'un sentiment collectif de décence. Ces consensus et ce sentiment seraient désormais choses désuètes et dépassées, indignes assurément de l'époque glorieuse des ordinateurs, des cellules spatiales et de la bombe à neutrons.

Ce serait chose désormais acquise et scellée : le pouvoir, pour la confrérie de ceux qui en disposent, est un pouvoir discrétionnaire.

