\section{Le dépouillement}

Comme je l'ai déjà laissé entendre tantôt, il m'a fallu surmonter des résistances intérieures considérables, ou plutôt les faire se résorber par un travail patient, méticuleux et tenace, pour parvenir à me séparer de certaines images familières, solidement assises, d'une inertie considérable, qui depuis des décennies avaient pris chez moi (comme chez tout le monde, et chez toi aussi, sûrement) la place d'une perception directe et nuancée de la réalité - en l'occurrence, de celle d'un certain monde mathématique, auquel je continue à être relié par un passé et par une œuvre. Une des plus fortement ancrées de ces images, ou idées toutes faites, c'est qu'il paraît exclu d'emblée qu'un savant de notoriété internationale, voire, un homme qui fait figure de grand mathématicien, puisse se payer (ne fût-ce qu’à titre exceptionnel, et encore moins comme une chère habitude...) des escroqueries petites ou grandes ; ou s'il s'abstient (par vieille habitude encore) d'y tremper la main lui-même, qu'il puisse néanmoins accueillir à bras ouverts telles opérations "(défiant tout sentiment de décence, par moments)" montées par un autre, et où, pour une raison ou une autre, il trouve son compte.

Cette inertie de l'esprit a été telle chez moi, que c'est il y a moins de deux mois seulement, au terme d'une longue réflexion qui s'était poursuivie déjà pendant une année entière, que j'ai fini par entrevoir timidement que Serre y était peut-être aussi pour quelque chose, dans cet Enterrement - chose qui à présent m'apparaît comme une évidence, indépendamment même de la conversation éloquente que j'ai eue avec lui dernièrement. Comme pour tous les membres du "milieu Bourbaki" qui m'avait accueilli avec bienveillance à mes débuts, et tout particulièrement dans son cas, il y avait pour moi une sorte de "tabou" tacite autour de sa personne. Il représentait l'incarnation même d'une certaine "élégance" - d'une élégance qui ne se limite nullement à la forme, mais qui inclut aussi une rigueur, une probité scrupuleuse.

Avant que je ne découvre l'Enterrement, le 19 avril l'an dernier, l'idée ne me serait pas venue, même en rêve, qu'un de ceux qui avaient été mes élèves soit capable, d'une malhonnêteté dans l'exercice de son métier, que ce soit vis-à-vis de moi ou de quiconque ; et c'est pour le plus brillant d'entre eux, celui aussi qui avait été le plus proche de moi, qu'une telle supposition m'aurait semblé la plus aberrante ! Pourtant, dès le moment déjà de mon départ et tout au long des années qui ont suivi et jusqu'à aujourd'hui même, j'avais eu ample occasion de me rendre compte à quel point sa relation à moi était divisée. Plus d'une fois, aussi, je l'ai vu user (pour le seul plaisir, aurait-on dit) du pouvoir de décourager et d'humilier, quand l'occasion était propice. J'en ai été à chaque fois profondément affecté (plus, sans doute, que je n'aurais voulu me l'admettre...). C'étaient là des signes bien assez éloquents d'un dérèglement profond, lequel (j'avais eu ample occasion de le constater) n'était nullement limité à sa seule personne, même dans le cercle des plus limités de ceux qui avaient été mes élèves. Un tel dérèglement, par la perte du respect de la personne d'autrui, n'est pas moins flagrant et moins profond, que celui qui se manifeste par ce qu'on appelle une "malhonnêteté professionnelle". N'empêche que la découverte d'une telle malhonnêteté est venue pour moi comme une surprise totale et comme un choc.

Dans les semaines qui ont suivi cette révélation époustouflante, suivie par toute une "cascade" d'autres de la même eau, je me suis d'ailleurs rendu compte peu à peu qu'un certain magouillage, parmi certains de mes élèves \footnote{Voir la précédente note de b. de p .}, avait commencé déjà dès les années qui ont précédé mon départ. Cela a été particulièrement flagrant, justement, chez le plus brillant d'entre eux - celui, après mon départ, qui a donné le ton et (comme j'écrivais tantôt) "pris la direction discrète et efficace des opérations". Avec le recul de près de vingt ans, ce magouillage m'apparaît à présent comme une évidence, il "crevait les yeux". Si j'ai alors choisi de fermer les yeux sur ce qui se passait, tout à la poursuite de la "baleine blanche" dans un monde "où tout n'est qu'ordre et beauté" (comme il me plaisait à me l'imaginer), je constate aujourd'hui que je n'ai pas su assumer alors la responsabilité qui m'incombait, vis-à-vis d'élèves apprenant à mon contact un métier que j'aime ; un métier qui est autre chose encore qu'un simple savoir-faire, ou le développement d'un certain "flair". Par une complaisance vis-à-vis d'élèves brillants, qu'il m'a plu (par décret tacite) de traiter en "êtres à part" et au dessus de tout soupçon, j'ai contribué alors ma part \footnote{Cette "contribution"- là apparaît notamment dans la note "L'être à part" ( $\mathrm{n}^{\circ} 67^{\prime}$ ), ainsi que dans les deux notes "L'ascension" et "L"ambiguïté" ( $\mathrm{n}^{\circ} \mathrm{s} 63^{\prime}, 63^{\prime \prime}$ ), et à nouveau (dans un éclairage un peu différent) à la fin de la note "L'éviction ( $\mathrm{n}^{\circ} 169$ ). Un autre type de "contribution" apparaît dans "Fatuité et Renouvellement", avec des attitudes de fatuité vis-à-vis de jeunes mathématiciens moins brillamment doués. Cette prise de conscience d'une part de responsabilité dans une dégradation générale culmine dans la section "La mathématique sportive" ( $\mathrm{n}^{\circ} 40$ ).} à l'éclosion de la corruption (sans précédent, me semble-t-il) que je vois s'étaler aujourd'hui dans un monde et parmi des êtres qui m'avaient été chers.

Certes, vue leur inertie immense, il a fallu un travail intense et soutenu pour me séparer de ce qu'on a coutume d'appeler des "illusions" (non sans quelque intonation de regret...), et que j'appellerais plutôt des idées toutes faites ; sur moi-même, sur un milieu auquel je m'étais identifié naguère, sur des personnes que j'ai aimées et que peut-être j’aime encore - me "séparer" de ces idées, ou plutôt, les laisser se détacher de moi. Cela a été un travail, ça oui, mais jamais une lutte - un travail qui m'a apporté, parmi beaucoup d'autres choses de prix, des moments de tristesse parfois, mais jamais un moment de regret ni d'amertume. L'amertume est un des moyens d'éluder une connaissance, d'éluder le message d'un vécu; de se maintenir dans une certaine illusion tenace sur soi-même, au prix d'une autre "illusion" (en négatif, en quelque sorte) sur le monde et sur autrui.

C'est sans amertume et sans regret que je vois se détacher de moi une à une, comme autant de poids encombrants voire écrasants, ces idées toutes faites qui m'avaient été "chères", par vieille habitude et parce qu'elles étaient par la "depuis toujours". Elles étaient devenues, c'est sûr, comme une seconde nature. Mais cette "seconde nature" n'est pas "moi". De m'en séparer morceau par morceau n'est pas un déchirement ni même une frustration, de celui qui se verrait dépouillé de choses qui ont pour lui du prix. Le "dépouillement" dont je parle vient comme la récompense et le fruit d'un travail. Son signe est un soulagement immédiat et bienfaisant, une libération bienvenue.





