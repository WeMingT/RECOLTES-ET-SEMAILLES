\section{Spontanéité et rigueur}

Spontanéité et rigueur sont les deux versants "ombre" et "lumière" d'une même qualité indivise. C'est de leurs épousailles, seulement, que naît cette qualité particulière d'un texte, ou d'un être, qu'on peut essayer d'évoquer par une expression comme "qualité de vérité". Si dans mes publications passées, la spontanéité a été (sinon absente, du moins) à la portion congrue, je ne pense pas que par son tardif épanouissement en moi, la rigueur soit devenue moindre pour autant. Plutôt, la présence à part entière de sa compagne yin donne à la rigueur une dimension, une fécondité nouvelles.

Cette rigueur s'exerce vis-à-vis d'elle-même, veillant à ce, que le "tri" délicat qu'elle doit opérer dans la multitude de ce qui passe dans le champ de la conscience, pour y décanter sans cesse le significatif ou l'essentiel du fortuit ou de l'accessoire, ne s'épaississe et ne se fige en des automorphismes de censure et de complaisance. Seule la curiosité, la soif de connaître en nous éveille et stimule une telle vigilance sans lourdeur, une telle vivacité, à l'encontre de l'inertie immense, omniprésente, des "pentes (dites) naturelles", taillées par les idées toutes faites, expressions de nos peurs et de nos conditionnements.

Et cette même rigueur, cette même attention vigilante se dirige aussi vers la spontanéité comme vers ce qui en prend les aspects, pour y faire la part, là encore, de ces "pentes" tout ce qu'il y a de naturelles, certes, et les distinguer de ce qui véritablement jaillit des couches profondes de l'être, de la pulsion originelle de connaissance et d'action, nous portant à la rencontre du monde.

Au niveau de l'écriture, la rigueur se manifeste par un souci constant de cerner de façon aussi fine, aussi fidèle que possible, à l'aide du langage, les pensées, sentiments, perceptions, images, intuitions... qu'il s'agit d'exprimer, sans se contenter d'un terme vague ou approximatif là où la chose à exprimer est à contours nettement tranchés, ni d'un terme d'une précision factice (et par là, tout aussi déformant) pour exprimer une chose qui reste entourée des brumes de ce qui n'est encore que pressenti. Quand nous essayons de la capter telle qu'elle est dans l'instant, et alors seulement, la chose inconnue nous révèle sa nature véritable, et jusqu'en la pleine lumière du jour peut-être, si elle est faite pour le jour et que notre désir l'incite à se dépouiller de ses voiles d'ombre et de brumes. Notre rôle n'est pas de prétendre décrire et fixer ce que nous ignorons et qui nous échappe, mais de prendre connaissance humblement, passionnément, de l'inconnu et du mystère qui nous entourent de toutes parts.

C'est dire que le rôle de l'écriture n'est pas de consigner les résultats d'une recherche, mais bien le processus même de la recherche - les travaux de l'amour et des œuvres de nos amours avec Notre Mère le Monde, l'Inconnue, qui sans relâche nous appelle en elle pour la connaître encore en son Corps inépuisable, partout en elle où nous portent les voies mystérieuses du désir.

Pour rendre ce processus, les retours en arrière, qui nuancent, précisent, approfondissent et parfois corrigent le "premier jet" de l'écriture, voire un deuxième ou un troisième, font partie de la démarche même de la découverte. Ils forment une partie essentielle du texte et lui donnent tout son sens. C’est pourquoi les "notes" (ou "annotations") placées à la fin de Fatuité et Renouvellement, et auxquelles il est référé ici et là au cours des cinquante "sections" qui constituent le "premier jet" du texte, sont une partie inséparable et essentielle de celui-ci. Je te conseille vivement de t'y reporter au fur et à mesure, et au moins en fin de lecture de chaque section où figurent un ou plusieurs renvois à de telles "notes". Il en est de même pour les notes de bas de page dans les autres parties de Récoltes et Semailles, ou les renvois, dans telle "note" (constituant ici le "texte principal"), à telle note ultérieure, qui fait dès lors fonction de "retour" sur celle-ci, ou d'annotation. C'est là, avec mon conseil de ne pas te séparer en cours de lecture de la table des matières, la principale des recommandations de lecture que je vois à te faire.

Une dernière question, pratique, qui va clore (un peu prosaïquement) cette lettre qu'il est temps de terminer. Il y a eu un peu de "panique" par moments, pour préparer les différents fascicules de Récoltes et Semailles pour le tirage par le Service de duplication à la Fac, à temps pour que le tirage se fasse (si possible) avant les grandes vacances. Dans la hâte, il y a toute une feuille de notes de bas de page de dernière minute, à rajouter au fascicule 2 (L'Enterrement (1) - ou La robe de l'Empereur de Chine), qui a "sauté". Il s'agissait surtout de la rectification de certaines erreurs matérielles, apparues dernièrement seulement, en cours d'écriture des Quatre Opérations. Il y a une de ces notes de bas de page qui est plus conséquente que les autres, et que je voudrais signaler ici. Il s'agit d'une annotation à la note "La victime - ou les deux silences" ( $\mathrm{n}^{\circ} 78^{\prime}$, page 304). Cette note, où je me suis efforcé, entre autres, de cerner mes impressions (toutes subjectives, certes) au sujet de la façon dont mon ami Zoghman Mebkhout "intériorisait" à cette époque la spoliation unique dont il faisait les frais, a été ressentie par lui comme injuste à son égard, alors que j'avais l'air quasiment de le mettre "dans le même sac" avec ses spoliateurs. Ce qui est sûr, c'est que dans cette note, qui ne prétend pas donner autre chose que des impressions liées à un "moment" particulier, je ne présente qu'un seul son de cloche, en laissant dans le non-dit (et comme chose allant de soi, sans doute) certains autres sons tout aussi réels (et moins discutables peut-être). Toujours est-il que la réflexion sur ce sujet délicat s'approfondit considérablement, à un an de distance, dans la note "Racines et Solitude" ( $\mathrm{n}^{\circ}$ 171). Celle-ci n'a pas suscité de réserves de la part de Zoghman. D'autres éléments de réflexion sur ce même sujet se trouvent également dans les deux notes "Trois jalons - ou l'innocence", et "Les pages mortes" ( $\mathrm{n}^{\circ}$ s 171 (x) et (xii)). Ces trois notes font partie de "L'Apothéose", qui est la partie des Quatre Opérations consacrées à l'opération d'appropriation et de détournement de l'œuvre de Zoghman Mebkhout.

Il ne me reste plus qu'à te souhaiter bonne lecture - et au plaisir de te lire à mon tour !

\hfill Alexandre Grothendieck







