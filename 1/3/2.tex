\section{Naissance de Récoltes et Semailles (une rétrospective - éclair)}

Dans cette pré-lettre, je voudrais maintenant te dire en quelques pages (si faire se peut) de quoi il est question dans Récoltes et Semailles - et le dire de façon plus circonstanciée que ne le fait le seul sous-titre : "Réflexions et témoignage sur un passé de mathématicien" (le mien de passé, tu l'auras deviné\ldots). Il y a beaucoup de choses dans Récoltes et Semailles et les uns et les autres y verront sans doute beaucoup de choses différentes : un voyage à la découverte d'un passé ; une \textit{méditation} sur l'existence ; un \textit{tableau de moeurs} d'un milieu et d'une époque (ou le tableau du glissement insidieux et implacable d'une époque à une autre\ldots) ; une \textit{enquête} (quasiment policière par moments, et en d'autres frisant le roman de cape et d'épée dans les bas-fonds de la mégapolis mathématique\ldots) ; une vaste \textit{divagation mathématique} (qui sèmera plus d'un\ldots) ; un traité pratique de psychanalyse appliquée (ou, au choix, un livre de "\textit{psychanalyse-fiction}") ; un panégyrique de la \textit{connaissance de soi} ; "\textit{Mes confessions}" ; un journal intime ; une psychologie de la \textit{découverte} et de la \textit{création} ; un \textit{réquisitoire} (impitoyable, comme il se doit\ldots), voire un \textit{règlement de comptes} dans "le beau monde mathématique" (et sans faire de cadeaux\ldots). Ce qui est sûr, c'est qu'à aucun moment je ne me suis ennuyé en l'écrivant, alors que j'en ai appris et vu de toutes les couleurs. Si tes importantes tâches te laissent le loisir de le lire, ça m'étonnerait que tu t'ennuies en me lisant - à moins de te forcer, bien sûr\ldots

Visiblement, ça ne s'adresse pas qu'aux mathématiciens. Il est vrai aussi qu'à certains moments, ça s'adresse aux mathématiciens plus qu'à d'autres. Dans cette pré-lettre à la "lettre Récoltes et Semailles", je voudrais résumer et faire ressortir surtout, justement, ce qui peut te concerner plus particulièrement comme mathématicien. Le plus naturel, pour ce faire, sera de te raconter simplement comment j'en suis venu, de fil en aiguille, à écrire coup sur coup ces quatre ou cinq "livres" dont il a été question.

Comme tu le sais, j'ai quitté "le grand monde" mathématique en 1970, à la suite d'une histoire de fonds militaires dans mon institution d'attache (l' IHES) Après quelques années de militantisme anti-militariste et écologique, style "révolution culturelle", dont tu as sans doute eu quelque écho ici et là, je disparais pratiquement de la circulation, perdu dans une université de province Dieu sait où. La rumeur dit que je passe mon temps à garder des moutons et à forer des puits. La vérité c'est qu'à part beaucoup d'autres occupations, j'allais bravement, comme tout le monde, faire mes cours à la Fac (c'était la mon peu original gagne-pain, et ça l'est encore aujourd'hui). Il m'arrivait même ici et là, pendant quelques jours, voire quelques semaines ou quelques mois, de refaire des maths à brin de zinc - j'ai des cartons pleins avec mes gribouillés, que je dois être le seul à pouvoir déchiffrer. Mais c'était sur des choses très différentes, à première vue du moins, de ce que j'avais fait dans le temps. Entre 1955 et 1970, mon thème de prédilection avait été la cohomologie, et plus particulièrement, la cohomologie des variétés en tous genres (algébriques, en particulier). Je jugeais en avoir assez fait dans cette direction-là pour que les autres se débrouillent sans moi, et tant qu'à faire des maths, il était temps que je change de disque\ldots

En 1976 est apparue dans ma vie une nouvelle passion, aussi forte qu'avait été jadis ma passion mathématique, et d'ailleurs proche parente de celle-ci. C'est la passion pour ce que j'ai appelé "la méditation" (puisqu'il faut bien des noms aux choses). Ce nom, comme le ferait le tout autre nom, ne peut manquer de susciter d'innombrables malentendus. Comme en mathématique, il s'agit là d'un travail de découverte. Je m'exprime à son sujet ici et là au cours de Récoltes et Semailles. Toujours est-il que, visiblement, il y avait là de quoi m'occuper jusqu'à la fin de mes jours. Et plus d'une fois, en effet, j'ai bien cru que la mathématique, c'était du passé et du dorénavant, je n'allais plus m'occuper que de choses plus sérieuses - que j'allais "méditer".

J'ai pourtant fini par me rendre à l'évidence (il y a quatre ans) que la passion mathématique n'était pas éteinte pour autant. Et même, sans trop savoir comment et à ma propre surprise, moi qui (depuis près de quinze ans) ne pensais plus publier une ligne de maths de ma vie, je me suis vu soudain embarqué dans l'écriture d'un ouvrage de maths qui visiblement n'en finissait pas et qui allait avoir des volumes et des volumes ; et tant que j'y étais, j'allais balancer ce que je croyais avoir à dire en maths dans une série (infinie ?) de livres qui s'appellerait "Réflexions Mathématiques", et qu'on n'en parle plus.

C'était il y a deux ans, printemps 1983. J'étais alors trop occupé déjà à écrire (le volume 1 de) "A la Poursuite des Champs", lequel devait constituer aussi le volume 1 des "Réflexions" (mathématiques), pour me poser des questions sur ce qui m'arrivait. Neuf mois plus tard, comme il se doit, ce premier volume était terminé autant dire, il n'y avait plus que l'introduction à écrire, relire le tout, des annotations - et à l'impression\ldots

Le volume en question n'est toujours pas terminé à l'heure qu'il est - il n'a pas bougé d'un poil depuis un an et demi. L'introduction qui restait à écrire a dépassé le cap des douze cent pages (dactylographiées), quand ce sera terminé vrai de vrai il y en aura bien quatorze cent. Tu auras deviné que ladite "introduction" n'est autre que Récoltes et Semailles. Aux dernières nouvelles, elle est censée former les volumes 1 et 2 plus une partie du volume 3 de la fameuse "série" prévue. Celle-ci du coup change de nom et s'appellera "Réflexions" (tout court, pas forcément mathématiques). Le reste du volume 3 sera formé surtout de textes mathématiques, à présent plus brillants pour moi que la Poursuite des Champs. Celle-ci attendra bien l'an prochain, pour les annotations, les index, plus, bien sûr, une introduction\ldots

Fin du premier Acte !