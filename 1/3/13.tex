\phantomsection
\section*{Epilogue en Post-scriptum - ou contexte et préalables d'un débat}
\addcontentsline{toc}{section}{Epilogue en Post-scriptum - ou contexte et préalables d'un débat}

\hfill Février 1986

\section{Le spectographe à bouteilles}

Voilà sept mois bien tassés que cette Lettre a été écrite, et près de quatre mois qu'elle est envoyée, avec le "pavé" qui va avec. Et avec une dédicace de ma main dans chacune \footnote{Il y a quelques rares exceptions, comprenant surtout les collègues que je ne connais pas personnellement, et qui ont reçu seulement les fascicules 0 et 4 du tirage provisoire, en prime pour leur participation active à mon Enterrement.}. Comme une "bouteille à la mer", ou plutôt, comme toute une flopée de telles bouteilles vagabondes, mon message est allé atterrir et circuler jusque dans les coins les plus reculés de ce microcosme mathématique qui me fut familier. Et par les échos directs et indirects qui m'en reviennent au fil des jours, des semaines et des mois, me voilà inopinément comme devant une vaste radiographie du milieu mathématique, laquelle serait prise par un spectographe tentaculaire, dont mes innocentes "bouteilles" seraient autant d'antennes voyageuses. Du coup (noblesse oblige !), moi qui pourtant ne manque pas de quoi m'occuper, me voilà placé devant la nouvelle tâche de déchiffrer la radio et de rendre compte, du mieux que je pourrai, de ce que j'y ai lu. Ce sera pour une sixième (et dernière, c'est promis !) partie de Récoltes et Semailles. Celle-ci viendra donc couronner, si Dieu me prête vie, "la grande œuvre sociologique de mes vieux jours". Pour le moment, quelques premiers commentaires.

Pour accueillir ma modeste flottille très artisanale, ce qui semble dominer et de loin, c'est le ton goguenard, mi-hargneux, sur l'air du "voilà Grothendieck qui devient parano sur ses vieux jours", ou "en voilà un qui se prend bien au sérieux" - et le tour est joué! Je n'ai eu pourtant qu'une seule lettre de ce style-là \footnote{Cette lettre provient d'un de ceux qui furent mes élèves, et de plus, un de mes coenterreurs.}, plus deux autres encore dans celui d'une dérision feutrée et ravie d'elle-même \footnote{De la part de deux de mes anciens collègues de travail au sein de Bourbaki, et dont l'un est un des aînés qui m'avaient accueilli avec une chaleureuse bien-veillance, lors de mes débuts.}. La plupart de mes destinataires mathématiciens, y compris parmi ceux qui furent mes élèves, ont répondu par le silence \footnote{Pour cent trente-et-un envois à des mathématiciens, il y a eu jusqu'à présent cinquante-trois parmi les destinataires qui ont donné signe de vie, ne fut-ce que pour accuser réception. Parmi ceux-ci, il y a six de mes ex-élèves - je n'ai pas eu signe de vie d'aucun des huit autres.} - un silence qui m'en dit long.

Cela n'empêche que j'ai eu déjà une volumineuse correspondance. La grande plupart des lettres sont dans les tons de l'embarras poli, lequel souvent se voudrait amical, comme par un souci de bienséance. Deux ou trois fois j'ai senti, derrière cet embarras et comme tamisé par lui, la chaleur d'un sentiment toujours vivant. Le plus souvent, quand l'embarras ne s'exprime par des protestations de bons sentiments (pour son propre compte, ou pour celui d'autrui), c'est par des compliments - je n'en aurai jamais tant reçu de ma vie! Sur l'air du "grand mathématicien", "pages superbes" (sur la créativité "et tout ça"...), "incontestable écrivain", et j'en passe. Pour faire bonne mesure, j'ai même eu droit à un compliment bien senti (et nullement ironique) sur la richesse de ma vie intérieure. Inutile de dire que dans toutes ces lettres-là, mon correspondant n'a garde d'entrer dans le vif d'aucune question, et encore moins, de s'y impliquer personnellement ; le ton serait plutôt de celui qui aurait été "sollicité de donner son opinion" (pour reprendre les termes d'une de ces lettres), sur une affaire un peu scabreuse et ce qui plus est, hypothétique voire imaginaire, et en tous cas et surtout, une affaire qui ne le concerne pas personnellement. Quand il fait mine pourtant d'y toucher, à une de ces questions, c'est du bout des doigts et pour la tenir aussi loin de lui qu'il le peut - que ce soit à la faveur de bons conseils à moi prodigués, ou par des conditionnels prudents, ou par les lieux communs d'usage quand on ne sait trop quoi dire, ou de toute autre façon. Certains quand même ont laissé entendre qu'il y avait peut-être des choses pas très normales qui se sont passées - tout en prenant soin de laisser dans le plus grand vague de quoi et de qui il s'agit...

J'ai eu aussi des échos franchement chaleureux, de la part de quinze ou seize de mes anciens et nouveaux amis. Certains exprimaient une émotion, sans velléité de vouloir s'en cacher ou de la faire taire. Ces échos, et d'autres tout aussi chaleureux me venant d'en dehors du milieu mathématique, auront été ma récompense pour un long et solitaire travail, fait non seulement pour moi-même, mais pour tous.

Et parmi les quelques cent-trente collègues qui ont reçu ma Lettre, il en est trois qui y ont répondu, au plein sens du terme, en s'impliquant eux-mêmes, au lieu de se borner à un commentaire lointain sur les événements du siècle. J'ai reçu un autre tel écho encore d'une correspondante non mathématicienne. C'étaient des vraies réponses à mon message. Et c'était là aussi la meilleure de mes récompenses.

