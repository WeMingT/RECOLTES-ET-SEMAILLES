\section{``Mes proches'' - ou la connivence}

Ce n'est pas mon propos dans cette lettre de passer en revue tous les ``moments forts'' (ou tous les ``moments sensibles'') dans l'écriture de Récoltes et Semailles, ou dans telle de ses étapes\footnote{Tu trouveras une courte rétrospective-bilan, de l'ensemble des trois premières parties de Récoltes et Semailles, dans les deux groupes de notes ``Les fruits du soir'' (n$^{\circ}$s 179-182) et ``Découverte d'un passé'' (n$^{\circ}$ 183-186).}. Qu'il me suffise de dire qu'il y a eu, dans ce travail, quatre grandes étapes nettement marquées ou quatre ``souffles'' - comme les souffles d'une respiration, ou comme les vagues successives dans un train de vagues surgit, je ne saurais dire comment, de ces vastes masses muettes, immobiles et mouvantes, sans limites et sans nom, d'une mer inconnue et sans fond qui est ``moi'', ou plutôt, d'une mer infiniment plus vaste et plus profonde que ce ``moi'' qu'elle porte et qu'elle nourrit. Ces ``souffles'' ou ces ``vagues'' se sont matérialisées en les quatre parties de Récoltes et Semailles écrites à présent. Chaque vague est venue sans que je l'aie appelée ni moins du monde prévue, et à aucun moment je n'aurais su dire où elle allait me porter ni quand elle prendrait fin. Et quand elle avait pris fin et qu'une nouvelle vague déjà avait pris sa suite, pendant un temps encore je me croyais toujours sur la fin d'une lancée (qui serait aussi, à la fin des fins, la fin de Récoltes et Semailles !), alors que j'étais pourtant soulevé et porté à présent par un autre souffle d'un autre et vaste mouvement. C'est avec le recul seulement que celui-ci apparaît clairement et que se révèle sans équivoque une \textbf{structure} dans ce qui avait été vécu comme acte et comme mouvance.

Et sûrement, ce mouvement-là n'a pas pris fin avec mon point final (tout provisoire !) à Récoltes et Semailles, et ne prendra fin non plus avec le point final à cette lettre à toi, laquelle est un des ``temps'' de ce mouvement. Et il n'est pas né en un jour de juin 1983, ou de février 1984, quand je me suis assis devant ma machine à écrire pour écrire (ou reprendre) une certaines introduction à un certain ouvrage mathématique. Il est né (ou plutôt, il est re-né\ldots), le jour où la méditation est apparue dans ma vie\ldots

Mais à nouveau je digresse, me laissant porter (et emporter\ldots) par les images et associations nées de l'instant, au lieu de m'en tenir sagement au fil d'un ``propos'', du prévu. Mon propos aujourd'hui avait été d'enchaîner avec le récit, si succinct soit-il, de la ``découverte de l' Enterrement'' au mois d'avril dernier, à un moment où depuis deux semaines je croyais avoir terminé Récoltes et Semailles - comment me sont dégringolées dessus en cascade, en l'espace de trois ou quatre semaine à peine, des découvertes les unes plus grosses et plus incroyables que les autres - si grosses et si dingues même que pendant des mois encore, j'ai eu le plus grand mal ``à en croire le témoignage de mes saines facultés'', à me libérer d'une insidieuse \textbf{incrédulité} devant l'évidence\footnote{J'essaye d'exprimer cette difficulté, par le conte ``La robe de l'Empereur de Chine'', dans la note de même nom (n$^{\circ}$ 77'), et y reviens à nouveau dans la note ``Le devoir accompli - ou l'instant de vérité'' (n$^{\circ}$ 163).}. Cette incrédulité secrète et tenace n'a fini par se dissiper qu'au mois d'octobre dernier (six mois après la découverte de ``l' Enterrement dans toute sa splendeur''), à la suite de la visite chez moi de mon ami ex-élève (occulte, il est vrai) Pierre Deligne\footnote{Je fais le récit de cette visite dans la note que je viens de citer (dans la précédente note de b. de p.).}. Pour la première fois, je me suis vu alors confronté à l' Enterrement non plus par le truchement de textes, en parlant (en termes certes éloquents !) du débinage, du pillage et du massacre d'une œuvre, et de l'enterrement (en la personne du maître absent) d'un certain style et d'une certaines approche de la mathématique - mais d'une façon cette fois directe et tangible, sous des traits familiers et par une voix bien connue, aux intonations affables et ingénues. L' Enterrement était là devant moi enfin, ``en chair et en os'', sous ces traits affairés et anodins que je reconnaissais bien désormais, mais que pour la première fois je regardais avec des yeux nouveaux, une attention nouvelle. Voici donc se déployer devant moi celui qui, au cours de ma réflexion des mois précédents, s'était révélé comme le Grand Officiant à mes Obsèques solennelles, comme le ``Prêtre en chasuble'' en même temps que le principal artisan et le principal ``bénéficiaire'' d'une ``opération'' sans précédent, héritier occulte d'une œuvre livrée à la dérision et au pillage\ldots

Cette rencontre se place aux débuts de la ``troisième vague'' dans Récoltes et Semailles, alors que je venais de m'engager dans la longue méditation sur le yin et le yang, à la poursuite d'une élusive et tenace association d'idées. Sur le coup, ce court épisode ne laisse que la trace d'un écho de quelques lignes, en passant. Il marque pourtant un moment important, dont les fruits n'apparaîtront clairement que des mois plus tard.

Il y a eu un deuxième tel moment de confrontation à ``L' Enterrement en chair et en os''. C'était il y a dix jours à peine, et venait relancer une fois encore, ``en dernière minute'', une enquête qui n'en finissait pas de repartir sans cesse Cette fois, c'était un simple coup de fil à Jean-Pierre Serre\footnote{C'est là, à peu de choses près, une citation de la note ``Le Fossoyeur - ou la Congrégation toute entière'' (n$^{\circ}$ 97, page 417).}. Cette conversation ``à bâtons rompus'' est venue confirmer de façon saisissante et au delà même de toute attente, ce que (quelques jours avant à peine) je venais de m'expliquer longuement\footnote{Dans la partie c. (``Celui entre tous - ou l'acquiescement'') de la même note (n$^{\circ}$ 173).}, et à mon corps défendant quasiment, au sujet du rôle joué par Serre dans mon Enterrement et sur un ``secret acquiescement'' en lui à ce qui se passait ``juste sous son nez'', sans qu'il fasse mine de rien voir ni rien sentir.

La encore, comme de juste, la conversation était tout ce qu'il y a de ``cool'' et d'amicale, et visiblement ces dispositions amicales en Serre à mon égard sont aussi tout ce qu'il y a de sincères et véritables. Cela n'empêche que cette fois j'ai pu voir véritablement, ou ``toucher'' aurais-je envie d'écrire, cet ``acquiescement'' que je venais de finir par m'admettre : ``secret'' sans doute (comme j'avais écrit précédemment) mais surtout empressé, comme j'ai pu alors le voir sans possibilité de doute. Un acquiescement empressé et sans réserve, pour que soit enterré ce qui doit être enterré, et pour que, partout où cela s'avère souhaitable et quels que soient les moyens, une paternité réelle (que Serre connaît de première main) et indésirable, soit remplacée par une paternité fictice et bienvenue\ldots\footnote{C'est là, à peu de choses près, une citation de la note ``Le Fossoyeur - ou la Congrégation toute entière'' (n$^{\circ}$ 97, page 417).} C'était là une confirmation saisissante d'une intuition apparue une année auparavant déjà, quand j'écrivais\footnote{Cette citation est extraite de la même note (voir les notes de b. de p. précédente), à la même page 417.} :

\begin{quote}
    ``Vu dans cette lumière\footnote{``A la lumière'' de ce propos délibéré, dont il venait d'être question, d'éliminer à tout prix des ``paternités indésirables'' (voir, ``indigestes'', pour reprendre l'expression employée dans le texte cité).}, le principal officiant Deligne apparaît non plus comme celui qui aurait façonné une œuvre à l'image des forces profondes qui déterminent sa propre vie et ses actes, mais plutôt comme \textbf{l'instrument} tout désigné (de par son rôle d' ``héritier légitime''\footnote{Ce rôle d' ``héritier'' de Deligne est un rôle à la fois occulte (alors que pas une ligne publiée de Deligne ne peut faire soupçonner qu'il puisse avoir appris quelque chose par ma bouche), et en même temps clairement senti et admis par tous. C'est là un des aspects typiques du double-jeu de Deligne et de son ``style'' particulier, où il a su jouer avec maestria sur cette ambiguïté, et encaisser les avantages de ce rôle tacite d'héritier, tout en désavouant le défunt maître et en prenant la direction d'opérations d'enterrement de vaste envergure.}) d'une \textbf{volonté collective} d'une cohérence sans failles, s'attachant à l'impossible tâche d'effacer et mon nom et mon style personnel de la mathématique contemporaine.''
\end{quote}

Si Deligne m'est apparu alors comme l' ``instrument'' tout désigné (en même temps que le premier et principal ``bénéficiaire'') d'une ``volonté collective d'une cohérence sans failles'', Serre m'apparaît à présent comme \textbf{l'incarnation} de cette même volonté collective, et comme le garant de son acquiescement sans réserve, un acquiescement à toutes les magouilles et escroqueries innombrables et jusques aux vastes ``opérations'' de mystification collective et d'appropriation sans vergogne, aussi longtemps que celles-ci concernent à cette ``impossible tâche'' vis-à-vis de ma modeste et défunte personne, ou vis-à-vis de tel autre\footnote{Je pense ici à Zoghman Mebkhout, dont il est question pour la première fois dans l'introduction, à (``L'Enterrement''), puis dans la note ``Mes orphelins'' (n$^{\circ}$ 46), et dans les notes (écrites ultérieurement, après la découverte de l'Enterrement) ``Echec d'un enseignement (2) - ou création et fatuité'' et ``Un sentiment d'injustice et d'impuissance'' (n$^{\circ}$ s 44', 44''). Je découvre l'inique opération d'escamotage et d'appropriation de l'œuvre de pionnier de Mebkhout, au fil des onze notes formant le Cortège VII de l'Enterrement, ``Le Colloque - ou faisceaux de Mebkhout et Perversité'' (n$^{\circ}$ s 75-80). Une enquête et un récit plus circonstanciés sur cette (quatrième et dernière) ``opération'' forme la partie la plus étoffée de l'enquête ``Les quatre opérations'', sous le nom qui s'imposait ``\textbf{L'Apothéose}'' (notes n$^{\circ}$ s 171 (i) à 171).} qui a osé se réclamer de moi et faire figure, envers et contre tous, de ``continuateur de Grothendieck''.

C'est un des aspects paradoxaux et déconcertants, parmi de nombreux autres dans l' Enterrement, que celui-ci soit l'œuvre avant tout, pour ne pas dire exclusivement, de ceux qui avaient été mes amis ou mes élèves, dans un monde où jamais je ne m'étais connu d'ennemis. C'est à ce titre surtout, je crois, que Récoltes et Semailles te concerne plus qu'un autre, et que cette lettre que je suis en train de t'écrire se veut \textbf{interpellation} à son tour. Car si tu est mathématicien, et si tu es un de ceux qui furent mes élèves, ou qui furent mes amis, tu n'es sans doute pas étranger à l' Enterrement, que ce soit par actes ou par connivence, et ne serait-ce que par ton silence vis-à-vis de moi, au sujet d'une chose qui se déroule devant le pas de ta porte. Et si (par extraordinaire) tu accueilles mes humbles paroles et le témoignage qu'elles te portent, plutôt que de rester enfermé derrière tes portes closes et de renvoyer ces messages malvenus, tu apprendras alors, peut-être, que ce qui a été enterré par toi et avec ta participation (active, ou par tacite acquiescement), ce n'est pas seulement l'œuvre d'un autre, fruit et vivant témoignage de mes amours avec la mathématique ; mais qu'à un niveau plus secret encore que cet enterrement (qui jamais ne dit son nom\ldots) et plus profond, c'est une part vivante et essentielle de ton propre être, de ton pouvoir originel de connaître, d'aimer et de créer, qu'il t'a plu d'enterrer par tes propres mains en la personne d'un autre.

Parmi tous mes élèves, Deligne avait occupé une place bien à part, sur laquelle je m'étends longuement au cours de la réflexion\footnote{Voir surtout, à ce sujet, le groupe des dix-sept notes ``Mon ami Pierre'' (n$^{\circ}$s 60-71) dans RS II.}. Il a été, et de très loin, le plus ``proche'', le seul aussi (élève ou pas) à avoir assimilé intimement et fait sienne\footnote{Cette ``vaste vision'', que Deligne a bel et bien ``assimilé intimement et fait sienne'', avait exercé une fascination puissante sur lui, et continue à le fasciner malgré lui, alors qu'une force impérieuse le pousse en même temps à la détruire, à briser sciemment son unité foncière et à s'emparer des morceaux épars. Ainsi, son antagonisme occulte vis-à-vis d'un maître renié et ``défunt'' est l'expression d'une division en son être, qui a profondément marqué son œuvre après mon départ - œuvre qui est restée très loin en deçà des moyens assez prodigieux que je lui avais connus.} une vaste vision qui était née et avait grandi en moi longtemps déjà avant notre rencontre. Et parmi tous mes amis partageant avec moi une commune passion pour la mathématique, c'était Serre, lequel avait en même temps fait un peu figure d'aîné, qui était le plus proche (et de loin, également), comme celui (notamment) qui pendant une décennie avait joué dans mon travail un rôle unique de ``détonateur'' pour certains de mes grands investissements, et pour la plupart des grandes idées-force qui ont inspiré ma pensée mathématique au cours des années cinquante et soixante, jusqu'au moment de mon départ. Cette relation très particulière que l'un et l'autre avait à ma personne n'est pas sans liens, certes, avec les moyens exceptionnels de l'un et de l'autre, qui leur a assuré un ascendant vraiment exceptionnel sur les mathématiciens de leur génération, et de celles qui ont suivi. Mis à part ces points communs, les tempéraments et les façons de Serre et de Deligne me paraissent d'ailleurs aussi dissemblables qu'il est possible, aux antipodes l'un de l'autre à bien des égards.

Quoi qu'il en soit, s'il y a eu des mathématiciens qui, à un titre ou à un autre, ont été ``proches'' de ma personne et de mon œuvre (et, ce qui plus est, connus pour tels), c'est bien Serre et Deligne : l'un, un aîné et une source d'inspiration dans mon œuvre pendant une période cruciale de gestation d'une vision ; l'autre, le plus doué de mes élèves, pour lequel j'ai été à mon tour (et suis resté, Enterrement ou pas\ldots) sa principale (et secrète\ldots) source d'inspiration\footnote{Voir à ce sujet la précédente note de b. de p.}. Si un Enterrement s'est mis en branle aux lendemains de mon départ (devenu ``décès'' en bonne et due forme), et s'est concrétisé en un interminable cortège d' ``opérations'' grandes et petites au service d'une même fin, cela n'a pu se faire qu'avec le concours conjugué et étroitement solidaire de l'un et de l'autre, de l'ex-aîné et de l'ex-élève (voir, ex-``disciple'') : l'un prenant la direction discrète et efficace des opérations, tout en sonnant le ralliement de certains de mes élèves\footnote{Il s'agit ici, très exactement, des cinq autres élèves qui ont choisi comme thème principal (tout comme Deligne) celui de la cohomologie des variétés.}, en mal de massacre du Père (sous l'effigie grotesque et dérisoire d'une pléthorique et bominante \textbf{super-nana}) ; et l'autre donnant un ``feu vert'' sans réserve, inconditionnel et illimité à la poursuite des (quatre) opérations (de débinages, carnage, dépeçage et de partage d'une inépuisable dépouille\ldots).

