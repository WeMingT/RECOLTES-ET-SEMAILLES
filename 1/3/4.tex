\section{Un vent d'enterrement\ldots}

Pourtant, s'il m'a plu tout au long de ces années d'étudier la perception diffuse d'un Enterrement de grande envergure, celui-ci n'a pas manqué de se rappeler obstinément à mon bon souvenir, sous d'autres visages et de moins anodins, que celui d'une simple désaffection pour une œuvre. J'ai su peu à peu, je ne saurais trop dire comment, que plusieurs notions qui faisaient partie de la vision oubliée, étaient non seulement tombées en désuétude, mais étaient devenues, dans un certain beau monde, objet d'un condescendant dédain. Tel a été le cas, notamment, de la notion unificatrice cruciale de topos, au cœur même de la géométrie nouvelle - celle-là même qui fournit l'intuition géométrique commune pour la topologie, la géométrie algébrique et l'arithmétique - celle aussi qui m'a permis de dégager aussi bien l'outil cohomologique étale et $\ell$-adique, que les idées maîtresses (plus ou moins oubliées depuis, il est vrai\ldots) de la cohomologie cristalline. A vrai dire, c'était mon nom même, au fil des ans, qui insidieusement, mystérieusement, était devenu objet de dérision - comme un synonyme de vaseux bouquinages à l'infini (tels ceux sur ces fameux ``topos'', justement, ou ces ``motifs'' dont il vous rabattait les oreilles et que personne n'avait jamais vus\ldots), de découpage de cheveux en quatre à longueur de mille pages, et de pléthorique et gigantesque bavardage sur ce que, de toutes façons, tout le monde connaissait déjà depuis toujours et sans l'avoir attendu\ldots Un peu sur ces tons-là, mais en sourdine, par sous-entendus, avec toute la délicatesse qui est de mise ``parmi les gens de bonne compagnie''.

Au cours de la réflexion poursuivie dans Récoltes et Semailles, je crois avoir mis le doigt sur les forces profondes à l'œuvre chez les uns et les autres, derrière ces airs de dérision et de condescendance devant une œuvre dont la portée, la vie et le souffle, leur échappent. J'ai découvert également (mis à part les traits particuliers de ma personne qui ont marqué mon œuvre et mon destin) le secret ``catalyseur'' qui a incité ces forces à se manifester sous cette forme du mépris désinvolte devant les signes éloquents d'une créativité intacte : le Grand Officiant aux Obsèques, en somme, en cet Enterrement feutré par la dérision et par le mépris. Chose étrange, c'est aussi celui, curieusement, qui est et a été le plus proche de moi : le seul aussi qui ait assimilé un jour et fait sienne une certaine vision, emplie de vie et de force intense. Mais j'anticipe\ldots

A vrai dire, ces ``bouffées de discrète dérision'' qui me revenaient ici et là, ne m'atteignaient pas outre mesure. Elles restaient en quelque sorte anonymes, jusqu'il y a trois ou quatre ans encore. J'y voyais certes un signe des temps peu réjouissant, mais elles ne me mettaient pas en cause vraiment, et ne suscitaient en moi angoisse ni inquiétude. Une chose par contre qui me touchait plus directement, c'étaient les signes de prise de distance par rapport à ma personne, me venant ici et là de la part de bon nombre de mes amis d'antan dans le monde mathématique, amis auxquels (nonobstant mon départ d'un monde qui nous fut commun) je continuais à me sentir relié par des liens de sympathie, en plus de ceux que crée une passion commune et un certain passé en commun. La encore, à chaque fois j'en ai été peiné, je ne m'y suis pourtant guère arrêté, et la pensée ne m'est jamais venue (pour autant que je me souvienne) de faire un rapprochement entre ces trois séries de signes : les chantiers abandonnés (et la vision oubliée), le ``vent de dérision'', et la prise de distance de nombre parmi ceux qui furent des amis. J'ai écrit à chacun d'eux, et je n'ai reçu de réponse d'aucun. Ce n'était pas rare d'ailleurs, désormais, que des lettres que j'écrivais à d'anciens amis ou élèves, sur des choses qui me tenaient à cœur, restent sans réponse. Nouveaux temps, nouveaux mœurs - qu'y pouvais-je faire ? Je me suis borné à m'abstenir de leur écrire encore. Et pourtant (si tu es un de ceux-là) cette lettre que je suis en train d'écrire, elle sera l'exception - une parole qui t'est à nouveau offerte - à toi de voir si tu l'accueilles cette fois, ou t'y fermes à nouveau\ldots

Les premiers signes d'une prise de distance de certains anciens amis par rapport à ma personne remontent, si je ne me trompe, à 1976. C'est l'année aussi où a commencé à apparaître une autre ``série'' de signes encore, dont il me reste à parler, avant de revenir à Récoltes et Semailles. Pour mieux dire, ces deux dernières séries de signes sont apparues alors conjointement. En ce moment même où j'écris, il m'apparaît qu'elles sont à vrai dire indissociables, que ce sont au fond deux aspects ou ``visages'' différents d'une même réalité, faisant irruption en cette année-là dans le champ de mon propre vécu. Pour l'aspect dont je m'apprêtais à parler à l'instant, il s'agit d'une ``fin de non recevoir'' systématique, discrète et sans réplique, réservée par un ``consensus sans failles''\footnote{Ce ``consensus sans failles'' est évoqué sporadiquement ici et là dans l'Avant et Renouvellement, et finit par devenir l'objet d'un témoignage circonstancié et d'une réflexion dans la partie suivante, L'Enterrement (1), avec le ``Cortège X'' ou ``Le Fourgon Funèbre'', forme des ``notes-cercueils'' (n$^{\circ}$ 93-96) et la note ``Le Fossoyeur - ou la Congrégation toute entière''. Celle-ci clôt cette partie de Récoltes et Semailles, et constitue en même temps un premier aboutissement de ce ``deuxième souffle'' de la réflexion.} aux quelques élèves-et-assimilés d'après 1970 qui, par leurs travaux, leur style de travail et leur inspiration, portaient clairement la marque de mon influence. C'est peut-être bien à cette occasion également que, pour la première fois, j'ai perçu ce "souffle de dérision" qui, à travers eux, visait un certain style et une certaine \textbf{approche} de la mathématique - un style et une vision qui (selon un consensus qui était apparemment déjà devenu universel alors dans l'establishment mathématique) \textbf{n'avait pas lieu d'être}.

La preuve, c'était une chose clairement perçue au niveau inconscient. Elle a fini même, cette même année encore à s'imposer à mon attention consciente, après qu'un même scénario aberrant (illustrant l'impossibilité de faire publier une thèse visiblement brillante) s'était répété cinq fois d'affilée, avec l'obstination burlesque d'un gag de cirque. En y repensant à présent, je me rends compte qu'une certaine réalité "me faisait signe" alors avec une insistance bienveillante, alors que je faisais mine de ne pas la saisir (ou de faire la sourde oreille : "Tu, regarde donc grand dadais, fais attention un peu à ce qui se passe là juste sous ton nez, ça concerne mais oui. . . !"). Je me suis secoué un peu, j'ai regardé (l'espace d'un instant), à demi ahuri et distrait à demi : "ah oui, bon, un peu étrange, on dirait bien qu'on en veut à quelqu'un là, quelque chose qui a dû mal tourner décidément, et avec un ensemble aussi parfait encore, c'est même à peine croyable ma parole !"

C'était même à tel point peu croyable que je me suis empressé d'oublier et le gag, et le cirque. Il est vrai que je ne manquais pas d'autres occupations intéressantes. Ça n'a pas empêché le cirque de se rappeler à mon bon souvenir dans les années suivantes encore - non plus dans les tons du gag maintenant, mais bien dans ceux d'une secrète délectation à humilier, ou celui du coup de poing assené en pleine gueule ; à cela près qu'on est entre gens distingués et que le coup de poing prend ici des formes plus discrètes, mais toutes aussi efficaces, laissées à l'inventivité des gens distingués en question. . .

L'épisode que j'ai ressenti comme "un coup de poing en pleine gueule" (d'un autre) se situe en octobre 1981\footnote{Cet épisode est raconté dans la note "Cercueil 3 - ou les jacobiniennes un peu trop relatives" (n° 95), notamment pages 404-406.}. Cette fois-là, et pour la première fois depuis que me parvenaient les signes insistants d'un esprit nouveau, j'étais atteint - plus fortement sans doute que si c'était sur moi que ça avait cogné, au lieu qu'un autre encaisse, que j'avais en affection. Il faisait un peu figure d'élève, et c'était de plus un mathématicien remarquablement doué, et qui venait de faire de belles choses - mais c'est là un détail, après tout. Ce qui n'était pas un détail, par contre, c'est que trois de mes élèves "d'avant" étaient alors directement solidaires d'un acte reçu par l'intéressé (et non sans raison) comme une humiliation et un affront. Deux autres de mes élèves d'antan avaient eu l'occasion déjà de le traiter avec condescendance, en gens cosus envoyant promener un traînesavates\footnote{Il en est question en passant, dans la note citée dans la précédente note de bas de page.}. Un autre élève encore allait d'ailleurs emboîter le pas trois ans plus tard (et dans le style "coup de poing dans la gueule" encore) - mais ça je ne le savais pas encore bien sûr. Ce qui m'interpellait alors était largement suffisant. C'était comme si mon passé de mathématicien, jamais vraiment passant, soudain me marquait dans un, rictus hideux, par la personne de cinq parmi ceux qui furent mes élèves, devenus personnages importants, puissants et dédaigneux. . .

Ça aurait été le moment où jamais alors de poser, de sonder le sens de ce qui m'interpellait soudain avec une telle violence. Mais quelque part en moi il avait été décidé (sans que jamais la chose n'ait eu à être dite. . . ) que ce passé "d'avant" ne me concernait plus au fond, qu'il n'y avait pas lieu que je m'y arrête ; que ce qui semblait m'interpeller maintenant d'une voix que je ne reconnaissais que trop bien - celle du temps du mépris - il y avait décidément maldonne. Et pourtant, j'étais noué d'angoisse, pendant des jours et peut-être des semaines, sans seulement en prendre acte. (C'est l'an dernier seulement, par l'écriture de Récoltes et Semailles qui m'a fait revenir sur cet épisode, que j'ai fini par prendre connaissance de cette angoisse, qu'avait été prise sous contrôle aussitôt qu'apparue.) Au lieu d'en faire le constat et d'en sonder le sens, je me suis agité, j'ai écrit à droite et à gauche, ``les lettres qui s'imposaient''. Les intéressés ont même pris la peine de me répondre, des lettres choisies il va de soi et qui n'entraient dans le fond de rien. Les vagues ont fini par se calmer, et tout est rentré dans l'ordre. Je n'ai guère dû y repenser, avant l'an dernier. Cette fois, pourtant, il était resté comme une blessure, ou comme une écharde douloureuse, plutôt, qu'on évite de toucher ; une écharde qui entretient cette blessure qui ne demande qu'à se refermer\ldots

Ça a été là, sûrement, l'expérience la plus douloureuse et la plus pénible que j'ai vécue dans ma vie de mathématicien - quand il m'a été donné de voir (sans pourtant consentir à vraiment \textbf{prendre connaissance} de ce que mes yeux voyaient) ``tel élève ou compagnon d'antan que j'ai aimé, prendre plaisir à écraser discrètement tel autre que j'aime et en qui il me reconnaît''. Elle m'a marqué alors plus fortement, sûrement, que les découvertes pourtant assez dingues que j'ai faites l'an dernier, et qui (pour un regard superficiel) peuvent paraître tout autrement incroyables\ldots Il est vrai que cette expérience avait fait entrer en résonance plusieurs autres, dans les mêmes tonalités mais moins violentes, et qui sur le coup avaient un peu ``passé à l'as''.

Cela me fait me rappeler, aussi, que cette même année 1981 a été celle aussi d'un tournant draconien dans ma relation au seul parmi les élèves d'antan avec lequel je sois resté en relations régulières après mon départ, et celui aussi qui depuis une quinzaine d'années, avait fait figure d' ``interlocuteur privilégié'' pour moi, au niveau mathématique. C'est l'année en effet où ``les signes d'une affectation de dédain'' qui étaient apparus depuis quelques années déjà\footnote{Il est question de cet épisode dans la note ``Deux tournants'' (n$^{\circ}$ 66).} ``se sont soudain faits si brutaux'' que j'ai cessé alors toute communication mathématique avec lui. C'était quelques mois avant l'épisode-coup-de-pointe de tantôt. Avec le recul la coïncidence me paraît saisissante, mais je ne crois pas avoir fait alors le moindre rapprochement. J'étais rangé dans des ``casiers'' séparés ; des casiers, dont quelqu'un, au surplus, avait déclaré qu'ils ne tiraient pas vraiment à conséquence - la cause était entendue !

Ce qui me frappe, aussi, c'est qu'au mois de juin de cette même année 1981 encore, avait eu lieu déjà un certain brillant Colloque, mémorable à plus d'un titre - un colloque qui aura bien mérité d'entrer dans l'Histoire (ou dans ce qui en reste\ldots) sous le nom indélébile de ``Colloque Pervers''. J'ai fait connaissance (ou plutôt, il m'a dégringolé dessus !) le 2 mai l'an dernier, deux semaines après la découverte (le 19 avril) de l' Enterrement en chair et en os, et j'ai compris aussitôt que je venais de tomber sur ``l' \textbf{Apothéose}''. L'apothéose d'un enterrement, certes, mais aussi, une \textbf{apothéose} du mépris de ce qui, depuis plus de deux mille ans que notre science existe, a été le fondement tacite et immuable de l'éthique du mathématicien : savoir, cette règle élémentaire, de ne pas présenter comme siens les idées et résultats pris chez un autre. Et en prenant note à l'instant de cette coïncidence remarquable dans le temps, entre deux événements qui peuvent sembler de nature et de portée très différentes, je suis saisi de voir se révéler ici le lien profond et évident entre le \textbf{respect de la personne}, et celui des règles éthiques élémentaires d'un art ou d'une science, qui font de son exercice autre chose qu'une ``foire d'empoigne'', et de l'ensemble de ceux qui sont connus pour y exceller et qui y donnent le ton, autre chose qu'une ``maffia'' sans scrupules. Mais à nouveau j'anticipe\ldots

