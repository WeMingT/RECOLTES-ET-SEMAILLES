\section{Trois pieds dans un plat}

Plusieurs parmi mes collègues et amis mathématiciens ont exprimé l'espoir que Récoltes et Semailles ouvre un large débat dans le milieu mathématique, sur l'état des mœurs dans ce milieu, sur l'éthique du mathématicien, et sur le sens et la finalité de son travail. Pour le moment, le moins qu'on puisse dire, c'est que ça n'en prend pas le chemin. Dès à présent (et pour faire le jeu de mots de rigueur) le débat sur un Enterrement a tout l'air d'être remplacé d'office par l'enterrement d'un débat !

Cela n'empêche, qu'on le veuille ou non et malgré le silence et l'apathie du grand nombre, qu'un débat se trouve bel et bien ouvert. Il est peu probable qu'il prenne jamais l'ampleur d'un véritable débat public, voire même (qu'à Dieu ne plaise !) la pompe et la raideur du débat "officiel". Nombreux en tous cas sont ceux qui d'ores et déjà ont pris les devants vite fait, pour le fermer en leur for intérieur avant même d'en avoir pris connaissance, forts du sempiternel et immuable consensus que "tout est pour le mieux dans le meilleur des mondes" (mathématiques, en l'occurrence). Peut-être pourtant qu'une mise en cause finira par venir du dehors, progressivement, par des "témoins" qui, ne faisant pas partie du même milieu, ne sont pas prisonniers de ses consensus de groupe, et qui ne se sentent donc pas (même en leur for intérieur) mis en cause personnellement.

Dans presque tous les échos reçus, je constate une même confusion au sujet des deux questions préalables : sur quoi porte le "débat" posé (du moins tacitement) par Récoltes et Semailles; et qui est apte à en prendre connaissance et à s'y prononcer, ou encore : à se faire une opinion en pleine connaissance de cause. À ce propos, je voudrais ici bien marquer trois "points de repère". Cela n'empêchera pas, certes, ceux qui tiennent à la confusion de continuer à s'y maintenir. Du moins, pour ceux qui voudraient savoir de quoi il retourne, peut-être cela pourra-t-il les aider à ne pas se laisser distraire par les bruitages tous azimuts (y compris même les mieux intentionnés...).

a) Tels amis sincères m'assurent que "tout va finir par s'arranger" (ou "tout", j'imagine, signifie des "choses" qui se seraient malencontreusement abîmées...); que je n'avais qu'à faire ma rentrée, "m'imposer par de nouveaux travaux", donner des conférences etc - et les autres feraient le reste. On dira généreusement "On a été un peu injuste quand même avec ce sacré Grothendieck", et de rectifier le tir discrètement et avec plus ou moins de conviction \footnote{J'ai eu occasion de noter déjà plusieurs tels signes discrets, montrant qu'on a pris bonne note que le lion s'est réveillé...}; voire, le lui tapoter l'épaule d'un air paterne en lui donnant du "grand mathématicien", histoire de calmer un quidam somme toute respectable, qui fait mine hélas de s'énerver et de faire des vagues indésirables.

Il ne s'agit nullement, comme le suggèrent ces amis, de "lâcher du lest" ou d'en faire lâcher. Je n'ai, pour ma part, nul besoin de compliments ni même d'admirateurs sincères, et pas non plus d' "aliés", pour "ma" cause ou pour quelque cause que ce soit. Ce n'est pas de moi qu'il s'agit, qui me porte à merveille, ni de mon œuvre, qui parle pour elle-même, fût-ce à des sourds. Si ce débat concerne aussi, entre autres, ma personne et mon œuvre, c'est simplement à titre de révélateurs d'autre chose, à travers la réalité d'un Enterrement (des plus révélateurs en effet).

S'il y a "quelqu'un" qui me paraît devoir inspirer un sentiment d'alarme, d'inquiétude et d'urgence, ce n'est nullement ma personne, ni même aucun des mes "coenterrés". Mais il s'agit d'un être collectif, à la fois insaisissable et très tangible, dont on parle souvent et qu'on se garde bien d'examiner jamais, et qui a nom "la communauté mathématique".

Au cours de ces dernières semaines, j'ai fini par la voir comme une personne en chair et en os, et dont le corps serait frappé d'une gangrène profonde. La meilleure nourriture, les plats les plus choisis, en elle se tournent en poison, qui fait se propager et s'incruster davantage le mal. Pourtant, il y a une boulimie irrésistible de se gaver encore et toujours davantage, comme une façon sûrement de se donner le change, au sujet d'un mal dont elle ne voudrait prendre connaissance à aucun prix. Quoi qu'on puisse lui dire est peine perdue - les mots mêmes les plus simples ont perdu leur sens. Ils cessent d'être porteurs d'un message, et ne servent plus qu'à déclencher les déclics de la peur et du refus...

b) La plupart de mes collègues ou anciens amis même bien disposés, quand ils hasardent une opinion, s'entourent de conditionnels prudents, du genre "s'il était vrai que... ce serait en effet inadmissible" - histoire de se recoucher contents sur leurs deux oreilles. J'avais cru pourtant être clair...

Avec le recul de sept mois, je puis préciser maintenant que pour la quasi-totalité des faits rapportés et commentés dans Récoltes et Semailles, leur réalité ne fait l'objet d'aucune controverse. Je reviendrai plus loin sur les quelques rares exceptions, qui seront d'ailleurs signalées comme telles, chacune en son lieu, pour tous les autres faits, après l'écriture de la version primitive de Récoltes et Semailles, une confrontation soigneuse avec certains des principaux concernés (à savoir, Pierre Deligne, Jean-Pierre Serre et Luc Illusie) a permis d'éliminer les erreurs de détail, et d'arriver à un accord sans ambiguïté au sujet des faits matériels eux-mêmes \footnote{Je suis heureux d'exprimer ma reconnaissance à tous les trois, pour la bonne volonté dont ils ont fait preuve en cette occasion, et leur donne acte pour leur bonne foi totale, pour tout ce qui concerne les questions de faits matériels.}.

Ainsi, le débat ne porte nullement sur la réalité des faits, laquelle n'est pas en cause, mais sur la question si les pratiques et les attitudes décrites par ces faits doivent être considérées comme admises et comme "normales", ou non.

Il s'agit ici de pratiques que dans mon témoignage je qualifie (à tort peut-être...) de scandaleuses ; comme des abus de confiance ou de pouvoir et comme des malhonnêtetés flagrantes, atteignant plus d'une fois la dimension de l'unique et de l'éhonté. La chose assez inimaginable qu'il me restait à apprendre encore, après avoir pris connaissance de ces faits (impensables il y a encore quinze ans), c'est qu'une grande majorité parmi mes collègues mathématiciens, et jusque parmi ceux qui furent mes élèves ou des amis, considère aujourd'hui ces pratiques comme normales et parfaitement honorables.

c) Il y a une deuxième façon pour beaucoup de mes collègues et anciens amis pour maintenir une confusion. C'est sur l'air du : "désolé, mais on n'est pas spécialiste en la matière - ne nous demande pas de prendre connaissance de faits, qui nous passent (providentiellement... ) par dessus la tête...".

J'affirme, au contraire, que pour prendre connaissance des faits principaux, point n'est besoin d'être "spécialiste" (désolé à mon tour !), ni même de connaître sa table de multiplication ou le théorème de Pythagore. Pas même d'avoir lu "Le Cid" ou les Fables de la Fontaine. Un enfant de dix ans normalement développé en est tout aussi capable que le plus réputé des spécialistes (voire même, mieux que lui...) \footnote{Bien entendu, ce n'est pas à l'intention de l'enfant de dix ans que j'ai écrit Récoltes et Semailles, et pour m'adresser à lui je choisirais un langage qui lui soit familier.}.

Qu'on me permette d'illustrer ce point par juste un exemple, le "premier venu" tiré de l'Enterrement \footnote{Il s'agit de la première "grande opération" d'Enterrement que j'aie découverte, un certain 19 avril 1984, où c'est aussi imposé à moi le nom "l'Enterrement". Voir à ce sujet les deux notes écrites le même jour, "Souvenir d'un rêve - ou la naissance des motifs", et "L'Enterrement - ou le Nouveau père" (Res III, n ${ }^{\circ}$ s 51, 52). On y trouve aussi la référence complète du livre dont il va être question.}. Point n'est besoin de connaître les tenants et aboutissants de la notion mathématique multiforme et fort délicate de "motif", ni d'avoir seulement son certificat d'études, pour prendre connaissance des quelques faits suivants, et pour porter un jugement à leur sujet.

$1^{\circ}$ ) Entre 1963 et 1969 j'ai introduit la notion de "motif" ; et j'ai développé autour de cette notion une "philosophie" et une "théorie", restées partiellement conjecturales. A tort ou à raison (peu importe ici), je considère la théorie des motifs comme ce que j'ai apporté de plus profond à la mathématique de mon temps. L'importance et la profondeur du "yoga motivique" n'est d'ailleurs aujourd'hui plus contestée par personne (après dix ans d'un silence quasi-complet à son sujet, dès après mon départ de la scène mathématique).

$2^{\circ}$ ) Dans le premier et seul livre (publié en 1981), consacré pour l'essentiel à la théorie des motifs (et où ce nom, introduit par moi, figure dans le titre du livre), le seul et unique passage qui puisse faire soupçonner au lecteur que ma modeste personne soit liée de près ou de loin à quelque théorie qui pourrait ressembler à celle développée en long et en large dans ce livre, se trouve à la page 261. Ce passage (de deux lignes et demie) consiste à expliquer au lecteur que la théorie développée là n'a rien à voir avec celle d'un dénommé Grothendieck (théorie mentionnée là pour la première et dernière fois, sans autre référence ni précision).

$3^{\circ}$ ) Il y a une conjecture célèbre, dite "conjecture de Hodge" (peu importe de quoi elle parle au juste), dont la validité impliquerait que la soi-disante "autre" théorie des motifs développée dans le brillant volume, est identique à (un cas très particulier de) celle que j'avais développée, au vu et su de tous, près de vingt ans avant.

Je pourrais ajouter un $4^{\circ}$ ) que le plus prestigieux parmi les quatre cosignataires du livre a été mon élève, et que c'est de nul autre que de moi qu'il a appris au fil des ans les brillantes idées qu'il présente là comme s'il venait de les trouver à l'instant \footnote{Je n'entends pas dire qu'il n'y a pas dans ce livre des idées, et même de belles idées, dues à cet auteur ou aux autres co-auteurs. Mais toute la problématique du livre, et le contexte conceptuel qui lui donne son sens, et jusqu'y compris la théorie délicate des $X$-catégories (appelées à tort "tannakiennes"), laquelle techniquement constitue le cœur du livre, sont mon œuvre.}, et $5^{\circ}$ ) que ces deux circonstances sont de notoriété publique parmi les gens bien informés, mais que c'est en vain qu'on chercherait dans la littérature une trace écrite attestant que ledit brillant auteur pourrait avoir appris quelque chose par ma bouche \footnote{A l'exception cependant d'une ligne dans un rapport de la plume de Serre, en 1977, dont il sera question en son lieu.}, et que $6^{\circ}$ ) la délicate question d'arithmétique qui (selon ce que m'en a expliqué l'auteur principal en personne) constitue le problème central du livre (et sans que mon nom ne soit prononcé), avait été dégagée par moi dans les années soixante, dans la foulée du "yoga des motifs", et que c'est par moi que l'auteur en a eu connaissance ; et je pourrais empiler encore des $7^{\circ}$ et $8^{\circ}$ etc (ce que je ne manque certes pas de faire en son lieu).

Ce qui précède suffira à mon propos, qui est celui-ci. Pour prendre connaissance de tels faits et porter un jugement à leur sujet, point n'est besoin de "compétences" particulières - ce n'est pas à ce niveau-là "que ça se passe". La faculté qui est en jeu ici, à part la saine raison (dévolue en principe à tout un chacun) est ce que j'appellerais du nom de sentiment de décence.

Le livre en question est dès à présent un des plus cités de la littérature mathématique, et son "auteur principal", un des mathématiciens les plus prestigieux de l'époque. Ceci dit et bien vu, la chose à présent de loin la plus remarquable à mes yeux, dans cette histoire, c'est que personne parmi les innombrables lecteurs de ce livre, y compris parmi ceux qui savent de première main de quoi il retourne, et qui furent mes élèves, ou mes amis - que personne n'y a rien vu d'anormal. Il n'y en a pas un en tous cas, jusqu'à aujourd'hui encore où j'écris ces lignes, qui se soit fait connaître à moi pour exprimer au sujet de ce livre prestigieux la moindre réserve \footnote{(**)Il y a eu en tout et pour tout deux collègues (y compris Zoghman Mebkhout) qui m'aient exprimé de telles "réserves". Ni l'un ni l'autre ne peuvent passer pour "lecteurs" de ce livre. Ils l'ont regardé par curiosité, histoire de se rendre compte...}(**).

Quant à ceux, parmi mes collègues et anciens amis, qui n'ont jamais tenu ce livre entre leurs mains et qui s'en prévalent pour plaider l'incompétence, je leur dis : point n'est besoin d'être "spécialiste" pour demander le volume dans la première bibliothèque mathématique venue, le feuilleter, et constater par vous-même ce qui n'est contesté par personne...

