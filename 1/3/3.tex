\section{Le décès du patron - chantiers à l'abandon}

Il est temps, je sens, de donner quelques explications : pourquoi j'ai quitté si abruptement un monde dans lequel, apparemment, je m'étais senti à l'aise pendant plus de vingt ans de ma vie ; pourquoi j'ai eu l'idée étrange de ``revenir'' (tel un revenant\ldots) alors qu'on s'était fort bien passé de moi pendant ces quinze ans ; et pourquoi enfin une introduction à un ouvrage mathématique de six ou sept cent pages en est arrivée à en faire douze (ou quatorze) cents. Et ici aussi, en entrant dans le vif du sujet, de je vais sans doute te chagriner (désolé !), voire même te fâcher. Car nul doute que, comme moi naguère, tu aimes à voir ``en rose'' le milieu dont tu fais partie, où tu as ta place, ton nom et tout ça. Je sais ce que c'est\ldots Et là, ça va grincer un peu\ldots

Je parle ici et là dans Récoltes et Semailles de l'épisode de mon départ, sans trop m'y arrêter. Ce ``départ'' y apparaît plutôt comme une césure importante dans ma vie de mathématicien - c'est par rapport à ce ``point'' que constamment se situent les événements de ma vie de mathématicien, comme ``avant'' et ``après''. Il a fallu un choc d'une grande force pour m'arracher à un milieu où j'étais fortement enraciné, et à une ``trajectoire'' fortement tracée. Ce choc est venu par la confrontation, dans un milieu auquel j'étais identifié fortement, à une certaine forme de corruption\footnote{Il s'agit ici de la collaboration sans réserve, ``establishment'' en tête, de l'ensemble des scientifiques de tous les pays avec les appareils militaires, comme source commode de financements, de prestige et de pouvoir. Cette question est à peine effleurée en passant, une ou deux fois, dans Récoltes et Semailles, par exemple dans la note ``Le respect'' du 2 avril dernier (n° 179, pages 1221 - 1223).} sur laquelle jusque là j'avais choisi de fermer les yeux (en m'abstenant simplement de ne pas y participer). Avec le recul, je me rends compte qu'au delà de l'événement, il y avait pourtant une force plus profonde à l'œuvre en moi. C'était un intense \textbf{besoin de renouvellement intérieur}. Un tel renouvellement ne pouvait s'accomplir et se poursuivre dans les cadres ambiants d'une scientifique d'une institution de grand standing. Derrière moi, vingt ans de créativité mathématique intense et d'investissement mathématique démesuré - et, en même temps aussi, vingt longues années de stagnation spirituelle, en ``vase clos''\ldots Sans m'en rendre compte, j'étouffais - c'est de l'air du large que j'avais besoin ! Mon ``départ'' providentiel a marqué la fin soudaine d'une longue stagnation, et un premier pas vers une élucidation des forces profondes en mon être, pliées et vissées dans un état de déséquilibre intense, figé\ldots Ce départ a été, véritablement, un \textbf{nouveau départ} - le premier pas dans un nouveau voyage\ldots

Comme je l'ai dit, ma passion mathématique n'était pas éteinte pour autant. Elle a trouvé expression dans des réflexions qui sont restées sporadiques, dans des voies toutes différentes de celles que je m'étais tracées ``avant''. Quant à \textbf{l’oeuvre} que je laissais derrière moi, celle ``d'avant'', celle publiée ou sur blanc que celle, plus essentielle peut-être, qui n'avait pas trouvé encore le chemin de l'écriture ou du texte publié - il pouvait bien sembler, et il me semblait en effet, qu'elle s'était détachée de moi. Avant l'an dernier, avec Récoltes et Semailles, l'idée ne m'était jamais venue de ``poser'' tant soit peu sur les ébats épars qu'm'en revenaient, ici et là. Je savais bien que tout ce que j'avais fait en maths, et plus particulièrement, dans ma période ``géométrique'' de 1955 à 1970, étaient des choses qui \textbf{devaient} être faites - et les choses que j'avais vues ou entrevues, étaient des choses qui \textbf{devaient} apparaître, qu'il \textbf{fallait} tirer au grand jour. Et aussi, que le travail que j'avais fait, et celui que j'avais fait faire, était du travail bien fait, du travail où je m'étais mis tout entier. J'y avais mis toute ma force et tout mon amour, et (ainsi me semblait-il) il était autonome désormais - une chose vivante et vigoureuse - qui n'avait plus besoin que je la materne. De ce côté là, je suis parti l'esprit parfaitement tranquille. Je n'avais aucun doute que ces choses écrites et non écrites que je laissais, je les laissais en de bonnes mains, qui sauraient veiller à ce qu'elles se déploient, qu'elles croissent et se multiplient suivant leur nature propre de choses vivantes et vigoureuses.

Dans ces quinze ans de travail mathématique intense, avait éclos, mûri et grandi en moi une vaste \textbf{vision} unificatrice, s'incarnant en quelques idées-force très simples. La vision était celle d'une "géométrie arith\-métique", synthèse de la topologie, de la géométrie (algébrique et analytique), et de l'arithmétique, dont j'ai trouvé un premier embryon dans les conjectures de Weil. C'est elle qui a été ma principale source d'inspi\-ration en ces années, qui pour moi sont celles surtout où j'ai dégagé les idées maîtresses de cette géométrie nouvelle, et où j'ai façonné quelques uns de ces principaux outils. Cette vision et ces idées-force sont deve\-nues pour moi comme une seconde nature. (Et après avoir cessé tout contact avec elles pendant près de quinze ans, je constate aujourd'hui que cette "seconde nature" est toujours vivante en moi !) Elles étaient pour moi si simples, et si évidentes, qu'il allait de soi que "tout le monde" les avait assimilées et fait siennes au fur et à mesure, en même temps que moi. C'est tout dernièrement seulement, en ces derniers mois, que je me suis rendu compte que ni la vision, ni ces quelques "idées-force" n'avaient été mon guide constant, ne se trouvent écrits en toutes lettres dans aucun texte publié, si ce n'est tout au plus entre les lignes. Et surtout, que cette vision que j'avais crû communiquer, et ces idées-force qui la portent, restent aujourd'hui encore, vingt ans après avoir atteint une pleine maturité, ignorées de tous. C'est moi, l'ouvrier, et le serviteur de ces choses que j'ai eu le privilège de découvrir, qui suis aussi le seul en qui elles soient toujours vivantes.

Tel outil et tel autre que j'avais façonné, est utilisé ici et là pour "fracturer" un problème réputé difficile, comme on forcerait un coffre-fort. L'outil apparemment est solide. Pourtant, je lui connais une autre "force" encore que celle d'une pince monseigneur. Il fait partie d'un Tout, comme un membre fait partie du corps - un Tout dont il est issu, et qui lui donne son sens et dont il tire sa force. Tu peux utiliser un os s'il est arraché pour fracturer un crâne, c'est une chose entendue. Mais ce n'est pas là sa vraie fonction, sa raison d'être. Et je vois ces outils épars dont se sont emparés les uns et les autres, un peu comme des os, soigneusement dépecés et nettoyés, qu'ils auraient arrachés à un corps - à un corps vivant qu'ils feraient mine d'ignorer\ldots

Ce que je dis là en termes purement passés, au terme d'une longue réflexion, a dû être perçu par moi peu à peu et de façon diffuse, au fil des ans, au niveau de l'informulé qui ne cherche encore à prendre forme dans une pensée et dans des images conscientes, et par la parole clairement articulée. J'avais décidé que ce passé, au fond, ne me concernait plus. Les échos qui me parvenaient de loin en loin, tout filtrés qu'ils étaient, étaient pourtant éloquents, pour peu que je m'y arrête. Je m'étais crû un ouvrier parmi d'autres, s'affairant sur cinq ou six "chantiers"\footnote{Je m'exprime au sujet de ces "chantiers" désertés, et les passe finalement en revue, dans la suite des notes "Les chantiers désolés" (n\textsuperscript{o} 176 à 178), d'il y a trois mois. Une première fois où je reprends contact avec mon œuvre et sur le sort qui a été le sien, dans la note "Mes orphelins" (n\textsuperscript{o} 46).} en pleine activité - un ouvrier plus expérimenté peut-être, l'aîné qui naguère avait œuvré seul en ces mêmes lieux, pendant de longues années, avant que ne vienne une relève bienvenue ; l'aîné, soit, mais au fond pas différent des autres. Et voilà que, celui-là parti, c'était comme une entreprise de maçonnerie qui aurait déclaré faillite, suite au décès imprévu du patron ; du jour au lendemain, autant dire, les chantiers été désertés. Les "ouvriers" sont partis, chacun emportant sous son bras les menues bricoles dont il pensait avoir l'usage chez lui. La caisse était partie, et il n'y avait plus aucune raison désormais qu'il continue à se fatiguer à bosser\ldots

C'est là encore, une formulation qui s'est décantée d'une réflexion et d'une enquête se poursuivant sur plus d'une année. Mais sûrement, c'était une chose perçue "quelque part" déjà, dès les premières années après mon départ. Mettant à part les travaux de Deligne sur les valeurs absolues des valeurs propres de Frobenius (la "question prestige", comme j'ai compris dernièrement\ldots) - quand il m'arrivait de loin en loin de rencontrer d'anciens élèves d'antan, avec lesquels j'avais travaillé sur les mêmes chantiers, et que je lui demandais alors\ldots "?", c'était toujours le même geste éloquent, les bras en l'air comme pour demander grâce\ldots Visible\-ment, eux étaient occupés à des choses plus importantes que celles qui me tenaient à cœur - et visiblement, aussi, alors que tous s'affairaient avec des airs occupés et importants, pas grand chose ne se faisait. L'essentiel avait disparu - une \textbf{unité} qui donnait leur sens aux tâches partielles, et une \textbf{chaleur} aussi, je crois. Il restait un éparpillement de tâches détachées d'un tout, chacun dans son coin couvant son petit magot, ou le faisant fructifier tant bien que mal.

Alors même que j'aurais voulu m'en défendre, ça me peinait bien sûr d'entrevoir que tout c'était arrêté net : de ne plus entendre parler ni de motifs, ni de topos, ni des six opérations, ni des coefficients de De Rham, ni de ceux de Hodge, ni du ``foncteur mystérieux'' qui devait relier entre elles, en un même éventail, autour des coefficients de De Rham, les coefficients $\ell$-adique pour tous les nombres premiers, ni des cristaux (si ce n'est pour apprendre qu'ils en sont toujours au même point), ni des ``conjectures standard'' et autres que j'avais dégagées et qui, à l'évidence, représentaient des questions cruciales. Même le vaste travail de fondements commencé avec les Éléments de Géométrie Algébrique (avec l'inlassable assistance de Dieudonné), qu'il aurait suffi quasiment de continuer sur la lancée déjà acquise, était laissé pour compte : tout le monde se contentait de s'installer dans les murs et dans les meubles qu'un autre avait patiemment assemblés, montés et briqués. L'ouvrier parti, il ne serait venu à l'idée de personne de retrousser ses manches à son tour et de mettre la main à la truelle, pour construire les nombreux bâtiments qui restaient à construire, des maisons, bonnes pour y vivre, pour soi-même et pour tous\ldots

Je n'ai pu m'empêcher encore, à nouveau, d'enchaîner avec des images pleinement conscientes, qui se sont dégagées et sont remontées par la vertu d'un travail de réflexion. Mais il n'y a aucun doute pour moi que ces images-là devaient déjà être présentes sous une forme ou une autre, dans les couches profondes de mon être. J'ai dû sentir déjà la réalité insidieuse d'un \textbf{Enterrement} de mon œuvre en même temps que de ma personne, qui s'est imposé à moi soudain, avec une force irrécusable et avec ce nom même, ``L'Enterrement'', le 19 avril de l'an dernier. Au niveau conscient, par contre, je n'aurais guère songé à m'offusquer ni même à m'affliger. Après tout, ``proche'' de naguère ou pas, ça ne regardait que l'intéressé, à quoi il choisissait d'occuper son temps. Si ce qui avait semblé le motiver ou l'inspirer naguère ne l'inspirait plus, c'était là son affaire, et pas la mienne. Si la même chose semblait arriver, avec un ensemble parfait, à tous mes ex-élèves sans exception, c'était encore là l'affaire de chacun d'eux séparément et j'avais d'autres chats à fouetter que d'aller chercher quel sens ça pouvait avoir, un point c'est tout ! Quant à ces choses que j'avais laissées - et auxquelles un lien profond et ignoré continuait à me relier - alors même qu'elles étaient visiblement laissées à l'abandon, sur ces chantiers désolés, je savais bien, moi, qu'elles n'étaient pas de celles qui craignent ``l'injure du temps'' ni les fluctuations des modes. Si elles n'étaient entrées encore dans le patrimoine commun (comme il m'avait pourtant semblé naguère), elle ne pourraient manquer de s'y enraciner tôt ou tard, dans dix ans ou dans cent, peu importait au fond\ldots