\section{La lettre de mille pages}

Mai 1985

Le texte que je te fais parvenir ici, tapé et tiré à un nombre limité d'exemplaires par les soins de mon université, n'est pourtant ni un tirage à part, ni un preprint. Son titre, Récoltes et Semailles, l'annonce bien assez clairement. Je te l'envoie comme j'enverrais une longue lettre - une lettre tout ce qu'il y a de personnelle, en plus. Si je te l'envoie, au lieu me contenter que tu en prennes connaissance un jour (si tu en as la curiosité) dans quelque volume en vente en librairie (s'il y a éditeur assez fou pour courir l'aventure\ldots), c'est parce que j'y adresse à toi plus qu'à d'autres. Plus d'une fois en l'écrivant j'ai pensé à toi - il faut dire que ça fait plus d'une année que je l'écris, cette lettre, en m'y mettant tout entier. C'est un don que je te fais, et j'ai pris grand soin en écrivant de donner ce que j'avais (à chaque moment) de meilleur à offrir. Je ne sais si le don sera accueilli - ta réponse (ou ta non-réponse\ldots) me le fera savoir\ldots

En même temps qu'à toi, je fais parvenir Récoltes et Semailles à tous ceux de mes collègues, amis ou (ex-)élèves dans le monde mathématique, auxquels j'ai été lié de près à quelque moment, ou qui figurent dans ma réflexion d'une façon ou d'une autre, nommément ou non. Il y a des chances que tu y figures, et si tu lis avec ton coeur et non seulement avec les yeux et la tête, sûrement tu te reconnaîtras même là où tu n'es pas nommé. J'envoie également Récoltes et Semailles à quelques autres amis encore, scientifiques ou non.

Cette "lettre d'introduction" que tu es en train de lire, qui t'annonce et te présente une "lettre de mille pages" (pour commencer\ldots), tiendra lieu aussi d'Avant-Propos. Ce dernier n'est pas écrit encore au moment d'écrire ces lignes. Récoltes et Semailles consiste par ailleurs en cinq parties (sans compter une introduction "à tiroirs"), Je t'envoie ici les parties I (Fatuité et Renouvellement), II (L'Enterrement (1) - ou la Robe de l'Empereur de Chine), et IV (L'Enterrement (3) - ou les Quatre Opérations) \footnote{Je mets à part les collègues qui figurent dans ma réflexion à un titre ou un autre, mais que je ne connais pas personnellement. Je me borne à leur envoyer "Les Quatre Opérations" (qui les concerne plus particulièrement), en même temps que le "fascicule 0" consistant en cette lettre, et en l'introduction à Récoltes et Semailles (plus la table des matières détaillée de l'ensemble des quatre premières parties).}. Ce sont celles dont il m'a semblé qu'elles te concerneraient plus particulièrement. La partie III (L'Enterrement (2) - ou la Clef du Yin et du Yang) est sans doute la partie la plus personnelle de mon témoignage, et celle en même temps qui, plus encore que les autres, me paraît avoir une valeur "universelle", au delà des circonstances particulières qui ont entouré sa naissance. Je réfère à cette partie ici et là dans la partie IV (Les Quatre Opérations), laquelle pourtant peut être lue indépendamment, et même (dans une large mesure) indépendamment des deux parties qui précèdent\footnote{De façon générale, tu pourras constater que chaque "section" (dans Fatuité et Renouvellement) ou chaque "note" (dans quelconque des trois parties suivantes de Récoltes et Semailles) a son unité et son autonomie propres. Elle peut être lue indépendamment du reste, tout comme on peut trouver intérêt et plaisir à regarder une main, un pied, un doigt ou un oeil ou toute autre portion grande ou petite du corps tout entier, sans oublier pour autant que c'est là une partie d'un Tout, et que c'est ce Tout seulement (lequel reste dans le non-dit) qui donne tout son sens.}.

Si la lecture de ce que je t'envoie ici t'incite à me répondre (comme c'est mon souhait), et si elle te donne envie de lire aussi la partie manquante, fais-le moi savoir. Je me ferai un plaisir de te la faire parvenir, pour peu que ta réponse me fasse sentir que ton intérêt dépasse celui d'une curiosité toute superficielle.
