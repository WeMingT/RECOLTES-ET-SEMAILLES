\section{Mouvement et structure}

Je crois que j'ai fini de faire le tour des choses les plus importantes que j'avais envie de te dire au sujet de Récoltes et Semailles, pour te faire savoir déjà "de quoi il s'agit", sûrement, j'en ai dit plus qu'assez pour te permettre de juger si toi, tu considères que la lettre de (plus de) mille pages qui doit suivre "te concerne", ou non - et par suite, si tu vas ou non continuer ta lecture. Pour le cas où ce serait "oui", il me semble utile de joindre encore quelques explications (de nature pratique, notamment) au sujet de la forme de Récoltes et Semailles.

Cette forme est le reflet et l'expression d'un certain esprit, que j'ai essayé de faire "passer" dans les pages qui précèdent. Par rapport à mes publications passées, s'il y a une qualité nouvelle qui apparaisse dans Récoltes et Semailles, et également dans "A la Poursuite des Champs" dont il est issu, c'est sans doute la spontanéité. Certes, il y a des fils conducteurs, et des grandes interrogations, qui donnent sa cohérence et son unité à l'ensemble de la réflexion. Celle-ci pourtant se poursuit au jour le jour, sans "programme" ou "plan" préétabli, sans qu'il soit question jamais de me fixer d'avance "ce qu'il fallait démontrer". Mon propos n'est pas de démontrer, mais bien de découvrir, de pénétrer plus avant dans une substance inconnue, de faire se condenser ce qui n'est encore que pressenti, soupçonné, entrevu. Je peux dire, sans aucune exagération vraiment, que dans ce travail, il n'y a pas un seul jour ni une seule nuit de réflexion qui se soit déroulé dans le champ du "prévu", en termes des idées, images, associations qui étaient présentes au moment où je me suis assis devant la feuille blanche, pour y poursuivre obstinément un "fil" tenace, ou pour en reprendre un autre qui vient d'apparaître. À chaque fois, ce qui apparaît dans la réflexion est autre que ce que j'aurais su prédire, si je m'étais hasardé à essayer de décrire d'avance tant bien que mal ce que je croyais voir devant moi. Le plus souvent, la réflexion s'engage dans des voies entièrement imprévues au départ, pour déboucher sur des paysages nouveaux, tout aussi imprévus. Mais alors même qu'elle s'en tiendrait à un itinéraire plus ou moins prévu, ce que me révèle le voyage au fil des heures diffère autant de l'image que j'en avais en me mettant en route, qu'un paysage réel, avec ses jeux d'ombre fraîche et de chaude lumière, sa perspective délicate et changeante au gré des pas du randonneur, et ces sons innombrables et ces parfums sans nom portés par une brise qui fait danser les herbes et chanter les futaies... - qu'un tel paysage vivant, insaisissable, diffère d'une carte postale, si belle et réussie, si "juste" soit-elle.

C'est la réflexion poursuivie d'une traite, au cours d'une journée ou d'une nuit, qui constitue l'unité indivise, la cellule vivante et individuelle en quelque sorte; dans l'ensemble de la réflexion (Récoltes et Semailles, en l'occurrence). Celle-ci est à chacune de ces unités (ou ces "notes" \footnote{Originellement, en écrivant Fatuité et Renouvellement, le nom "note" était pour moi synonyme d' "annotation", jouant le rôle d'une note de bas de page. Pour des raisons de commodité typographique, j'avais préféré rejeter ces annotations à la fin du texte (notes 1 à 44, pages 141 et 171). Une des raisons pour ce faire, était que certaines de ces "notes" ou "annotations" s'étendent sur une ou plusieurs pages, et deviennent plus longues même que le texte qu'elles sont censées commenter. Quant aux "unités" indivises du "premier jet" de la réflexion, à défaut d'un meilleur nom je les ai appelées alors "sections" (moins rébarbatif que "paragraphes"!).

Cette situation, et la structure du texte, change avec la partie suivante, qui initialement s'appelait "L'Enterrement", et qui est devenue "L'Enterrement (1)" (ou "La robe de l'Empereur de Chine"). Cette réflexion a enchaîné sur la double-note "Mes orphelins" et "Refus d'un héritage - ou le prix d'une contradiction" (notes n ${ }^{\circ}$ s 46, 47, pages 177, 192), venant en annotation à la "section" ultime de Récoltes et Semailles (ou plutôt, de ce qui allait être sa partie I, ou Fatuité et Renouvellement), "Le poids d'un passé" ( $\mathrm{n}^{\circ} 50$, p. 131). Par la suite, s'y sont rajoutées d'autres annotations à cette même section (les notes $\mathrm{n}^{\circ} \mathrm{s} 44^{\prime}$ et 50 ), et d'autres notes encore venant en annotations à "Mes orphelins", qui à leur tour donnaient naissance à de nouvelles notes annotantes; sans compter, cette fois, de véritables notes de bas de page, quand les annotations prévues étaient (et restaient, une fois mises noir sur blanc) de dimensions modestes. Ainsi, théoriquement, toute cette partie-là de Récoltes et Semailles (qui était censée alors en constituer la partie deuxième et terminale) apparaissait comme un ensemble de "notes" à la "section" "Poids d'un passé". Par l'inertie acquise, cette subdivision en "notes" (au lieu de "sections") s'est maintenue encore dans les trois parties suivantes, où j'utilise conjointement, comme moyen d'annotation pour un "premier jet" de la réflexion, aussi bien la note de bas de page (quand ses dimensions le permettent), que la note ultérieure auquel il est fait renvoi dans le texte.

Typographiquement, la "note" se distingue de la "section" (utilisée dans RS I comme unité de base du "premier jet" de la réflexion) par un signe tel que (1), (2) etc (comprenant le numéro de la note placé entre parenthèses et "en l'air", suivant un usage répandu pour les renvois à des annotations), placé soit au début de la note en question, soit à titre de renvoi à l'endroit approprié du texte qui réfère à elle. Les sections sont désignées par les chiffres arabes de 1 à 50 (à l'exclusion de rébarbatifs indices et exposants, comme j'ai été amené à en utiliser pour les notes, par des impératifs de nature pratique). Cela dit, on peut dire qu'il n'y a aucune différence essentielle entre la fonction des "sections" dans la première partie de Récoltes et Semailles et celle des "notes" dans les parties ultérieures. Les commentaires que je fais au sujet de cette fonction dans la présente partie de ma lettre ("Spontanéité et structure") s'appliquent aussi bien aux "sections" de RS I, alors même que j'utilise le nom commun "notes".

Pour d'autres précisions et conventions, concernant notamment la lecture de la table des matières de l'Enterrement (1), je renvoie à l'Introduction, 7 (L'Ordonnancement des Obsèques), et notamment pages xiv - xv.}, formant "mélodie...") dont chacune dès lors reçoit son propre nom et par là acquiert une identité et une autonomie propres. En d'autres moments par contre, une réflexion qui s'était trouvée écourtée pour une raison ou une autre (fortuite le plus souvent), se prolonge spontanément le lendemain ou surlendemain; ou une réflexion poursuivie sur deux ou plusieurs journées consécutives apparaît pourtant, rétrospectivement, comme si elle s'était poursuivie d'une seule traite ; on dirait que seul le besoin du sommeil nous ait obligé, à notre corps défendant, d'y inclure quelque césure (en quelque sorte "physiologique"), marquée seulement par une lapidaire indication de date (voire, par plusieurs) entre tels alinéas consécutifs de la "note" envisagée, laquelle se distingue alors comme telle par un nom unique.
Ainsi, chacune des notes de Récoltes et Semailles a son individualité propre, un visage et une fonction qui la distinguent de toute autre. Pour chacune, j'ai essayé d'exprimer sa particularité propre par son nom, censé restituer ou évoquer l'essentiel, ou tout au moins quelque chose d'essentiel, de ce qu'elle "a à dire". Chacune, je la reconnais véritablement, avant toute autre chose, par son nom, et c'est par ce nom aussi que je l'appelle, chaque fois que par la suite j'ai besoin de son concours.

Souvent le nom s'est présenté à moi spontanément, avant même que j'y aie songé. C’est son apparition inopinée qui me signale, alors, que cette note-là que je suis encore en train d'écrire est sur le point d'être achevée - qu'elle a dit ce qu'elle avait à dire, le temps de terminer l'alinéa que je suis en train d'écrire... Souvent aussi, le nom apparaît, tout aussi spontanément, en relisant les notes de la veille ou de l'avant-veille, avant de poursuivre ma réflexion. Parfois, il se modifie quelque peu au cours des jours ou des semaines qui suivent l'apparition de la note nouvelle venue, où il s'enrichit d'un deuxième nom auquel je n'avais pas songé tout d'abord. Beaucoup de notes ont un double nom, exprimant deux éclairages différents, parfois complémentaires, de son message. Le premier de ces doubles-noms qui se soit présenté à moi, dès les débuts de "Fatuité et Renouvellement", est "Rencontre avec Claude Chevalley - ou liberté et bons sentiments" ( $\mathrm{n}^{\circ}$ 11).

Deux fois seulement ai-je eu déjà un nom en tête avant de commencer une note - et les deux fois, d'ailleurs, il a été bousculé par la suite des événements !

C'est, avec le recul seulement, recul de semaines, voire de mois, qu'apparaît un mouvement d'ensemble et une structure dans l'ensemble des notes se suivant au jour le jour. J'ai essayé de saisir l'un et l'autre par divers groupements et sous-groupements de notes, chacun d'eux avec son nom à lui, qui lui confère son existence propre et sa fonction ou son message; un peu comme pour les organes et les membres d'un même corps (pour reprendre l'image de tantôt), et telles parties de ses membres. Ainsi, dans "le Tout" Récoltes et Semailles, il y a les cinq "parties" dont j'ai déjà parlé, dont chacune a une structure bien à elle : Fatuité et Renouvellement se groupe en huit "chapitres" I à VIII \footnote{Dans Fatuité et Renouvellement, je réfère à l'occasion à ces chapitres comme des "parties" de Récoltes et Semailles, qu'il ne faut pas confondre, bien sûr, avec les cinq parties dont il a déjà été question, et qui ne sont apparues qu'ultérieurement.}, et l'ensemble des trois parties formant ${ }^{\text {I }}$ ' Enterrement (qui, elles aussi se sont dégagées progressivement au fil des mois...) est formé d'une longue et solennelle Procession de douze "Cortèges" I à XII. Le dernier de ceux-ci, ou plutôt la "Cérémonie Funèbre" (c'est là son nom) vers quoi s'étaient acheminés (sans trop se douter de rien, sûrement...) les onze Cortèges précédents, est de dimensions véritablement gigantesques, à la mesure de l'Œuvre dont elle consacre les solennelles Obsèques : elle englobe la quasi-totalité de RS III (L' Enterrement (2)) et la totalité de RS IV (L' Enterrement (3)), avec ses près de huit cents pages et dans les cent cinquante notes (alors qu'initialement, cette fameuse cérémonie n'était prévue pour en comporter que deux !). Conduite avec doigté (et avec sa modestie bien connue...) par le grand officiant en personne, la cérémonie se poursuit en neuf "temps" ou actes liturgiques séparés, ouverte par l' Eloge Funèbre (on s'en serait douté), et s'achevant (comme il se doit) en le De Profundis final. Deux autres parmi ces "temps", nommés l'un "La Clef du Yin et du Yang", l'autre "Les Quatre Opérations", constituent chacun (et de loin) la plus grande part de la partie (III ou IV) de Récoltes et Semailles dans laquelle il s'insère, et donne d'ailleurs son nom à celle-ci.

Tout au long de Récoltes et Semailles, j'ai pris soin (comme de la prunelle de mes yeux !) de la table des matières, la remaniant sans cesse pour tenir compte de l'afflux toujours renouvelé de notes imprévues \footnote{Parmi ces notes imprévues, il y a notamment celles qui sont "issues d'une note de bas de page qui a pris des dimensions prohibitives". Le plus souvent, je l'ai placée immédiatement après la note à laquelle elle se rapporte, en lui donnant le même numéro affecté d'un exposant ' ou ", voire "' au besoin - ce qui évite la tâche prohibitive d'avoir à renuméroter du même coup l'ensemble de toutes les notes ultérieures déjà écrites ! Ces notes, issues d'une note de bas de page à une autre, sont précédées dans la table des matières par le signe ! (tout au moins dans l'Enterrement (1)).}, et lui faire refléter de façon aussi fine que je le pouvais le mouvement d'ensemble de la réflexion et la structure délicate qui s'y fait jour. C'est dans les parties III et surtout IV (dont il vient d'être question), "La Clef" et "Les Quatre Opérations", que cette structure se trouve être la plus complexe et la plus imbriquée.

Pour préserver au texte le caractère de spontanéité, et les aspects d'imprévu de la réflexion telle qu'elle s'est poursuivie et qu'elle a été vécue réellement, je n'ai pas voulu faire précéder les notes par leur nom, alors que celui-ci à chaque fois n'est apparu qu'après-coup seulement. C'est pourquoi je te conseille, en fin de lecture de chaque note, de te reporter à la table des matières pour y apprendre comment cette note s'appelle ; et aussi, à l'occasion, pour pouvoir apprécier en un simple coup d'œil comment elle s'insère dans la réflexion déjà poursuivie, voire même, dans celle encore à venir. Autrement tu risques de te perdre sans espoir dans un ensemble en apparence indigeste et hétéroclite de notes aux numérotations parfois bizarres, pour ne pas dire rébarbatives \footnote{Pour la raison d'être de telles numérotations d'apparence peut-être saugrenue par moments, je te réfère à la précédente note de bas de page à cette intarissable lettre.}; comme un voyageur égaré dans une ville étrangère (poussée là bizarrement au gré du caprice des générations et des siècles...), sans un guide ni seulement un plan pour l'aider à s'y orienter. Dans le manuscrit destiné à l'impression, je compte inclure au fil du texte les noms de "chapitres" et autres groupements de notes et de sections, à la seule exclusion des notes (ou sections) elles-mêmes. Mais même alors, le recours occasionnel à la table des matières me paraît indispensable, pour ne pas se perdre dans un fouillis de centaines de notes, se suivant à la queue-leu-leu sur plus de mille pages...

