\section{Le respect et la fortitude}

Mais à nouveau je digresse, en anticipant sur ce que la réflexion m'a enseigné. J'étais parti d'un double propos, clairement présent en moi dès avant même les débuts de celle-ci : le propos d'une ``méditation d'intentions'', et (intimement lié à celui-ci, comme il vient d'apparaître) celui de m'exprimer au sujet : de la nature du travail créateur. Il y avait pourtant un troisième propos encore, moins clairement présent sûrement au niveau conscient, mais répondant à un besoin plus profond et plus essentiel. Il était suscité par ces ``interpellations'' parfois déconcertantes, me parvenant de mon passé de mathématicien par la voix de ceux qui avaient été mes élèves ou mes amis (ou du moins, de bon nombre d'entre eux). Au niveau épidermique, ce besoin se traduisait par une envie de ``vider mon sac'', de dire quelques ``vérités désagréables''. Mais plus profondément, sûrement, il y avait le besoin de \textbf{faire connaissance} enfin avec un certain passé, que j'avais choisi jusque là d'éluder. C'est de ce besoin-là, avant tout, qu'est issu Récoltes et Semailles. Cette longue réflexion a été ma ``réponse'', au jour le jour, à cette pulsion de connaissance en moi, et à l'interpellation sans cesse renouvelée qui me venait du monde extérieur, du ``monde mathématique'' que j'avais quitté sans esprit de retour. Mis à part les toutes premières pages de ``Fatuité et Renouvellement'', celles qui en forment les deux premiers chapitres (``Travail et découverte'' et ``Le rêve et le Rêveur''), et dès le chapitre qui enchaîne ``Naissance de la crainte'' (p. 18), avec un ``témoignage'' qui n'était nullement prévu au programme, c'est ce besoin de faire connaissance de mon passé et de l'assumer pleinement, qui (je crois) a été la force principale en œuvre dans l'écriture de Récoltes et Semailles.

L'interpellation qui m'était venue du monde des mathématiciens, et qui revenait sur moi avec une force nouvelle tout au cours de Récoltes et Semailles (et surtout, au cours de l' ``enquête'' poursuivie dans les parties II et IV), avait pris d'emblée le masque de la suffisance, quand ce n'était celui du dédain (``délicatement dosé''), de la dérision ou du mépris, que ce soit vis-à-vis de moi (parfois) ou (surtout) vis-à-vis de ceux qui avaient osé s'inspirer de moi (sans se douter, certes, de ce qui les attendait) et qui étaient ``classés'' comme ayant partie liée à moi, par quelque décret tacite et implacable. Et à nouveau je vois apparaître ici le lien ``évident'' et profond'', entre le \textbf{respect} (ou l'absence de respect) pour la personne d'autrui ; celui pour l'acte de création et pour certains de ses fruits les plus délicats et les plus essentiels ; et enfin le respect pour les règles les plus évidentes de l'éthique scientifique : celles qui s'enracinent dans un respect élémentaire de soi et d'autrui et que je serais tenté d'appeler les ``règles de décence'' dans l'exercice de notre art. Ce sont là autant d'aspects, sûrement, d'un élémentaire et essentiel ``respect de soi''. Si j'essaie, en une seule formule lapidaire, de faire le bilan de ce que m'a enseigné Récoltes et Semailles au sujet d'un certain monde qui fut le mien, un monde auquel je m'étais identifié pendant plus de vingt ans de ma vie, je dirais : c'est un monde qui a \textbf{perdu le respect}\footnote{La encore, c'est une formulation qui ne s'applique pas seulement à un certain milieu limité, où j'ai eu ample occasion de voir la chose de près, mais elle me paraît résumer une certaine dégradation dans l'ensemble du monde contemporain. (Comparer avec la note de b. de p. page 1. 19.) Dans le cadre plus limité du bilan d'une ``enquête'' poursuivie dans Récoltes et Semailles, cette formulation apparaît dans la note du 2 avril dernier, ``Le respect'' (n$^{\circ}$ 179).}.

C'était là une chose déjà fortement sentie, sinon formulée, dès les années qui avaient précédé. Elle n'a fait que se confirmer et se préciser, de façon imprévue toujours et parfois stupéfiante, tout au cours de Récoltes et Semailles. Elle est clairement apparente dès le moment déjà où une réflexion de nature ``philosophique'' et générale devient soudain un témoignage personnel (dans la section ``L'étranger bienvenu'' (n$^{\circ}$ 9, p. 18) ouvrant le chapitre déjà cité ``Naissance de la crainte''.

Cette perception n'apparaît pourtant pas sur le ton de la récrimination acerbe ou amère, mais (par la logique interne de l'écriture et par l'attitude différente que celle-ci suscite) sur celui d'une \textbf{interrogation} : quelle a été ma propre part dans cette dégradation, dans cette perte du respect que je constate aujourd'hui ? C'est là l'interrogation principale qui traverse et porte cette première partie de Récoltes et Semailles, jusqu'au moment où elle se résoud finalement en une constatation claire et sans équivoque\footnote{Dans les sections ``La mathématique sportive'' et ``Fin la marge'' (n$^{\circ}$ s 40, 41).}. Auparavant, cette dégradation m'était apparue comme ``tombée du ciel'' soudain, de façon inexplicable et d'autant plus outrageuse, intolérable. Au cours de la réflexion, je découvre qu'elle s'était poursuivie insidieusement, sans que personne sûrement ne la décèle autour de lui ni en lui-même, tout au long des années cinquante et soixante, \textbf{y compris dans ma propre personne}.

La constatation de cet humble fait, bien évident sûrement et sans apparence, marque un premier tournant crucial dans le témoignage, et un changement qualitatif immédiat\footnote{Dès le lendemain, le témoignage s'approfondit en une méditation sur moi-même, et garde cette qualité particulière dans les semaines qui suivent, jusqu'à la fin de ce ``premier souffle'' de Récoltes et Semailles (avec la section ``Le poids d'un passé'', n$^{\circ}$ 50).}. C'était là une première chose essentielle que j'avais à apprendre, sur mon passé de mathématicien et sur moi-même. Cette connaissance d'une \textbf{part de responsabilité} qui m'incombait dans la dégradation générale (connaissance plus ou moins aiguë suivant les moments de la réflexion) est restée comme une note de fond et comme un rappel, tout au cours de Récoltes et Semailles. Il en a été ainsi, surtout, aux moments où ma réflexion prenait les allures d'une enquête sur les disgrâces et sur les iniquités d'une époque. Conjointement au désir de comprendre, à la curiosité donc qui anime et porte en avant tout vrai travail de découverte, c'est cette humble connaissance (maintes fois oubliée en chemin et refaisant surface malgré tout, là où on s'y attendait le moins\ldots) qui a préservé mon témoignage de jamais virer (je crois) à la récrimination stérile sur l'ingratitude du monde, voire au ``règlement de compte'' avec certains de ceux qui avaient été mes élèves ou des amis (ou les deux). Cette absence de complaisance vis-à-vis de moi-même m'a donné également ce calme intérieur, ou cette fortitude, qui m'ont préservé des pièges de la complaisance vis-à-vis d'autrui, où ne serait-ce que ceux d'une fausse ``discrétion''. Tout ce que je croyais avoir à dire, à un moment ou à un autre de la réflexion, que ce soit sur moi, ou sur tel de mes collègues, ex-élèves ou amis, ou sur un milieu, ou sur une époque, je l'ai dit, sans avoir jamais à bousculer mes réticences. Pour celles-ci, il a suffi à chaque fois que je les examine avec attention, pour qu'elles s'évanouissent sans laisser de traces.




