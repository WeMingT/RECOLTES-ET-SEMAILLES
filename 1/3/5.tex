\section{Le voyage}

Je crois que j'ai à peu près fait le tour, là, du contexte dans lequel s'est placé mon ``retour aux maths'', et, de fil en aiguille, l'écriture de Récoltes et Semailles. C'est fin mars l'an dernier, dans la toute dernière section de Fatuité et Renouvellement (``Le poids d'un passé'' (n$^{\circ}$ 50)), que je songe enfin à m'interroger sur les raisons et sur le sens de ce retour inattendu. Pour ce qui est des ``raisons'', la plus forte de toutes sûrement était l'impression, diffuse et impérieuse en même temps, que ces choses fortes et vigoureuses, que j'avais crû naguère confier entre des mains aimantes - c'est dans un tombeau, coupé des bienfaits du vent, de la pluie et du soleil qu'elles ont croupi pendant ces quinze ans où je les avais perdues de vue\footnote{Voir ``Le poids d'un passé'' (section n$^{\circ}$ 50), notamment p. 137, (**).}. J'ai dû comprendre, peu à peu et sans que jamais avant aujourd'hui j'aie songé à me le dire, que ce ne serait nul autre que moi qui ferait enfin sauter ces planches vermoulues, retenant prisonnières des choses vivantes faites, non pour pourrir en cercueils clos, mais pour s'épanouir au grand air. Et ces airs de fausse componction et d'insidieuse dérision autour de ces cercueils capitonnés et pléthoriques (à l'image du rejeton défunt, à ne pas douter\ldots), ont dû aussi ``finir par réveiller en moi une fibre de combativité qui s'était quelque peu assoupie au cours des dernières dix années'' et l'envie de me lancer dans la mêlée\ldots\footnote{Citation extraite de la note ``La mélodie au tombeau - ou la suffisance'' (n$^{\circ}$ 167), page 826.}.

C'est ainsi, il y a deux ans, que ce qui était d'abord prévu comme une rapide prospection, de quelques jours ou de quelques semaines à tout casser, d'un de ces ``chantiers'' laissés pour compte, est devenu un grand feuilleton mathématique en N volumes, s'insérant dans la fameuse nouvelle série des ``Réflexions'' (``mathématiques'', en attendant d'élaguer ce qualificatif inutile). Dès l'instant d'ailleurs où j'ai su que j'étais en train d'écrire un ouvrage mathématique destiné à publication, j'ai su aussi que j'allais y joindre, en plus d'une introduction ``mathématique'' plus ou moins conforme aux usages, une autre ``introduction'' encore, de nature plus personnelle. Je sentais qu'il était important que je m'explique sur mon ``retour'', lequel n'était nullement le retour dans un \textbf{milieu}, mais le ``retour'' seulement à un investissement mathématique intense et à la publication de textes mathématiques de ma plume, pendant une durée indéterminée. Également, je voulais m'expliquer sur l'esprit dans lequel j'écrivais maintenant les maths, très différent à certains égards de l'esprit de mes écrits d'avant mon départ - l'esprit ``journal de bord'' d'un voyage de découverte. Sans compter qu'il y avait d'autres choses que j'avais sur le cœur, liées à celles-ci sans doute, mais que je sentais plus essentielles encore. Il était bien entendu pour moi que j'allais prendre mon temps pour dire ce que j'avais à dire. Ces choses-là, encore diffuse, étaient inséparables pour moi du sens qu'allaient avoir ces volumes que je m'apprêtais à écrire, et les ``Réflexions'' dans lesquelles ils allaient s'insérer. Il n'était pas question de les glisser là à la sauvette, comme en m'excusant d'abuser du temps précieux d'un lecteur pressé. S'il y avait choses dans ``A la Poursuite des Champs'' dont il était bon, pour lui et pour tous, qu'il prenne connaissance, c'étaient celles justement que je me réservais de dire dans cette introduction. Si vingt ou trente pages ne devaient pas y suffire, à les dire, j'y mettrais quarante, voire cinquante, qu'à cela ne tienne - sans compter que je n'obligeais personne à me lire\ldots

C'est ainsi qu'est né Récoltes et Semailles. J'ai écrit les premières pages de l'introduction prévue au mois de juin 1983, à un moment creux dans l'écriture du volume premier de La Poursuite des Champs. Puis j'ai remis à en février l'an dernier, alors que mon volume était pratiquement terminé depuis plusieurs mois\footnote{[Entre-temps] avais passé un bon mois à réfléchir à la ``surface structurale'' pour un système de pseudo-droites, obtenue en termes de l'ensemble de toutes les ``positions relatives'' possibles d'une pseudo-droite rapport à un tel système. J'ai également écrit ``L'Esquisse d'un Programme'', qui sera inclus dans le volume 3 des Réflexions.}. Je comptais bien que cette introduction serait une occasion pour m'éclairer sur deux ou trois choses qui restaient un tantinet floues dans mon esprit. Mais je n'avais aucun soupçon que ça allait être, tout comme le volume que je venais d'écrire, un \textbf{voyage de découverte} ; un voyage dans un monde autrement plus riche encore et de plus vastes dimensions que celui que je m'apprêtais à prospecter, dans le volume écrit et dans ceux qui devaient suivre. C'est au fil des jours, des semaines et des mois, sans trop me rendre compte de ce qui arrivait, que s'est poursuivi ce nouveau voyage, à la découverte d'un certain passé (obstinément éludé pendant plus de trois décennies\ldots) et de moi-même et des liens qui me relient à ce passé ; à la découverte aussi de certains de ceux qui furent mes proches dans le monde mathématique, et que j'ai si mal connus ; et enfin même, dans la foulée et par surcroît, un voyage de découverte mathématique, alors que pour la première fois depuis quinze ou vingt ans\footnote{Dans les années cinquante et soixante, j'avais souvent réprimé mon envie de me lancer à la poursuite de telles questions juteuses et brûlantes, accaparé que j'étais par d'interminables tâches de fondements, que personne n'aurait su ou voulu poursuivre à ma place, et que personne après mon départ n'a eu non plus à cœur de continuer\ldots}, je prenais loisir de revenir sur certaines des questions que j'avais laissées, brûlantes, au moment de mon départ. Je peux dire, en somme, que ce sont \textbf{trois} voyages de découverte, intimement entrelacés, que je poursuis dans les pages de Récoltes et Semailles. Et aucun des trois n'est achevé avec le point final, à la page douze cents et quelques. Les échos, déjà, que va recueillir mon témoignage (et jusques y compris l'écho par le silence\ldots) feront partie de la ``suite'' du voyage. Quant à son à terme, ce voyage sûrement est de ceux qui ne sont jamais menés à terme - pas même, si ça se trouve, au jour de notre mort\ldots

Et me voilà enfin revenu au point de départ : te dire d'avance, si faire se peut, ``de quoi il est question'' dans Récoltes et Semailles. Mais il est vrai aussi que sans l'avoir même cherché, les pages précédentes te l'ont déjà dit peu ou prou. Il sera plus intéressant, peut-être, de continuer sur ma lancée et de \textbf{raconter}, plutôt que d' ``annoncer''.

\begin{flushright}
Juin 1985
\end{flushright}


