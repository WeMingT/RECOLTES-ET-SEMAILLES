\section{Le versant d'ombre - ou création et mépris}

Les pages précédentes ont été écrites à la faveur d'un court ``moment creux'', le mois dernier. Entre-temps, j'ai enfin fini de mettre la dernière main aux ``Quatre Opérations'' (la quatrième partie de Récoltes et Semailles) - il ne me reste plus qu'à terminer encore cette lettre ou ``pré-lettre'' (qui elle aussi finit mine de prendre des dimensions prohibitives\ldots) pour que tout soit prêt enfin pour la frappe et pour la duplication. Je n'y croyais plus, à force, depuis bientôt un an et demi que je suis ``sur le point de terminer'' ces fameuses notes ! En me mettant à cette ``introduction'' de nature un peu inhabituelle pour un ouvrage mathématique, au mois de février l'an dernier (et déjà l'année d'avant, au mois de juin), il y avait (je crois) trois genres de choses surtout sur lesquelles j'avais envie alors de m'exprimer. Tout d'abord, je voulais m'expliquer sur mes intentions en revenant à une activité mathématique, et sur l'esprit dans lequel j'avais écrit ce premier volume de ``A la Poursuite des Champs'' (que je venais de déclarer terminé), et sur l'esprit aussi dans lequel je comptais poursuivre un voyage de prospection et de découverte mathématique plus vaste encore, avec les ``Réflexions''. Il ne s'agirait plus pour moi, désormais, de présenter des fondations méticuleuses et à quatre épingles pour quelque nouvel univers mathématique en gésine. Ce seraient des ``carnets de bord'' plutôt, où le travail se poursuivrait au jour le jour, sans rien en cacher et tel qu'il se poursuit vraiment, avec ses ratés et ses erreurs, ses insistants retours en arrière et aussi ses soudains bonds en avant - un travail tiré en avant irrésistiblement jour après jour (et nonobstant les incidents et imprévus innombrables), comme par un invisible fil - par quelque vision élusive, tenace et sûre. Un travail tâtonnant bien souvent, surtout en ces ``moments sensibles'' où affleure, à peine perceptible, quelque intuition sans nom encore et sans visage ; ou au départ de quelque nouveau voyage, à l'appel et à la poursuite de quelques premières idées et intuitions, élusives souvent et réticentes à se laisser saisir dans les mailles du langage, alors que c'est justement le langage adéquat pour les saisir avec délicatesse qui souvent fait encore défaut. C'est un tel langage, avant toute autre chose, qu'il s'agit alors de faire se condenser hors d'un apparent néant de brumes impalpables. Ce qui n'est encore que pressenti, avant d'être seulement entrevu et encore moins ``vu'' et touché du doigt, peu à peu se décante de l'impondérable, se dégage de son manteau d'ombre et de brumes pour prendre forme et chair et poids\ldots

C'est cette partie-là du travail, de piètre apparence pour ne pas dire (bien des fois) foireux, qui en est aussi la partie la plus délicate et la plus essentielle - celle où, véritablement, quelque chose de nouveau fait son apparition, par l'effet d'une attention intense, d'une sollicitude, d'un respect pour cette chose fragile, infiniment délicate, sur le point de naître. C'est la partie créatrice entre toutes - celle de la conception et d'une lente gestation dans les chaudes ténèbres de la matrice nourricière, depuis l'invisible double gamète originelle, devenant informe embryon et se transformant au fil des jours et des mois, par un travail obscur et intense, invisible et sans apparence, en un nouvel être en chair et en os.

C'est là aussi la partie ``obscure'', la partie ``yin'' ou ``\textbf{féminine}'' du travail de découverte. L'aspect complémentaire, la partie ``claire'', ou ``yang'' ou ``\textbf{masculine}'', s'apparenterait plutôt au travail à coups de marteau ou de masse, sur un burin bien affûté ou sur un coin de bon acier trempé. (Des outils déjà tout prêts à l'usage, et d'une efficacité qui a fait déjà ses preuves\ldots) L'un et l'autre aspect a sa raison d'être et sa fonction, en symbiose inséparable l'un avec l'autre - ou pour mieux dire, ce sont la Yin et le Yang et l'Époux du couple indissoluble des deux forces cosmiques originelles, dont l'étreinte sans cesse renouvelée fait resurgir sans cesse les obscurs labeurs créateurs de la conception, de la gestation et de la naissance - de la naissance de \textbf{l'enfant}, de la chose nouvelle.

La deuxième chose sur laquelle je sentais le besoin de m'exprimer, dans ma fameuse ``introduction'' personnelle et ``philosophique'' à un texte mathématique, c'était au sujet de la nature du travail créateur justement. Je m'étais rendu compte déjà, depuis des années, que cette nature était généralement ignorée, occultée par des clichés à tout venant et par des répressions et des peurs ancestrales. A quel point il en est bien ainsi, je l'ai découvert après seulement, progressivement, au fil des jours et des mois, tout au cours de la réflexion et de l' ``enquête'' poursuivie dans Récoltes et Semailles. C'est dès le ``moment d'envoi'' de cette réflexion, au cours des quelques pages datées de juin 1983, que je suis pour la première fois saisi par la portée de ce fait d'anodine apparence, et pourtant stupéfiant, pour peu seulement qu'on s'y arrête tant soit peu : que cette partie ``créatrice entre toutes'' dont je viens de parler dans le travail de découverte, ne \textbf{transparaît pratiquement nulle part} dans les textes ou discours qui sont censés présenter un tel travail (ou du moins, ses fruits les plus tangibles) ; que ce soient des manuels et autres textes didactiques, ou les articles et mémoires originaux, ou les cours oraux et exposés de séminaires etc. Il y a, depuis des millénaires semblerait-il, depuis les origines même de la mathématique et des autres arts et sciences, une sorte de ``conspiration du silence'' autour de ces ``\textbf{inavouables labeurs}'' qui préludent à l'éclosion de toute idée nouvelle, grande ou petite, venant renouveler notre connaissance d'une portion de ce monde, en création perpétuelle, où nous vivons.

Pour tout dire, il semblerait que la répression de la connaissance de cet aspect-là ou de ce stade-là, le plus crucial de tous dans tout travail de découverte (et dans le travail créateur en général) ; soit à tel point efficace, et tel point intériorisée par ceux-là même qui pourtant connaissent tel travail de première main, que souvent on jurerait que même ceux-là en ont éradiqué toute trace de leur souvenir conscient. Un peu comme dans une société puritaine à outrance, une femme aurait éradiqué de son souvenir, en relation à chacun de ces enfants qu'elle se fait un devoir de moucher et de torcher, le moment de l'étreinte (subie à contre-cœur) qui le fit concevoir, les longs mois de la grossesse (vécue comme une inconvenance), et les longues heures de l'accouchement (endurées comme un pur ragoûtant calvaire, suivi enfin d'une délivrance).

Cette comparaison peut paraître outrée, et elle l'est peut-être en effet, si je l'applique à ce dont je me rappelle aujourd'hui de l'esprit que j'ai connu dans le milieu mathématique dont je faisais moi-même partie, il y a encore vingt ans. Mais au cours de ma réflexion dans Récoltes et Semailles j'ai pu me rendre compte, et de façon saisissante en ces tout derniers mois surtout (avec l'écriture des ``Quatre Opérations''), qu'il y a eu depuis mon départ de la scène mathématique une stupéfiante \textbf{dégradation} dans l'esprit qui aujourd'hui fait loi dans les milieux que j'avais connus, et (me semble-t-il, dans une large mesure au moins) dans le monde mathématique en général\footnote{Cette dégradation ne se limite d'ailleurs nullement au seul ``monde mathématique''. On la constate également dans l'ensemble de la vie scientifique, et au delà encore de celle-ci, dans le monde contemporain à l'échelle planétaire. Une amorce de constat et de réflexion dans ce sens se trouve dans la note ``Le respect et la fortitude'' qui suivra la réflexion sur le yin et le yang (note n$^{\circ}$ 106).}. Il est possible même, tant par ma personnalité mathématique très particulière que par les conditions qui ont entouré mon départ, que celui-ci ait agi comme un catalyseur dans une évolution qui était déjà en train de se faire\footnote{C'est l'évolution examinée dans la note citée dans la précédente note de b. de p. Des liens entre celle-ci et l'Enterrement (de ma personne et de mon œuvre) font leur apparition et s'éclaircissent progressivement au cours de la réflexion sur ``le yin et le yang enterrés yin (4)'', ``La circonstance providentielle - ou l'Apothéose'', ``Le désaveu (1) - ou le rappel'', ``Le désaveu (2) - ou la métamorphose'' (n$^{\circ}$ 124, 151, 152, 153). Voir également les notes plus récentes (dans RS IV) ``Les détails inutiles'' (n$^{\circ}$ 171 (v)), et la note (c) ``Des choses qui ressemblent à rien - ou le dessèchement'') et ``L'album de famille'' (n$^{\circ}$ 173, partie c. ``Celui entre tous - ou l'acquiescement'').} - une évolution dont je n'ai alors rien su percevoir (pas plus qu'aucun autre de mes collègues et amis, à la seule exception peut-être de Claude Chevalley). L'aspect de cette dégradation auquel je pense surtout ici (qui en est juste \textbf{un} aspect parmi de nombreux autres\footnote{L'aspect qui le plus souvent au centre de l'attention dans Récoltes et Semailles, et plus particulièrement dans les deux parties ``enquête'' (RS II ou ``La robe de l'Empereur de Chine'', et RS IV ou ``Les Quatre Opérations''), et celui aussi, peut-être, qui m'a le plus ``secoué'', est la dégradation de l'éthique du métier, s'exprimant par un pillage, un débinage et un maquillage sans vergogne, pratiqué parmi certains des plus prestigieux et des plus brillants des mathématiciens du moment, et ceci (dans une très large mesure) au vu et au su de tous. Pour certains autres aspects plus délicats, et directement liés d'ailleurs à celui-là, je renvoie à la note déjà citée (n$^{\circ}$ 106 partie c.) ``Des choses qui ressemblent à rien - ou le dessèchement''.}) est le \textbf{mépris tacite}, quand ce n'est la dérision sans équivoque, à l'encontre de ce qui (en mathématique, en l'occurrence) ne s'apparente pas au pur travail du marteau sur l'enclume ou sur le burin - le mépris des processus créateurs les plus délicats (et souvent de moindre apparence) ; de tout ce qui est \textbf{inspiration}, \textbf{rêve}, \textbf{vision} (si puissantes et si fertiles soient-elles), et même (à la limite) de toute idée, si clairement conçue et formulée soit-elle : de tout ce qui n'est écrit et publié noir sur blanc, sous forme d'énoncés purs et durs, répertoriables et répertoriés, puis pour les ``banques de données'' engouffrées dans les inépuisables mémoires de nos mégaordinateurs.

Il y a eu (pour reprendre une expression de C.L. Siegel\footnote{Cette expression est citée et commentée dans la note qui vient d'être citée dans la précédente note de b. de p.}) un extraordinaire ``\textbf{aplatissement}'', un ``\textbf{rétrécissement} de la pensée mathématique, dépouillée d'une dimension essentielle, de tout son ``versant d'ombre'', du versant ``yin''. Il est vrai que par une tradition ancestrale, ce versant-là du travail de découverte restait dans une large mesure occultée, personne (autant dire) n'en \textbf{parlait} jamais - mais le contact vivant avec les sources profondes du rêve, qui alimentent les grandes visions et les grands desseins, n'avait jamais encore (à ma connaissance) été perdu. Il semblerait que dès à présent nous soyons déjà entrés dans une \textbf{époque} de dessèchement, où cette source est, non point tarie certes, mais où l'accès à elle est condamné, par le verdict sans appel du mépris général et par les représailles de la dérision.

Nous voilà approcher du moment, semble-t-il, où sera éradiqué en chacun non seulement le \textbf{souvenir} de tout travail proche de la source, du travail ``au féminin'' (ridiculisé comme ``vaseux'', ``mou'', ``inconsistant'' - ou au bout opposé comme ``trivialités'', ``enfantillages'', ``baguenaude''\ldots), mais où sera extirpé également ce travail même et ses fruits : celui où sont conçues, s'élaborent et naissent les notions et les visions nouvelles. Ce sera l'époque aussi où l'exercice de notre art sera réduit à d'arides et vaines exhibitions de ``poids et haltères'' cérébraux, aux surenchères des prouesses pour ``craquer'' les problèmes au concours (``de difficulté proverbiale'') - l'époque d'une hypertrophie ``supermacho'' fiévreuse et stérile, prenant la suite de plus de trois siècles de renouvellement créateur.

