30 janvier 1986

Il ne manquait plus que l'avant-propos à écrire, pour confier Récoltes et Semailles à l'imprimeur. Et je jure que j'étais de la meilleure volonté du monde pour écrire quelque chose qui fasse l'affaire. Quelque chose de \textbf{raisonnable}, cette fois. Trois quatre pages pas plus, mais bien senties, pour présenter cet énorme ``pavé'' de plus de mille pages. Quelque chose qui ``accroche'' le lecteur blasé, qui lui fasse entrevoir que dans ces peu rassurantes ``plus de mille pages'', il pourrait y avoir des choses qui l'intéressent (voir même, qui le concernent, qui sait?). C'est pas tellement mon style, l'accroche, ça non. Mais là j'allais faire l'exception, pour une fois! Il fallait bien que ``l'éditeur assez fou pour courir l'aventure'' (de publier ce monstre, visiblement impubliable) rentre dans ses frais tant bien que mal.

Et puis non, c'est pas venu. J'ai fait de mon mieux pourtant. Et pas qu'un après-midi, comme je comptais le faire, vite fait. Demain ça fera trois semaines pile que je suis dessus, que les feuilles s'entassent. Ce qui est venu, c'est sûr, n'est pas ce qu'on pourrait décemment appeler un ``avant-propos''. C'est encore loupé, décidément! On se refait plus à mon âge -- et je suis pas fait pour, pour vendre ou faire vendre. Même quand il s'agit de faire plaisir (à soi-même, et aux amis\ldots).

Ce qui est venu, c'est une sorte de longue ``promenade'' commentée, à travers mon oeuvre de mathématicien. Une promenade à l'intention surtout du ``profane'' -- de celui qui ``n'a jamais rien compris aux maths''. Et à mon intention aussi, qui n'avais jamais pris le loisir d'une telle promenade. De fil en aiguille, je me suis vu amené à dégager et à dire des choses qui jusque là étaient toujours restées dans le non-dit. Comme par hasard, ce sont celles aussi que je sens les plus essentielles, dans mon travail et dans mon oeuvre. C'est des choses qui n'ont rien de technique. A toi de voir si j'ai réussi dans ma naïve entreprise de les ``faire passer'' -- une entreprise un peu folle sûrement, elle aussi. Ma satisfaction et mon plaisir, ce serait d'avoir su te les faire sentir. Des choses que beaucoup parmi mes savants collègues ne savent plus sentir. Peut-être sont-ils devenus trop savants et trop prestigieux. Ça fait perdre contact, souvent, avec les choses simples et essentielles.

Au cours de cette ``Promenade à travers une oeuvre'', je parle un peu de ma vie aussi. Et un petit peu, ici et là, de quoi il est question dans Récoltes et Semailles. J'en reparle encore et de façon plus détaillée, dans la ``Lettre'' (datée de Mai l'an dernier) qui suit la ``Promenade''. Cette Lettre était destinée à mes ex-élèves et à mes ``amis d'antan'' dans le monde mathématique. Mais elle non plus n'a rien de technique. Elle peut être lue sans problème par tout lecteur qui serait intéressé à apprendre, par un récit ``sur le vif'', les tenants et aboutissants qui m'ont finalement amené à écrire Récoltes et Semailles. Plus encore que la Promenade, ça te donnera aussi un avant-goût d'une certaine ambiance, dans le ``grand monde'' mathématique. Et aussi (tout comme la Promenade), de mon style d'expression, un peu spécial paraît-il. Et de l'esprit aussi qui s'exprime par ce style -- un esprit qui lui non plus n'est pas apprécié par tout le monde.

Dans la Promenade et un peu partout dans Récoltes et Semailles, je parle du \textbf{travail mathématique}. C'est un travail que je connais bien et de première main. La plupart des choses que j'en dis sont vraies, sûrement, pour tout travail créateur, tout travail de découverte. C'est vrai tout au moins pour le travail dit ``intellectuel'', celui qui se fait surtout ``par la tête'', et en écrivant. Un tel travail est marqué par l'éclosion et par l'épanouisse-ment d’une \textbf{compréhension} des choses que nous sommes en train de sonder. Mais, pour prendre un exemple au bout opposé, la passion d’amour est, elle aussi, pulsion de découverte. Elle nous ouvre à une connaissance dite "charnelle", qui elle aussi se renouvelle, s’épanouit, s’approfondit. Ces deux pulsions - celle qui anime le mathématicien au travail, disons, et celle en l’amant ou en l’amant - sont bien plus proches qu’on ne le soupçonne généralement, ou qu’on n’est disposé à se l’admettre. Je souhaite que les pages de Récoltes et Semaille puissent contribuer à te le faire sentir, dans ton travail et dans ta vie de tous les jours.

Au cours de la Promenade, il sera surtout question du travail mathématique lui-même. J’y reste quasiment muet par contre sur le contexte où ce travail se place, et sur les \textit{motivations} qui jouent en dehors du temps de travail proprement dit. Cela risque de donner de ma personne, ou du mathématicien ou du "scientifique" en général, une image flatteuse certes, mais déformée. Genre "grande et noble passion", sans correctif d’aucune sorte. Dans la ligne, en somme, du grand "Mythe de la Science" (avec S majuscule s’il vous plaît). Le mythe héroïque, "prométhéen", dans lequel cervains et savants sont tombés (et continuent à tomber) à qui mieux mieux. Il n’y a guère que les historiens, peut-être, qui y résistent parfois, à ce mythe si séduisant. La vérité, c’est que dans les motivations "du scientifique", qui parfois le poussent à investir sans compter dans son travail, l’ambition et la vanité jouent un rôle aussi important et quasiment universel, que dans toute autre profession. Ça prend des formes plus ou moins grossières, plus ou moins subtiles, suivant l’intéressé. Je ne prétends nullement y faire exception. La lecture de mon témoignage ne laissera, j’espère, aucun doute à ce sujet.

Il est vrai aussi que l’ambition la plus dévorante est impuissante à découvrir le moindre énoncé mathématique, ou à le démontrer - tout comme elle est impuissante (par exemple) à "faire bander" (au sens propre du terme). Qu’on soit femme ou homme, ce qui "fait bander" n’est nullement l’ambition, le désir de briller, d’exhiber une puissance, sexuelle en l’occurrence - bien au contraire! Mais c’est la perception aiguë de quelque chose de fort, de très réel et de très délicat à la fois. On peut l’appeler "la beauté", et c’est là un des mille visages de cette chose-là. D’être ambitieux n’empêche pas forcément de sentir parfois la beauté d’un être, ou d’une chose, d’accord. Mais ce qui est sûr, c’est que ce n’est pas l’ambition qui nous la fait sentir. . .

L’homme qui, le premier, a découvert et maîtrisé le feu, était quelqu’un exactement comme toi et moi. Pas du tout ce qu’on se figure sous le nom de "héros", de "demi-dieu" et j’en passe. Sûrement, comme toi et comme moi, il a connu la morsure de l’angoisse, et la pommade vaniteuse éprouvée, qui fait oublier la morsure. Mais au moment où il a "connu" le feu, il n’y avait ni peur, ni vanité. Telle est la vérité dans le mythe héroïque. Le mythe devient insipide, il devient pommade, quand il nous sert à nous cacher un autre aspect des choses, tout aussi réel et tout aussi essentiel.

Mon propos dans Récoltes et Semaille a été de parler de l’un et de l’autre aspect - de la pulsion de connaissance, et de la peur et de ses antidotes vaniteux. Je crois "comprendre", ou du moins connaître la pulsion et sa nature. (Peut-être un jour découvrirai-je, émerveillé, à quel point je me faisais illusion. . .) Mais pour ce qui est de la peur et de la vanité, et les insidieux blocages de la créativité qui en dérivent, je sais bien que je n’ai pas été au fond de cette grande énigme. Et j’ignore si je verrai jamais le fond de ce mystère, pendant les années qui me restent à vivre. . .

En cours d’écriture de Récoltes et Semaille deux images ont émergé, pour représenter l’un et l’autre de ces deux aspects de l’aventure humaine. Ce sont l’\textbf{enfant} (alias l’ouvrier), et le \textbf{Patron}. Dans la Promenade qu’on va faire tantôt, c’est de "l’enfant" qu’il sera question presque exclusivement. C’est lui aussi qui figure dans le sous-titre "L’enfant et la Mère". Ce nom va s’éclairer, j’espère, au cours de la promenade.

Dans tout le reste de la réflexion, c’est le Patron par contre qui prend surtout le devant de la scène. Il n’est pas patron pour rien! Il serait d’ailleurs plus exact de dire qu’il s’agit non pas d’un Patron, mais des Patrons d’entreprises concurrentes. Mais il est vrai aussi que tous les Patrons se ressemblent sur l’essentiel. Et quand on commence à parler des Patrons, ça signifie aussi qu’il y a y avoir des "vilains". Dans la partie I de la réflexion ("Fatigue et Renouvellement", qui fait suite à la présente partie introductive, où le "Prélude en quatre Mouvements"), c’est surtout moi, "le vilain". Dans les trois parties suivantes, c’est surtout "les autres". Chacun son tour !

C’est dire qu’il y aura, en plus de profondes réflexions philosophiques et de "confessions" (nullément contrites), des "portraits au vitriol" (pour reprendre l’expression d’un de mes collègues et amis, qui s’est trouvé un peu malmené. . .). Sans compter des "opérations" de grande envergure et pas piquées de vers. Robert Jaulin\footnote{Robert Jaulin est un ami de vieille date. J’ai cru comprendre que vis-à-vis de l’établissement du milieu ethnologique, il se trouve dans une situation (de "loup blanc") un peu analogue à la mienne vis-à-vis du "beau monde" mathématique.} m’a assuré (en plagiant à demi) que dans Récoltes et Semaille je faisais "l’ethnologie du milieu mathématique" (ou peut-être la sociologie, je ne saurais plus trop dire). On est flatté bien sûr, quand on apprend que (sans même le savoir) on fait des choses savantes ! C’est un fait qu’au cours de la partie "enquête" de la réflexion (et à mon corps défendant. . .), j’ai vu défiler, dans les pages que j’étais en train d’écrire, une bonne partie de l’établissement mathématique, sans compter nombre de collègues et d’amis au statut plus modeste. Et ces derniers mois, depuis que j’ai fait les envois du tirage provisoire de Récoltes et Semaille au mois d’octobre dernier, ça "remis ça" encore. Décidément, mon témoignage est venu comme un pavé dans la mare. Il y a eu des échos un peu sur tous les tons vraiment (sauf celui de l’ennui. . .). Presque à chaque coup, c’était pas du tout ce à quoi je me serais attendu. Et il y a eu aussi beaucoup de silence, qui en dit long. Visiblement, j’en avais (et il me reste) à en apprendre encore, et de toutes les couleurs, sur ce qui se passe dans la caboche des uns et des autres, parmi mes ex-élèves et autres collègues plus ou moins bien situés - pardon, sur la "sociologie du milieu mathématique" je voulais dire! À tous ceux venus d’ores et déjà apporter leur contribution à la grande œuvre sociologique de mes vieux jours, je tiens à exprimer ici-même mes sentiments reconnaissants.

Bien sûr, j’ai été particulièrement sensible aux échos dans les tonalités chaleureuses. Il y a eu aussi quelques rares collègues qui m’ont fait part d’une émotion, ou d’un sentiment (resté inexprimé jusqu’alors) de crise, ou de dégradation à l’intérieur de ce milieu mathématique dont ils se sentent faire partie.

En dehors de ce milieu, parmi les tout premiers à faire un accueil chaleureux, voire ému, à mon témoignage, je voudrais nommer ici Sylvie et Catherine Chevalley\footnote{Sylvie et Catherine Chevalley sont la veuve et la fille de Claude Chevalley, le collègue et ami à qui est dédié la partie centrale de Récoltes et Semaille (RES III, "La Clef du Yin et du Yang"). En plusieurs endroits de la réflexion, je parle de lui, et du rôle qui fut le sien dans mon itinéraire.}, Robert Jaulin, Stéphane Deligeorge, Christian Bourgois. Si Récoltes et Semaille va connaître une diffusion plus étendue que celle du tirage provisoire initial (à l’intention d’un cercle des plus restreints), c’est surtout grâce à eux. Grâce, surtout à leur conviction communicative : que ce que je suis efforcé de saisir et de dire, devait être dit. Et que cela pouvait être entendu dans un cercle plus large que celui de mes collègues (souvent maussades, voire hargneux, et mille fois disposés à se remettre en cause. . .). C’est ainsi que Christian Bourgois n’a pas hésité à courir le risque de publier l’impensable, et Stéphane Deligeorge, de me faire l’honneur d’accueillir mon indigeste témoignage dans la collection "Epistémè", aux côtés (pour le moment) de Newton, de Cuvier et d’Arago. (Je ne pouvais rêver meilleure compagnie!) À chacune et à chacun, pour leurs marques répétées de sympathie et de confiance, survenant à un moment particulièrement "sensible", je suis heureux de dire ici toute ma reconnaissance.

Et nous voilà sur le départ d’une Promenade à travers une œuvre, comme entrée en matière pour un voyage à travers une vie. Un long voyage, oui, de mille pages et plus, et bien tasséé chacune. J’ai mis une vie à le faire, ce voyage, sans l’avoir épuisé, et plus d’une année à le redécouvrir, page après page. Les mots parfois ont été hésitants à venir, pour exprimer tout le jus d’une expérience se dérobant encore à une compréhension hésitante - comme du raisin mûr et dru entassé dans le pressoir semble, par moments, vouloir se dérober à la force qui l’étreint... Mais même en les moments où les mots semblent se bousculer et couler à flots, ce n’est pas au bonheur-la-chance pourtant qu’ils se bousculent et qu’ils coulent. Chacun d’eux a été pesé au passage, ou sinon après-coup, pour être ajusté avec soin s’il a été trouvé trop léger, ou trop lourd. Aussi cette réflexion-témoignage-voyage n’est pas faite pour être lue vite fait, en un jour ou en un mois, par un lecteur qui aurait hâte d’en venir au mot de la fin. Il n’y a pas "de mot de la fin", pas de "conclusions" dans Récoltes et Semailles, pas plus qu’il n’y en a dans ma vie, ou dans la tienne. Il y a un vin, vieilli pendant une vie dans les fûts de mon être. Le dernier verre que tu boiras ne sera pas meilleur que le premier ou que le centième. Ils sont tous "le même", et ils sont tous différents. Et si le premier verre est gâté, tout le tonneau l’est ; autant alors boire de la bonne eau (s’il s’en trouve), plutôt que du mauvais vin.

Mais un bon vin ne se boit pas à la va-vite, ni au pied levé.








