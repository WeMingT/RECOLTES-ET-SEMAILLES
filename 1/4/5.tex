\subsection{5. Une dette bienvenue}

Je voudrais conclure cette introduction par quelques mots au sujet des deux dédicaces au présent volume "Récoltes et Semailles".

La dédicace "à ceux qui furent mes élèves, à qui j’ai donné du meilleur de moi-même - et aussi du pire" a été présente en moi tout au moins dès l'été dernier, et notamment quand j'ai écrit les premières quatre sections de ce qui était encore censé être une introduction à un ouvrage mathématique. C'est dire que je savais bien, en fait depuis quelques années déjà, qu'il y avait un "pire" à examiner - et c'était maintenant le moment ou jamais ! (Mais je ne me doutais pas que ce "pire" finirait par me mener à travers une méditation de près de deux cents pages.)

Par contre, la dédicace "à ceux qui furent mes aînés" est apparue en cours de route seulement, tout comme le nom même de cette réflexion (qui est devenu aussi celui d'un volume). Celle-ci m'a révélé le rôle important qui a été le leur dans ma vie de mathématicien, un rôle dont les effets restent vivants encore aujourd'hui. Cela apparaîtra sans doute assez clairement dans les pages qui suivent - pour qu'il soit inutile ici de m'étendre à se sujet. Ces "aînés", par ordre (approximatif) d'apparition dans ma vie alors que j'avais vingt ans, sont Henri Cartan, Claude Chevalley, André Weil, Jean-Pierre Serre, Laurent Schwartz, Jean Dieudonné, Roger Godement, Jean Delsarte. Le nouveau venu ignare que j'étais a été accueilli avec bienveillance par chacun d'eux, et par la suite beaucoup parmi eux m'ont donné une amitié et une affection durables. Il me faut aussi mentionner ici Jean Leray, dont l'accueil bienveillant lors de mon premier contact avec le "monde des mathématiciens" (en 1948/49) a été également un encouragement précieux. Ma réflexion a fait apparaître une dette de reconnaissance envers chacun de ces hommes "d'un autre monde et d'un autre destin". Cette dette-là n'est nullement un poids. Sa découverte est venue comme une joie, et m'a rendu plus léger.

\hfill Fin mars 1984

