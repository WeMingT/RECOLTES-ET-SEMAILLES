\subsection{2. L'esprit d'un voyage}

Cette réflexion qui a fini par devenir "Récoltes et Semailles" avait commencé comme une "introduction" au premier volume (en cours d'achèvement) de "A la Poursuite des Champs", ce premier travail mathématique que je destine à une publication depuis 1970. J'avais écrit les premières quelques pages à un moment creux, au mois de juin l'an dernier, et j'ai repris cette réflexion il y a moins de deux mois, au point où je l'avais laissée. Je me rendais compte qu'il y avait pas mal de choses à regarder et à dire, je m'attendais donc à une introduction relativement étoffée, de trente ou quarante pages. Puis, pendant les près de deux mois qui ont suivi, jusqu'à maintenant même où j'écris cette nouvelle introduction à ce qui fut d'abord une introduction, je croyais chaque jour que c'était celui où je terminais ce travail, ou que ce serait le lendemain ou le surlendemain au pis. Quant au bout de quelques semaines j'ai commencé à approcher du cap de la centaine de pages, l'introduction a été promue "chapitre introductif". Après quelques semaines encore, quand les dimensions dudit "chapitre" se sont trouvées excéder de loin celles des autres chapitres du volume en préparation (tous terminés au moment d'écrire ces lignes, sauf le dernier), j'ai enfin compris que sa place n'était pas dans un livre de maths, que décidément cette réflexion et ce témoignage y seraient à l'étroit. Leur vraie place était dans un volume séparé, qui sera le volume 1 de ces "Réflexions Mathématiques" que je compte poursuivre dans les années qui viennent, sur la lancée de la Poursuite des Champs.

Je ne dirais pas que Récoltes et Semailles, ce premier volume dans la série des Réflexions Mathématiques (qui sera suivi des deux ou trois volumes de la Poursuite des Champs, pour commencer) est un volume "d'introduction" aux Réflexions. Plutôt, je vois ce premier volume comme le fondement de ce qui est à venir, ou pour mieux dire, comme celui qui donne la note de fond, l'esprit dans lequel j'entreprends ce nouveau voyage, que je compte poursuivre dans les années à venir, et qui me mènera je ne saurais dire où.

Pour terminer ces précisions au sujet de la partie maîtresse du présent volume, quelques indications de nature pratique. Le lecteur ne s'étonnera pas de trouver dans le texte de Récoltes et Semailles des références occasionnelles au "présent volume" - sous-entendu, le premier volume (Histoire de Modèles) de la Poursuite des Champs, dont je crois encore être en train d'écrire l'introduction. Je n'ai pas voulu "corriger" ces passages, tenant avant tout à conserver au texte sa spontanéité, et son authenticité de témoignage non seulement sur un passé lointain, mais aussi sur le moment même où j'écris.

C'est pour la même raison aussi que mes retouches du premier jet du texte se sont bornées à corriger des maladresses de style ou une expression parfois confuse qui nuisaient à la compréhension de ce que je voulais exprimer. Ces retouches parfois m'ont conduit à une appréhension plus claire ou plus fine qu'au moment d'écrire le premier jet. Des modifications tant soit peu substantielles de celui-ci, pour le nuancer, le préciser, le compléter ou (parfois) le corriger, sont l'objet d'une cinquantaine de notes numérotées, groupées à la fin de la réflexion, et qui constituent plus du quart du texte \footnote{Il s'agit ici du texte de la première partie de Récoltes et Semailles, "Fatuité et Renouvellement". La deuxième partie n'était pas écrite au moment d'écrire ces lignes.}. J'y renvoie par des sigles comme (1) etc... Parmi ces notes, j'en ai distingué une vingtaine qui m'ont paru d'une importance comparable (par leur longueur ou leur substance) à celle d'une quelconque des cinquante "sections" ou "paragraphes" en lesquels spontanément la réflexion s'est organisée. Ces notes plus longues ont été incluses dans la table des matières, après la liste des cinquante sections. Comme il fallait s'y attendre, pour certaines des notes longues, il s'est trouvé le besoin d'ajouter une ou plusieurs notes à la note. Celles-ci sont alors incluses à la suite de celle-ci, avec le même type de renvois, sauf des notes assez courtes, qui figurent alors sur la même page en "notes de bas de page", avec des renvois tels que ou.

J'ai eu grand plaisir à donner un nom à chacune des sections du texte, ainsi qu'à chacune des notes les plus substantielles - sans compter que par la suite, cela s'est avéré même indispensable pour m'y retrouver. Il va sans doute sans dire que ces noms ont été trouvés après-coup, alors qu'en commençant une section ou une note un peu longue je n'aurais su dire pour aucune quelle en serait la substance essentielle. Il en est de même à fortiori des noms (comme "Travail et découverte", etc ...) par lesquels j'ai désigné les huit parties I à VIII en lesquelles j'ai groupé après-coup les cinquante sections qui composent le texte.

Pour le contenu de ces huit parties, je me bornerai à de très brefs commentaires. Les deux premières I (Travail et découverte) et II (Le rêve et le Rêveur) contiennent des éléments d'une réflexion sur le travail mathématique, et sur le travail de découverte en général. Ma personne y est impliquée de façon beaucoup plus épisodique et beaucoup moins directe que dans les parties suivantes. Ce sont celles-ci surtout qui ont qualité de témoignage et de méditation. Les parties III à VI sont surtout une réflexion et un témoignage sur mon passé de mathématicien "dans le monde mathématique", entre 1948 et 1970. La motivation qui a animé cette méditation a été avant tout le désir de comprendre ce passé, dans un effort pour comprendre et assumer un présent dans certains aspects parfois décevants ou déroutants. Les parties VII (L'Enfant s'amuse) et VIII (L'aventure solitaire) concernent plutôt l'évolution de ma relation à la mathématique depuis 1970 jusqu'à aujourd'hui, c'est-à-dire depuis que j'ai quitté "le monde des mathématiciens" pour ne plus y retourner. J'y examine notamment les motivations, et les forces et circonstances, qui m'ont amené (à ma propre surprise) à reprendre une activité mathématique "publique" (en écrivant et faisant publier les Réflexions Mathématiques), après une interruption de plus de treize ans.



