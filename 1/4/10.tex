\subsection{10. Un acte de respect}

Cette réflexion qui a fini par prendre le nom "L'Enterrement" avait commencé comme un acte de respect. Un respect pour des choses que j'avais découvertes, que j'ai vues se condenser et prendre forme dans un néant, dont j'ai été le premier à connaître le goût et la vigueur et auxquelles j'ai donné un nom, pour exprimer et la connaissance que j'avais d'elles, et mon respect. À ces choses, j'ai donné du meilleur de moi-même. Elles se sont nourries de la force qui repose en moi, elles ont poussé et se sont épanouies, comme des branches multiples et vigoureuses jaillissant d'un même tronc vivant aux racines vigoureuses et multiples. Ce sont là choses vivantes et présentes, non des inventions qu'on peut faire ou ne pas faire - des choses étroitement solidaires dans une unité vivante qui est faite de chacune d'elles et qui donne à chacune sa place et son sens, une origine et une fin. Je les avais laissées il y a longtemps et sans aucune inquiétude ni regret, car je savais que ce que je laissais était sain et fort et n'avait nul besoin de moi pour croître et s'épanouir encore et se multiplier, suivant sa propre nature. Ce n'était pas un sac d'écus que je laissais, qu'on pouvait voler, ni un tas d'outils, qui pouvaient rouiller ou pourrir.

Pourtant, au fil des ans, alors que je me croyais bien loin d'un monde que j'avais laissé, me revenaient ici et là jusqu'à dans ma retraite comme des bouffées de dédain insidieux et de discrète dérision, désignant telles de ces choses que je connaissais fortes et belles, qui avaient leur place et leur fonction unique qu'aucune autre chose ne pourrait jamais remplir. Je les sentais comme des orphelines dans un monde hostile, un monde malade de la maladie du mépris, s'acharnant sur ce qui est sans armure. C'est dans ces dispositions qu'a commencé cette réflexion, comme un acte de respect vis-à-vis de ces choses et par là, vis-à-vis de moi-même - comme le rappel d'un lien profond entre ces choses et moi : celui qui se plaît à affecter un dédain vis-à-vis d'une de ces choses qui ont été nourries de mon amour, c'est moi qu'il se plaît à dédaigner, et tout ce qui est issu de moi.

Et il en est de même de celui qui, connaissant de première main ce lien qui me relie à telle chose qu'il a apprise par nul autre que moi, fait mine de tenir pour négligeable ou d'ignorer ce lien ou de revendiquer (fût-ce tacitement et par omission) pour son compte ou pour celui d'autrui une "paternité" factice. J'y vois bien clairement un acte de mépris pour une chose née de l'ouvrier comme pour l'obscur et délicat travail qui a permis à cette chose de naître, et pour l'ouvrier, et avant tout (d'une façon plus cachée et plus essentielle) pour lui-même.

Si mon "retour aux maths" ne devait servir qu'à me faire me rappeler de ce lien et à susciter en moi cet acte de respect devant tous - devant ceux qui affectent de dédaigner et devant les témoins indifférents - ce retour n'aura pas été inutile.

Il est vrai que j'avais vraiment perdu contact avec l'œuvre écrite et non écrite (ou du moins non publiée) que j'avais laissée. En commençant cette réflexion - je voyais les branches assez distinctement, sans trop me rappeler cependant qu'elles étaient partie d'un même arbre. Chose étrange, il a fallu que peu à peu se dévoile à mes yeux le tableau d'un saccage de ce que j'avais laissé, pour retrouver en moi le sens de l'unité vivante de ce qui était ainsi saccagé et dispersé. L'un a emporté des écus et l'autre un outil ou deux pour s'en prévaloir ou même pour s'en servir - mais l'unité qui fait la vie et la vraie force de ce que j'avais laissé, elle a échappé à chacun et à tous. J'en connais bien un pourtant qui a senti profondément cette unité et cette force, et qui au fond de lui-même la sent aujourd'hui encore, et qui se plaît à disperser la force qui est en lui à vouloir détruire cette unité qu'il a sentie en autrui à travers son œuvre. C'est dans cette unité vivante que réside la beauté et la vertu créatrice de l'œuvre. Nonobstant le saccage, je les retrouve intacts comme si je venais de les quitter sauf que j'ai mûri et les vois aujourd'hui avec des yeux neufs.

Si quelque chose pourtant est saccagé et mutilé, et désamorcé de sa force originelle, c'est en ceux qui oublient la force qui repose en eux-mêmes et qui s'imaginent saccager une chose à leur merci, alors qu'ils se coupent seulement de la vertu créatrice de ce qui est à leur disposition comme elle est à la disposition de tous, mais nullement à leur merci ni au pouvoir de personne.

Ainsi cette réflexion, et à travers elle ce "retour" inattendu, m'aura aussi fait reprendre contact avec une beauté oubliée. C'est d'avoir senti pleinement cette beauté qui donne tout son sens à cet acte de respect qui s'exprime maladroitement dans la note "Mes orphelins" \footnote{Cette note ( $\mathrm{n}^{\circ} 46$ ) est chronologiquement la première de toutes celles qui figurent dans l'Enterrement.}, et que je viens de réitérer en pleine connaissance de cause ici même.



