\subsection*{7. L'Ordonnancement des Obsèques}

Sous le nom "L'Enterrement", j'ai donc regroupé dans la table des matières l'imposant défilé des principales "notes" se rapportant à cette section d'anodine apparence "Le poids d'un passé" (s.50), donnant ainsi tout son sens au nom qui d'emblée s'était imposé à moi pour cette section ultime du "premier jet" de Récoltes et Semailles.

Dans cette longue procession de notes aux multiples parentés, celles qui s'y sont jointes au cours des quatre semaines écoulées (notes (51) à (97)) \footnote{Il faut y ajouter encore la note $\mathrm{n}^{\circ}$ 104, du 12 mai 1984. Les notes $\mathrm{n}^{\circ}$ 98 et suivantes (à l'exception de la note précédente $\mathrm{n}^{\circ}$ 104) constituent le "troisième souffle" de la réflexion, à partir du 22 septembre 1984. Elles sont également datées.} se distinguent comme les seules datées (du 19 avril au 24 mai) \footnote{Dans une suite de notes consécutives écrites le même jour, seule la première est datée. Les autres notes non datées sont les notes n°s 44' à 50 (formant les cortèges I, II, III). Les notes n°s 46, 47, 50 sont du 30 ou 31 mars, les notes n°s 44', 48, 48', 49 de la première quinzaine d'avril, enfin la note $\mathrm{n}^{\circ}$ 44'' est datée (du 10 mai).}. Il m'a paru le plus naturel de les donner dans l'ordre chronologique où elles se succèdent dans la réflexion \footnote{J'ai parfois fait une inversion de faible amplitude dans cet ordre chronologique, au bénéfice d'un ordre "dit logique", quand il m'a semblé que l'impression d'ensemble de la démarche de la réflexion n'en était pas faussée. Comme seules exceptions, je signale cependant onze notes (dont le numéro est précédé du signe !) issues de notes de b. de p. ultérieures à une note et qui ont pris des dimensions prohibitives, et que j'ai placées chacune à la suite de la note à laquelle elle se rapporte (sauf la note $\mathrm{n}^{\circ}$ 98, se rapportant au $\mathrm{n}^{\circ}$ 47).}, plutôt que dans quelque autre ordre dit "logique"; ou dans l'ordre d'apparition des références à ces notes dans des notes antérieures. Pour pouvoir retrouver ce dernier ordre (nullement linéaire) de filiation entre notes participantes, j'ai fait suivre (dans la table des matières) le numéro de chacune par celui de la note (parmi celles qui la précèdent) où il est fait d'abord référence à elle \footnote{Quand la référence à une note (telle (45)) se trouve dans la section "Le poids d'un passé" elle-même ; c'est le numéro (50) de cette dernière, placé entre parenthèses, qui est placé après celui de la note, comme dans 46 (50).}, ou (à défaut) par le numéro de celle dont elle constitue une continuation immédiate \footnote{Le numéro d'une note qui est continuation immédiate d'une note précédente (lesquels numéros se suivent alors) est précédé du signe * dans la table des matières. Ainsi *47, 46 indique que la note $\mathrm{n}^{\circ}$ 47 est une continuation immédiate de la note $\mathrm{n}^{\circ}$ 46 (qui n'est d'ailleurs pas ici celle qui la précède immédiatement, laquelle est la note $\mathrm{n}^{\circ}$ 46$_9$).}. (Cette dernière relation est indiquée dans le texte lui-même par un sigle de référence placé à la fin de la première note, tel $(\Rightarrow 47)$ placé à la fin de la dernière ligne de la note (46), qui réfère à la note (47) qui la continue.) Enfin, certaines précisions de nature tant soit peu technique à une note sont regroupées à la fin de celle-ci en des sous-notes numérotées par des indices consécutifs au numéro de la note primitive - comme dans les sous-notes $\left(46_{1}\right)$ à $\left(46_{9}\right)$ de la note (46) "Mes orphelins".

Pour structurer quelque peu l'ordonnancement d'ensemble de l'Enterrement et pour permettre de s'y reconnaître dans la multitude des notes qui s'y pressent, il m'a paru séant pour la circonstance d'inclure dans la procession quelques sous-titres gravement suggestifs, chacun précédant et menant un cortège long ou court de notes consécutives reliées par un thème commun.

J'ai eu ainsi le plaisir de voir s'assembler un à un, dans une longue procession solennelle venant honorer mes obsèques, dix \footnote{(29 Septembre) En fait, il y a finalement douze cortèges, en y incluant le Fourgon Funèbre (x), et "Le défunt (toujours pas décédé)" (XI) qui vient in extremis de se faufiler encore dans la procession...} cortèges - certains humbles, d'autres imposants, certains contrits et d'autres secrètement en liesse, comme il ne peut en être autrement en semblable occasion. Voici donc s'avancer : l'élève posthume (que tout un chacun se fait un devoir d'ignorer), les orphelins (fraîchement exhumés pour la circonstance), la Mode et ses Hommes illustres (j'ai bien mérité ça), les motifs (derniers nés et derniers exhumés de tous mes orphelins), mon ami Pierre menant modestement le plus important des cortèges, suivi de près par l'Accord Unanime des notes (silencieusement) concertantes et par le Colloque (dit "Pervers") au grand complet (se démarquant de l'élève posthume, alias l'Élève Inconnu, par cortèges funéraires interposés portant fleurs et couronnes) ; enfin, pour clore dignement l'imposant défilé, voici encore s'avancer l'Élève (nullement posthume et encore moins inconnu) alias le Patron, suivi de la troupe affairée de mes élèves (munis de force pelles et cordes) et enfin du Fourgon Funèbre (arborant quatre beaux cercueils de chêne solidement vissés, sans compter le Fossoyeur)... dix cortèges enfin au grand complet (il était temps), s'acheminant lentement vers la Cérémonie Funèbre.

Le clou de la Cérémonie est l'Éloge Funèbre, servi avec un doigté parfait par nul autre que mon ami Pierre en personne, présidant aux obsèques en réponse aux vœux de tous et à la satisfaction générale. La Cérémonie s'achève en un De Profundis final et définitif (du moins on l'espère), chanté comme une sincère action de grâces par le regretté défunt lui-même, qui à l'insu de tous a survécu à ses impressionnantes obsèques et mène en a pris de la graine, à sa satisfaction complète - laquelle satisfaction forme la note finale et l'ultime accord du mémorable Enterrement.

