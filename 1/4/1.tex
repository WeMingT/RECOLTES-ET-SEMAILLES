\section{(I) Le trèfle à cinq feuilles}

\subsection*{1. Rêve et accomplissement}

Il va y avoir trois ans au mois de juillet, j'ai fait un rêve peu ordinaire. Si je dis "peu ordinaire", c'est-là une impression qui est apparue après-coup seulement, en y repensant au réveil. Le rêve lui-même m'est venu comme la chose la plus naturelle, la plus évidente du monde, sans tambour ni trompette - au point même qu'au réveil, j'ai failli ne pas y faire attention, le pousser sans plus dans les oubliettes pour passer à "l'ordre du jour". Depuis la veille j'étais embarqué pour une réflexion sur ma relation à la mathématique. C'était la première fois de ma vie que je prenais la peine d'y aller voir - et encore, si je m'y suis mis à ce moment-là, c'était que vraiment j'y étais quasiment contraint et forcé ! Il y avait des choses si étranges, pour ne pas dire violentes, qui s'étaient passées dans les mois et dans les années précédentes, des sortes d'explosions de passion mathématique faisant irruption dans ma vie sans crier gare, qu'il n'était vraiment plus possible de continuer à ne pas regarder ce qui se passait.

Le rêve dont je parle ne comportait aucun scénario ni action d'aucune sorte. Il consistait en une seule image, immobile, mais en même temps très vivante. C'était la tête d'une personne, vue de profil. On la voyait regardant de droite à gauche. C'était un homme d'âge mûr, imberbe, chevelure folle faisant autour de la tête comme une auréole de force. L'impression surtout qui se dégageait de cette tête était celle d'une force juvénile, joyeuse, qui semblait jaillir de l'arc souple et vigoureux de la nuque (qu'on devinait plus qu'on ne le voyait). L'expression du visage était plus celle d'un garnement espiègle, ravi de quelque coup qu'il viendrait

