\section{(I) Le trèfle à cinq feuilles}

\subsection*{1. Rêve et accomplissement}

Il va y avoir trois ans au mois de juillet, j'ai fait un rêve peu ordinaire. Si je dis "peu ordinaire", c'est-là une impression qui est apparue après-coup seulement, en y repensant au réveil. Le rêve lui-même m'est venu comme la chose la plus naturelle, la plus évidente du monde, sans tambour ni trompette - au point même qu'au réveil, j'ai failli ne pas y faire attention, le pousser sans plus dans les oubliettes pour passer à "l'ordre du jour". Depuis la veille j'étais embarqué pour une réflexion sur ma relation à la mathématique. C'était la première fois de ma vie que je prenais la peine d'y aller voir - et encore, si je m'y suis mis à ce moment-là, c'était que vraiment j'y étais quasiment contraint et forcé ! Il y avait des choses si étranges, pour ne pas dire violentes, qui s'étaient passées dans les mois et dans les années précédentes, des sortes d'explosions de passion mathématique faisant irruption dans ma vie sans crier gare, qu'il n'était vraiment plus possible de continuer à ne pas regarder ce qui se passait.

Le rêve dont je parle ne comportait aucun scénario ni action d'aucune sorte. Il consistait en une seule image, immobile, mais en même temps très vivante. C'était la tête d'une personne, vue de profil. On la voyait regardant de droite à gauche. C'était un homme d'âge mûr, imberbe, chevelure folle faisant autour de la tête comme une auréole de force. L'impression surtout qui se dégageait de cette tête était celle d'une force juvénile, joyeuse, qui semblait jaillir de l'arc souple et vigoureux de la nuque (qu'on devinait plus qu'on ne le voyait). L'expression du visage était plus celle d'un garnement espiègle, ravi de quelque coup qu'il viendrait ou méditerait de faire, que celle de l'homme mûr, ou de celui qui aurait pris de l'assiette, mûr ou pas. Il s'en dégageait surtout une joie de vivre intense, contenue, fusant en jeu...

Il n'y avait pas une deuxième personne présente, un "je" qui aurait regardé ou contemplé cette autre, dont on ne voyait que la tête. Mais il y avait une perception intense de cette tête, de ce qui se dégageait d'elle. Il n'y avait personne non plus pour ressentir des impressions, les commenter, les nommer, ou pour coller un nom à la personne perçue, la désigner comme "un tel". Il n'y avait que cette chose très vivante, cette tête d'homme, et une perception également vivante, intense de cette chose.

Quant au réveil, sans propos délibéré, je me suis souvenu des rêves de la nuit écoulée, la vision de cette tête d'homme ne ressortait pas sur le nombre avec une intensité particulière, elle ne se poussait pas vers l'avant pour me crier ou me souffler : c'est moi qu'il te faut regarder! Quand ce rêve est apparu dans le champ de mon rapide regard sur les rêves de la nuit, dans la chaude quiétude du lit, j'ai eu bien sûr ce réflexe de l'esprit éveillé de mettre un nom sur ce qui avait été vu. Je n'avais pas d'ailleurs à chercher, il suffisait que je pose la question pour savoir aussitôt que cette tête d'homme qui avait été là dans ce rêve n'était autre que la mienne.

Elle est pas mal celle-là, j'ai pensé alors, il faut quand même le faire, se voir soi-même en rêve comme ça, comme si c'était un autre! Ce rêve venait là un peu comme si, en me promenant et par le plus grand des hasards, j'étais tombé sur un trèfle à quatre feuilles, ou même à cinq, pour m'en ébahir quelques instants comme il se doit, et poursuivre mon chemin comme si rien ne s'était passé.

C'est comme ça tout au moins que ça a failli se passer. Heureusement, comme il m'est arrivé bien des fois dans des situations de ce genre, j'ai quand même et par acquit de conscience noté noir sur blanc ce petit incident "pas mal", en commençant une réflexion qui était censée continuer sur la lancée de celle de la veille. Puis, de fil en aiguille, la réflexion de ce jour-là s'est bornée à me plonger dans le sens de ce rêve sans prétention, de cette image unique, et du message sur moi-même qu'il m'apportait.

Ce n'est pas le lieu ici de m'étendre sur ce que cette méditation d'un jour m'a enseigné et apporté. Ou plutôt, ce que ce rêve m'a enseigné et apporté, une fois que je m'étais mis dans les dispositions d'attention, d'écoute qui m'ont permis d'accueillir ce qu'il avait à me dire. Un premier fruit immédiat du rêve et de cette écoute a été un soudain afflux d'énergie nouvelle. Cette énergie a porté la méditation de longue haleine qui s'est poursuivie dans les mois suivants, à l'encontre de résistances intérieures opiniâtres, qu'il m'a fallu démonter une à une par un travail patient et obstiné.

Depuis cinq ans que je commençais à faire attention à certains des rêves qui me venaient, celui-ci était le premier "rêve messager" qui ne se présentait pas sous les apparences, reconnaissables désormais, d'un tel rêve, avec des moyens scéniques impressionnants et une intensité de vision exceptionnelle, parfois bouleversante. Celui-ci était tout ce qu'il y a de "cool", avec rien pour forcer l'attention, la discrétion même - c'était à prendre, ou à laisser, sans histoires.

Quelques semaines plus tôt m'était venu un rêve messager dans l'ancien style, sur le diapason dramatique et même sauvage, qui a mis une fin soudaine et immédiate à une longue période de frénésie mathématique. La seule parente apparente entre les deux rêves, c'est que dans l'un ni dans l'autre il n'y avait d'observateur. Par une parabole d'une force lapidaire, ce rêve montrait quelque chose qui se passait alors dans ma vie, sans que je prenne la peine d'y accorder attention - une chose que je prenais même grand soin d'ignorer, pour tout dire. C'est ce rêve qui m'a fait comprendre alors l'urgence d'un travail de réflexion, dans lequel je me suis engagé quelques semaines plus tard, et qui s'est alors poursuivi sur près de six mois. J'ai occasion d'en parler tant soit peu dans la dernière partie de cette réflexion-témoignage "Récoltes et Semailles", qui ouvre le présent volume et lui donne son nom \footnote{Voir notamment section 43, "Le patron trouble-fête - ou la marmite à pression"}.

Si j'ai commencé cette introduction par l'évocation de cet autre rêve, de cette image-vision de moi-même ("Traumgesicht meiner selbst" comme je l'ai appelé dans mes notes en allemand), c'est parce que dans ces dernières semaines la pensée de ce rêve m'est revenue plus d'une fois, pendant que la méditation "sur un passé de mathématicien" s'acheminait vers sa fin. A vrai dire, rétrospectivement, les trois années qui se sont écoulées depuis ce rêve m'apparaissent comme des années de décantation et de maturation, vers un accomplissement de son message simple et limpide. Le rêve me montrait "tel que je suis". Il était clair également que dans ma vie éveillée je n'étais pas pleinement celui que le rêve me montrait - des poids et des raideurs venant de loin faisaient (et font encore) obstacle bien souvent à ce que je sois pleinement et simplement moi-même. Pendant ces années, alors que la pensée de ce rêve ne me revenait que rarement pourtant, ce rêve a dû agir d'une certaine façon. Ce n'était nullement comme une sorte de modèle ou d'idéal auquel je me serais efforcé de ressembler, mais comme le rappel discret d'une simplicité joyeuse qui "était moi", qui se manifestait de bien des façons, et qui était appelée à se libérer de ce qui continuait à peser sur elle et à s'épanouir pleinement. Ce rêve était un lien délicat et vigoureux à la fois, entre un présent lesté encore par bien des poids provenant du passé, et un "demain" tout proche que ce présent contient en germe, un "demain" qui est moi dès à présent, et qui est en moi depuis toujours sûrement...

Sûrement, si en ces dernières semaines ce rêve rarement évoqué a été à nouveau bien présent, c'est qu'à un certain niveau qui n'est pas celui d'une pensée qui sonde et analyse, j'ai dû "savoir" que le travail que j'étais en train de faire et de mener à sa fin, travail qui reprenait et approfondissait cet autre travail d'il y a trois ans, était un nouveau pas vers l'accomplissement du message sur moi-même qu'il m'apportait.

Tel est à présent pour moi le sens principal de Récoltes et Semailles, de ce travail intense de près de deux mois. Maintenant seulement qu'il est achevé, je me rends compte à quel point il était important que je le fasse. Au cours de ce travail, j'ai connu beaucoup de moments de joie, d'une joie souvent malicieuse, blagueuse, exubérante. Et il y a eu également des moments de tristesse, et des moments où je revivais des frustrations ou des peines qui m'avaient touché douloureusement en ces dernières années - mais il n'y a pas eu un seul moment d'amertume. Je quitte ce travail avec la satisfaction complète de celui qui sait qu'il a mené un travail à son terme. Il n'y a chose si "petite" soit elle que j'y aie éludée, ou qu'il m'aurait tenu à cœur de dire et que je n'aurais pas dite, et qui en cet instant laisserait en moi le résidu d'une insatisfaction, d'un regret, si "petits" soient-ils.

En écrivant ce témoignage, il était clair pour moi qu'il ne plaira pas à tout le monde. Il est même bien possible que j'ai trouvé moyen de mécontenter tout le monde sans exception. Ce n'était pourtant nullement mon propos, ni même de mécontenter quiconque. Mon propos était simplement de regarder les choses simples et importantes, les choses de tous les jours, de mon passé (et parfois de mon présent aussi) de mathématicien, pour découvrir enfin (mieux vaut tard que jamais !) et sans l'ombre d'un doute ou d'une réserve, ce qu'elles étaient et ce qu'elles sont ; et, chemin faisant, dire en des mots simples ce que je voyais.

