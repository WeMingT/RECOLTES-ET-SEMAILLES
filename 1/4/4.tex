\subsection{4. Un voyage à la poursuite des choses évidentes}

Il me faudrait encore dire quelques mots plus circonstanciés sur ce voyage entrepris depuis un peu plus d'un an, les Réflexions Mathématiques. Je m'explique de façon assez détaillée, dans les huit premières sections de Récoltes et Semailles (i.e. dans les parties I et II de la réflexion), au sujet de l'esprit dans lequel j'entreprends ce voyage, et qui, je pense, est apparent dès à présent dans le présent premier volume, comme aussi dans celui qui lui fait suite (l'Histoire de Modèles, qui est le volume 1 de la Poursuite des Champs), en cours d'achèvement. Il me semble donc inutile de m'étendre à ce sujet dans cette introduction.

Je ne puis certes prédire ce que sera le voyage entrepris, chose que je découvrirai au fur et à mesure qu'il se poursuivra. Je n'ai pas à présent un itinéraire prévu même dans les grandes lignes, et je doute qu'il s'en dégagera un prochainement. Comme je l'ai dit précédemment, les thèmes principaux qui vont sans doute inspirer ma réflexion sont esquissés peu ou prou dans l'Esquisse d'un Programme, le "texte-boussole". Parmi ces thèmes, il y a aussi le thème principal de la Poursuite des Champs, c'est-à-dire les "champs", dont j'espère bien faire le tour (et m'en tenir là) au cours de cette année encore, en deux ou peut-être trois volumes. Au sujet de ce thème j'écris dans l'Esquisse : "... c'est un peu comme une dette dont je m'acquitterais vis-à-vis d'un passé scientifique où pendant une quinzaine d'années (entre 1955 et 1970), le développement d'outils cohomologiques a été le Leitmotiv constant dans mon travail de fondements de la géométrie algébrique". C'est donc là, parmi les thèmes prévus, celui qui s'enracine le plus fortement dans mon "passé" scientifique. C'est celui aussi qui est resté présent comme un regret tout au long de ces quinze années écoulées, comme la lacune la plus flagrante de toutes peut-être du travail que j'avais laissé à faire lors de mon départ de la scène mathématique, et qu'aucun de mes élèves ou amis d'antan ne s'est soucié de combler. Pour plus de détails sur ce travail en cours, le lecteur intéressé pourra se reporter à la section pertinente dans l'Esquisse d'un Programme, ou à l'introduction (la vraie cette fois !) du premier volume, en cours d'achèvement, de la Poursuite des Champs.

Comme autre legs de mon passé scientifique qui me tient particulièrement à cœur, il y a surtout la notion de motif, qui attend toujours de sortir de la nuit où elle est restée maintenue, depuis une bonne quinzaine d'années pourtant qu'elle a fait son apparition. Il n'est pas exclu que je finisse par me mettre au travail de fondements qui s'impose ici, si personne de mieux placé que moi (par un âge plus jeune, aussi bien que par les outils et connaissances dont il dispose) ne se décide à le faire dans les toutes prochaines années.

Je prends cette occasion pour signaler que la fortune (ou plutôt, l'infortune...) de la notion de motif, et de quelques autres parmi celles que j'ai tirées au jour et qui entre toutes me paraissent (en puissance) les plus fécondes, font l'objet d'une réflexion rétrospective de près d'une vingtaine de pages, formant la plus longue (et une des toutes dernières) des "notes" à Récoltes et Semailles \footnote{Cette double note ( $\mathrm{n}^{\circ} \mathrm{s} 46,47$ ) et ses sous-notes ont été incluses dans la deuxième partie "L'Enterrement" de Récoltes et Semailles, qui en constitue une continuation directe.}. J'ai après-coup subdivisé cette note en deux parties ("Mes orphelins" et "Refus d'un héritage - ou le prix d'une contradiction"), en plus des trois "sous-notes" qui la suivent \footnote{Il s'agit des sous-notes $\mathrm{n}^{\circ} \mathrm{s} 48,49,50$ (la note $\mathrm{n}^{\circ} 48^{\prime}$ a été rajoutée ultérieurement).}. L'ensemble de ces cinq notes consécutives est la seule partie de Récoltes et Semailles où sont évoquées des notions mathématiques autrement que par allusions en passant. Ces notions deviennent l'occasion pour illustrer certaines contradictions à l'intérieur du monde des mathématiciens, qui elles-mêmes reflètent des contradictions en les personnes elles-mêmes. J'ai songé à un moment à séparer cette note tentaculaire du texte dont elle provient, pour la joindre à l'Esquisse thématique. Cela aurait eu l'avantage de mettre celle-ci en perspective, et d'insuffler un peu de vie à un texte qui ressemble un peu trop à un catalogue. Je me suis pourtant abstenu de le faire, dans un souci de préserver l'authenticité d'un témoignage dont cette mégnote, que cela me plaise ou non, fait bel et bien partie.

A ce qui est dit dans Récoltes et Semailles sur les dispositions dans lesquelles j'aborde les "Réflexions", je voudrais ajouter ici une seule chose, sur laquelle je me suis exprimé déjà dans une des notes ("Le snobisme des jeunes - ou les défenseurs de la pureté"), quand j'écris : "Mon ambition de mathématicien ma vie durant, ou plutôt ma joie et ma passion, ont été constamment de découvrir les choses évidentes, et c'est ma seule ambition aussi dans le présent ouvrage" (A la Poursuite des Champs). C'est là ma seule ambition également pour ce nouveau voyage que je poursuis depuis un an avec les Réflexions. Il n'en a pas été autrement dans ces Récoltes et Semailles qui (pour mes lecteurs du moins, s'il s'en trouve) ouvrent ce voyage.


