\subsection{8. La fin d'un secret}

Au cours de cette étape ultime (on l'espère) de la réflexion m'est apparu l'intérêt de joindre en "Appendice" au présent volume 1 des Réflexions Mathématiques deux autres textes, de nature mathématique, en plus des trois dont il a été question précédemment \footnote{De plus, je pense adjoindre à l'Esquisse Thématique (voir "Boussole et bagages", Introduction, 3) un "commentaire" donnant quelques précisions au sujet de mes contributions aux "thèmes" qui y sont passés en revue sommairement, et au sujet aussi des influences qui ont joué dans la genèse des principales idées-force dans mon œuvre mathématique. La rétrospective des dernières six semaines a fait déjà apparaître (à ma propre surprise) un rôle de "détonateur" de Serre, pour le démarrage de la plupart de ces idées, comme aussi pour certaines des "grandes tâches" que je m'étais posées, entre 1955 et 1970.

Enfin, comme autre texte de nature mathématique (au sens courant), et le seul qui figure (incidemment) dans le texte non technique "Récoltes et Semailles", le signale la sous-note $\mathrm{n}^{\circ} 87$ à la note "Le massacre" ( $\mathrm{n}^{\circ} 87$ ), où j'explicite avec le soin qu'elle mérite une variante "discrète" (conjecturale) du théorème de Riemann-Roch-Grothendieck familier dans le contexte cohérent. Cette conjecture figure (parmi un nombre d'autres) dans l'exposé de clôture du séminaire SGA 5 de 1965/66, exposé dont il ne reste trace (pas plus que de nombreux autres) dans le volume publié onze ans plus tard sous le nom SGA 5. Les vicissitudes de ce séminaire crucial aux mains de certains de mes élèves, et les liens de celles-ci avec une certaine "opération SGA $4 \frac{1}{2}$ ", se révèlent progressivement au cours de la réflexion poursuivie dans les notes $\mathrm{n}^{\circ} \mathrm{s} 63^{\prime \prime}, 67,67^{\prime}, 68,68^{\prime}, 84,85,85^{\prime}$, $86,87,88$.

Comme autre note donnant des commentaires mathématiques assez étoffés, sur l'opportunité de dégager un cadre "topossique" commun (dans la mesure du possible) pour les cas connus où on dispose d'un formalisme de dualité dit "des six opérations", je signale aussi la sous-note $\mathrm{n}^{\circ} 81_{2}$ à la note "Thèse à crédit et assurance tous risques", $\mathrm{n}^{\circ} 81$.}.

Cette esquisse d'un formulaire cohérent sera pour moi le premier pas évident vers ce "vaste tableau d'ensemble du rêve des motifs", qui depuis plus de quinze ans "attend le mathématicien hardi qui voudra bien le brosser". Selon toute apparence, ce mathématicien ne sera autre que moi-même. Il est grand temps en effet que ce qui était né et confié dans l'intimité il y a près de vingt ans, non pour rester le privilège d'un seul mais pour être à la disposition de tous, sorte enfin de la nuit du secret, et naisse une nouvelle fois à la pleine lumière du jour.

Il est bien vrai qu'un seul, à part moi, avait une connaissance intime de ce "yoga des motifs", pour l'avoir appris de ma bouche au fil des jours et des années qui ont précédé mon départ. Parmi toutes les choses mathématiques que j'avais eu le privilège de découvrir et d'amener au jour, cette réalité des motifs m'apparaît encore comme la plus fascinante, la plus chargée de mystère - au cœur même de l'identité profonde entre "la géométrie" et "l'arithmétique". Et le "yoga des motifs" auquel m'a conduit cette réalité longtemps ignorée est peut-être le plus puissant instrument de découverte que j'aie dégagé dans cette première période de ma vie de mathématicien.

Mais il est vrai aussi que cette réalité, et ce "yoga" qui s'efforce de la cerner au plus près, n'avaient nullement été tenus secrets par moi. Absorbé par des tâches impératives de rédaction de fondements (que tout le monde depuis est bien content de pouvoir utiliser tels quels dans son travail de tous les jours), je n'ai pas pris les quelques mois nécessaires pour rédiger une vaste esquisse d'ensemble de ce yoga des motifs, et le mettre ainsi à la disposition de tous. Je n'ai pas manqué pourtant, dans les années précédant mon départ inopiné, d'en parler au hasard des rencontres et à qui voulait l'entendre, en commençant par mes élèves, qui (à part l'un d'entre eux) l'ont oublié comme tous l'ont oublié. Si j'en ai parlé, ce n'était pas pour placer des "inventions" qui porteraient mon nom, mais pour attirer l'attention sur une réalité qui se manifeste à chaque pas, dès qu'on s'intéresse à la cohomologie des variétés algébriques et notamment, à leurs propriétés "arithmétiques" et aux relations entre elles des différentes théories cohomologiques connues à ce jour. Cette réalité est aussi tangible que l'était jadis celle des "infiniments petits", perçue longtemps avant l'apparition du langage rigoureux qui permettait de l'appréhender de façon parfaite et de "l'établir". Et pour appréhender la réalité des motifs, nous ne sommes aujourd'hui nullement à court d'un langage souple et adéquat, ni d'une expérience consommée dans l'édification de théories mathématiques, qui manquaient à nos prédécesseurs.

Si ce que j'ai naguère crié sur les toits est tombé en des oreilles sourdes, et si le mutisme dédaigneux de l'un a recueilli en écho le silence et la léthargie de tous ceux qui font mine de s'intéresser à la cohomologie (et qui ont pourtant des yeux et des mains tout comme moi...), je ne puis en tenir pour responsable celui-là seul qui a choisi de garder par-devers lui le "bénéfice" de ce que je lui avais confié à l'intention de tous. Force est de constater que notre époque, dont la productivité scientifique effrénée rivalise avec celle investie dans les armements ou dans les biens de consommation, est très loin de ce "dynamisme hardi" de nos prédécesseurs du dix-septième siècle, qui "n'y sont pas allés par quatre chemins" pour développer un calcul des infiniments petits, sans se laisser arrêter par le souci si ce calcul était "conjectural" ou non; ni attendre non plus que tel homme prestigieux parmi eux daigne leur donner le feu vert, pour empoigner ce que chacun voyait bien de ses propres yeux et sentait de première main.

