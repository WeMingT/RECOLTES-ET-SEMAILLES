\section{(II) Un acte de respect}

(- 4 mai - ... juin)

\subsection{6. L'Enterrement}

Un événement imprévu a relancé une réflexion qui était menée à terme. Il a inauguré une cascade de découvertes grandes et petites au cours des semaines écoulées, dévoilant progressivement une situation qui était restée floue et en avivant les contours. Cela m'a conduit notamment à entrer de façon circonstanciée et approfondie dans des événements et situations dont il n'avait été question précédemment qu'en passant ou par allusion. Du coup la "réflexion rétrospective d'une quinzaine de pages" sur les vicissitudes d'une œuvre, dont il a été question précédemment (Introduction, 4), a pris des dimensions inattendues, s'augmentant de quelques deux cents pages supplémentaires.

Par la force des choses et par la logique intérieure d'une réflexion, j'ai été amené en chemin à impliquer autrui autant que moi-même. Celui qui est impliqué plus que tout autre (à part moi-même) est un homme auquel me lie une amitié de près de vingt ans. J'ai écrit de lui (par euphémisme \footnote{Sur le sens de cet "euphémisme"; voir la note "L'être à part", $\mathrm{n}^{\circ} 57$ '.}) qu'il avait "fait un peu figure d'élève", en les toutes premières années de cette amitié affectueuse enracinée dans une passion commune, et pendant longtemps et en mon for intérieur je voyais en lui une sorte d' "héritier légitime" de ce que je croyais pouvoir apporter en mathématique, au-delà d'une œuvre publiée restée fragmentaire. Nombreux seront ceux qui déjà l'auront reconnu : c'est Pierre Deligne.

Je ne m'excuse pas de rendre publique avec ces notes, entre autres, une réflexion personnelle sur une relation personnelle, et de l'impliquer ainsi sans l'avoir consulté. Il me paraît important, et sain pour tous, qu'une situation restée longtemps occulte et confuse soit enfin portée au grand jour et examinée. Ce faisant, j'apporte un témoignage, subjectif certes et qui ne prétend ni épuiser une situation délicate et complexe, ni être exempt d'erreurs. Son premier mérite (comme celui de mes publications passées, ou de celles sur lesquelles je travaille à présent) est d'exister, à la disposition de ceux qu'il peut intéresser. Mon souci n'a été ni de convaincre, ni de me mettre à l'abri de l'erreur ou du doute derrière les seules choses dites "patentes". Mon souci est d'être vrai, en disant les choses telles que je les vois ou les sens, en chaque instant - comme un moyen pour les approfondir et pour comprendre.

Le nom "L'Enterrement", pour l'ensemble de toutes les notes se rapportant au "Poids d'un passé", s'est imposé avec une force croissante au cours de la réflexion \footnote{Vers la fin de cette réflexion, un autre nom s'est présenté, exprimant un autre aspect saisissant d'un certain tableau qui s'était progressivement dévoilé à mes yeux au cours des cinq semaines écoulées. C'est le nom d'un conte, sur lequel je vais revenir en son lieu : "La robe de l'Empereur de Chine"...}. J'y joue le rôle du défunt anticipé, en la funèbre compagnie des quelques mathématiciens (beaucoup plus jeunes) dont l'œuvre se place après mon "départ" en 1970 et porte la marque de mon influence, par un certain style et par une certaine approche de la mathématique. Au premier rang de ceux-ci se trouve mon ami Zoghman Mebkhout, qui a eu ce lourd privilège d'avoir à affronter tous les handicaps de celui traité en "élève de Grothendieck après 1970", sans avoir eu pour autant l'avantage d'un contact avec moi et de mon encouragement et de mes conseils, alors qu'il n'a été "élève" que de mon œuvre à travers mes écrits. C'était à l'époque où (dans le monde qu'il hante) je faisais déjà figure de "défunt" au point que pendant longtemps l'idée même d'une rencontre ne s'est apparemment pas présentée, et qu'une relation suivie (tant personnelle que mathématique) n'a fini par se nouer que l'an dernier.

\footnote{Vers la fin de cette réflexion, un autre nom s’est présenté, exprimant un autre aspect saisissant d’un certain tableau qui s’était progressivement dévoilé à mes yeux au cours des cinq semaines écoulées. C’est le nom d’un conte, sur lequel je vais revenir en son lieu : "La robe de l’Empereur de Chine". . .}

Cela n'a pas empêché Mebkhout, à contre-courant d'une mode tyrannique et du dédain de ses aînés (qui furent mes élèves) et dans un isolement quasi-complet, de faire œuvre originale et profonde, par une synthèse imprévue des idées de l'école de Sato et des miennes. Cette œuvre fournit une prise nouvelle sur la cohomologie des variétés analytiques et algébriques, et porte la promesse d'un renouvellement de grande envergure dans notre compréhension de cette cohomologie. Nul doute que ce renouvellement serait chose accomplie dès à présent et depuis des années, si Mebkhout avait trouvé auprès de ceux tout désignés pour cela l'accueil chaleureux et le soutien sans réserve qu'ils avaient naguère reçus auprès de moi. Du moins, depuis octobre 1980 ses idées et travaux ont fourni l'inspiration et les moyens techniques d'un redémarrage spectaculaire de la théorie cohomologique des variétés algébriques, sortant enfin (mis à part les résultats de Deligne autour des conjectures de Weil) d'une longue période de stagnation.

Chose incroyable et pourtant vraie, ses idées et résultats sont depuis près de quatre ans utilisés par "tous" (au même titre que les miens), alors que son nom reste soigneusement ignoré et tu par ceux-là même qui connaissent son œuvre de première main et l'utilisent de façon essentielle dans leurs travaux. J'ignore si à aucune autre époque la mathématique a connu une telle disgrâce, quand certains des plus influents ou des plus prestigieux parmi ses adeptes donnent l'exemple, dans l'indifférence générale, du mépris de la règle la plus universellement admise dans l'éthique du métier de mathématicien.

Je vois quatre hommes, mathématiciens aux moyens brillants, qui ont eu et qui ont droit avec moi aux honneurs de cet enterrement par le silence et par le dédain. Et je vois en chacun la morsure du mépris sur la belle passion qui l'avait animé.

A part ceux-là, je vois surtout deux hommes, placés l'un et l'autre sous les feux de la rampe sur la place publique mathématique, qui officient aux obsèques en nombreuse compagnie et qui en même temps (dans un sens plus caché) sont enterrés et de leurs propres mains, en même temps que ceux qu'ils enterrent de propos délibéré. J'ai déjà nommé l'un d'eux. L'autre est également un ancien élève et un ancien ami, Jean-Louis Verdier. Après mon "départ" de 1970, le contact entre lui et moi ne s'est pas maintenu, à part quelques rencontres hâtives au niveau professionnel. C'est pourquoi sans doute il ne figure dans cette réflexion qu'à travers certains actes de sa vie professionnelle, alors que les motivations éventuelles de ces actes, au niveau de sa relation à moi, ne sont pas examinées et m'échappent d'ailleurs entièrement.

S'il est une interrogation pressante qui s'est imposée à moi tout au long des années écoulées, qui a été une motivation profonde de Récoltes et Semailles et qui m'a suivie aussi tout au long de cette réflexion, c'est celle de la part qui me revient dans l'avènement d'un certain esprit et de certaines mœurs qui rendent possible des disgrâces comme celle que j'ai dite, dans un monde qui fût le mien et auquel je m'étais identifié pendant plus de vingt ans de ma vie de mathématicien. La réflexion m'a fait découvrir que par certaines attitudes de fatuité en moi, s'exprimant par un dédain tacite des collègues aux moyens modestes, et par une complaisance à moi-même et à tels mathématiciens pourvus de moyens brillants, je n'ai pas été étranger à cet esprit que je vois s'étaler aujourd'hui parmi ceux-là même que j'avais aimés, et parmi ceux-là aussi auxquels j'ai enseigné un métier que j'aimais ; ceux que j'ai mal aimés et mal enseignés et qui aujourd'hui donnent le ton (quand ils ne font la loi) dans ce monde qui m'était cher et que j'ai quitté.

Je sens souffler un vent de suffisance, de cynisme et de mépris. "Il souffle sans se soucier de "mérite" ni de "démérite", brûlant de son haleine les humbles vocations comme les plus belles passions...". J'ai compris que ce vent-là est la prolifique récolte de semailles aveugles et insouciantes que j'ai contribué à semer. Et si son souffle revient sur moi et sur ce que j'avais confié à d'autres mains, et sur ceux que j'aime aujourd'hui et qui ont osé se réclamer ou seulement s'inspirer de moi, c'est là un retour des choses dont je n'ai pas lieu de me plaindre, et qui a beaucoup à m'enseigner.

