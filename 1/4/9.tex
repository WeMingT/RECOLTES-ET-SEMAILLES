\subsection{9. La scène et les Acteurs}

Par sa propre structure interne et par son thème particulier, "L'Enterrement" (qui forme maintenant plus de la moitié du texte de Récoltes et Semailles) est dans une large mesure et au point de vue logique indépendant de la longue réflexion qui le précède. C'est là pourtant une indépendance toute superficielle. Pour moi cette réflexion, autour d'un "enterrement" sortant progressivement des brumes du non-dit et du pressenti, est inséparable de celle qui l'avait précédée, dont elle est issue et qui lui donne tout son sens. Commencée comme un rapide coup d'œil "en passant" sur les vicissitudes d'une œuvre que j'avais un peu (beaucoup) perdue de vue, elle est devenue, sans l'avoir prévu ni cherché, une méditation sur une relation importante dans ma vie, me conduisant à son tour à une réflexion sur le sort de cette œuvre aux mains de "ceux qui furent mes élèves". Séparer cette réflexion de celle dont elle est spontanément issue me paraît une façon de la réduire à un simple "tableau de mœurs" (voire même, à un règlement de comptes dans le "beau monde" mathématique).

Il est vrai que si on y tient, la même réduction à un "tableau de mœurs" peut être faite pour Récoltes et Semailles tout entier. Certes, les mœurs qui prévalent à une époque et dans un milieu donnés et qui contribuent à façonner la vie des hommes qui en font partie, ont leur importance et méritent d'être décrites. Il sera clair pourtant pour un lecteur attentif de Récoltes et Semailles que mon propos n'est pas de décrire des mœurs, c'est-à-dire une certaine scène, changeant avec le temps et d'un lieu à l'autre, sur laquelle se déroulent nos actions. Cette scène dans une large mesure définit et délimite les moyens à la disposition de diverses forces en nous, leur permettant de s'exprimer. Alors que la scène et ces moyens qu'elle fournit (et les "règles du jeu" qu'elle impose) varient à l'infini, la nature des forces profondes en nous qui (au niveau collectif) façonnent les scènes et qui (au niveau de la personne) s'expriment sur elles, semble bien être la même d'un milieu ou d'une culture à l'autre, et d'une époque à l'autre. S'il est une chose dans ma vie, hors la mathématique et hors l'amour de la femme, dont j'aie senti le mystère et l'attirance (sur le tard, il est vrai), c'est bien la nature cachée de ces quelques forces qui ont pouvoir de nous faire agir, pour le "meilleur" comme pour le "pire", pour enfouir et pour créer.

