\subsection{3. Boussole et bagages}

Il me faudrait dire quelques mots au sujet des deux autres textes qui constituent avec Récoltes et Semailles le présent volume de même nom.

L' "Esquisse d'un Programme" donne une esquisse des principaux thèmes de réflexion mathématique que j'ai poursuivis au cours des dix dernières années. Je compte tout au moins en développer tant soit peu quelques-uns dans les années qui viennent, dans une série de réflexions informelles dont j'ai eu occasion déjà de parler, les "Réflexions Mathématiques". Cette esquisse est la reproduction textuelle d'un rapport que j'ai écrit en janvier dernier pour appuyer ma candidature à un poste de chercheur au CNRS. Je l'ai inclus dans le présent volume, parce que visiblement ce programme dépasse de loin les possibilités de ma modeste personne, même s'il m'était donné de vivre encore cent ans et que je choisisse de les employer à poursuivre aussi loin que je peux les thèmes en question.

L' "Esquisse thématique" a été écrite en 1972 à l'occasion d'une autre candidature (à un poste de professeur au Collège de France). Elle contient une esquisse, par thèmes, de ce que je considérais alors comme mes principales contributions mathématiques. Ce texte se ressent des dispositions dans lesquelles il a été écrit, à un moment où mon intérêt pour la mathématique était tout ce qu'il y a de marginal, à dire le moins. Aussi cette esquisse n'est-elle guère mieux qu'une énumération sèche et méthodique (mais qui fort heureusement ne vise pas à être exhaustive...). Elle ne paraît pas portée par une vision ou par le souffle d'un désir - comme si ces choses que j'y passe en revue comme par acquit de conscience (et c'étaient bien là en effet mes dispositions) n'avaient jamais été effleurées par une vision vivante, ni par une passion de les tirer au jour alors qu'elles n'étaient encore que pressenties derrière leurs voiles de brume et d'ombre...

Si pourtant je me suis décidé à inclure ici ce rapport peu inspirant, je crains, c'est surtout pour clore le bec (à supposer que ce soit là chose possible) à certains collègues de haut vol et à une certaine mode, qui depuis mon départ d'un monde qui nous fut commun affectent de regarder de haut ce qu'ils appellent aimablement des "grothendieckeries". C'est là, paraît-il, synonyme de bombinage sur des choses trop triviales pour qu'un mathématicien sérieux et de bon goût consente à perdre sur elles un temps certes précieux. Peut-être ce "digest" indigeste leur paraîtra-t-il plus "sérieux"! Quant aux textes de ma plume qu'une vision et une passion anime, ils ne sont pas pour ceux qu'une mode maintient et justifie dans une suffisance, les rendant insensibles aux choses qui m'enchantent. Si j'écris pour d'autres que pour moi-même, c'est pour ceux qui ne trouvent pas leur temps et leur personne trop précieux pour poursuivre sans jamais se lasser les choses évidentes que personne ne daigne voir, et pour se réjouir de l'intime beauté de chacune des choses découvertes, la distinguant de toute autre qui nous était connue dans sa propre beauté.

Si je voulais situer les uns par rapport aux autres les trois textes qui constituent le présent volume, et le rôle de chacun dans ce voyage dans lequel me voilà embarqué avec les Réflexions Mathématiques, je pourrais dire que la réflexion-témoignage Récoltes et Semailles reflète et décrit l'esprit dans lequel j'entreprends ce voyage et qui lui donne son sens. L'Esquisse d'un Programme décrit mes sources d'inspiration, qui fixent une direction sinon certes une destination pour ce voyage dans l'inconnu, à la manière un peu d'une boussole, ou d'un vigoureux fil d'Ariane. L'Esquisse thématique enfin passe en revue rapidement un bagage, acquis dans mon passé de mathématicien d'avant 1970, dont une partie au moins sera utile et la bienvenue dans telle ou telle étape du voyage (comme mes réflexes d'algèbre cohomologique et topossique me sont indispensables dès maintenant dans la Poursuite des Champs). Et l'ordre dans lequel ces trois textes se suivent, comme aussi leurs longueurs respectives, reflètent bien (sans propos délibéré de ma part) l'importance et le poids que je leur accorde dans ce voyage, dont la première étape approche de sa fin.


