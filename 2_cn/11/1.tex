\section{(46) 禁果}

我不得不在笔记中中断了两天。经过仔细重读,我觉得前述的情节大体上确实是对现实的描述,现在需要对此进行更深入的挖掘。特别是,我需要更仔细地审视冥想与数学这两匹“马”各自的优点;同时,也要试图理解是哪些事件或情境最终引发了“老板”投注的“翻转”,违背了那种惯性力量——这种力量本会推动他无限期地维持一个即使是输局的投注。

或许还应该探究“同一个”的偏好。现在这已是共识,他时不时想要换个游戏玩,而老板显然具备最起码的灵活性,不会不惜一切代价强迫他总是玩这个而从不玩那个。近几年来,他学会了考虑“那个小家伙”的感受,与他妥协,而不是等到压力锅爆炸。虽非完全的和谐,但也不再是战争,更像是一种友好的默契,偶尔的紧张反而有助于缓和关系,而非使其僵化。

只要不被过于严厉地反对,“小家伙”在偏好上的本性是相当灵活的。(这不像老板,只是到了晚年才迫不得已学会了一点灵活性……)但“同一个”灵活并不意味着他没有自己的偏好,也不意味着他不会对某件事物比另一件更感兴趣。

要弄清楚这一点往往并不容易,要区分“同一个”的愿望与老板的偏好,甚至是老板一劳永逸的决定,绝非显而易见。当我过去对自己说:冥想更好、更重要、更严肃等等,比数学优越,理由是这样那样的(当然是最有说服力的理由),其实是老板在事后找理由说服自己,他所做的投注确实是“正确的”。“同一个”并不会说某件事物“更好”或“更重要”。他不擅长发表言论。当他想做什么时,只要没人阻止,他就去做,不会去问这件事是否“重要”或“更好”。他的欲望在不同事物之间、不同时刻之间强弱不一。要察觉他的偏好,听老板那些自称代表“小家伙”的解释性话语毫无用处,因为老板只能代表他自己说话。只有观察“小家伙”在游戏中的表现,或许才能窥见他的倾向。即使如此,也不总是显而易见的:当他兴致勃勃地玩这个时,并不一定意味着如果老板不插手,他不会同样兴高采烈地玩别的。

显然,最吸引他的,首先是未知——是深入夜的朦胧褶边,将那些对他、对所有人皆未知的事物带入光明。我感觉,当我加上“对所有人”时,这确实是“孩子”的愿望,而非老板的虚荣心,想要炫耀给他人或自己看。还有一点也是共识,“同一个”每次从无尽阁楼与地窖的幽暗中带回的东西,都是“显而易见”的、孩子气的东西。越是显得显而易见,他越是满足。如果不显而易见,那就是他还没干完活,半途停在了黑暗与光明之间。

在数学中,“显而易见”的事物,也是那些迟早有人会发现的。它们不是可以选择发明或不发明的“创造”。它们早已存在,永恒不变,所有人都在它们身边擦肩而过却视而不见,要么绕个大圈,要么每次经过时都跌跌撞撞。一年后或千年后,总有人不可避免地注意到它,围绕它挖掘,将它挖出,从各个角度审视,擦拭干净,最后为之命名。这类工作,是我最喜欢的工作,每次都可能由他人完成,而且迟早必然有人完成 \footnote{毋庸赘言,我这里忽略了一种绝非不可能的假设,即一场突如其来的核战争或其他类似“乐事”的爆发,可能突然且一劳永逸地终结这场名为“数学”的集体游戏,以及更多其他事物……} (44)。

而在“我”的发现中,在那绝非集体的“冥想”游戏里,情况完全不同。我所发现的,世上没有其他人,无论现在还是任何时候,能代替我发现。它只属于我独自去发现,也就是说:去承担。这个未知并非注定被知晓,几乎是出于必然,只要我肯费心关注。若它在沉默中等候被知晓的时刻,有时,当时机成熟,我听见它在召唤,只有我,只有我内心的“孩子”,被呼唤去知晓它。这不是一个暂时的未知。当然,我可以自由选择回应这召唤,或逃避它,说“明天”或“某天”。但这召唤是向我、只向我发出的,别人听不见,也无法回应。

每次我回应这召唤,“企业”中总有些许变化。效果立竿见影,当场便被感知为一种福祉——有时像突然的解放,一种巨大的宽慰,卸下我常未察觉的重担,而这重担的真实性通过这种宽慰、这种解放显现出来。在较小的幅度上,这类体验在所有发现工作中都很常见,我也曾谈及此事。然而,自我发现的工作(无论是在光天化日下还是隐秘进行)与其他发现工作的区别在于,它确实改变了“企业”本身。这不是量的变化,不是产量的增加,也不是产品规模或质量的差异。它是老板与“工人-孩子”之间关系的变化。或许老板自身也有变化,如果这变化不仅仅体现在他与“同一个”的关系上。例如,他可能不再那么关注产出——但这也是他与工人关系的一部分,可能是某种此前陌生的关怀或尊重出现了。在我冥想的每一次案例中,变化都朝向澄清与平静的方向发展,除非某些冥想流于表面,仅是“应景之作”,迫于即时且有限的需求,这种澄清与平静一直延续至今。

这赋予了自我发现工作与其他发现工作不同的意义,尽管两者有许多共同的本质面向。自我认知及其发现工作中,有一种维度,使其区别于其他任何认知与工作。或许这就是“知识之树”的“禁果”。或许冥想对我的吸引,或者说它揭示的那些奥秘对我的吸引,正是禁果的诱惑。我跨越了一道门槛,恐惧消失了。通往认知的唯一障碍是惯性,一种有时颇为巨大的惯性,但有限,绝非不可逾越。我几乎每迈一步都能感受到这惯性,隐秘而无处不在。它有时让我恼火,但从未让我气馁。(在数学工作中也是如此,惯性同样是主要障碍,但其重量远不及此。)这种惯性成了游戏的重要成分之一,更准确地说,是这个微妙且不对称的游戏中的主角之一——或者说有三个主角:一边是跃跃欲试的孩子,与老板(即惯性本身)竭力阻挠(却假装不在),另一边是隐约可见的美丽陌生者,充满神秘,既近又远,既逃避又召唤……