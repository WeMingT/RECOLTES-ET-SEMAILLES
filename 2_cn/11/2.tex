\section{(47) 孤独的探险}

这种对“冥想”的迷恋对我来说具有巨大的力量——其力量之大,堪比昔日对“女性”的吸引,而如今冥想似乎已取代了后者的位置。如果我刚写下“曾经是”,这并不意味着这种迷恋如今已然熄灭。自从一年前我投入数学以来,它只是退居幕后。经验告诉我,这种状态可能在一夜之间翻转,正如当前的状态本身也是一个完全未曾预料的翻转的结果。事实上,在我经历过的四段长时间的冥想时期中(其中一段持续了近一年半),我理所当然地认为自己会一直保持这种势头,直到生命的最后一息,去探寻我所能触及的生命奥秘和人类存在的奥秘。当笔记堆积成令人震撼的纸山,几乎要淹没我的工作房间时,我甚至特意定制了一件家具来容纳它们,并通过一个简单的算术级数计算,大手笔地预留了空间,以容纳未来几年必然会增加的更多笔记;如果我没记错的话,我预留了大约十五年的余量(这已经相当可观了!)。那件家具的老板干得真不错,就管理而言,这可是顶级的管理!这件事情,加上一次大规模整理所有与冥想工作或多或少相关的个人文件,竟成了他最后完成(几乎)的一项任务,就在偏好和投入发生转变之前。这不禁让人怀疑,他是否早有某种隐秘的打算,是否早已预见到“数学反思”「Réflexions Mathématiques;Mathematical Reflections」的卷册会填满那些原本据说留给未来“笔记”的空荡荡的架子。

诚然,冥想的激情,对自我发现的激情,广阔得足以填满我余生的每一天。同样真实的是,数学的激情并未耗尽,但或许这种饥渴将在未来几年内得到满足。我内心深处有某种东西希望如此,并将数学视为一种束缚,阻碍我追随一条唯有我才能独自前行的孤独探险之路。而我感到,这种内心的“某种东西”并非那个“老板”,也不是老板的某种冲动(老板的本性是分裂的)。我认为,数学的激情仍带有老板的印记,无论如何,追随它会让我的生命在一个封闭的圆圈中旋转;在一种轻松的圆圈中,在一种惯性的运动中,绝非一种焕然一新的运动。

我曾探究过这种数学激情在我生命中顽强持续的意义。当我追随它时,它并未真正填满我的生活。它带来欢乐,也带来满足,但它本身并不足以带来真正的充实,一种圆满。作为一种纯粹的智力活动,长时间高强度的数学活动反而有种令人迟钝的效果。我在他人的身上观察到这一点,尤其是在我自己身上,每次重新投入时都更加明显。这种活动如此片面,仅调动了我们直觉和感性能力中极小的一部分,以至于这些能力因长期不用而变得迟钝。很长一段时间,我并未察觉这一点,显然我的大多数同事也未比我更早觉察。自从我开始纯粹地冥想后,我似乎才对这件事变得敏感。只要稍加留意,这一点便显而易见——大量数学让人变得迟钝。即使在两年前半的冥想之后,数学激情被确认为一种真正的激情,一种我生命中重要的东西——如今当我投入这份激情时,仍存有一丝保留,一种不愿完全交付的感觉。我知道,所谓的“完全交付”实际上是一种放弃,是顺从惯性,是一种逃避,而非真正的给予。

对于冥想,我心中却没有丝毫这样的保留。当我投入其中时,我是全然交付的,这种给予中没有一丝分裂。我知道,在这种交付中,我与自己、与世界完全和谐——我忠于自己的本性,“我即是道”。这种给予对我自己和所有人都是有益的。它向我敞开自己,也向他人敞开,以爱解开我内心仍未松绑的结。

冥想让我向他人敞开,它有能力解开我与他人的关系,即便对方仍未解开自己的结。然而,能与他人稍稍交流冥想工作,或分享其带给我的某些认知的机会却极为罕见。这绝非因为这些内容“过于个人化”。用一个不完美的比喻来说,我无法与一个缺乏必要知识储备的数学家交流某一时刻吸引我的数学问题,除非对方在同一时刻也愿意对此感兴趣。有时,我会被某些数学问题迷住数年,却未曾遇到(也未主动寻找)一位可以与之交流的数学家。但我知道,如果我去寻找,总能找到这样的同行者;即便找不到,也不过是运气或时机的缘故;我感兴趣的东西,迟早会引起某些人甚至一群人的兴趣,无论是在十年后还是百年后,这本质上并无差别。这正是我工作意义的来源,即便这份工作是在孤独中完成的。如果世上没有其他数学家,且将来也不会再有,我不认为做数学对我来说还有意义——我怀疑任何其他数学家,或任何领域的“研究者”,情况亦是如此。这与我之前的观察相呼应,对我而言,“数学未知”是指无人知晓的东西——这不仅取决于我个人,而是关乎一种集体现实。数学是一场延续千年的集体冒险。

而在冥想中,若要与人交流,“知识储备”的问题并不存在,至少在我目前的阶段如此,我怀疑将来是否会有所不同。唯一的问题在于他人是否怀有与我相呼应的兴趣。这是一种对自我与他人真实内在的好奇,超越那些严谨的外表——只要真心感兴趣,这些外表其实掩盖不了什么。但我发现,在一个人身上出现这种兴趣的时刻,那些“真相的瞬间”,是稀有而短暂的。当然,遇到“对心理学感兴趣”的人并不罕见,他们读过弗洛伊德「Freud;Freud」和荣格「Jung;Jung」,还有其他许多著作,乐于进行“有趣的讨论”。他们带着自己的“文化”行李,或重或轻,这构成了他们自我形象的一部分,并强化了这种形象,而他们从不审视这形象,正如那些对数学、飞碟或钓鱼感兴趣的人一样。我刚才说的“储备”或“兴趣”,并非这种类型的——尽管此处相同的词语指代的是性质截然不同的东西。

换句话说:冥想是一场孤独的冒险。其本性即是孤独的。不仅冥想的工作是孤独的——我认为一切发现的工作皆如此,即便它嵌入集体之中。但冥想工作所诞生的认知是一种“孤独的认知”,一种无法分享、更遑论“交流”的认知;即便能分享,也仅在极少数瞬间。这是一项工作,一种认知,与最根深蒂固的共识背道而驰,它让所有人感到不安。这种认知固然能以简单的方式表达,用简洁明了的词语。当我对自己表达时,我在表达中学习,因为表达本身就是工作的一部分,承载着强烈的兴趣。但这些简单明了的词语,在面对冷漠或恐惧的封闭之门时,却无力传递意义。即便是梦的语言——那种力量迥异、资源无穷的语言,由一位不知疲倦且仁慈的“梦者”不断更新——也无法穿越这些门。

没有哪种冥想不是孤独的。若有一丝寻求他人认可、确认或鼓励的念头,便无冥想的工作,也无自我的发现。有人会说,一切真正的发现工作,在工作的那一刻,皆是如此。的确。但在工作之外,他人的认可——无论是亲人、同事,还是整个所属的群体——这种认可对工作的意义至关重要。这种认可与鼓励是最有力的激励之一,使得“老板”(借用这个比喻)毫无保留地点头放行,让人尽情投入。正是这些因素,在很大程度上决定了老板的投入。在我对数学的投入中,亦是如此,卡尔唐「Cartan;Cartan」、施瓦茨「Schwartz;Schwartz」、迪厄多内「Dieudonné;Dieudonné」、戈德芒「Godement;Godement」等人,以及后来的其他人的善意、温暖与信任,激励着我。然而,对于冥想工作,却没有这样的激励。这是一种与“老板”同工的激情,而老板不过是出于宽容,才多少容忍了它,因为它“毫无回报”。它确实结出果实,但这些果实并非老板所追求的。当老板不自欺欺人时,显然他不会将精力投入冥想——老板天性是群居的!

唯有孩童天性孤独。