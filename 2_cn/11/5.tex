\section{(50) 过去的重量}

几天前,我完成了对《收获与播种》「Récoltes et Semailles;Harvests and Sowings」的最后润色——此前一个多月,我一直以为自己即将在几天内完成。即使这次,在“最后润色”之后,我仍不确定是否真正结束了——事实上,有一个问题我尚未解决。那就是“理解哪些事件或情境最终引发了‘老板’投注的‘翻转’”,从冥想转向数学,逆着巨大的惯性力量。在毫无刻意的情况下,最近几天我的思绪一再回到这个问题,尽管我已开始转向完全不同的其他事务,包括数学问题(关于共形几何)。不如趁着这冥想“余势未尽”,稍作挖掘,清理出一片净土。

当我凭直觉试图回答“为何我要重拾数学”(指一种重要的、计划长期投入,至少数年的努力)时,几个联想浮现出来。其中最强烈的,或许与我近六七年来在教学活动中逐渐感到的慢性挫败有关。这种挫败感愈发强烈,我觉得自己“未被充分利用”,甚至常常将最好的自己投入,却面对一群漠然的学生,他们对我所能给予的毫不在意。

我四处看见美妙的事物等待着手,它们渴望着被完成。通常,只需最基本的知识储备即可着手,这些事物本身会启示我们发展何种语言去理解它们,获取何种工具去挖掘它们。只要我因教学活动与数学保持规律接触(即便水平再谦卑),我便无法不看见它们,即便在我生命中对数学兴趣最淡薄的时期。每当瞥见一件事物,稍作探究,便有更多美妙之物浮现,它们层层掩映,又层层揭示……无论在数学还是其他领域,只要以真兴趣注目,便会看见一种丰饶显现,一种深邃展开,令人感到无穷无尽。我所说的挫败,是无法向学生传递这种丰饶与深邃之感——哪怕只是点燃一丝欲望,去探索那些触手可及之物,在他们已决意投入数月或数年于所谓“研究”活动(为准备某项文凭)的时间里尽情享受。除了过去十年中两三个学生外,他们似乎连“尽情享受”的念头都感到恐惧,宁愿数月数年垂手踱步,或艰难地做着不知来龙去脉的鼹鼠活,只要最后有文凭在手。这种创造力的瘫痪,与是否有“天赋”或“能力”无关——这让我联想起反思之初,我曾略提及此类障碍的深层原因。但这并非我此处的目的,我意在确认,这种反复出现的情境,在过去七年教学活动中积累出的慢性挫败,已在我内心扎根。

“解决”这种挫败的显而易见方式,至少对“我心中的数学家”而非“教师”而言,是亲自去做那些我绝望地盼望学生最终能着手的事物。过去,我确实零星地做过一些,或在教学之余用几小时乃至几天偶尔思考,或在数学狂热期(有时如爆炸般袭来……)持续数周或数月。这类断续的零星工作,通常只能对一个问题进行初步梳理,所得视野极为零碎——更像是对未来工作的清晰展望,而工作本身始终未完成,且因更清晰,反而愈发炽热。两个月前,我概述了几个主要主题的轮廓,这些是我已稍有把握的课题。这就是我曾提及的“纲领草图”「Esquisse d'un Programme;Sketch of a Program」,最终将与本次反思合为“数学反思”「Réflexions Mathématiques;Mathematical Reflections」的第1卷。

显然,仅这种“私人”探勘工作不足以消解我的挫败。这种“未被充分利用”之感,反映的无疑是一种欲望(我认为源自自我,即“老板”的欲望),渴望施展行动。这里的行动,与其说是对他人的影响(比如推动学生、传递某些东西,或助他们取得文凭以谋职),不如说是“数学家”的行动:促成未知事实的发现,推动某理论的萌发……这立即与我先前的观察相呼应,即数学是一场“集体冒险”。回顾过去十年我做数学时的心境——那时我从未想过会再次发表成果,也清楚现今或未来的学生对我的探勘工作漠不关心——我立刻意识到,这绝非一个仅为个人愉悦或内在需求(与他人无关)而做事之人的心境。我做数学时,内心深处似乎认定,这些数学是为与他人分享而做,是我参与的更广大事物的一部分,这绝非个体性质。我可称之为“数学”,或更准确地说,“我们对数学事物的认知”。这里的“我们”,首先具体指向我认识并与之有共同兴趣的数学家群体;但无疑,它超越这一小圈子,也超越我个人。这“我们”指向我们的物种,那些历代通过某些个体对数学对象世界之现实感兴趣的人们。此刻写下这些文字前,我从未思考过这“事物”在我生命中的存在,更遑论探究其本质及其在我作为数学家与教师的生活中的角色。

我提到的施展行动的欲望,在我作为数学家的生命中,似乎表现为:将未知于众的事物从阴影中带出,不仅对我个人未知(正如我先前所述),且为供所有人使用,丰富一种共同的“遗产”。换言之,是渴望为这“事物”或“遗产”的扩展与丰厚做出贡献,它超越了我个人。

在这欲望中,诚然,通过我的作品扩展自我的欲望并非不存在。从这方面,我看到了“增长”与“扩展”的渴求,这是自我、“老板”的特征之一;其侵略性乃至极限上的破坏性(参见注释 \({44}^{\prime }§{13.1.1}\),第260页)。然而,我也意识到,增加那些(短时或长久)或多或少带有我名字的事物的欲望,远不足以涵盖或穷尽这更广阔的欲望或力量,它推动我为共同遗产的扩展出力。我感到,这种欲望或可找到满足(若非在“我的企业”中,因老板过于强势,至少在一个更成熟的数学家身上),即便其个人角色保持匿名。这或许是自我扩展倾向的一种“升华”形式,通过与超越自身的某种事物认同。除非这种力量本身并非自我性质,而是更微妙、更深刻的本质,表达一种深层需求,独立于任何条件,证明个人生命与整个物种生命之间的深刻联系,这联系是我们个体存在的意义之一。我不知晓,这也不是我此处的目的,去探究如此宏大的问题。

我的目的更谦逊,考察我个人的具体处境:一种挫败感,伴随数学活动的零星与暂时宣泄。于是,这处境的逻辑,迟早会引领我分享我的发现。直到去年,我并未准备为数学热情投入大规模且长期的精力,以通过详尽的“实证工作”挖掘并出版我发现的宝藏。于是,我选择将最珍视的内容,至少传达给一些足够“在行”的数学家朋友。

我想,若过去十年我找到一位数学家朋友,作为我的对话者与信息来源(如同50、60年代塞尔「Serre;Serre」在很大程度上扮演的角色),同时作为我传递“信息”的中继(这角色塞尔当年无需承担,因我自担此责),我“在数学中施展行动”的欲望或已得到足够满足,化解我的挫败,同时仅以适度的间歇性能量投入数学,将更多精力留给新热情。1975年,我首次怀着(至少内心隐含的)期待联系一位数学家朋友,最后一次是1982年,距今一年半。有趣的巧合,两次我都试图“推广”(希望被传递,甚至最终发展!)同一套同调代数与同伦的“纲领”,其萌芽可追溯至50年代,至60年代末已完全“成熟”(我深信如此);这纲领的初步发展和大致轮廓,正是我此刻应为《域的追逐》「Poursuite des Champs;Pursuit of Fields」撰写引言的主题!然而,因各次情况不同,我的尝试——重建如1970年前与塞尔、德利涅「Deligne;Deligne」那样的“优选对话者”关系——均告失败。共同情境是我对数学的有限投入。这无疑在1975年与1982年的两次交流中,使沟通跛足。事实上,我主要想“推广”某些东西,却不太在意努力“更新自己”,以成为对方——远比我“在行”(至少就同伦当前技术而言!)——满意的对话者。

我可将《致……的信》——《域的追逐》的首章(去年2月信,距今刚过一年)——视为我最后一次尝试,在昔日朋友中寻找共鸣,回应我如今的想法与关注。这封信开启(或重启)的反思,在数周后(我尚未察觉)成为1970年以来首篇拟出版的数学文本。近一年后,我才收到对此实质性信件的间接回应(参见注释 \footnote{这些笔记实为致……的长信延续,成为其首章。为便于昔日朋友及两三位可能感兴趣者(尤以罗尼·布朗「Ronnie Brown;Ronnie Brown」为首)阅读,我用打字机录入。此信从未获回复,收信人近一年后(我问及是否收到)真诚惊讶,我竟以为他会读它,鉴于他们预期我写的数学类型……} (38))。这回应比我迄今收到的任何数学家同事信件更雄辩,让我感受到自离开共同圈子后,朋友们对我的普遍态度。此信出自一位我以温暖友善姿态视为朋友之人,却带有刻意的嘲讽,强烈提醒我近年来愈发清晰察觉之事。此前,我主要注意到数学“大世界”对我个人的疏远,尤其是昔日亲近朋友(45)。如今,这不再是个人层面的疏远,而是一种共识,类似时尚,自视为理所当然,在“在行”者间流传:那种成千页的数学、我在十年或二十年间喋喋不休的概念(46,47),若认真考量并不严肃;不过是些喧嚣,围绕概形「Schéma;Scheme」与 étale 上同调的“一般胡言”(有时倒也有用,唉,不得不承认),其余最好仁慈地遗忘;若有人仍效仿吹响“格罗滕迪克式号角”,违背品味与严肃标准,就与他们宣称或未宣称的大师同类,若被如其应得对待,只能怪自己……

无疑,自1976年(50),尤其近两三年,许多此类回响(我刚直白转述)唤醒了我沉睡近十年的斗志。如反射般激起我冲进混战,堵住那些一无所知小子的嘴——十足愚蠢的反射,如斗牛见红布一挥便忘路而动!我认为这反射颇为表层,单凭它不足以撼动我。幸好,做数学远比冲向红布、被四面刺伤更有魅力。但坚持我风格与方法的数学,也是“投入混战”;是面对轻蔑与拒绝的自我肯定——这些无疑回应了我旧友感到的(或以为感到的)轻蔑,若非针对他们,至少针对他们仍全心认同的圈子。于是,这也多少是追逐红布,而非我的路。

这想法近几周多次浮现,或许今日反思主要朝此审视进发。途中,另一面向浮现,自我力量无疑占大比重,但非单纯斗志反射。而是我内心的欲望,其本质我尚不清楚:赋予过去十或十二年数学工作以意义,或见其全然显现意义;这意义(我深信)非私人愉悦或个人冒险所能涵盖。即使这欲望本质未明,因我未暇深究,此反思足以显示,正是在此欲望中,潜伏着压迫我、迫使我转向数学投入的力量——“翻转”之力。无论有无红布,它都会起作用。若它标志对过去的依恋,那是指1970年后的十年,而非已成白纸黑字、已完成之事,即1970年前。

我内心对这些已成之事及其未来、“后世”命运毫无忧虑(尽管是否有后世都存疑……)。我对此过去的兴趣,非我已完成之事(及其得失),而是当时眼前宏大纲领中未完成的部分,仅小部分由我及偶尔加入的朋友与学生实现。未曾预料或追求,这纲领自我更新,与我对数学事物的视野与方法同步。多年来,主题与目的皆移:不再是完成宏大细致的基础任务,我首要目的现为探究最迷我的奥秘,如“模体”「motifs;motives」之谜,或 \(\mathbb{Q}\) 上 \(\mathbb{Q}\) 的伽罗瓦群「Galois group;Galois group」的“几何”描述。途中,我难免为奠基勾勒草图,如《穿越伽罗瓦理论的长征》「La longue Marche à travers la théorie de Galois;The Long March through Galois Theory」中已开始,或《域的追逐》中正进行。然而,目的与表达风格已变。

换言之:近十年,我瞥见数学世界中神秘而绝美之物。它们非我独有,是为分享而生——我感到,瞥见它们的意义在于分享,使之被接受、理解、吸收……但分享,即便对自己,也需深化与发展——这是工作。我深知,即便余我百年,也无法完成此工。但今日我不必忧虑,在余生探索世界时,将多少岁月或月奉献于此,因另一工作待我独担。我无权也无责安排生命的季节。