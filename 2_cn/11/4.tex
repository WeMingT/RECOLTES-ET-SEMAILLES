\section{(49) 分裂的确认}

关于当前工作意义、给予与接纳的短暂反思,仿佛是思绪主线中的一段离题;或者更像是对某些特质的例证,这些特质将“冥想”与其他发现工作——尤其是数学工作——区分开来。昨天我意识到,这些特质具有双重效应,具体而言,是两种相反方向的效应:对“那个小家伙”产生独特的吸引力,而对“老板”则完全缺乏兴趣。看来,这种双重效应是事物本性使然,绝无可能通过任何妥协或调整加以缓和。无论怎么做,当“小家伙”追随他真正的偏好时,老板完全找不到自己的回报,毫无所得!

无疑,这正是那场翻转的意义所在,这场翻转可能在未来几年彻底清空我生命中的冥想(除了三个月前的“应景冥想”之类)。我并不认为这些年会因此完全荒芜,正如过去一年并未荒芜。但同样真实的是,我在这一年中学到的(数学之外的)东西微乎其微,若与前四年的任何一年相比。奇怪的是,我经历的四段长时间冥想,每一段都是极度充实的时光,没有任何迹象让我怀疑内心有何受挫。然而,若有压力锅爆炸,说明某处必有压力,这种压力不可能是当天的突发;它必定早已存在,潜伏在我视野之外,持续数周或数月,而我却全然沉浸于冥想,毫无察觉。

但此刻我似乎被笔锋(或者更准确地说,打字机的节奏)带跑了。现实是(除了最后一段冥想,因一系列事件与情境的交汇而被中途打断),冥想的强度从某一刻起逐渐减弱,就像一波浪潮,紧接着另一波浪潮准备取而代之……充实感,实话说,也跟随同样的节奏起伏,不同之处在于,它仅存在于“冥想浪潮”之时,而不在“数学浪潮”之中。

我试图厘清的处境,在我看来,已不再是冲突的处境,但显然它仍蕴含着冲突的种子、冲突的可能性。如今,这对我而言或许是最显著的标志,通过它对我生命历程的影响,显现出我内心的分裂。这种分裂正是老板与孩子之间的分裂。

我无法终结这种分裂。现在我能做的,既然已清楚察觉到它的存在,至少在这一表现形式中,就是保持关注,继续追踪它的迹象与演变,在未来的岁月里。或许这份对数学有些不合时宜的激情——必须承认——会因持续燃烧而耗尽(正如我内心的另一种激情已然耗尽……),为发现自我与命运的唯一激情腾出空间。

这份激情,我曾说过,广阔得足以填满我的生命——而我的一生,必定不足以将它耗尽。