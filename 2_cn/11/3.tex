\section{(48) 给予与接纳}

昨天在谈论冥想那孤独的本质时,我脑海中掠过一个念头:近六周来我所写的这些笔记,最终演变成了一种冥想,却注定要出版。这一点无疑以多种方式影响了冥想的形式,尤其是对简洁的关注,以及对谨慎的考量。冥想的一个核心特质——即在工作当下对内心发生之事的持续关注——仅偶尔显现,且流于表面。显然,这一切必定影响了工作的进程与品质。然而,我仍感到它具备冥想的特质,首先是因为其果实的性质,因为一种自我认知的浮现(尤其是关于某些过去的认知),而这些是我此前一直回避的。另一个方面是自发性,在这近五十个“章节”或“段落”中——反思自发地聚集成这些单元——我从未能在开始时预知其内容为何;每一次,内容都是在途中逐渐显现,每一次,这工作都带出新的现实,或以新的光芒照亮此前被忽略的事实。

这项工作最直接的意义在于与我自己的对话,因而是一种冥想。然而,这场冥想注定要出版,并且作为即将推出的“数学反思”「Réflexions Mathématiques;Mathematical Reflections」的“开场”,这一事实绝非无关紧要的附带情境,在工作中也从未被置之不理。对我而言,这是这项工作意义的核心部分。如果我昨天暗示老板肯定从中“找到自己的回报”(他可是个“样样都能找到回报”的大师,几乎无一例外!),这绝不意味着其意义仅限于此——不过是那匹著名的三脚马迟来的、几乎是身后之回报!我也多次感到,一个行为深刻的意义,有时超越了激发它的动机(无论是显见的还是隐秘的)。在这“回归数学”中,我察觉到另一种意义,不仅仅是某些心理力量在我身上特定时刻因特定原因交汇的结果。

这场我正在进行的冥想,是为了献给那些我在数学世界中认识并深爱过的人们——若我感到它是所瞥见的意义的重要部分,这并非寄望于这份给予会被接纳。它是否被接纳,不取决于我,而仅取决于接受它的人。当然,我并非对它是否被接纳无动于衷。但这不是我的责任。我唯一的责任是在这份给予中保持真实,也就是说,做我自己。

冥想让我认识的是那些谦卑而显见的事物,那些看似不起眼的东西。它们也是我在任何书籍或论著中——无论多么博学、深刻、天才——都无法找到的,是他人无法替我发现的。我探究了一片“迷雾”,我费心倾听,从中得知一个关于“竞技态度”及其显而易见意义的谦卑真相,无论是在我与数学的关系中,还是与他人的关系中。即使我“原文”阅读了《圣经》、古兰经、奥义书,再加上柏拉图「Platon;Plato」、尼采「Nietzsche;Nietzsche」、弗洛伊德「Freud;Freud」和荣格「Jung;Jung」,成为广博深邃的学问奇才——这一切只会让我远离那个真相,一个童稚的、显而易见的真相。我可以百遍重复基督的话:“那些如孩童的人有福了,因为天国属于他们”,并对此作出精妙的评论,但这只会让我更加远离我内心的孩子,远离那些让我不安、唯有孩子能见的谦卑真相。这些,才是我能献上的最好之物。

我也深知,当这些事物被说出并献上,用简单而明澈的语言,它们并不会因此被接纳。接纳,不仅仅是接收信息,带着尴尬甚或兴趣:“哎呀,谁能想到呢……!”或是:“这倒也不算太意外……”接纳,往往是在给予者身上认出自己。通过他人的存在,与自己相识。