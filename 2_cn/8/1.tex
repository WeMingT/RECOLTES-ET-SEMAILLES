\section{(25) 学生与计划}

我尚未完全梳理完与那些数学家的关系,那时我觉得自己与他们同属一个世界,一个“数学共同体”。我仍需尤其审视我与学生们的关系——我所经历的那些,以及对那些视我为前辈的人的关系。

总体而言,我相信自己可以毫无保留地说,我与学生们的关系是基于尊重的。至少在这点上,我认为,我从前辈那里接受的东西,在岁月流转中并未退化。由于我以做“难懂”的数学闻名(这概念确实极为主观!),且比其他导师更严格(这已不那么主观),来找我的学生从一开始便动力十足:他们“有心”!只有一个学生起初有些“吊儿郎当”,不太明确他是否会起步——但他最终还是启动了,无需我多加推动……

据我回忆,我接受了所有请求与我共事的学生。其中两人,几周或数月后发现我的工作风格不适合他们。实话说,如今我觉得这两次皆是某种阻滞,我当时却草率解读为数学工作的无能。今天我对这类判断会谨慎得多。我毫不犹豫地向这两人表达了我的印象,建议他们不要继续从事我认为不符其天赋的职业。事实上,我后来得知,至少其中一人证明我错了:这位年轻研究者在代数几何与数论交界的难题上声名鹊起。至于另一位女学生,我不知她在与我受挫后是否继续。或许我对其能力的印象过于武断,打击了她的信心,而她本可能如他人般出色完成工作。我觉得自己对这些学生,如同对其他学生一样,给予了信任与支持。但面对显然是阻滞而非无能的迹象,我缺乏分辨力。\footnote{(18)\par 我认为这并非疏忽,而是成熟不足、无知使然。十年后,我才开始关注阻滞机制——在我自身、亲近之人及学生中——并体会其在每个人生命中的巨大作用,不只在学校或大学。当然,我遗憾当时未有更成熟的分辨力,但不后悔清楚表达我的印象,无论是否准确。若某次我察觉工作缺乏认真,指出事实对我而言必要且有益。若另次我的结论过于仓促且无据,我并非唯一负责任者。被震动的学生仍有选择:或从中汲取教训(或许首次如此),或受挫而放弃,甚至改行(未必坏事!)。}(18)

从六十年代初起,约十年间,十一名学生在我指导下完成国家博士论文。\footnote{(19) 耶稣与十二使徒 \par 耶稣与十二使徒。自1970至今,另有一学生,伊夫·拉德盖耶里「Yves Ladegaillerie;Yves Ladegaillerie」,在我指导下准备并通过论文。第一时期的十一名学生是P.贝尔特洛「P. Berthelot;P. Berthelot」、M.德马祖尔「M. Demazure;M. Demazure」、J.吉罗「J. Giraud;J. Giraud」、M.哈基姆夫人「Mme M. Hakim;Mme M. Hakim」、黄春生夫人「Mme Hoang Xuan Sinh;Mme Hoang Xuan Sinh」、L.伊吕西「L. Illusie;L. Illusie」、P.茹瓦努卢「P. Jouanolou;P. Jouanolou」、M.雷诺「M. Raynaud;M. Raynaud」、M.雷诺夫人「Mme M. Raynaud;Mme M. Raynaud」、N.萨维德拉「N. Saavedra;N. Saavedra」、J.L.韦迪耶「J.L. Verdier;J.L. Verdier」。(其中六人在1970年后完成论文,那时我的数学可用性极有限。)其中,米歇尔·雷诺「Michel Raynaud;Michel Raynaud」地位特殊,他自行发现论文核心问题与概念,完全独立发展,我作为“论文指导”的角色仅限于阅读完成稿、组建并参与评审团。

若由我提议题目,我小心限于我有强烈关联的领域,确保必要时能支持学生工作。一个显著例外是米歇尔·雷诺夫人「Mme Michèle Raynaud;Mme Michèle Raynaud」关于基本群局部与全局勒夫谢茨定理「Théorèmes de Lefschetz;Lefschetz Theorems」的工作,用适Étale位点上的1-层表述。我认为此题(事实证明)困难,且对所提猜想(几乎无疑)无证明思路。她于七十年代初继续此工,如其夫先前,无我或他人协助,独立发展出精妙原创方法。这优秀工作开启了将结果扩展至n-层的问题,我认为这是概形「Schéma;Scheme」语境下“弱勒夫谢茨定理”类型的自然归宿。此处相关猜想的表述(同样几无疑义)本质依赖n-层概念,其探索应为此书主旨[实为《数学反思》第3卷,而非本卷1《收获与播种》——见引言,页(v)。],如其名“追寻层”「À la Poursuite des Champs;In Pursuit of Fields」所示。我们或将在适当处再议。

另一特例是黄夫人「Mme Sinh;Mme Sinh」,我于1967年12月在河内「Hanoï;Hanoi」被疏散的大学讲授一月课程-研讨会时初识她。次年我为其提议论文题目。她在战时艰难条件下工作,与我仅偶尔通信。她于1974/75年赴法(借温哥华国际数学家大会之机),在巴黎通过论文(评审团由卡尔唐「Cartan;Cartan」主持,包括施瓦茨「Schwartz;Schwartz」、德尼「Deny;Deny」、齐斯曼「Zisman;Zisman」与我)。

最后须提及皮埃尔·德利涅「Pierre Deligne;Pierre Deligne」与卡洛斯·孔图-卡雷尔「Carlos Contou-Carrère;Carlos Contou-Carrère」,他们或多或少似我的学生,前者约1965-68年,后者约1974-76年。两人显然(且始终)才华卓越,用法迥异,际遇亦大不同。德利涅来布雷「Bures;Bures」前略受蒂茨「Tits;Tits」(在比利时)指导——我疑他是否在常规意义上做过谁的学生。孔图-卡雷尔曾师从桑塔洛「Santalo;Santalo」(在阿根廷),及短时随托姆「Thom;Thom」!两人接触我时已具数学家风范,只是孔图-卡雷尔缺方法与技艺。

我对德利涅的数学角色仅限于每周略述我所知的代数几何,他如听童话般吸收——仿佛早已知晓;同时提出问题,他多当场或数日后解答。这是我知晓的德利涅早期工作。其后1970年起(对他及“正式学生”皆然),我仅从零星遥远回音得知[实为《数学反思》第3卷,而非本卷1《收获与播种》——见引言,页(v)。]。

对孔图-卡雷尔,据其论文开篇自述,我的角色仅限于引入概形语言。我对其近年国家博士论文工作——涉及我能力之外的最新题目——仅遥远关注。他在大千世界几经波折后,最近迫不得已(今看来违其本意),临阵求助于我,担任论文指导并组建评审团。(这让他冒着“1970年后格罗滕迪克学生”之风险,在当时或有严重不便……)我尽力履行职责,这或是我最后一次(在国家博士论文层面)行使此职。在这特殊情境下,我尤感欣慰于让·吉罗「Jean Giraud;Jean Giraud」的友好协助,他抽出一两月细读厚稿,作详尽温暖报告。}(19) 选定合意题目后,他们各以热情投入工作,我感到他们对自己选题有强烈认同。然有一例外,一学生或许无真信念,选了个“该做”的题目,带有乏味一面,是对已有观念的技术整理,有时艰难甚至枯燥,无甚惊喜或悬念可期。\footnote{(20)\par 这让我想到莫妮克·哈基姆「Monique Hakim;Monique Hakim」的题目,实话说不更吸引人,我不知她如何保持士气!若她偶感艰难,至少未至悲伤或阴郁,我们间的合作氛围友好轻松。}(20) 我为宏大计划需人手,提议此题时或缺心理洞察,未必适合此学生个性。他则未必意识到自己陷入何种困境!总之,我们皆未及时看出开局不利,最好改弦易辙。

他显然工作无真热情,总带几分悲伤与阴郁。我想我当时已不太在意这些——尽管我该记得,这在任何研究工作中,甚至不限于研究,决定成败!我角色仅限于工作拖延时感烦躁,重启时松口气,最终计划“完成”时如释重负。

1970年觉醒数年后,与这位旧生(如今如众人般成教授,那时风气宽松!)通信,我才想到此案例确有不妥,或非全然成功。今看,这是个失败,尽管“计划完成”(绝非敷衍!)、获文凭与职位。我负主要责任,因将计划需求置于个人之上——此人信任托付于我。我方才自诩“毫无保留”的对学生的“尊重”,在此流于表面,脱离真正尊重之魂:对人需求的深情关注,至少在其满足依赖我时。此处需求,是工作中的喜悦,否则工作失意义,成负担。

此反思中我曾言及“无爱之世界”,并在自身寻其种子。此乃一大种子——我不知其在他人中如何发芽。这缺乏关注与真爱的肤浅尊重,亦是我给予子女的尊重。与他们,我有幸见此种子发芽繁茂。我也略懂,面对收获无益抗拒……