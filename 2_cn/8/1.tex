\section{(25) L'élève et le programme}

Je n'ai pas terminé de faire le tour de ce qu'ont été mes relations aux autres mathématiciens, au temps où je me sentais faire partie avec eux d'un même monde, d'une même "communauté mathématique". Il me reste surtout à examiner ce qu'ont été mes relations à mes élèves, telles que je les ai vécues, et à d'autres pour lesquels je faisais figure d'aîné.

De façon générale, je crois pouvoir dire, sans aucune réserve, que mes relations à mes élèves ont été des relations de respect. A ce sujet tout au moins, je crois, ce que j'avais reçu de mes aînés au temps où j'ai été moi-même élève, ne s'est pas dégradé au cours des années. Comme j'avais la réputation de faire des maths "difficiles" (notion il est vrai des plus subjectives !), et de plus d'être plus exigeant que d'autres patrons (chose déjà moins subjective), les étudiants qui venaient vers moi étaient dès le début assez fortement motivés : "ils en voulaient"! Il y a eu juste un élève qui au début était un peu "ollé ollé", c'était pas tellement clair s'il allait démarrer - et puis si, il s'est déclenché sans que j'aie eu à pousser...

Pour autant que je puisse me rappeler, j'ai accepté tous les élèves qui demandaient à travailler avec moi. Pour deux d'entre eux, il s'est avéré au bout de quelques semaines ou mois que mon style de travail ne leur convenait pas. A vrai dire, il me semble maintenant qu'il s'est agi les deux fois de situations de blocage, que j'ai alors interprété hâtivement comme signes d'inaptitude au travail mathématique. Aujourd'hui je serais beaucoup plus prudent pour faire de tels pronostics. Je n'ai eu aucune hésitation à faire part de mes impressions aux deux intéressés, en leur conseillant de ne pas continuer dans une carrière qui, me semblait-il, ne correspondait pas à leurs dispositions. En fait, j'ai su que pour un de ces deux élèves tout au moins, j'avais fait erreur : ce jeune chercheur a acquis par la suite une notoriété dans des sujets difficiles, aux confins de la géométrie algébrique et de la théorie des nombres. Je n'ai pas su si l'autre élève, une jeune femme, a continué ou non après sa déconvenue avec moi. Il n'est pas exclu que mon impression sur ses aptitudes, exprimée de façon trop péremptoire, l'ait découragée, alors qu'elle était peut-être toute aussi capable qu'un autre de faire du bon travail. Il me semble que j'avais fait crédit et confiance à ces élèves comme aux autres. J'ai manqué par contre de discernement pour faire la part des choses devant ce qui étaient sûrement des signes de blocage, plutôt que d'inaptitude \footnote{(18)\par Je crois que ce manque de discernement ne provenait pas d'une négligence de ma part en ces deux occasions, mais plutôt d'un manque de maturité, d'une ignorance. Ce n'est qu'une dizaine d'années plus tard que j'ai commencé à prêter attention aux mécanismes de blocage, aussi bien dans ma propre personne que dans mes proches ou chez des élèves, et à mesurer le rôle immense qu'ils jouent dans la vie de chacun, et pas seulement à l'école ou à l'université. Bien sûr, je regrette de n'avoir pas eu en ces deux occasions le discernement d'une maturité plus grande, mais non pas d'avoir exprimé clairement mes impressions, fondées ou non. Quand je constatais dans tel cas un travail fait sans sérieux, le fait de nommer ces choses pour ce qu'elles sont me paraît une chose nécessaire et bienfaisante. Si dans tel autre cas encore, la conclusion que j'en tirais était hâtive et non fondée, je n'étais pas le seul pourtant dont la responsabilité était engagée. L'élève ainsi secoué avait le choix encore, soit d'en prendre de la graine (c'est peut-être ce qui s'est passé une première fois), soit de se laisser décourager, et peut-être alors de changer de métier (ce qui n'est pas nécessairement une mauvaise chose non plus!).}(18).

A partir du début des années soixante, donc pendant une dizaine d'années, onze élèves ont fait une thèse de doctorat d'état avec moi \footnote{(19) Jésus et les douze apôtres \par Jésus et les douze apôtres. Depuis 1970 jusqu'à aujourd'hui un élève encore, Yves Ladegaillerie, a préparé et passé une thèse avec moi. Les élèves de la première période sont P. Berthelot, M. Demazure, J. Giraud, Mme M. Hakim, Mme Hoang Xuan Sinh, L. Illusie, P. Jouanolou, M. Raynaud, Mme M. Raynaud, N. Saavedra, J.L. Verdier. (Six parmi eux ont d'ailleurs terminé leur travail de thèse après 1970, donc à une époque où ma disponibilité mathématique était des plus limitées.) Parmi ces élèves, Michel Raynaud prend une place à part, ayant trouvé par lui-même les questions et notions essentielles qui font l'objet de son travail de thèse, qu'il a de plus développé de façon entièrement indépendante; mon rôle de "directeur de thèse" proprement dit s'est donc borné à lire la thèse terminée, à constituer le jury et à en faire partie.

Quand c'était mo i qui proposais un sujet, je prenais bien soin de me borner à ceux auxquels j' avais une relation suffi samment forte pour me sentir en mesure, en cas de besoin, d'épauler le travail de l'élève. Une exception notable a été le travail de Mme Michèle Raynaud sur les théorèmes de Lefschetz locaux et globaux pour le groupe fondamental, formulés en termes de 1-champs sur des sites étales convenables. Cette question me paraissait(et s' est bel et bien avérée) diff i cile, et je n' avais pas d'idée de démonstration pour les conjectures que je proposais(lesquelles ne pouvaient d' ailleurs guère faire de doute). Ce travail s' est poursuivi aux débuts des années 70, et Mme Raynaud (comme ce fût le cas précédemment pour son mari) a développé une méthode délicate et originale sans aucune assistance de ma part ou d' ailleurs. Cet excellent travail ouvre d' ailleurs la question d' une extension des résultats de Mme Raynaud au cas des n-champs, qui me semble devoir représenter l' aboutissement natural,dans le contexte des schémas, des théorèmes du type"théorème de Lefschetz faible". La formulation de la conjecture pertinente ici(qui ne peut guère faire de doute non plus) utilise cependant de façon essentielle la notion de n-champs, dont la poursuite est censée être l' objet principal du présent ouvrage [II s' agit en fait du volume 3 des Réflexions Mathématiques, et non du présent volume 1 Récoltes et Semailles - voir Introduction,p.(v).], comme son nom"A la Poursuite des Champs"l' indique. Nous y reviendrons sans doute en son lieu,

Un autre cas assez à part est celui de Mme Sinh, que j'avais d'abord rencontrée à Hanoï en décembre 1967, à l'occasion d'un cours-séminaire d'un mois que j'ai donné à l'université évacuée de Hanoi. Je lui ai proposé l'année suivante son sujet de thèse. Elle a travaillé dans les conditions particulièrement difficiles des temps de guerre, son contact avec moi se bornant à une correspondance épisodique. Elle a pu venir en France en 1974/75 (à l'occasion du congrès international de mathématiciens à Vancouver), et passer alors sa thèse à Paris (devant un jury présidé par Cartan, et comprenant de plus Schwartz, Deny, Zisman et moi).

Il me faut enfin mentionner encore Pierre Deligne et Carlos Contou-Carrère, qui l'un et l'autre ont fait un peu figure d'élève, le premier vers les années 1965-68, le second vers les années 1974-76. L'un et l'autre avaient visiblement (et ont toujours) des moyens peu communs, dont ils ont fait usage de façon très différente et avec des fortunes très différentes aussi. Avant de venir à Bures, Deligne avait été un peu élève de Tits (en Belgique) - je doute qu'il ait été élève de quelqu'un en mathématique, au sens courant du terme. Contou-Carrère avait été élève de Santalo (en Argentine), et pendant quelque temps de Thom! peu ou prou). L'un et l'autre avaient déjà la stature d'un mathématicien au moment où le contact s'est établi, à cela près que Contou-Carrère manquait de méthode et de métier.

Mon rôle mathématique auprès de Deligne s'est borné à le mettre au courant, à la petite semaine, du peu que je savais en géométrie algébrique, qu'il a appris comme on écoute un conte - comme s'il l'avait toujours su; et chemin faisant aussi, à soulever des questions auxquelles le plus souvent il trouvait réponse, sur le champ ou dans les jours suivants. Ce sont là les premiers travaux de Deligne que j' ai connus. Ceux d'après 1970(pour lui comme aussi pour mes"élèves offi ciels") ne me sont connus que par des échos très épars et lointains [II s' agit en fait du volume 3 des Réflexions Mathématiques, et non du présent volume 1 Récoltes et Semailles - voir Introduction,p.(v).].

Mon rôle auprès de Contou-Carrère, suivant ce qu' il en dit lui-même au début de sa thèse, s' est borné à l' introduire au language des schémas. Je n' ai suivi que de très loin en tous cas le travail qu' il a préparé comme thèse de doctorat d'état en ces dernières années, sur un sujet des plus actuels qui échappe à ma compétence. C' est à la suite de quelques mésaventures dans le vaste monde que Contou-Carrère s' est vu fi nalement conduit récemment, in extremis et (m' apparaît-il maintenant) à son corps défendant, à faire appel à mes services pour faire fonction de directeur de thèse et constituer un jury.(Cela l' exposait au risque de faire fi gure d'élève de Grothendieck"après 1970", dans une conjecture où cela peut présenter de sérieux inconvénients……).Je me suis acquitté de cette tâche du mieux que j' ai pu, et il est probable que c' est là la dernière fois que j' aura i exercé cette fonction (au niveau d' une thèse de doctorat d'état). Je suis d' autant plus heureux, dans cette circonstance un peu particulière,de l' amical concours de Jean Giraud, qui a aussi pris sur son temps un mois ou deux pour faire une lecture minutieuse du volumineux manuscript, dont il a fait un rapport circonstancié et chaleureux.}(19). Après avoir choisi un sujet à leur convenance, ils ont chacun fait leur travail avec entrain, et (ainsi l'ai-je senti) ils se sont fortement identifiés au sujet qu'ils avaient choisi. Il y a eu pourtant une exception, dans le cas d'un élève qui avait choisi, peut-être sans véritable conviction, un sujet "qui devait être fait", mais qui avait des aspects ingrats aussi, s'agissant d'une mise au point technique, parfois ardue, voire aride, d'idées qui étaient déjà acquises, alors qu'il n'y avait plus guère de surprises ni de suspense en perspective \footnote{(20)\par Cela me fait penser au sujet qu'avait pris Monique Hakim, qui n'était pas plus engageant à vrai dire, je me demande comment elle a fait pour garder le moral ! Si elle a peiné par moments, ce n'était pas en tous cas au point de la rendre triste ou maussade, et le travail entre nous s'est fait dans une ambiance cordiale et détendue.}(20). Emporté par les nécessités d'un vaste programme pour lequel j'avais besoin de bras, j'ai dû manquer de discernement psychologique en proposant ce sujet qui ne convenait pas, sûrement, à la personnalité particulière de cet élève. Lui de son côté ne devait pas trop se rendre compte dans quelle galère il s'embarquait là ! Toujours est-il que ni lui ni moi n'avons su voir à temps que c'était parti du mauvais pied, et qu'il valait mieux repartir sur autre chose.

Visiblement il travaillait sans véritable conviction, et sans se départir d'un air toujours un peu triste, maussade. Je crois que j'en étais arrivé déjà à un point où je ne faisais pas trop attention à ces choses-là, qui pourtant (j'aurais dû m'en souvenir) font le jour et la nuit dans tout travail de recherche, et pas seulement de recherche ! Mon rôle alors s'est borné à être ennuyé quand le travail faisait mine de traîner en longueur, et de pousser un "ouf !" de soulagement quand ça repartait, puis quand enfin le programme prévu a fini par être "bouclé".

Ce n'est que des années après mon réveil de 1970, ayant eu à correspondre avec cet ancien élève (devenu professeur, comme tout le monde d'ailleurs en ces temps cléments !), que l'idée m'est venue que décidément quelque chose avait cloché dans ce cas-là, que ce n'était peut-être pas un succès total. Aujourd'hui, il m'apparaît comme un échec, malgré le "programme bouclé" (nullement bâclé !), le diplôme, et le poste à la clef. Et je porte une large part de responsabilité, pour avoir fait passer les besoins d'un programme avant ceux d'une personne - d'une personne qui s'en était remise à moi avec confiance. Le "respect" dont tantôt je me suis prévalu ("sans réserve aucune"), dont j'aurais fait preuve vis-à-vis de mes élèves, est resté ici superficiel, séparé de ce qui fait l'âme véritable du respect : une attention affectueuse aux besoins de la personne, dans la mesure tout au moins où leur satisfaction dépendait de moi. Besoin, ici, d'une joie dans le travail, sans quoi celui-ci perd son sens, devient contrainte.

J'ai eu l'occasion au cours de cette réflexion de parler d'un "monde sans amour", et je cherchais en ma propre personne les germes de ce monde-là que je récusais. En voilà un de taille - et je ne saurais dire aujourd'hui comment il a levé en autrui. Ce respect superficiel, dénué d'attention, de véritable amour, est le "respect" aussi que j'ai accordé à mes enfants. Avec eux, j'ai eu ce privilège de voir lever cette graine et la voir proliférer. Et j'ai compris aussi tant soit peu, que rien ne sert à rechigner devant la récolte...



