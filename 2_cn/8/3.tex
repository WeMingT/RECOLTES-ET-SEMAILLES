\section{(27) la bavure - ou vingt ans après}

Sauf peut-être dans le cas des deux étudiants dont j’ai parlé, avec qui une relation de travail ne s'est finalement pas établie, je ne me rappelle pas que les autres étudiants qui venaient me trouver pour demander de travailler avec moi, soient venus avec un "trac" ou une crainte. Sans doute ils devaient déjà me connaître peu ou prou, pour avoir suivi ne serait-ce que quelque temps mon séminaire à l' IHES. Si gêne il y avait au commencement de notre relation, celle-ci finissait par se dissiper, sans plus laisser de traces, au cours du travail. Je devrais cependant faire ici deux exceptions. L'une concerne l'élève qui n'est pas arrivé à prendre vraiment goût à son travail, et qui est resté monosyllabique même pendant notre travail en commun. Peut-être aussi est-il venu à un moment où ma disponibilité allait devenir moins grande, et qu'il n'y a pas eu avec lui des séances de travail sur pièces, pendant des après-midi et des jours entiers. Non, en effet je ne me rappelle pas de telles séances ; je crois plutôt qu'on se voyait surtout en coup de vent, pendant une heure ou deux, pour faire le point où il en était. Décidément c'est lui qui a dû le moins bien tomber avec moi !

L'autre élève par contre dont je voulais parler a travaillé avec moi à l'époque où j'avais encore une disponibilité complète pour mes élèves. Notre relation a été cordiale depuis les débuts. Il fait même partie des quelques élèves avec lesquels s'est établi une relation amicale, ceux qu'il m'arrivait de voir chez eux tout comme ils venaient chez moi, une relation un peu de famille à famille. Il est vrai que même dans ces cas-là, la relation restait toujours à un niveau relativement superficiel, tout au moins en ce qui me concerne. Au niveau conscient, alors que déjà je ne me rendais pas compte de grand-chose de ce qui se passait chez moi, sous mon propre toit, je ne savais presque rien finalement sur la vie de mes amis mathématiciens, élèves ou non, à part les noms de l'épouse et des enfants (et encore, il m'arrivait de les oublier, sans que jamais on m'en veuille !). Peut-être que je représentais un cas extrême de "polard", mais je crois que dans le milieu mathématique que j'ai connu, la plupart sinon toutes les relations, même amicales et affectueuses, restaient à ce niveau superficiel où on ne sait finalement que très peu de choses l'un de l'autre, si ce n'est ce qui est perçu au niveau de l'informulé. C'est une des raisons, sûrement, pourquoi le conflit entre personnes était si rare dans ce milieu, alors qu'il est clair pour moi que la division a existé à l'intérieur de la plupart de mes collègues et amis, et à l'intérieur de leurs familles, tout autant que chez moi et que partout ailleurs.

Je ne crois pas que ma relation à cet élève se soit distinguée de ma relation à d'autres, et je n'avais pas non plus le sentiment à l'époque qu'inversement, sa relation à moi se distinguait d'une façon notable de celle d'autres élèves, et notamment de ceux avec qui des liens amicaux se sont liés. Ce n'est que depuis peu que j'ai pu me rendre compte qu'il a dû s'agir d'une relation plus forte que pour la plupart de mes autres élèves. Les manifestations visibles d'un conflit inexprimé sont venues comme une révélation inattendue, près de vingt ans après l'époque où il a été mon élève. C'est alors seulement que j'ai fait le rapprochement avec un "petit" fait depuis longtemps oublié. Pendant longtemps, peut-être même pendant toute la période (de quelques années donc) où il nous arrivait de travailler ensemble plus ou moins régulièrement, et élève avait conservé un certain "trac". Celui-ci se manifestait à chaque rencontre, par des signes qui ne trompent pas. Ces signes disparaissaient assez rapidement ensuite, au cours du travail en commun. J'étais bien sûr gêné par ces signes de malaise, et je sentais qu'il l'était davantage. On faisait l'un et l'autre semblant d'ignorer la chose, comme de juste. Sûrement l'idée d'en parler ne serait venue à l'un ni à l'autre, ni même celle d'accorder quelque attention par devers soi à une situation étrange, visiblement digne d'intérêt ! Par lui comme par moi, ce "trac" devait être ressenti comme une simple "bavure", qui n'avait pas lieu d'être. La "bavure" se rappelait à notre bon souvenir régulièrement, mais à chaque fois, elle avait le bon goût de disparaître, le temps de nous laisser loisir de nous occuper tranquilles de choses sérieuses, des maths - et en même temps d'oublier "ce qui n'avait pas lieu d'être". Je ne me rappelle pas m'être arrêté une seule fois, pour me poser quelque question sur la signification de la bavure, et je suis persuadé qu'il en était de même du côté de mon élève et ami. Rien sans doute, dans ce que nous avions connu l'un et l'autre autour de nous, depuis notre première enfance, ne pouvait suggérer en lui ou en moi l'idée d'une autre attitude vis-à-vis d'une chose gênante, que celle de l'écarter dans la mesure du possible, pour qu'elle cesse de gêner. Dans ce cas-là c'était tout à fait possible et facile même, et on était parfaitement d'accord pour n'avoir rien vu rien senti rien entendu.

Par bien des échos et recoupements qui me reviennent depuis deux ou trois ans, je me rends compte pourtant que ce qu'on avait écarté comme n'ayant pas lieu d'être, n'a pas dû cesser pour autant d'être, et de se manifester. Ce qui me revenait parfois n'a pas non plus "lieu d'être" - et pourtant "c'est", et maintenant ne peut plus être écarté du revers d'une main...


