\section{(27) 瑕疵——或二十年后}

除去我提到的两位未建立工作关系的学生,我不记得其他来找我请求共事的学生带着“怯场”或恐惧。他们多半已略知我,或因在IHES「Institut des Hautes Études Scientifiques;IHES」短期听过我的研讨会。若关系初有尴尬,也会在共事中消散,不留痕迹。然而,我须提及两例外。一是我前述那位未能真正喜欢工作的学生,即便共事时他也寡言。或许他来的时机不佳,我的可用性正减少,与他未有整日整下午的深入工作会。我确不记得此类会面,多是匆匆一两小时,了解进展。显然,他与我配合最不理想!

另一学生则在我对学生全然可用时与我共事。我们关系自始友好。他是少数与我建立家庭式友谊的学生之一,我会去他家,他也来我家。即便如此,至少对我而言,这关系仍相对表面化。当时我已有意识地觉察不到自家屋檐下之事,对数学家朋友——无论学生与否——的生活也几乎一无所知,除配偶与子女姓名(有时还忘,却无人怪我!)。我或是个极端“书呆子”,但我认为,在我所知的数学环境中,大多数乃至所有关系,即便友好深情,也停留在这浅层,彼此知之甚少,除非在未言明的层面感知。这或是一因,解释为何此环境中人与人冲突极少,而我清楚,大多数同事与朋友内心及家庭中,如我及他处一般,存在分裂。

我不觉我与此学生的关系有别于其他,且当时也不觉他对我与其他学生——尤其那些结下友谊者——的关系有显著不同。直到最近,我才意识到,这关系对他或比对多数学生更深。未表达冲突的明显迹象,如意外启示,在他做我学生近二十年后浮现。我才将之与一早已忘的“小事”联系。那时(数年间),我们或多或少定期共事,他始终保留某种“怯场”。每次见面,这怯场以 unmistakable 迹象显现,随共事很快消退。我自然为此不适,他更甚。我们都假装无视,理所当然。想必彼此从未想过讨论,甚至私下关注这怪况——显然值得探究!对他与我,这“怯场”仅是“瑕疵”,不应存在。它定期提醒我们,但每次都识趣消失,让我们安心处理正事——数学——同时忘却“那不应存在之物”。我不记得曾停下来探究其意义,确信他亦然。我们自幼周遭经历,无一暗示面对尴尬之物可有他态,唯尽可能排除以免干扰。此处极易做到,我们默契地视若无睹。

然而,近两三年诸多回音与交汇让我明白,那被我们排除为“无存”之物,未因此停止存在与显现。有时传回之物也“无存”——却“存在”,如今无法一挥手抹去……