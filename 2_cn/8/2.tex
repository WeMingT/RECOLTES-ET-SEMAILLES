\section{(26) Rigueur et rigueur}

Si je fais exception de cet élève, qui sûrement n'était pas moins "doué" que les autres, je peux dire que les relations entre mes élèves et moi ont été cordiales, souvent même affectueuses. Par la force des choses, tous ont appris à être patient vis-à-vis de mes deux principaux défauts comme "patron" : celui d'avoir une écriture impossible (pourtant tous je crois ont fini par apprendre à me déchiffrer) et, chose plus sérieuse certes (et dont je ne me suis aperçu que beaucoup plus tard), ma difficulté foncière à suivre la pensée d'autrui, sans que je ne l'aie d'abord traduite dans mes images à moi, et repensée dans mon propre style. J'étais beaucoup plus porté à communiquer à mes élèves une certaine vision des choses dont je m'étais imprégné fortement, plutôt que d'encourager en eux l'éclosion d'une vision personnelle, peut-être assez différente de la mienne. Cette difficulté dans la relation à mes élèves n'a pas disparu encore aujourd'hui, mais il me semble que ses effets sont atténués, du fait que je me rends compte de cette propension en moi. Peut-être que mon tempérament, inné ou acquis, me prédispose-t-il plus au travail solitaire, qui a été le mien d'ailleurs pendant les quinze premières années de mon activité mathématique (de 1945 à 1960 environ), qu'au rôle de "maître" au contact d'élèves dont la vocation et la personnalité mathématiques ne sont pas entièrement formés \footnote{(21)\par Il serait peut-être plus exact de dire que pour le tempérament qui est le mien, c'est la maturité nécessaire qui me fait encore défaut pour assumer pleinement un rôle d'enseignant. Mon tempérament acquis a été longtemps marqué par une prédominance excessive des traits "masculins" (ou "yang"), et un des aspects de la maturité est justement un équilibre "yin-yang" à dominante "féminine" (ou "yin"). 

(Rajouté ultérieurement.) Plus encore que d'une maturité, je vois que c'est une certaine générosité qui m'a fait défaut dans ma vie d'enseignant jusqu'à aujourd'hui - une générosité qui s'exprime de façon plus délicate que par une disponibilité en temps et en énergie, et qui est plus essentielle. Ce manque ne s'est pas manifesté de façon visible (par une accumulation de situations d'échec disons) dans ma première période d'enseignement, sans doute surtout parce qu'il était compensé par une forte motivation en les élèves qui choisissaient de venir travailler avec moi. Dans la deuxième période par contre, de 1970 à aujourd'hui, il me semble que ce manque est pour le moins une des raisons, et celle en tous cas qui m'implique le plus directement, pour l'échec global que je constate dans mon enseignement au niveau de recherche (à partir du niveau d'un DEA donc). Voir à ce sujet "Esquisse d'un programme", par.8, et par. 9 "Bilan d'une activité enseignante", où transparaît le sentiment de frustration sur lequel m'a laissé cette activité depuis sept ou huit ans [Comparer aussi la note (23iv), rajoutée ultérieurement.].}(21). Il est vrai aussi, pourtant, que depuis ma petite enfance j'ai aimé enseigner, et que depuis les années soixante jusqu'à aujourd'hui, les élèves que j'ai pu avoir ont pris dans ma vie une place importante. C'est dire aussi que mon activité enseignante, mon rôle d'enseignant ont eu dans ma vie et y gardent une grande place \footnote{(22)\par Plus pour bien longtemps peut-être, puisque j'ai pris la décision de demander mon admission au Centre National de la Recherche Scientifique, et mettre fin ainsi à une activité enseignante en milieu universitaire, qui depuis quelques années est devenue de plus en plus problématique.}(22).

Pendant cette première période de mon activité enseignante, il n'y a pas eu de conflit apparent entre aucun de mes élèves et moi, qui se serait exprimé ne serait-ce que par un "froid" passager dans nos relations. Une seule fois, je me suis vu obligé de dire à un élève qu'il manquait de sérieux dans son travail et que ça ne m'intéressait pas de continuer avec lui si ça continuait comme ça. Il savait bien sûr tout aussi bien que moi de quoi il retournait, il s'est repris et l'incident a été clos sans laisser de nuage. Une autre fois, au début des années soixante-dix déjà, alors que le plus clair de mon énergie était engagé dans les activités du groupe "Survivre et Vivre", un élève à qui j'avais montré (comme c'est mon habitude) le rapport de thèse que je venais d'écrire sur son travail, s'est mis en colère, jugeant que certaines considérations dans ce rapport mettaient en cause la qualité de son travail (ce qui n'était nullement mon intention). Cette fois c'est moi qui ai rectifié le tir sans faire de difficulté. Il ne m'a pas semblé alors que ce court incident puisse laisser une ombre dans notre relation, mais il se peut que je me sois trompé. La relation entre cet élève et moi avait été plus impersonnelle qu'avec les autres élèves (mis à part "l'élève triste" dont j'ai parlé), une bonne relation de travail sans plus, sans une véritable chaleur qui aurait passé entre nous. Je ne pense pas pourtant que c'est un manque de bienveillance inconscient en moi qui m'aurait fait mettre dans mon rapport les considérations qu'il jugeait désavantageuses à son égard, ajoutant "qu'il n'allait pas laisser passer" la chose comme avait fait un camarade à lui, qui avait déjà passé sa thèse avec moi. Avec cet autre élève, d'un naturel sensible et affectueux, j'étais lié par une relation particulièrement amicale ; si j'avais inclus dans mon rapport sur sa thèse le même genre de considération qui avait tant déplu à son camarade, ce n'était sûrement pas par manque de bienveillance ! Par ailleurs, pour l'un et pour l'autre, comme pour tous mes élèves, je n'aurais pas donné le feu vert pour une soutenance, si je n'avais été pleinement satisfait du travail qu'ils présentaient. Aucun de mes élèves de cette période n'a d'ailleurs eu de difficulté à trouver rapidement un poste à sa mesure ; une fois sa thèse passée.

Jusqu'en l'année 1970, j'avais vis-à-vis de mes élèves une disponibilité pratiquement illimitée \footnote{(22’)\par Même après 1970, quand mon intérêt pour les maths est devenu sporadique et marginal dans ma vie, je ne crois pas qu'il y ait eu d'occasion où je me sois récusé, quand un élève faisait appel à moi pour travailler avec lui. Je peux même dire qu'à part deux ou trois cas, l'intérêt de mes élèves d'après 1970 pour le travail qu'ils faisaient était loin en deçà de mon propre intérêt pour leur sujet, même en les périodes où je ne me préoccupais guère de maths que les jours où je mettais les pieds à la Fac. Aussi le genre de disponibilité que j'avais à mes élèves d'avant 1970, et l'extrême exigence dans le travail qui en était un signe principal, n'auraient-ils eu aucun sens vis-à-vis de la plupart de mes élèves ultérieurs, qui faisaient des maths sans conviction, comme par un continuel effort qu'ils auraient dû faire sur eux-mêmes...}(22’). Quand le temps était mûr et chaque fois alors que cela pouvait être utile, je passais avec l'un ou l'autre des journées entières s'il le fallait, à travailler telles questions qui n'étaient pas au point, ou à revoir ensemble les états successifs de la rédaction de leur travail. Tel que j'ai vécu ces séances de travail, il ne me semble pas que j'y aie jamais joué le rôle de "directeur" prenant des décisions, mais que c'était chaque fois une recherche commune, où les discussions se faisaient d'égal à égal, jusqu'à satisfaction complète de l'un comme de l'autre. L'élève apportait un investissement d'énergie considérable, sans commune mesure bien sûr à celui que j'étais appelé à apporter moi-même, qui avais par contre une plus grande expérience, et parfois un flair plus exercé.

La chose cependant qui me paraît la plus essentielle pour la qualité de toute recherche, qu'elle soit intellectuelle ou autre, n'est aucunement question d'expérience. C'est l'exigence vis-à-vis de soi-même. L'exigence dont je veux parler est d'essence délicate, elle n'est pas de l'ordre d'une conformité scrupuleuse avec des normes quelles qu'elles soient, de rigueur ou autres. Elle consiste en une attention extrême à quelque chose de délicat à l'intérieur de nous-mêmes, qui échappe à toute norme et à toute mesure. Cette chose délicate, c'est l'absence ou la présence d'une compréhension de la chose examinée. Plus exactement, l'attention dont je veux parler est une attention à la qualité de compréhension présente à chaque moment, depuis la cacophonie d'un empilement hétéroclite de notions et d'énoncés (hypothétiques ou connus), jusqu'à la satisfaction totale, l'harmonie achevée d'une compréhension parfaite. La profondeur d'une recherche, que son aboutissement soit une compréhension fragmentaire ou totale, est dans la qualité de cette attention. Une telle attention n'apparaît pas comme résultat d'un précepte qu'on suivrait, d'une intention délibérée de "faire gaffe", d'être attentif - elle naît spontanément, il me semble, de la passion de connaître, elle est un des signes qui distinguent la pulsion de connaissance de ses contrefaçons égotiques. Cette attention est aussi parfois appelée "rigueur". C'est une rigueur intérieure, indépendante des canons de rigueur qui peuvent prévaloir à un moment déterminé dans une discipline (disons) déterminée. Si dans ce livre je me permets de prendre des libertés avec des canons de rigueur (que j'ai enseignés et qui ont leur raison d'être et leur utilité), je ne crois pas que cette rigueur plus essentielle y soit moindre que dans mes publications passées, en style canonique. Et si j'ai pu, peut-être, malgré tout, transmettre à mes élèves quelque chose d'un plus grand prix qu'un langage et un savoir-faire, c'est sans doute cette exigence, cette attention, cette rigueur - sinon dans la relation à autrui et à soi-mêmes (alors qu'à ce niveau elle me faisait défaut autant qu'à quiconque), du moins dans le travail mathématique \footnote{(23) L’Enfant et le maître \par Le terme "transmettre" ici ne correspond pas vraiment à la réalité des choses, qui me rappelle à une attitude plus modeste. Cette rigueur n'est pas une chose qu'on puisse transmettre, mais tout au plus réveiller ou encourager, alors qu'elle est ignorée ou découragée depuis le plus jeune âge, par l'entourage familial aussi bien que par l'école et l'université. Aussi loin que je puisse me rappeler, cette rigueur a été présente dans mes quêtes, celles de nature intellectuelle tout au moins, et je ne pense pas qu'elle m'ait été transmise par mes parents, et encore moins par des maîtres, à l'école ou parmi mes aînés mathématiciens. Elle me sembler faire partie des attributs de l'innocence, et par là, des choses qui sont dévolues à chacun à la naissance. Cette innocence très tôt "en voit des vertes et des pas mûres", qui font qu'elle est obligée de plonger plus ou moins profond, et que souvent il n'en apparaît plus guère trace dans le restant de la vie. Chez moi, pour des raisons que je n'ai pas songé encore à sonder, une certaine innocence a survécu au niveau relativement anodin de la curiosité intellectuelle, alors que partout ailleurs elle a plongé profond, ni vu, ni connu ! comme chez tout le monde. Peut-être le secret, ou plutôt le mystère, de "l'enseignement" au plein sens du terme, est de retrouver le contact avec cette innocence en apparence disparue. Mais il n'est pas question de retrouver ce contact en l'élève, s'il n'est déjà d'abord présent ou retrouvé dans la personne de l'enseignant lui-même. Et ce qui est "transmis" alors par l'enseignant à l'élève n'est nullement cette rigueur ou cette innocence (innées en l'un et l'autre), mais un respect, une revalorisation tacite pour cette chose communément rejetée.}(23). C'est là, certes, une chose bien modeste, mais peut-être, malgré tout, mieux que rien.





