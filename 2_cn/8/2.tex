\section{(26) 严谨与严谨}

如果我把那个学生除外——他肯定不比其他学生“天赋”差——我可以说,我与学生们的关系是友好的,甚至常常是深情的。迫于形势,他们都学会了对我的两个主要“老板”缺陷保持耐心:一是我的字迹难以辨认(不过我想他们最终都学会了破译),二是更严重的事(我很久以后才意识到),我天生难以跟随他人的思路,除非先将其转化为我自己的意象,并以我自己的风格重新思考。我更倾向于向学生们传递我自己深深浸润其中的某种事物视野,而不是鼓励他们萌发或许与我颇不相同的个人视野。这种与学生关系的困难至今未消,但我觉其影响已减弱,因为我意识到了自己这种倾向。或许我的性情——无论是天生还是后天——使我更适合独自工作,如我在数学活动头十五年(约1945至1960)所做的那样,而不适合扮演“导师”角色,与那些数学志向和个性尚未完全成型的学生接触\footnote{(21)\par 更准确地说,对于我这样的性情,我还缺乏充分承担教师角色的必要成熟。我后天形成的性情长期以来过度偏重“阳”性特质,而成熟的一个方面恰恰是“阴阳”平衡,以“阴”为主。

(后补) 我发现,较之成熟,我在迄今的教师生涯中更缺乏一种慷慨——一种比时间和精力上的可用性更精微、更本质的慷慨。这种缺乏在我的首期教学中并未明显表现(比如说,未累积失败情境),无疑主要是因为选择与我共事的学生的强烈动机弥补了这一点。反之,在第二期,从1970年至今,我认为这种缺乏至少是原因之一,且是最直接涉及我的原因,导致我在研究级教学(从DEA水平起)中的整体失败。关于这一点,见《计划纲要》第8段和第9段《教学活动总结》,其中透露出近七八年来这种活动给我留下的挫败感[另见后补注(23iv)]。}(21)。然而,确实,自幼我便喜爱教学,且自六十年代至今,我所教过的学生在我生命中占据了重要位置。这也说明,我的教学活动和教师角色在我的生命中曾有并仍占有一席重要之地\footnote{(22)\par 或许不会太久了,因为我已决定申请加入国家科学研究中心「Centre National de la Recherche Scientifique;National Center for Scientific Research」,从而结束在大学环境中愈发棘手的教学活动。}(22)。

在我教学活动的首期,我与任一学生间都未有明显冲突,连短暂的“冷淡”也未曾在我们的关系中出现过。仅有一次,我不得不对一名学生说,他的工作缺乏认真,如果继续这样,我不感兴趣与他继续合作。他当然和我一样清楚问题所在,他改正了,事件就此了结,没有留下任何阴影。另一次,已经是七十年代初,当时我大部分精力都投入到“生存与生活”「Survivre et Vivre;Survive and Live」小组的活动中,我向一名学生展示了我为他的工作写的论文报告(这是我的习惯),他却生气了,认为报告中的某些评论质疑了他工作的质量(这绝非我的本意)。这次我毫不费力地调整了方向。当时我并不觉得这短暂的插曲会在我们的关系中留下阴影,但我可能错了。我与这名学生的关系比与其他学生(除了我提到的“悲伤学生”)更缺乏个人情感,只是一份良好的工作关系,我们之间缺乏那种真正的温暖。然而,我不认为是我无意识的缺乏善意导致我在报告中写入他认为对其不利的评论,他还说“他不会像他的一个已经在我指导下通过论文的同学那样容忍此事”。与那位天性敏感且深情的另一名学生,我有着特别友好的关系;如果我在他的论文报告中纳入了同样令他的同学如此不悦的评论,那绝不是因为缺乏善意!此外,对他们两位以及所有我的学生,如果我对他们呈现的工作不完全满意,我是不会同意他们进行答辩的。顺便说一句,我那个时期的学生在论文通过后,都没有遇到困难,很快就找到了适合自己的职位。

直到1970年,我对我的学生几乎是无限制地可用\footnote{(22’)\par 即使在1970年后,当我对数学的兴趣在生活中变得时有时无且边缘化时,我不记得有拒绝过学生请求与我共事的场合。我甚至可以说,除了两三个例外,1970年后的学生对他们工作的兴趣远不如我对他们题目的兴趣,即使在我几乎只在进学院的日子才关注数学的时期。因此,我对1970年前学生的那种可用性和工作中的极高要求——这是主要标志——对大多数后来的学生毫无意义,他们做数学缺乏信念,如同持续的自我勉强……}(22’)。当时机成熟且每次在可能有用时,我会与他们中的某一位整日共事,如有必要,处理那些尚未完善的特定问题,或一起审阅他们工作的连续草稿。就我所经历的这些工作会晤而言,我不觉得自己曾扮演过“指导者”的角色,做出决策,而是每次都是一次共同的研究,讨论在平等的基础上进行,直到双方都完全满意。学生投入了巨大的精力,当然,这与我需要投入的精力不可同日而语,我拥有更多的经验,有时还有更敏锐的直觉。

然而,我认为对任何研究——无论是智力上的还是其他——的质量最为关键的,绝非经验问题,而是对自己的要求。我所说的这种要求本质上是很微妙的,它不属于对任何规范——无论是严谨性还是其他——的严格遵守。它在于对我们内在某种微妙事物的极度关注,这种事物超脱于所有规范和衡量。这种微妙的事物,就是对所考察事物的理解的有无。更确切地说,我所说的关注是对每一刻理解质量的关注,从杂乱堆砌的概念和命题(无论是假设的还是已知的)的嘈杂,到完全满意、理解完美的和谐。研究的深度,无论其结果是片面的理解还是全面的理解,都在于这种关注的质量。这种关注并非源自遵循某条准则、刻意“小心翼翼”或保持警觉——我认为,它自发地诞生于求知的热情,它是区分真正的求知欲与其自我中心的伪装的标志之一。这种关注有时也被称为“严谨”。这是一种内在的严谨,独立于某一特定时刻在某一(比如说)特定学科中可能盛行的严谨标准。如果在本书中,我允许自己对严谨标准(我曾教授过且它们有其存在的理由和用处)有所放宽,我不认为这种更本质的严谨会比我过去规范风格的出版物中更少。而且,如果我或许尽管如此,还是向我的学生们传递了比语言和技艺更有价值的东西,那无疑就是这种要求、这种关注、这种严谨——即使在与他人的关系和对自己的态度上(在这一点上,我和任何人一样都缺乏这种严谨),至少在数学工作中是如此\footnote{(23) 孩子与导师 \par 此处的“传递”一词并不真正符合事物的现实,这提醒我采取更谦逊的态度。这种严谨并非可以传递之物,而至多是唤醒或鼓励之物,尽管它自幼便被家庭、学校和大学所忽视或抑制。据我所忆,这种严谨始终存在于我的探索中,至少在智力性质的探索中,我不认为它是由父母传递给我的,更不用说学校或数学前辈导师了。它似是纯真的属性之一,因此是每个人生来就拥有的。这种纯真很早就“经历风霜”,迫使它或多或少深藏,以致在余生中常难觅其踪迹。在我身上,出于我尚未探寻的原因,某种纯真在相对无害的智力好奇层面幸存,而在其他方面则如常人般深藏不露!或许“教学”一词全义的秘密或奥秘,在于重连这看似消失的纯真。但若教师自身未先拥有或重获此联系,则无从在学生中寻回。而教师“传递”给学生的,并非这种严谨或纯真(两者皆天生具备),而是对这常被摒弃之物的尊重和默然的重估。}(23)。这确实是一件很谦逊的事情,但或许,尽管如此,也比什么都没有要好。