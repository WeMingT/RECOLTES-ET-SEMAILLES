\section{(29) 敌对的父亲 (1)}

1970年转折后开始与我合作的学生,在一个完全不同的省立大学环境中,与之前的学生截然不同。只有两位学生与我合作完成了国家博士论文。其他人的工作则停留在DEA或第三阶段博士论文的水平。我还应该包括许多学生,他们对某些``研究入门''课程产生了浓厚的兴趣,这些课程使他们有机会提出意想不到的数学问题,有时还能想出原创的方法来解决这些问题。我在为一年级学生开设的某些``选修课程''中遇到了最积极的参与。对于那些在大学环境中坚持了几年的学生来说,他们的某种新鲜感、兴趣能力和个人视野已经或多或少地消失了。在选修课程的学生中,有几位显然有成为优秀数学家的潜质。鉴于当时的形势,我没有鼓励任何人走这条路,尽管这条路可能会吸引他们,而且他们本可以在其中表现出色。

对于那些为了准备硕士文凭而参加我的``课程''的学生来说,我们的关系通常在一年后就结束了。每次,我都觉得我们的关系很快就变得友好和轻松,除了一个被``怯场''困扰的学生\footnote{(23”) 表演的恐惧 \par 这位学生在整个学年里与我合作完成了DEA的``实习工作'',但在与我的工作关系中一直保持着``紧张''状态。这是一种坦诚友好的关系,充满了毋庸置疑的相互同情。然而,确实存在这种``怯场'',这种恐惧,其真正原因并不是对我本人的恐惧,尽管表面上看起来是这样。如果这位学生没有自己告诉我,我可能甚至不会注意到这一点,他这样做无疑是为了``解释''他在学年工作中持续的完全停滞的原因。 \par 就像其他学生一样,他在开始时对某种几何实质产生了浓厚的兴趣,但当需要进行``实际工作''时,即把陈述写成正式的形式,或者仅仅理解我建议作为语言基础的这些陈述的意义和``游戏规则''时,停滞就出现了。面对被要求``进行研究''的情况,学生几乎总是会采用由大师传授的一个或多个隐含的``游戏角度''作为``给定'',并且绝不试图明确或理解这些规则。这些隐含规则的具体形式是语义或计算的``配方'',模仿经典的数学书籍(或任何其他常见的教学书籍)。学生期望从大师那里得到一个``证明……''的列表,这是他在过去经验中遇到的唯一形式的``数学思考''。(我认为大多数专业数学家和其他科学家的倾向本质上并没有什么不同——只是``大师''被``共识''所取代,后者确定了当前的游戏规则并将其视为不可改变的给定。这个共识还确定了需要解决的``问题'',每个人都可以根据自己的喜好选择,甚至在工作中修改它们,甚至发明新的问题……)。我发现,我对需要探索的数学实质的态度,以及对学生的态度,与众不同,几乎肯定会引发困惑,其中一个迹象就是焦虑。像所有焦虑一样,这种焦虑往往会投射到一个面孔上,投射到一个看似合理的外在``原因''上。焦虑最常见的一个面孔正是恐惧。 \par 在我教学活动的早期,这种困难很少出现,除了也许在两三个``师生''关系中,这种关系没有持续超过几个星期,也许(我不能确定)在``狡猾的学生''的情况下,他们只关心``做好''他们根本不感兴趣的事情,尽管他们完全有自由改变。在我温和地提到的那个学生的情况下,他长期以来一直被某种怯场所困扰,原因显然在别处。他在工作中并没有被阻挡,相反,他对所选择的主题非常自在,并对其基础进行了非常敏锐的工作。事实上,我那个时期的大多数学生都是高等师范学校「École Normale;Normal School」的前学生,他们与亨利·卡尔坦「Henri Cartan;Henri Cartan」的接触已经向他们展示了一种不同的``方法''来处理数学。在我作为教师的第二阶段,在蒙彼利埃大学「Université de Montpellier;University of Montpellier」,在一年级学生中,我提到的焦虑对反思工作的干扰最少。在这些学生中,许多人对不同方法的惊讶并没有引发焦虑或封闭,相反,他们开放并渴望做一些有趣的事情!根据我的观察,大学几年的学习对学生的倾向产生了彻底而毁灭性的影响。奇怪的是,在这方面,漫长的中学岁月的影响似乎相对较小。原因可能是}(23''), 对于那些在我的指导下正式准备研究工作的学生来说,情况也是如此,无论是在哪个层面。与我之前的学生相比(在许多其他方面!),一个区别是我们的关系并不局限于共同的数学工作。学生和我之间的交流往往更深入地涉及我们的个人 \footnote{(23v) \par 这种差异的一个特别显著的迹象是在``外国人事件''中表现出来的,我有机会在(第24节)中提到过。当时,我收到了许多完全陌生的人的同情表示,但我记得1970年之前的任何一个学生都没有想过以这种方式表现自己,更不用说在我所从事的行动中提供任何帮助。相反,我认为我的第二阶段的学生或前学生中没有一个不向我表达同情和声援的,许多人积极参与了我在当地开展的运动。除了这个小圈子,1945年的法令事件还在许多只知道我名字的学院学生中引起了一定的情绪,许多人在我被传唤的那天来到法院,表达他们的声援。最后一种情况表明,我在1970年``之前''和``之后''的学生态度上的差异可能更多地反映了心态的差异,而不是我们之间关系的差异。显然,我``之前''的学生已经成为重要人物,而重要人物很难被感动……但我在1970年离开高等科学研究所「IHES;IHES」并参与激进行动的事件似乎表明,不仅仅是这样。那时,他们中没有一个人真正成为重要人物,然而我不记得他们中的任何一个对我所从事的活动表现出丝毫的兴趣。我认为,这种活动肯定让他们所有人感到不舒服。这仍然表明心态的差异,但不能仅仅归因于社会地位的差异。}(23v)。因此,在我教学活动的第二阶段,与某些学生的关系中的冲突元素以更清晰、更直接、甚至更激烈的方式出现,这并不奇怪。在我第一阶段的前学生中,有两个人在后来表现出系统和明确的敌对态度(我曾有机会顺便提到过),但这些态度仍然停留在未说出口的层面,甚至可能是无意识的。在更长的第二阶段,有三个学生对我表现出敌意。其中两个人的敌意表现得非常尖锐。

在其中一个学生身上,敌意在一夜之间出现在一个曾经非常友好的关系中,那是在这位朋友不再是我的学生多年之后。我怀疑冲突的原因与其说是我不可告人的行为和个性,不如说是长期压抑的不满,因为他的工作(非常出色)没有得到他理应期待的认可。这是``1970年后''让我作为导师的可疑特权的反面,他一定对我怀有怨恨,尽管他自己内心并没有完全承认这一点。

在另一个学生身上,尖锐的敌意在一年半的工作后就出现了,而当时的气氛似乎非常友好。这是我第一次也是唯一一次在学生还处于学生身份时,遇到与学生之间的关系困难。这使得继续共同工作变得不可能,尽管工作在一开始时看起来前景光明,充满了最有希望的热情,针对一个壮丽的反思主题,我必须说。我感觉到这位年轻的研究员对自己做好工作的能力有一种隐秘的缺乏信心(这种能力对我来说是毫无疑问的),而敌意的尖锐表现是一种``逃避'',以抢先面对所恐惧的失败,并将责任预先推到一个可憎的导师身上 \footnote{(23''') \par 两兄弟。这个学生的敌意从一开始就采取了``阶级敌对''的形式:我是``老板'',对他的数学未来拥有``生杀大权'',我可以随心所欲地决定……当然,事件只能证实这种看法,因为我不久就结束了对我这个学生(变得痛苦的)责任。这使他处于一个微妙的境地,在当前的时代,找到一个``老板''并不容易,尤其是当主题已经选定的时候。在另一个学生身上,他对自己的合理期望感到沮丧,敌意采取了类似的形式。我被视为专横的``mandarin'',不能容忍他所认为的从属者(学生或低级同事)的反对。}。

在第一阶段与我的学生的关系中,从未表现出丝毫的``阶级态度''。显而易见的原因是,在1970年之前的形势下,毫无疑问,学生一旦通过论文,就会获得讲师的职位,因此享有与我相同的社会地位,即``大学教授''。数字很能说明问题:1970年之前开始与我合作的11名学生在完成工作后立即获得了讲师职位,而在我指导下或多或少工作的约20名学生中,没有一个获得过这样的职位。确实,他们中只有两人有足够的动力去完成国家博士论文(而且两人都做得非常出色)。

因此,在第二阶段,某些矛盾心理(其深层根源仍然隐秘)以阶级敌对的形式出现,对``老板''的不信任(被呈现和感受到为``本能的'')并不奇怪。对于那些或多或少扮演学生角色的人来说,友好的关系持续了大约十年,没有明显的敌对事件,但仍然带有同样的模糊性,通过一种``保留''的不信任态度来表达,隐藏在明显的同情后面。说实话,我从未被这种``命令式''的不信任所欺骗,在我看来,这主要是这位朋友认为有必要给自己一个理由,以不冒险走出他所选择的明确领域,无论是职业生活还是个人生活——尽管他有自由这样做,而没有人(除了他自己!)要求他解释……

这三个案例是我在整个教学经验中,唯一的学生(或那些或多或少扮演学生角色的人)与我之间的关系中的矛盾心理通过``阶级态度''表达出来的情况。当这种态度在大学``群体''内的同事之间表现出来时,特别具有模糊性,他们都享有与普通人相比过分的特权,这些特权使得等级(和薪水)差异相对微不足道。我还注意到,一旦当事人自己被提升到他前一天还在指责他人的地位,这种态度就会如魔法般消失(原因很充分!)。

我在大多数(如果不是全部)数学世界内部(以及外部)目睹的冲突情况下都发现了类似的模糊性。那些``安顿好''的人,无论他们的等级是否符合他们的期望(无论是否合理),都享有惊人的特权,这是其他职业或职业生涯无法提供的。那些``未安顿''的人渴望同样的安全和特权(这并不一定阻止他们对数学本身感兴趣,有时还会做出美丽的事情)。在竞争激烈的时代,未安顿的人经常被当作落魄者对待:我不止一次感觉到,喜欢羞辱的人和被羞辱的人之间的默契——后者吞咽并压抑自己。他的痛苦和敌意的真正对象不是那个使用权力的人,而是他自己,他压抑了自己,并将权力赋予了他人,后者愉快地使用它。喜欢羞辱的人也是那个寻求报复并补偿(但从未抹去……)长期遭受的羞辱的人,这种羞辱早已被埋藏和遗忘。而那个默认自己被羞辱的人是他的兄弟和模仿者,他暗中嫉妒,并在痛苦中埋藏了羞辱,以及它带给他的关于自己的谦卑信息。

在我教授数学家职业近25年来,学生与我之间的所有冲突出现的一个共同方面是强烈的矛盾心理。在所有这些案例中,毫无例外地,敌意是在事后出现的,往往是隐秘的,在一种无可置疑的同情关系中。我甚至可以说,在所有这些案例中,以及在许多其他没有表现出明显的敌对成分的案例中,我的个人魅力一直很强,而且仍然很强。无疑,正是这种吸引力的强度也滋养了敌意的强度,并确保了其持续性。情况仍然如此,无疑,在敌意采取激烈反感、愤怒拒绝的形式的情况下;以及在另一个极端相反的情况下,在友好的尊重这一严格旗帜下,(当机会合适时)表达了一种故作轻蔑的态度,小心翼翼地加以调配……

说实话,这种矛盾心理的情况并不局限于我与某些学生或前学生的关系。事实上,自从我30岁(即我母亲去世后)以来,它们在我的整个成人生活中比比皆是。无论是在我的情感或婚姻生活中,还是在我与男性的关系中,特别是与那些比我年轻得多的男性关系中,都是如此。我最终意识到,我身上有些东西,无论是天生的还是后天获得的,我不太确定,似乎使我倾向于扮演父亲的角色。我想,我有理想的体格和有利的共鸣,使我成为完美的养父!必须说,父亲的角色对我来说就像手套一样合适——仿佛我生来就是这样。我不会试图计算我对另一个人扮演这个角色的次数,我们之间有完美的默契。通常,这种父子或父女的角色分配是未说出口的,甚至是无意识的,但有时也会以或多或少明确的方式表达出来。在某些情况下,我甚至在完全不知道发生了什么的情况下,无论是自觉还是无意识地,就扮演了父亲的角色。

我在1972年“生存与生活”「Survivre et Vivre;Survive and Live」时期,第一次意识到自己扮演了养父的角色,当时我突然面对一位年轻朋友的激烈拒绝态度。(有趣的巧合,他是一个脱离常规的数学学生!)我对第三方的行为中的某些东西让他失望了。我想,我会毫不困难地承认他的失望是有道理的,在这种情况下我缺乏慷慨——但反应的激烈程度当时真的让我震惊。这就像一场突然爆发的激烈仇恨,几乎立即平息下来,当他显然没有真正成功让我动摇时。(差一点,但我把这一点留给自己……)。我不知道当时我是如何直觉到他将与父亲未解决的冲突投射到我这个被理想化的人身上的。这个突然的直觉,虽然被遗忘,并没有阻止我在接下来的几年里继续以同样的信念扮演父亲的角色,丝毫没有戒备。当然,每次当我随后面对冲突的迹象,无论是隐秘的还是激烈的,我总是感到同样的痛苦惊愕,不相信我的眼睛或其他。

在对我父母的生活进行了六七个月的强烈独处工作后,我以一种未曾预料的光芒看到了他们的人格,我明白了养父母角色的虚幻之处,这个角色(当然更好,这是事先约定的!)取代了一个真实存在的父母,后者被宣布(哪怕只是通过默契)为“缺失”。这是在帮助他人逃避冲突所在的地方,比如他与父亲的关系,将其投射到一个完全无关的第三者(在这种情况下是我)身上。自从1979年8月至1980年3月的冥想以来,我对自己保持警惕,不再闭着眼睛沉迷于我不幸的父性使命。这并没有阻止错误的情况再次发生(比如在我与那个我不得不停止工作的学生的关系中)——但现在,我相信,没有我的默许。

如果我把那个在合理期望中受挫的学生的情况放在一边,对我来说,毫无疑问,在所有其他我面对学生或前学生的敌意的情况下,这都是对父亲冲突原型的再现:父亲既被仰慕又被恐惧,被爱又被恨——那个需要面对、战胜、取代、也许羞辱的人……但也是那个暗中想成为的人,剥夺他的力量据为己有——另一个自我,被恐惧、被恨、被逃避……