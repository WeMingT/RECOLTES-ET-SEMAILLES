

L'éthique dont me parlait Dieudonné en termes tout ce qu'il y a de terre à terre, est morte en tant qu'éthique d'un certain milieu. Ou plutôt, ce milieu lui-même est mort en même temps que cette probité qui en faisait l'âme. Cette probité s'est conservée en certaines personnes isolées, et elle est réapparue ou réapparaîtra dans certaines autres où elle s'était dégradée. Son apparition ou sa disparition dans tel d'entre nous fait partie des épisodes cruciaux de l'aventure spirituelle de l'un et de l'autre. Mais la scène sur laquelle se déroule cette aventure est profondément transformée. Un milieu qui m'avait accueilli, que j'avais fait mien, dont j'étais secrètement fier, n'est plus. Ce qui faisait son prix est mort en moi-même, ou du moins s'est vu envahi et supplanté par des forces d'une autre nature, bien avant que l'éthique tacite qui le réglait se trouve ouvertement reniée dans les usages comme dans les professions de foi. Si j'ai pu depuis m'étonner et m'offusquer, c'était par ignorance délibérée. Ce qui m'est revenu de ce milieu qui fut mien avait un message à m'apporter sur moi-même, qu'il m'a plu d'éluder jusqu'à aujourd'hui.


