\section{(32) 数学家的伦理}

昨天我记录的那个案例,现在我终于费心将其写下,显得意义重大,从某些方面看,比之前提到的另外三个(无疑也具典型性的)案例影响更深。在那些案例中,自负的力量深深扰乱了我自然的善意与尊重态度。这一次,我利用了一个非常真实的权力地位(尽管我像所有人一样假装无视这种权力),用它来劝阻一位善意的学者,并拒绝了一份值得发表的工作。这就是所谓的滥用权力。这种行为即便不触犯刑法条款,也同样明显。当时的形势比现在宽松,这让我感到庆幸,我相信那位学者并未因此受太大阻碍,依靠一些比我更友善的同事的支持发表了他的工作,他的数学家生涯并未因我的不当行为受到严重干扰,更不用说被毁。我事后为此感到欣慰,但并不想以此作为“减轻罪责的情节”。在更苛刻的环境中,我可能会更谨慎——但这只是假设,与此无关。我仍相信,我心中并无隐秘的恶意或因所述恼怒而生的伤害意图。我对这种恼怒的反应是“本能的”,毫无自我批判的意愿,更不用说稍稍审视我内心的变化,或我的反应可能对他人的生活产生的影响。我并未衡量自己拥有的权力,也从未在这种关系中想到与之相伴的责任(哪怕只是鼓励或劝阻的权力)。这是一个典型的失责行为案例,如同街头巷尾随处可见,无论在科学界还是其他地方。

我记得的这类唯一案例可能是极端情况之一,与其他类似情况并存。触发这种无善意态度的是虚荣的恼怒,它迫不及待地看到“随便谁”擅自闯入禁猎区,抢夺只应归于此地主人的一些小猎物……这种恼怒自有现成的理性化解释,显得更高尚,这不难猜想。并非我这微不足道的人受到威胁,而是对艺术和数学的热爱,这个年轻人连天才的借口都没有,反而笨拙不堪,他会毁了一切,我们可怎么办,若他能比我做得更好还好说,但我预想的美好秩序全被他抛诸脑后,真是厚颜无耻……!贯穿始终的潜台词是精英主义的主题:只有最优秀者(如我)才有资格进入我的领域,或者那些置身于这些优秀者庇护之下的人!(至于不太常见的情况,即另一位大人物闯入我的地盘,那是另一回事——每日烦恼已够多矣!)在这个具体案例中,我几乎不再怀疑,还有另一种完全无意识的力量朝同一方向作用,这在我与那位不懈朋友的早期关系中已强烈显现:对某种不符合我从母亲那里继承的“男子气概”标准的人的自动排斥。但这一情况虽对我理解自身有意义和兴趣,对我当前的主旨——即在过去身处某个圈子时的态度与行为中寻找今日我所见的深刻退化迹象——却相对无关。

这个案例之所以比其他我缺乏善意与尊重的案例意义更大,是因为它违反了数学家职业中某种基本的伦理 \footnote{(24) \par 我所说的伦理同样适用于围绕研究活动形成的任何其他圈子,在这些圈子中,能否公布成果并获得认可,对每个成员的社会地位乃至作为圈子成员的“生存”至关重要,这对其本人及其家庭都有深远影响。}(24)。在我职业起步时接纳我的圈子,即布尔巴基「Bourbaki;Bourbaki」及其周边,这个伦理通常是隐性的,但依然存在,鲜活且(我认为)受一种无形的共识支持。据我回忆,唯一明确向我表述这一伦理的人是迪厄多内「Dieudonné;Dieudonné」,可能是在我首次作为客人拜访他于南锡「Nancy;Nancy」时。他可能在其他场合也提到过。显然他认为这很重要,我也一定感受到他对此的重视,三十五年后我仍记得此事。仅凭我前辈群体的道德权威,尤其是当时明显表达群体共识的迪厄多内的权威,我默默接受了这一伦理,却从未反思片刻,也未理解其重要性。老实说,我甚至没想过有必要反思,早就确信我的父母和我自己各是伦理、负责且无可挑剔(或几近如此)的化身 \footnote{(25) 职业共识与信息控制 \par 除了与迪厄多内的对话,我不记得在我的数学家生涯中参与或见证过任何讨论职业伦理或同行间“游戏规则”的对话。(这里我排除七十年代初围绕“生存与生活”「Survivre et Vivre;Survive and Live」运动关于科学家与军事机构合作的讨论,这些讨论与数学家之间的关系无关。我在该运动中的许多朋友,包括谢瓦利「Chevalley;Chevalley」和盖德「Guedj;Guedj」,都感到我当时——尤其早期——对这一我特别敏感的问题的强调,使我疏远了更基本的日常现实,正如我在此反思中探究的那样。)我和学生之间从未讨论过这些。隐性共识似乎仅限于一条规则:不得将得知的他人想法据为己有。我认为,这一共识自古就有,且至今未在任何科学圈子中受到挑战。但若无另一条补充规则——保障每个学者公布其想法与成果的机会——这条规则形同虚设。如今在科学界,享有声望与权力的人对科学信息有随意控制权。在我熟悉的圈子中,这种控制不再受迪厄多内所述共识的约束,或许这种共识仅存在于他所代表的小群体中。有权者几乎接收所有他认为有用的信息(往往还超出其需),并有权阻止其中大部分信息的发表,同时保留其益处,将其斥为“无趣”“或多或少已知”“琐碎”等……我在注释(27)中再谈此情况。}(25)。

迪厄多内并未对我长篇大论——这既非他的风格,也非布尔巴基中任何朋友的风格。他可能只是顺带一提,当作理所当然的事。他仅强调了一条看似平凡却最简单的规则:任何发现值得关注成果的人都有权且应有机会发表,唯一条件是该成果尚未发表。因此,即便一或多人已知该成果,只要他们未费心将其写下并发表,使之服务于(嗯!)“数学共同体”,任何通过自身努力发现该成果的人(言下之意:包括那所谓的“随便谁”!)都应能依其方式与视角发表(言下之意:无论其方式与视角如何,无论在更知情者眼中是否“狭隘”……)。我记得迪厄多内补充说,若不遵守此规则,将为最恶劣的滥用敞开大门——可能正是在他口中,我首次得知高斯以雅可比的想法早已为他所知为由拒绝其工作的历史案例。

这条简单规则是对迪厄多内(及其他布尔巴基成员)及我自身“精英主义”态度的关键矫正。遵守此规则是诚实的保障。我很高兴据我今日所知,布尔巴基初始群体中的每个人都保持了这种基本诚实 \footnote{(26) \par 布尔巴基的“创始成员”是亨利·卡尔坦「Henri Cartan;Henri Cartan」、克洛德·谢瓦利「Claude Chevalley;Claude Chevalley」、让·德尔萨特「Jean Delsarte;Jean Delsarte」、让·迪厄多内「Jean Dieudonné;Jean Dieudonné」、安德烈·韦伊「André Weil;André Weil」。除德尔萨特在五十年代早逝——当时职业伦理仍普遍受尊重——外,他们都健在。

重读文本时,我曾想删除这段,因我可能给人颁发“诚实证书”(或非诚实)的印象,而相关人士对此无所谓,我亦无权如此。此段引起的保留或许有理。但为证言的真实性,我保留了它,因它确实反映了我的感受,即便这些感受不妥。}(26)。我注意到,其他加入布尔巴基或其圈子的人并非如此。我自身也未能保持这种诚实。

迪厄多以平实语言向我述说的伦理,作为某个圈子的伦理已死。或者更准确地说,那个圈子连同作为其灵魂的诚实一同消亡。这种诚实保留在某些孤立个体中,并在某些一度退化的人身上重现或将重现。其在某人身上的出现或消失,是我们各自精神冒险的关键篇章。但这场冒险的舞台已深刻改变。曾接纳我、被我视为己有、令我暗自骄傲的圈子不复存在。其珍贵之处在我心中已死,或至少被另一种性质的力量侵占并取代,远在我所熟悉的隐性伦理在惯例与信仰中被公开否定之前。若我后来感到惊讶与愤怒,那是有意无视的结果。从这个曾属我的圈子传回的信息,带来关于我自身的启示,我却一直选择回避至今。