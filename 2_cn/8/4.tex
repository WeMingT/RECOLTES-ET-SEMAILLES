\section{(28) La récolte inachevée}

Jusqu'au moment du premier "réveil", en 1970, les relations à mes élèves, tout comme ma relation à mon propre travail, était une source de satisfaction et de joie, un des fondements tangibles, irrécusables d'un sentiment d'harmonie dans ma vie, qui continuait à lui donner un sens, alors qu'une destruction insaisissable sévissait dans ma vie familiale. À cette époque, il n'y avait à mes yeux aucun élément de conflit apparent dans ces relations, dont aucune n'a été alors, à aucun moment même fugitif, cause d'une frustration ou d'une peine. C'est une chose qui peut paraître paradoxale, que le conflit dans la relation à tel de mes élèves ne soit devenu apparent qu'après ce fameux réveil, après un tournant donc qui donnait à ma vie une ouverture qu'elle n'avait pas connue avant, et à ma personne un petit début de souplesse peut-être - des qualités donc qui, pourrait-on penser, devraient être de nature à résoudre ou à éviter le conflit, et non à le provoquer ou à l'exacerber.

En y regardant de plus près pourtant, je vois bien que le paradoxe n'est qu'apparent, et qu'il disparaît, sous quelque angle qu'on le regarde. Le premier qui me vient : pour qu'un conflit ait une chance de se résoudre, il faut tout d'abord qu'il se soit manifesté. Le stade du conflit manifesté représente un mûrissement par rapport à celui du conflit caché ou ignoré, dont par ailleurs les manifestations existent bel et bien, et sont d'autant plus "efficaces" que le conflit qui s'exprime par elles reste ignoré. Aussi : pour qu'un conflit puisse se manifester de façon reconnaissable, il faut d'abord qu'une distance se soit réduite ou ait disparu. Les changements qui se sont faits dans ma vie depuis bientôt quinze ans, au cours de "réveils" successifs notamment, ont tous été des changements, il me semble, de nature à réduire une distance, à effacer un isolement. Un conflit qui a du mal à s'exprimer vis-à-vis d'un patron prestigieux, admiré, en prend plus à son aise vis-à-vis de quelqu'un dépouillé d'une position de pouvoir (volontairement en l'occurrence), qui s'est exilé d'un certain milieu détenteur d'autorité et de prestige, qui de moins en moins est perçu comme une incarnation ou un représentant privilégié de quelque entité (telle la mathématique), et de plus en plus comme une personne comme les autres : une personne non seulement susceptible d'être atteinte, mais qui, de plus, est de moins en moins encline à se cacher de blessures ou de peines. Et en troisième lieu et surtout : l'évolution qui a été la mienne depuis le premier réveil, surtout à cette époque-là et dans les années qui ont suivi, était de nature à susciter (ou à réveiller peut-être) des questions, une inquiétude, une "remise en question" dans l'univers bien ordonné de mes anciens élèves. J'ai eu ample occasion de me rendre compte qu'il en a été ainsi non seulement pour ceux-ci, mais aussi parmi mes amis et compagnons d'antan dans le monde mathématique, et parfois même parmi des collègues scientifiques qui ne me connaissent que par ouï-dire.

Il faut dire aussi que la résolution d'un conflit tant soit peu profond est une chose des plus rares. Le plus souvent, nonobstant toutes trêves et réconciliations de surface, le cortège grandissant de nos conflits nous suit sans guère nous quitter d'une semelle pendant la vie entière, pour ne nous lâcher finalement qu'entre les mains maussades des croquemorts. Il m'a été donné parfois de voir un conflit se dénouer tant soit peu, et parfois même le voir se résoudre en connaissance - mais jusqu'à présent une telle chose ne s'est pas produite au cours et à l'occasion de ma relation à un de mes élèves, ou à un de mes amis d'antan dans le monde mathématique. Et je sais bien aussi qu'il n'est nullement sûr qu'une telle chose se produise jamais, même si je devais vivre encore cent ans.

C'est une chose remarquable que le moment même de ma rupture avec un certain passé, je veux dire l'épisode de mon départ de l' IHES (de l'institution donc qui représentait un peu comme la "matrice" du microcosme mathématique qui s'était formé autour de moi) - que cet épisode décisif ait été en même temps la première occasion où un antagonisme profond d'un de mes élèves à mon égard s'est exprimé. C'est cette circonstance sûrement qui a rendu cet épisode particulièrement pénible, particulièrement douloureux, comme un accouchement ou une naissance qui se seraient faits dans des conditions particulièrement difficiles. Bien sûr, je ne pouvais alors voir cet épisode, dont le sens m'échappait, dans la lumière où j'ai appris à le voir depuis. Longtemps après encore, cette surprise douloureuse est restée. Pourtant, dès l'été de cette même année, ce départ dans l'amertume s'était révélé comme une libération - à l'image d'une porte qui soudain s'était grande ouverte (il avait suffi que je la pousse !) sur un monde insoupçonné, m'appelant à le découvrir. Et chaque nouveau réveil depuis lors a été aussi une nouvelle libération : la découverte d'un assujettissement, d'une entrave intérieure, et la redécouverte de la présence d'un inconnu immense, caché derrière l'apparence familière de ce qui était censé "connu". Mais tout au long aussi de ces quinze années et jusqu'à aujourd'hui même, cet antagonisme opiniâtre, discret et sans failles m'a suivi, comme la seule et grande source durable de frustration que j'aie connue dans ma vie de mathématicien \footnote{(23’)\par Il y a eu pourtant depuis sept ou huit ans une autre "source de frustration" chronique dans ma vie de mathématicien, mais qui s'est exprimée au fil des ans de façon beaucoup plus discrète. Elle a fini par devenir apparente par un effet de répétition, d'accumulation obstinée du même type de situation "frustrante" dans mon activité enseignante, et par éclater finalement en une sorte de "ras-le-bol!", me faisant mettre fin pratiquement à toute activité dite de "direction de recherches". J'effleure cette question une ou deux fois au cours de ma réflexion, pour finalement l'examiner au moins tant soit peu tout à la fin. J'y décris tout au moins cette frustration, et examine le rôle qu'elle a joué dans mon "retour aux maths" (cf. par. 50. "Poids d'un passé").}(23’). Je pourrais dire peut-être qu'elle a été le prix que j'ai payé pour cette première libération, et pour celles qui l'ont suivie. Mais je sais bien que libération et maturation intérieure sont choses étrangères à un "prix à payer", qu'elles ne sont pas question de "profits" et de "pertes". Ou pour le dire autrement : quand la récolte est menée à son terme, quand elle est achevée, il n’y a pas de perte - cela même qui semblait "perte" est devenu "profit". Et il devient clair que je n'ai pas su encore mener jusqu'à son terme cette récolte-là, qui reste, en ce moment encore où j'écris ces lignes, inachevée.



