\section{(28) 未完的收获}

直至1970年首次“觉醒”前,我与学生们的关系,如同我对自己工作的关系,是满足与喜悦的源泉,是我生活中和谐感的坚实、无可辩驳的基础,持续赋予其意义,尽管家庭生活中一种难以捉摸的毁灭正肆虐。当时,在我眼中,这些关系无任何明显冲突,甚至片刻挫折或痛苦也未曾有过。这似是悖论:与某学生的关系冲突,在那著名觉醒后方显露——那转折赋予我生命此前未有的开阔,或许也予我些许柔软——这些特质,似应化解或避免冲突,而非引发或加剧。

然而细察之下,这悖论仅表面,任何角度看皆消散。首先浮现的是:冲突要获化解,须先显现。显性冲突较隐秘或被忽视的冲突成熟,后者虽存,却因未被认知,其表现更“有效”。其次,冲突要以可辨形式显现,须先缩短或消除某种距离。近十五年来,尤其历次“觉醒”,我生命中的变化皆似在减少距离、消解孤立。对一位受敬仰的权威“老板”,冲突难表达;对一位自愿放弃权力、自我放逐于某权威与声望环境、愈发不再被视为某实体(如数学)化身或特使,而渐如常人——不仅可被触及,且愈不愿掩藏伤痛或悲哀者——冲突则更易流露。第三,且尤重要:自首次觉醒起,尤其那时期及随后数年,我的演变似能激发(或唤醒)疑问、不安、对我旧生井然宇宙的“重新审视”。我屡次察觉,这不仅发生在他们中,也在我昔日数学界朋友与同伴间,甚至某些仅耳闻我的科学界同事中。

须说,化解稍深的冲突极罕见。多半,无论表面休战或和解如何,我们的冲突队列伴随一生,鲜有离弃,仅在殡仪员阴郁手中方释。我偶见冲突稍解,有时甚至在认知中化解——但至今,这未在我与某学生或旧日数学界朋友的关系中发生。我也深知,即便再活百年,亦未必发生。

值得注目的是,我与某过去的决裂之时——即我离开IHES「Institut des Hautes Études Scientifiques;IHES」(那机构似我周遭数学微观世界的“母体”)的插曲——这决定性时刻,也是某学生对我深层对立的首次表达。此情境无疑使这插曲尤为痛苦,如艰难中的分娩或诞生。当然,当时我无法在今日习得的光中看这插曲,其意尚模糊。许久后,这痛苦意外仍存。然而,那年夏天,这苦涩离去已现为解放——如一扇门骤然敞开(我仅需推之!),通往未知世界,呼唤我探索。此后每次觉醒皆是新解放:发现内在束缚与桎梏,重识那藏于“已知”熟悉表象后的浩瀚未知。但这十五年间乃至今日,这顽强、隐秘、无隙的对立如影随形,是我数学家生涯唯一持久的重大挫败源。\footnote{(23’)\par 近七八年来,我数学家生涯另有一慢性“挫败源”,多年间表达甚微。其通过教学活动中同一“挫败”情境的固执累积渐显,最终爆发为“受够了!”之感,使我几乎终止所谓“研究指导”活动。我反思中一两次触及此,末尾至少稍作审视,描述此挫败,探究其在我“回归数学”中的作用(见第50段“过往之重”)。}(23’) 或可说,这是我为首次解放及后续解放付的代价。但我知,内在解放与成熟无关“代价”,非“得失”之事。换言之:当收获圆满完成,无“失”——看似“失”者已成“得”。而今书写此行,我明晰尚未完成那收获,仍未完。