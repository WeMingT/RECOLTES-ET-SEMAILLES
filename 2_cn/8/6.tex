\section{(30) 敌对的父亲 (2)}

1970年的重大转折并未在某些前学生与我之间创造出敌意,这种敌意并非源自一个田园诗般、无云的过去。它仅仅是使那些在更为传统的导师-学生(或前导师-前学生)关系框架中难以表达的敌意变得可见。我怀疑,这类冲突在科学界中并不少见,但它们通常以更迂回、更不易辨识的方式表达出来,而不像我所涉及的关系那样明显。

回想起来,我最终并不觉得,在与学生们的关系中,我有太多倾向于扮演父亲的角色——甚至,我无法回忆起哪怕一个多少与此相关的记忆片段。就我个人而言,我感到我投入到与学生关系中的几乎全部精力,与我投入到数学以及实现一个宏大计划中的精力是同一回事。在第一阶段,我只记得一个例子,在这个例子中,我对一个学生的人格产生了兴趣,这种兴趣带有某种亲和力或同情,其强度与数学兴趣相当(即便不完全相等)。但即便在这种情况下,我也并不觉得我对他扮演了父亲的角色。至于我可能对他的个人或其他学生的个人在某种程度上施加的影响,这类事情在我与学生的关系中是我完全不予关注的。(即使是今天,我仍然倾向于对此不加注意,无论是对近几年与我合作的学生,还是对其他人。)当然,在所有这些情况下,学生与我之间的关系绝非“对称”的,意思是至少在师生关系存续期间(甚至很可能在之后的大多数时间里),一个学生在我生命中的重要性,与我在他生命中的重要性不可同日而语,我们各自在这种关系中投入的精神力量也是如此。除了五六个案例中这些力量以清晰可辨的敌意迹象表现出来,我意识到,在我超过二十年的教学活动中,我与不同学生及前学生的关系本质对我而言完全是个谜!不过,探究这些谜团并不是我的职责,而是他们每个人对自己部分的职责。但若是对自己的人格感兴趣,与其探究与前导师关系的来龙去脉,或许有更迫切的事情值得关注……无论如何,尽管我并未表现出对学生扮演父亲角色的倾向,但鉴于我之前提到的特殊“心理特质”,以及我在这种情境中不可避免地至少扮演长者的动态,我很可能常常在他们眼中或多或少地成为了养父的形象。

无论如何,在我提及的几个案例中,这种学生与我之间关系的特殊色彩对我来说毫无疑问。在我的职业生活之外,还有许多其他例子,在这些例子中,无论是否有我的默许,我显然对一些比我年轻的男性和女性扮演了养父的角色,他们被我的人格吸引,并首先通过相互的同情与我建立联系,但与我并无任何亲缘关系。至于我自己的孩子,我对他们的父性情感一直很强烈,从他们幼年起,他们就在我的生活中占据了重要位置。然而,命运的奇特讽刺在于,我的五个孩子中没有一个接受我作为他们的父亲这一事实。在我得以近距离了解的四个孩子的生活中,尤其是在近些年,他们与我关系的这种分裂反映了他们内心深处的一种分裂;特别是对他们身上那些与我——他们的父亲——相似的特质的拒绝……但这里不是探究这种分裂根源的地方,这些根源既深植于他们破碎的童年,也深植于我的童年和我父母的童年;同样也深植于他们母亲的童年,以及她父母的童年。这里也不是衡量这种分裂对他们自己生活或他们孩子的生活的影响的地方……