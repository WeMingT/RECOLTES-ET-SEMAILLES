\section{(31) 使人气馁的力量}

为了结束这次对1948年至1970年间我在数学界所建立的关系的概述,我还需谈谈我与年轻的数学家们的关系,他们或多或少都是初学者,因而不能严格称为"同事",同时我对他们也并非扮演"导师"的角色。这里指的是我在IHES的研讨会上相遇一两年的年轻研究者,或者是在哈佛或其他地方的课程或研讨会上,有时也通过信件往来,例如当我收到一位年轻作者的工作,他期待着我的评论,当然也期待着鼓励。

与初学研究者的关系是一种不如作为某些学生的"导师"那么明显的角色,但同样重要,正如我后来意识到的。在那个时期,我并不像在六七年前那样意识到,对一个知名数学家而言,这一角色代表着相当大的权力。首先是鼓励、激励的力量,无论是在明显出色的工作(但可能因表达不当或"技巧"不足而受到影响)还是在仅仅扎实的工作中都存在;甚至在那些按照一个处于强大手段、经验丰富和信息广泛的前辈标准来看,只代表很微小,甚至可以忽略不计或毫无贡献的工作中也存在。只要提交给我们的工作是认真写就的——通常在阅读前几页就能辨别出来,鼓励的力量就存在。

而使人气馁的力量同样存在,无论什么样的工作都可以随意行使。这就是柯西「Cauchy;Cauchy」对待伽罗瓦「Galois;Galois」,高斯「Gauss;Gauss」对待雅可比「Jacobi;Jacobi」时所使用的力量——这种力量由来已久,杰出而令人敬畏的人一直在使用它!如果历史记录了这两个案例,那是因为受其害的人有足够的信念和自信继续前行,尽管当时在数学界呼风唤雨的权威人士并不善待他们。雅可比找到了一本期刊来发表他的想法,而伽罗瓦则用他最后一封信的纸张作为"期刊"。

如今,对于一个不为人知或知名度不高的数学家来说,要让人们了解他比上个世纪肯定更困难。而知名数学家的权力不仅存在于心理层面,也存在于实际层面。他有权接受或拒绝一项工作,也就是说:给予或拒绝对发表的支持。不管对错,我觉得"在我那个时代",即五十年代和六十年代,拒绝并非没有上诉的余地——如果工作呈现出"值得关注"的结果,它有机会获得另一位杰出人士的支持。今天,情况显然不再如此,因为即使是在他所在领域,也很难找到哪怕一位有影响力的数学家愿意浏览(以他喜欢的态度)一项工作,除非作者已经获得了名声,或者被一位知名同事推荐。

在过去几年里,我曾看到有影响力且才华横溢的数学家使用他们使人气馁和拒绝的权力,不论是对显然必须完成的扎实工作,还是对明显展示出作者力量和独创性的大型工作。多次,这样行使自由裁量权的人恰好是我的一位前学生。这无疑是我在数学家生涯中经历过的最痛苦的经历。

但我偏离了主题,我本应探讨在我热衷于扮演"知名数学家"角色的年代,我如何运用我所拥有的鼓励和使人气馁的力量。我应该补充说,在1970年后,在我科学活动的更为平常的层面上,作为一所省级大学中的普通教师之一,这种力量并没有因此而消失,无论是对我的学生,还是(诚然很少)对偶尔的通信者。但对于我现在的目的,我生活中作为数学家的第一个时期才是唯一重要的。

关于与我的学生的关系,从我有的第一个学生直到今天,我相信我可以毫无保留地说,我尽我所能鼓励他们在他们所选择的工作中前进\footnote{(23iv) 教学的失败 (1) \par 自从写下这些文字以来,我有机会与我1970年后的两位前学生交谈,试图与他们一起探究我在蒙彼利埃大学研究层面教学失败的原因。他们告诉我,我倾向于低估对他们来说可能存在的困难——即掌握对我来说熟悉但对他们来说并不熟悉的技术,这对他们产生了一种使人气馁的效果,因为他们总感觉自己达不到我对他们的期望。此外(在我看来这一点影响更大),当我通过给他们一个我早已准备好的现成的命题来"泄露天机",而不是让他们有乐趣通过自己的方法在他们已经非常接近的时刻发现它时,他们感到很沮丧。在这之后,他们只剩下证明有关命题的"练习"(这并不特别令他们感到兴奋)。这就是我之前在一条注释(注21)中提到但没有详细说明的我身上的"缺乏慷慨"之处。正是这样的挫折,尤其是,代表了我个人对他们俩在经历了极好的开端后对研究兴趣消失的贡献。

我意识到,在1970年前我并不比之后更慷慨。如果我那时没有遇到同样的困难,那可能是因为那个时期来找我的那类学生足够有动力,即使在"长练习"中也能找到魅力,这是学习技艺和一路上学习大量东西的机会;同样地,对于我"泄露天机"的起始命题,他们能通过自己的方法推导出一系列远远超出第一个命题的其他命题。当我改变了教学活动场所时,我对为新学生提出的思考主题做了必要的调整,选择了可以通过直观立即理解的数学对象,而不依赖于任何技术背景。但这种必要的调整本身是不够的,因为(我的新学生与以前的学生相比)在倾向性上的差异比仅仅是背景知识的差异更为重要。这与之前关于我作为"导师"角色某种不足的观察(第25段开始)相符,这种不足在我作为教师的第二个时期比第一个时期表现得更为明显。}(23iv)。即使在今天,在"导师"与学生的关系中,情况很少会有所不同,特别是在一个拥有资源培养优秀学生并与他们一起开垦等待耕种的广阔领域的导师的情况下。令人难以置信但却是真实的是,甚至存在这种极端情况:享有声望的导师会乐于熄灭那些天赋卓越的学生身上的数学激情,而这种激情在他自己年轻时曾激励着他。

但我又离题了!现在我要考察的是我与那些不是我学生的年轻研究者的关系。在这样的关系中,知名人士身上的自我力量不会那么倾向于推动他鼓励他人,因为那位向他求助的年轻陌生人的成功对他自己的荣耀几乎没有贡献。恰恰相反,我认为,在缺乏真正善意的情况下,仅仅是自我力量的作用,几乎总是倾向于朝着相反的方向推动,使用使人气馁和拒绝的力量。在我看来,这既不多也不少,就是那条可以在社会各个领域观察到的通则:证明自己重要性的自我欲望以及满足这种欲望时伴随的秘密快乐通常更强烈、更受欢迎,当一个人拥有的权力有机会造成他人的失望,甚至是他人的羞辱,而不是相反。这条法则在某些特殊环境中表现得特别残酷,比如战争环境,集中营宇宙,监狱或精神病院,甚至在像我们这样的国家里,普通医院的环境......但即使在最日常的环境中,我们每个人都有机会面对证明这条法则的态度和行为。对这些态度的纠正首先是文化纠正,源于在特定环境中关于什么被视为"正常"或"可接受"行为的共识;另一方面是非自我性质的力量,比如对特定人的同情,或者有时,一种自发的善意态度,甚至不依赖于接受这种善意的人。这样的善意无疑是罕见的,无论在什么环境中寻找都是如此。至于数学界的文化纠正,我认为它在过去二十年中已经被严重削弱。无论如何,在我所知道的圈子里,情况确实如此。

我显然坚持偏离我的主题,我的目的不是对这个世纪进行演讲,而是对我自己以及我与那些不是我学生的或多或少初学的研究者的关系进行思考。我不认为我提到的"法则"在这些关系中找到了表达的机会。由于一些在此不必考察的原因,似乎我身上的自我力量,尽管与任何人一样强烈,但在我的生活中并没有采取这种方式来以他人为代价表现自己(除了一些可以追溯到我童年的情况)。我甚至敢说,有机会审视这一点,我对他人的基本倾向是一种善意的倾向,也就是一种当我能帮助时去帮助,当我能减轻时去减轻,当我有能力鼓励时去鼓励的愿望。即使在像对待那个"不知疲倦的朋友"这样深刻分裂的关系中,我心中的自负从未使我迷失到考虑(哪怕是通过无意识的意图)伤害他的地步。(我本有可能这样做,当然是"带着最好的良心"。)我相信在大多数情况下,这种普遍善意的倾向(即使它们可能只是表面的)也标志着我在数学界的关系,包括与那些虽然不在我的学生之列,但可能需要我的支持或鼓励的初学数学家。

我相信至少在五十年代,直到六十年代初,情况无一例外地如此。在那些日子里,我认为这种善意并不仅限于像广中平佑「Heisuke Hironaka;Heisuke Hironaka」或迈克·阿廷「Mike Artin;Mike Artin」这样明显才华横溢的年轻人(尽管那时还没有名声证明他们的能力)。但在六十年代,在自我力量的影响下,这种善意可能在或多或少程度上消失了。如果有人能就此给我提供任何证词,我将特别感激。

我的记忆只恢复了一个具体案例,我将谈论它,超出这个案例,就是那著名的"迷雾",它不会凝结成任何其他具体的案例或事实,而是给我一种内在态度的感觉。当另一个数学家"踩到我的地盘"而没有表示要询问我什么,好像他这个年轻的白痴是在自己家一样,我会感到一定的恼怒!这主要涉及的确是年轻人的情况,他们不太了解行情,他们设法重新发现,有时甚至是在特殊情况下,那些我已经知道多年且更高层次的东西。我想这种情况不会经常发生,但也许两次、三次,也许四次,我说不太准。正如我刚才所说,我只记得一个具体案例,也许是因为同一位年轻数学家多次以这种或那种形式重演了这种情况。我可以说,从各方面来看,这位年轻研究者,其所属大学在国外,一直非常正确,他将自己刚完成的工作寄给我,我被认为是最了解情况的人。每次,我都反应得很冷淡,原因就是我说的那样。我甚至无法确定地说我是否坦率地告诉他,他所做的事情我早已知晓,而且出于这个原因,如果他不在引言中向我行个小礼,我会对他的发表感到烦恼。当然,如果他是我的学生,作者的这种自负就不会那么起作用,一方面是因为与学生已经建立了一种同情的关系,但也因为无论如何,学生的工作包含导师的想法是理所当然的,除非另有说明!我想这种情况可能发生过两次,甚至三次,都是与同一位研究者,每次我都同样冷淡,同样使人气馁。如果我没记错的话,我从未同意推荐这位研究者的工作发表在某本期刊上,也没有同意成为论文答辩委员会的一员(我记得这个问题被提出过)。这几乎就像我决定把他当作替罪羊一样。最精彩的是,他的工作每次都是完全有效的——我相信它们写得很认真,我没有理由怀疑他没有自己找到他在其中发展的想法,这些想法在当时并不那么普遍,只有像塞尔「Serre;Serre」、卡蒂埃「Cartier;Cartier」、我和其他一两个了解情况的人才(或多或少)"熟知"。我不理解的是,这位年轻同事(他当然最终获得了应得的博士学位和职位)没有厌倦每次都向我求助而我每次都对他"冷淡相待",而且他显然从未对我怀恨在心。我还记得他有一次对我的犹豫表示惊讶,他显然不明白发生了什么。如果他期待我的解释,他会很难受的!他有一个漂亮的头,有点像古典希腊风格,非常年轻——特征相当柔和、平静,唤起一种内在的平静感......现在我第一次试图确定他的人和面相所散发的印象,我突然意识到他真的很像我曾经提到的那个"不知疲倦的朋友";他们可能是兄弟,这个和我同龄的朋友带着微笑的语调,而这位比我年轻二十岁的研究者,则更倾向于严肃的语调,但绝不悲伤。这种相似性可能起了作用,我可能将对一个人的蔑视投射到另一个人身上,这种蔑视没有找到机会与另一个人表达,因为他被如此忠诚的友谊迹象所解除武装!我确实必须发展出一层非常厚的外壳,才不会被这个肯定很讨人喜欢的年轻人明显的诚意和做好事的意愿所解除武装,他不厌倦地一次又一次地返回,而我甚至不屑于赐给他哪怕一个微笑!