\section{(21) 守得严实的公开秘密}

昨天我一定遗漏了一些小插曲,比如与某同事关系中短暂的“冷淡”,多半源于我的敏感。我还该补充三四次自尊受挫的场合,当一些同事兼朋友在某些出版物中,未提及我曾与他们分享的某想法或结果,而这些(在我看来)显然在其工作中起了作用。我至今记得这些,表明那是我的敏感点,或许并未随年龄完全消退!除了一次,我从未向当事人提及,他们的诚意无疑无可置疑。反过来,这种情况也一定发生过,只是未有回音传到我耳中。在我的“微观世界”里,我未听说因优先权问题引发不和或敌意,甚至连当事人间的酸言酸语也没有。然而,当我唯一一次就此(在我看来显而易见的情况)展开讨论时,确实发生了一场小小的争执,却净化了氛围,未留下任何怨恨。那是一位尤为杰出的同事,他有诸多才能,其中之一是能以惊人速度吸收听到的内容,但我感觉他常有种令人遗憾的倾向,将刚从他人处听来的想法据为己有。

这里有个难题,或多或少存在于所有数学家(不仅限于他们)之中,不仅仅源于驱使我们大多数人(我也不例外)追求“功绩”——无论是真实的还是假想的——的自我训练。对情境(数学或其他)的理解,无论我们如何达成,是否借助他人,本质上都是一件个人之事,一种个人经验,其果实是视野,必然也具个人性。视野有时可传递,但传递的视野不同于最初的视野。因此,需高度警觉,方能分辨他人对自己视野形成的贡献。我自己显然并非总有这种警觉,那是我最不关心的事,尽管我却期待他人对我如此!迈克·阿廷「Mike Artin;Mike Artin」是第一个也是唯一一个,以玩笑般揭露公开秘密的口吻告诉我的人,他说试图分辨“自己的”和“他人的”部分,既不可能也毫无意义,当你能全力抓住某事物并理解它时。这让我有些困惑,因这完全不符合卡尔唐「Cartan;Cartan」、迪厄多内「Dieudonné;Dieudonné」、施瓦茨「Schwartz;Schwartz」等人以身作则教我的职业伦理。然而,我模糊感到,他的话语与他那戏谑的目光中,藏着我此前未察觉的真相。\footnote{(9月30日)关于事情的另一面,可参见6月1日的笔记(比本文晚三个月)“模棱两可”「L’ambiguïté;The Ambiguity」(第63’’号),探讨对自身及他人的某种纵容所隐藏的陷阱。} 我的数学关系(尤其是数学生产)深受自我驱动,而迈克并非如此。他给人的印象真是像个玩耍的孩子般做数学,却并未忘记生计。