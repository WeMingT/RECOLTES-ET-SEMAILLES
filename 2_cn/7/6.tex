\section{(21) Un secret de Polichinelle bien gardé}

J'ai sûrement oublié hier quelques épisodes mineurs, comme des "froids" passagers dans ma relation à tel collègue, dûs notamment à ma susceptibilité. Je devrais ajouter aussi trois ou quatre occasions où mon amour-propre se trouvait déçu, quand il arrivait que des collègues et amis ne se rappellent pas, dans telles de leurs publications, que telle idée ou résultat dont je leur avais fait part avait dû jouer un rôle dans leur travail (ainsi me semblait-il). Le fait que je m'en rappelle encore montre que c'était là un point sensible, et qui peut-être n'a pas entièrement disparu avec l'âge ! Sauf une fois, je me suis abstenu d'en faire mention aux intéressés, dont la bonne foi était certes au-dessus de tout soupçon. La situation inverse a sûrement dû se produire également, sans que j'en reçoive d'écho. Je n'ai pas eu connaissance d'un cas, dans mon "microcosme", où une question de priorité soit l'occasion d'une brouille ou d'une inimitié, ni même de propos aigres-doux entre les intéressés. Quand même, la seule fois où j'ai eu une telle discussion (dans un cas qui me semblait flagrant) il y a eu une sorte de prise de bec, qui a assaini l'atmosphère sans laisser un résidu de ressentiment. Il s'agissait d'un collègue particulièrement brillant, qui avait entre autres capacités celle d'assimiler avec une rapidité impressionnante tout ce qu'il entendait, et il me semble qu'il avait souvent une fâcheuse tendance à prendre pour siennes les idées d'autrui qu'il venait d'apprendre de leur bouche.

Il y a là une difficulté qui doit se retrouver sous une forme plus ou moins forte chez tous les mathématiciens (et pas seulement chez eux), et qui n'est pas seulement due à l'entraînement égotique qui pousse la plupart d'entre nous (et je n'y fais pas exception) à s'attribuer des "mérites", aussi bien réels que supposés. La compréhension d'une situation (mathématique ou autre), quelle que soit la façon dont nous y parvenions, avec ou sans l'assistance d'autrui, est en elle-même une chose d'essence personnelle, une expérience personnelle dont le fruit est une vision, nécessairement personnelle aussi. Une vision peut parfois se communiquer, mais la vision communiquée est différente de la vision initiale. Cela étant, il faut une grande vigilance pour néanmoins décerner la part d'autrui dans la formation de sa vision. Sûrement moi-même n'ai pas toujours eu cette vigilance, qui était le dernier de mes soucis, alors que pourtant je l'attendais chez les autres vis-à-vis de moi ! Mike Artin a été le premier et seul qui m'ait fait entendre un jour, avec l'air blagueur de celui qui divulgue un secret de Polichinelle, que c'était à la fois impossible et parfaitement vain, de se fatiguer à vouloir discerner quelle est la part "à soi", quelle celle "d'autrui" quand on arrive à prendre une substance à bras le corps et à y comprendre quelque chose. Cela m'avait un peu dérouté, alors que ce n'était pas du tout dans la déontologie qui m'avait été enseignée par l'exemple par Cartan, Dieudonné, Schwartz et d'autres. Je sentais pourtant confusément qu'il y avait dans ses paroles, et tout autant dans son regard rieur, une vérité qui m'avait échappée jusque là \footnote{(30 Septembre)Pour un autre aspect des choses, voir cependant la note du 1 juin (postérieure de trois mois au présent texte), "L'ambiguïté" (n° 63’’), examinant les pièges d'une certaine complaisance à soi et à autrui.}. Ma relation à la mathématique (et surtout, à la production mathématique) était fortement investie par l'ego, et ce n'était pas le cas chez Mike. Il donnait vraiment l'impression de faire des maths comme un gosse qui s'amuse, et sans pour autant oublier le boire et le manger.

