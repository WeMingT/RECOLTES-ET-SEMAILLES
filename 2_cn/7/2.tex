\section{(17) 特里·米尔基尔}

说实话,在那次关键的转折之前,我就已与一些伙伴建立了友谊(他们后来成了“同事”),如果我当时考虑过这个问题(以及如果他们真是我的朋友……),我可能会将他们归入“沼泽”。直到这次反思,迫使我挖掘记忆,那些零散的回忆才得以聚拢。我是在最初的时光里认识这三位朋友的,那时我与他们在南锡「Nancy;Nancy」一起学习这份职业——那是一个我们尚在同一处境的时刻,还没人将我视为“权威”。这绝非偶然,而在随后的二十年间,再无类似友谊的出现。我们四人都是外来者,这无疑是一个重要的纽带——我与那些年轻的“师范生”关系远不如与他们亲密,他们和我一样被空降到南锡,但我们仅在学院里见面。其中一位朋友在一两年后移民南美。他和我一样被CNRS的研究吸引,我总有种感觉,他自己也不太清楚在“寻找”什么,他在CNRS的处境逐渐变得有些棘手。我们偶尔见面或通信,时间一长,便失去了联系。与另外两位朋友的关系则更持久,也更深厚,远非泛泛之交。数学兴趣在其中几乎不起作用,甚至毫无影响。

我和特里·米尔基尔「Terry Mirkil;Terry Mirkil」及其妻子普蕾索西亚——她娇小脆弱,而他身材敦实——两人都带着一种柔和的气质,我们常在南锡共度夜晚,有时彻夜不眠,唱歌、弹琴(那时是特里弹奏),谈论音乐——那是他们的热情所在——以及生活中重要的事物和其他话题。诚然,不是最重要的事——那些总是被小心翼翼掩藏的事……然而,这份友谊带给我许多。特里有着我所欠缺的敏锐与洞察力,那时我的大部分精力已极度聚焦于数学。他比我更能保持对简单而本质事物的感知——阳光、雨水、大地、风声、歌声、友谊……

特里在达特茅斯学院「Dartmouth Collège;Dartmouth College」找到一份合意的职位后,那里离我常去的哈佛「Harvard;Harvard」不算太远(从五十年代末开始),我们继续见面和通信。其间,我得知他深受抑郁困扰,这让他长时间住在“疯人院”里——他在唯一一封简短提及此事的信中这样称呼,那是他在一次“可怕的住院”后写给我的。见面时,我们从不谈及这些——除了一两次非常偶然的提及,那是为了回应我对他们夫妇未领养孩子的惊讶。我不认为自己曾萌生过与他深入探讨问题本质的想法,甚至连触及表面的念头都没有——或许我压根没想过,我的朋友生活中,或者我自己生活中,可能有些问题值得正视……对此,我们之间有一道无言且无法逾越的禁忌。

渐渐地,见面与通信变得稀疏。确实,我越来越被任务和角色所困,尤其是那种近乎执念的意志——不断超越自己,积累成就,或许是为了逃避其他东西——而我的家庭生活却神秘而不可阻挡地恶化……

有一天,我从特里在达特茅斯的一位同事的信中得知,我的朋友自杀了(那已是他在死后被埋葬很久之后……),这消息仿佛穿过迷雾传来,像是从一个遥远的世界传来的回声,一个我不知何时已离开的世界。或许是我内心的某个世界,早在特里结束自己被无法平息的焦虑摧毁的生命之前,就已死去——那焦虑,他未能或不愿化解,而我未能或不愿察觉……