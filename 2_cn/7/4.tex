\section{(19) 无爱之世界}

“大老板”老了,幸好,他略微有些崩塌,而那个孩子自此得以更自在些。关于这段与那些真正坚韧的朋友的关系,我似乎确实触及了我生命中最明显、最荒诞的一个例子,体现了某种虚荣(以及其他因素)在一份个人关系中的效应。或许我仍在自欺,但我想这也是我与数学界(甚至其他领域)的同事或朋友关系中,唯一被虚荣长期渗透的例子,而非仅偶尔以隐秘而短暂的方式显现。此外,我觉得在我当时数学界众多朋友中,那些我乐于相伴的人,没有一个我能想象他们会在与同事——无论是否朋友——的关系中,经历类似的迷失。在所有朋友中,我或许是最不“酷”的,最“极端”,最不愿流露一丝幽默(这只在晚年才逐渐显现),也最倾向于极端严肃地对待自己。显然,我也不会特别寻求像我这样的人的陪伴(假设真有这样的人存在)!

令人惊讶的是,我的朋友们,无论属于“沼泽”与否,都能忍受我,甚至对我怀有深情。这是一件好事,也很重要,值得在此一提——尽管我们常仅为讨论数学而聚在一起,耗费数小时、数日:深情仍在流动,如同今日仍在流动的那样,在我与此刻的朋友之间(因缘分有时颇为偶然),自从1949年我在南锡「Nancy;Nancy」初次被接纳时,那份深情便已开始,在洛朗与海伦·施瓦茨「Laurent et Hélène Schwartz;Laurent and Hélène Schwartz」家中(我几乎成了家庭一员),在迪厄多内「Dieudonné;Dieudonné」家中,在戈德芒「Godement;Godement」家中(有一段时间我也常去那里)。

这份温暖的深情,环绕着我在数学界迈出的最初步伐,我却有些倾向于将其遗忘,它对我整个数学家生涯至关重要。无疑正是它,为我与前辈们所代表的环境的关系,赋予了同样的温暖基调。它赋予了我对这一环境认同的全部力量,也为“数学共同体”这个名称注入了全部意义。

显然,对如今许多年轻数学家而言,他们在学习时期,甚至之后很久,都被隔绝于任何深情与温暖的流动之外;他们的工作映照在一位疏远导师的眼神中,以及他吝啬的评语里,仿佛在阅读一份来自研究与工业部的通告,剪断了工作的翅膀,使其丧失了比谋生——那阴郁而不确定的谋生——更深刻的意义。

但我说到这种不幸时,有些超前了,这或许是七十年代与八十年代数学界最深的不幸——那个由我昔日的学生,以及我旧日朋友的学生们定调的数学世界。在那个世界里,导师常如扔骨头给狗般为学生指定研究课题——要么这个,要么没有!如同为囚犯指定牢房:你将在此服完你的孤独!某项细致而扎实的工作,多年耐心努力的果实,却被手握大权、洞悉一切者的轻蔑微笑否决:“这工作让我不感兴趣!”问题就此了结。扔进垃圾桶,不必再提……

我知道,这种不幸在我所认识的环境中,在五十年代与六十年代我常伴的朋友间,并不存在。诚然,我在1970年得知,这在数学之外的科学界倒是家常便饭——即便在数学界,这种公开的轻蔑与权力的明目张胆滥用(且无处申诉)似乎也非罕见,甚至出现在一些我曾偶遇的知名同事身上。但在我天真地视为“那个”数学世界,或至少是其忠实缩影的朋友圈中,我从未经历过这些。

然而,轻蔑的种子一定早已埋下,由我的朋友们和我播种,在我们的学生中发芽。不仅在学生中,也在某些我昔日的同伴与朋友身上。但我的角色并非揭发,甚至也不是抗争:腐败无法被抗争。看到我曾深爱的学生,或昔日同伴中的某人如此行事,我内心紧缩——与其接受痛苦带来的认知,我常拒绝痛苦,挣扎着逃入否认与抗争的态度:这种事不该存在!然而它确实存在——甚至,我内心深处明白其意义。我并非与之全然无关,若我曾深爱的某个学生或同伴,乐于悄然压迫另一个我所爱之人,并在后者身上看到我的影子。

我又跑题了,甚至可说是双重跑题——仿佛轻蔑之风只在我居所周围吹拂!然而,我正是通过它吹向我,尤其是吹向我亲近与珍视之人,才感受到并认识它的。但现在尚未到谈论之时,除非仅对自己,在沉默中自语。现在更该拾起我的反思-见证之线索,或许可命名为“追寻轻蔑”——在我自身及周围,在五十年代与六十年代属于我的那个数学环境中的轻蔑。