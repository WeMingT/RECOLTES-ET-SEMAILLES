\section{(19) Le monde sans amour}

Le Grand Chef a vieilli, heureusement, il s'est effrité un tantinet, et le gosse depuis a pu en prendre plus à son aise. Pour ce qui est de cette relation avec ces amis vraiment endurants, il me semble bien avoir mis le doigt là sur le cas dans ma vie le plus flagrant, le plus grotesque des effets d'une certaine fatuité (entre autres) dans une relation personnelle. Peut-être que je suis encore en train de m'abuser, mais je crois bien que c'est aussi le seul cas où ma relation à un collègue ou à un ami dans le milieu mathématique (ou même ailleurs) ait été investi de façon durable par la fatuité, au lieu que celle-ci ne se contente de se manifester occasionnellement, de façon discrète et fugace. Il me semble d'ailleurs que parmi les nombreux amis que j'avais alors dans le monde mathématique et que j'aimais à fréquenter, il n'y en a aucun pour lequel je pourrais m'imaginer qu'ils aient connu un semblable égarement, dans une relation à un collègue, ami ou pas. Parmi tous mes amis, j'étais le moins "cool" peut-être, le plus "polard", le moins enclin à laisser percer une pointe d'humour (ça a fini par me venir sur le tard seulement), le plus porté à se prendre terriblement au sérieux. Sûrement même, je n'aurais pas tellement recherché la compagnie de gens comme moi (à supposer qu'il s'en soit trouvé) !

L'étonnant, c'est que mes amis, "marais" ou pas "marais", me supportaient et même me prenaient en affection. C'est une chose bonne et importante à dire ici - alors même que souvent on ne se voyait guère que pour discuter maths à longueur d'heures et de jours : l'affection circulait, comme elle circule encore aujourd'hui, entre les amis du moment (au gré d'affinités parfois fortuites) et moi, depuis ce premier moment où j'ai été reçu avec affection à Nancy, en 1949, dans la maison de Laurent et Hélène Schwartz (où je faisais un peu partie de la famille), celle de Dieudonné, celle de Godement (qu'en un temps je hantais également régulièrement).

Cette chaleur affectueuse qui a entouré mes premiers pas dans le monde mathématique, et que j'ai eu tendance un peu à oublier, a été importante pour toute ma vie de mathématicien. C'est elle sûrement qui a donné une semblable tonalité chaleureuse à ma relation au milieu que mes aînés incarnaient pour moi. Elle a donné toute sa force à mon identification à ce milieu, et tout son sens à ce nom de "communauté mathématique".

Visiblement, pour beaucoup de jeunes mathématiciens aujourd'hui, c'est d'être coupés dans leur temps d'apprentissage, et souvent bien au-delà, de tout courant d'affection, de chaleur ; de voir reflété leur travail dans les yeux d'un patron distant et dans ses parcimonieux commentaires, un peu comme s'ils lisaient une circulaire du ministère de la recherche et de l'industrie, qui coupe les ailes au travail et lui enlève un sens plus profond que celui d'un gagne-pain maussade et incertain.

Mais j'anticipe, en parlant de cette disgrâce-là, la plus profonde de toutes peut-être, du monde mathématique des années 70 et 80 - le monde mathématique où ceux qui furent mes élèves, et les élèves de mes amis d'antan, donnent le ton. Un monde où, souvent, le patron assigne son sujet de travail à l'élève, comme on jette un os à un chien - ça ou rien ! Comme on assigne une cellule à un prisonnier : c'est là que tu purgeras ta solitude ! Où tel travail minutieux et solide, le fruit d'années de patients efforts, se trouve rejeté par le mépris souriant de celui qui sait tout et qui a le pouvoir en mains : "ce travail ne m'amuse pas !" et la question est classée. Bon pour la poubelle, n'en parlons plus...

De telles disgrâces, je le sais bien, n'existaient pas dans le milieu que j'ai connu, parmi les amis que je hantais, dans les années cinquante et soixante. Il est vrai que j'ai appris en 1970 que c'était là plutôt le pain quotidien dans le monde scientifique en dehors des maths - et même dans les maths ce n'était pas si rare apparemment, le mépris à visage ouvert, l'abus de pouvoir flagrant (et sans recours), même chez certains collègues de renom et que j'avais eu l'occasion de rencontrer. Mais dans le cercle d'amis que j'avais naïvement pris pour "le" monde mathématique, ou tout au moins comme une expression miniature fidèle de ce monde, je n'ai rien connu de tel.

Pourtant, les germes du mépris devaient y être déjà, semés par mes amis et par moi et qui ont levé en nos élèves. Et non seulement en nos élèves, mais aussi en tels de mes anciens compagnons et amis. Mais mon rôle n'est pas de dénoncer ni même de combattre : on ne combat pas la corruption. De la voir en tel de mes élèves que j'ai aimé, ou en tel des compagnons d'antan, quelque chose en moi se serre - et plutôt que d'accepter la connaissance que m'apporte une douleur, souvent je refuse la douleur et me débats et me réfugie dans le refus et une attitude de combat : telle chose n'a pas lieu d'être ! Et pourtant elle est - et même, je sais au fond quel en est le sens. A plus d'un titre, je n'y suis pas étranger, si tel élève ou compagnon d'antan que j'ai aimé, se plaît à écraser discrètement tel autre que j'aime et en qui il me reconnaît.

A nouveau je digresse, doublement je pourrais dire - comme si le vent du mépris ne soufflait qu'autour de ma demeure ! C'est pourtant par son souffle sur moi surtout et sur ceux qui me sont proches et chers que j'en suis touché et le connais. Mais le temps n'est pas mûr pour en parler, si ce n'est à moi-même seulement, dans le silence. Il est temps plutôt que je reprenne le fil de ma réflexion-témoignage, qui pourrait bien prendre le nom "A la poursuite du mépris" - le mépris en moi-même et autour de moi, dans ce milieu mathématique qui fut le mien, dans les années cinquante et soixante.



