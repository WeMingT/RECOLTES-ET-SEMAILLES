\section{(18) 二十年的虚荣,或:不倦的朋友}

我与特里「Terry;Terry」的友谊,我相信从未因我们在数学界地位的差异,或我可能从中获得的优越感而扭曲。这份友谊,以及那段时间生命赠予我的另外一两段友谊(无视我是否“配得上”),无疑是当时对抗我内心隐秘虚荣的少数解药之一。这种虚荣由社会地位滋养,更由我对自己数学能力的觉醒及其自我赋予的价值所助长。然而,与第三位朋友的关系却并非如此。这位朋友,以及后来他的妻子(他大约在我们于南锡「Nancy;Nancy」相识时认识了她),在这些年里,每次我们在他家或我家相聚,他们都以温暖、细腻而单纯的友谊待我。在这份友谊中,显然没有任何与地位或智力相关的隐秘动机。然而,我与他们的关系,二十多年来始终带着我内心深处的那种矛盾,那种我曾提及的撕裂,它标志着我作为数学家的生活。每次与他们相处,我都无法抑制地感受到他们友爱的温暖,并几乎违背自己意愿地回应!与此同时,二十多年来,我却成功地以一种高高在上的姿态,轻蔑地看待我的朋友。这一定从我们在南锡的最初几年便已开始,并且很长一段时间,我的偏见也延伸到他妻子身上,仿佛早已认定,她只能与他一样“微不足道”。我与母亲之间,常以一个嘲弄的绰号称呼他,这个绰号在我母亲1957年去世后,仍在我心中刻印了许久。现在看来,我态度背后的力量之一,至少是母亲强势性格在她有生之年对我的影响,以及她去世后近二十年,我仍沉浸在她生命中主导的价值观念中。我朋友那温和、亲切、毫无争斗性的天性,被默认为是“微不足道”,因而成为我嘲弄轻蔑的对象。直到此刻,我第一次费心审视这段关系,才发现它长期以来被我对他人温暖同情的疯狂隔离所笼罩。我的朋友特里「Terry;Terry」,并不比这位朋友更具斗争性或冲击力,却有幸被我母亲认可,未成为她嘲笑的对象——我怀疑,正因如此,我与特里的关系才能在我内心毫无抗拒地绽放。他的数学投入并不更热烈,天赋也不更突出,但我并未因此以轻蔑与自负为壁垒,将他与他的妻子隔绝在外!

在这另一段关系中,我至今仍无法理解的是,我朋友那深情的友谊,为何从未因我每次见面时他必能感受到的冷淡而气馁。然而,如今我清楚,我不仅是那层外壳与轻蔑,不仅是那块引以为傲的脑力肌肉与虚荣。如同在他们身上,我心中也有个孩子——那个我假装忽视、被我轻视的孩子。我与他割裂,但他仍在我内心某处活着,如我出生之日般健康而充满活力。无疑,我朋友的深情是献给这个孩子的,他们比我更少与自己的根源割裂。而这个孩子,也一定是在“大老板”背过身时,偷偷地、匆忙地回应着。