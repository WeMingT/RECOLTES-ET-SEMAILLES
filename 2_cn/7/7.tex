\section{(22) 布尔巴基,或我的莫大幸运——及其代价}

在深入探究可见表面之下之前,一个事实已然跃入我眼帘:我在五六十年代浸润其中的数学环境,确乎是一个“无冲突的世界”,可谓如此!这本身便颇为非凡,值得我稍作停留。

我需立刻澄清,这指的是一个极为狭小的圈子,是我数学微观世界的核心,仅限于我的“直接环境”——那二十余位我定期相会、联系最紧密的同事与朋友。回顾这些人,我惊觉其中过半是布尔巴基学派「Bourbaki;Bourbaki」的活跃成员。显然,这个微观世界的核心与灵魂便是布尔巴基。几乎可说,是布尔巴基及其最亲近的数学家们。到了六十年代,我已不再是该群体一员,但与某些成员的关系依然紧密如初,尤其是与迪厄多内「Dieudonné;Dieudonné」、塞尔「Serre;Serre」、泰特「Tate;Tate」、朗「Lang;Lang」、卡蒂埃「Cartier;Cartier」。那时我仍是布尔巴基研讨会「Séminaire Bourbaki;Bourbaki Seminar」的常客,或者说,那时我才真正成为常客,也是在那段时间,我在会上发表了关于概形理论「Théorie des schémas;Theory of Schemes」的大多数报告。

想必在六十年代,布尔巴基群体的“基调”逐渐滑向愈发显著的精英主义,我当时无疑是其中的一份子,也正因如此,我不太可能察觉。我仍记得1970年时的惊讶,发现布尔巴基之名在数学界广大阶层(我此前未曾留意)中何其不受欢迎,几乎成了精英主义、狭隘教条主义、崇尚“规范”形式而牺牲活泼理解、晦涩难懂、扼杀自发性的代名词,等等!不仅在“沼泽”中布尔巴基声誉不佳,六十年代,甚至更早,我便偶尔从一些思维方式不同的数学家那里听到类似反馈,他们对“布尔巴基风格”颇为“过敏”。\footnote{(15)\par 我并未感到这种对布尔巴基风格的“过敏”导致这些数学家与我或其他布尔巴基成员或支持者之间沟通困难,若群体精神真如小圈子或精英中的精英那般,情况或会如此。超越风格与潮流,所有成员对数学实质皆有敏锐感知,无论其来源为何。只是到了六十年代,我忆起某位朋友将不感兴趣的数学家工作斥为“烦人家伙”。对那些我几乎一无所知的事物,我常倾向于将这类轻率自信的评价奉为圭臬——直到某天发现某“烦人家伙”实为原创而深刻的思想者,只是未获我那才华横溢的朋友青睐。我感到,在某些布尔巴基成员中,面对未知或不甚理解的工作时,那种谦逊(或至少克制)的态度最先消蚀,而那“数学直觉”——能感知丰富实质或扎实工作的能力,无需倚赖名声——仍存。据我零星听闻,如今在曾属我的数学环境中,谦逊与直觉皆已稀有。}(15) 作为无条件支持者,我对此既惊讶又有些伤感——我原以为数学能令众心合一!然而我早该记起,初涉之时,吞咽布尔巴基文本并非总是轻松或启发,尽管效率颇高。那规范文本几乎未透露其创作氛围,至少可说如此。如今我觉得,这正是布尔巴基文本的主要缺陷——连偶尔的微笑也无从让人猜想,这些文字出自活生生的人,且这些人彼此的联系,远非对严苛规范无情忠诚的某种誓言所能定义……

但关于滑向精英主义的问题,如同布尔巴基写作风格,皆为题外话。此处令我震撼的是,我所选为职业环境的“布尔巴基微观世界”,竟是一个无冲突的世界。这尤为引人注目,因其中主角各具鲜明的数学个性,许多被视为“伟大数学家”,每人皆有足够分量自成一派,成为自己微观世界的中心与无可争议的领袖!\footnote{(16)\par 实则,多位布尔巴基成员确有各自的微观世界,或大或小,或独立或超越布尔巴基微观世界。但在我身上,这类微观世界在我退出布尔巴基后方形成,且仅在我将全副精力投入个人任务后,或许并非偶然。}(16) 这些鲜明的个性在同一微观世界、同一个工作群体中,二十年间的友好共处甚至深情相伴,在我看来是如此非凡,或许独一无二。这与昨日关于布尔巴基“异常成功”的印象相呼应。

看来,我初接触数学世界时,确有非凡幸运,正好落入时空中的特权之地——几年前方形成的一个品质卓越、或许因这品质而独一无二的数学环境。这环境成了我的归属,始终是我理想“数学共同体”的化身,这理想或许在那时(超出我所认知的环境)或数学史上任何时刻,皆未必真正存在,除非在某些同样狭小的群体中(或许如围绕毕达哥拉斯「Pythagore;Pythagoras」形成的那般,精神迥异)。

我对此环境的认同极强,与我在四十年代末诞生的数学家新身份密不可分。这是除家庭外,首个以温暖接纳我、视我为其中一员的群体。另有一层不同的联系:我对数学的approach,在群体及其成员——我的新环境——中得到印证。它与“布尔巴基式”approach并非全同,但显然是兄弟。

此外,这环境对我而言,成了那理想之地(或近乎如此!),那无冲突之地——对它的追求无疑引领我走向数学,这门在我看来毫无冲突可能的科学!若我此前提及“非凡幸运”,心中已然明白,这幸运有其代价。它让我得以发展能力,在前辈变同侪的环境中展现数学家的价值,同时也成了我逃避生活中冲突的便利手段,以及长久精神停滞的途径。