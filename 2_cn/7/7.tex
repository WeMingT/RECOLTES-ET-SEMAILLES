\section{(22) Bourbaki, ou ma grande chance - et son revers}

Avant même de plonger un peu plus en dessous de la surface visible, il y a une constatation qui s'impose à moi dès à présent : c'est que le milieu mathématique que je hantais pendant deux décennies, en les années 50 et 60, était bel et bien un "monde sans conflit", autant dire ! C'est là une chose assez extraordinaire par elle-même, et qui mérite que je m'y arrête quelque peu.

Il me faudrait préciser tout de suite qu'il s'agit d'un milieu très restreint, la partie centrale de mon microcosme mathématique, limitée à mon "environnement" immédiat, - les quelques vingt collègues et amis que je rencontrais régulièrement, et auxquels j'étais le plus fortement lié. Les passant en revue, j'ai été frappé par le fait que plus de la moitié de ces collègues étaient des membres actifs de Bourbaki. Il est clair que le noyau et l'âme de ce microcosme était Bourbaki. C'était, à peu de choses près, Bourbaki et les mathématiciens les plus proches de Bourbaki. Dans les années 60 je ne faisais plus partie moi-même du groupe, mais ma relation à certains des membres restait aussi étroite que jamais, notamment avec Dieudonné, Serre, Tate, Lang, Cartier. Je continuais d'ailleurs à être un habitué du Séminaire Bourbaki ou plutôt, je le suis devenu à ce moment, et c'est à cette époque que j'y ai fait la plupart de mes exposés (sur la théorie des schémas).

C'est sans doute dans les années soixante que le "ton" dans le groupe Bourbaki a glissé vers un élitisme de plus en plus prononcé, dont j'étais sûrement partie prenante alors, et dont pour cette raison je ne risquais pas de m'apercevoir. Je me rappelle encore de mon étonnement, en 1970, en découvrant à quel point le nom même de Bourbaki était devenu impopulaire dans de larges couches (de moi ignorées jusque là) du monde mathématique, comme synonyme plus ou moins d'élitisme, de dogmatisme étroit, de culte de la forme "canonique" aux dépens d'une compréhension vivante, d'hermétisme, d'antispontanéité castratrice et j'en passe! Ce n'est d'ailleurs pas que dans le "marais" que Bourbaki avait mauvaise presse : dans les années soixante, et peut-être dès avant, j'en avais eu des échos occasionnels de la part de mathématiciens ayant une autre tournure d'esprit, allergique au "style Bourbaki" \footnote{(15)\par Je n'ai pas eu l'impression que cette "allergie" au style Bourbaki ait donné lieu à des difficultés de communication entre ces mathématiciens et moi ou d'autres membres ou sympathisants de Bourbaki, comme il aurait été le cas si l'esprit du groupe avait été esprit de chapelle, d'élite dans l'élite. Au-delà des styles et des modes, il y avait chez tous les membres du groupe un sens vif pour la substance mathématique, d'où qu'elle provienne. C'est au cours des années soixante seulement que je me rappelle tel de mes amis, qualifiant d' "emmerdeurs" tels mathématiciens dont le travail ne l'intéressait pas. S'agissant de choses dont je ne savais pratiquement rien par ailleurs, j'avais tendance à prendre pour argent comptant de telles appréciations, impressionné par tant d'assurance désinvolte - jusqu'au jour où je découvrais que tel "emmerdeur" était un esprit original et profond, qui n'avait pas eu l'heur de plaire à mon brillant ami. Il me semble que chez certains membres Bourbaki, une attitude de modestie (ou tout au moins de réserve) devant le travail d'autrui, quand on ignore ce travail ou le comprend imparfaitement, s'est érodée d'abord, alors que subsistait encore cet "instinct mathématique" qui fait sentir une substance riche ou un travail solide, sans avoir à se référer à une réputation ou à un renom. Par les échos qui me parviennent ici et là, il me semble que l'une comme l'autre, modestie comme instinct, sont aujourd'hui devenus choses rares dans ce qui fut mon milieu mathématique.}(15). En adhérant inconditionnel j'en avais été surpris et un peu peiné - moi qui croyais que la mathématique faisait l'accord des esprits! Pourtant j'aurais dû me rappeler que lors de mes débuts, ce n'était pas toujours facile ni inspirant d'ingurgiter un texte Bourbaki, même si c'était expéditif. Le texte canonique ne donnait guère une idée de l'ambiance dans lequel il était écrit, à dire le moins. Il me semble maintenant que c'est là justement la principale lacune des textes Bourbaki - que pas même un sourire occasionnel puisse y laisser soupçonner que ces textes aient été écrits par des personnes, et des personnes liées par bien autre chose que par quelque serment de fidélité inconditionnelle à d'impitoyables canons de rigueur...

Mais la question du glissement vers un élitisme, comme celle du style d'écriture de Bourbaki, est ici une digression. La chose qui me frappe ici, c'est que ce "microcosme bourbakien" que j'avais choisi pour milieu professionnel, était un monde sans conflit. La chose me semble d'autant plus remarquable que les protagonistes dans ce milieu avaient chacun une forte personnalité mathématique, et bon nombre sont considérés comme des "grands mathématiciens", dont chacun assurément faisait le poids pour former son propre microcosme à lui, dont il aurait été le centre et le chef incontesté ! \footnote{(16)\par A vrai dire, plusieurs des membres Bourbaki avaient sûrement leur propre microcosme "à eux", plus ou moins étendu, à part ou au-delà du microcosme bourbakien. Mais ce n'est peut-être pas un hasard si dans mon propre cas, un tel microcosme ne s'est constitué autour de moi qu'après que j'aie cessé de faire partie de Bourbaki, et que toute mon énergie a été investie dans des tâches qui m'étaient personnelles.}(16) C'est la convivance cordiale et même affectueuse, pendant deux décennies, de ces fortes personnalités dans un même microcosme et dans un même groupe de travail, qui m'apparaît comme une chose si remarquable, peut-être unique. Cela rejoint l'impression de "réussite exceptionnelle" qui s'était déjà dégagée hier à propos de Bourbaki.

Il semblerait finalement que j'ai eu cette chance exceptionnelle, lors de mon premier contact au monde mathématique, de tomber pile sur le lieu privilégié, dans le temps et dans l'espace, où venait de se former depuis quelques années un milieu mathématique d'une qualité exceptionnelle, peut-être unique par cette qualité-là. Ce milieu est devenu le mien, et est resté pour moi l'incarnation d'une "communauté mathématique" idéale, qui probablement n'existait pas plus à ce moment-là (au-delà du milieu qui pour moi l'incarnait) qu'à aucun autre dans l'histoire des mathématiques, si ce n'est peut-être dans quelques groupes tout aussi restreints (tel celui peut-être, qui s'était formé autour de Pythagore dans un esprit tout différent).

Mon identification à ce milieu a été très forte, et inséparable de ma nouvelle identité de mathématicien, née à la fin des années quarante. C'était le premier groupe, au-delà du groupe familial, où j’aie été accueilli avec chaleur, et accepté comme un des leurs. Autre lien, d'une autre nature : ma propre approche des mathématiques trouvait confirmation dans celle du groupe, et dans celles des membres de mon nouveau milieu. Elle n'était pas identique à l'approche "bourbachique", mais il était clair que les deux étaient frères.

Ce milieu par surcroît, devait pour moi représenter ce lieu idéal (ou peu s'en fallait !), ce lieu sans conflit dont la quête sans doute m'avait dirigé vers les mathématiques, la science entre toutes où toute velléité de conflit me semblait absente! Et si j'ai parlé tantôt de ma "chance exceptionnelle", il était présent dans mon esprit que cette chance-là avait son revers. Si elle m'a permis de développer des moyens, et de donner ma mesure comme mathématicien dans le milieu de mes aînés devenus mes pairs, elle a été aussi le moyen bienvenu d'une fuite devant le conflit dans ma propre vie, et d'une longue stagnation spirituelle.




