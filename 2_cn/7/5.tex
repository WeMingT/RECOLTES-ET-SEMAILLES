\section{(20) Un monde sans conflit?}

J'avais pensé parler du "marais" en quelques lignes, par acquit de conscience, juste pour dire qu'il était là mais que je ne le fréquentais pas - et comme souvent dans la méditation (et aussi dans le travail mathématique); le "rien" qu'on regarde s'est révélé riche de vie et de mystère, et de connaissance jusque-là négligée. Comme cet autre "rien", qui se situait aussi à Nancy comme par hasard (décidément le berceau de ma nouvelle identité !), le "rien" de cet élève un peu nul sûrement qui se faisait traiter fallait voir comme... J'y ai repensé en flash tantôt, quand j'ai écrit (un peu vite peut-être ?) que "ces disgrâces", ça n'existait pas encore "chez nous". Disons que c'est là le seul et unique incident du genre que je puisse rapporter, qui ressemble (il faut bien le reconnaître) à la "disgrâce" à laquelle je faisais allusion, sans trop m’appesantir sur une description circonstanciée. Ceux qui l'ont subie savent bien de quoi je veux parler, sans avoir à faire de dessin. Et aussi ceux qui, sans l'avoir subie, ne s'empressent pas de fermer les yeux chaque fois qu'ils y sont confrontés. Quant aux autres, ceux qui méprisent à coeur joie comme ceux qui se contentent de fermer les yeux (comme je le fis moi-même avec succès pendant vingt ans), même un album de dessins serait peine perdue...

Il me reste à examiner mes relations personnelles et professionnelles à mes collègues et à mes élèves, pendant ces deux décennies, et incidemment aussi, ce que j'ai pu connaître des relations de mes collègues les plus proches entre eux, et avec leurs élèves. La chose qui me frappe le plus aujourd'hui, c'est à quel point il semblerait que le conflit ait été absent de toutes ces relations. Je dois ajouter aussitôt que c'est là une chose qui dans ce temps-là me semblait toute naturelle - un peu comme la moindre des choses. Le conflit, entre gens de bonne volonté, mentalement et spirituellement adultes et tout ça (la moindre des choses, encore une fois !), n'avait pas lieu d'être. Quand conflit il y avait quelque part, je le regardais comme une sorte de regrettable malentendu : avec la bonne volonté de rigueur et en s'expliquant, ça ne pourrait qu'être réglé dans les plus brefs délais et sans laisser de traces ! Si j'ai choisi dès mon jeune âge la mathématique comme mon activité de prédilection, c'est sûrement parce que je sentais que c"est dans cette voie-là que cette vision du monde avait le plus de chances de ne pas se heurter à chaque pas à des démentis troublants. Quand on a démontré quelque chose, après tout ; tout le monde est mis d'accord c'est-à-dire les gens de bonne volonté et tout ça, s'entend.

Il se trouve que j'avais bien senti juste. Et l'histoire de ces deux décennies passée dans la quiétude du monde "sans conflit" (?) de ma chère "communauté mathématique", est aussi l'histoire d'une longue stagnation intérieure en moi, yeux et oreilles bouchés, sans rien apprendre sauf des maths ou peu s'en faut - alors que dans ma vie privée (d'abord dans les relations entre ma mère et moi, puis dans la famille que j'ai fondée sitôt après sa mort) sévissait une destruction silencieuse qu'en aucun moment pendant ces années je n'ai osé regarder. Mais c'est là une autre histoire. . . Le "réveil" de 1970, dont j'ai parlé souvent dans ces lignes, a été un tournant non seulement dans ma vie de mathématicien, et un changement radical de milieu, mais un tournant aussi (à une année près) dans ma vie familiale. C'est l'année aussi où pour la première fois, au contact de mes nouveaux amis, je risquais un coup d'oeil occasionnel, bien furtif encore, sur le conflit dans ma vie. C'est le moment où un doute a commencé à poindre en moi, qui a mûri au long des années qui ont suivi, que le conflit dans ma vie, et celui aussi que parfois j'appréhendais dans la vie d'autrui, n'était pas qu'un malentendu, une "bavure" qu'on enlevait avec un coup d'éponge.

Cette absence (au moins relative) de conflit, dans ce milieu que j'avais choisi comme mien, me paraît rétrospectivement une chose assez remarquable, alors que j'ai fini par apprendre que le conflit fait rage partout où vivent des humains, dans les familles tout comme sur les lieux de travail, que ceux-ci soient des usines, des laboratoires ou des bureaux de professeurs ou d'assistants. Il semblerait presque que je sois tombé pile, en Septembre ou Octobre 1948, débarquant à Paris sans me douter de rien, sur l'îlot paradisiaque et unique dans l' Univers, où les gens vivent sans conflit les uns avec les autres!

La chose tout d'un coup me semble vraiment extraordinaire, après tout ce que j'ai appris depuis 1970. Sûrement elle mérite d'être regardée de plus près - est-ce un mythe, ou une réalité ? Je vois bien l'affection qui circulait entre tant de mes amis et moi, et plus tard entre des élèves et moi, je n'ai pas à l'inventer - mais il semblerait presque que je sois obligé d'inventer du conflit, dans ce monde paradisiaque d'où le conflit semble banni !

C'est vrai, dans cette réflexion j'ai eu l'occasion quand même d'effleurer deux situations de conflits, comme révélateurs à chaque fois d'une attitude intérieure en moi : L'un est l'incident de "l'élève nul" à Nancy, dont j'ignore les tenants et aboutissants entre les protagonistes directs. L'autre est une situation de conflit en moimême, une division, dans ma relation à "l'ami infatigable" - mais celle-ci ne s'est jamais exprimée sous forme d'un conflit entre personnes, la seule forme du conflit généralement reconnue. Chose remarquable, au sens conventionnel du terme, la relation entre ces amis et moi a été entièrement exempte de conflit - elle n'a à aucun moment connu le moindre nuage. La division était en moi, non en eux.

Je continue le recensement. Une des premières pensées : le groupe Bourbaki ! Pendant les années où j'y participais plus ou moins régulièrement, donc jusque vers la fin des années cinquante, ce groupe incarnait pour moi l'idéal d'un travail collectif fait dans le respect aussi bien du détail en apparence infime dans ce travail lui-même, que de la liberté de chacun de ces membres. A aucun moment, je n'ai senti parmi mes amis du groupe Bourbaki l'ombre d'une velléité de contrainte, que ce soit sur moi ou sur quiconque d'autre, membre chevronné ou invité, venu à l'essai pour voir si ça allait "accrocher" entre lui et le groupe. A aucun moment, l'ombre d'une lutte d'influence, que ce soit à propos de différences de points de vue sur telle ou telle question à l'ordre du jour, ou une rivalité pour une hégémonie à exercer sur le groupe. Le groupe fonctionnait sans chef, et personne apparemment n'aspirait en son for intérieur, pour autant que j'aie pu m'en apercevoir, à jouer ce rôle. Bien entendu, comme dans tout groupe, tel membre exerçait sur le groupe, ou sur tels autres membres, un ascendant plus grand que tel autre. Weil jouait à ce sujet un rôle à part, dont j'ai parlé. Quand il était présent, il faisait un peu "meneur de jeu"\footnote{(14)\par On pourrait penser que cela contredit l'affirmation de l'absence de chef, alors qu'il n'en est rien. Pour les anciens de Bourbaki, il me semble que Weil était perçu comme l'âme du groupe, mais jamais comme un "chef". Quand il était là et quand il lui plaisait, il devenait "meneur de jeu" comme j'ai dit, mais il ne faisait pas la loi. Quand il était mal luné il pouvait bloquer la discussion sur tel sujet qu'il avait en aversion, quitte à reprendre le sujet tranquille à un autre congrès quand Weil n'était pas là, voire même le lendemain quand il ne faisait plus obstruction. Les décisions étaient prises à l'unanimité des membres présents, considérant qu'il n'était nullement exclu (ni même rare) qu'une personne soit dans le vrai contre l'unanimité de toutes les autres. Ce principe peut sembler aberrant pour un travail en groupe. La chose extraordinaire, c'est que ça marchait pourtant !})(14). Deux fois je crois, ma susceptibilité s'en était offusquée, et je suis parti - ce sont les seuls signes de "conflit" dont j'aie eu connaissance. Progressivement, Serre a exercé sur le groupe un ascendant comparable à celui de Weil. Du temps où je faisais partie de Bourbaki, cela n'a pas donné lieu à des situations de rivalité entre les deux hommes, et je n'ai pas eu connaissance d'une inimitié qui se serait établie entre eux plus tard. Avec le recul de vingt-cinq années encore, Bourbaki, tel que je l'ai connu dans les années cinquante, me semble toujours un exemple de réussite remarquable au niveau de la qualité des relations, dans un groupe formé autour d'un projet commun. Cette qualité du groupe m'apparaît d'une essence plus rare encore que la qualité des livres qui en sont sortis. Cela a été un des nombreux privilèges de ma vie, comblée de privilèges, que d'avoir fait la rencontre de Bourbaki, et d'en avoir fait partie pendant quelques années. Si je n'y suis pas resté, ce n'est nullement par suite de conflits ou parce que la qualité dont j'ai parlé se serait dégradée, mais parce que des tâches plus personnelles m'attiraient plus fortement encore, et que je leur ai consacré la totalité de mon énergie. Ce départ d'ailleurs n'a jeté d'ombre ni sur ma relation au groupe, ni sur ma relation à aucun de ses membres.

Il me faudrait passer en revue les situations de conflit dans lesquelles j'ai été impliqué, qui m'ont opposé à un de mes collègues ou un de mes élèves, entre 1948 et 1970. La seule chose qui ressorte tant soit peu, ce sont les deux brouilles passagères avec Weil, dont il a déjà été question. Quelques ombres passagères, très passagères sur mes relations à Serre, à cause de ma susceptibilité vis à vis d'une certaine désinvolture parfois déconcertante qu'il avait à couper court quand un entretien avait fini de l'intéresser, ou à exprimer son manque d'intérêt, voire son aversion pour tel travail dans lequel j'étais engagé, ou telle vision des choses sur laquelle j'insistais, peut-être un peu trop et trop souvent ! Ça n'a jamais pris l'ampleur d'une brouille. Au-delà des différences de tempérament, nos affinités mathématiques étaient particulièrement fortes, et il devait sentir comme moi que nous nous complétions l'un l'autre.

Le seul autre mathématicien auquel j'aie été lié par une affinité comparable et même plus forte, a été Deligne. A ce propos, me vient le souvenir que la question de la nomination de Deligne à l' IHES en 1969 a donné lieu à des tensions, que je n'ai pas perçues alors comme un "conflit" (lequel se serait exprimé disons par une brouille, ou par un tournant dans une relation entre collègues).

Il me semble que j'ai fait le tour - qu'au niveau du conflit entre personnes, visible par des manifestations tangibles, dans les relations entre collègues ou entre collègues et élèves dans le milieu que je hantais, c'est tout pendant ces vingt-deux ans, si incroyable que cela puisse paraître. Autant dire, pas de conflit dans ce paradis que j'avais choisi - donc, faut-il croire, pas de mépris? Une contradiction encore dans les mathématiques?

Décidément, il faudra que j'y regarde de plus près !

