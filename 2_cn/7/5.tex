\section{(20) 无冲突之世界?}

我原以为只需寥寥数行提及“沼泽”,以示问心无愧,仅说它存在但我未曾涉足——正如冥想(以及数学工作)中常有的那样,那被注视的“虚无”却显露出丰饶的生命、神秘与此前被忽视的认知。如同另一个“虚无”,恰巧也在南锡「Nancy;Nancy」(显然是我新身份的摇篮!),那个想必有些愚钝的学生被对待的方式,真是令人瞠目……我刚才一闪念回想起这事,在写下“这些不幸”在我们这里尚不存在时(或许有些仓促?)。姑且说这是我能记起的唯一类似事件,颇似我所提及的“不幸”,但我并未过多停留于详述细节。经历过的人自知我在说什么,无需画图说明。那些未曾经历却不急于每次面对时闭眼的人也是如此。至于其他人,那些尽情轻蔑或如我一样二十年成功闭眼的人,即便给他们一整本画册也无济于事……

我仍需审视这两十年间,我与同事及学生的个人与职业关系,顺便也包括我所了解的亲近同事之间及其与学生之间的关系。如今最令我震撼的,是这些关系中冲突似乎全然缺席。我必须立刻补充,这在当时对我而言完全自然——仿佛理所当然。冲突,在怀有善意、心智与精神成熟的人之间(再次强调,理所当然!),不应存在。若某处出现冲突,我视其为一种遗憾的误会:只要有必要的善意并加以解释,定能在最短时间内解决,不留痕迹!我自幼选择数学作为心仪之事,想必是因为我感到,这条道路最能让这种世界观免于每迈一步便遭遇令人不安的反驳。毕竟,一旦证明了某事,所有人都达成一致——当然,指的是那些怀有善意的人。

事实证明,我的直觉没错。这二十年在亲爱的“数学共同体”那“无冲突”(?)世界的宁静中度过的历史,也是我内心长久停滞的历史,耳目闭塞,除数学外几乎一无所学——而与此同时,我的私人生活(先是母亲与我之间的关系,后是我在她去世后立即组建的家庭)正悄然走向毁灭,这一切我在那些年从未敢于正视。但这是另一个故事……我常在此提及的1970年“觉醒”,不仅是数学家生涯的转折点与环境的剧变,也(前后不过一年)是我家庭生活的转折点。那一年,在新朋友的接触中,我首次冒险偶尔瞥一眼生活中的冲突,尽管仍极为隐秘。也是那时,我心中开始萌生一丝怀疑,在随后岁月里逐渐成熟:我生命中的冲突,以及我偶尔在他者生命中察觉的冲突,不仅仅是误会或可用一抹海绵擦去的“瑕疵”。

这种(至少相对的)冲突缺席,在我所选择的环境中,回想起来颇为引人注目。而我后来得知,冲突在人类生活的每个角落——家庭与工作场所,无论是工厂、实验室还是教授与助教的办公室——皆肆虐不休。我几乎像是1948年九月或十月,懵懂抵达巴黎时,恰好落入宇宙中那座独特的天堂孤岛,在那里,人们彼此无冲突地共存!

在1970年之后所知的一切映衬下,这事突然显得格外非凡。它无疑值得更仔细审视——是神话,还是现实?我清楚记得许多朋友与我之间流淌的深情,后来学生与我之间也是如此,这无需虚构——但我几乎得凭空捏造冲突,在这看似逐出冲突的天堂世界中!

诚然,在此反思中,我确曾触及两起冲突情境,每每揭示我内心的某种态度:一是南锡那“愚钝学生”的事件,我不知直接相关者间的始末。另一是我与“不倦的朋友”关系中的内在冲突,一种撕裂——但这从未以人与人之间的冲突形式表达,那是通常被认可的唯一冲突形式。值得注意的是,按常规定义,这几位朋友与我的关系完全无冲突——从未蒙上一丝阴云。撕裂在我心中,而非在他们。

我继续清点。第一个念头:布尔巴基学派「Bourbaki;Bourbaki」!在我或多或少定期参与的岁月里,直到五十年代末,这个群体对我而言是集体工作的理想体现,既尊重工作中看似微不足道的细节,也尊重每个成员的自由。我从未在布尔巴基的朋友间感到一丝强制的阴影,无论是对我还是其他任何人——无论是资深成员还是受邀试水的客人,探寻他与群体是否“契合”。从未有权力斗争的痕迹,无论是因议题上的观点分歧,还是为争夺群体主导权的竞争。群体无领袖运作,似乎无人内心渴望扮演此角色,至少我未察觉。如同任何群体,某成员对群体或某些其他成员的影响大于他人。韦伊「Weil;Weil」在这方面角色独特,我曾提及。他在场时,略似“游戏引领者”。\footnote{(14)\par 有人或以为这与无领袖之说相悖,其实不然。对布尔巴基的老成员而言,我觉得韦伊被视为群体之魂,却从未是“领袖”。他在场且兴致高时,如我所说,成为“游戏引领者”,但他并不定律。若心情不佳,他可能阻挠某个他厌恶的话题讨论,但待他不在的另次会议,或甚至次日他不再阻挠时,话题便平静重启。决议由在场成员一致通过,且不排除(甚至不罕见)一人对抗全体仍为正确。这原则看似不适于群体工作。奇妙的是,它却行得通!}(14) 我记得两次,我的敏感因此受挫,我便离开——这是我所知的唯一“冲突”迹象。渐渐地,塞尔「Serre;Serre」对群体的影响与韦伊相当。在我参与布尔巴基期间,这未引发两人间的竞争,我也不知后来他们之间是否生出敌意。再过二十五年回望,我在五十年代所知的布尔巴基,仍是围绕共同目标形成的群体中关系品质卓越的范例。这种群体品质,在我看来,比其产出的书籍品质更珍稀。能遇见布尔巴基并参与数年,是我充满特权的生命中诸多特权之一。我未留下,非因冲突或品质衰退,而是更个人的任务更强地吸引我,我将全部精力投入其中。这离开未给群体或我与任一成员的关系蒙上阴影。

我需回顾1948至1970年间,我卷入的与同事或学生对立的冲突情境。唯一稍显突出的是与韦伊的两次短暂不和,已提及。还有与塞尔关系上的几抹极淡阴影,因我对他偶尔令人困惑的轻率敏感——他会在对话不再吸引他时 abruptly 结束,或表达对某项我投入的工作或我过于坚持的观点的兴趣缺失乃至厌恶!这从未发展成不和。超越性情差异,我们的数学亲和力极强,他想必和我一样感到彼此互补。

另一位与我有类似甚至更强亲和力的数学家是德利涅「Deligne;Deligne」。由此忆起,1969年德利涅获IHES提名时确有紧张氛围,我当时未视为“冲突”(如以不和或同事关系转折表达)。

我想我已梳理完毕——在人与人之间,以 tangible 形式显现的冲突,在我所处环境中同事间或师生间,二十二年间仅此而已,难以置信却如此。换言之,我选择的天堂无冲突——那么,是否该信,也无轻蔑?数学中的又一矛盾?

显然,我得更仔细探究!