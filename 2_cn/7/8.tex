\section{(23) 深渊之声}

这个“布尔巴基式”的环境无疑对我个人及我对世界与自身定位的看法产生了深远影响。此处并非试图界定这种影响及其在我生命中的表达之地。我仅想说,我并不觉得我对虚荣的倾向及其精英主义的合理化,是因接触布尔巴基学派「Bourbaki;Bourbaki」或融入“布尔巴基环境”而被激发——至少在四十年代末与五十年代并非如此。这些种子早已在我内心埋下,在任何其他环境中都可能找到发芽的机会。我提及的“愚钝学生”事件绝非这一环境氛围的典型,恰恰相反,我再次强调,它仅反映了我自身的一种矛盾态度。布尔巴基的氛围是对人的尊重,是自由的氛围——至少我如此感受;这种氛围天然抑制并缓和任何走向支配或虚荣的倾向,无论是个体还是集体的。

这个品质卓越的环境已不复存在。它何时消逝,我无从知晓,想必无人察觉,也无人暗自为它敲响丧钟。我猜想,一种不易察觉的退化在人身上悄然发生——我们都逐渐“老练”起来,变得沉稳。我们成了重要人物,受人倾听,拥有权力,被人畏惧与追逐。火花或许仍在,但纯真已在途中遗失。或许有人会在死前找回它,如同一次新生,但那个接纳我的环境已逝,我若期待它复活,徒劳无益。一切已回归常态。

或许尊重也在途中遗失。当我们有了学生时,传递最好的东西或许已太迟——火花仍在,但纯真不再,尊重亦然,除非是对“同侪”与“自己人”。

风可骤起,吹刮炽烧——我们各据厚墙庇护,与“自己人”同在。

一切已回归常态……