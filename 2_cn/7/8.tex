\section{(23) De Profundis}

Ce milieu "bourbachique" a sûrement exercé une forte influence sur ma personne et sur ma vision du monde et de ma place dans le monde. Ce n'est pas le lieu ici d'essayer de cerner cette influence, et comment elle s'est exprimée dans ma vie. Je dirais seulement qu'il ne me semble nullement que mes penchants vers la fatuité, et leurs rationalisations méritocratisantes, aient été stimulés par mon contact avec Bourbaki et par mon insertion dans le "milieu bourbachique" - tout au moins pas à la fin des années quarante et dans les années cinquante. Les germes en avaient été semés de longue date en moi, et auraient trouvé occasion à se développer dans tout autre milieu. L'incident de "l'élève nul" que j'ai rapporté n'est nullement typique, bien au contraire, d'une ambiance qui aurait prévalu dans ce milieu, je le répète, mais uniquement d'une attitude ambiguë en ma propre personne. L'ambiance dans Bourbaki était une ambiance de respect pour la personne, une ambiance de liberté - c'est ainsi du moins que je l'ai ressenti ; et elle était de nature à décourager et à atténuer tout penchant vers des attitudes de domination ou de fatuité, qu'elles soient individuelles ou collectives.

Ce milieu de qualité exceptionnelle n'est plus. Il est mort je ne saurais dire quand, sans que personne, sans doute, ne s'en aperçoive et en sonne le glas, même en son for intérieur. Je suppose qu'une dégradation insensible a dû se faire dans les personnes - on a tous dû "prendre de la bouteille", se rassir. On est devenus des gens importants, écoutés, puissants, craints, recherchés. L'étincelle peut-être y était encore, mais l'innocence s'est perdue en route. Tel d'entre nous la retrouvera peut-être avant sa mort, comme une nouvelle naissance mais ce milieu qui m'avait accueilli n'est plus, et il serait vain que je m'attende qu'il ressuscite. Tout est rentré dans l'ordre.

Et le respect aussi peut-être s'est perdu en route. Quand nous avons eu des élèves, c'était peut-être trop tard pour que le meilleur se transmette - il y avait une étincelle encore, mais plus l'innocence, ni le respect, sauf pour "ses pairs" et pour "les siens".

Le vent peut se lever et souffler et brûler - nous sommes à l'abri derrière d'épaisses murailles, chacun, avec "les siens".

Tout est rentré dans l'ordre...