\section{(16) 沼泽与前排}

但我尚未完成关于我在轻蔑的出现及其在这个世界中的演进中所扮演角色的反思,我依然愉快地称这个世界为“数学共同体”。现在我感到,这份反思是我能给予这个世界中我所深爱之人最好的东西,此时我正准备——并非重返其中,而是再次在此表达自己。

我认为,我仍需审视的主要是,在我尚属这个世界一员的那些岁月里,我与其中的各色人等保持着怎样的关系。

现在回想起来,我震惊于一个事实:在这个世界中,有一部分我虽定期接触,却仿佛从未进入我的注意,仿佛它并不存在。那时我一定将这一部分视为某种“沼泽”,在我心中没有明确的功能,甚至连“共鸣箱”的作用都未必有——就像一团灰色的、匿名的群体,在研讨会和学术会议中,他们总是坐在最后几排,仿佛生来就被指定如此,他们在报告中从不开口冒险提问,想必事先确信自己的问题只会离题。如果他们向像我这样被认为“紧跟潮流”的人提问,那也是在走廊里,当显而易见“权威人士”们并未打算彼此交谈时——他们会匆匆提问,仿佛踮着脚尖,羞于占用我们这些重要人物的宝贵时间。有时问题确实显得离题,我便尽量(我想)用几句话解释原因;更多时候,问题切中要害,我也尽力作答,我想是这样。无论哪种情况,这种情境下(或者更该说,这种氛围下)提出的问题很少会有后续的第二问,去 уточ 或深化它。或许我们这些“前排之人”在那种时候确实过于匆忙(即便我们有时显然努力不表现出来),以至于对面的恐惧无法消散,也无法让真正的交流诞生。我当然能感觉到,正如我的对话者也从他那边感受到,我们身处的这种情境有多虚假、多矫饰——只是我从未当时明确表达出来,而他,想必也未曾如此。彼此间,我们如同奇怪的自动机器运作着,一种奇特的共谋将我们 связ在一起:假装忽视那紧攫其中一方的焦虑,而另一方模糊地察觉到它——这缕焦虑是弥漫于会场那茫然焦虑氛围的一部分,所有人无疑都如我们一样感知到,却都默契地选择视而不见。\footnote{(13)\par 显然,前述描述并无更多意图,仅是试图通过笨拙的言辞,尽可能还原记忆中那团“迷雾”所呈现的景象。这迷雾从未在当时凝结成任何稍显具体的时刻,让我能在此给出哪怕略为“现实”或“客观”的描述。若将这段文字解读为,那些不愿坐上前排、或未拥有明星或权威地位的同事,在与后者交谈时必然满怀焦虑,那是对我意图的扭曲。对我在这个环境中认识的大多数朋友而言,显然并非如此,即便其中一些人常会出现在学术会议或研讨会上。毫无保留的真相是,“权威”地位制造了一道屏障、一条鸿沟,与那些缺乏类似地位的人隔开,而这鸿沟很少会缩小,哪怕只是在一次讨论的片刻。我还需补充,那种主观上(却在我看来真实)的“前排”与“沼泽”之分,绝不能简单归结为社会学标准(社会地位、职位、头衔等),甚至也不是“地位”或名声,而是涉及性情或更难界定的心理特质的细微差异。当我二十岁抵达巴黎时,我知道自己是个数学家「Mathématicien;Mathematician」,我做过数学,尽管有过我曾提及的超越感,我内心深处仍觉得自己是“他们中的一员”,只是我独自知晓这一点,甚至最初还不确定是否会继续从事数学。如今,若有机会(这种机会已然稀少),我更倾向于坐在最后几排。}(13)

这种对焦虑的模糊感知,直到1970年第一次“觉醒”后的日子,才在我心中变得清晰。那时,这片“沼泽”从我此前乐于让其沉浸其中的昏暗中浮现出来。这并非出于某种刻意的决定,也非我当时立即意识到的转变,我只是悄然离开了一个环境,进入了另一个——从“前排之人”的圈子进入了“沼泽”——突然间,我的大多数新朋友,竟是一年前我还轻易归入那片无名无形之地的那些人。所谓的沼泽骤然鲜活起来,通过与我因共同冒险而相连的朋友们的面孔焕发生机——那是另一场冒险!