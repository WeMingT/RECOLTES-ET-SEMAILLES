\section{(24) 我的告别,或:异乡人}

这场对数学家生涯的回顾,走上了一条我未曾预料的路径。实话说,我并未打算回顾,只是想用几行字,至多一两页,述说如今我与那已离弃的世界之关系,或许也反过来,根据偶尔传来的回音,谈谈昔日朋友与我的关系。相反,我原打算更仔细审视我在那些数学工作密集岁月里引入的某些观念与概念的奇特际遇——更该说,是我有幸瞥见并从全然未知的暗夜中拉至幽影,有时甚至至白昼最明亮之光的新对象与结构类型!如今这意图似与这场冥想格格不入,这冥想关乎过去,旨在更理解并承担某个有时令人困惑的现在。那关于某“几何学派”的预定反思——它在我推动下成型,却几乎未留痕迹地消散——将等待更合适的时机。\footnote{这“更合适的时机”来得比预期早,该反思成为《收获与播种》第二部分“葬礼”「L’Enterrement;The Burial」的主题。} 眼下,我的关注是完成这场关于我在数学家世界中数学家生涯的回顾,而非赘述某作品及其命运。

过去五天,被其他任务占据,未写这些反思笔记,一个记忆却执拗浮现。它将作为我停笔于《深渊之声》的尾声。

那是1977年末。几周前,我因“无偿收留并供养一名非法居留的外国人”(即在法居留证件不齐的外国人)被传唤至蒙彼利埃「Montpellier;Montpellier」矫正法庭。借此传唤,我才知晓1945年外国人管理条例中那令人震惊的条款,禁止任何法国人以任何形式援助“非法居留的外国人”。此法,即便在希特勒德国对犹太人亦无类似规定,显然从未按字面执行。却因极怪的“巧合”,我荣幸成为首个试验品,见证这独特条款首次生效。

数日间,我惊愕不已,如遭麻痹,深陷沮丧。骤然间,我仿佛回到三十五年前,那时生命——尤其是外国人的生命——轻如鸿毛……随后我振作起来。几月间,我倾尽全力,试图动员舆论,先在我的大学与蒙彼利埃,后在全国。正是在这为一场后来证明无望的事业奔走的紧张时期,发生了我今日可称为“告别”的事件。

为在全国采取行动,我写信给五位科学界知名“人物”(包括一位数学家),告知他们这条至今仍令我难以置信的法律。我信中提议联合行动,反对这恶法,它等同于将数十万在法外国人置于法外,并将数百万其他外国人如麻风病人般推向公众猜忌,使之成为嫌疑人,可能给不加防范的法国人带来最糟的麻烦。

令人惊讶且完全出乎意料的是,这五位“人物”无一人回信。显然,我还有 многое要学……

于是我决定赴巴黎,借布尔巴基研讨会「Séminaire Bourbaki;Bourbaki Seminar」之机——我定会遇见众多旧友——先在最熟悉的数学界动员舆论。我以为,这环境对外国人境遇尤为敏感,因所有数学家同事,如我一般,每日与外国同事、学生相处,他们大多——若非全部——都曾因居留证件遭遇困难,在警署走廊与办公室面对专横乃至轻蔑。洛朗·施瓦茨「Laurent Schwartz;Laurent Schwartz」知我计划后,表示将在首日报告结束后给我发言机会,向在场同事陈述情况。

那天我抵达,手提箱塞满传单,准备分发给同事。阿兰·拉斯库「Alain Lascoux;Alain Lascoux」协助我在亨利·庞加莱研究所「Institut Henri Poincaré;Henri Poincaré Institute」走廊分发,在首场前及两场报告间“幕间休息”时。若我记得没错,他还自制了一小份传单——他是少数两三位听闻此事后动容、在我赴巴黎前联系我并提议帮助的同事之一。\footnote{(17)\par 我为之事奔走时,主要在科学界外遇到热烈回响与实际援助。除阿兰·拉斯库与罗杰·戈德芒「Roger Godement;Roger Godement」的友好支持,还须特别提及让·迪厄多内「Jean Dieudonné;Jean Dieudonné」,他专程赴蒙彼利埃,出席矫正法庭听证,为这无望事业与其他证人一道,献上温暖证词。}(17) 罗杰·戈德芒也在其中,他制作的传单标题是“诺贝尔奖得主入狱?”。他真够义气,但显然我们不在同一波长:仿佛丑闻在于针对“诺贝尔奖得主”,而非任意无名小卒!

首日布尔巴基研讨会果然人潮汹涌,许多我或深或浅认识的人,包括昔日布尔巴基的友人与同伴,我想大多都在场。还有几位旧学生。快十年未见这些人,我很高兴借此机会重逢,即便人数众多!总会找到机会小范围聚聚……

然而,重逢“并非如此”,从一开始便很明显。许多手伸出握手,确是如此,许多人问“嘿,你怎么在这儿,什么风把你吹来?”,是的——但欢快语气背后,有种难以名状的尴尬。是因我带来的事业本质上不吸引他们——他们是为这场三年一度的数学盛典而来,全神贯注?还是无论我为何而来,我本身便引发这尴尬,恰如一个叛教神父在纯正神学生中引起的局促?我无从分辨——或许两者皆有。我这边,则不禁注意到某些曾熟悉乃至友好的面孔所发生的转变。它们似已僵硬,或塌陷。我曾熟悉的那种灵动仿佛消失,像是从未存在。我如面对异乡人,仿佛从未与他们有任何联结。冥冥中,我感到我们已不在同一世界。我原以为这次特殊机缘能找回兄弟,却发现面对的是异乡人。得承认,他们很有教养,我不记得有尖酸评论,也不记得传单被扔在地上。实际上,分发的传单(或几乎全部)想必都被阅读,出于好奇。

这并未动摇那恶法的根基!我有五分钟,或许用了十分钟,谈及那些对我如兄弟、却被称为“外国人”者的处境。满满的阶梯教室,同事们比我讲数学报告时更沉默。或许我对他们诉说的信念已不再。如昔日那般,同情与兴趣的流动已逝。想必有人赶时间,我心想,便缩短发言,提议立即与关心的同事聚首,更具体商讨可行之事……

会议结束宣布散场时,众人如潮水般涌向出口——显然,每人都有即将出发的火车或地铁,绝不能错过!一两分钟内,埃尔米特「Hermite;Hermite」阶梯教室空无一人,堪称奇迹!偌大的空教室,刺眼灯光下,我们三人独留,包括我和阿兰。我不认识第三人,猜是个不愿承认身份的外国人,伴着可疑之人,且居留状况不合法!我们未多费唇舌评论眼前这颇具说服力的场景。或许唯我不信自己的眼睛,两位朋友则体贴地未置一词。显然,我刚“登陆”……

当晚在阿兰及其前妻杰奎琳家中结束,我们评估形势,检视可行之事,也略作彼此了解。那日或之后,我未暇将此经历置于过往中定位。然而,那天我定是在无言中明白,我曾熟识并深爱的某个环境、某个世界已逝,那我曾期望重拾的活泼温暖,想必早已消散。

这并未阻止那已失温暖的世界,年复一年传来的回音,仍多次令我困惑,痛楚触及。我疑这场反思能否改变未来——或许,仅是让我不再如此触动时过于抗拒……