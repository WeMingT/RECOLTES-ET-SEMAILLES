\section{(6) 梦者}

事实上,我凭经验深知,当内心渴求认知而非逃避(或手持专利模板应对——实则无异于逃避),梦绝非不愿“成形”——它乐于被轻柔描绘,传递其讯息,总是简单,从不愚蠢,有时令人震撼。恰恰相反,我们内心的梦者「Rêveur;Dreamer」是无与伦比的大师,总能因时制宜,或全新创造最适切的语言,绕过我们的恐惧,摇醒我们的昏沉。其舞台手段无穷变幻:从全无视觉或感官元素,到令人叹为观止的场景布置。当祂显现,绝非为隐匿,而是鼓励我们(虽几乎总徒劳,祂的仁慈却从不倦怠……)走出自我,摆脱祂所见的沉重——有时祂不动声色,以滑稽色彩戏仿这沉重。倾听内心的梦者,便是与自我对话,抗衡那极力阻隔的强大壁垒。

能者多劳,能大亦能小。若我们能通过梦与自我沟通,揭示自我,那么以同样简单的方式,向他人传递数学之梦那并无隐私的讯息,理应可行——它所激发的阻力远非同等量级。实话说,我作为数学家的过去,不正是追随、“梦”至极致,直至它们以最显明、最坚实的形式呈现——无可辩驳——那些从浓雾密织中一片片剥离的梦之碎片吗?我曾多少次因执意将每块凝结自梦的宝石或半宝石打磨至最后一面而焦躁不安,而非顺从更深的冲动:追随母体织物的多形奥秘——在梦与其可“发表”的明证化身之间那模糊边界!彼时我正欲顺此冲动,投身“数学科幻”工作,一种关于“动机”「motifs;motives」理论的“清醒之梦”——当时纯属假设,至今仍如此,因缺乏另一位“清醒梦者”接续此冒险。那是六十年代末,我浑然不觉,生活即将全然转向,接下来十年,我的数学激情被边缘化,甚至遭弃。

但细想之下,《追寻场》「A la Poursuite des Champs;In Pursuit of Fields」——这沉默十四年后的首篇发表之作——实承接了那未曾书写的“清醒之梦”之精神,至少暂时如此。诚然,这两个梦的主题乍看截然不同,数学主题间似难再有更大差异;更别提前者(动机)似处于现有手段“可实现”的地平线,而后者(所谓“场”及其同类)看似触手可及。这些差异可称偶然或表面,或许消散得比预期更快 \footnote{(3)\par 我尤其想到已故的莫德尔「Mordell;Mordell」、泰特「Tate;Tate」、沙法列维奇「Chafarévitch;Shafarevich」猜想,去年法尔廷斯「Faltings;Faltings」以四十页手稿证明三者,当时“圈内”共识却认为它们“遥不可及”!我珍视的“阿贝尔代数几何”纲领的关键“基本猜想”,恰与莫德尔猜想相近。(甚至似后者为其推论,足见此纲领非严肃人故事……)} (3)。在我看来,这些差异对两主题引发的工种影响相对有限——只要是“清醒之梦”,或用较温和的说法:将概念粗加工至整体愿景,具足够连贯性与精确性,令人或多或少确信其本质契合事物现实。对本书主题而言,这大致意味着验证此愿景有效性,成纯粹技艺之事。虽需大量工作,兼具巧思与想象,且无疑有意外转折与新视角,使之远非纯例行(安德烈·韦伊「André Weil;André Weil」所谓“长练习”),幸莫如此。

这正是我过去反复饱尝、烂熟于心的工作,在余年无需重做。若我再次投入数学,精力最佳去处必在“清醒之梦”的边界。此选择非出于效益考量(假若有人受此驱动),而是源于梦,或多个梦。若这新动力在我体内焕发力量,必自梦中汲取!