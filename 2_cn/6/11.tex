\section{(15) 收获与播种}

我不再考虑,也不会再激愤于那阵吹过的风,当我已清楚看到,我并非与这风毫无关联,正如我内心的虚荣曾多么希望让我相信的那样。即使我真的与它无关,我的愤怒对那些被羞辱的人和那些羞辱他人的人——我同样深爱着这两者——也不过是微不足道的献礼。

我并非与这风毫无瓜葛,因为我在自己选择的世界里,与轻蔑和恐惧共谋。这对我来说很方便,让我对这些瑕疵视而不见,就像对职业生涯和家庭生活中的许多其他瑕疵一样。在这两者中,我收获了自己播下的种子——以及在我之前或与我一起播种的他人所撒下的种子,无论是我的父母(以及我父母的父母……),还是我昔日的新朋友们。而如今,收获这些发芽的种子的人不再只有我,还有我的孩子们(以及我孩子们的孩子们),以及我现在的某个学生,被我昔日的某个学生以轻蔑相待。

在谈论播种与收获时,我心中没有苦涩,没有顺从,也没有自怜。因为我已懂得,即便在苦涩的收获中,也蕴含着实质的养分,我们只需选择汲取它。当这养分被我们食用,成为我们血肉的一部分时,那苦涩便消失了,它不过是我们面对命定食粮时抗拒的表征。

我也知道,没有哪次收获不同时是另一次播种,为更苦涩的收获埋下种子,往往比之前的更加苦涩。面对这看似无尽的链条——一代代传承与接续的漫不经心的播种和苦涩的收获——我内心仍会偶尔感到紧缩。但我不再为此感到压抑或反叛,仿佛面对一种残酷而不可避免的宿命,更不再是那顺从而盲目的囚徒,如我从前那样。因为我明白,在我遭遇的一切中,无论种子是我亲手播下还是他人所为,都蕴藏着滋养的实质——我只需食用它,便能将其转化为认知。对我的孩子们,对我曾深爱和此刻深爱的所有人而言也是如此,当他们收获我在虚荣与轻率之时播下的种子,或我今日偶尔仍在播撒的种子时,亦是如此。