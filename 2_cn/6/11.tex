\section{(15) Récoltes et semailles}

Je ne songe pas, je ne songerais plus à m'indigner d'un vent qui souffle, alors que j'ai vu clairement que je ne suis pas étranger à ce vent, comme une fatuité en moi aurait bien voulu me le faire croire. Et alors même que j'y aurais été étranger, mon indignation aurait été une offrande bien dérisoire à ceux qui sont humiliés comme à ceux qui humilient, et que j'ai aimés les uns comme les autres.

Je n'ai pas été étranger à ce vent, par ma connivence avec le mépris et avec la crainte, dans ce monde que j'avais choisi. Cela m'arrangeait de fermer les yeux sur ces bavures, comme sur bien d'autres, aussi bien dans ma vie professionnelle que dans ma vie familiale. Dans l'une et l'autre, j'ai récolté ce que j'ai semé - et ce que d'autres aussi ont semé avant moi ou avec moi, aussi bien mes parents (et les parents de mes parents...) que mes nouveaux amis d'antan. Et d'autres encore que moi récoltent aujourd'hui ces semailles qui ont levé, aussi bien mes enfants (et les enfants de mes enfants), que tel de mes élèves d'aujourd'hui, traité avec mépris par tel de mes élèves d'antan.

Et il n'y a amertume ni résignation en moi, ni apitoyement, en parlant des semailles et de la récolte. Car j'ai appris que dans la récolte même amère, il y a une chair substantielle dont il ne tient qu'à nous de nous nourrir. Quand cette substance est mangée et qu'elle est devenue part de notre chair, l'amertume a disparu, qui n'était que le signe de notre résistance devant une nourriture à nous destinée.

Et je sais aussi qu'il n'y a récoltes qui ne soient aussi semailles d'autres récoltes, plus amères souvent que celles qui les avaient précédées. Il arrive encore que quelque chose en moi se serre devant la chaîne apparemment sans fin de semailles insouciantes et de récoltes amères, transmise et reprise de génération à génération. Mais je n'en suis plus accablé ni révolté comme devant une fatalité cruelle et inéluctable, et encore moins je n'en suis le prisonnier complaisant et aveugle, comme je le fus naguère. Car je sais qu'il y a une substance nourricière dans tout ce qui m'arrive, que les semailles soient de ma main ou de celle d'autrui - il ne tient qu'à moi de manger et de la voir se transformer en connaissance. Et il n'en est pas autrement pour mes enfants et pour tous ceux que j'ai aimés et ceux que j'aime en cet instant, lorsqu'ils récoltent ce que j'ai semé en des temps de fatuité et d'insouciance, ou ce qu'il m'arrive de semer encore aujourd'hui.


