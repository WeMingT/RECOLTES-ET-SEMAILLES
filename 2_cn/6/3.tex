\section{(7) 伽罗瓦的遗产}

在所有自然科学中,似乎唯有数学将我所谓的“梦”或“清醒之梦”置于近乎绝对、逾两千年的禁忌之下。在其他科学中,包括被誉为“精确”的物理学,梦至少被容忍,甚至(依时代而定)受鼓励,只是冠以更“体面”之名,如“推测”、“假设”(如德谟克利特「Démocrite;Democritus」源于梦——抱歉,源于推测——的著名“原子假设”)、“理论”……从不敢自称之梦到“科学真理”的转变,渐进无痕,通过共识逐步扩大。然而在数学中,至少如今多半如此,这转变骤然发生,仰仗证明的魔法棒一挥 \footnote{(4)\par 即便今日,仍有地位不明的“证明”。如格劳尔特「Grauert;Grauert」为其命名的有限性定理的证明,多年无人能读懂(好心人并非不多!)。此困惑因更透明、甚至更深入的后续证明而解,取代了最初版本。更极端的是“四色问题”的“解”,其计算部分耗费数百万美元借助计算机完成。这“证明”不再基于理解数学情境的内在确信,而依赖一台无理解能力的机器,其结构与运作连数学家使用者亦不知。即便其他计算机以不同程序确认计算,我仍不认为四色问题已了结。它仅面貌一变,不再寻反例,而求可读证明(自不待言!)。} (4)。在数学定义与证明的概念尚未如今日般清晰且(大致)达成共识的时代,重要概念却常具模糊存在——如“负数”(帕斯卡「Pascal;Pascal」拒之)或“虚数”。此模糊至今映于用语。

定义、命题、证明、数学理论概念的逐步澄清,在此极为有益。它让我们意识到手中工具的威力——虽简单如童稚,却能以近乎日常语言的严谨运用,完美表述看似不可言说之物。自幼令我着迷于数学的,正是这以言语捕捉并完美表达数学事物本质的力量——它们初现时如此 elusive 或神秘,仿佛超越言语……

然而,这威力与完美精确及证明提供的资源,其心理副作用堪忧:它进一步强化了传统对“数学之梦”的禁忌,即对任何未以常规精确形式(哪怕牺牲更广视野)呈现、未经规范证明“正宗”担保之物的排斥——或至少(如今愈发如此……)需有可规范化证明的轮廓。偶尔容忍猜想,须满足问卷式精确条件,仅允“是”或“否”答复。(更不用说,提出者须在数学界有声望。)据我所知,从未有以“实验”名义发展一明确假设性的数学理论。诚然,按现代标准,自十七世纪起的“无穷小”计算——后成微积分——堪称清醒之梦,仅在两世纪后经柯西「Cauchy;Cauchy」魔法一挥,化为严肃数学。这让我忆起埃瓦里斯特·伽罗瓦「Evariste Galois;Evariste Galois」的清醒之梦,他未受柯西青睐;但这次不到百年,乔丹「Jordan;Jordan」(若我记得没错)的魔法一挥,使此梦获认可,改名“伽罗瓦理论”「théorie de Galois;Galois theory」。

由此可见,对“1984年数学”不利的是,若牛顿「Newton;Newton」、莱布尼茨「Leibnitz;Leibniz」、伽罗瓦等人(我非史家,定漏许多……)受今规束缚,在彼时仅顾发现而未暇规范化,多亏未然!

伽罗瓦的例子不请自来,触动我心弦。我似忆起初闻他及其奇命运时——或为高中或大学生——便生兄弟般的同情。如他,我怀数学激情;如他,我感自身边缘,与(我以为)拒他的“美好世界”格格不入。然我终融入此界,后无悔离去……这半忘的亲近感最近以全新面貌重现,写《纲领概览》「Esquisse d'un Programme;Sketch of a Program」时(为申请国家科学研究中心「Centre National de la Recherche Scientifique;National Center for Scientific Research」研究员)。此报告主要勾勒近十年反思主题。其中最迷我、拟未来数年深耕者,乃典型数学之梦,与“动机之梦”「rêve des motifs;dream of motives」相连,提供新路径。写此概览,我忆起近十四年最长的一次连续数学反思——1981年1至6月的“穿越伽罗瓦理论的长征”「La longue Marche à travers la théorie de Galois;The Long March through Galois Theory」。由此,我意识到近年断续追逐、终名“阿贝尔代数几何”「géométrie algébrique anabélienne;anabelian algebraic geometry」的清醒之梦,乃伽罗瓦理论的延续,“其终极归宿,或依伽罗瓦之精神”。

此连续性显现时——写下引述那行之际——一股喜悦穿透我身,至今未散。这是完全孤独工作中之回报。其现身如昔日向两三位“圈内”同事兼旧友(其一曾为我学生)倾诉发现时的冷遇般意外,当时我心喜尚热……

今日继承伽罗瓦遗产,定也须接纳他当年的孤独风险。或许时代变迁不如我们所想,此“风险”对我却常非威胁。虽有时因所爱之人佯装冷漠或轻蔑而感伤挫败,但多年未觉数学或其他孤独为重。若有一忠友,我离之便渴盼重逢,便是她!