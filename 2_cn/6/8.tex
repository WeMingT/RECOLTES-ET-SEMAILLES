\section{(12) 功绩与轻蔑}

我想借我有限的个人经验,仔细审视轻蔑何时何以在数学家世界,尤其在我那如第二故乡的同事、朋友与学生“微观世界”中扎根,同时探究我在此转变中的角色。

我可毫无保留地说,1948-49年间,在我先前所述以初代布尔巴基「Bourbaki;Bourbaki」为中心的数学家圈中,我未见丝毫轻蔑、贬低或傲慢痕迹,无论对我或任何前来学数学的年轻人——不论法国人或外邦人。那些因地位或声望居要角者,如勒雷「Leray;Leray」、卡尔唐「Cartan;Cartan」、韦伊「Weil;Weil」,我未畏之,我信我的同侪亦然。除勒雷与卡尔唐颇具“绅士风范”,我颇久方觉那些随意到来、与卡尔唐互称“你”、明显“在行”的家伙,皆如卡尔唐般为大学教授,非我般勉强度日,而是薪资对我而言天文数字,且为国际知名数学家。

依韦伊建议,我后三年在南锡「Nancy;Nancy」度过,时为布尔巴基某种总部,德尔萨特「Delsarte;Delsarte」、迪厄多内「Dieudonné;Dieudonné」、施瓦茨「Schwartz;Schwartz」、戈德芒「Godement;Godement」(后有塞尔「Serre;Serre」)执教大学。仅四五年轻人与我同在(我忆有利翁「Lions;Lions」、马尔格朗热「Malgrange;Malgrange」、布吕阿「Bruhat;Bruhat」、贝尔热「Berger;Berger」,若未混淆),远不如巴黎“淹没于众”。氛围更亲密,皆熟识,似皆互称“你”。然回溯记忆,正是在此,我首见——且唯一见——一数学家对学生公然轻蔑。那倒霉者从外地来一日,与导师工作。(他在备博士论文,后体面通过,似获一定名声。)我颇震惊。若有人对我如此一刻,我必当即摔门而去!此事中,我熟识“导师”,与他称“你”,仅略识学生。我前辈除广博学识(不限于数学)与犀利才智,有种不容置疑的威严,彼时(及至七十年代初)令我敬畏,对我颇有影响。我不忆是否问他态度,仅得出结论:此可怜学生定极差,方“该”如此待——类似此。我未想,若学生真差,应劝其改行、终止合作,而非轻蔑。我认同如前辈般的“数学强者”,以“该”轻蔑的“无能”为代价。我遂走上与轻蔑共谋的现成路,因其便利,凸显我被“有功者”、数学强者的兄弟会接纳!\footnote{(7)\par 前段乃引言中首段于初稿大幅涂改、多重增补处。事件描述、措辞初甚逆意——一股力明显欲匆匆略过,仅为释怀,奔“正事”。此乃熟悉的抗拒迹象,此处针对阐明此事件及其揭示内心态度。情境酷似引言开篇(第2段)所述数学工作中“关键”时刻——发现矛盾及其意义:乃心智惰性,厌弃舍弃错误或不足愿景(我身未涉),扮演“抗拒”角色。此为主动、必要时具创意的掩饰,连水无也淹鱼;而我所述惰性仅被动。此处,远超数学工作,简单明了的发现瞬时带来卸重与解放之感。此非仅情感——乃对刚发生之解放的敏锐感恩感知。}(7)

当然,我与任何人一样,不会明言:尝试数学未成者该被轻蔑!若彼时或他时闻人如此,我必严正驳斥,真心为其惊人精神无知遗憾。实则我陷于矛盾,双面行事互不交融:一面是美好原则与情怀,一面是:可怜家伙,真够差才被如此待(暗指:我绝不会如此倒霉,确定!)。

我终觉所述事件,尤我看似无足轻重的角色,实典型反映我内在矛盾,随我数学人生廿年,至1970年“觉醒”后方消 \footnote{(8)\par 后文将明,此矛盾未“1970年觉醒后即消”。此乃“自我”典型战略退却,弃“觉醒前”于得失,立“觉醒”为“完美后”之界!}(8),此前我未明察,直至今写此行。甚憾彼时未觉。或时机未熟。总之,当时关于轻蔑盛行的见证——我选择视而不见——未涉我身,亦未涉我亲近微观世界的同事与友 \footnote{(9)\par 此不全准,至少一亲近同事例外,后文显。我记忆“偷懒”典型,常略去与根深蒂固熟悉愿景不符之事。}(9)。反倒似叹:啊!得知(或告知你等)此事多可悲,谁料,须真混蛋(险说:差,抱歉!)才如此待活人!与前调无大异,仅“差”换“混蛋”、“被待”换“待”,戏法成!正义斗士的荣誉,自保全!

此显我与轻蔑态度的共谋,至少追溯至五十年代初,即卡尔唐及其友善迎我后的数年。若后未“见”,当轻蔑渐普遍,因我不想见——不比此孤例,须极力装瞎装聾!

此共谋与我新身份——受敬的“有功者”、数学强者团体一员——紧密共生。我忆尤满足,甚至骄傲,于我选并接纳我的世界,非社会地位或(绝非!)仅名声为重,须名副其实——纵为大学教授、院士,若仅平庸数学家(可怜家伙!),则无足轻重,唯功绩、深刻原创思想、技术 virtuosité、广阔愿景为重!

此“功绩”意识形态,我毫无保留认同(虽隐而不言),至1970年“觉醒”后受重创。我不确信其时已无痕消散。须我清晰自察,而我似多斥之于他人。谢瓦利与我经“生存”「Survivre;Survive」识之丹尼·盖德「Denis Guedj;Denis Guedj」首警我此意识形态(彼称“精英制”或类似),及其暴力与轻蔑。谢瓦利言(或初访他宅谈“生存”时),因此他不忍布尔巴基氛围,断足。我今思,他定察觉我深涉此意识形态,或尚存余迹。但我不忆他暗示。或他又留我自点他画的“i”,我至今日方点。迟总胜于无!