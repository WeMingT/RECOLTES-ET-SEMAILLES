\section{(13) force et épaisseur}

Il est bien possible que l'incident que j'ai rapporté marque aussi le moment d'un basculement intérieur en moi, vers une identification plus ou moins inconditionnelle avec la confrérie du mérite, aux dépens des gens considérés comme nuls, ou simplement "sans génie" comme on aurait dit quelques générations avant (ce terme n'était plus en vogue déjà de mon temps) : les gens ternes, médiocres - tout au mieux des "caisses de résonance" (comme Weil a écrit quelque part) pour les grandes idées de ceux qui comptent vraiment. . . Le seul fait que ma mémoire, qui si souvent agit en fossoyeur même pour des épisodes qui sur le moment mobilisent une énergie psychique considérable, ait retenu cet épisode-là, ne se rattachent à aucun autre souvenir directement lié, et se présentent sous une apparence tellement anodine, rend plausible ce sentiment d'un "basculement" qui aurait eu lieu alors.

Dans une méditation d'il y a moins de cinq ans, j'ai d'ailleurs fini par me rendre compte que cette idéologie du "nous, les grands et nobles esprits...", sous une forme particulièrement extrême et virulente, avait sévi en ma mère depuis son enfance, et domine sa relation aux autres, qu'elle se plaisait à regarder du haut de sa grandeur avec une commisération souvent dédaigneuse, voire méprisante. Je vouais d'ailleurs à mes parents une admiration sans réserve. Le premier et seul groupe auquel je me sois identifié, avant la fameuse "communauté mathématique", a été le groupe familial réduit à ma mère, mon père et moi, qui avais eu l'honneur d'être reconnu par ma mère comme digne de les avoir comme parents. C'est dire que les germes du mépris ont dû être semés dans ma personne dès mon enfance. Le moment serait peut-être mûr de suivre les vicissitudes, à travers mon enfance et ma vie d'adulte, de ces germes, et des récoltes d'illusion, d'isolement et de conflit en quoi certains d'eux ont levé. Mais ce n'est pas le lieu ici, où je suis un dessein plus limité. Je crois pouvoir dire que cette attitude de mépris n'a jamais pris dans ma vie une véhémence et une force destructrice comparables à celles que j'ai vues dans la vie de ma mère, (quand je me suis donné la peine de regarder la vie de mes parents, vingt-deux ans après la mort de ma mère, et trente-sept ans après celle de mon père). Mais c'est le moment maintenant ou jamais d'examiner avec attention, ici, au moins quelle a été la place de cette attitude dans ma vie de mathématicien.

Avant cela, pour situer dans son contexte général l'incident rapporté au paragraphe précèdent, je voudrais insister sur ce fait, qu'il est entièrement isolé parmi mes souvenirs des années cinquante, et même de plus tard. Même de nos jours, alors que je constate pourtant une érosion parfois déconcertante de certaines formes élémentaires de la courtoisie et du respect d'autrui dans le milieu qui fût le mien \footnote{(10)\par Par exemple, je ne compte plus le nombre de lettres, sur des questions aussi bien mathématiques que pratiques ou personnelles, envoyées à des collègues ou des ex-élèves que je considérais comme des amis, et qui n'ont jamais reçu de réponse. Il ne semble pas que ce soit seulement un traitement de faveur réservé à ma personne, mais bien un signe d'un changement de moeurs, d'après des échos dans le même sens. (Ceux-ci concernent, il est vrai, des cas où celui qui envoyait une lettre mathématique n'était pas connu du destinataire, mathématicien en vue...)} (10) l'expression directe et non déguisée du mépris de patron à élève doit être une chose assez rare. Pour ce qui est des années cinquante, j'ai très peu de souvenirs qui aillent dans le sens d'une crainte qui aurait entouré alors une figure de notoriété, ou d'attitude de mépris ou simplement dédaigneuse. Si je fouille dans ce sens, je peux dire que lors de la première fois où j'ai été reçu chez Dieudonné à Nancy, avec l'amabilité pleine de délicatesse qu'il a toujours eue avec moi, j'ai été un peu éberlué par la façon dont cet homme raffiné et affable parlait de ses étudiants - tous des abrutis autant dire! C'était une corvée de leur faire des cours, auxquels il était évident qu'ils ne comprenaient rien... Après 1970 j'ai entendu les échos venant du côté amphithéâtre, et j'ai su que Dieudonné était bel et bien craint des étudiants. Pourtant, alors qu'il était réputé pour avoir des opinions tranchées et pour les servir avec une franchise parfois tonitruante, je ne l'ai jamais vu se comporter d'une façon blessante ou humiliante, y compris en présence de collègues dont il avait piètre estime, ou aux moments de ses légendaires grosses colères, qui s'apaisaient aussi rapidement et aisément qu'elles avaient surgi.

Sans m'associer aux sentiments exprimés par Dieudonné au sujet de ses étudiants, je ne prenais pas non plus mes distances par rapport à son attitude, présentée comme la chose la plus évidente du monde, comme allant presque de soi de la part d'une personne qui avait une passion pour la mathématique. L'autorité pleine de bienveillance de mon aîné aidant, cette attitude-là m'apparaissait alors comme tout au moins une des attitudes possibles qu'on pouvait raisonnablement avoir vis-à-vis des étudiants et des tâches d'enseignement.

Il me semble que pour Dieudonné comme pour moi, imprégnés l'un et l'autre de cette même idéologie du mérite, l'effet isolant de celle-ci se trouvait dans une large mesure neutralisée lorsque nous nous trouvions devant une personne en chair et en os, dont la seule présence nous rappelait silencieusement des réalités plus essentielles que celles du soi-disant "mérite", et rétablissait un lien oublié. La même chose devait se passer pour la plupart de nos collègues ou amis, non moins imprégnés que Dieudonné ou moi du syndrome si répandu de supériorité. Sûrement tel est le cas encore aujourd'hui pour beaucoup d'entre eux.

Weil avait également la réputation d'être craint par ses étudiants, et il est le seul de mon microcosme, en les années cinquante, dont j'aie eu l'impression qu'il était craint même parmi les collègues, de statut (ou simplement de tempérament) plus modeste. Il lui arrivait d'avoir des attitudes de hauteur sans réplique, qui pouvaient déconcerter l'assurance la mieux accrochée. Ma susceptibilité aidant, cela a été l'occasion une ou deux fois de brouilles passagères. Je n'ai pas perçu en ses façons une nuance de mépris ou une intention délibérée de blesser, d'écraser; plutôt des attitudes d'enfant gâté, prenant un plaisir (parfois malicieux) à mettre mal à l'aise, comme une façon de se convaincre d'un certain pouvoir qu'il exerçait. Il avait d'ailleurs un ascendant véritablement étonnant sur le groupe Bourbaki, qu'il me donnait parfois l'impression de mener à la baguette, un peu comme une maîtresse d'école maternelle une troupe d'enfants sages.

Je ne me rappelle qu'une seule autre occasion en les années cinquante, où j'aie senti une expression brutale, non déguisée de mépris. Elle provenait d'un collègue et ami étranger, à peu près de mon âge. Il avait une puissance mathématique peu commune. Quelques années avant, où cette puissance était pourtant déjà bien manifeste, j'avais été frappé par sa soumission (qui me paraissait quasiment obséquieuse) au grand professeur dont il était encore le modeste assistant. Ses moyens exceptionnels lui valurent rapidement une réputation internationale, et un poste-clef dans une université particulièrement prestigieuse. Il y régnait alors sur une petite armée d'assistants-élèves, de façon apparemment toute aussi absolue que son patron avait régné sur lui et ses camarades. A ma question (si je me rappelle bien) s'il avait quelques élèves (sous-entendu : qui faisaient du bon travail avec lui), il a répondu, avec un air de fausse désinvolture (je traduis en français) : "douze pièces !" - où "pièces" était donc le nom par lequel il référait à ses élèves et assistants. Il est certes rare qu'un mathématicien ait un tel nombre d'élèves à la fois faisant de la recherche sous sa direction - et sûrement mon interlocuteur en tirait un secret orgueil, qu'il essayait de cacher sous cet air négligent, comme pour dire : "oh, juste douze pièces, pas la peine même d'en parler !". Ça devait être vers 1959, j'avais déjà une bonne carapace alors sûrement, j'ai pourtant eu un haut le coeur ! J'ai dû le lui dire sur le champ d'une façon ou d'une autre, et je ne crois pas qu'il m'en ait voulu. Peut-être même sa relation à ses élèves n'était-elle pas aussi sinistre que son expression pouvait le laisser supposer (je n'ai pas eu le témoignage d'un de ses élèves), et qu'il s'était trouvé simplement pris au piège de son puéril-désir de se pavaner devant moi dans toute sa gloire. Rétrospectivement, je vois que cet incident a dû marquer un tournant dans nos relations, qui avaient été des relations d'amitié - je sentais en lui une sorte de fragilité, une finesse aussi, qui attiraient ma sympathie affectueuse. Ces qualités s'étaient émoussées, corrodées par sa position d'homme important, admiré et craint. Après cet incident, un malaise est resté en moi vis à vis de lui - décidément je ne me sentais pas faire partie du même monde que lui...

Pourtant on faisait bien partie du même monde - et sans m'en rendre plus compte que lui, sûrement je m'épaississais, moi aussi. A ce sujet il m'est resté un souvenir vivace, se situant au Congrès International d' Edimburgh, en 1958, Depuis l'année précédente, avec mon travail sur le théorème de Riemann-Roch, j'étais promu grande vedette, et (sans que j'aie eu à me le dire en termes clairs alors) j'étais aussi une des vedettes du Congrès.(J'y ai fait un exposé sur le vigoureux démarrage de la théorie des schémas en cette même année.) Hirzebruch (une autre vedette du jour, avec son théorème de Riemann-Roch à lui) faisait un discours d'ouverture, en l'honneur de Hodge qui allait partir à la retraite cette année. A un moment, Hirzebruch a laissé entendre que les mathématiques se faisaient par le travail des jeunes surtout, plus que par celui des mathématiciens d'âge mûr. Cela avait déclenché dans la salle du Congrès, où les jeunes formaient une majorité, un tollé général d'approbation. J'étais enchanté et très d'accord bien sûr, j'avais trente ans pile ça pouvait encore passer pour jeune et le monde m'appartenait! Dans mon enthousiasme, j'ai dû crier à haute voix et taper des grands coups sur la table. Il se trouvait que j'étais assis à côté de Lady Hodge, l'épouse du mathématicien éminent qu'on était censé honorer en cette occasion, alors qu'il allait prendre sa retraite. Elle s'est tournée vers moi avec de grands yeux et m'a dit quelques mots, dont je n'ai plus souvenir - mais j'ai dû voir reflété par ses yeux étonnés l'épaisseur dénuée de tact qui venait de se déchaîner sans retenue devant cette dame sur la fin de sa vie. J'ai senti alors quelque chose, dont le mot "honte" donne une image peut-être déformée - une humble vérité plutôt concernant celui que j'étais alors. Je n'ai plus dû donner des grands coups sur les tables ce jour-là...




