\section{(13) 力量与厚重}

我所述事件或标志我内在转折,向“功绩兄弟会”近乎无条件认同,牺牲被视为“无能”或如前几代所言“无天才”(我时此词已不时髦)者:平庸、暗淡之人——至多为真正重要者的伟大思想之“共鸣箱”(如韦伊「Weil;Weil」某处所述)。仅凭记忆——常埋葬耗费心力的片段——保留此看似琐碎、无直接关联的事件,便令人信当时确有“转折”。

五年前一冥想中,我终觉此“吾等伟大高尚精神……”意识形态,于我母自幼极端剧烈,主宰其对他人的关系,常以高傲怜悯——多带轻蔑甚至鄙视——俯视。我对父母钦佩无保留。加入“数学共同体”前,我首认同的团体乃狭小家庭——母、父与我,我荣幸被母视为配为其子。这表明轻蔑种子或自幼植于我身。或时机已熟,追索此种子于我童年及成年的起伏,及其幻觉、孤立与冲突之果。然此处非地,我意更有限。我信此轻蔑态度于我数学人生未具我母生命中(我费心审视父母人生——母殁廿二年、父殁卅七年后)所见的激烈破坏力。今乃审视其于我数学人生地位之时,机不可失。

先置前段所述事件于更广背景,我强调其于五十年代回忆中孤立,甚至更晚亦然。即今,我见昔日圈子中礼貌与尊重他人的基本形式有时令人困惑地侵蚀 \footnote{(10)\par 例如,我寄给视为朋友的同事或旧生之信——涉数学、实用或私人——无数未获回。似非独对我,依同类反馈,乃风气变。(诚然,此类涉寄数学信予知名数学家,而发信人不为其所识……)} (10),然导师对学生直接赤裸轻蔑当罕见。五十年代,我少有回忆涉当时围绕名人的恐惧,或轻蔑、贬低态度。若深挖,我可说首访南锡「Nancy;Nancy」迪厄多内「Dieudonné;Dieudonné」宅,他以一贯温雅相待,我却为其谈学生方式愕然——皆蠢货无异!教课乃苦役,显然他们一无所解……1970年后,我闻讲堂侧反馈,知学生确畏迪厄多内。然虽他以观点鲜明、直言震耳著称,我未见其伤人或羞辱,即便对低估的同事或其传奇怒火——来得快去亦速——时亦然。

我未附和迪厄多内对学生的看法,亦未与其态度——似理所当然,近乎数学热忱者的自然流露——拉开距离。仗前辈温和权威,此态度对我似为合理对待学生与教学的可能姿态之一。

我觉对迪厄多内与我——皆浸此“功绩”意识形态——其孤立效应在面对活人时多被中和,其存在无声提醒比所谓“功绩”更本质的现实,重联遗忘纽带。多同事友亦然,不比我俩少受此普遍优越症影响。今或亦多如此。

韦伊亦以令学生畏知名,五十年代微观世界中,我唯觉他连同事——地位或性格温和者——亦畏。他偶现无可辩驳的高傲,可动摇最坚定的自信。我敏感使然,一两次致短暂不和。我未感其方式含轻蔑或故意伤人之意,更似被宠坏的孩子,乐于(有时恶作剧般)令人不安,以确其某种权力。他对布尔巴基确有惊人影响,时似挥棒操群,如幼儿园女师领乖童。

五十年代我仅忆另一例,感赤裸残酷轻蔑。来自一外国同事兼友,近我龄,数学力量罕见。数年前,其力已显,我为其对大教授——他尚为其谦卑助手——近乎谄媚的顺服震惊。其卓越才干速获国际声誉及顶尖大学要职。他遂统小群助手-学生,似如其导师绝对统治他及其同侪。我问(若忆准)他有无学生(暗指:共创佳作),他故作漫不经心答(我译法语):“十二件!”——“件”乃其称学生与助手之词。罕有数学家同时指导如此多研究学生——他或暗自骄傲,掩于此漫态,似言:“哦,仅十二件,不值一提!”约1959年,我已有厚壳,仍觉恶心!必当场有所表示,他似未怨。或他与学生的关系不似其言阴森(我未闻其学生证词),仅陷于幼稚炫耀欲,欲在我前展其荣光。回顾,此事或为我俩友谊转折——我感其脆弱与细腻,引我温情。此质渐钝,蚀于其显赫、受敬畏地位。此后,我对他存不适——我确感与他非同世界……

然我俩确同世界——我未如他自察,定亦渐厚。此处我忆鲜明,1958年爱丁堡「Edimburgh;Edinburgh」国际大会。前年凭黎曼-罗赫「Riemann-Roch;Riemann-Roch」定理工作,我晋大明星,(未明言)亦为大会明星。(我述当年方案「schémas;schemes」理论蓬勃起步。)赫泽布鲁赫「Hirzebruch;Hirzebruch」(当日另一星,其黎曼-罗赫定理)开幕致敬霍奇「Hodge;Hodge」——当年退休。他暗示数学主要由年轻人推动,超成熟数学家。会场——青年占多——哗然赞同。我甚喜甚同,三十正龄尚算年轻,世界我有!我兴奋或高喊、大敲桌。我坐霍奇夫人旁——应致敬的显赫数学家之妻,因其退休。她转头大眼看我,说了几词(不复忆)——我必从其惊目中见我无拘厚重,于此暮年女士前肆放。我感某物,“羞耻”或失真——乃关乎当时我之谦卑真相。那日我未再大敲桌……