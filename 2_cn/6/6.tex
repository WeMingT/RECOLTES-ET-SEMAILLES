\section{(10) “数学共同体”:虚构与现实}

若从那年起——或许在我内心深处,且在随后岁月愈发清晰——我自觉是这世界一员,称其为“数学共同体”「Communauté mathématique;Mathematical Community」,对我满载意义,自不奇怪。在写下这些前,我从未审视这名称于我何意,虽我已在很大程度上与之认同。如今明了,它无非是我初遇那友善世界的理想延伸,穿越时空;那世界接纳我为“自己人”,且由主宰我人生的一大激情相连。

这“共同体”——我渐与之认同——并非全然虚构,乃初迎我的数学圈子的某种延展。初始圈子缓缓扩大,我意指:因共同兴趣与个人亲近,我常接触的数学家圈子,在初识后十廿年间渐广。具体而言,是同事与友人的圈子,或更像同心结构:从最亲近者(初为迪厄多内「Dieudonné;Dieudonné」、施瓦茨「Schwartz;Schwartz」、戈德芒「Godement;Godement」,后尤塞尔「Serre;Serre」,再后如安德烈奥蒂「Andreotti;Andreotti」、朗「Lang;Lang」、泰特「Tate;Tate」、扎里斯基「Zariski;Zariski」、广中「Hironaka;Hironaka」、芒福德「Mumford;Mumford」、博特「Bott;Bott」、麦克·阿廷「Mike Artin;Mike Artin」,及逐渐扩大的布尔巴基「Bourbaki;Bourbaki」团体,乃至六十年代起趋我的学生……),至偶遇此处彼处的其他同事,因亲疏不一的共鸣相连不等——此微观世界,由际遇与共鸣偶成,构成了这对我温馨共鸣之名的具体内涵:数学共同体。我视其为活体、温暖,与之认同,实则认同此微观世界。

直至1970年“重大转折”——应称初醒——我方知这温馨友好的微观世界仅占“数学世界”一隅,我乐于赋予这未知、不曾探究之世界的特质,皆虚构。

廿二年间,此微观世界面貌已变,周遭世界亦然。我无疑亦变,随岁月潜移默化,如周遭一般。我友与同事是否比我更觉此变——于周遭、其微观世界及自身——我不知。我亦难言此奇变何时何由——或悄然而至,蹑足潜行:名声之人令人畏惧。我亦被畏——非学生、友人或熟识我者,而是仅知我名声、自身无同等名声护体者。

近十五年前“觉醒”后,我方察觉数学世界(及其他科学圈子尤甚)的恐惧。前十五年,我渐入“大师”角色,跻身数学界“名人录”,浑然不觉。我亦受此角色囚困,隔绝于众,仅少数“同侪”及少数执着学生(即便如此……)除外。脱离此角色后,围绕其部分恐惧消散。多年缄口的舌头松动。

它们见证的不止恐惧,亦有轻蔑。尤在位者对他人的轻蔑,滋生并助长恐惧。

我鲜历恐惧,却深谙轻蔑——彼时人命与人生轻如鸿毛。我乐于忘却轻蔑时代,然它重叩记忆!或它从未止息,我仅自以为换了世界,移目他处,或佯装不视不闻,除却迷人无尽的数学讨论?彼日,我终愿知晓,轻蔑遍布我选为己有、与之认同、予我认可并宠爱之世界。