\section{(9) 受欢迎的异乡人}

此刻似是时候谈谈我与数学家世界的关系。这与我对数学的关系截然不同。后者自幼便存在且强烈,远在我知晓数学家世界与圈子之前。那是个复杂世界,有学术团体、期刊、会议、研讨会、大会,有明星与苦工,有权力结构、幕后大佬,还有更多默默无闻、为论文或文章挣扎的可塑之才,以及少数富于资源与创意却屡遭闭门羹者——他们绝望地寻求那些忙碌而令人畏惧的权势人物支持,以期获那魔法般的力量:发表一篇文章……

1948年,我二十岁抵巴黎,始知数学世界存在。行李简薄,仅一蒙彼利埃大学「Université de Montpellier;University of Montpellier」科学学士学位,与一手稿——字迹密布,正反书写,无边距(纸张昂贵!),载三载孤独反思,抵京后方知乃众所周知的“测度论”「théorie de la mesure;measure theory」或“勒贝格积分”「intégrale de Lebesgue;Lebesgue integral」。此前未遇他人,我真以为自己是世上唯一“做数学”者,唯一数学家。(对我而言二者等同,至今多少如此。)我玩弄自称可测的集合(未遇不可测者……),研究几乎处处收敛,却不知拓扑空间为何物。我在一小册子(似为阿佩尔「Appert;Appert」所著,刊于《科学与工业动态》「Actualités Scientifiques et Industrielles;Scientific and Industrial News」)中捞到十余“抽象空间”与紧性定义,颇为迷失——天晓得如何得之。数学语境中,我未闻怪词或蛮语如群、域、环、模、复形、同调(等等!),它们却骤然齐袭,猝不及防。冲击猛烈!

我若“幸存”此冲击,继续数学并以此为业,因彼时数学世界远未似今日模样。或我幸降于一较友善角落。我携蒙彼利埃学院一教授苏拉「Soula;Soula」的模糊推荐信(他与同事罕见我上课!),他是卡尔唐「Cartan;Cartan」(父或子,不甚明了)学生。埃利·卡尔唐「Elie Cartan;Elie Cartan」已“退场”,其子亨利·卡尔唐「Henri Cartan;Henri Cartan」成我首个“同类”相遇者。我未料此乃吉兆!他以特有的温和礼遇待我——为多代巴黎高师「normaliens;École Normale students」熟知,他们有幸随他初试锋芒。他或未全知我无知之深,从他指导我学业的建议看。无论如何,其善意显系对人,而非学识、天赋,或(后来的)名声……

次年,我旁听卡尔唐在“学校”讲授的微分形式与流形课程,死抓不放;亦列席“卡尔唐研讨会”「Séminaire Cartan;Cartan Seminar」,惊叹他与塞尔「Serre;Serre」论战,挥洒“谱序列”「Suites Spectrales;Spectral Sequences」(毛骨悚然!)与满布箭头的图表(称“图解”)。那是“层”「faisceaux;sheaves」、“壳”「carapaces;carapaces」及一整套武器的英雄时代,我全然不明其意,却勉力吞咽定义、命题,验证证明。研讨会常现谢瓦利「Chevalley;Chevalley」、韦伊「Weil;Weil」,至布尔巴基研讨会「Séminaires Bourbaki;Bourbaki Seminars」日(参与与旁听者不过二三十),其他布尔巴基「Bourbaki;Bourbaki」团伙——迪厄多内「Dieudonné;Dieudonné」、施瓦茨「Schwartz;Schwartz」、戈德芒「Godement;Godement」、德尔萨特「Delsarte;Delsarte」——如喧闹友群登场。他们互称“你”,操我几乎全不懂的语言,烟雾缭绕,常笑——若有啤酒箱,便齐全,实则以粉笔与海绵代之。与勒雷「Leray;Leray」在法兰西学院「Collège de France;College of France」的课程(论无限维空间中肖德尔「Schauder;Schauder」拓扑度理论,可怜我!)氛围迥异——我依卡尔唐建议旁听,曾访勒雷,问其课程内容(若我忆准)。我不记得他说了什么,也不记得我懂了多少——只觉同样善意迎我这初来乍到者。仅此而已,定使我坚持听课,勇敢紧随卡尔唐研讨会与勒雷课程,虽后者内容我几全不懂。

奇妙的是,在这新入、语言不通的世界,我不觉陌生。虽少有机会(有因!)与韦伊、迪厄多内这类乐天派,或卡尔唐、勒雷、谢瓦利这类风度翩翩者交谈,我却感被接纳,几如“自己人”。我未忆起被这些前辈傲慢相待,或我求知之渴、后来的发现之喜遭自大或轻蔑拒斥 \footnote{(5)\par 此事实尤显,因至1957年前,多位布尔巴基成员对我存保留——他们最终似不情愿接纳我。一善意玩笑称我为“危险专家”(函数分析)。我有时感卡尔唐有未言的严肃保留——数年,我或予他热衷空泛肤浅概括之感。他惊讶于我为布尔巴基首次(且唯一)长篇撰稿(论流形微分形式)具实质反思——我提议时他并不热心。(此反思多年后助我从相干对偶视角发展残数形式。)我常在布尔巴基大会迷失,尤其共读稿件时,难跟上节奏与讨论。我或不适集体工作。然我融入共事的困难,或因他因在卡尔唐等人处引发的保留,未招嘲讽或冷落,除韦伊一两次略显轻蔑(真特例!)。卡尔唐从未失其温和与亲切,带他特有的幽默——于我与其人不可分。} (5)。若非如此,我不会“成为数学家”——我会选他业,施展才华而不面对轻蔑……

“客观上”我对此界、彼国皆异乡人,然与这些不同背景、文化、命运者相连的,是共同激情。我疑那关键年——我初识数学家世界时——无人,甚至卡尔唐(我略为其徒,他却有众多不似我迷茫的学生),察觉我与他们同怀此情。对他们,我或仅是听课与研讨会中记笔记、明显“不谙此道”的一员。若我稍异于众,或因我不惧提问——多暴露我对语言与数学的无知。答或简短、或惊讶,我这茫然怪人从未遭冷落或“归位”,无论在布尔巴基的随意圈子,或勒雷课程的严肃氛围。抵巴黎那年,携致埃利·卡尔唐的信,我从未觉面对封闭或敌意之界。若我深知轻蔑引发的内心紧缩,非在此界,至少彼时非然。尊重人格似我呼吸的空气。无需挣得尊重或证明自我,即被接纳且友善相待。奇哉,仅为人、有张人脸,便足矣。