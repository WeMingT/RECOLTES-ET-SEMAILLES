\section{(8) 梦与证明}

让我们回到梦,以及数学中数千年对其的禁忌。这或许是所有先入为主中最根深蒂固者,常隐而不宣,深植于习惯,裁定何为“数学”,何非也。须历千年,诸如几何图形的对称群、拓扑形态、零、集合等简单且无处不在之物,方获准进入圣殿!我向学生谈及球的拓扑,或由球加柄衍生的形态——幼儿不觉惊奇,学生却困惑,因他们自以为知“数学为何”——初闻即脱口而出:这不是数学!数学当然是毕达哥拉斯定理、三角形高、二阶多项式……这些学生不比你我愚钝,他们的反应与古今全球数学家无异,除毕达哥拉斯「Pythagore;Pythagoras」、黎曼「Riemann;Riemann」及五六人外。即便是颇有建树的庞加莱「Poincaré;Poincaré」,也以哲学化的“A加B”论证无限集非数学!想必曾有一时,三角与方形亦非数学——不过是孩子或陶匠在沙上、陶上勾画,不容混淆……

此种被“知识”窒息的心智惰性,非数学家独有。我稍稍偏离主题:那针对数学之梦的禁忌,及由此对一切未以惯常成品形式、即食状态呈现之物的排斥。我对其他自然科学的浅知足以让我明白,若施以同样严苛禁忌,它们将陷于不育,或如中世纪龟速前行——当时无人敢质疑《圣经》字面。但我也深知,发现的深层源泉及其路径的一切本质面向,在数学与其他宇宙可知领域——身与心可触及者——并无二致。禁梦即禁源,将其逼入隐秘存在。

我亦凭自幼与数学初恋至今未曾失真的经验深知:在数学中展开广阔或深刻愿景时,此愿景与理解的展开、逐步渗透,恒先于证明,使之可能并赋予意义。当一情境——无论最朴素或最宏大——于本质上被理解,其证明(及其余)如熟果自然坠地。而如青果强摘自知识之树的证明,留下的则是未尽的不满,未能平息我们的渴求。我数学生涯中两三次迫于无奈强摘果实,非说做错或后悔。但我最擅、最爱者,皆自愿摘取,而非强取。若数学予我无尽欢乐,于成熟之年仍令我着迷,非因我强索的证明,而是其无穷奥秘与完美和谐,总待爱的手与目揭示。