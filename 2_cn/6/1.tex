1984年2月

\section{(5) 被禁之梦}

趁着《追寻场》「La Poursuite des Champs;In Pursuit of Fields」写作中断三月之机,我重拾去年六月搁置的引言。时隔半年多,我仔细重读,并添加了几个小标题。

撰写此引言时,我深知这类反思难免引发诸多“误解”——试图预先化解不过是徒劳,只会叠加更多误解于原有之上!对此我仅补充一点:我绝无意对千年惯例确立的科学写作风格宣战。那是我人生二十余年勤勉实践并传授给学生的风格,是数学家职业的重要部分。无论对错,至今我仍如此认为,并继续传授。甚至我可能显得老派,坚持工作做到极致,从头至尾手工缝制,不放过任何幽暗角落。若近十年我有所妥协,也纯属形势所迫!“正式撰写”对我而言仍是数学工作的关键步骤,既是发现的工具——检验并深化未经此过程则粗略零散的理解,也是传递此理解的途径。从教学角度,严谨的演绎式 exposition,不排除描绘广阔画卷的可能性,显然具有简洁与引用便利的优势。这些是实实在在且重要的优势,尤其针对数学家,特别是那些已熟悉论述主题或相关领域的数学家。

然而,若 exposition 面向儿童、年轻人或完全“不谙此道”的成人,这些优势荡然无存。他们的兴趣尚未被唤醒,且多半对发现工作的真实路径一无所知(也因故将持续如此……)。更恰当地说,这些读者甚至不知此类工作的存在——它人人可及,只需好奇心与常识。这工作不断孕育并更新我们对宇宙事物的智识,包括如欧几里得「Euclide;Euclid」的《几何原本》「Eléments;Elements」或达尔文「Darwin;Darwin」的《物种起源》「L'Origine des Espèces;The Origin of Species」般宏大的体系。对此工作存在及其本质的彻底无知几乎无处不在,从小学教师到大学教授皆然。这是个惊人之事实,去年借引言第一部分反思之机,我初见其全貌,同时瞥见其令人困惑的深层根源……

即便面向完全“在行”的读者,严谨 exposition 仍有一重要之事被禁止传递。这在严肃人群中——如我们科学家——亦颇不受待见。我指的是梦。梦,以及它轻声诉说的愿景——起初如梦般无形,常不愿成形。或许穷尽多年甚至一生紧张劳作,也未必能让某梦中愿景完全显现,凝结并打磨至钻石的坚硬与光芒。这是我们手或心的匠人之工。当工作或其部分完成,我们将有形成果置于最明亮的光线下展示,欣喜并常引以为傲。然而,启发我们雕琢的,非这长久打磨的钻石。或许我们锻造了一件精密高效的工具——但工具如人手所造之物,皆有限,即便看似伟大。最初无名无形的愿景,薄如雾中碎片,引导我们手持工具,俯身劳作,不觉时光流逝——或数小时,或数年。这碎片悄然脱离无底的雾海与幽暗……我们内心的无垠是她,这孕育不息之海,当我们的渴望使其受孕,便从中涌出梦,如胚胎依偎于滋养的母体,等待那将它引向光明的第二次诞生的隐秘劳作。

若一个世界轻视梦,则此世亦轻视我们内心深处——祸哉斯世。我不知在我们这电视、计算机与洲际火箭的文化之前,是否其他文化宣扬此轻蔑。这或为我们区别于前人——那些被我们彻底取代、几乎从地球表面抹除的前人——的诸多特质之一。我未闻其他文化不敬梦、不感其深根、不为其所共认。有哪个人或民族的宏伟事业,非自梦而生、非由梦滋养,直至绽放于光天化日?然而在我们这里(是否已该说:无处不在?),敬梦被斥为“迷信”。众所周知,我们的心理学家与精神病学家已将梦量尽——不过塞满一台小型计算机的内存罢了,诚然。同样真实的是,“我们”无人再知如何生火,无人敢在家中迎接新生儿或送别父母——有诊所与医院代劳。谢天谢地……我们的世界,骄傲于其原子弹兆吨威力和图书馆与计算机中的信息量,或许也是每个人面对生命简单本质之物时的无能、恐惧与轻蔑达巅峰的世界。

幸而梦,如最压抑社会中的原始性冲动,生命力顽强!无论迷信与否,它仍暗中执着地低语知识——那清醒之心太沉重或胆怯而无法领会之知,为它启发的计划赋予生命与翅膀。

我曾暗示梦常不愿成形,那只是表象,未触本质。“不愿”更应归于清醒之心在其常态下的“姿态”——“不愿”一词还是轻描淡写!更确切是深深的戒备,掩盖着古老的恐惧——认知之惧。以字面意义的梦而言,此恐惧愈强,其屏障愈密,因梦的讯息愈切近我们,愈携改变我们自身的深重威胁,若被听见。但须信,即便在相对无害的“数学之梦”中,此戒备亦存且有效;以至于一切梦似被逐出文本(我未见任何文本存其痕迹),乃至同事间私下讨论,无论小圈子或一对一。

若真如此,非因数学之梦不存在或已消亡——否则我们的科学将枯竭,事实显然非然。此表象缺席、此沉默共谋之因,定与另一共识密切相关——那小心抹去发现工作及其更新世界认知痕迹的共识。或更准确地说,围绕梦及其激发、启发、滋养之工作的,是同一沉默。以至于“数学之梦”一词对许多人似无稽,因我们常被一按即得的陈词滥调驱使,而非源于对简单、日常、重要现实的直接经验。