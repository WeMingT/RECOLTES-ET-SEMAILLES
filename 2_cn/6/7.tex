\section{(11) 与克洛德·谢瓦利相遇,或:自由与善意}

前文或予人印象,我被几乎一夜间涌来的见证所震撼。实则不然。这些见证仅留于表层,单纯叠加于我新知或早已知晓却避而不视之事上。今我将当时教训表述为:“科学家”,无论显赫或默默,皆与常人无异!我曾自得以为“我们”更优、多些什么——顽固幻觉,耗一二年方褪!

助我者中,唯独一人来自我决然离去之圈子 \footnote{(6) 我的“生存与生活”朋友 \par
其中或应计入皮埃尔·萨缪尔「Pierre Samuel;Pierre Samuel」,我早先主要于布尔巴基「Bourbaki;Bourbaki」识之,如谢瓦利「Chevalley;Chevalley」,他在“生存与生活”「Survivre et Vivre;Survive and Live」团体中亦要角。我不觉萨缪尔甚执科学家优越幻觉。他贡献尤多,我感,因其常识与笑意盎然的乐观,融入共事、讨论、待人,亦优雅担“可怕改良派”角色于一倾向激进分析与选择的团体。我退出后,他留“生存与生活” некоторое время,掌同名刊物,觉不再有益时,欣然离去(加入“地球之友”「Amis de la Terre;Friends of the Earth」)。

萨缪尔与我同属狭圈,却不碍他成为我那沸腾岁月所学(虽我劣徒……)之友。他与谢瓦利行事虽异,皆为我“精英倾向”的更好解药,胜于犀利分析!

今觉此期所有教我之友,皆因其行事与异于我的敏感“某物”渐传予我,而非解释、讨论等……尤忆,除谢瓦利、萨缪尔外,丹尼·盖德「Denis Guedj;Denis Guedj」(对“生存与生活”影响甚大)、丹尼尔·西博尼「Daniel Sibony;Daniel Sibony」(置身事外,半轻半嘲旁观其演变)、戈登·爱德华兹「Gordon Edwards;Gordon Edwards」(1970年6月蒙特利尔“运动”共创者,多年竭力维持英文版“生存与生活”刊)、让·德洛尔「Jean Delord;Jean Delord」(近我龄物理学家,温润热情,喜我及生存圈)、弗雷德·斯内尔「Fred Snell;Fred Snell」(美国布法罗物理学家,1972年我数月寄居其乡宅)。

此众友,五为数学家,二为物理学家,皆科学家——似示那年我最近圈子仍科学家,尤数学家。} (6)。即克洛德·谢瓦利「Claude Chevalley;Claude Chevalley」。他不作演说,亦无意听我高谈,我却信从他学到比前述更重要、更隐秘之事。“生存”时期我与他常聚(他半信加入“生存”「Survivre;Survive」团体,后成“生存与生活”),常令我困惑。我难言其如何,但感他握有我未及之知,对简单本质事物的理解,或可用简词表达,却无法轻易传递。我今觉他我间有成熟度差异,常使我与他似聾者對話,非因互缺好感或尊重。他未明言(我忆),但对他或显而易见,我当时——或独或随“生存”共思与行动——对“科学家社会角色”、“科学”等“质疑”,本质仍浅表。它们关乎我身处世界及其中角色,却未深涉我自身。那沸腾岁月,我对己之见未变分毫。自我认知非彼时始。六年后,我首度摆脱一顽固幻觉,非关他人或周遭,而关乎我。此乃另一觉醒,较前次——为其铺路者——影响更深,乃一系列觉醒之首,我盼余年延续。

我不忆谢瓦利曾提及自我认知,或更妥曰“自我发现”。回顾,他显然早已识己。有时他自述,仅因某事几词,简单得令人不安。他是我未闻陈词滥调的二三人之一。他少言,所语非采纳之理念,而是个人感知与理解。故他常令我不安,早在布尔巴基共聚时。他言常颠覆我珍视、因而视为“真”之见。他有我缺的内在自主,我在“生存与生活”时始隐约察觉。此自主非智识、言辞之属,非可“采纳”如观点等。我幸未萌“据为己有”他自主之念。我须觅己之自主,亦即:学(或重学)做自己。然那年,我未觉自身欠成熟、缺内在自主。若终察觉,与谢瓦利相遇定为暗中在我发酵之因,虽我忙于大计。非演说或词语播此酵种。仅需路上一人无需言辞,仅做自己,便足。

我觉七十年代初,借“生存与生活”刊物共聚时,谢瓦利试——未强加——传我一讯息,我当时太笨拙或陷激进任务未解。我隐约觉他教我自由——内在自由。我趋以大道德原则行事,自“生存”首刊起吹此号角,视之为理所当然,他却尤厌说教。我想这是“生存”初他最令我困惑之处。对他,此论仅是强加于人,叠于无数外在压迫之上。可终生辩其正反。此见全然颠覆我那(自不待言)高尚慷慨情怀驱动之观。我伤感,难解为何我极敬、如战友的谢瓦利,竟乐于不共享此情!我未明真理、现实,非关善意、观点或偏好。谢瓦利见一简单真事,我不见。非他读来;见一事与读其述毫无共性。可用手(盲文)或耳(他人朗读)读文,唯己眼可见事物本身。我不信谢瓦利眼胜我。他用眼,我否。我忙于善意等,无暇观其于我及他人——始于我儿——之效。

他定见我常不用眼,甚至无意如此。奇哉,他从未暗示。或曾示,我未闻?或他觉无用而止?或他未萌此念——用眼与否,毕竟我事,非他事!