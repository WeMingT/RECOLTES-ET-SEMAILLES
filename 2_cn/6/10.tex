\section{(14) 恐惧的诞生}

大约是在这个时候,我猜想,当我(并未刻意追求)开始被数学界视为一位明星时,一种特定的恐惧也一定开始围绕着我这个人,对许多不认识或不太知名的同事而言。我这样推测,却无法通过一个具体的记忆、一个让我印象深刻并固定在脑海中的画面来定位它,就像之前提到的那个事件(无疑标志着我在所选择的环境中首次遭遇轻蔑)。这件事一定是在不知不觉中发生的,没有引起我的注意,也没有通过某个特别的、典型的、会被记忆保留的事件表现出来,带着一种或许同样刻意显得平淡无奇的光芒,就像那个其他事件一样。我对那些过渡岁月的整体记忆告诉我,当时并不少见的是,那些与我交谈的人——无论是在我的研讨会之后,还是在诸如布尔巴基研讨会「Séminaire Bourbaki;Bourbaki Seminar」或某些学术会议、代表大会等场合——需要克服一种怯场,这种怯场在我们的交谈中或多或少显而易见,如果有交谈的话。当这种交谈持续超过几分钟时,这种尴尬通常会在我们交谈的过程中逐渐消散,谈话也随之变得活跃起来。但偶尔,也很少见地,这种尴尬可能会持续存在,甚至成为数学讨论这一非个人层面上沟通的真正障碍,我当时隐约感到对面有一种无能为力的痛苦,这种痛苦对自己感到恼怒。我谈论这一切时,并非真正“记得”,仿佛透过一层迷雾,然而这迷雾却还原了一些一定被记录下来的印象,或许也是逐渐被我排遣的印象。我完全无法在时间上定位这种尴尬——这种恐惧的表现——只能通过推测来判断。

我不认为这种恐惧源自我个人,也不认为它局限于某种态度或行为,使我区别于我的同事。如果真是如此,我觉得到了七十年代初,当我脱离了此前一直扮演的角色——正是那个明星、“大老板”的角色时,我应该会听到一些回响。我相信,是这个角色,而不是我个人,被恐惧所包围。而这个角色,带着这种与尊敬毫无共通之处的恐惧光环,在五十年代初并不存在,至少在我1948年进入数学界并被接纳的那个数学环境中,还没有出现。

在1970年的“觉醒”之前,我甚至不会想到将这种怯场、这种尴尬称为“恐惧”,这些是我有时在不太熟悉的同事身上遇到的。我自己也会因其显现而感到不安,并尽力去消解它。一个值得注意的现象,典型地反映了我亲爱的微观世界对这类事情的漠不关心:在整整二十年里,我身处这个环境,却不记得有哪怕一次,与一位同事之间,或在他人面前,讨论过这个问题!\footnote{(11) 阿尔多·安德烈奥蒂「Aldo Andreotti;Aldo Andreotti」,约内尔·布库尔「Ionel Bucur;Ionel Bucur」 \par 当然,我并非不可能有所遗忘——更不用说,我当时那种格外“极端”的性情,恐怕既不鼓励别人与我谈论这类事情,也不让我记住可能发生过的这类对话。可以肯定的是,讨论恐惧的问题(甚至不以这个名字称呼它……)必定是极为罕见的,至少当时如此,今天在“上流社会”中想必也是如此。

在那个世界的众多朋友中,除了谢瓦莱「Chevalley;Chevalley」,他一定至少在六十年代意识到了这种恐惧氛围外,唯一一个我认为也一定清楚察觉到它的,是阿尔多·安德烈奥蒂「Aldo Andreotti;Aldo Andreotti」。我是在1955年认识他的,还有他的妻子芭芭拉和他们当时还很小的双胞胎孩子(我想是在芝加哥韦伊「Weil;Weil」家的一次晚会上)。直到1970年的“大转折”,当我离开我们共同的环境并与他们有些疏远之前,我们一直保持着密切联系。阿尔多有着极其敏锐的感受力,这种感受力并未因与数学或像我这样的“极端分子”的交往而有所减弱。他身上有一种对接近之人自然流露的同情天赋,这使他在我认识的所有数学界朋友,甚至其他朋友中,显得与众不同。在他那里,友谊总是优先于共同的数学兴趣(这些兴趣并不少),他是少数几个我与之稍微谈及我的生活、而他也谈及他生活的数学家之一。他的父亲和我的一样,是犹太人,他曾在墨索里尼时期的意大利因此受苦,就像我在希特勒时期的德国一样。他总是乐于鼓励和支持年轻研究者,在一个越来越难以被学术权威接受的环境中,他始终保持着这种态度。他那发自内心的兴趣,首先总是关注人本身,而非数学“潜力”或名声。他是我有幸遇到的最令人珍视的人之一。

提起阿尔多,让我想起了约内尔·布库尔「Ionel Bucur;Ionel Bucur」,他也意外地英年早逝,与阿尔多一样,我想他更被人怀念的是作为一个让人乐于重逢的朋友,而非数学讨论的伙伴。他身上有一种善良,伴着一种罕见的谦逊,总倾向于不断退让。一个如此不把自己当回事、不试图给任何人留下深刻印象的人,如何最终成为布加勒斯特科学学院的院长,真是个谜;或许是因为他从不拒绝那些他并不觊觎、却被同事或政治权威加诸他肩上的职责——不得不说,他的肩膀相当坚实。他是农民之子(在一个“阶级标准”重要的国家,这一点一定起了作用),带着农民的常识与纯朴。他一定也察觉到围绕知名人物的恐惧,但这种恐惧对他来说想必是理所当然的,是权力地位的天然属性。然而,我不认为他自己曾让任何人感到恐惧——无论是他的妻子弗洛丽卡、女儿亚历山德拉,还是他的同事或学生——我听到的反馈也确实如此。}(11) 这团代替记忆的“迷雾”也没有给我带来任何因这类情境而产生的自觉或不自觉的满足感。我不认为自己在意识层面上有过这种感觉,但也不敢断言在早年无意识层面上从未偶尔触及过。如果有的话,那一定是转瞬即逝的,没有反映在某种固化尴尬的行为上。这并不是说我的虚荣没有投入到我所扮演的角色中!但如果说我毫无保留地投入这个角色,当时驱动我自我的,不是给“普通同事”留下深刻印象的野心,而是不断超越自己,以不断赢得“同侪”——尤其是那些信任我、早在我展现能力之前就接纳我为其中一员的前辈们——的持续尊敬。在我看来,我对成为恐惧对象的内在态度,是尽力忽略它,同时在它显现时尽力消解它——这种态度在六十年代我所属的那个环境(那个“微观世界”)中,可以说是典型的。

此后十年或十五年间,情况进一步显著恶化,至少从我偶尔听到的那个世界的迹象,以及我亲眼目睹甚至有时参与其中的情境来看是这样。不止一次,在我昔日最珍视的朋友或学生中,我面对的是熟悉而无可辩驳的轻蔑迹象;一种看似“无缘无故”的、旨在打击、羞辱、压倒的意愿。一阵轻蔑之风不知何时刮起,吹过这个我曾珍视的世界。它肆意吹刮,不顾“功绩”或“过失”,以其炽热的气息焚毁谦卑的志向与最炽烈的激情。在我当年的同伴中,有谁——各自被坚固的壁垒保护着,与“自己人”共存,沉浸(正如我曾经那样)在围绕其个人的柔和恐惧中——有谁能感受到这阵风?在我昔日的朋友中,我只知道一个,且仅一个,察觉到它并与我谈及,尽管未直呼其名。还有另一个,也曾在某天不情愿地感知到它,却在次日便急于忘却。\footnote{(12)\par 此处“次日”应按字面理解,而非隐喻。}(12) 因为对当年的朋友或我自己而言,感受到这阵风并接受它,也意味着愿意审视自己。