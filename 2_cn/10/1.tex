\section{(42) 孩子}

显然,必定还有些角落未被扫帚触及。无妨,它们自会引起我的注意,到时再处理也不迟。但就我那著名的“数学家过去”而言,大扫除已完成,毫无疑问。

现在,我再次确认自己并不比他人优越,可别又掉进那个老套的陷阱,以为自己比过去的自己更优秀!别以为现在的我——脱离了旋转木马之类的一切——比十五年前、甚至十五天前的我更“优秀”。这十五年间我确实学到了些东西,这点毋庸置疑,过去十五天也是如此,甚至从昨天起也是。每次学到东西,我便成熟一分,不再完全是原来的我。然而,学会某事时的我,并不比尚未学会时的我“更优秀”。成熟的果实并不比未熟或青涩的果实“更优秀”。一个季节并不比前一个季节“更优秀”。成熟果实的味道或许更宜人,或不那么宜人,因人而异。我感觉自己一年比一年更适应自己的状态,看来我内心的变化“合我口味”——但并非所有朋友或亲人都如此认同。每当我重拾数学,总会收到各方赞美,语气不外乎:“他之前干别的真是异想天开!现在一切回归正轨,恰逢其时!”看到有人改变,总会令人不安……

我学习,我成熟,我改变——有时甚至难以在回忆或他人意外的见证中认出曾经的自己。我在变,但也有某种东西始终“如一”。它似乎自始至终都在那里,或许从我出生时便已存在,甚至更早。近年来,我觉得自己能清楚地辨识它。我称它为“孩子”。因这东西,我此刻并不比生命中任何其他时刻更优秀;它一直都在,即便往往难以察觉它的存在。因这东西,我也不比任何人更优秀,任何人也不比我更优秀。在某些时刻或某些人身上,这孩子更为鲜明。它带来莫大的慰藉。但这并不意味着某人比他人“更优秀”,或比自己其他时刻更优秀。

通常,当我从事数学、做爱或冥想时,是那孩子在嬉戏。它并非总是唯一“玩耍”的存在。但若它不在,便无数学、无爱、无冥想。装模作样毫无意义——我极少演这种戏。

当然,不只有孩子。還有“自我”、“老板”或“大头目”,随你怎么称呼。显然,这老板对企业的运转不可或缺。既然有老板,必然有其作用。它负责后勤事务,但像所有老板一样,它总有种令人讨厌的侵占倾向。它极其自视甚高,总想比对面的老板更优秀。不管是否侵占,它只是老板,不是工人。它组织、指挥,当然也“收获”——将收益视为理所应当,将损失视为奇耻大辱。但它无法创造。唯有工人有创造之力,而那工人正是孩子。

老板与工人和谐共存的企业实属罕见。通常,工人的踪影全无,被锁在不知何处。老板假装接管了车间,结果可想而知。即便工人确实在场,老板也常与之交战,或激烈冲突,或小规模摩擦——这样的车间难有产出!有时,老板对工人抱有戒备的容忍,嘀咕着放手让他干,却始终紧盯着不放。这如同一场从未停歇的战争中不断续签的休战。工人则在休战中勉力劳作。

我刚完成的冥想,未必能神奇地消除我对数学的占有态度!至少,我需更仔细审视那些占有欲的表现——我才刚触及一处,并名之为此。这里不是详述之处,这篇“引言”已变为“引章”,而此章已然拖得太长!不过,今夜有件事“叮”地敲响了我,我想略作回顾——那是两三年前我带着几分惊讶记下的一件事。

当时我正钻研一个数学问题,具体是什么已记不清。某一刻(因某种我不明的缘由),我发现自己研究的这个问题或许已被他人审视,可能已在一本书中被详述,只要我去图书馆查阅即可。这一简单可能性带来的冲击令人震惊:一瞬间,欲望荡然无存。那个我或许已投入数周、还打算再投入更多时间的问题,突然对我毫无吸引力!这不是懊恼,而是兴趣骤然全失。即使那本书在我手中,我也不会费神翻开。

事实上,那可能性并未证实,于是欲望重燃,我继续投入,仿佛什么也没发生。但我仍感困惑。当然,若我真需要此事去完成其他工作,兴趣不会如此戏剧性地消退。我常重做已知之事,哪怕明知或怀疑其已为人知,也毫不在意。当时我正处于一种状态:以自己的方式、在当下视角下做事,比翻阅书籍或文章更省力,也远更有趣。我在“顺势而为”中奔向别处,奔向欲望所指引的目标。当然,我足够“在行”,知道那终点不在任何书本或文章中。

这让我意识到,数学工作——即便多年独力进行——并非纯粹个人的、个体的劳动,不像冥想,至少对我而言并非如此。我在数学中追逐的“未知”,若要如此强烈地吸引我,不仅需对我未知,更需对所有人未知。书本中写下的不再是未知,即便我从未耳闻。阅读书籍或文章从未吸引我,我总是尽可能避开。它们能告诉我的绝非未知,我对其的兴趣也不具备欲望的特质。那只是情境下的“兴趣”,对可能有用的信息感兴趣,如同工具,而非欲望的对象。

细想之下,我提及的事件似乎并非嫉妒或占有倾向的标志,也非虚荣受挫的迹象。我毫无懊恼或失望,只是前一刻还炽烈的欲望骤然消失。那时,我完全未想发表什么,也不曾幻想某天再发表什么。那欲望并非虚荣、知识积累或头衔渴求的表达——它确是真切的欲望,孩子沉迷游戏时的欲望。可突然间——什么也没了!谁能理解,我也不懂……抱歉!