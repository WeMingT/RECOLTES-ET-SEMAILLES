\section{(42) L'enfant}

C'est même sûr qu'il doit y avoir des recoins où le balai n'a pas passé. C'est pas grave, ils vont bien se signaler à mon attention et il sera toujours temps alors de m'en occuper. Mais pour ce qui est de mon fameux ``passé de mathématicien'', le gros nettoyage est fait, pas de doute.

Maintenant que je viens de voir une nouvelle fois que je ne suis pas meilleur que les autres, il ne faudrait pas que je retombe dans le sempiternel panneau de me prendre pour meilleur que moi-même ! De me prendre pour meilleur maintenant, sorti du manège et tout et tout, que celui que j'étais il y a quinze ans, ou quinze jours. J'ai appris quelque chose pendant ces quinze ans, ça c'est sûr, et pendant les quinze jours aussi et même depuis hier. Quand j'apprends quelque chose je mûris, je ne suis plus tout à fait le même. Je ne suis pas ``meilleur'' quand j'ai appris quelque chose, que quand cette chose à apprendre était encore devant moi. Un fruit plus mûr n'est pas ``meilleur'' qu'un fruit moins mûr, ou vert. Une saison n'est pas ``meilleure'' que celle qui la précède. Le goût du fruit le plus mûr peut être plus agréable, ou moins agréable, cela dépend des goûts. Je me sens mieux dans ma peau d'une année à l'autre, il faut croire que les changements qui se font en moi sont ``à mon goût'' - mais ils ne sont pas au goût de tous mes amis ou proches. Chaque fois que je me remets à faire des maths, je reçois de tous côtés des compliments, sur le ton : ``quelle idée aussi qu'il avait de faire autre chose ! Tout rentre dans l'ordre, il était temps !''. Ça inquiète de voir quelqu'un changer...

J'apprends, je mûris, je change - au point que parfois j'ai du mal à me reconnaître dans celui que j'étais et que je redécouvre, par un souvenir ou par le témoignage inattendu d'autrui. Je change, et il y a aussi quelque chose qui reste ``le même''. C'était là depuis toujours, depuis ma naissance sûrement, et peut-être dès avant. Il me semble que j'arrive à bien le reconnaître, depuis quelques années. Je l'appelle ``l'enfant''. Par cette chose, je ne suis pas meilleur en ce moment qu'en aucun autre moment de ma vie ; il était là, même si ça aurait été difficile souvent de deviner sa présence. Par cette chose aussi, je ne suis meilleur que personne, et personne n'est meilleur que moi. En certains moments ou en certaines personnes, l'enfant est plus présent. Et c'est une chose qui fait beaucoup de bien. Ça ne signifie pas que quelqu'un soit ``meilleur'' que quelqu'un d'autre, ou que lui-même à un autre moment.

Souvent, quand je fais des maths, ou quand je fais l'amour, ou quand je médite, c'est l'enfant qui joue. Il n'est pas toujours le seul à ``jouer''. Mais quand il n'est pas là, il n'y a ni maths, ni amour, ni méditation. C'est pas la peine de faire semblant - et c'est rare que j'aie joué cette comédie-là.

Il n'y a pas que l'enfant, c'est sûr. Il y a le ``moi'', le ``patron'' ou le ``grand chef'', qu'on l'appelle comme on voudra. Sûrement qu'il est indispensable, le patron, à la marche de l'entreprise. S'il y a un patron ça doit bien être pour quelque chose. Il veille à l'intendance, et comme tous les patrons, il a une fâcheuse tendance à devenir envahissant. Il se prend terriblement au sérieux et veut à toute fin être meilleur que le patron d'en face. Envahissant ou pas, il n'est que le patron, c'est pas lui l'ouvrier. Il organise, il commande, et il encaisse c'est sûr ! - il encaisse les bénéfices comme son dû, et subit les pertes comme un outrage. Mais il ne crée rien. Seul l'ouvrier a puissance de créer, et l'ouvrier n'est autre que l'enfant.

C'est rare, l'entreprise où patron et ouvrier s'entendent. Le plus souvent, on ne voit trace de l'ouvrier, enfermé Dieu sait où. C'est le patron qui a fait mine de prendre sa place dans l'atelier, avec les résultats qu'on devine. Et souvent aussi, quand l'ouvrier y est bel et bien, le patron lui fait la guerre, guerre violente ou d'escarmouches - de cet atelier ne sort pas grand chose! Parfois aussi il y a en le patron une tolérance méfiante vis à vis de l'ouvrier, il le laisse faire en maugréant, et sans le quitter de l'oeil. C'est comme une trêve constamment reconduite dans une guerre qui n'a jamais cessé. Et l'ouvrier peut travailler tant soit peu à la faveur de la trêve.

Ce n'est pas sûr du tout que par la vertu de la méditation que je viens de faire, l'attitude de possessivité en moi vis à vis de la mathématique ait disparu comme par enchantement! Il me faudrait pour le moins regarder de beaucoup plus près les manifestations de possessivité, dont je viens seulement d'effleurer une en l'appelant par son nom. Ce n'est pas le lieu dans cette ``introduction'', qui est devenue un ``chapitre introductif'', lequel à son tour déjà commence à se faire long ! Une chose pourtant avait fait ``tilt'' cette nuit, sur laquelle j'ai envie de revenir tant soit peu maintenant, une chose que j'avais notée avec une certaine surprise il y a deux ou trois ans.

J'étais lancé sur une question mathématique, je ne saurais plus dire quoi, et à un moment (par je ne sais quelle circonstance) il s'est trouvé que la question que je regardais avait peut-être déjà été regardée, qu'elle pouvait bien être traitée noir sur blanc dans tel bouquin, qu'il ne tenait qu'à moi de consulter à la bibliothèque. L'évocation de cette simple éventualité a eu un effet foudroyant, qui m'a stupéfié : d'un moment à l'autre, le désir avait disparu. Tout d'un coup, la question sur laquelle j'avais peut-être passé des semaines, et me disposais à en passer d'autres encore, avait perdu pour moi tout intérêt ! Ce n'était pas un dépit, c'était un manque d'intérêt soudain et total. Si j'avais eu le bouquin dans les mains, je n'aurais pas pris la peine de l'ouvrir.

En fait, l'éventualité ne s'est pas confirmée, et du coup le désir est revenu et j'ai continué sur ma lancée comme si rien ne s'était passé. Je restais quand même interloqué. Bien sûr, si j'avais vraiment eu besoin de ce que j'étais en train de faire pour faire autre chose, il n'y aurait pas eu une chute d'intérêt aussi spectaculaire. Ça m'est arrivé souvent de refaire des choses connues, sachant ou me doutant qu'elles l'étaient sans m'en soucier le moins du monde. J'étais alors sur une lancée où il était plus économique, et bien plus intéressant surtout, de faire les choses à ma façon, dans l'optique où elles se présentaient à moi, que d'aller fouiller dans des livres ou articles. Je le faisais alors ``dans la foulée'' vers autre chose, vers quoi me portait le désir. Et bien sûr, j'étais assez ``dans le coup'' pour savoir que ce qui était au bout ne se trouvait dans aucun livre ni article.

Cela rappelle à mon attention que le travail mathématique, alors même qu'il se ferait dans la solitude pendant des années, n'est pas un travail purement personnel, individuel, comme l'est la méditation - du moins pas chez moi. ``L'inconnu'' que je poursuis dans la mathématique, pour qu'il m'attire avec une telle force, ne doit pas seulement être inconnu de moi, mais inconnu de tous. Ce qui est écrit dans des livres mathématiques n'est pas un inconnu, alors même que moi-même n'en aurais jamais entendu parler. Lire un livre ou un article ne m'a jamais attiré, je l'ai évité chaque fois que j'ai pu. Ce qu'il peut me dire n'est jamais l'inconnu, et l'intérêt que je lui accorde n'a pas la qualité du désir. C'est un ``intérêt'' de circonstance, l'intérêt pour une information qui peut m'être utile, comme instrument d'un désir dont elle n'est nullement l'objet.

Réflexion faite, il ne me semble pas que l'événement que j'ai rapporté soit le signe de dispositions jalouses, possessives, le signe d'une vanité qui se trouvait déçue. Il n'y avait en moi aucun dépit, aucune déception, simplement la disparition soudaine d'un désir qui, l'instant d'avant encore, avait été intense. C'était en un temps où je ne songeais absolument pas à publier quoi que ce soit, ni qu'un jour il me prendrait fantaisie de publier encore quelque chose. Ce désir n'était pas expression de la vanité, de la fringale d'accumulation de connaissances, de titres et de crédits - c'était bel et bien un vrai désir, le désir de l'enfant passionné au jeu. Et tout d'un coup - plus rien ! Comprenne qui pourra, moi je ne comprends pas... Désolé !


