\section{(44) 再次扭转局面}

这会儿我已经有一年半没冥想了,除了去年十二月为了厘清一个紧急问题而花的几小时。而过去一年,我把大部分精力投入到数学「mathématique;mathematics」中。这股“浪潮”如同以往的浪潮——无论是数学浪潮还是冥想浪潮——来时从不预告。或者即便有所预告,我也从未听见!老板似乎对冥想略有偏爱,这得承认:每次冥想浪潮过后,总紧接着数学浪潮;我原以为冥想会永恒延续,而数学浪潮(在我看来)不过是几天、最多几周的事,却拖延开来,绵延数月,甚至可能,谁知道呢,延续数年。但老板终于明白,这些节奏不由他掌控,试图调节也无济于事。

不过,或许老板那“小小的偏爱”最终还是发生了某种转变,因为近一年来,这已成定局且决意如此:我至少要用几年时间“重拾数学”,甚至可以说这是正式的——我甚至向法国国家科学研究中心「CNRS;CNRS」申请了一个职位!更重要的是,一年前完全意想不到的是,我重新开始发表文章。即使在1981年的那场冥想(我之前提到过)之后,当做数学的欲望不再被视为次要时,我也从未想过会再次发表数学文章。其他东西倒还罢了,比如一本关于冥想、梦境或梦者「Rêveur;Dreamer」的书——即便如此,我当时忙于手头的事,根本无意为此写书!何必呢?!

因此,这是一个颇为重要的决定,影响了我未来几年的生活轨迹,而且是以一种不经意的方式做出的,我甚至不太清楚何时、如何决定的。有一天,当一叠打字笔记堆积起来时(瞧瞧!此前我一直满足于手写我的数学思考……\footnote{(38) \par 这些笔记实际上是对……的长信的延续,那封信成了第一章。它们被打字是为了让一位老友可读,也为了两三位其他朋友(尤其是罗尼·布朗「Ronnie Brown;Ronnie Brown」),我认为他们可能会感兴趣。这封信从未收到回复,收信人也没读过。近一年后(当我问他是否收到时),他真诚地惊讶于我竟会以为他可能去读它,鉴于人们对我期待的那种数学……}(38),内容涉及场与同伦模型等等……),事情就这么定了:出版吧!既然如此,不如大干一场,启动一个数学反思小系列,名字都现成,只需加上大写字母:“数学反思「Réflexions Mathématiques;Mathematical Reflections」”!这大致是我此刻从那著名的“迷雾”中还原出的记忆——那迷雾常代替我的回忆。显然,这回忆被大大压缩了。不管怎样,值得注意的是,这一切发生时,我甚至未曾停下来审视自己要去往何处,是什么推动或承载着我……这正是我在这次意外冥想后仍想做的事,好让它真正感到完结。

一个问题立刻浮现:我刚察觉的这“值得注意的事”,是老板(所谓?)“低调”的标志吗?他无论如何不愿干扰(哪怕只是偷瞄一眼……)这无需他介入的自发而美妙的运动?还是恰恰相反,这表明他彻底偏向一方,那所谓的“小小偏爱”正全力推动他朝数学方向前进?

只需把问题写下来,答案便显现了!不是那孩子——他只是投入了一场比以往更持久的游戏——决意要连绵X年不中断,乖乖填满所需页数,凑成一套体面的大写标题系列的若干卷!是老板把一切规划妥当,孩子只需照办。或许孩子乐意如此,谁也无法预知——但这是次要问题。孩子的欲望,至少在某种程度上,取决于环境,而环境主要由老板掌控。

老板已做出选择,这很清楚。而且他刚展现了一定灵活性,因为这场冥想已在一个多月前开始,并在他的善意注视下延续。不过,他的善意并非无私,因为冥想的具体产物——我正在写的这些笔记——将成为他已然设想建造的高塔中最美的基石,那塔由孩子-工人慷慨雕琢的石块垒成,孩子似乎颇为乐在其中。显然,现在夸他“灵活”还为时过早!过去一年半中,除了三个月前的几小时冥想,总共就这么点时间,实在有些单薄!

然而,我并不觉得这段时间里,有什么冥想的欲望被压抑或受挫。去年十二月的几小时里,我理清了思路,看到了该看的东西;那已足以转变一个不明朗的局面。我重拾中断的数学工作,无需中断其他事务。我不认为有什么冲突暗中重现——我指的是两年前已解决的那种冲突,这次以相反形式再现。老板有偏好是他的本性,也是他的权利——假装禁止自己有偏好才愚蠢(尽管比这更蠢的事也常发生……)。这并非冲突的标志,尽管它常是冲突之因。就目前情况看,实在无需责怪他缺乏灵活性!

这点看清后,我还需试着探究老板的“动机”,为何如此悄无声息地扭转局面,而若仔细审视,这转变颇为惊人。