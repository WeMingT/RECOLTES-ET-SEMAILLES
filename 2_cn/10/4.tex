\section{(45) 非大师之大师——或三腿马}

这立刻让我回想起那场从1981年7月持续到12月的冥想,此前我刚经历了四个月的某种数学狂热。那段有些疯狂的时期(从数学角度看非常富有成果\footnote{(39) \par 这段时间包括,特别值得一提的是,``穿越伽罗瓦理论的长征'',在``纲领概览''(第3段:“与儿童画相关的数域”)中有所提及。}(39))在一夜之间因一个梦而结束。那是一个梦,通过一个具有野性不可抗拒力量的寓言,描述了我生活中正在发生的事情——对那场狂热的寓言。其信息如闪电般清晰,然而我却花了两天紧张的工作才接受其显而易见的意义\footnote{(40) 显然

关于这个梦的工作是一封长篇英文信的对象,写给一位朋友兼同事,他在前一天曾匆匆来我家拜访。梦者用来从看似虚空中唤起这个逼真梦境的一些素材,显然借自这位我近十年未见的老友短暂拜访的片段。因此,在工作的第一天,与我过去的经验相反,我以为可以得出结论,这个梦与我的朋友有关,而非与我有关——应该是他做这个梦,而不是我!这是逃避梦的信息的一种方式,而我本该凭过去的经验立刻知道,这个梦只关乎我一人。在第一阶段肤浅的工作后的那个夜晚,我终于意识到了这一点;第二天,我在同一封信中继续深入。从那封难忘的信之后,我再未收到这位我最亲密朋友之一的任何消息。

这次工作是我唯一一次以书信形式(而且是用英语)进行的冥想,因此我没有留下任何书面痕迹。这一事件尤其让我印象深刻,在众多其他事件中,它显示出一切超越某种表面的工作迹象——将简单却通常被刻意忽视的事实带入光明——是如何在他人心中引发不安与恐惧。我将在后文再谈及此(见第47段,“孤独的冒险”)。}(40)。完成后,我知道自己该做什么。在随后六个月的工作中,我未再回头审视这个梦,但我的所作所为无非是更深入地探究其意义并完全吸收其信息。梦后的第三天,这个信息在表面和粗略的层面上已被理解。我尤其需要深入的是“我的”关系——我是说老板与两种欲望之间的关系,这两种欲望在我看来是对立的。

自那次冥想以来,我的生命中发生了太多事情,以至于它仿佛属于遥远的过去。如果我试图总结它关于“老板”动机的教导,我会这样说:在1970年“第一次觉醒”后的十二年间,老板押注的显然是“错误的马”:在数学与冥想之间(他乐于将两者对立),他选择了冥想。

这是一种表达方式,因为“冥想”这个词及其概念直到1976年10月才进入我的生活,那是五年前。但在1970年那个被重新粉刷一新的珍贵自我形象中,冥想六年后适时出现,用它的光芒提升了一种早已被察觉却从未被审视的态度或姿态,直到1981年的这场冥想。我称之为“大师综合症”,有些人(颇有道理地)也称之为我的“大师姿态”。我选择前者而非后者,或许是因为它便于混淆事物的本质,而我乐于维持这种混淆。从我幼年时起,我内心就有一种自发的教学乐趣,它与学习的乐趣并不对立,也绝非姿态。尤其是在我与学生的关系中,这种力量在起作用;这种关系虽浅显,但强烈且真诚,我的意思是:没有姿态。在我称之为1970年“觉醒”之后,那个曾熟悉的宇宙几乎消失,学生和“教学”——分享我所知且对我有意义、有价值的事物的机会——也随之远去,老板便尽可能地报复:与其教数学——那种仅够谋生、但除此之外配不上我新伟大身份的东西,我想象自己通过生活和榜样教导某种“智慧”。我当然小心翼翼,既不对自己也不对他人明确表达这种想法,当我听到这方面的回声时,我肯定会否认,痛心于朋友或亲人如此误解我。我费尽口舌解释,他们却固执不解,真是令人失望的学生!

我读过一两本克里希那穆提「Krishnamurti;Krishnamurti」的书,给我留下了深刻印象,我的头脑转瞬便吸收了某些信息和价值观\footnote{(41) 克里希那穆提,或解放变为枷锁

若说从这些书中我仅汲取了某些词汇,以及将其据为己有并最终以之为现实的倾向,未免不准确。我读到的第一本克里希那穆提的书让我深受震撼(即便我只来得及读几章),是因为他所说彻底颠覆了我认为理所当然的许多东西,我立刻意识到那些是从未察觉却一直呼吸的空气中的陈词滥调。同时,这次阅读首次让我注意到一些意义深远的事实,尤其是逃避现实——作为人类心智最强大、最普遍的条件之一。这为我理解此前难以理解的处境提供了关键钥匙,而这些处境(在我五六年后发现冥想前未曾察觉)一直是焦虑的源头。我立刻在周围验证了这种逃避的现实。这解开了一些焦虑,却未改变任何本质,因为我只在他人身上看到这种现实,想当然地认为它在我身上不存在,我是例外,证实了规则(却未对此惊人例外提出任何疑问)。其实,我对他人和自己毫无好奇。这把“钥匙”只有在渴望深入者手中才能开启。在我手中,它成了驱邪术和姿态。

1974年初,我首次不得不承认,我生命中的毁灭并非全来自他人,我内心有某种东西在吸引、滋养、延续它。那是一个谦卑与开放的时刻,适合焕然一新。然而,由于缺乏深入工作,这焕新仍外围而短暂。我内心的“某种东西”仍模糊。我清楚那是爱的缺失,但要更贴近地审视我在哪里、如何缺失了爱,它如何显现,又有哪些具体后果,这样的想法既非来自我此前认识的任何环境或人,也非来自克里希那穆提。(恰恰相反,K.乐于强调一切工作的虚妄,自动将其等同于自我的“成为之渴望”。)因此,仅凭借来的“智慧”作为指南,我别无他法,只能耐心等待“爱”如圣灵恩典降临我身。

然而,我在波谷深处学到的谦卑真理激起了一股强大新能量的浪潮,堪比两年半后推动我首次投身冥想的那股力量。当时,这能量并非完全未被利用。几个月后,因一次天意事故卧床不起时,它支撑了一场(书面)反思,那是我首次审视构成我与他人关系基础却从未言明的世界观,这世界观来自我的父母,尤其是母亲。我清楚意识到,这世界观已破产,无法解释人与人关系的现实,也无法促进我自身及与他人关系的成长。这反思带有“克里希那穆提风格”,也受其对真正理解工作的禁忌影响。然而,它使几个月前诞生的模糊而难以捉摸的认知变得切实而不可逆转。这认知,当时世上无书无人能赋予我。

要成为冥想,这反思尤缺对我自身及自我形象的审视,而非仅对世界观、对一个我不真正“有血有肉”在内的公理体系的审视。还缺在反思当下对自己的审视(这仍未达真正工作的境界);这审视会让我察觉借来的风格、对这些笔记文学性的某种自满、因而缺乏自发性与真实性。尽管不足,其直接影响于我与他人的关系相对有限,但在我看来,这反思是从起点迈向两年后更深焕新的一步,或许是必要的。那时,我终于发现了冥想——发现了一个未曾预料的事实:关于我自身有待发现之事——这些几乎完全决定了我生命历程及与他人关系本质的事物……}(41)。这足以让我相信一切已然达成(当然表面上否认如此)。我无需多读,我能即兴说出或写下最纯正的克里希那穆提,话语连贯无懈可击。但这美丽无瑕的话语从未对我或他人有任何用处。这持续多年,我却不以为意。冥想发现后,这术语一夜间离我而去,不留痕迹。我才知话语与认知之别。

大老板立刻调整方向:克里希那穆提丢一边,冥想上肩头!当然悄无声息,他现在得用全新手法。时代变了,这孩子在他腿间跑来跑去,有时眼尖得很。看来孩子当时忙别处去了。总之,直到五年后某个压力锅爆炸,孩子跑来看发生了什么,大老板的把戏才暴露。

那其实不算太久,才刚过两年,那个不露大师相的大师终于暴露——又一个伪装丢一边!可怜的老板,几乎要一丝不挂了。换句话说:“冥想”这匹马,取代了那匹无名马(绝不能叫“克里希那穆提式”!),回报实在可笑,尤其比起遥远年代老板还押注时“数学”那匹马的丰厚回报。他若如此长久押错,纯属惯性——他已换过一次注,不算常见,且需一次震撼事件全力推动\footnote{(42) 救赎的撕裂

所谓“震撼事件”,是1969年末发现我所属机构部分资金来自军方部门,这与我的基本公理相悖(至今依然)。这事件是连锁反应中的首件(一件比一件揭示!),导致我离开IHES「Institut des Hautes Etudes Scientifiques;Institute for Advanced Scientific Studies」,进而彻底改变环境与投入。

IHES英雄年代,迪厄多内「Dieudonné;Dieudonné」与我是唯二成员,也唯二赋予其科学界信誉与关注者,迪厄多内靠编辑“数学出版物”(首卷1959年问世,即IHES由莱昂·莫尚「Léon Motchane;Léon Motchane」创立次年),我靠“代数几何研讨会”。初年,IHES生存极脆弱,资金不定(靠几家公司慷慨赞助),唯一场地是巴黎蒂耶基金会借出的一间厅(明显不情不愿)供我研讨会用[近期IHES为庆25周年出版的小册子(尼科·库伊珀「Nico Kuiper;Nico Kuiper」好心寄我一本)对此艰难开端只字未提,或许认为不配隆重场合,去年盛大庆祝]。我自觉如与迪厄多内同为“科学共同创始人”,打算在此终老!我强烈认同IHES,我的离开(因同事漠视)如从另一“家”撕裂,最终却成解放。

回想,我早有焕新需求,不知从何时起。非巧合的是,离开IHES前一年,我的精力突然转向,抛下前日还炽手可热的任务与最吸引我的问题,在生物学家朋友米尔恰·杜米特雷斯库「Mircea Dumitrescu;Mircea Dumitrescu」影响下投身生物学。我本打算在IHES内长期投入(符合其跨学科使命)。这显然只是更深焕新需求的出口,在IHES“科学温室”氛围中无法实现,这焕新在“觉醒瀑布”中完成,我已提及七次,最后一次在1982年。“军方资金”事件天意般触发首次觉醒。军方与IHES前同事,最终都得我感激!}(42)。老板不爱换注——这次甚至像退回前注。

自1973年退居乡下,新马回报渐薄,与往昔相比尤甚。三年后冥想意外出现,稍有回升。1979年3月至7月有一巅峰插曲,我不赘述,那时我又成使徒,颂扬一部自创诗作中古老又崭新的智慧,最终未交出版社\footnote{“我创作的诗作”含许多我亲知之事,今在我生命及“生命一般”中仍重要如初,当时意欲出版。未付梓,主因我后来发现其形式刻意“诗化”,整体构思过分雕琢,多处缺乏自发性,有时僵硬或浮夸难忍。这时而浮夸的形式,反映我当时状态,显然常由“老板”笨拙领舞……}(43)。但两年后,大师彻底失灵,冥想马似断一腿(对老板回报而言)——无论手法如何,再扮大师无门!

此后不久,三腿马连同诗人使徒、非大师之大师及不敢提名的克里希那穆提一并丢弃。数学万岁!

我们拭目以待后续……