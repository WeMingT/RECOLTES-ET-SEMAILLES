\section{(43) 扫兴的老板——或压力锅}

我感觉自己终于完成了对我的数学家生涯的回顾。当然,我并未穷尽这一主题——要做到这一点需要好几卷书,假设这样一个主题能够被``穷尽''。这并不是我的目的。我的目的是弄清楚我是否参与并共同促成了某种我如今时而感受到的``氛围''的出现,如果是的话,是以何种方式。现在我已心知肚明,这感觉很好。进一步深入,去探究那些仅被瞥见或触及的事物,可能会非常有趣。有那么多引人入胜的事情值得去看、去做、去发现!至于我作为数学家的过去,我觉得为了承担这段过去,我需要审视的部分已经看清了。

当然,如果深入这场沉思,我无疑会了解到许多关于我现在生活的有趣事情。这项工作几乎每一步都让我感受到,我对这段过去是多么依恋,直到今天它在我对自己的形象以及与他人的关系中仍占据重要位置;尤其是在我与那些我在某种意义上已经离开的人的关系中。无疑,在这个过程中,我与这段过去的关系发生了转变,朝向一种超脱,或者一种更大的轻松。未来会告诉我更多。但很可能一种依恋仍会持续,只要我的数学激情尚未燃尽并满足——只要我还在``做数学''。我并不急于猜测或预测它是否会在我之前熄灭……

在超过十年的时间里,我曾以为这种激情已熄灭。更准确地说,我曾下令宣布它已熄灭。那是我停下做数学的那一天,也是我重新发现世界的那一天!在那之后的三年或四年里,我全身心投入一种极为激烈的活动,以至于我从前的激情恐怕连一丝缝隙都找不到来显现自己。那是充满学习的几年,在某种程度上仍停留在相当表面的层次。在随后的几年里,数学激情以突如其来的、完全无法预料的方式显现。这些发作持续几周或几个月,而我固执地忽视它们那相当明显的意义。我曾一劳永逸地决定,这种对做数学的渴望,这种显然毫无用处的东西,已经是过时之物,句号!然而,这个``毫无用处''的东西却不这么认为——而我这边,则保持聋病。

看似矛盾的是,在我发现冥想(1976年)之后,随着一种新激情进入我的生活,旧激情的重现变得尤为强烈,几乎是猛烈的——仿佛每次都有一个盖子在过大压力下被掀开。只是五年后,在事件的推动下——确实可以这么说——我才费心去审视发生了什么。这是我对一个看似界限分明的问题进行的最长一次冥想:我花了六个月顽强而紧张的工作,才绕过一座冰山,其可见的顶端已变得足够碍眼,迫使我几乎违背自己的意愿去探究一番。不得不承认存在一种冲突状态,从表面上看,这是两种力量或欲望的冲突:冥想的欲望和做数学的欲望。

在这场漫长的冥想中,我一步步了解到,我轻视的做数学的欲望,与我极度推崇的冥想欲望一样,都是孩童的渴望。孩童才不管什么轻视或大厨兼老板的那种谦虚骄傲!孩童的欲望接连涌现,随着时间与日子流逝,仿佛一场舞蹈中彼此诞生的动作。这是它们的本性。它们并不比一首歌的诗节或一首康塔塔、一首赋格曲的连续乐章更对立。是那个糟糕的指挥老板宣称这个乐章``好''、那个``坏'',在和谐之处制造了冲突。

这场冥想之后,老板变得温和些了,不再那么装模作样地插手他不该管的事。这次的工作很长,而我原以为几天就能完成。工作完成后,``结果''显得显而易见,用几句话就能概括\footnote{(37) \par 我想几乎无需多说,这场漫长的工作日复一日地显现出远不止我以简洁形式给出的``结果''。一次冥想工作与一次由特定问题驱动的数学工作并无不同。往往,追随这条路(无论它是否导向对初始问题的或多或少完整的澄清)的曲折经历,比初始问题或``最终结果''本身更引人入胜。}(37)。但如果有人在我开始或进行这项工作时就敏锐地告诉我这些话,恐怕对我毫无帮助。这项工作如此漫长,是因为阻力强大且深层。老板因此挨了不少教训,但他从未吭声,因为这是在一种他无法发脾气的氛围中进行的。可以肯定的是,这六个月用得很好,我无法省略它们;就像一个女人无法省略九个月的怀孕,最终生下像婴儿这样``显而易见''的东西。
