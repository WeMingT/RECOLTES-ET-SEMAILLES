\section{(43) Le patron trouble-fête - ou la marmite à pression}

J'ai le sentiment d'avoir finalement terminé cette rétrospective de ma vie de mathématicien. Bien sûr, je n'ai pas épuisé mon sujet - il y faudrait des volumes, à supposer qu'un tel sujet puisse être ``épuisé''. Ce n'était pas là mon propos. Mon propos était d'en avoir le coeur net si oui ou non j'avais été partie prenante et co-acteur dans l'apparition d'un certain ``air'' que je sens aujourd'hui par bouffées, et si oui, de quelle façon. J'en ai le coeur net maintenant, et ça fait du bien. Ça pourrait être passionnant d'aller plus loin, d'approfondir ce qui n'a été qu'entrevu ou effleuré. Il y a tant de choses passionnantes à regarder, à faire, à découvrir ! Pour ce qui est de mon passé de mathématicien, il me semble que ce qu'il fallait que je regarde, pour assumer ce passé, a été vu.

Sûrement, en approfondissant cette méditation, je ne manquerais pas d'apprendre bien des choses intéressantes sur mon présent. Une chose que ce travail m'a fait sentir déjà presque à chaque pas, c'est à quel point je suis resté attaché à ce passé, l'importance qu'il a eu jusqu'à aujourd'hui encore dans mon image de moi-même, et aussi dans ma relation aux autres ; surtout dans ma relation à ceux que j'ai, en un certain sens, quittés. Sûrement ma relation à ce passé s'est transformée au cours de ce travail, dans le sens d'un détachement, ou d'une plus grande légèreté. L'avenir m'en dira plus. Mais il est probable qu'un attachement restera, aussi longtemps que ne sera pas brûlée et assouvie ma passion mathématique - aussi longtemps que je ``ferai des maths''. Et je n'ai nul souci de vouloir deviner ou prédire si elle s'éteindra avant moi...

Pendant plus de dix ans j'avais crû cette passion éteinte. Il serait plus vrai de dire que j'avais décrété qu'elle était éteinte. C'était le jour où je me suis arrêté pour un temps de faire des maths, et où j'ai redécouvert le monde! Pendant trois ou quatre ans j'ai été absorbé alors par une activité si intense, que mon ancienne passion n'a pas dû trouver le moindre interstice par où se glisser pour se manifester. C'étaient des années d'apprentissage intense, à un certain niveau qui restait assez superficiel. Dans les années qui ont suivi celles-là, la passion mathématique s'est manifestée par des accès soudains, totalement imprévus. Ces accès duraient quelques semaines ou mois, et je m'obstinais à ignorer leur sens pourtant assez clair. J'avais décidé une bonne fois que la fringale de faire des maths, décidément bonne à rien, était désormais chose dépassée, point final ! La ``bonne à rien'' pourtant ne l'entendait pas de cette oreille - et moi de mon côté, je restais sourd.

Chose qui peut sembler paradoxale, c'est après la découverte de la méditation (en 1976), avec l'entrée dans ma vie d'une nouvelle passion, que les réapparitions de l'ancienne se sont faites particulièrement fortes, violentes presque - comme si à chaque fois un couvercle sautait sous l'effet d'une pression trop forte. C'est cinq ans plus tard seulement, sous la poussée des événements c'est le cas de le dire, que j'ai pris la peine d'examiner ce qui se passait. Ça a été la plus longue méditation que j'aie faite sur une question d'apparence bien délimitée : il m'a fallu six mois d'un travail obstiné et intense pour faire le tour d'une sorte d'iceberg, dont le sommet visible avait fini par devenir assez gênant pour m'obliger, à mon corps défendant presque, d'y aller voir. Force était de constater une situation de conflit, qui de toute apparence était le conflit de deux forces ou envies : l'envie de méditer, et l'envie de faire des maths.

Au cours de cette longue méditation, j'ai appris pas à pas que l'envie de faire des maths, que je traitais avec dédain, était, tout comme l'envie de méditer, que je valorisais à fond, un désir de l'enfant. L'enfant n'a rien à faire du dédain ni de la fierté modeste du grand chef et patron! Les désirs de l'enfant se suivent, au fil des heures et des jours, comme les mouvements d'une danse naissant les uns des autres. Telle est leur nature. Ils ne s'opposent pas plus que ne s'opposent les strophes d'un chant, ou les mouvements successifs d'une cantate ou d'une fugue. C'est le patron mauvais chef d'orchestre qui déclare que tel mouvement est ``bon'' et tel autre ``mauvais'' et qui crée le conflit là où il y a harmonie.

Après cette méditation, le patron s'est assagi, il fait moins mine de mettre son nez là où il n'a rien à faire. Le travail cette fois était long, alors que je croyais que ce serait fait en quelques jours. Une fois le travail fait, le ``résultat'' apparaît comme évident, et se formule en quelques mots \footnote{(37) \par Il est à peine besoin d'ajouter, je pense, que ce travail de longue haleine a fait apparaître, au jour le jour, bien autre chose encore que le ``résultat'' que je viens de livrer sous forme lapidaire. Il n'en va pas autrement pour un travail de méditation que pour un travail mathématique motivé par une question particulière qu'on se proposait d'examiner. Bien souvent les péripéties de la route suivie (qui mène ou ne mène pas vers un éclaircissement plus ou moins complet de la question initiale) sont plus intéressants que la question initiale ou que le ``résultat final''.}(37). Mais quelqu'un de perspicace m'aurait dit ces mots avant ou au cours du travail, que cela ne m'aurait sans doute avancé en rien. Si le travail a été si long, c'est que les résistances étaient fortes, et profondes. Le patron en a pris plein la gueule d'ailleurs, et il n'a jamais moufté, car ça se passait dans une ambiance où il n'y avait pas moyen qu'il se fâche. Ce qui est sûr, c'est que ça a été six mois bien employés, et dont je n'aurais pas pu faire l'économie ; pas plus qu'une femme ne peut faire l'économie des neufs mois de grossesse pour finalement accoucher de quelque chose d'aussi ``évident'' qu'un marmot.