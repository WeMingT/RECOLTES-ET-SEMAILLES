\section{(36) 欲望与冥想}

我所提及的那一夜,新的激情取代了永远消逝的旧有恐惧,那也是我发现冥想的夜晚。那是我第一次“冥想”的夜晚,它在一种迫切、紧急的需求压力下显现出来。在此前的几天,我仿佛被一波波焦虑淹没。如同所有焦虑一般,这或许是一种“起飞的焦虑”,它执意向我发出信号,指出我自身一个朴素而显而易见的现实,与一个已有四十年从未被我质疑的自我形象之间的脱离。无疑,当时在我内心既有强烈的认知渴求,也有相当大的逃避力量,以及想要摆脱焦虑、恢复从前平静的愿望。于是,一场激烈的工作展开了,持续了数小时直到结束,而我尚不知晓这一切的意义,更不知自己将走向何方。在这过程中,一个个托词被逐一识别;或者更准确地说,是这场工作让这些托词逐一显现,每一个都以一种内心信念的面貌出现。我终于肯将它们黑白分明地记录下来,仿佛要更深地融入其中,而此前它们一直处于一种模糊的状态,便于隐藏。我满心欢喜地记录着,毫无戒备,这些信念必定有某种诱人之处——当时我正处于一种毫无疑虑的心态,对于我而言,仅是将未曾言明的信念写成黑白文字,便是其真实性的无可辩驳的标志,是其正当性的证明。若非我内心那不识趣甚至有些放肆的欲望——我说的是认知的欲望——我本会在每次达到这“圆满结局”时停下来,而每阶段的结束也确实带着这种圆满结局的心态。然而,可怜的我!不知怎的,天晓得为何,我突发奇想,要更仔细地审视我刚写下的、令我完全满意的内容:它就写在那里,黑白分明,只需重读即可!于是我认真地、单纯地重读,发现有些地方稍有不妥,不是那么清晰,嗯,有点意思!再稍加细看,便清楚地发现根本不是那么回事,甚至可以说是胡扯,我刚把自己蒙骗了一番!每次这样的部分发现都如惊喜般降临,“哇!这可真是妙不可言!”这种愉悦的惊喜为思考注入了新的能量,继续前行,我们一定能找到最终答案,肯定就在此刻,只要顺势而为!稍作总结,理清思路……很快又一个内心信念浮现,带着“故事结局”的全部表象,我们只想相信这次一定是真的,但为了谨慎起见还是记下来,而且记录这些如此明智且感受深刻的内容本身就是一种乐趣,若不同意这些显而易见的诚意,真是用心险恶,这已是完美无缺了!

这就是新一阶段的结束,又一个圆满结局,若非那个淘气到极点的小坏蛋再次捣乱,我本会满足地停下来。这个无可救药的家伙又开始挑刺,硬要再次插手这最新的“最终答案”和圆满结局。拦不住他,新一轮阶段又开始了!

就这样,在四个小时里,一个个阶段接连展开,仿佛我剥开了一层又一层的洋葱(这是那晚结束时我脑海中浮现的意象),最终抵达核心——一个简单而显而易见的真相,说实话,这真相一目了然,然而我却在几天、几周(乃至我一生中)成功地将其掩藏在一层层“洋葱皮”之下,彼此遮掩。

这朴素真相的最终显现带来巨大的宽慰,一种意想不到的彻底解放。那一刻,我知道自己触及了焦虑的核心。前五天的焦虑确已消解、溶化,转化为在我内心刚刚形成的认知。这焦虑不仅从我眼前消失,如同整个冥想过程中那样,也如同前五天中的几次经历;它所转化的认知绝非某种观念,或是我为了平静而做出的妥协(正如那晚我曾偶尔为之);它不是我当时采纳或获取的外在事物,用以附着于我身。这是一种全然的认知,来自第一手的、朴素而显而易见的认知,从此成为我的一部分,如同我的血肉一般。此外,它被以清晰无误的语言表述——不是长篇大论,而是一个三四字的简短句子。这一表述是此前持续工作的最后一步,这一结果稍纵即逝、可逆,直到这最后一步完成。在整个过程中,对浮现的想法、呈现的观念进行小心甚至一丝不苟的表述,是这项工作的核心部分,每一次新的开始都是对刚走过阶段的反思,而这一阶段通过我刚写下的书面见证为我所知(无法在记忆的迷雾中被掩藏!)。

在发现与解放的那一刻之后的几分钟里,我也明白了刚刚发生之事的全部分量。我发现了一些比过去几天那朴素真相更为珍贵的东西。那就是我内在的力量,只要我有兴趣,便能探知我内心发生之事的究竟,了解任何分裂与冲突的情境——并由此完全凭借自身能力,化解我所意识到的任何内心冲突。这种化解并非如我前些年倾向相信的那样,来自某种恩赐的效果,而是通过激烈、顽强且一丝不苟的工作,运用我普通的才能得以实现。若有“恩赐”,它不在于冲突在我们内心的突然且永久消失,也不在于对冲突的理解如天上掉下的馅饼般唾手可得(如同丰饶之国的烤鸡!)——而在于认知欲望的存在或出现\footnote{(31)\par 我在此想到的是认知欲望的“阳”形式——那种探查、发现、命名显现之物……正是被命名,使得显现的认知不可逆、不可抹去(即便它后来被埋藏、遗忘,不再活跃……)。认知欲望的“阴”形式、“女性”形式,则在于一种开放、一种接纳,在于对浮现于我们存在更深层——思想无法触及之处——的认知的沉默欢迎。这种开放的出现,以及随之而来的突然认知——它暂时抹去一切冲突痕迹——仍如恩赐般降临,触及深处,尽管其显见效果或许短暂。我却怀疑,这种无声的认知,在我们生命中某些罕见时刻降临时,同样不可抹去,其作用甚至超越我们所能保留的记忆。} (31)。正是这欲望引领我,在几小时内抵达冲突的核心——正如爱的欲望让我们无误地找到通往心爱女性深处的路径。

无论是自我发现还是数学探索,若无欲望,一切所谓的“工作”不过是一场伪装,无处可去。充其量,它让满足于此的人“绕着罐子打转”无休无止——罐中的内容只留给饥饿而欲食之人!如所有人一样,我也有欲望与饥渴缺席之时。当涉及对自身的认知欲望时,我对自己及我所处情境的认知便停滞不动,我的行为不再基于认知,而是随根深蒂固的单纯机制而动,随之带来一切后果——有点像一辆由电脑而非人驾驶的汽车。但无论是冥想还是数学,若无欲望、无此饥渴,我不会假装“工作”。因此,我从未在未经学习的情况下冥想数小时,或做数学数小时\footnote{(32) 百事齐头并进,或:枯坐无益!

在我还从事函数分析的年代,即直到1954年,我常无休止地固执于一个无法解决的问题,即便我已无新思路,仅在旧有想法的圈子里打转,显然这些想法已不再“咬合”。至少有一整年是这样,尤其是在拓扑向量空间中的“逼近问题”,这个问题直到二十年后才由完全不同的方法解决,而这些方法在我当时的位置上无从触及。那时驱使我的不是欲望,而是固执,以及对自己内心状态的无知。那是艰难的一年——我生命中唯一一次做数学变得痛苦的时刻!这一经历让我明白,“枯坐”毫无用处——一旦工作到达停滞点,且停滞被察觉,就该转向他处——待更适宜的时机再回过头来处理悬而未决的问题。这一时机几乎总会很快到来——问题在无人触及的情况下自然成熟,单凭我在其他看似无关问题上兴致勃勃的工作即可促成。我深信,若当时固执己见,即便十年也一无所获!自1954年起,我在数学中养成了同时拥有众多“热铁在火中”的习惯。我一次只专注于其一,但通过某种不断重现的奇迹,我对其中之一的工作也惠及其他静静等待时机的项目。冥想亦是如此,从我初次接触时便无意中如此——随着反思的深入,亟待探究的炽热问题日益增多……} (32),而往往(若非总是)学到一些意外且不可预知的东西。这与我拥有而他人缺乏的才能无关,而仅源于我不在无真实欲望时假装工作。(正是这“欲望”的力量,独自孕育了我别处提及的那种苛求,使人在工作中不满足于大致了然,唯有彻悟——哪怕是谦卑的彻悟——方能满意。)在发现的领域,无欲望的工作是荒谬的伪装,正如无欲望的爱毫无意义。实话说,我从未有过浪费精力假装做无心之事的心动,世上有太多引人入胜之事可做,即便只是睡觉(与做梦……)——当睡意来临时。

我相信,正是在那同一夜,我明白了认知的欲望与认知及发现的能力乃同一事物。只要我们信任并追随它,这欲望便引领我们直抵所欲认知之物的核心。它亦使我们在无需刻意寻觅的情况下,找到最有效的认知方法,且最适合我们自身。对于数学,似乎自古以来,书写对任何“做数学”的人都是不可或缺的手段:做数学,首先是书写\footnote{(33) “年轻人的势利”,或纯洁的捍卫者

这并不意味着工作中无纸(或其变体黑板!)的时刻不重要,尤其在“敏感时刻”——新直觉浮现时,此时需以更整体、更直观的方式“结识”它,而非通过“逐件工作”,这种非正式反思阶段为此做准备。对我而言,这类反思多在床上或散步时进行,我认为其在总工作时间中占比相对较小。同样观察也适用于我迄今所实践的冥想工作。} (33)。在任何以智识为主的发现工作中,或许皆是如此。但在“冥想”——我意指自我发现的工作——中未必如此。然而,就我目前的情况而言,书写在冥想中一直是有效且不可或缺的手段。如同数学工作,它是固定反思节奏的物质载体,是注意力凝聚的标志与归依,否则我的注意力易如四散之风。此外,书写为刚完成的工作留下有形痕迹,随时可供回顾。在长时间的冥想中,常常有必要回顾前几日甚至几年前某冥想时刻的书面记录。

因此,思想及其一丝不苟的表述,在我迄今所实践的冥想中扮演重要角色。然而,它并非仅限于思想的工作。单凭思想无法把握生命。它尤其擅长察觉我们对自身及与他人关系的看法中的矛盾——这些矛盾往往巨大至荒诞;但通常,它不足以理解这些矛盾的意义。对于怀有认知欲望者,思想是常有用且有效的工具,甚至不可或缺,只要我们意识到其局限——在冥想中显而易见(在数学工作中则较隐蔽)。重要的是,在敏感时刻,当别的东西浮现时,思想须知趣地悄然退场——或许以突然而深刻的感情形式显现,而手或许仍在纸上奔走,为之赋予笨拙而结巴的表达……