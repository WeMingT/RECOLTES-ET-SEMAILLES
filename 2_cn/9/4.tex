\section{(36) Désir et méditation}

La nuit dont j'ai parlé, où une passion nouvelle a pris la place d'une vieille peur qui s'est évanouie à jamais, est la nuit aussi où j'ai découvert la méditation. C'est la nuit de ma première ``méditation'', apparue sous la pression d'un besoin impérieux, urgent, alors que j'avais été comme submergé dans les jours précédents par des vagues d'angoisse. Comme toute angoisse peut-être, c'était là une ``angoisse de décollage'', qui me signalait avec insistance le décollage entre une réalité humble et évidente concernant ma personne, et une image de moi vieille de quarante ans et jamais mise en doute par moi. Sûrement il devait y avoir une grande soif de connaître, à côté de forces de fuite considérables, et du désir d'échapper à l'angoisse, d'être tranquille comme avant. Il y a eu alors un travail intense, qui s'est poursuivi pendant quelques heures jusqu'à son dénouement, sans que je sache encore le sens de ce qui se passait et encore moins où j'allais. Au cours de ce travail, les faux-fuyants ont été reconnus l'un après l'autre ; ou pour mieux dire, c'est ce travail qui a fait apparaître un à un ces faux-fuyants, sous les traits chacun d'une intime conviction que je prenais enfin la peine de noter noir sur blanc comme pour mieux m'en pénétrer, alors qu'elle était restée jusque là dans un flou propice. Je la notais tout content, sans m'en méfier le moins du monde, elle devait avoir de quoi me séduire sûrement - dans les dispositions alors de celui qui ne doute de rien, et pour qui le seul fait d'avoir écrit noir sur blanc une conviction informulée était le signe irrécusable de son authenticité, la preuve qu'elle était fondée. S'il n'y avait eu en moi ce désir indiscret, pour ne pas dire indécent, le désir de connaître je veux dire, je me serais à chaque fois arrêté sur ce ``happy end'', et c'est bien dans ces dispositions du happy end que se terminait l'étape. Puis, malheur à moi ! il me prenait fantaisie, Dieu sait comment et pourquoi, de regarder d'un peu plus près ce que je venais d'écrire à mon entière satisfaction : c'était écrit là noir sur blanc, il y avait qu'à relire ! Et en relisant avec attention, naïvement, je sentais que ça clochait un tout petit peu, que ce n'était pas tellement clair, tiens tiens ! Puis, prenant la peine de regarder d'un peu plus près, il devenait clair que ce n'était pas ça du tout même, que c'était du bidon autant dire, que je venais de me faire prendre des vessies pour des lanternes ! Cette découverte partielle à chaque fois venait comme une fameuse surprise, ``ça alors ! elle est pas piquée de vers celle-là !'', une surprise joyeuse qui relançait la réflexion avec un afflux d'énergie nouvelle. En avant, on va finir par connaître le fin mot, sûrement il va venir pas plus tard que maintenant, il y a qu'à continuer sur la lancée ! Un petit bilan, faire le point\ldots{} et voilà déjà monter une autre intime conviction, avec toutes les apparences du ``fin mot de l'histoire'', nous on demande qu'à croire ça doit être ça cette fois, on va quand même noter par acquit de conscience et puis c'est un plaisir même de noter des choses aussi judicieuses et bien senties, faudrait vraiment avoir l'esprit mal tourné pour ne pas être d'accord, une bonne foi aussi évidente, on peut pas faire mieux c'est parfait comme ça !

C'était là la nouvelle fin d'étape, le nouveau happy end, sur lequel je me serais arrêté tout content, s'il n'y avait eu le mauvais garnement polisson au possible qui a nouveau se mettait à faire des siennes, s'avisant, incorrigible décidément, de mettre encore son nez dans ce dernier ``fin mot'' et happy end. Il y avait pas à l'arrêter, c'était reparti pour une nouvelle étape encore !

C'est ainsi que pendant quatre heures, les étapes se sont succédées une à une, comme un oignon dont j'aurais enlevé les couches les unes après les autres (c'est là l'image qui m'est venue à la fin de cette nuit-là), pour arriver à la fin des fins au coeur - à la vérité toute simple et évidente, une vérité qui crevait les yeux à vrai dire et que pourtant j'avais réussi pendant des jours et des semaines (et ma vie durant, pour tout dire) à escamoter sous cette accumulation de ``couches d'oignon'' se cachant les unes derrière les autres.

L'apparition enfin de l'humble vérité a été un soulagement immense, une délivrance inattendue et complète. Je savais en cet instant que j'avais touché au noeud de l'angoisse. L'angoisse de ces cinq derniers jours était bel et bien résolue, dissoute, transformée en la connaissance qui venait de se former en moi. L'angoisse n'avait pas seulement disparu de ma vue, comme tout au long de la méditation, et plusieurs fois aussi au cours des cinq jours précédents ; et la connaissance en quoi elle s'était transformée n'était nullement dans la nature d'une idée, d'une concession que j'aurais faite disons pour être quitte et tranquille (comme il m'était arrivé ici et là au cours de la même nuit) ; ce n'était pas une chose extérieure que j'aurais alors adoptée ou acquise pour l'adjoindre à ma personne. C'était une connaissance au plein sens du terme, de première main, humble et évidente, qui désormais était part de moi, tout comme ma chair et mon sang sont une part de moi. Elle était, de plus, formulée en termes clairs et sans équivoque - pas en un long discours, mais en une petite phrase toute bête de trois ou quatre mots. Cette formulation avait été l'étape ultime du travail qui venait de se poursuivre, qui restait éphémère, réversible aussi longtemps que ce dernier pas n'était pas franchi. Tout au long de ce travail, la formulation soigneuse, méticuleuse même, des pensées qui se formaient, des idées qui se présentaient, avait été une part essentielle de ce travail, dont chaque nouveau départ était une réflexion sur l'étape que je venais de parcourir, qui m'était connue par le témoignage écrit que je venais d'en faire (sans possibilité de l'escamoter dans les brouillards d'une mémoire défaillante !).

Dans les minutes qui ont suivi le moment de la découverte et de la délivrance, j'ai su aussi toute la portée de ce qui venait de se passer. Je venais de découvrir quelque chose d'un plus grand prix encore que l'humble vérité de ces derniers jours. Cette chose, c'était le pouvoir en moi, pour peu que je sois intéressé, de connaître le fin mot de ce qui se passe en moi, de toute situation de division, de conflit - et par là-même la capacité de résoudre entièrement, par mes propres moyens, tout conflit en moi dont j'aurais su prendre conscience. La résolution ne se fait pas par l'effet de quelque grâce, comme j'avais eu tendance à croire dans les années précédentes, mais par un travail intense, obstiné et méticuleux, faisant usage de mes facultés ordinaires. Si ``grâce'' il y a, elle est non dans la disparition soudaine et définitive d'un conflit en nous, ou dans l'apparition d'une compréhension du conflit qui nous viendrait toute cuite (comme les poulets au pays de Cocagne !) - mais elle est dans la présence ou dans l'apparition de ce désir de connaître\footnote{(31)\par Je pense ici à la forme ``yang'' du désir de connaître - celui qui sonde, découvre, nomme ce qui apparaît\ldots{} C'est d'avoir été nommée qui rend la connaissance apparue irréversible, ineffaçable (alors même qu'elle viendrait par la suite à être enterrée, oubliée, qu'elle cesserait d'être active\ldots{}). La forme ``yin'', ``féminine'' du désir de connaissance est dans une ouverture, une réceptivité, dans un silencieux accueil d'une connaissance apparaissant en des couches plus profondes de notre être, où la pensée n'a pas accès. L'apparition d'une telle ouverture, et d'une connaissance soudaine qui pour un temps efface toute trace de conflit, vient comme une grâce encore, qui touche profond alors que son effet visible est peut-être éphémère. Je soupçonne pourtant que cette connaissance sans paroles qui nous vient ainsi, en certains rares moments de notre vie, est toute aussi ineffaçable, et son action se poursuit au-delà même de la mémoire que nous pouvons en avoir.} (31). C'est ce désir qui m'avait guidé et mené en quelques heures au coeur du conflit - tout comme le désir d'amour nous fait trouver infailliblement le chemin qui mène au plus profond de la femme aimée.

Qu'il s'agisse de la découverte de soi ou de la mathématique, en l'absence de désir, tout soi-disant ``travail'' n'est qu'une simagrée, qui ne mène nulle part. Dans le meilleur des cas, elle fait ``tourner autour du pot'' sans fin celui qui s'y complaît - le contenu du pot est réservé à celui qui a faim pour manger ! Comme à tout le monde, il m'arrive que désir et faim soient absents. Quand il s'agit du désir de connaissance de moi-même, alors ma connaissance de ma personne et des situations dans lesquelles je suis impliqué reste inerte, et j'agis non pas en connaissance de cause, mais au gré de simples mécanismes invétérés, avec toutes les conséquences que cela implique - un peu comme une voiture qui serait conduite par un ordinateur, non par une personne. Mais qu'il s'agisse de méditation ou de mathématique, je ne songerais pas à faire mine de ``travailler'' quand il n'y a pas désir, quand il n'y a pas cette faim. C'est pourquoi il ne m'est pas arrivé de méditer ne serait-ce que quelques heures, ou de faire des maths ne serait-ce que quelques heures\footnote{(32) Cent fers dans le feu, ou : rien ne sert de sécher ! 

Au temps où je faisais encore de l' Analyse Fonctionnelle, donc jusqu'en 1954 il m'arrivait de m'obstiner sans fin sur une question que je n'arrivais pas à résoudre, alors même que je n'avais plus d'idées et me contentais de tourner en rond dans le cercle des idées anciennes qui, visiblement, ne ``mordaient'' plus. Il en a été ainsi en tous cas pendant toute une année, pour le ``problème d'approximation'' dans les espaces vectoriels topologiques notamment, qui allait être résolu une vingtaine d'années plus tard seulement par des méthodes d'un ordre totalement différent, qui ne pouvaient que m'échapper au point où j'en étais. J'étais mû alors, non par le désir, mais par un entêtement, et par une ignorance de ce qui se passait en moi. Ça a été une année pénible - le seul moment dans ma vie où faire des maths était devenu pénible pour moi ! Il m'a fallu cette expérience pour comprendre qu'il ne sert à rien de ``sécher'' - qu'à partir du moment où un travail est arrivé à un point d'arrêt, et sitôt l'arrêt perçu, il faut passer à autre chose - quitte à revenir à un moment plus propice sur la question laissée en suspens. Ce moment presque toujours ne tarde pas à apparaître - il se fait un mûrissement de la question, sans que je fasse mine d'y toucher par la seule vertu d'un travail fait avec entrain sur des questions qui peuvent sembler n'avoir aucun rapport avec celle-là. Je suis persuadé que si je m'obstinais alors, je n'arriverais à rien même en dix ans ! C'est à partir de 1954 que j'ai pris l'habitude en maths d'avoir toujours beaucoup de fers dans là feu en même temps. Je ne travaille que sur un d'eux à la fois, mais par une sorte de miracle qui se renouvelle constamment, le travail que je fais sur l'un profite aussi à tous les autres, qui attendent leur heure. Il en a été de même, sans aucun propos délibéré de ma part, dès mon premier contact avec la méditation - le nombre de questions brûlantes à examiner est allé augmentant de jour en jour, au fur et à mesure que la réflexion se poursuivait\ldots{}} (32), sans y avoir appris quelque chose ; et le plus souvent (pour ne pas dire toujours) quelque chose d'imprévu et imprévisible. Cela n'a rien à voir avec des facultés que j'aurais et que d'autres n'auraient pas, mais vient seulement de ce que je ne fais pas mine de travailler sans en avoir vraiment envie. (C'est la force de cette ``envie'' qui à elle seule crée aussi cette exigence dont j'ai parlé ailleurs, qui fait que dans le travail on ne se contente pas d'un à-peu-près, mais n'est satisfait qu'après être allé jusqu'au bout d'une compréhension, si humble soit-elle.) Là où il s'agit de découvrir, un travail sans désir est non-sens et simagrée, tout autant que de faire l'amour sans désir. A dire vrai, je n'ai pas connu la tentation de gaspiller mon énergie à faire semblant de faire une chose que je n'ai nulle envie de faire, alors qu'il y a tant de choses passionnantes à faire, ne serait-ce que dormir (et rêver\ldots{}) quand c'est le moment de dormir.

C'est dans cette même nuit, je crois, que j'ai compris que désir de connaître et puissance de connaître et de découvrir sont une seule et même chose. Pour peu que nous lui fassions confiance et le suivions, c'est le désir qui nous mène jusqu'au coeur des choses que nous désirons connaître. Et c'est lui aussi qui nous fait trouver, sans même avoir à la chercher, la méthode la plus efficace pour connaître ces choses, et qui convient le mieux à notre personne. Pour les mathématiques, il semble bien que l'écriture de tout temps a été un moyen indispensable, quelle que soit la personne qui ``fait des maths'' : faire des mathématiques, c'est avant tout écrire\footnote{(33) Le ``snobisme des jeunes'', ou les défenseurs de la pureté

Cela ne signifie pas que les moments du travail où le papier (ou le tableau noir, qui en est une variante ! est absent, ne soient importants dans le travail mathématique. Il en est ainsi surtout dans les ``moments sensibles'' où une intuition nouvelle vient d'apparaître, quand il s'agit de ``faire connaissance'' avec elle d'une façon plus globale, plus intuitive que par un ``travail sur pièces'', que ce stade informel de la réflexion prépare. Chez moi, ce genre de réflexion se fait surtout au lit ou en promenade, et il me semble qu'il représente une part relativement modeste du temps total consacré du travail. Les mêmes observations s'appliquent également au travail de méditation tel que je l'ai pratiqué jusqu'à présent.} (33). Il en va de même sans doute dans tout travail de découverte où l'intellect prend la plus grande part. Mais sûrement ce n'est pas le cas nécessairement de la ``méditation'', par quoi j'entends le travail de découverte de soi. Dans mon cas pourtant et jusqu'à présent, l'écriture a été un moyen efficace et indispensable dans la méditation. Comme dans le travail mathématique, elle est le support matériel qui fixe le rythme de la réflexion, et sert de repère et de ralliement pour une attention qui autrement a tendance chez moi à s'éparpiller aux quatre vents. Aussi, l'écriture nous donne une trace tangible du travail qui vient de se faire) auquel nous pouvons à tout moment nous reporter. Dans une méditation de longue haleine, il est utile souvent de pouvoir se reporter aussi aux traces écrites qui témoignent de tel moment de la méditation dans les jours précédents, voire même des années avant.

La pensée, et sa formulation méticuleuse, jouent donc un rôle important dans la méditation telle que je l'ai pratiquée jusqu'à présent. Elle ne se limite pas pour autant à un travail de la seule pensée. Celle-ci à elle seule est impuissante à appréhender la vie. Elle est efficace surtout pour détecter les contradictions, souvent énormes jusqu'au grotesque, dans notre vision de nous-mêmes et de nos relations à autrui ; mais souvent, elle ne suffit pas pour appréhender le sens de ces contradictions. Pour celui qui est animé du désir de connaître, la pensée est un instrument souvent utile et efficace, voire indispensable, aussi longtemps qu'on reste conscient de ses limites, bien évidentes dans la méditation (et plus cachées dans le travail mathématique). Il est important que la pensée sache s'effacer et disparaître sur la pointe des pieds aux moments sensibles où autre chose apparaît - sous la forme peut-être d'une émotion subite et profonde, alors que la main peut-être continue à courir sur le papier pour lui donner au même moment une expression maladroite et balbutiante\ldots
