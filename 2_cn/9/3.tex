\section{(35) 我的激情}

在我成年生活中,三大激情占据了主导地位,除此之外还有其他不同性质的力量。我最终认识到,这三大激情是同一深层冲动的三种表达;是认知冲动在我内心所采取的三条路径,而在无穷的世界中,这冲动本有无穷的路径可供选择。

首先在我生命中显现的是对数学的激情。十七岁时,刚从中学毕业,我放手追随一种单纯的兴趣,这兴趣随即演化为一种激情,在接下来的二十五年里指引了我的人生方向。我“认识”数学的时间远远早于我认识第一个女人(除了我出生时便认识的那位),而如今,在我成熟的年纪,我发现这种激情依然未被耗尽。它不再主宰我的生活,我也无意去主宰它。有时它沉睡,甚至让我以为它已熄灭,但它总会在毫无预兆时重现,依然如往昔般炽烈。它不再像从前那样吞噬我的生命,那时我将生命交付给它去吞噬。它继续在我生命中留下深刻的印迹,如同恋人身上挚爱女子的印痕。

我生命中的第二大激情是对女性的追寻。这种激情常常以寻找伴侣的面貌呈现于我。直到这一追寻接近尾声时,我才分辨出这两者的差异,那时我明白,我所追逐的东西无处可寻,或者说,它早已在我内心。对女性的激情真正得以展开,是在母亲去世之后(那是我初次恋爱并生下一个儿子之后的第五年)。那时,我二十九岁,组建了一个家庭,又生下三个孩子。对子女的依恋最初是与对母亲的依恋不可分割的一部分,是吸引我投向女性的那种力量的一部分。这是爱的激情所结出的果实之一。

我并未将这两种激情的共存视为冲突,无论是在最初还是后来。我必定是模糊地感知到了这两者深层的同一性,这种同一性在很久之后,第三种激情出现时,才清晰地呈现出来。然而,这两种激情对我生命的影响截然不同。对数学的热爱将我引向某个世界——数学对象的世界,这个世界无疑拥有它自身的“现实”,但并非人类生活展开的世界。对数学事物的深入了解几乎未曾让我认识自己,更不用说了解他人——对数学的发现冲动只会让我远离自己和他人。有时,两个或更多的人可能在这一冲动中产生共鸣,但这种共鸣是表面的,实际上让每个人既远离了自己,也远离了他人。因此,对数学的激情在我生命中并非一种成熟的力量,我怀疑这种激情能否促成任何人的成熟\footnote{(29) 害怕游戏与两兄弟

我想在此谈论对数学或其他纯粹智识活动的强烈而持久的投入。相反,这种激情的展开可能是一种重新认识我们内在被遗忘力量的方式,是一个与顽强物质较量的机会,并在这一过程中,通过真正属于我们个人的东西,更新并丰富我们的身份感——这样的展开很可能是一个内在旅程中的重要阶段,是成熟的一部分。} (29)。如果说我长期以来给予这种激情如此过度的位置,必定也是因为它让我得以逃避对冲突的认知,以及对自身的认知。

然而,性的冲动,无论我们愿不愿意,都将我们直接抛向与他人的相遇,直接抛入我们自身及他人内心的冲突之中!我在生命中对“伴侣”的追寻,是对无冲突幸福的追寻——这不是我乐于相信的认知冲动或性冲动,而是一种对他人及我自身冲突认知的无尽逃避。(这是我必须学会的两件事之一,才能让这场虚幻的追寻及其如影随形的焦虑终结……)幸运的是,无论我们如何逃避冲突,性总会迅速将我们拉回其中!

有一天,我不再拒绝冲突通过我所爱或曾爱的女性,以及由此爱情诞生的孩子们所执意带给我的教诲。当我终于开始倾听和学习时,在之后的数年里,我发现我所学的一切,皆通过我曾爱或正爱的女性而得\footnote{(30)\par 近几年来,我的孩子们接过了这一角色,向一个有时不愿接受的学生传授人类存在的奥秘……} (30)。直到1976年,四十八岁时,对女性的追寻是我生命中唯一重要的成熟力量。如果这种成熟只在随后的几年——也就是过去七年间——得以实现,那是因为我以一切可用的方式保护自己免于成熟(正如我从父母及我所处的环境中学会的那样)。其中最有效的方式便是我对数学激情的投入。

第三大激情出现在我生命中的那天——1976年10月的一个夜晚——我对学习的巨大恐惧消失了。那也是对平凡现实的恐惧,对关于我自身 прежде всего 或我珍视之人的朴素真相的恐惧。奇怪的是,在那晚之前,在我四十八岁时,我从未察觉自己内心的这种恐惧。我在恐惧被认出的那一夜发现了它,同时也迎来了新的激情——认知激情的又一次显现。可以说,它占据了被认出后恐惧所留下的位置。多年来,我清楚地在他人的身上看到这种恐惧,但由于一种奇异的盲点,我未曾在自己身上看到。害怕看见的恐惧阻止我看见这种害怕看见的恐惧!我如所有人一样,强烈依附于某种自我形象,这个形象自童年起大体未变。我所说的那晚,也是这一旧形象首次崩塌的夜晚。其他相似的形象接连取代它,在顽强的惯性力量支持下,维持了几天、数月,甚至一两年,直到在审视的目光下再次崩塌。懒于审视常常推迟这种新的觉醒——但害怕审视的恐惧从未再现。有好奇心的地方,恐惧便无立足之地。当我对自己产生好奇时,我并不害怕发现什么,正如我渴望弄清一个数学情境的究竟时那样:那是一种喜悦的期待,有时急切却始终坚定,准备迎接一切到来之物,无论预料之中还是意料之外——一种对明确信号的热切关注,这些信号在最初的虚假、半真和可能的混沌中揭示真相。

对自己的好奇中蕴含着爱,这种爱不因我们所见是否符合期待而动摇。实话说,对自己的爱已在那个夜晚前的数月中悄然萌芽,而那个夜晚也是这种爱化为行动、充满进取之时,可以说,它毫不留情地颠覆了外衣与布景!如我所述,其他外衣与布景很快如魔法般重现,又被依次颠覆,没有咒骂,也没有咬牙切齿……

在过去七年中,这种新激情在我生命中的显现,最终让我觉得像是波浪起伏的节奏,如同一次次平静而深广的呼吸。此处并非试图描摹其蜿蜒多变的轨迹,或与之对位的数学激情表现之场所。我已放弃试图掌控这两者的进程——如今,这双重运动共同调节着我生命的进程,或者更准确地说,它们本身就是我生命的进程。

在新的激情出现前的数月中——孕育与充实的月份——对女性的追寻开始改变面貌。它开始从曾浸透其中的焦虑中分离出来,仿佛一股“气息”从压迫中解放,重获其应有的幅度与节奏。又如一团半熄的火,因缺乏出口而窒息,在清新空气的吹拂下,突然迸发为噼啪作响、灵动活泼的火焰!

火焰已燃尽饥渴。一度看似永不满足的饥饿终于饱足。近两三年间,这一追寻似已燃尽,无灰烬残留,为两种激情的歌声与和声留出广阔空间。其一,是我青年时代的激情,三十年来,它助我与被否定的童年分离。其二,是我成熟之年的激情,它让我重拾了那个孩子,以及我的童年。