\section{(35) Mes passions}

Trois grandes passions ont dominé ma vie d'adulte, à côté d'autres forces de nature différente. J'ai fini par reconnaître en ces passions trois expressions d'une même pulsion profonde ; trois voies qu'a prise la pulsion de connaissance en moi, parmi une infinité de voies qui s'offrent à elle dans notre monde infini.

La première à se manifester dans ma vie a été ma passion pour les mathématiques. A l'âge de dix-sept ans, au sortir du lycée, lâchant les rênes à un simple penchant, celui-ci s'est déployé en une passion, qui a dirigé le cours de ma vie pendant les vingt-cinq ans qui ont suivi. J'ai ``connu'' la mathématique longtemps avant que je connaisse la première femme (à part celle que j'ai connue dès la naissance), et aujourd'hui en mon âge mûr, je constate qu'elle n'est toujours pas consumée. Elle ne dirige plus ma vie, pas plus que je ne prétends la diriger. Parfois elle s'assoupit, au point parfois que je la crois éteinte, pour réapparaître sans s'annoncer, aussi fougueuse que jamais. Elle ne dévore plus ma vie comme jadis, quand je lui donnais ma vie à dévorer. Elle continue à marquer ma vie d'une empreinte profonde, comme l'empreinte dans un amant de la femme qu'il aime.

La deuxième passion dans ma vie a été la quête de la femme. Cette passion souvent se présentait à moi sous les traits de la quête de la compagne. Je n'ai su distinguer l'une de l'autre que vers le temps où celle-ci se terminait, quand j'ai su que ce que je poursuivais ne se trouvait nulle part, ou aussi : que je le portais en moi-même. Ma passion pour la femme n'a pu vraiment se déployer qu'après la mort de ma mère (cinq ans après ma première liaison amoureuse, dont est né un fils). C'est alors, à l'âge de vingt-neuf ans, que j'ai fondé une famille, dont sont issus trois autres enfants. L'attachement à mes enfants a été à l'origine une part indissoluble de l'attachement à la mère, une part de cette puissance émanant de la femme qui m'attirait en elle. C'est un des fruits de cette passion de l'amour.

Je n'ai pas vécu la présence en moi de ces deux passions comme un conflit, ni dans les débuts, ni plus tard. J'ai dû sentir obscurément l'identité profonde des deux, qui m'est apparue clairement bien plus tard, après l'apparition dans ma vie de la troisième. Pourtant, les effets sur ma vie de l'une et l'autre passion ne pouvaient être que très différents. L'amour des mathématiques m'attirait dans un certain monde, celui des objets mathématiques, qui sûrement a sa propre ``réalité'' à lui, mais qui n'est pas celui où se déroule la vie des hommes. L'intime connaissance de choses mathématiques ne m'a rien appris sur moi-même autant dire, et encore moins sur les autres - l'élan de découverte vers la mathématique ne pouvait que m'éloigner de moi-même et des autres. Il peut y avoir parfois communion de deux ou plusieurs dans ce même élan, mais c'est là une communion à un niveau superficiel, qui en fait éloigne chacun et de lui-même et des autres. C'est pourquoi la passion pour la mathématique n'a pas été dans ma vie une force de maturation, et je doute qu'une telle passion puisse favoriser une maturation en quiconque\footnote{(29) La peur de jouer \& Les deux frères

Je veux parler ici d'un investissement intense et de longue haleine dans la mathématique, ou dans une autre activité entièrement intellectuelle. Par contre, le déployement d'une telle passion, qui peut être une façon de refaire connaissance avec une force oubliée en nous, et l'occasion de se mesurer à une substance réticente et chemin faisant aussi, de renouveler et enrichir notre sentiment d'identité par quelque chose qui nous soit vraiment personnel - un tel déployement peut fort bien être une étape importante dans un itinéraire intérieur, dans un mûrissement.} (29). Si j'ai donné à cette passion une place aussi démesurée dans ma vie pendant longtemps, c'est sûrement aussi, justement, parce qu'elle me permettait d'échapper à la connaissance du conflit et à la connaissance de moi-même.

La pulsion du sexe, par contre, que nous le voulions ou non, nous lance droit à la rencontre d'autrui, et droit dans le noeud du conflit en nous-mêmes comme en l'autre ! La quête de ``la compagne'' dans ma vie, elle, a été la quête de la félicité sans conflit - ce n'était pas la pulsion de connaissance, la pulsion du sexe, comme il me plaisait à croire, mais une fuite sans fin devant la connaissance du conflit en l'autre et en moi-même. (C'était là une des deux choses qu'il me fallait apprendre, pour que cette quête illusoire prenne fin, et l'inquiétude qui l'accompagne comme son ombre inséparable\ldots) Heureusement, on a beau fuir le conflit, le sexe se charge de nous y ramener vite fait !

Un jour j'ai renoncé à récuser l'enseignement qu'obstinément le conflit m'apportait, à travers les femmes que j'aimais ou que j'avais aimées, et à travers les enfants nés de ces amours. Quand j'ai commencé enfin à écouter et à apprendre, et pendant des années encore, il se trouvait que tout ce que j'apprenais, c'est par les femmes que j'avais aimées ou que j'aimais que je l'apprenais\footnote{(30)\par Depuis quelques années, ce sont mes enfants qui ont pris le relais, pour enseigner à un élève parfois réticent les mystères de l'existence humaine...} (30). Jusqu'en 1976, à l'âge de quarante-huit ans, c'est la quête de la femme qui a été la seule grande force de maturation dans ma vie. Si cette maturation ne s'est faite que dans les années qui ont suivi, donc depuis sept ans, c'est parce que je m'en préservais (comme j'avais appris à le faire par mes parents et par les entourages que j'ai connus) par tous les moyens à ma disposition. Le plus efficace de ces moyens était mon investissement dans la passion mathématique.

Le jour où est apparu dans ma vie la troisième grande passion - une certaine nuit du mois d' Octobre 1976 - s'est évanouie la grande peur d'apprendre. C'est la peur aussi de la réalité toute bête, des humbles vérités concernant ma personne avant tout, ou des personnes qui me sont chères. Chose étrange, je n'avais jamais perçu cette peur en moi avant cette nuit, à l'âge de quarante-huit ans. Je l'ai découverte la nuit même où est apparue cette nouvelle passion, cette nouvelle manifestation de la passion de connaître. Celle-ci a pris, si on peut dire, la place de la peur enfin reconnue. Cela faisait des années que ce voyais cette peur en autrui bien clairement, mais par un étrange aveuglement, je ne la voyais pas en moi-même. La peur de voir m'empêchait de voir cette même peur de voir ! J'étais fortement attaché, comme tout le monde, à une certaine image de moi-même, qui pour l'essentiel n'avait pas bougé depuis mon enfance. La nuit dont je parle est celle aussi où, pour la première fois, cette vieille image-là s'est affaissée. D'autres images à sa ressemblance ont pris sa suite, se maintenant pendant quelques jours ou mois, voire un an ou deux, à la faveur de forces d'inertie tenaces, pour s'affaisser à leur tour sous un regard scrutateur. La paresse de regarder souvent retardait un tel nouvel éveil - mais la peur de regarder n'est jamais réapparue. Où il y a curiosité, la peur n'a plus de place. Quand il y a en moi une curiosité pour moi-même, il n'y a pas plus de peur de ce que je vais trouver que lorsque j'ai envie de connaître le fin mot d'une situation mathématique : il y a alors une expectative joyeuse, impatiente parfois et pourtant obstinée, prête à accueillir tout ce qui voudra bien venir à elle, prévu ou imprévu - une attention passionnée à l'affût des signes sans équivoque qui font reconnaître le vrai dans la confusion initiale du faux, du demi-vrai et du peut-être.

Dans la curiosité pour soi-même, il y a amour, que ne trouble aucune peur que ce que nous regardons ne soit conforme à ce que nous aimerions y voir. Et à vrai dire, l'amour de moi-même avait éclos en silence dans les mois déjà qui avaient précédé cette nuit, qui est celle aussi où cet amour a pris forme agissante, entreprenante si on peut dire, bousculant sans ménagement costumes et décors ! Comme j'ai dit, d'autres costumes et décors sont réapparus bientôt comme par enchantement, pour être bousculés à leur tour, sans invectives ni grincements de dents...

Les manifestations de cette nouvelle passion dans ma vie en ces dernières sept années ont fini par m'apparaître comme le haut-et-bas mouvant de vagues se suivant les unes les autres, comme les souffles d'une respiration vaste et paisible. Ce n'est pas ici le lieu d'essayer d'en tracer la ligne sinueuse et changeante, ou celle, en contrepoint, des manifestations de la passion mathématique. J'ai renoncé à vouloir régler le cours de l'une ou de l'autre - c'est ce double mouvement plutôt de l'une et l'autre qui aujourd'hui règle le cours de ma vie - ou pour mieux dire, qui en est le cours.

Dans les mois déjà qui avaient précédé l'apparition de la nouvelle passion - mois de gestation et de plénitude - la quête de la femme s'est mise à changer de visage. Elle a commencé alors à se séparer de l'inquiétude dont elle avait été imprégnée, comme un "souffle" encore qui se serait libéré d'une oppression qui avait pesé sur lui, et qui retrouverait l'amplitude et le rythme qui sont les siens. Ou comme un feu qui aurait couvé s'étouffant à demi, faute d'échappée, et qui sous un soufflé d'air frais se déployerait soudain en flammes crépitantes, agiles et vives !

Le feu a brûlé à satiété. Une faim qui semblait inextinguible s'est trouvée rassasiée. Depuis deux ans ou trois, il semble bien que cette quête-là est consumée sans résidu de cendres, laissant champ libre au chant et contre-chant de deux passions. L'une, la passion de mes jeunes années, m'avait pendant trente ans servi à me séparer d'une enfance reniée. L'autre est la passion de mon âge mûr, qui m'a fait retrouver et l'enfant, et mon enfance.



