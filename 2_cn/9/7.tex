\section{(39) Belle de nuit, belle de jour (ou : les écuries d' Augias)}

Le souvenir de l'émerveillement en un de mes enfants se situe tout à la fin des années cinquante et tout au début des années soixante. S'il ne m'est pas resté de semblable souvenir pour les autres enfants qui sont nés par la suite, c'est peut-être que ma propre capacité d'émerveillement s'était émoussée, que j'étais devenu trop lointain pour communier en le ravissement d'un de mes enfants, ou pour en être seulement témoin.

Je n'ai jamais songé encore à suivre les vicissitudes de cette capacité dans ma vie, de mon enfance jusqu'à aujourd'hui. Sûrement il y aurait là un fil conducteur, un ``détecteur'' d'une grande sensibilité. Si je n'ai jamais songé à suivre ce fil, c'est sûrement que cette capacité est d'une nature si humble, d'aspect si insignifiant presque, que l'idée ne me serait guère venue d'y accorder une attention particulière, absorbé que j'étais à découvrir et à sonder ce que j'appelais ``les grandes forces'' dans ma vie (qui continuent aujourd'hui encore à s'y manifester). Pourtant, cette capacité d'aspect si humble fournit un signe entre tous de la présence ou de l'absence de la ``force'' en nous la plus rare et du plus grand prix\ldots

Je n'ai jamais été entièrement coupé de cette force, à travers toute ma vie d'adulte. Quelque aride par ailleurs qu'ait pu devenir ma vie, je retrouvais dans l'amour l'émerveillement de l'enfant, le ravissement de la découverte. A travers bien des déserts, la passion de l'amour est resté le lien vivant et vigoureux avec quelque chose que j'avais quitté, un cordon ombilical qui continuait en silence à me nourrir d'un sang chaud et généreux. Et pendant longtemps aussi l'émerveillement en la femme aimée était inséparable de l'émerveillement en les nouveaux êtres qu'elle enfantait - ces êtres tout neufs, infiniment délicats et intensément vivants qui attestaient et héritaient de sa puissance.

Mais mon propos ici est surtout de suivre tant soit peu les vicissitudes de cette ``force d'innocence'' à travers ma vie de mathématicien, à l'époque où j'ai fait partie du ``monde des mathématiciens'', de 1948 à 1970. Sûrement, l'émerveillement n'a jamais imprégné ma passion mathématique à un point comparable comme dans la passion d'amour. Chose étrange, si j'essaye de me souvenir d'un moment particulier de ravissement ou d'émerveillement, dans mon travail mathématique, je n'en trouve aucun! Mon approche des mathématiques, depuis l'âge de dix-sept ans quand j'ai commencé à m'y investir à fond, a été de me poser des grandes tâches. C'étaient toujours, dès le début, des tâches de ``mise en ordre'', de grand nettoyage. Je voyais un apparent chaos, une confusion de choses hétéroclites ou de brumes parfois impondérables, qui visiblement devaient avoir une essence commune et receler un ordre, une harmonie encore cachée qu'il s'agissait de dégager par un travail patient, méticuleux, souvent de longue haleine. C'était un travail souvent à la serpillère et au balais-brosse, pour la grosse besogne qui déjà absorbait une énergie considérable, avant d'en venir aux finitions au plumeau, qui me passionnaient moins mais qui avaient aussi leur charme et, en tous cas; une évidente utilité.

Il y avait dans le travail au jour le jour une satisfaction intense de voir peu à peu se dégager cet ordre qu'on devinait, qui toujours se révélait plus délicat, d'une texture plus riche que ce qui avait été entrevu et deviné. Le travail a été riche constamment en épisodes imprévus, surgissant le plus souvent de l'examen de ce qui pouvait sembler un détail infime et qu'on avait jusque là négligé. Souvent le fignolage de tel ``détail'' jetait une lumière inattendue sur le travail fait des années auparavant. Parfois aussi, il conduisait à des intuitions nouvelles, dont l'approfondissement devenait l'objet d'une autre ``grande tâche''.

Ainsi, dans mon travail mathématique (à part ``l'année pénible'' vers 1954 dont j'ai eu occasion de parler), il y avait un suspense continuel, l'attention constamment était maintenue en haleine. La fidélité à mes ``tâches'' m'interdisait d'ailleurs des échappées trop lointaines, et je rongeais mon frein dans une impatience d'être arrivé au bout de toutes et m'élancer enfin dans l'inconnu, le vrai - alors que la dimension de ces tâches était devenue telle déjà, que pour les mener à bonne fin, même avec l'aide de bonnes volontés qui avaient fini par arriver à la rescousse, le restant de mes jours n'y aurait pas suffi!

Mon principal guide dans mon travail a été la recherche constante d'une cohérence parfaite, d'une harmonie complète que je devinais derrière la surface turbulente des choses, et que je m'efforçais de dégager patiemment, sans jamais m'en lasser. C'était un sens aîgu de la ``beauté'', sûrement, qui était mon flair et ma seule boussole. Ma plus grande joie a été, moins de la contempler quand elle était apparue en pleine lumière, que de la voir se dégager peu à peu du manteau d'ombre et de brumes où il lui plaisait de se dérober sans cesse. Certes, je n'avais de cesse que quand j'étais parvenu à l'amener jusqu'à la plus claire lumière du jour. J'ai connu alors, parfois, la plénitude de la contemplation, quand tous les sons audibles concourent à une même et vaste harmonie. Mais plus souvent encore, ce qui était amené au grand jour devenait aussitôt motivation et moyen d'une nouvelle plongée dans les brumes, à la poursuite d'une nouvelle incarnation de Celle qui restait à jamais mystérieuse, inconnue - m'appelant sans cesse, pour La connaître encore\ldots

Le plaisir et le ravissement de Dieudonné était surtout, il me semble, de voir la beauté des choses se manifester en pleine lumière, et ma joie a été avant tout de la poursuivre dans les replis obscurs des brumes et de la nuit. C'est là peut-être la différence profonde entre l'approche de la mathématique chez Dieudonné, et chez moi. Le sens de la beauté des choses, pendant longtemps tout au moins, n'a pas dû être moins fort en moi qu'en Dieudonné, alors qu'il s'est peut-être émoussé au cours des années soixante, sous l'action d'une fatuité. Mais il semblerait que la perception de la beauté, qui se manifestait chez Dieudonné par l'émerveillement, prenait chez moi des formes différentes: moins contemplatives, plus entreprenantes, moins manifestes aussi au niveau de l'émotion ressentie et exprimée. S'il en est ainsi, mon propos serait donc de suivre les vicissitudes de cette ouverture en moi à la beauté des choses mathématiques, plutôt que du mystérieux ``don d'émerveillement''.