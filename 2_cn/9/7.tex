\section{(39) 夜美人,日美人(或:奥吉亚斯的马厩)}

我对我的一个孩子的惊叹的记忆可以追溯到五十年代末和六十年代初。如果我对后来出生的其他孩子没有类似的记忆,可能是因为我自己的惊叹能力已经减弱,我变得太遥远,无法与我孩子的欣喜共鸣,或者甚至无法见证它。

我从未想过要追踪这种能力在我生命中的起伏,从我的童年到今天。肯定会有一个线索,一个极具敏感性的``探测器''。如果我从未想过要追随这条线索,那肯定是因为这种能力本质上如此谦逊,外表几乎微不足道,以至于我几乎不会想到要特别关注它,因为我全神贯注于发现和探索我所说的``伟大力量''(这些力量今天仍然在我的生活中继续显现)。然而,这种外表如此谦逊的能力提供了一个在我们身上最稀有和最宝贵的``力量''存在或缺失的标志……

在我成年后的生活中,我从未完全与这种力量断绝联系。尽管我的生活变得多么贫瘠,我在爱中重新发现了孩子的惊叹,发现的喜悦。穿过许多沙漠,爱的激情一直是我与我离开的东西的生动而有力的联系,一根持续默默地用温暖而慷慨的血液滋养我的脐带。长期以来,对心爱女人的惊叹与对她所生的新生命的惊叹密不可分——这些全新的、极其微妙和充满活力的生命证明并继承了她的力量。

但我在这里的目的是稍微追踪一下这种``天真力量''在我作为数学家的生活中的起伏,从1948年到1970年我成为``数学家世界''的一部分的那段时间。当然,惊叹从未像在爱的激情中那样渗透到我的数学激情中。奇怪的是,如果我试图回忆起在我的数学工作中某个特定的欣喜或惊叹的时刻,我一个也想不起来!自从我十七岁开始全身心投入数学以来,我对数学的approach一直是给自己设定伟大的任务。从一开始,这些任务就一直是``整理''、大扫除的任务。我看到了一个明显的混乱,异质事物的混杂或有时是无形的迷雾,它们显然应该有一个共同的本质,并隐藏着一个秩序,一种尚未被揭示的和谐,需要通过耐心的、细致的、通常是长期的工作来揭示。这通常是一个用拖把和刷子工作的过程,对于已经消耗了大量精力的粗重工作,然后是使用掸子进行收尾工作,这让我不那么感兴趣,但也有其魅力,并且在任何情况下都有明显的用处。

在日常工作中,有一种强烈的满足感,看到逐渐显现出我们猜测的秩序,这种秩序总是比预见和猜测的更微妙,质地更丰富。工作一直充满了意想不到的情节,通常是由于对似乎是微不足道的细节的检查,而这些细节之前被忽视了。通常,对某个``细节''的打磨会为几年前所做的工作投下意想不到的光芒。有时,它也会导致新的直觉,深化这些直觉成为另一个``伟大任务''的目标。

因此,在我的数学工作中(除了我曾提到过的1954年左右的``艰难年份''),有一种持续的悬念,注意力始终保持警觉。对我的``任务''的忠诚禁止我走得太远,我在不耐烦中咬紧牙关,渴望完成所有任务并最终跃入未知,真正的未知——而这些任务的规模已经变得如此之大,以至于即使有最终前来救援的善意帮助,要完成它们,我余下的日子也不够!

我工作中的主要指南是不断追求完美的连贯性,追求我猜测在事物动荡表面背后的完全和谐,并努力耐心地揭示它,从未厌倦。这是一种敏锐的``美''感,毫无疑问,这是我的嗅觉和唯一的指南。我最大的快乐,与其说是当它在明亮的光线下出现时凝视它,不如说是看到它逐渐从阴影和迷雾中显现出来,而它喜欢不断地躲藏在其中。当然,我只有在成功地将它带到最清晰的日光下时才会罢休。然后,我有时会体验到沉思的充实感,当所有可听的声音都汇聚成一个宏大而和谐的整体时。但更常见的是,被带到光天化日之下的东西立即成为一种动力和手段,促使我再次潜入迷雾中,追求那个永远神秘、未知的存在的另一个化身——她不断呼唤我,让我再次认识她……

迪厄多内「Dieudonné;Dieudonné」的快乐和欣喜,我认为,主要是看到事物的美在明亮的光线下显现,而我的快乐首先是在迷雾和黑夜的幽暗中追求它。这可能就是迪厄多内和我对数学的approach之间的深刻差异。对事物美的感知,至少在很长一段时间内,在我身上并不比在迪厄多内身上弱,尽管在六十年代,由于自负的影响,它可能已经减弱了。但似乎对美的感知,在迪厄多内身上通过惊叹表现出来,而在我身上则采取了不同的形式:不那么沉思,更有进取心,在情感层面上也不那么明显。如果是这样的话,我的目的就是追踪这种对数学事物美的开放性在我身上的起伏,而不是神秘的``惊叹的天赋''。