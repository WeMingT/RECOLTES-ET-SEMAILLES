\section{(37) 惊奇}

这场关于冥想发现的回顾完全出乎意料地降临,几乎违背了我的初衷——这绝非我起初打算探讨的内容。我原本想谈谈惊奇。那一夜,充满了诸多事物,也充满了面对这些事物的惊奇。在工作过程中,每揭露一个新的托词,便有一种难以置信的惊奇,仿佛我曾心满意足地——简直不可思议!——将一件粗糙的、用粗线缝制的戏服当作了最真实的东西,且态度严肃至极!此后多年,在那第一次冥想之夜的后续岁月中,我多次重拾这种惊奇,面对我所发现的事实之巨大,以及此前掩盖这些事实的托词之粗劣。最先吸引我的是其滑稽的一面,我由此开始探索我内心那未曾料想的世界,这个世界在日复一日、月复一月、年复一年的揭示中,展现出惊艳的丰富。那第一夜,尽管如此,我惊奇的对象已不仅限于闹剧般的片段。那是我首次重新接触内心沉睡已久的某种力量的一夜,那力量的本质对我而言尚不明朗,只知它确是一种力量,且随时为我所用。

而在此前的数月中,我已因一种自始便在我内心的东西而充满无声的惊奇,那时我刚刚与之重新建立联系。我感受到这东西并非一种力量,而更像一种隐秘的柔美,一种平静却又令人心动的美。后来,在发现那长久被忽视的力量的狂喜中,我遗忘了那些沉默孕育的月份,仅有一些零散的诗篇见证了它们——那些爱的诗篇,若置于我的冥想笔记中,或许大多时候会显得格格不入……

多年之后,我才回想起那些因世界之美及我内心所感之美而惊奇的时光。我那时明白,我曾感受到的柔美与美好,以及随后发现的那深刻改变我人生的力量,是同一事物的两个不可分割的面向。

如今我也看到,我们内在那多样的创造性,其柔和、沉静、无声的一面,自然通过惊奇表达出来。同样,在对所爱之人揭示的难以言喻的内在美的惊奇中,男子认识了他所爱的女子,而她也认识了他。当对所探索之物或所爱之人的惊奇缺席时,我们与世界的拥抱便失去了其中最美好的部分——失去了使之成为对自身及世界之福祉的那部分。没有惊奇的拥抱是无力的拥抱,仅是占有姿态的重复。它无力孕育除重复之外的任何东西,或许更大、更粗、更厚,无关紧要,却绝非更新\footnote{(34) 无力的拥抱

对我而言,“拥抱”绝非单纯的隐喻,此处日常语言反映了一种深刻的同一性。有人或会言之有理地说,若字面意义上如此,无惊奇的拥抱并非无力——否则大地早已人烟稀少乃至荒芜。极端情形如强暴,其中确无惊奇可言,却有时在被强暴的女子身上孕育出一个生命。无疑,从此类拥抱中诞生的孩子必带着其痕迹,这痕迹成为他所得“包裹”的一部分,由他承担;但这并不妨碍一个新生命的孕育与诞生,确有创造,彰显一种力量。同样,我也见过某些充满自负的数学家,寻得并证明了优美的定理,显示其与数学的拥抱并非无力!但若此数学家的生活被自负扼杀(正如我自己在某段时期在某种程度上所经历的),这些与数学拥抱的果实便既非对他自身的福祉,亦非对任何人的福祉。同样可言及强暴所生之子的父母。若我说“无力的拥抱”,我主要意指其无力在自以为创造之人心中孕育更新,他所创造的仅是外在于他的产物,与其内心无深层共鸣;这产物远非解放他或在他内心创造和谐,反而更紧地束缚他于其自负之中,他被其囚禁,不断被驱使生产与再生产。这是一种深层次的无力,隐藏在看似“创造力”的表象之下,而实则不过是无拘的产出。

我也多次察觉,自负与无惊奇的能力本质上是一种真正的盲目,是对自然敏感与直觉的阻塞;若非完全且永久的阻塞,至少在某些特定情境中显而易见。在这种状态下,某位声名显赫的数学家在其擅长的领域中,有时显得如最顽固的学生般愚钝!而在其他场合,他可能展现技艺的惊人 virtuosité。我却怀疑他是否还能发现那些简单而显明的事物,那些有能力更新一门学科或科学的发现。它们对他而言过于低下,他已不屑一顾!要看见无人愿见之物,需保有他已丧失或摒弃的纯真……近二十年来,数学产量的惊人增长,以及试图“稍稍跟上潮流”的数学家所面对的新结果的令人迷乱的泛滥,这并非偶然。然而,据我从各方听闻的回声判断,却鲜有真正的更新,鲜有大范围的变革(而非仅靠累积),在我所熟悉的任何重大反思主题上皆是如此。更新非数量之事,与投入量无关,无法以某水平数学家在某主题上投入的“数学家-日”数来衡量。一百万个数学家-日也无力催生如零般幼稚的事物,而零更新了我们对数的认知。唯有纯真拥有此力量,其可见标志便是惊奇……}(34)。当我们还是孩子, готов 惊叹于世界万物及自身之美时,我们也 готов 更新自身,并如柔顺的工具置于工匠之手,通过祂的手与我们,让存在与事物得以更新。

我清楚记得,在四十年代末及随后几年,那个对我而言代表数学界的无拘朋友群体,有时喧嚣而自信,语气略显武断并不罕见(但并未夹杂自负)——在那个环境中,惊奇随时有其空间。惊奇最显而易见者是迪厄多内「Dieudonné;Dieudonné」。无论他作报告,还是仅作为听众,当关键时刻到来,突然豁然开朗,你会看到迪厄多内如入天堂,容光焕发。那是纯粹的惊奇,具感染力,无法抗拒——其中“自我”的痕迹荡然无存。此刻回想,我意识到这种惊奇本身即是一种力量,在他周围即刻产生作用,如同他为其源头的光芒。若我见过哪位数学家运用了强大而基本的“鼓励之力”,那便是他!此前我从未再思及此,但现在我记得,他也是以这种姿态迎接我在南锡「Nancy;Nancy」的最初成果,解决了由他与施瓦茨「Schwartz;Schwartz」提出的关于 (\(F\)) 与 (\(LF\)) 空间的问题。那些成果颇为谦逊,绝非天才之作或非凡之举,可谓无甚可惊奇之处。此后我见过远更重要的成果被自视为伟大数学家的同事毫不留情地轻蔑拒绝。迪厄多内毫无此类自负的负担,无论正当与否。他毫无阻碍地为小事欣喜。

这种欣喜的能力中蕴含一种慷慨,对愿让其在内心绽放者及其周围皆为福祉。这种福祉无意取悦任何人,简单如花之香、如日之暖。

在我认识的所有数学家中,迪厄多内的这一“天赋”展现得最为耀眼、最具感染力,或许也最具作用力,我无法断言\footnote{(35) \par 此“天赋”非任何人独有,我们皆与之俱生。当它在我身上看似缺席,是我将其驱逐,而是否再次接纳,全在我一念之间。在我或某人身上,此“天赋”以不同方式表达,或不如他人那样具感染力或不可抗拒,但它依然存在,我无法说其作用是否更弱。}(35)。但在我喜爱的数学家朋友中,无一人缺乏此天赋。它或以更内敛的方式,在任何时刻找到显现的机会。每当我向他们分享刚发现且令我欣喜的事物时,它便显现出来。

若我在数学家生涯中经历过挫折与痛苦,首先是因在某些我曾珍爱之人身上,再未寻回我曾熟知的慷慨,以及对事物美感——无论“渺小”或“伟大”——的敏感;仿佛他们生命中那颤动的活力已熄灭无踪,被自负扼杀,他们眼中的世界不再美得值得欣喜。

当然,还有另一种痛苦,见昔日朋友以居高临下或轻蔑对待我今日之友。但归根结底,这痛苦源于同样的封闭。敞开于事物之美——哪怕最谦卑——之人,一旦感受到那美,便不由得对孕育或成就它的人心生敬意。在人手所成之物的美中,我们感受到创作者内在美的映射,感受到他倾注其中的爱。当我们感受到这美、这爱时,心中无处容纳居高临下或轻蔑,正如在感受女子之美及其美所彰显的力量的瞬间,无人能对她心生居高临下或轻蔑。