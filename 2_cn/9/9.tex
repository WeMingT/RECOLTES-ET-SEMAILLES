\section{(41) Krishnamurti, ou la libération devenue entrave}

Je me suis arrêté cette nuit sur un sentiment de soulagement, de grande satisfaction, le contentement de celui qui n'a pas perdu son temps! Je me sentais léger soudain, et joyeux - une joie un peu malicieuse par moments, fusant en rires espiègles - un rire de garnement blagueur. Pourtant je n'avais pas fait grand chose au fond, j'avais tout juste regardé un épisode déjà ``connu'', celui du fameux ``blanc-bec qui... '', sous un angle un peu différent. Un angle montrant ma relation à la mathématique elle-même, en certaines circonstances, non seulement ma relation à des mathématiciens. Il n'en a pas fallu plus pour qu'un mythe qui m'avait été cher parte en fumée.

A vrai dire, ce n'est pas la première fois que je regardais ma relation à la mathématique. Il y a deux ans et demi j'avais été conduit déjà à y consacrer quelques semaines ou mois. Je m'étais alors rendu compte (entre autres choses) de l'importance des forces égotiques, des forces d'autoagrandissement, dans mon investissement passé dans les maths. Mais la nuit dernière je venais de mettre le doigt sur un aspect qui m'avait alors échappé. Maintenant que je reviens là-dessus, je m'aperçois que cet aspect-là, l'aspect donc de l'attitude jalouse dans ma relation aux maths, rejoint la découverte ``toute bête'' qui était venue en dénouement de la première nuit où j'ai ``médité'' (méditant alors sans le savoir, comme Monsieur Jourdain faisait de la prose...). Il est bien possible que cela ait eu sa part dans cette exultation joyeuse qui a suivi. Même si ce n'était pas perçu consciemment, c'était un peu comme la reconfirmation, sous un jour nouveau, de quelque chose que j'avais découvert naguère - et le plaisir alors est le même qu'en mathématique, quand sans l'avoir cherché on tombe, par un biais entièrement différent, sur quelque chose qu'on connaît, qu'on a trouvé peut-être des années avant. A chaque fois cela s'accompagne d'un sentiment d'intime satisfaction, alors que se révèle une nouvelle fois l'harmonie des choses, et qu'en même temps se renouvelle peu ou prou la connaissance que nous en avons.

De plus, je crois que cette fois, j'ai bel et bien ``fait le tour''! Ça faisait des jours que je sentais bien qu'il restait encore quelque chose à tirer au jour, sans que j'aurais su dire très clairement quoi. Je n'ai pas essayé de forcer, je sentais qu'il n'y avait qu'à laisser venir, en laissant se dérouler librement le fil que je suivais, à travers des paysages à la fois familiers et imprévus. Imprévus, parce que je n'avais jamais pris la peine jusqu'à maintenant de les regarder. C'est au pas de promenade que je me suis approché du ``point chaud'' qui restait. Et je crois bien que c'est le dernier, dans le voyage que je viens de faire et qui touche à sa fin.

Et j'ai eu l'impression, sitôt arrivé à ce point, de celui qui arrive à un belvédère, d'où il voit se déployer le paysage qu'il vient de parcourir, dont à chaque moment il ne pouvait encore percevoir qu'une portion. Et il y a maintenant cette perception d'étendue et d'espace, qui est une libération.

Si j'essaye de formuler par des mots ce que me livre le paysage devant moi, il vient ceci : tout ce qui m'est venu, et souvent malvenu et mal accueilli, dans ma vie de mathématicien en ces dernières années, est récolte et message de ce que j'ai semé, aux temps où je faisais partie du monde des mathématiciens.

Bien sûr, cette chose-là, je me la suis dite et redite bien des fois au cours de ces années, et dans ces notes même que je viens d'écrire. Je me le suis dit, par analogie un peu avec d'autres récoltes qui me sont venues avec insistance, que j'ai longtemps récusées et que j'ai fini par accueillir et faire miennes. Dès la première que j'ai ainsi accueillie, avant même que je connaisse la méditation, j'ai compris que toute récolte devait avoir son sens, et que rechigner ne faisait qu'éluder un sens et reculer l'échéance d'un dénouement. Cette connaissance m'a été précieuse, car elle m'a gardé souvent de la pitié de moi, et de l'indignation vertueuse qui souvent en est une forme déguisée. Cette connaissance est en moi comme une demi-maturité, qui ne met nullement fin encore au réflexe invétéré de refuser les récoltes quand elles paraissent amères. Quand je me dis ``rien ne sert à rechigner'', la récolte n'est pas accueillie pour autant. Je ne me prends pas en pitié ni ne m'indigne peut-être, et pourtant je ``rechigne''! Tant que le plat n'est pas mangé, il n'est pas accueilli - et ne pas manger; c'est rechigner.

D'accueillir et manger est un travail : une certaine énergie travaille, un travail se fait au grand jour ou dans l'ombre, quelque chose se transforme... Alors que rechigner est le gaspillage d'une énergie qui se disperse - à ``rechigner''! Et on ne peut faire l'économie du travail de manger, de digérer, d'assimiler. Le seul fait de passer à travers des événements, de ``faire'' ou ``acquérir'' une expérience, n'a rien de commun avec un travail. C'est simplement un matériau possible pour un travail qu'on est libre de faire, ou de ne pas faire. Depuis trente-six ans que j'ai rencontré le monde des mathématiciens, j'ai fait usage de cette liberté-là que j'ai, en éludant un travail, alors que le matériau, la substance à manger et à digérer augmentait d'année en année. Ce sentiment de libération joyeuse que j'éprouve depuis hier est le signe sûrement que le travail qui était devant moi, que je repoussais sans cesse en faveur d'autres travaux ou tâches ; vient enfin d'être fait. Il était temps en effet !

Il est trop tôt encore pour être assuré qu'il en est bien ainsi, qu'il ne reste pas quelque recoin obscur et tenace qui aurait échappé à mon attention, sur lequel il me faudra revenir. Mais il est vrai aussi que ce sentiment de libération ne trompe pas - chaque fois que je l'ai ressenti dans ma vie, j'ai pu constater par la suite qu'il était bien le signe d'une libération, en effet; de quelque chose de durable, d'acquis, fruit d'une compréhension, d'une connaissance qui est devenue une part de moi-même. Je suis libre, s'il me plaît, d'ignorer cette connaissance, l'enterrer où je veux et comme je veux. Mais il n'est au pouvoir de moi ni de personne de la détruire, pas plus qu'on ne peut détruire la maturité d'un fruit, le faire revenir à un état de verdeur qui n'est plus le sien.

C'est un grand soulagement de voir confirmé, une nouvelle fois, que je ne suis pas ``meilleur'' que les autres. Bien sûr, ça aussi, c'est une chose que je me répète assez souvent - mais répéter et voir n'est pas pareil, décidément! A défaut de l'innocence et de la mobilité de l'enfant, qui voit comme il respire, souvent pour voir l'évidence il faut un travail - et voilà, c'est fait, j'ai fini par voir celle-ci : je ne suis pas ``meilleur'' que tels collègues ou ex-élèves qui, il y a quelques jours encore, me ``coupaient le souffle''! Qu'on juge du poids dont me voilà débarrassé ! C'est peut-être gratifiant d'une certaine façon de se croire meilleur que les autres, mais c'est aussi très fatiguant. C'est un gaspillage d'énergie extraordinaire même - comme chaque fois qu'il s'agit de maintenir une fiction. On s'en rend rarement compte, mais il en faut déjà de l'énergie, rien que pour maintenir la fiction contre vents et marées, alors que l'évidence à chaque pas clame dans mes oreilles soigneusement bouchées que c'est du bidon, regarde donc idiot! C'est peut-être un travail parfois de voir, mais quand il est fait il est fait. Ça fait l'économie une bonne fois pour toutes de me promener comme ça en me bouchant à tout bout de champ yeux et oreilles, faut le faire ça aussi ! et de m'affliger comme d'un intolérable outrage chaque fois que quelque chose me tombe dessus que j'avais posé là par mégarde.

Ras-le-bol de ce manège ! Quand on a vu le manège, on en est déjà sorti. On a payé, d'accord, j'ai le droit d'y tourner à perpète, et même le devoir qu'à cela ne tienne, tout le monde me le dira : droit, devoir - à la tête du client. C'est très fatigant aussi tous ces droits qui sont des devoirs et tous ces devoirs qui sont des droits, qui me collent après quand je me prends pour meilleur que les autres. C'est normal après tout, quand on est meilleur, on encaisse discrètement(ça, c'est les ``droits'') et on ``paye'', on fait tout son devoir pour l'honneur de l'esprit humain et de la mathématique - c'est très beau c'est vrai, honneur, esprit, mathématique qui dit mieux, bravo ! bis ! C'est très beau, d'accord, mais c'est aussi très fatigant, ça finit par donner le torticolis. J'ai eu mon torticolis et maintenant ça suffit comme ça - je laisse la place aux autres pour se tenir raides.

C'est normal aussi (puisque je parlais d'élèves) que l'élève dépasse le maître. Je m'en étais offusqué, j'avais de l'énergie à gaspiller ! Fini tout ça !

Quel soulagement!

