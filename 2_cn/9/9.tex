\section{(41) 克里希那穆提,或自由变为桎梏}

昨夜,我停下来时,心中涌起一种解脱感,一种巨大的满足感,那是一个未曾虚度时光之人的欣慰!我突然感到轻盈而愉悦——那是一种带点顽皮的喜悦,时而迸发出淘气的笑声——一个顽童恶作剧般的笑声。然而,我其实并未做太多事情,不过是以一个略微不同的角度,重新审视了一个早已“熟悉”的片段,那个著名的“愣头青如何……”的故事。这角度揭示了我与数学「mathématique;mathematics」本身在某些情境下的关系,而不仅仅是我与数学家们的关系。仅此而已,一个我曾珍视的神话便烟消云散。

说实话,这并非我第一次审视自己与数学的关系。两年半前,我已为此投入了几周乃至数月的时间。当时我意识到(除其他方面外),自我膨胀的力量、自大倾向的力量,在我过去对数学的投入中占据了重要位置。但昨夜,我却触及了一个当时未曾察觉的层面。现在回想起来,我发现这个层面——即我在与数学关系中的嫉妒态度——与我第一次“冥想”那夜(当时尚不知自己在冥想,就像约尔丹先生「Monsieur Jourdain;Mr. Jourdain」不知不觉地在写散文……)所得的“极其简单”的发现不谋而合。这很可能在随后那欢欣雀跃的情绪中扮演了一定角色。即使未被有意识地感知,这就像以一种新的视角再次确认了往昔的发现——那种愉悦恰似数学中的体验:当你未曾刻意追寻,却以截然不同的路径偶然重逢早已熟知、或许多年前已发现的事物时,总伴随着一种内在的满足感。此时,事物和谐的美感再度显现,而我们对它的认知也多少得到了更新。

此外,我相信这一次,我确已“穷尽一切”!数日来,我隐约感到还有什么尚未揭示,却无法清晰言明。我并未强求,心中明白只需静待其来,放任自己跟随那条线索自由展开,穿越那些既熟悉又意外的风景。意外,是因为我从未在此之前费心去审视它们。我以漫步的节奏靠近了那个仍存的“热点”。我确信,这是我此番旅程——即将结束的旅程——中的最后一个热点。

抵达此处的那一刻,我仿佛置身于观景台,俯瞰刚刚穿越的风景——在此之前,每一刻我只能感知其中的片段。而现在,这种辽阔与空间的感知,带来了一种解放。

若试图以言语表达眼前风景带给我的启示,便是如此:近些年来,我在数学家生涯中所遭遇的一切——往往是不受欢迎、不被欣然接受的——皆是我曾播下的种子,在我尚属数学家世界一员时的收获与讯息。

当然,这一点,我在这些年里、在这些刚写下的笔记中,已对自己反复述说多次。我通过类比其他一再降临的收获——那些我曾长久拒绝、最终接纳并化为己有的收获——如此告诫自己。从我第一次接纳这样的收获时起,甚至在我知晓冥想之前,我便明白,每一份收获必有其意义,而抗拒只会回避意义、推迟结局的到来。这份认知对我弥足珍贵,它常使我免于自怜,以及那常以自怜伪装的义愤。这种认知在我心中如半熟的果实,尚未完全终止我拒绝苦涩收获的固有本能。当我说“抗拒无济于事”时,收获并未因此被接纳。我或许不再自怜或义愤填膺,然而我仍在“抗拒”!只要那盘菜未被吃下,它便未被接纳——不吃,便是抗拒。

接纳并吃下是一项工作:某种能量在运作,一项工作在明处或暗中进行,某物在转化……而抗拒则是能量的浪费——浪费在“抗拒”上!我们无法省略吃下、消化、吸收的工夫。仅仅经历事件、“获取”或“拥有”某种经验,与工作毫无共通之处。那不过是为一项工作——我们可以选择做或不做的工作——提供的可能素材。自从三十六年前我踏入数学家的世界,我便行使了这份自由,回避了一项工作,而可供吃下与消化的素材、实质,却逐年累积。我自昨日起感受到的这种欢愉的解放感,无疑是某种信号,表明我一直推迟、不断让位于其他工作或任务的那项工作,终于完成了。的确,恰逢其时!

现在断言一切已然如此、没有某个隐秘而顽强的角落逃过我的注意、需要我回头再探,或许为时尚早。但同样真实的是,这种解放感从不欺骗——在我生命中每次感受到它时,事后总能证实,它确实是某种解放的象征,是持久的、已获得的成果,是理解与认知的果实,成为我的一部分。我尽可随心所欲地忽视这认知,将其埋葬于任何我想埋葬之处、按任何我想的方式埋葬。但无论是谁,包括我自己,皆无权毁掉它,正如无人能毁掉一颗果实的成熟,使其退回不再属于它的青涩状态。

再次确认我并不“优于”他人,带来巨大的宽慰。当然,这也是我常对自己重复的话——但重复与看见,终究不同!若缺乏孩童那自然的纯真与灵动——他们如呼吸般看见——要看见显而易见之事,往往需要一番工夫。而现在,这工夫已完成,我终于看见:我并不“优于”那些同事或昔日学生——几天前,他们还令我“瞠目结舌”!可想而知,我卸下了多大的负担!觉得自己优于他人或许在某种意义上令人满足,但也极度疲惫。那是一种惊人的能量浪费——如同每次试图维持虚构之物时那样。我们极少察觉,但单单为了对抗风浪维持这虚构,已需耗费多少精力,而真相每一步都在我那被小心堵住的耳边高喊:那是假的,快看啊,傻瓜!看见或许需要工夫,但一旦完成,便一劳永逸地省去了我四处堵耳遮眼的劳顿——那也得费劲啊!——以及每次因自己不慎放置的东西砸到身上时,那种仿佛不堪忍受的屈辱感。

受够了这种把戏!一旦看穿了把戏,便已从中脱身。没错,我付了代价,我有权永远在其中打转,甚至有义务如此——管它呢,大家都会这么说:权利、义务——看人下菜碟罢了。那些既是义务的权利、又是权利的义务,也很累人,当我自认为优于他人时,它们便如影随形。毕竟很正常,若自认优越,就得不动声色地承受(这叫“权利”),还得“付出”,尽一切义务以捍卫人类精神与数学的荣誉——这很美,没错,荣誉、精神、数学,谁能说得更好? Bravo!再来一次!的确很美,但也很累,最终让人扭了脖子。我脖子扭够了,现在够了——让位给别人去硬撑吧。

学生超过老师也很正常(既然提到了学生)。我曾为此愤愤不平,真是浪费精力!现在都结束了!

多么大的解脱!