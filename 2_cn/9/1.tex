\section{(33) 笔记——或新伦理}

当然,一条职业道德规范唯有通过一种内在的态度——其灵魂所在——方能彰显意义。它无法凭空塑造出它所欲表达的尊重与公正之姿,充其量不过是在一个对此规范达成普遍共识的环境中,助力维系这种态度罢了。若内在态度缺席,即便此规范被口头高唱,它亦丧失一切意义与价值。任凭多么严谨、多么细致的诠释,也无法扭转这一事实。

近日,我一位昔日的朋友与同伴温和地向我解释道,在如今这时代,唉,由于众所周知的数学产出「production mathématique;mathematical production」如洪水般泛滥,“我们”无论情愿与否,都绝对有必要对撰写并提交发表的论文进行严格筛选,仅出版其中一小部分。他言辞间流露出真诚的遗憾,仿佛他自己也略为成了这无可逃避之宿命的受害者——那神情与他提及自己名列法国“六七人”之一时颇为相似,这几人决定着哪些文章得以发表,哪些被拒之门外。唉,世事如此,无奈至极!随着年岁渐长,我变得不那么健谈,仅默默聆听。此话题确有千言万语可述,但我深知徒劳无益。一两个月后,我方得知,这位同事几年前曾拒绝推荐一篇提交至《CR》「Comptes Rendus;Proceedings」的笔记发表,此笔记的作者及其主题(大约七八年前由我向他提议)皆令我深为挂怀。作者耗费两年光阴钻研此主题,诚然,它并非当下的流行之选(尽管在我看来,它始终切中时弊)。我认为他完成了一项卓越的工作(以第三阶段论文「thèse de 3e cycle;third-cycle thesis」的形式呈现)。我并非这位年轻研究者的“导师”,他天资聪颖,颇为出众(鉴于所受待遇,我不知他是否会继续在数学领域施展才华……),且他在全无与我联系的情况下独立完成此研究。然不可否认,他所发展主题的渊源显而易见;可怜的家伙,前路坎坷,却浑然不觉!这位同事倒也颇具礼节,至少这一点尚存,我对他亦不作更高期待,“真诚地感到遗憾,但您明白……”一个满怀热忱的初学者研究者两载辛劳,与一篇不过三页的《CR》笔记相较——这需耗费多少公共资金「deniers publics;public funds」?其中荒谬显而易见,两者间的巨大落差令人瞠目。诚然,若肯费心探究深层动机,这荒谬或可消解。唯有这位同事兼旧友能洞悉自身动机,正如我唯有自省方知己心。但无需深究,我也清楚,这绝非数学产出泛滥之故,亦非为了节省公共资金(抑或那虚构的《CR》“未知读者”「lecteur inconnu;unknown reader」的耐心)……

同一篇《CR》笔记项目此前已荣幸地提交给法国“六七人”中的另一位,他却将其退回作者的“导师”,理由是这些数学“并未令他感到有趣”(原文如此!)。(导师心生厌恶却小心翼翼,他自身处境颇为不稳,两度宁可忍气吞声,也不愿拂逆……)我曾有机会与这位同事兼前学生讨论此事,得知他确曾用心研读提交的笔记并深思其内容(想必勾起他诸多回忆……),且认为其中若干陈述本可呈现得更便于使用者理解。然他并未屈尊将宝贵时间浪费于向当事人提交意见:这位显赫人物的十五分钟,与一位默默无闻的年轻研究者两年的辛劳相较!数学确也“有趣”得足以令他抓住此契机,重探笔记所涉情境(这情境于他我二人心中,皆不免激起丰富的几何联想「associations géométriques;geometric associations」),吸收其描述,继而凭借自身学识与能力,轻易察觉其中拙劣或疏漏之处。他并未虚度时光:借一位初试锋芒的研究者两年勤勉之功,他对某数学情境的认知得以精进与丰厚;而这位大师,若仅论大略且不涉证明,想必数日便可完成此工。一旦汲取完毕,他便忆起自身身份——案已定夺,无名氏「Monsieur Personne;Mr. Nobody」两年的心血,不过垃圾一堆……

有些人在这风气吹拂时毫无所感——然时至今日,我仍为之屏息。这无疑是彼时所求效果之一(观其拒绝之精妙形式可知),却绝非唯一。在同次交谈中,这位旧友以谦抑的自豪之姿向我坦言,他仅在“陈述的结果令他惊叹,或他不知如何证明”时,方接受提交《CR》的笔记\footnote{(27) “年轻人的势利眼”,抑或纯洁的捍卫者

罗尼·布朗「Ronnie Brown;Ronnie Brown」曾转述J.H.C.怀特海德「J.H.C. Whitehead;J.H.C. Whitehead」(其学生)的一席话,论及“年轻人的势利眼,他们以为:一则定理因其证明琐碎「trivial;trivial」而琐碎”。我诸多旧友皆当深思此言。如今,这“势利眼”绝非年轻人专属,我知晓不少声名赫赫的数学家常习此态。我对此尤为敏感,因我在数学(乃至他处……)所创最佳成果,那些在我看来最具丰饶性的概念与结构,以及我经耐心与执着挖掘出的本质属性,皆被贴上“琐碎”之签。(若作者非已成名之人,此等成果今日几无望获《CR》接纳!)我毕生之数学抱负,或更确切言之,我的激情与欢愉,皆在于寻获显而易见之物,此亦是我于本书(包括此引章……)之唯一志向。往往,关键在于洞察未被察觉之问(无论答案为何,无论其是否已得),或提炼一陈述(即便仅为猜想「conjectural;conjectural」),以概括并涵括一未被看见或未被理解的情境;若获证明,证明是否琐碎无关紧要,此乃全然次要之事,甚至仓促而暂时的证明若被证伪,亦无大碍。怀特海德所言之势利眼,乃倦怠享乐者之态,唯在确信美酒价格不菲后方肯品味。近年,我屡受旧日激情驱使,献出我之最佳,卻见其被此傲慢拒之门外。我心伤痛犹存,喜悦化作失望——然我并未因此流落街头,幸而我亦非试图发表自家文章。

怀特海德所言之势利眼,乃权力之滥用与不诚,不仅是对事物之美的麻木或封闭。当此态由权势者施于受其摆布的研究者,彼可尽情吸纳并利用其思想,却以“显而易见”或“琐碎”故“无趣”为由,阻其发表。我并未虑及剽窃之极端情形,此于数学界当仍稀有。然于实践而言,受损研究者的处境无异,而促成此况的内在态度,于我观之并无太大不同。唯其更惬意,因伴有对他人的无限优越感,以及以数学无形纯洁「pureté de la mathématique;purity of mathematics」之不妥协捍卫者自居的良知与内心满足。}(27)。这或许是他鲜有发表之因。若以己之标准自衡,他当全然不发。(诚然,以其处境,他无此必要。)他通晓一切,要令他惊叹,或觅一可证之事而他不知如何证明,皆难如登天。(二十年间,此类事于我仅寥寥两三回,且近十余年未见!)他显然为其“质量”之准则自豪,此使其成为数学家职业中苛求至极之典范。我却从中见其对己之无可动摇的纵容,以及在微笑谦逊的表象下,对他人的肆意轻蔑。我亦见其从中获巨大满足。

此同事乃我于“新伦理”「nouvelle éthique;new ethics」代表中所遇最极端之例,然仍具典型性。此处,无论我所述事件,抑或为其辩护之信念宣言,皆存一种乌布式「ubuesque;ubuesque」的荒谬,以常识计之——其规模之巨,令这位头脑超群的旧友,及诸多地位稍逊的同事(彼等仅满足于不向他提交《CR》笔记)皆视而不见。欲见,须先观。当肯费心审视动机(首当其冲乃己之动机),荒谬便于光芒中尽显,同时因揭示其谦卑而显见的意义,而不再荒谬。

若近年我常感痛苦不堪,面对某些态度尤甚行为,想必因我隐约辨出其中似我昔日态度与行为的极致夸张,乃至怪诞或可憎,此等姿态经由旧生或友,重返于我。我心屡起旧习,欲谴责、欲对抗那明指的“恶”——然若偶屈于此,彼处此地,皆信念分裂。内心深处,我知斗争不过仍在事物表面滑行,不过逃避。我之角色非谴责,亦非“改善”我身处之世,或“改善”己身。我之天职乃学习,透过自身认知此世,透过此世自知。若我之人生能为己或他人带来些许福祉,唯赖我忠于此天职,与己和谐相处。是时当自警,以断绝我内旧制,此处彼欲推我为某事业(或曰某殞地之伦理)辩护,或欲说服(某取代之伦理的所谓“荒谬”性,或然),而非探求以发现与认知,或以描述为探求之途。书写前两三页时,无更明确目的,仅欲略谈今日常见态度如何取代昨日旧姿,我始终自惕,仿若随时准备以一笔划去所写,弃之筐中!然我将保留所书,虽不虚伪,却因我涉他人甚于涉己,而造一虚假之境。我深感书写中无所获,此必为我不安之源。果决之时已至,当回归更深之思,以自教,而非佯装教人或说服他人\footnote{(28) 书写前页时,我初陷于“倾囊而出”之欲与矜持或谨慎之虑间。故我流于大略,此必为我不安及“无所获”之感的主因。自记下此不安后,我两度重写此令我内心不满之页,更明涉己身,更深入探究。途中我确“有所获”,且信我亦同时触及一重要之物,超越个案与我身。}(28)。