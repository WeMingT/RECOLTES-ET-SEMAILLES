\section{(40) 竞技数学}

显然,对数学事物美的开放性在我身上从未完全消失,即便在六十年代至1970年期间,自负在我与数学及与其他数学家的关系中逐渐占据了越来越大的位置。如果没有对事物美的最起码的开放性,我就完全无法以数学家的身份``运作'',即便只是以最 скром的节奏——我怀疑是否有人能在数学中做出有用的工作,如果他内心对美的感知,哪怕只有一丝一毫,不再鲜活。我认为,数学家之间,或同一数学家不同工作之间的差异,与其说是所谓的``脑力'',不如说是这种开放性或敏感性的细腻程度、精妙程度在不同研究者之间或同一研究者的不同时刻有所不同。最深刻、最富有成果的工作,也正是那种见证了对隐藏之美最为敏锐的感知的工作\footnote{(36) \par 这种对美的精妙敏感性,在我看来,与我曾提及的一种品质密切相关,我称之为``要求''(对自己)或``严谨''(在最充分的意义上),我将其描述为``对我们内心某种精妙之物的关注'',一种对所探究事物理解品质的关注。对数学事物的这种理解品质,与对其特定``美''或多或少亲密、或多或少完美的感知密不可分。}(36)。

如果真是如此,那么这种敏感性一定在我身上保持了活力,至少在某些时刻,直到最后,因为正是在六十年代末\footnote{(8 août) 经过核查,我关于动机的思考起始于六十年代初,而非六十年代末。},我开始瞥见并稍稍揭示了我所发现的最隐秘、最神秘的数学事物——我称之为``动机''「motif;motif」的东西。这也是在我数学家生涯中对我最具吸引力的东西(如果不算最后几年的某些思考,那些思考与动机的现实密切相关)。毫无疑问,如果我的生活没有突然转向一个完全无法预料的方向,将我远远带离数学事物的宁静世界,我最终会追随这种强大吸引力的召唤,抛下那些迄今将我束缚的``任务''!

或许我可以说,在我工作房间的孤寂中,对美的感知直到1970年我的第一次``觉醒''时都保持不变,未真正受到常常标记我与同辈关系的自负的影响。某种``嗅觉''甚至随着岁月,在与数学事物的日常亲密接触中变得更加敏锐。我们对事物可能拥有的那种亲密认识,有时让我们能够超越当下所知,进一步深入知识——这种认识或成熟,以及作为其最明显标志的``嗅觉'',与对事物美与真的开放性密切相关。它促进并激发这种开放性,是所有先前开放时刻、所有``真理时刻''的总和与果实。

因此,我仍需探究的是,这种对美的自发敏感性在与某个同事的关系中得以显现的时刻,是否或多或少地受到了深刻干扰。

关于这一点,我的记忆并未凝结成一个具体而精确的事实,可以在此详述。记忆在这里仍只是一团迷雾,但它传递给我一种总体印象,我必须努力去界定。这是我内心某种态度留下的印象,这种态度最终似乎成了我的第二天性,每当我接收到与我能力范围内的数学信息时,它就会显现。坦白说,这种态度在某个相对无害的方面可能一直属于我,是我某种气质的一部分,我曾偶尔提及过。这是一种反射,即首先只愿意了解一个命题的表述,从不看其证明,先尝试将其置于我已知的框架中,看看在已知的基础上这个命题是否变得透明、显而易见。通常这会让我以或多或少深刻的方式重新表述命题,朝更大一般性或更精确的方向推进,往往两者兼具。只有当我无法用我的经验和意象``安置''这个命题时,我才准备好(有时几乎是违背意愿地!)去倾听(或阅读……)其来龙去脉,这些有时给出了事物的``原因'',或者至少提供了一个证明,无论我是否理解。

这是我approach数学的一个特点,我认为这让我区别于我参与布尔巴基「Bourbaki;Bourbaki」小组时的所有其他成员,这也让我几乎不可能像他们一样融入集体工作。这一特点肯定也在我的教学活动中构成了某种障碍,这种障碍一定被我所有的学生感受到,直到今天(随着年龄增长)它才有所软化。

我身上的这一特质无疑已指向某种开放性的缺陷。它只意味着一种部分的开放,仅准备迎接``恰逢其时''的东西,或者至少对其他一切极为抗拒。在选择我的数学投入,以及我愿意花在某些意外信息上的时间时,这种有意的\textbf{``部分封闭''}如今比以往任何时候都要强烈。如果我想追随最吸引我的召唤,而不至于再次将``我的生命奉献给吞噬''数学女神,这甚至是一种必要。

然而,那团``迷雾''带给我的不仅是这一特点,我几年前就已经意识到了这一点(迟知总比不知好!)。在某个时刻,这种反射成了一种荣誉点;如果我不能在比说这句话还短的时间内``掌握''这个命题(假设它对我来说还不完全熟悉),那真是见鬼了!如果命题的作者是一个名不见经传的人,还有一种额外的意味:难道我(毕竟被认为是在行的人!)袖子里还没装着这些东西吗?事实上,我常常确实已经掌握了,甚至更多,那时我的态度往往倾向于:“好吧,你可以回去收拾一下了——等你做得更好再回来吧!”

这正是我对待那个``踩进我地盘的年轻毛头小子''的态度。我甚至无法保证他所做之事中没有一些有趣的细节,是我的``秘密笔记''所未涵盖的——这毕竟是次要的\footnote{(8 août) 我后来意识到,这并非那么``次要'',它构成了从``竞技态度''到某种不诚实的开端的界线,而我可能已经越过了这条线……}。最终,这个插曲也照亮了我在此探究的问题:对数学事物美的开放性是否受到了深刻干扰。仿佛一旦我``完成''了某件事,其美感对我而言就消失了,剩下的只有要求认可和回报的虚荣。(尽管我并未屈尊花费时间去发表它——确实,太多了。)这是一种典型的占有态度,类似于一个男人,在认识一个女人后不再感受她的美,转而追逐其他百人,却不因他人认识她而感到痛苦。我在爱情生活中谴责这种态度,认为自己远高于这种虚荣,却小心翼翼地不去正视这一明显事实:这正是我对数学的态度!

我有一种感觉,这种粗糙的竞争倾向,这种所谓的``竞技''倾向,我在自己身上刚刚触及的这些,在“我的”数学环境中可能开始变得常见,正好与它们在我身上常见的时间相吻合。我很难准确指出它们出现的时间,或它们成为我们所呼吸的空气中亲密一部分的时间,或我的学生在与我接触时所呼吸的时间。我唯一能说的,是这应该发生在六十年代,或许从六十年代初,甚至五十年代末开始。(如果是这样,我所有的学生都得面对这个——对他们来说,要么接受,要么离开!)要定位这一点,我需要其他具体案例,而这些此刻完全从我的记忆中逃逸。

这种谦卑的现实显然与我对数学及对年轻研究者关系的崇高形象形成鲜明对比。我用来欺骗自己的粗糙伎俩带有功利主义的灵感:在这一形象中,我只保留与我学生(他们为我的声望增光,是其中最高贵的花朵!)以及特别杰出的年轻数学家的关系,我能识别他们的优点,与他们平等相待,就像对待我的学生一样,不等他们的头戴上桂冠(当然,这并未迟到——有嗅觉就是有嗅觉!)。至于那些既不是我学生、也不是我朋友的学生、也不是年轻天才的年轻人,我完全不关心我与他们的关系。他们不算数。

我认为,这种现实在与年轻研究者的个人接触中通常会软化、缓和,无论是在我的研讨会上遇到他们,还是他们通过信件联系我。那个``年轻毛头小子''的案例从这个角度看可能有些特殊,是个例外。我感觉,对于我刚提到的那些研究者,我可能有点像把他们视为``在我的庇护下'',这会唤醒我更仁慈的态度。在这种情况下,我想要出风头的欲望也能找到出口,通过对相关者发表评论并提出建议,或许以更广阔的视角或更深入的方式重塑他们的工作。在这种情况下,那个暂时扮演我学生角色的年轻研究者很可能也从中受益,并对我保持良好回忆。(任何这方面的反馈,无论是正面的还是负面的,我都很欢迎。)

我在这里主要考虑的是年轻研究者的案例,尽管这种``竞技''态度显然不仅限于我与他们的关系。但无疑,在与年轻研究者的关系中,一个知名数学家的心理和实际影响往往最为强烈,对他们未来的职业生涯影响也最为深远。