\section{(34) 淤泥与源泉}

在我看来,我大体上已经梳理清楚了我与各个年龄、各个阶层的其他数学家的关系,那是在我还是他们世界——数学家世界的一部分时的情况;与此同时,更重要的是,我通过自己的态度和行为,参与塑造了我如今在这个世界中所察觉到的某种精神,这种精神显然并非始于今日。在这一反思过程中,或者更确切地说,这一场旅途中,我曾四次遭遇一些情境,它们在我看来典型地反映了我个人身上某些态度与矛盾之处。在这些情境中,我对他人自然流露的善意与尊重倾向,有时会被自我的力量所扰乱,甚至完全被抹去,尤其是(至少在其中三个案例中)被一种自负所取代。这种自负主要依仗于一种所谓的优越感,这种优越感据称源自我某种脑力的强大,以及我对数学活动的过度投入。它在一种普遍共识中得到了确认与支撑,这种共识几乎毫无保留地推崇这种脑力与这种过度的投入。

在我所审视的最后一个情境中,即那个“踩在我地盘上的无礼年轻人”的案例,似乎对我的当前论述最为重要。前三个情境是我个人性格的典型反映,或者说是我在某个时期(当然也在某种背景下)某些面向的典型反映——但正如我曾多次提及并反复强调的,我并不认为这些情境在当时我所处的环境中具有普遍性。我也不认为它们在如今的法国数学界——比如说——是典型的。很可能,我与那个“不知疲倦的朋友”之间那种带有某种慢性迷失的关系,在当时如同今日一样,都是颇为罕见的。然而,我在面对那个“无礼年轻人”时的态度与行为,却典型地反映了当今数学界中日复一日发生的事情,无论你看向何处。在这个世界里,一个有影响力的数学家对一个默默无闻的年轻人表现出善意与尊重的态度,已成为极为罕见的例外,尤其是当这个年轻人不是他的学生(即便如此……),或者不是由一位地位相当的同事推荐的学生时。无疑,这正是我在1970年“觉醒”后的第二天便已察觉到的东西,那时沉默的 tongues 被解开了——但我当时听到的第一手证词对我而言仍显得遥远,因为它们既不直接涉及我本人,也不涉及我在那个环境中最为珍视的朋友。直到1976年左右,当我听到的回声或我亲眼目睹的事实涉及到某些朋友,甚至是已成为重要人物的前学生时,我才开始受到较深的触动;尤其是当那些遭到恶意对待的人是我熟知的人——往往是学生(当然是“1970年之后”的学生!)——他们的命运让我感同身受时,这种触动尤为强烈。在某些情况下,毫无疑问,缺乏善意,甚至是公开的轻蔑态度,至少被强化了——如果不是完全由这一事实引发的话——仅仅因为某个年轻研究者是我的学生,或者他敢于(未必是我的学生)去做我昔日的朋友和其他同事乐于称之为“格罗滕迪克式举动「Grothendieckeries;Grothendieckeries」”的事情……

那个“无礼年轻人”在70年代初再次给我写信,非常礼貌地(尽管他完全没有义务向我询问任何事情!)询问我是否介意他发表一个定理的证明,这个定理据说由我提出但从未发表过,而他找到了这个证明。我记得我以过去同样的不悦态度回复了他,似乎既未明确同意也未明确反对,只是暗示——在完全不了解他的证明(他当然愿意与我分享,但我忙于我的激进活动,根本不在乎!)的情况下——他的证明肯定无法为我的证明增添任何东西(然而,至少他的证明会被写成黑白分明的文字,公之于数学界,连同定理陈述本身!)。这清楚地表明,那所谓的“觉醒”仍是多么肤浅,对根深蒂固于自负和“功利主义”态度中的某些行为毫无影响——而与此同时,我很可能正在《生存与生活》中撰写犀利的文章,或在公开辩论中慷慨陈词,谴责这些态度……

这以一种非常具体的方式回答了我之前悬而未决的一个问题。不如在此坦然承认这一朴素的真相:在我身上,这种自负的态度绝非“一次解决永不复发”,我怀疑它们是否会在我死前被彻底克服。如果说有什么转变,那并不是因为虚荣的消失,而是因为一种对自己以及某些态度和行为真实本质的好奇(或重新觉醒的好奇)的出现。正是这种好奇让我对自身虚荣的表现变得多少有些敏感。这深刻地改变了我内在的某种动态,从而也改变了“虚荣”的效应;也就是说,这种力量常常驱使我掩盖或伪装我对现实的健全而敏锐的感知,以抬高自己、凌驾于他人之上,同时又假装并非如此。

或许有读者会感到困惑,正如我自己曾经困惑过,面对我作为数学家生活中虚荣的隐秘而顽强的存在(他或许也在自己的生活中偶尔瞥见这种虚荣)与我所谓的对数学的热爱或激情之间的明显矛盾(这种热爱或许也在他自身的数学体验中,或对某个他人或事物的体验中,激起了一丝共鸣)。如果他确实感到困惑,那么他内心已具备一切所需,去重新接触(正如我曾经做过的那样)事物的真实本质,这些真实是他能够亲手认知的,而不是像一只被困在无尽词语与概念笼中的松鼠般徒劳地旋转。

看到浑浊的水流,有人会说水与泥是同一回事吗?要认识不是泥的水,只需溯源而上,凝视并饮用。要认识不是水的泥,只需登上被阳光与风吹干的河岸,掰开一团粗糙的黏土,在手中细细揉碎。野心与虚荣或多或少能调节一个人在生活中对某种激情——如数学激情——的投入比例,若回报满足了它们,便可能让这种激情变得吞噬一切。然而,即便是最贪婪的野心,单凭自身也无力发现或认知哪怕最微小的事物,反而恰恰相反!在工作的时刻,当理解逐渐萌芽、成形、深入;当混乱中渐渐显现出秩序,或熟悉的事物突然呈现出陌生的面貌,继而令人不安,直至最终矛盾爆发,颠覆了看似永恒的认知图景——在这样的工作中,没有野心或虚荣的痕迹。那时引领一切的,是某种远超“自我”及其对无限膨胀(哪怕是对“知识”与“见识”的渴求)的贪婪的东西——无疑也远超我们个人乃至我们这个物种。

这就是源泉,它存在于我们每一个人之中。