\section{(4) 他人的无谬性与自我的轻蔑}

我先前谨慎地说“我的表达风格”变了,甚至暗示这不足为奇,真是个轻描淡写的说法:你懂的,十三年没写东西,怎能和从前一样,“表达风格”必然得变……差别在于,以前我像所有人一样“表达”(姑且这么说):先做工作,再反过来重做,小心抹去所有涂改痕迹。途中又有新涂改,有时比初稿还糟,甚至颠覆一切。于是再重来——有时三四次,直到完美无瑕。不但没有可疑角落或偷偷塞到家具下的尘屑(我从不喜欢角落里的尘屑,既然要扫就扫干净);更重要的是,读最终文本时,那种(如同所有科学文本)令人自得的印象是,作者(即区区在下)仿佛无谬性的化身。他总是无懈可击地命中“那些”恰当概念,继而“那些”恰当命题,如引擎润滑顺畅的低鸣,证明一个接一个“落地”,每每恰逢其时!

试想这对毫无戒心的读者有何影响:一个学习毕达哥拉斯定理或二次方程的高中生,或者我在所谓“高等”研究或教育机构(有心人自会意会!)的同事,费力解读某位声名显赫同行的某篇文章!这种体验在学生生涯,甚至研究者生涯中重复数百、数千次,伴随家庭与全球各类媒体的齐声附和,其效应可想而知。只要稍加留意,便能在自身与他人身上察觉:那是内在对自己无能的深信,与“那些知道”与“那些做到”者的能力与重要性形成鲜明对比。

这种内在信念有时通过发展记忆未解之物的能力,或某种操作技巧得以补偿,却从未消解或缓解:矩阵相乘,靠“正题”与“反题”拼凑一篇作文……总之是鹦鹉或训练有素的猴子的能力,如今比以往任何时候更受推崇,以令人垂涎的文凭认可,以舒适的职业回报。然而,即便那满身文凭、地位稳固、或许荣誉加身之人,在内心深处,也未被这些虚假的重要与“价值”标志蒙骗。即便那更罕见之人,将一切赌注押在真正天赋的发展上,在职业生涯中发挥全力并创造成果,他内心深处也未被声名的光环说服——那光环常是他用来掩饰自己与他人真相的道具。无论是他俩,还是最普通的差生,皆被同一未经审视的疑虑、同一或许永不敢正视的信念所困扰。

正是这疑虑,这未言明的内在信念,驱使他们不断超越自我,累积荣誉或作品,并将那暗中噬咬的自轻自贱投射于他人(尤其是他们能掌控之人……),试图通过堆砌“优于他人”的证据,徒劳逃离这心结 \footnote{(2)\par (1984年3月增补) 重读最后两段,我有些许不安,因写作时我涉及他人而非自身。显然,写下这些时,我未曾想过自己也可能牵涉其中。我显然毫无长进,仅将多年来在他人身上察觉并多次验证的事物写下(或许带着某种满足),止步于此。在后续反思中,我被引导忆起,生活中我并未缺少对他人的轻蔑态度。若我察觉到的他人轻蔑与自我轻蔑间的联系,在我身上缺席,未免奇怪;健全的理性(以及类似自我盲视的经验——我最终有所察觉)告诉我,绝非如此!然而目前,这仅是推论,其唯一用处或在于促使我亲眼审视真相,观察并探究(若真存在或曾存在)那仍属假设、深埋至今完全逃过我视线的自我轻蔑。的确,值得审视之事从不匮乏!此事突然显现为最关键之一,正因其隐藏如此之深…… [(1984年8月) 然而,参见关于此的反思,见注释“屠杀”「Le massacre;The Massacre」,第87号最后两段。].} (2)。