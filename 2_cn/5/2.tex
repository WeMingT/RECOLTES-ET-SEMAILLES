\section{(2) 错误与发现}

回到我数学工作本身的风格,或其"本质"或其"方法",它们现在和过去一样,是上帝亲自无言地教给我们每一个人的,天知道是什么时候,也许是在我们出生之前很久很久。我像他那样做。这也是每个人本能地去做的事情,只要好奇心驱使他去了解万物中的某样东西,一个从此被这种渴望、这种饥渴所投注的事物......

当我对某事物好奇时,无论是数学还是其他,我便向它提问。我提问,不去关心我的问题是否愚蠢或是否会显得愚蠢,也不必非要深思熟虑不可。常常,问题以断言的形式出现——一个实质上是探测试验的断言。我或多或少相信我的断言,这当然取决于我在理解正在研究的事物方面所处的阶段。通常,特别是在研究初期,断言完全是错误的——但要使自己相信它是错误的,必须首先提出它。通常,只需把它写下来,就会一目了然地发现它是错误的,而在写下来之前,模糊不清,如同一种不适,代替了这种明证。现在这使我们能够少了这一无知,带着一个也许不那么"离题"的问题-断言重新出发。更常见的是,按字面理解的断言被证明是错误的,但试图通过它表达的直觉是正确的,尽管仍然模糊。这种直觉将逐渐从起初同样无形的错误或不当想法的外壳中沉淀下来,它将逐渐从不理解(却渴望被理解)的朦胧中,从未知(却渴望被认知)的朦胧中走出来,采取属于它自己的形式,随着我对面前这些事物提出的问题变得更加精确或更加贴切,它的轮廓会越来越清晰锐利,越来越紧密地围绕着事物。

但有时通过这种方法,反复的探测试验会收敛到对情况的某种图景,从迷雾中浮现出足够鲜明的特征,使人开始相信这个图景确实表达了现实——然而事实并非如此,当这个图景被一个足以深刻扭曲它的重大错误所玷污时。从发现图景与某些明显事实之间的最初"偏离",或与我们同样信任的其他图景之间的偏离,到追查出这种错误想法的工作,有时是艰辛的——这项工作常常伴随着不断增长的紧张,随着我们接近矛盾的核心,矛盾从最初的模糊变得越来越尖锐——直到最终爆发的那一刻,错误被发现,某种事物的视角崩塌,带来了巨大的宽慰,如同一种解放。发现错误是所有发现工作中的关键时刻之一,是最具创造性的时刻之一,无论是数学工作还是自我发现的工作。这是我们对所探索事物的认识突然更新的时刻。

畏惧错误和畏惧真相是同一回事。害怕犯错的人无力发现。当我们害怕犯错时,我们内心的错误便如磐石般不可改变。因为在我们的恐惧中,我们紧抓着那些我们某天宣称为"真理"的东西,或从来就被呈现为真理的东西。当我们被驱使的不是对看到幻觉般安全感消失的恐惧,而是对知识的渴望时,错误就像痛苦或悲伤一样,穿越我们而不会凝固,它经过的痕迹是一种更新了的认识。