\section{(3) 不可言说的劳作}

真正的研究,其自发的路径几乎从不出现在那些旨在传递与沟通“所发现”实质的文本或话语中,这绝非偶然。文本与话语通常仅止于记录“结果”,以一种形式呈现,让凡夫俗子觉得它们仿佛是永恒镌刻在某种巨型图书馆花岗岩书板上的严苛不变法则,由全知之神口授给启迪的抄写学者及其同侪——那些撰写学术书籍与文章的人,那些从讲坛高处或研讨会的狭小圈子中传授知识的人。是否有哪怕一本教科书、一本供小学生、中学生、大学生乃至“我们的研究者”使用的教材,能让可怜的读者对研究为何物有一丝概念?——除非是那普遍接受的观念:研究就是那些学识渊博、通过无数考试甚至竞赛的大脑发达之人所为,比如巴斯德「Pasteur;Pasteur」、居里「Curie;Curie」、诺贝尔奖得主之类。而我们这些读者或听众,勉力吞咽这些伟人为人类福祉记录的“知识”,若努力些,至多能在年末通过考试,仅此而已……

有多少人,甚至包括那些为论文或文章苦恼的“研究者”,乃至最“博学”\footnote{我们中最负盛名者。}、最享有声誉之人,能单纯地看到“研究”无非是热情地询问事物——如孩子渴望知道自己或妹妹如何来到世上?寻找与发现,即提问与倾听,是世上最简单、最自发之事,无人独享其特权。这是我们自摇篮中便获的“天赋”,注定以无穷面貌,从一刻到下一刻、从一人到另一人,表达与绽放……

当你冒险说出这些,无论是自认愚钝的差生,还是确信自己博学超群的学者,回应皆是半尴尬半会意的微笑,仿佛你刚说了个有些过分的笑话,暴露了显而易见的幼稚;这话虽美,谁也不该贬低,这没错——但别太过分,差生就是差生,不是爱因斯坦「Einstein;Einstein」或毕加索「Picasso;Picasso」!

面对如此一致的共识,我若再坚持,未免失礼。冥顽不灵如我,又一次错失沉默的机会……

不,这绝非偶然,所有教导性或启发性书籍与各类教材,以完美的默契,将“知识”呈现为从天才大脑中整装而出、为我们福祉而记录之物。也不能说是恶意为之,即便在少数作者“深谙此道”、明知其文本暗示的图景与现实毫无相符之处时亦然。在这种情况下,论述或超越结果与方法的汇编,带有生气,活泼的洞见赋予其生命,有时甚至从作者传递给专注的读者。但一种似有巨大力量的默契共识,使文本中不留丝毫其孕育过程的痕迹,即便它以简练有力的语言表达了那工作的真正果实——对事物有时深刻的洞察。

实话说,在筹划撰写并发表这些“数学反思”「Réflexions Mathématiques;Mathematical Reflections」时,我亦隐约感到这股力量、这无声共识的重量。若试图探查这共识在我内心的隐秘形式,或更确切地说,由它触发的我对计划的抗拒形式,我脑海中立刻浮现“不得体”一词。这共识——不知何时在我内心根植(这是我首次费力将其从数周乃至数月来颇为执拗的低语中拉至光天化日,置于我视线之内)——对我说:“将发现工作的起伏、笨拙摸索乃至‘脏衣’公开展示于人,甚至公之于众,是不得体的。这只会浪费读者宝贵的时间。何况,这会增加无数页面,需排版、印刷——纸张昂贵,科学印刷何其浪费!真要如此展示无人感兴趣的东西,仿佛我的失误也值得注目,简直是自炫的机会。”更隐秘地,它还说:“发表这类真实进行的反思笔记是不得体的,如同在公共广场上做爱,或展示、甚至随意丢弃分娩劳作中沾血的床单……”

此处的禁忌,以性禁忌的阴险且强势形式呈现。直到写这引言时,我才开始窥见其惊人力量,以及这一事实本身的惊人意义——它见证了这力量:发现的真实路径,如此令人困惑的简单、孩童般的单纯,几乎无处显现;它被无声地抹去、忽视、否定。即便在相对无害的科学发现领域——谢天谢地,不是发现自己的小弟弟之类——这“发现”看似适合所有人接触,(人们或以为)无须隐藏……

若我想追随此处浮现的“线索”——这线索绝非纤细,而是粗壮有力——它定会带我远超那几百页同调代数-同伦代数的篇幅,终将完成并交付印刷。