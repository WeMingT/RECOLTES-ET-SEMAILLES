1983年6月

\section{(1) 孩童与上帝}

我现在正在撰写的数学笔记是十三年来我第一次准备公开发表的作品。读者不会对我在长时间沉默之后表达风格的改变感到惊讶。然而,这种表达方式的变化并不意味着工作风格或方法的改变\footnote{(1)\par(1984年3月补充) 说我的"风格"和"方法"没有改变无疑是言过其实,因为我在数学中的表达风格已经发生了深刻的变化。我在过去一年中投入到《追寻场论「La Poursuite des Champs;Pursuing Fields」》的大部分时间都花在了打字机上,敲打那些几乎按原样发表的思考(只是后来添加了一些相对简短的注释以便于阅读,如参考引用、错误更正等)。不再需要剪刀和胶水来费力地准备一份"最终"稿件(特别是一份不应透露任何导致它的思考过程的稿件)——这确实是"风格"和"方法"的改变!除非我们将数学工作本身与写作工作、结果呈现分离开来,但这是人为的,因为它不符合事物的现实,数学工作与写作是密不可分的。}(1),更不是我数学工作本质发生转变的标志。这种本质不仅保持不变——而且我已经确信,发现工作的本质在每个发现者身上都是相同的,它超越了由无限变化的环境制约和气质所创造的差异。

发现是孩童的特权。我所说的是幼小的孩童,那个尚未害怕犯错、害怕显得愚蠢、害怕不严肃、害怕与众不同的孩童。他也不害怕他所看到的事物有不符合他期望的坏味道,不符合它们应有的样子,或者更确切地说:不符合公认的它们应有的样子。他不知道那些成为我们呼吸空气一部分的无言而无缝的共识——那些所有被认为明智且众所周知的人的共识。天知道自远古以来,有多少被认为明智且众所周知的人!

我们的思想充斥着杂乱的"知识",恐惧与懒惰、渴望与禁忌的纠缠;随意获取的信息和一键式解释——一个封闭的空间,信息、渴望和恐惧在其中堆积,而海洋的风从不吹入其中。除了日常技能之外,这种"知识"的主要作用似乎是排除对这世界之物的活生生的感知与认识。它的效果主要是一种巨大的惯性,一种常常令人窒息的重量。

幼小的孩童发现世界如同呼吸一般——他呼吸的涨落使他在自己娇嫩的存在中接纳世界,也使他投射进接纳他的世界。成人也会发现,在那些罕见的时刻,当他忘记了自己的恐惧和知识,当他用睁大的眼睛、渴望知晓的眼睛,新鲜的眼睛——孩童的眼睛——看待事物或自己。

\begin{center}
    * \quad * \\
    *
\end{center}

上帝在发现世界的同时创造了世界,或者更确切地说,他永恒地创造着世界,随着他发现它——而他又随着创造它而发现它。他创造了世界,并且日复一日地创造着,一次又一次地重复数百万亿次,不间断地,摸索着、犯错数百万亿次并纠正方向,永不疲倦......每一次,在这种探测事物的游戏中,事物的回应("这一次不错",或者:"你这次完全搞砸了",或者"一切顺利,继续这样"),以及新的探测纠正或延续前一次探测,回应前一次回应......,在创造者与事物之间这种无限对话的每一个来回,发生在创造的每一瞬间和每一个地方,上帝学习,发现,他越来越亲密地了解事物,随着它们在他手中获得生命和形态并转变。

这就是发现和创造的过程,看来它从永恒就是这样(就我们所能知道的而言)。它一直如此,不需要人类迟来的登场,仅仅在一两百万年前,参与其中——伴随着我们所知的最近的不幸后果。

有时我们中的一个人发现了这个或那个事物。有时他在自己的生活中重新发现,带着惊奇,什么是发现。每个人都拥有发现这个广阔世界中吸引他的一切所需的一切,包括他内在的那种奇妙能力——世界上最简单、最明显的事物!(然而,这是很多人已经忘记的事物,就像我们忘记了歌唱,或者像孩子那样呼吸......)

每个人都可以重新发现什么是发现和创造,但没有人能发明它们。它们在我们之前就存在,并且是它们本来的样子。