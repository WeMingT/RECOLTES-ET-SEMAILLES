
\section{愿景——或十二个主题的和谐乐章}

或许可以说,“伟大思想”是这样一种视角:它不仅展现出新颖与丰饶,更在科学中引入一个崭新而宏大的主题,使之得以具象化。当我们不以权力与支配的工具,而以人类世代求知冒险的精神来理解科学时,科学无非是这种或广或狭、或丰或俭的和谐乐章——它随着时代更迭,在世代与世纪的长河中徐徐展开,凭借所有相继涌现的主题(仿佛自虚无中应召而来)的精妙对位,交织融合于其中。

在我揭示的诸多数学新观点中,经过沉淀,有十二个可称之为“伟大思想”\footnote{为满足数学读者之好奇,兹按时间顺序列出这十二个主导思想,即我著作中的“核心主题”:

\begin{enumerate}
    \item 拓扑张量积与核空间。
    \item “连续”与“离散”的对偶性(导出范畴,“六运算”)。
    \item Riemann-Roch-Grothendieck 瑜伽($K$理论,与相交理论的关系)。
    \item 概形。
    \item 拓扑斯。
    \item 平展上同调与$\ell$-进上同调。
    \item 动机与动机伽罗瓦群(Grothendieck的$\otimes$-范畴)。
    \item 晶体与晶体上同调,De Rham系数、Hodge系数等瑜伽。
    \item “拓扑代数”:$\infty$-叠、导出器;拓扑斯的上同调形式体系,作为新同伦代数的灵感来源。
    \item 适度拓扑。
    \item 远阿贝尔几何瑜伽,Teichmüller-伽罗瓦理论。
    \item 正多面体与各类正则构型的“概形”或“算术”观点。
\end{enumerate}

除第一个主题(其重要部分构成我1953年博士论文内容,并在1950至1955年泛函分析阶段得到发展)外,其余十一个主题均形成于我自1955年转向几何研究后的时期。

理解我的数学工作,就是要“感受”其中至少部分思想,以及这些思想所引出的宏大主题——它们构成了作品的脉络与灵魂。

由于事物本质,某些思想比其他思想“更宏大”(相应地,其他思想则“更渺小”)。换言之,在这些新主题中,有些涵盖范围更广,有些则更深邃地触及数学事物的奥秘核心\footnote{其中,我认为范畴最广的是拓扑斯,它提供了代数几何、拓扑与算术的综合理念。就目前已产生的成果广度而言,最突出的是概形主题(参见第20页脚注(*))。它为八项其他主题(即除第1、5、10项外的所有主题)提供了卓越框架,同时彻底革新了代数几何及其语言体系的核心概念。

与之相对,十二个主题中的首尾两项在我看来规模较为有限。然而对于最后一项——为正多面体与正则构型这一古老主题注入新视角——我怀疑即使数学家穷尽毕生精力也未必能完全发掘其内涵。至于首个主题(拓扑张量积),它更多是作为即用型工具而非后续发展的灵感源泉。尽管如此,直到近年我仍会零星收到相关研究进展,这些工作(在二三十年后)解决了我当年遗留的部分问题。

这十二个主题中最深刻的(在我看来)是动机理论,以及与之紧密关联的远阿贝尔几何与Galois-Teichmüller瑜伽。

就工具的完善程度与应用广度而言,过去二十年间在多个“前沿领域”研究中常规使用的“概形”与“平展及$\ell$-进上同调”部分最为显著。
对于一个消息灵通的数学家而言,我认为此刻几乎无需怀疑,诸如源于其的$\ell$-进上同调这样的概形工具,已成为本世纪少数重大成果之一,滋养并革新了我们这几代人的科学。

其中有三项(在我看来绝非微不足道)在我退出数学舞台后才出现,至今仍处于萌芽状态;"官方"意义上它们甚至不存在,因为没有正式发表的文献为它们颁发出生证明\footnote{唯一"半官方"提及这三个主题的文本是1984年1月为申请法国国家科研中心借调而撰写的《纲领草案》。该文本(在《引言三:罗盘与行囊》中亦有提及)原则上将收录于《沉思录》第四卷。}。

在我退出前就已出现的九个主题中,最后三个本处于蓬勃发展期,却因缺乏(我离开后)关爱之手照料这些"孤儿"的必要需求,至今仍停滞在幼年阶段,被遗弃在充满敌意的世界\footnote{这三个孤儿在我离开后不久便悄无声息地被埋葬,其中两个却在1981年和次年(因操作异常顺利)被大张旗鼓地重新发掘,且未提及原发掘者。}。

至于另外六个在我退出前二十年已完全成熟的主题,可以说(除一两点保留意见\footnote{"几乎"主要指格罗滕迪克对偶瑜伽(导出范畴与六运算)以及拓扑斯理论。这些将在《收获与播种》第二部分和第四部分(《葬礼(1)与(3)》)中结合其他内容详细讨论。})它们当时已进入公共知识库:尤其在几何学者群体中,如今"所有人"都像莫里哀笔下的茹尔丹先生不经意间使用散文般,终日不自觉地运用着它们。它们已成为呼吸的空气——当人们"从事几何"时,或进行稍带"几何"色彩的算术、代数或分析时。

我这十二大主题绝非彼此孤立。在我看来,它们构成精神与目标上的统一体,如同贯穿我所有"书面"与"非书面"作品中持续的低音。而撰写这些文字时,我似乎再次听见同样的音调——如同召唤!——在那三年"无偿"、狂热而孤独的工作岁月里,当我尚未关心世上是否还有其他数学家存在时,完全沉迷于那召唤我的事物……

这种统一性不仅源于同一工匠在其作品上留下的印记。这些主题通过无数精妙而显见的纽带相互联结,如同庞大对位法中清晰可辨又交织发展的各个声部——在和声的统摄下彼此赋予意义、动力与完整。每个局部主题似乎都诞生于这更宏大的和谐,并时刻从中重生,远胜过和声作为预先存在主题的"总和"或"结果"。坦白说,我无法抑制这种(或许荒谬的)感受……
) 从某种意义上说,正是这种尚未显现却必定早已"存在"的和谐,潜藏于未生之物的幽暗怀抱中——正是它依次唤起了那些唯有通过它才能获得完整意义的主题,也正是它在那些炽热孤独的青春岁月里,以低沉而迫切的声音召唤着我...

无论如何,我作品中的这十二个核心主题确实如同受到某种隐秘的宿命指引,共同汇入同一部交响乐——或者换种比喻,它们各自体现着不同的"视角",最终都指向同一个宏大的愿景。

这个愿景直到1957、58年间才逐渐从迷雾中浮现,显现出可辨识的轮廓——那是段激烈孕育的岁月\footnote{1957年,我提出了"Riemann-Roch"主题(格罗滕迪克版本),一夜之间成为"巨星"。这也是母亲去世的年份,成为我生命的重要分水岭。这是我创造力最旺盛的年份之一,不仅限于数学层面。十二年来我倾注全部精力于数学工作,那年却突然感到已"穷尽"数学工作的本质,或许是时候转向其他领域了。那是我生命中首次涌现内在革新的渴望,甚至考虑转行写作,数月未碰数学。最终决定至少把进行中的数学工作结稿,原以为只需数月,最多一年...

或许时机尚未成熟。当我重拾数学工作时,反而是它再次俘获了我,此后十二年再未放手。

<EMPTYLINE
随后的1958年可能是我数学生涯最丰产的年份。新几何学的两大核心主题在此年破土:概型理论强势启航(我在爱丁堡国际数学家大会的报告主题),以及"site"概念的诞生——这是关键性拓扑斯概念的临时技术版本。三十年后回望,可以说正是这一年,随着概型(代表传统"代数簇"概念的蜕变)和拓扑斯(代表更深刻的"空间"概念蜕变)这两大工具的诞生,新几何学的愿景真正成形。}。奇怪的是,这个愿景对我如此亲近、"不言自明",以至于直到一年前\footnote{1984年12月4日思考"Yin le Serviteur (2)-ou la générosité"笔记时(ReS III第637页),在子注释($\mathrm{n}^{\circ} 136_{1}$)中首次萌生为其命名的念头。},我竟从未想过为它命名。(尽管我向来热衷为发现的事物命名,视作把握它们的第一步...)事实上,我指不出某个特定时刻可以被视为这个愿景的诞生瞬间,即便回溯亦无从确认。
一种新视野是如此广阔,它的出现恐怕无法定位于某一特定时刻,而必须经过漫长岁月——若非数代人——逐渐渗透并占据那些观察与沉思者的心灵;仿佛新的眼睛需要在熟悉的眼睛背后艰辛地成形,最终将逐步取代后者。这视野同样过于浩瀚,以至于所谓"把握"它,就像抓住路边偶然闪现的第一个念头般徒劳。因此无怪乎,为如此浩瀚、如此贴近又如此弥散之物命名的念头,唯有在事后——当它完全成熟之时——才姗姗来迟。

坦白说,直到两年前,我与数学的关系仍局限于(除教学任务外)追逐某种冲动——它不断牵引我奔向那个持续吸引我的"未知"。我从未想过要停下脚步,哪怕片刻,去回望或许已形成的路径,更遑论定位一段完成的历程。(无论是将其置于我的生命中,视为那些长期被忽视的深刻联系依然维系着我的存在;还是置于"数学"这场集体冒险之中。)

更奇特的是,最终促使我"驻足"并重新认识这部半被遗忘的作品,或仅仅考虑为孕育其灵魂的视野命名,竟需要我突然面对一场规模空前的葬礼现实——通过沉默与嘲弄埋葬的,既是那个视野,也是孕育它的工匠……