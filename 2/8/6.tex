\section{(30) Le Père ennemi (2)}

Ce n'est pas le grand tournant de 1970 qui a créé des antagonismes entre certains ex-élèves et moi, sur l'arrière-fond d'un passé idyllique et sans nuages. Il a seulement rendu visible des antagonismes qui pouvaient difficilement s'exprimer dans le cadre plus conventionnel d'une relation patron-élève (ou ex-patron - ex-élève) typique. Je suspecte que de tels conflits ne doivent pas être rares dans le milieu scientifique, mais qu'ils s'expriment le plus souvent de façon plus détournée et moins reconnaissable que dans les relations dans lesquelles j'ai été impliqué.

En y repensant, je n'ai pas l'impression, finalement, que dans ces relations à mes élèves, j'aie tellement eu tendance à entrer dans un rôle paternel - et même, je n'arrive pas à accrocher un seul souvenir qui aille dans ce sens peu ou prou. Pour ce qui est de ma personne, il me semble que la quasi-totalité de l'énergie que j'investissais dans une relation à un élève était celle-là même que j'investissais aussi dans la mathématique, et dans la réalisation d'un vaste programme. Dans la première période, je ne vois qu'un seul cas où il y ait eu en moi un intérêt pour la personne d'un élève, dans la nature d'une affinité ou d'une sympathie, qui ait eu une force comparable (sinon égale) à celle de l'intérêt mathématique. Mais même dans ce cas-là, je n'ai pas l'impression que je sois entré vis-à-vis de lui dans un rôle paternel. Quant à l'ascendant que j'ai pu exercer sur sa personne ou sur celle d'autres élèves, à un niveau ou un autre, c'est le genre de choses à quoi je ne faisais nulle attention dans ma relation à mes élèves. (Même aujourd'hui encore, j'ai tendance à ne pas y être attentif, ni avec les élèves qui ont travaillé avec moi en ces dernières années, ni même avec d'autres personnes.) Bien sûr, dans tous ces cas, la relation entre l'élève et moi n'était nullement "symétrique", en ce sens que pendant le temps tout au moins de la relation maître-élève (et probablement même au-delà, le plus souvent), l'importance qu'un élève avait dans ma vie n'était pas comparable à celle que je devais prendre dans la sienne, ni les forces psychiques que la relation mettait en jeu dans ma personne et dans la sienne. Sauf dans les cinq ou six cas où ces forces se sont manifestées par des signes d'antagonisme clairement reconnus, je me rends compte que la nature des relations à moi de mes différents élèves puis ex-élèves, pendant plus de vingt ans d'activité enseignante, restent pour moi un mystère total ! Ce n'est d'ailleurs pas tellement mon boulot de sonder ces mystères-là, plutôt celui de chacun d'eux pour sa propre part. Mais tant qu'à prendre intérêt à sa propre personne, il peut y avoir des choses plus brûlantes à regarder que les tenants et aboutissants de sa relation à son ex-patron... Quoi qu'il en soit, alors même que je ne manifestais aucune propension vis-à-vis de mes élèves à entrer dans un rôle paternel, il n'a pas dû être rare que j'aie néanmoins peu ou prou fait pour eux figure de père d'adoption, vu mon "profil" psychique particulier dont j'ai parlé précédemment, et vue aussi la dynamique inhérente à une situation où je ne pouvais manquer de faire figure d'aîné, à dire le moins.

En tout état de cause, dans plusieurs cas que j'ai évoqués, cette coloration particulière de la relation entre un élève et moi ne fait pas pour moi le moindre doute. En dehors de ma vie professionnelle il y a eu de nombreux autres cas encore où, avec ou sans connivence de ma part, j'ai visiblement fait figure de père d'adoption vis-à-vis d'hommes ou de femmes plus jeunes, attirés par ma personne et liés à moi tout d'abord par une sympathie mutuelle, mais nullement par des liens de parenté. Quant à mes propres enfants, la fibre paternelle en moi vis-à-vis d'eux a été forte, et depuis leur plus jeune âge ils ont eu une place importante dans ma vie. Par une étrange ironie, il s'est trouvé pourtant qu'aucun de mes cinq enfants n'a accepté le fait de m'avoir pour père. Dans la vie des quatre d'entre eux que j'ai pu connaître de près, en ces dernières années surtout, cette division dans leur relation à moi est le reflet d'une division profonde en eux-mêmes; d'un refus notamment de tout cela en eux qui les apparente à moi, leur père... Mais ce n'est pas ici le lieu de sonder les racines de cette division, qui plongent aussi bien dans une enfance déchirée, que dans mon enfance et dans celle de mes parents; comme aussi dans l'enfance de la mère, et dans celle de ses parents. Ni le lieu ici d'en mesurer les effets, dans leur propre vie, ou dans celle de leurs enfants...
