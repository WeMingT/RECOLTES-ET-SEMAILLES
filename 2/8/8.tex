\section{(32) L'éthique du mathématicien}

Le cas que j'ai rapporté hier, maintenant que je viens enfin de prendre la peine de le noter noir sur blanc, m'apparaît d'une portée considérable, plus grande à certains égards que les autres trois cas (sans doute typiques également) rapportés précédemment, où des forces de fatuité ont perturbé profondément en moi une attitude naturelle de bienveillance et de respect. Cette fois, utilisant une position de pouvoir bien réel (alors que je faisais mine, comme tout le monde, d'ignorer ce pouvoir), j'en ai usé pour décourager un chercheur de bonne volonté, et refuser un travail qui méritait d'être publié. C'est ce qu'on appelle un abus de pouvoir. Il n'est pas moins flagrant, pour ne pas tomber sous le coup d'un article du code pénal. Il est heureux que la conjoncture en ce temps-là était moins dure qu'aujourd'hui, de sorte que ce chercheur a pu, sans trop de mal je crois, faire publier son travail avec l'appui de quelque collègue plus bienveillants que moi, et que sa carrière de mathématicien n'a pas été sérieusement perturbée, et encore moins cassée, par mon comportement abusif. J'en suis heureux après coup, sans vouloir pour autant en faire une "circonstance atténuante". Il est possible que dans une conjoncture plus dure, j'aurais fait plus attention - mais c'est là une simple supposition, qui n'a pas grand chose à faire ici. Je crois quand même pouvoir dire qu'il n'y avait pas en moi une malveillance secrète, un désir de nuire causé par l'irritation dont j'ai parlé. Je réagissais à cette irritation de façon "viscérale", sans la moindre velléité critique à mon propre égard, et encore moins sans la moindre velléité de regarder tant soit peu ce qui se passait en moi, ou ne serait-ce que la portée que ma réaction pouvait avoir dans la vie de l'autre. Je ne mesurais pas le pouvoir dont je disposais, et la pensée d'une responsabilité allant avec ce pouvoir (ne fût-ce que le pouvoir d'encourager ou de décourager) ne m'a jamais effleuré au cours de cette relation. C'était un cas-type de conduite irresponsable, comme on en rencontre à tous les coins de rue, dans le monde scientifique comme ailleurs.

Il est possible que ce seul cas de son genre dont j'aie gardé souvenir soit un cas extrême, parmi quelques autres semblables. Ce qui déclenche une attitude sans bienveillance est l'irritation d'une vanité, impatiente de voir "le premier venu" s'arroger le droit de marcher dans des chasses gardées et d'y prendre quelque menu gibier qui ne revient qu'aux maîtres de ces lieux... Cette irritation a des rationalisations toutes trouvées, qui ont plus noble allure, on s'en doute. Ce n'est pas ma modeste personne qui est en jeu mais non, mais l'amour de l'art et de la mathématique, ce jeune homme qui n'a pas même l'excuse d'être génial le genre pataud plutôt il va tout abîmer malheur à nous, si encore il faisait les choses mieux que je ne sais le faire, mais les beaux ordonnancements que j'avais prévus tous passé à l'as, faut être un peu sans gêne franchement...! En filigrane constant, il y a le Leitmotiv méritocratisant : il n'y a que les tout meilleurs (tels que moi) qui aient droit de cité chez moi, ou ceux qui se mettent sous la protection d'un de ceux-là ! (Quant au cas moins courant où c'est bel et bien un autre grand chef qui marche dans mes plates bandes, c'est une autre paire de manches - à chaque jour suffit sa peine !) Dans le cas d'espèce, il y a eu (je n'ai plus guère de doute à ce sujet) une autre force allant dans le même sens, entièrement inconsciente elle, qui avait déjà joué fortement dans ma relation à l'infatigable ami de mes débuts : un automatisme de rejet vis-à-vis d'un certain type de personne, ne correspondant pas aux canons de "virilité" que j'avais repris de ma mère. Mais cette circonstance, qui a sa signification et son intérêt pour une compréhension de moi-même, est relativement irrelevant pour mon propos actuel : celui de trouver en moi-même, dans des attitudes et comportements qui ont été miens aux temps où je faisais encore partie d'un certain milieu, les signes typiques d'une dégradation profonde que j'y constate aujourd'hui.

Si ce cas que je viens d'examiner m'apparaît d'une plus grande portée que les autres où j'ai manqué de bienveillance et de respect, c'est parce que c'est celui où se trouve enfreint une certaine éthique élémentaire dans le métier de mathématicien \footnote{(24)\par L'éthique dont je veux parler s'applique tout autant à tout autre milieu formé autour d'une activité de recherche, et où donc la possibilité de faire connaître ses résultats, et d'en recueillir le crédit ; est une question "de vie ou de mort" pour le statut social de tout membre, voire même de "survie" en tant que membre de ce milieu, avec toutes les conséquences que cela implique pour lui et sa famille.}(24). Dans le milieu où j'ai été accueilli dans mes débuts, le milieu Bourbaki donc et des proches de Bourbaki, cette éthique dont je veux parler restait généralement implicite, mais elle était néanmoins présente, vivante, objet (il me semble) d'un consensus intangible. Le seul qui me l'ait exprimée en termes clairs et nets, pour autant que je me rappelle, était Dieudonné, une des premières fois sans doute où j'ai été son hôte à Nancy. Il est possible qu'il y soit revenu en d'autres occasions encore. Visiblement il sentait que c'était une chose importante, et j'ai dû sentir alors l'importance qu'il y attachait, pour m'en être souvenu encore aujourd'hui, trente-cinq ans après. Par le seul fait de l'autorité morale du groupe de mes aînés, et de Dieudonné qui visiblement alors exprimait un consensus du groupe, j'ai dû faire mienne tacitement cette éthique, sans pourtant jamais lui avoir accordé un moment de réflexion, ni comprendre ce qui faisait son importance. A vrai dire, l'idée ne me serait pas même venue qu'il pourrait être utile que j'y accorde une réflexion, persuadé que j'étais depuis belle lurette que mes parents et ma propre personne représentions, chacun, une incarnation parfaite (ou peu s'en fallait) d'une attitude éthique, responsable et tout, et à toute épreuve \footnote{(25) Consensus déontologique - et contrôle de l'information \par En dehors de la conversation avec Dieudonné, je ne me rappelle pas d'une conversation dont j'aie été participant ou témoin, au cours de ma vie de mathématicien, où il ait été question de l'éthique du métier, des "règles du jeu" dans les relations entre membres de la profession. (J'excepte ici les discussions au sujet de la collaboration de scientifiques avec les appareils militaires, qui ont eu lieu aux débuts des années 70 autour du mouvement "Survivre et Vivre". Elles ne concernaient pas vraiment les relations des mathématiciens entre eux. Beaucoup de mes amis dans Survivre et Vivre, y compris Chevalley et Guedj, sentaient d'ailleurs que l'accent que je mettais à cette époque, surtout aux débuts, sur cette question à laquelle j'étais particulièrement sensibilisé, m'éloignait de réalités quotidiennes plus essentielles, du type justement de celles que j'examine dans la présente réflexion.) Il n'a jamais été question de ces choses entre un élève et moi. Le consensus tacite se bornait je crois à cette seule règle, de ne pas présenter comme siennes des idées d'autrui dont en a pu avoir connaissance. C'est là un consensus, me semble-t-il, qui a existé depuis l'antiquité et n'a été contesté dans aucun milieu scientifique jusqu'à aujourd'hui. Mais en l'absence de cette autre règle complémentaire, qui garantit à tout chercheur la possibilité de faire connaître ses idées et ses résultats, la première règle reste lettre morte. Dans le monde scientifique aujourd'hui, les hommes en position de prestige et de pouvoir détiennent un contrôle discrétionnaire de l'information scientifique. Ce contrôle n'est plus tempéré, dans le milieu que j'avais connu, par un consensus comme celui dont parlait Dieudonné, lequel peut-être n'a jamais existé en dehors du groupe restreint dont il se faisait le porte-parole. Le scientifique en position de pouvoir reçoit pratiquement toute l'information qu'il juge utile de recevoir (et souvent même au-delà), et il a pouvoir, pour une grande partie de cette information, d'en empêcher la publication tout en gardant le bénéfice de l'information et rejetée comme "sans intérêt", "plus ou moins bien connu", "trivial", etc... Je reviens sur cette situation dans la note (27).}(25).

Dieudonné ne m'a pas fait d'ailleurs de long discours - ce n'était pas plus son genre que celui d'aucun de ses amis dans Bourbaki. Il a dû m'en parler plutôt en passant, et comme une chose qui était censée aller de soi. Il insistait simplement sur une règle des plus simples, toute anodine en apparence, qui est celle-ci : toute personne qui trouve un résultat digne d'intérêt doit avoir le droit et la possibilité de le publier, à seule condition que ce résultat ne soit déjà l'objet d'une publication. Donc même si ce résultat était connu d'une ou plusieurs personnes, du moment que celles-ci n'ont pas pris la peine de le mettre noir sur blanc et de le publier, de façon à le mettre à la disposition de (hm !) la "communauté mathématique", toute autre personne (sous-entendu : y inclus le fameux "premier venu"!) qui trouve le résultat par ses propres moyens (sous-entendu : quels que soient ses moyens, ses points de vue et éclairages, et qu'ils semblent ou non "étriqués" aux gens censés plus dans le coup que lui...) doit avoir la possibilité de le publier, suivant ses propres moyens et éclairages. Je crois me rappeler que Dieudonné avait ajouté que si cette règle n'était pas respectée, cela ouvrait la porte aux pires abus - il est possible que c'est à cette occasion et par sa bouche que j'ai appris justement le cas historique de Gauss refusant le travail de Jacobi, sous prétexte que les idées de Jacobi lui étaient connues depuis longtemps.

Cette règle si simple était le correctif essentiel à l'attitude "méritocratique" qui existait en Dieudonné (et en d'autres membres de Bourbaki) tout comme en moi-même. Le respect de cette règle était garant d'une probité. Je suis heureux de pouvoir dire, par tout ce qui m'est parvenu jusqu'à aujourd'hui, que cette probité essentielle est restée intacte en chacun des membres du groupe Bourbaki initial \footnote{(26) \par Les "membres fondateurs" de Bourbaki sont Henri Cartan, Claude Chevalley, Jean Delsarte, Jean Dieudonné. André Weil. Ils sont tous en vie, à l'exception de Delsarte emporté avant l'âge dans les années cinquante, à un moment donc où l'éthique du métier restait encore généralement respectée. 

En relisant le texte, j'ai eu la tentation de supprimer ce passage, dans lequel je peux donner l'impression de décerner des certificats de "probité" (ou de non probité) dont les intéressés n'ont que faire, et qu'il ne m'incombe pas de faire. La réserve que ce passage peut susciter est sûrement justifiée. Je le conserve pourtant, par souci d'authenticité du témoignage, et parce que ce passage restitue bel et bien mes sentiments, même si ceux-ci sont déplacés.}(26). Je constate qu'il n'aura pas été ainsi pour d'autres mathématiciens qui ont fait partie du groupe ou du milieu Bourbaki. Elle n'est pas restée intacte dans ma propre personne.

L'éthique dont me parlait Dieudonné en termes tout ce qu' il y a de terre à terre, est morte en tant qu'éthique d' un certain milieu. Ou plutôt, ce milieu ïui-même est mort en même temps que cette probité qui en faisait l'âme. Cette probité s' est conservée en certaines personnes isolées, et elle est réapparue ou r éapparaîtra dans certaines autres où elle s'était dégradée. Son apparition ou sa disparition dans tel d' entre nous fait partie des épisodes cruciaux de l' aventure spirituelle de l' un et de l' autre. Mais la scène sur laquelle se déroule cette aventure est profondément transformée. Un milieu qui m' avait accueilli, que j' avais fait mien, dont j'étais secrètement fier, n' est plus. Ce qui faisait son prix est mort en moi-même, ou du moins s' est vu envahi et supplanté par des forces d' une au tre nature, bien avant que l'éthique tacite qui le réglait se trouve ouvertement reniée dans les usages comme dans les professions de foi. Si j' ai pu depuis m'étonner et m' offusquer,c'était par ignorance délibérée. Ce qui m' est revenu de ce milieu qui fut mien avait un message à m' apporter sur moi-même, qu' il m' a p lu d'éluder jusqu'à aujourd' hui.

