\section{(31) Le pouvoir de décourager}

Pour terminer ce tour sommaire à travers les relations que j'ai eues dans le milieu mathématique entre 1948 et 1970, il me reste à parler de mes relations aux mathématiciens plus jeunes, plus ou moins débutants et par suite sans statut de "collègue" à proprement parler, sans pour autant que je joue vis-à-vis d'eux le rôle de "patron". Il s'agit donc de jeunes chercheurs que je rencontrais pendant une année ou deux dans mon séminaire à l'IHES, ou à l'occasion de tels cours ou séminaire à Harvard ou ailleurs, ou aussi parfois, à l'occasion d'une correspondance, par exemple quand j'avais reçu un travail d'un jeune auteur pour lequel celui-ci attendait des commentaires, et sûrement aussi un encouragement.

Les relations aux chercheurs débutants font partie d'un rôle moins apparent que celui de "patron" de tels élèves, mais tout aussi important, comme je m'en suis aperçu depuis. A cette époque, je ne me rendais pas compte, comme je le fais depuis six ou sept ans, que ce rôle-là, pour un mathématicien en vue, représente un pouvoir considérable. C'est tout d'abord le pouvoir d'encourager, de stimuler, qui existe aussi bien dans le cas du travail visiblement brillant (mais peut-être desservi par des maladresses de présentation ou une insuffisance de "métier"), que dans celui d'un travail simplement solide; elle existe même dans le cas d'un travail qui ne représente qu'une contribution très modeste, voire négligeable ou même nulle suivant les critères d'un aîné en pleine possession de moyens puissants, d'une expérience éprouvée du sujet, et d'une information étendue. Le pouvoir d'encourager est présent, pour peu que le travail qui nous est soumis ait été écrit avec sérieux - chose généralement discernable dès les premières pages.

Et le pouvoir de décourager existe tout autant, et peut s'exercer à discrétion quel que soit le travail. C'est le pouvoir dont Cauchy a usé vis-à-vis de Galois, et Gauss vis-à-vis de Jacobi - ce n'est pas d'hier qu'il existe et que des hommes éminents et craints en font usage! Si l'histoire nous a rapporté ces deux cas-là, c'est parce que les hommes qui en avaient fait les frais avaient une foi et une assurance suffisantes pour continuer leur voie, en dépit de l'autorité sans bienveillance de ceux qui faisaient alors la pluie et le beau temps dans le monde mathématique. Jacobi a trouvé un journal pour publier ses idées, et Galois les feuilles de sa dernière lettre, faisant office de "journal".

De nos jours, pour un mathématicien inconnu ou peu connu, il est assurément plus difficile qu'au siècle dernier de se faire connaître. Et le pouvoir du mathématicien en vue ne se situe pas seulement au niveau psychologique, mais au niveau pratique également. Il a le pouvoir d'accepter ou de refuser un travail, c'est-à-dire : donner ou refuser son appui pour une publication. A tort ou à raison, il me semble que "de mon temps", dans les années cinquante et soixante, le refus n'était pas sans appel - si le travail présentait des résultats "dignes d'intérêt", il avait une chance de trouver l'appui d'une autre éminence. Aujourd'hui, il n'en est plus ainsi assurément, alors qu'il est devenu difficile de trouver ne serait-ce qu'un seul mathématicien influent qui consente à parcourir (dans les dispositions qu'il lui plaira d'avoir) un travail dans sa partie, quand l'auteur n'a déjà acquis une notoriété, ou ne lui est recommandé par un collègue connu.

Il m'est arrivé, au cours des dernières années, de voir des mathématiciens influents et brillants faire usage de leur pouvoir de décourager et de refuser, aussi bien vis-à-vis de tel travail solide qui visiblement devait être fait, que vis-à-vis de tels travaux d'envergure dénotant clairement la puissance et l'originalité de leurs auteurs. Plusieurs fois, celui qui usait ainsi de son pouvoir discrétionnaire s'est trouvé être un de mes anciens élèves. C'est là sans doute l'expérience la plus amère qu'il m'a été donné de vivre dans ma vie de mathématicien.

Mais je m'éloigne de mon propos, qui était d'examiner de quelle façon, aux temps où je me prêtais avec conviction au rôle de "mathématicien en vue", j'usais du pouvoir d'encourager et de décourager dont je disposais. Je devrais ajouter qu'au niveau plus modeste où mon activité scientifique s'est poursuivie après 1970, en tant qu'enseignant parmi d'autres dans une université de province, ce pouvoir n'a pas cessé pour autant d'exister, tant vis-à-vis de mes étudiants ou élèves, que (rarement il est vrai) vis-à-vis de correspondants occasionnels. Mais pour mon propos présent, c'est la première période de ma vie de mathématicien qui seule importe.

Pour ce qui est de la relation à mes élèves, depuis le premier que j'ai eu jusqu'à aujourd'hui même, je crois pouvoir dire sans restriction d'aucune sorte que j'ai fait tout ce qui était en mon pouvoir pour les encourager dans le travail qu'ils avaient choisi \footnote{(23iv) Echec d'un enseignement (1) \par Depuis que ces lignes ont été écrites, j'ai eu l'occasion de parler avec deux de mes ex-élèves d'après 1970, pour essayer de sonder avec eux la raison de l'échec de mon enseignement au niveau de recherche, à l'Université de Montpellier. Ils m'ont dit que la propension que j'avais de sous-estimer là difficulté que pouvait représenter pour eux l'assimilation de telles techniques familières pour moi, mais non pour eux, avait eu sur eux un effet décourageant, car ils se sont sentis constamment en deçà de l'expectative que j'avais vis-à-vis d'eux. De plus (chose qui me semble d'une plus grande portée encore), ils est arrivé qu'ils se sentent frustrés, quand je leur "vendais la mèche" en leur donnant un énoncé en forme que j'avais dans mes manches, au lieu de leur laisser le plaisir de le découvrir par leurs propres moyens, à un moment où ils en étaient déjà tout proches. Après ça, il ne leur restait plus qu'à faire l' "exercice" (qui ne les passionnait pas autrement) de démontrer l'énoncé en question. C'est ici que se place le "manque de générosité" en moi que j'avais constaté dans une note antérieure (note 21), sans m'étendre plus à ce sujet. Ce sont de telles déconvenues, surtout, qui représentent ma contribution personnelle dans la disparition de l'intérêt pour la recherche chez l'un et l'autre, après des débuts pourtant excellents. 

Je me rends compte que je n'étais pas plus généreux avant 1970 qu'après. Si je n'ai pas eu les mêmes difficultés alors, c'est sans doute que le genre d'élèves qui venaient vers moi à cette époque étaient assez motivés pour trouver un charme même à un "long exercice", qui était occasion d'apprendre le métier et une foule de choses chemin faisant; et également, pour un énoncé de démarrage dont je "vendais la mèche", d'en dégager par leurs propres moyens une flopée d'autres qui allaient bien au-delà du premier. Quand j'ai changé de lieu d'activité enseignante, j'ai fait l'ajustement qui s'imposait dans le choix des thèmes de réflexion que je proposais à mes nouveaux élèves, par le choix d'objets mathématiques qui pouvaient être saisis par une intuition immédiate, indépendamment de tout bagage technique. Mais cet ajustement indispensable était par lui-même insuffisant, à cause de différences de dispositions (en mes nouveaux élèves par rapport à ceux d'antan), plus importantes encore qu'une seule différence de bagage. Cela rejoint d'ailleurs la constatation faite précédemment (début du par.25) sur une certaine insuffisance en moi pour le rôle de "maître", laquelle est ressortie de façon beaucoup plus forte dans ma deuxième période comme enseignant, que dans la première.}(23iv). Il doit être rare, même de nos jours, qu'il en soit autrement dans la relation de "patron" à élève, et tout particulièrement dans le cas d'un patron qui dispose des moyens pour pouvoir former des élèves brillants, et défricher avec leur concours des vastes étendues prêtes pour les labours. La chose à peine croyable, et vraie pourtant, c'est qu'il existe même ce cas extrême du patron prestigieux, prenant plaisir à éteindre en des élèves brillamment doués la passion mathématique qui l'avait lui-même animé en un plus jeune âge.

Mais à nouveau je digresse! C'est ma relation aux jeunes chercheurs qui n'étaient pas mes élèves qu'il s'agit maintenant d'examiner. Dans de telles relations, les forces égotiques dans la personne de l'homme en vue auraient moins tendance à le pousser dans le sens d'un encouragement, alors que les succès du jeune inconnu qui s'adresse à lui n'apporteront rien ou peu à sa propre gloire. Bien au contraire, je pense que le seul jeu des forces égotiques, en l'absence d'une véritable bienveillance, auraient tendance presque invariablement à pousser dans le sens opposé, à user du pouvoir de décourager, de refuser. C'est là, il me semble, ni plus ni moins que cette loi générale, qu'on peut constater dans tous les secteurs de la société : que le désir égotique de prouver sa propre importance, et le plaisir secret qui accompagne son assouvissement sont généralement plus forts et plus appréciés, quand le pouvoir dont on dispose trouve occasion à causer la déconvenue du prochain, voire son humiliation, plutôt que l'inverse. Cette loi s'exprime de façon particulièrement brutale dans certains contextes exceptionnels, comme celui de la guerre, ou l'univers concentrationnaire, celui des prisons ou des asiles psychiatriques, voire simplement celui des hôpitaux à tout venant dans un pays comme le nôtre... Mais même dans les contextes les plus quotidiens, chacun de nous a eu occasion d'être confronté à des attitudes et comportements qui attestent de cette loi. Les correctifs à ces attitudes sont tout d'abord des correctifs culturels, provenant d'un consensus, dans un milieu donné, sur ce qui est considéré comme comportement "normal" ou "acceptable"; ce sont d'autre part les forces de nature non égotique, comme la sympathie vis-à-vis d'une personne déterminée, ou parfois, une attitude de bienveillance spontanée indépendante même de la personne à qui elle s'adresse. Une telle bienveillance est sans doute chose rare, quel que soit le milieu où on la chercherait. Quant au correctif culturel en milieu mathématique, il me semble qu'il s'est considérablement érodé au cours des deux décennies écoulées. Il en est certainement ainsi, en tous cas, dans les milieux que j'ai connus.

Décidément je m'obstine à m'éloigner de mon propos, qui n'était pas un discours sur le siècle, mais une méditation sur moi-même et sur ma relation aux chercheurs plus ou moins débutants qui n'étaient pas mes élèves. Je ne crois pas que la "loi" à laquelle j'ai fait allusion ait trouvé occasion à s'exprimer dans ces relations. Pour des raisons qu'il n'y a pas lieu d'examiner ici, il semblerait que les forces égotiques, tout aussi fortes en moi qu'en quiconque, n'ont pas pris dans ma vie cette voie-là pour se manifester aux dépens d'autrui (à part quelques cas remontant à mon enfance). Je crois même pouvoir dire, ayant eu l'occasion d'examiner la chose, que la tonalité de base de mes dispositions vis-à-vis d'autrui est une tonalité de bienveillance, un désir donc d'aider quand je peux aider, de soulager quand je peux soulager, d'encourager quand je suis en mesure d'encourager. Même dans une relation aussi profondément divisée que vis-à-vis de cet "ami infatigable" dont j'ai eu à parler, jamais la fatuité en moi ne m'a égaré au point que j'aurais songé (fût-ce par intention inconsciente) à lui nuire. (J'aurais eu la possibilité de le faire, et "avec la meilleure conscience du monde" bien sûr.) Et je crois que dans la plupart des cas ces dispositions de bienveillance générale (fussent-elles mêmes un peu à fleur de peau seulement) ont marqué aussi mes relations dans le monde mathématique, y compris avec les mathématiciens débutants qui, sans être parmi les élèves, pouvaient avoir besoin de mon appui ou de mon encouragement.

Je crois qu'il en a été ainsi sans exception tout au moins au cours des années cinquante, et jusque dans les débuts des années soixante. Il me semble qu'en ces temps-là tout au moins, cette bienveillance n'était pas limitée à des jeunes visiblement brillants comme Heisuke Hironaka ou Mike Artin (alors qu'aucune renommée encore n'attestait de leurs moyens). Mais il est possible qu'elle se soit effacée dans une plus ou moins grande mesure au cours des années soixante, sous l'effet de forces égotiques. Je serais particulièrement reconnaissant pour tout témoignage qui me parviendrait à ce sujet.

Ma mémoire ne me restitue qu'un cas précis, dont je vais parler, et au-delà de ce cas, ce fameux "brouillard" qui ne se condense en aucun autre cas ou fait précis, mais plutôt qui me livre une certaine attitude intérieure. Je ressentais une certaine irritation quand il arrivait qu'un autre mathématicien "marchait sur mes plates-bandes" sans faire mine de rien me demander, comme s'il était chez lui le jeune blanc-bec ! Il devait s'agir surtout de cas de jeunes en effet, pas trop dans le coup, qui s'avisaient de retrouver, parfois dans des cas bien particuliers ma foi, des choses que je connaissais depuis des années et de haut encore. Ça n'a pas dû se produire très souvent, je crois, mais peut-être quand même deux, trois fois, peut-être quatre, je ne saurais trop dire. Comme je viens de dire, je ne me rappelle que d'un cas d'espèce, peut-être parce que la situation s'est reproduite avec le même jeune mathématicien à plusieurs reprises, sous une forme ou sous une autre. Je peux dire qu'à tous égards ce jeune chercheur, dont l'université d'attache était à l'étranger, a été d'une correction parfaite, en m'envoyant à moi, qui étais censé être la personne la plus dans le coup, le travail qu'il venait de faire. A chaque fois, j'ai réagi très fraîchement, pour la raison que j'ai dite. Je ne saurais même plus dire avec certitude si je lui disais franchement que ce qu'il faisait m'était connu depuis belle lurette, et que pour cette raison ça m'ennuyait qu'il le publie sans au moins me faire une petite courbette dans l'introduction. Bien sûr, s'il avait été mon élève, cette fatuité d'auteur n'aurait pas tellement joué, d'une part à cause d'une relation de sympathie qui était déjà établie avec l'élève, mais aussi parce qu'il allait de soi de toutes façons que le travail de l'élève contenait aussi des idées du patron, sauf mention du contraire ! Je crois que la situation a dû se produire deux, peut-être même trois fois, avec ce même chercheur, et qu'à chaque fois j'ai eu une attitude également fraîche, également décourageante. Je n'ai jamais accepté, si je me rappelle bien, de recommander un travail de ce chercheur pour être publié dans tel journal, ni de faire partie d'un jury de thèse (je crois me rappeler que la question s'était posée). C'est presque comme si j'avais décidé de le choisir comme tête de turc. Le plus beau, c'est que son travail à chaque fois était parfaitement valable - je crois qu'il était écrit avec soin, et je n'ai aucune raison de supposer qu'il n'ait pas trouvé lui-même les idées qu'il y développait, qui à ce moment ne couraient pas encore tellement les rues, et n'étaient (plus ou moins) "bien connues" que d'une poignée de gens dans le coup, comme Serre, Cartier, moi et un ou deux autres. Ce qui m'est incompréhensible, c'est que ce jeune collègue (il a fini bien sûr par avoir une thèse et un poste bien mérités) ne se soit pas lassé de s'adresser à moi qui "le battais froid" à chaque coup, et qu'il ne m'en ait apparemment jamais voulu. Je me rappelle quand même de la surprise qu'il m'a exprimé une fois devant ma réticence, visiblement il ne comprenait pas ce qui se passait. Il aurait eu du mal, s'il attendait mes explications! Il avait une belle tête, un peu à la grecque classique, très juvénile - des traits plutôt doux, paisibles, évoquant un calme intérieur... Maintenant que j'essaye pour la première fois de cerner l'impression que dégageait sa personne et sa physionomie, je me rends compte tout d'un coup qu'il ressemblait vraiment beaucoup à cet "ami infatigable" dont j'ai eu occasion de parler; ils auraient pu être frères, cet ami de mon âge dans la tonalité souriante, et ce chercheur, de vingt ans plus jeune, plutôt dans les tons un peu graves, mais nullement tristes. Il n'est pas impossible que cette ressemblance ait joué, que j'aie projeté sur l'un un dédain qui n'avait pas trouvé occasion de s'exprimer avec l'autre, désarmé qu'il était par les signes d'une amitié aussi fidèle! Et il fallait en effet que j'aie développé une carapace vraiment épaisse, pour ne pas être désarmé par la bonne foi évidente et la volonté de bien faire chez ce jeune homme attachant sûrement, qui ne se lassait pas de revenir à la charge, sans que je daigne le gratifier ne serait-ce que d'un sourire !



