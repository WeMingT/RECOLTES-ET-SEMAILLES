\section{(29) Le Père ennemi (1)}

Le genre d'élèves qui ont commencé à travailler avec moi après le tournant de 1970, dans le milieu complètement différent d'une université de province, a été très différent aussi des élèves d'avant. Il n'y a en a plus eu que deux qui ont travaillé avec moi au niveau d'une thèse de doctorat d'état. Le travail des autres s'est situé au niveau du DEA ou de thèses de doctorat de troisième cycle. Je devrais encore inclure un bon nombre d'étudiants qui ont accroché fortement à certains ``cours'' d'initiation à la recherche, lesquels ont été l'occasion pour eux de se poser des questions mathématiques souvent imprévues, et parfois d'imaginer des méthodes originales pour les résoudre. J'ai rencontré la participation la plus active dans certains ``cours d'option'' pour des étudiants de première année. Chez les étudiants qui contre tout défi subi l'ambiance universitaire pendant quelques années, une certaine fraîcheur, une capacité d'intérêt, de vision personnelle sont déjà plus ou moins éteintes. Parmi les étudiants des cours d'option, plusieurs avaient l'étoffe visiblement pour faire un excellent mathématicien. Vu la conjoncture, je me suis gardé d'en encourager aucun à se lancer dans cette voie-là, qui pourtant aurait pu les attirer et où ils auraient pu exceller.

Avec les étudiants qui suivaient tels de mes ``cours'' pour préparer des diplômes de maîtrise, les relations ne se sont pas poursuivies, le plus souvent, au-delà de l'année. A chaque fois, j'ai eu l'impression qu'elles sont vite devenues cordiales et décontractées, dans l'ensemble. Sauf chez un élève affligé d'un ``trac'' envahissant\footnote{(23”) La peur de jouer \par Cet élève avait travaillé avec moi sur un ``travail de stage'' de DEA pendant toute une année, et est resté ``contracté'' dans sa relation de travail avec moi jusqu'à la fin. C'était une relation franchement amicale, traversée par une sympathie mutuelle qui ne pouvait faire aucun doute. Il y avait pourtant ce ``trac'', cette peur, dont la cause véritable n'était vraiment pas une crainte devant ma personne, alors même que ça en prenait l'apparence. Je ne me serais peut-être pas même aperçu de la chose, si cet élève ne m'en avait parlé lui-même, sans doute pour ``expliquer'' peu ou prou la raison d'un blocage incessamment complet dans son travail au cours de l'année. \par Comme cela a eu lieu avec d'autres élèves qui, comme lui, ont bien accroché au début à une certaine substance géométrique, le blocage s'est manifesté dès le moment où il s'agissait de faire un ``travail sur pièces'', donc mettre noir sur blanc des énoncés en forme, ou seulement saisir le sens de la signification de ces énoncés et que je proposais d'admettre comme fondements d'un langage, comme ``règle du jeu''. Les réflexes ``scolaires'' poussent presque toujours l'élève confronté à une situation où il est censé ``faire de la recherche'', à adopter comme un ``donné'' à la fois un ou plusieurs ``angles du jeu'' implicites qui lui sont transmis par le Maître, et qu'il ne s'agit surtout pas d'essayer d'expliciter, et encore moins de comprendre. La forme concrète que prennent ces règles implicites sont les ``recettes'' de sémantique ou de calcul, sur le modèle des livres de taupe classiques (ou de tout autre livre d'enseignement courant). L'élève attend du Maître une liste de la forme ``démontrer que\ldots'', qui a été le seule forme de ``réflexion'' mathématique qu'il ait rencontrée dans son expérience passée. (Je ne crois pas d'ailleurs que les dispositions de la plupart des mathématiciens professionnels, et des autres scientifiques également, soient essentiellement différentes - à ceci près que le ``maître'' est remplacé par le ``consensus'' qui fixe les règles du jeu du moment et le considère comme un donné immuable. Ce consensus fixe également quels sont les ``problèmes'' qu'il s'agit de résoudre, entre lesquels chacun se sent latitude de choisir à son goût, se permettant même de les modifier au cours de son travail, voire même d'en inventer d'autres\ldots). J'ai rencontré que l'attitude entièrement différente qui est la mienne vis-à-vis d'une substance mathématique qu'il s'agit de sonder, et donc aussi vis-à-vis de l'élève, déclenche presque à coup sûr un désarroi, dont un des signes est l'angoisse. Comme toute angoisse, celle-ci aura tendance à se projeter sur un visage, à se projeter sur une ``raison'' extérieure plausible ou non. Un des visages les plus communs de l'angoisse est justement la peur. \par De telles difficultés ne se sont guère présentées dans la première période de mon activité enseignante, sauf peut-être dans les deux ou trois situations ``enseignant-élève'' ne s'est pas poursuivie au-delà de quelques semaines, et peut-être (je ne saurais dire) dans les cas de ``l'élève rusé'', uniquement préoccupé de ``faire bien'' ce qu'il ne l'intéressait nullement, alors qu'il avait pourtant toute latitude d'en changer. Dans le cas de l'élève dont j'ai parlé gentiment) qui s'est resté affligé d'un certain trac pendant longtemps, il est clair que la raison en est ailleurs. Il n'était nullement bloqué dans son travail, mais au contraire parfaitement à l'aise avec le thème qu'il avait choisi, sur lequel il a fait un travail de fondements très perspicace. La plupart de mes élèves de cette période étaient d'ailleurs des anciens élèves de l'Ecole Normale, et leurs contacts avec Henri Cartan leur avaient déjà montré l'exemple d'une autre ``approche'' des mathématiques. A l'extrémité opposée (pour ainsi dire) de ceux-ci, dans ma deuxième période comme enseignant, à l'Université de Montpellier, c'est chez les étudiants de première année que l'angoisse dont j'ai parlé a le moins interféré avec un travail de réflexion. Chez beaucoup de ces étudiants, l'étonnement devant une approche différente ne provoquait ni angoisse ni fermeture, mais au contraire ouverture et certain pour faire, pour une fois, des choses intéressantes! D'après mes observations, l'effet de quelques années de fac sur les dispositions de l'étudiant est radical et dévastateur. C'est une chose étrange qu'à cet égard l'effet des longues années de lycée semble relativement anodin. La raison en est peut-être que}(23''), il en a été de même avec les élèves qui étaient censés officiellement préparer un travail de recherche sous ma direction, à un niveau ou un autre. Une différence (parmi beaucoup d'autres!) avec mes élèves d'avant, c'est que notre relation ne s'est pas autant bornée à un travail mathématique commun. Souvent l'échange entre l'élève et moi a impliqué nos personnes de façon moins superficielle \footnote{(23v)\par Un signe particulièrement frappant de cette différence s'est manifesté à l'occasion de "l'épisode des étrangers", dont j'ai eu occasion de parler (section 24). Alors que j'ai reçu alors des témoignages de sympathie de la part de bien des personnes qui m'étaient entièrement étrangères, je ne me souviens pas qu'aucun de mes élèves d'avant 1970 ait songé à se manifester dans ce sens, et encore moins à me proposer une aide quelconque dans l'action dans laquelle je m'étais engagé. Par contre, il me semble qu'il n'y a aucun de mes élèves ou ex-élèves de la seconde période qui ne m'ait exprimé sa sympathie et sa solidarité, et plusieurs se sont associés activement à la campagne que je menais au niveau local. Au-delà de ce cercle restreint, l'affaire de l'ordonnance de 1945 a créé également une certaine émotion parmi de nombreux étudiants de la Faculté qui me connaissaient tout au plus de nom, et il en est venu un bon nombre au Palais de Justice le jour de ma citation, pour manifester leur solidarité. Cette dernière circonstance suggère d'ailleurs que la différence que j'ai constatée entre les attitudes de mes élèves "d'avant" et "d'après" 1970 exprime peut-être moins la différence des relations entre eux et moi, qu'une différence de mentalités. Visiblement, mes élèves "d'avant" étaient devenus des personnages importants, et il en faut beaucoup pour que les gens importants consentent à s'émouvoir... Mais l'épisode de mon départ de l'IHES en 1970 et de mon engagement dans une action militante semble montrer qu'il n'y a pas que cela. C'était là un moment où aucun d'eux ne faisait encore tellement figure de personnage important, et pourtant je ne me rappelle pas qu'aucun d'eux ait manifesté le moindre intérêt pour l'activité dans laquelle je m'engageais. Je pense plutôt que celle-ci a dû les mettre mal à l'aise, tous sans exception. Cela va bien encore dans le sens d'une différence de mentalité, mais qui ne peut être mise sur le compte de la seule différence de statut social.}(23v). Il n'est donc pas étonnant que dans cette deuxième période de mon activité enseignante, les éléments conflictuels dans la relation à certains élèves soient apparus de façon plus claire et plus directe, voire même véhémente. Parmi mes ex-élèves de la première période, il en est deux chez qui sont apparus par la suite des attitudes d'antagonisme systématique et sans équivoque (que j'ai eu l'occasion d'évoquer en passant), restées pourtant au niveau de l'informulé, et peut-être même de l'inconscient. Dans la deuxième période, plus longue, il y a eu trois élèves chez qui j'ai été confronté à un antagonisme. Chez deux d'entre eux, cela s'est manifesté de façon aiguë.

Chez un de ces élèves, l'antagonisme est apparu du jour au lendemain dans une relation qui avait été des plus amicales, de longues années après que cet ami ait cessé d'être mon élève. Je soupçonne que la cause du conflit n'était pas tant ma conduite et ma personnalité inqualifiables, qu'une insatisfaction longtemps refoulée de n'avoir trouvé pour son travail (qui avait été excellent) l'accueil qu'il aurait été en droit d'en attendre. C'était là le revers du douteux privilège de m'avoir eu comme patron "après 1970", et il devait m'en vouloir, sans trop se le reconnaître même en son for intérieur.

Chez l'autre élève, un antagonisme aigu est apparu déjà au bout d'une année et demi de travail, dans une ambiance qui avait semblé très cordiale. C'est la première et unique fois où une difficulté relationnelle entre un élève et moi soit apparue à un moment où il était encore en situation d'élève. Elle a rendu impossible la continuation d'un travail commun, qui s'était pourtant annoncé sous d'heureux auspices, avec un enthousiasme du meilleur augure, pour un thème de réflexion magnifique, il faut dire. J'ai eu le sentiment qu'il y avait en ce jeune chercheur un insidieux manque de confiance en son aptitude à faire du bon travail (aptitude qui pour moi ne faisait aucun doute), et que la manifestation au diapason aigu de l'antagonisme a été une sorte de "fuite en avant" pour prendre les devants sur un échec redouté, et en rejeter d'avance la responsabilité sur la personne d'un patron odieux \footnote{(23''')\par Les deux frères. L'antagonisme chez cet élève a pris la forme, d'emblée, d'un "antagonisme de classe" : j'étais le "patron" qui avait "pouvoir de vie et de mort" sur son avenir mathématique, dont je pouvais décider selon mon bon plaisir... Bien entendu, l'événement n'a pu que confirmer cette vision, puisque je n'ai pas tardé à mettre fin à mes responsabilités (devenues pénibles) vis-à-vis de cet élève. Cela l'a mis dans une situation délicate, par les temps qui courent où ce n'est pas si évident de trouver un "patron", surtout quand le sujet est déjà choisi. Chez l'autre élève, frustré dans ses légitimes expectatives, l'antagonisme a pris une forme analogue. J'étais ressenti comme le "mandarin" tyrannique, qui ne saurait tolérer de contradiction de la part de ceux (élèves ou collègues de moindre rang) qu'il considère comme ses subordonnés.

Une telle "attitude de classe" ne s'est jamais manifestée, si peu que ce soit, au cours de la relation à mes élèves de la première période. La raison évidente, c'est que dans la conjoncture d'avant 1970, il ne faisait aucun doute que l'élève, une fois sa thèse passée, aurait un poste de maître de conférences, et jouirait donc d'un statut social identique au mien, celui de "professeur d'université". Chiffres loquaces : les onze élèves qui ont commencé à travailler avec moi avant 1970 ont eu des postes de maîtres de conférences dès achèvement de leur travail, alors qu'aucun des quelque vingt élèves qui ont travaillé peu ou prou sous ma direction n'a eu accès à un tel poste. Il est vrai que deux seulement d'entre eux ont été assez motivés pour faire une thèse de doctorat d'état (d'ailleurs excellente pour l'un et pour l'autre).

Ce n'est donc pas chose étonnante si dans cette deuxième période, certaines ambivalences (dont l'origine profonde restait occultée) ont pris la forme d'un antagonisme de classe, de la méfiance (présentée et ressentie comme "viscérale") vis-à-vis du "patron". Pour un de ceux qui avaient peu ou prou fait figure d'élève, des relations amicales se sont poursuivies pendant une dizaine d'années sans épisode d'apparence antagoniste, et pourtant marquées par cette même ambiguïté, s'exprimant par une attitude de méfiance, tenue "en réserve" derrière une sympathie manifeste. Je n'ai à vrai dire jamais été dupe de cette "méfiance" de commande, qui m'est apparue surtout comme une raison que cet ami croit bon de se donner pour ne pas se hasarder hors du domaine bien délimité qu'il a choisi comme le sien, dans sa vie professionnelle comme dans sa vie tout court - chose qu'il est libre de faire pourtant sans que personne (sauf tout au plus lui-même !) lui demande des comptes...

Ces trois cas sont d'ailleurs les seuls, dans toute mon expérience d'enseignant, où une certaine ambivalence dans la relation entre un élève (ou quelqu'un qui peu ou prou fasse figure d'élève) et moi se soit exprimée par une "attitude de classe". Une telle attitude apparaît particulièrement ambiguë quand elle se manifeste entre collègues au sein d'un "corps" universitaire où ils jouissent l'un et l'autre de privilèges exorbitants en comparaison de la situation du commun des mortels, privilèges qui font apparaître les différences de rang (et de salaires) comme relativement insignifiants. J'ai remarqué d'ailleurs que ces attitudes disparaissent comme par enchantement (et pour cause !), dès que l'intéressé se voit promu lui-même à la situation dont la veille encore il faisait grief à autrui.

Je décèle d'ailleurs une ambiguïté similaire dans la plupart, sinon toutes, les situations de conflit dont j'ai pu être témoin à l'intérieur du monde mathématique (et souvent aussi en dehors). Ceux qui sont "casés", que leur rang corresponde ou non à leurs expectatives (justifiées ou non), jouissent de privilèges assez inouïs, qu'aucune autre profession ou carrière ne peut offrir. Ceux qui ne sont pas casés aspirent à la même sécurité et aux mêmes privilèges (ce qui ne les empêche pas nécessairement de s'intéresser aux maths elles-mêmes, et de faire parfois de belles choses). Par les temps qui courent où la concurrence est serrée pour se caser et où le non-casé est souvent traité en traîne-savates : j'ai plus d'une fois senti la connivence entre celui qui se plaît à humilier, et celui qui est humilié - et qui avale et s'écrase. Le véritable objet de son amertume et de son animosité n'est pas celui qui a fait usage d'un pouvoir, mais n'est nul autre que lui-même, qui s'est écrasé et qui a investi l'autre de ce pouvoir dont il use à plaisir. Celui qui se plaît à humilier est celui aussi qui prend sa revanche et compense (sans jamais l'effacer...) une longue humiliation subie et depuis longtemps enfouie et oubliée. Et celui qui acquiesce à sa propre humiliation est son frère et émule, qui secrètement l'envie et dans l'amertume enfouit et l'humiliation, et l'humble message sur lui-même qu'elle lui porte.}(23''').

Un aspect commun à toutes ces apparitions de conflit entre des élèves et moi, depuis bientôt vingt-cinq ans que j'enseigne le métier de mathématicien, est une forte ambivalence. Dans tous ces cas sans exception, l'antagonisme se manifeste après-coup, insidieusement souvent, dans une relation de sympathie qui, elle, ne peut faire l'objet d'aucun doute. Je puis même dire qu'en tous ces cas, comme en bien d'autres aussi où une composante franchement antagoniste ne s'est pas manifestée, ma personne a exercé et exerce encore une forte attirance. C'est sûrement la force même de cette attirance qui alimente aussi la force de l'antagonisme et assure sa continuité. Il en est encore ainsi, sûrement, dans les cas où l'antagonisme prend la forme d'une antipathie violente, d'un rejet outragé ; comme aussi dans tel autre cas, à l'extrême opposé, où sous le pavillon de rigueur d'un amical respect s'exprime (quand l'occasion est bonne) une affectation de dédain désinvolte et délicatement dosé...

De telles situations d'ambivalence, à vrai dire, ne sont pas particulières à ma relation à certains de mes élèves ou ex-élèves. En fait, elles ont abondé à travers toute ma vie d'adulte, depuis au moins l'âge de trente ans (c'est-à-dire depuis la mort de ma mère). Il en a été ainsi aussi bien dans ma vie sentimentale ou conjugale, que dans ma relation aux hommes et, plus précisément, à des hommes surtout qui sont nettement plus jeunes que moi. J'ai fini par comprendre que quelque chose en moi, d'inné ou acquis je ne saurais trop le dire, semble me prédisposer pour faire figure paternelle. J'ai, faut-il croire, la carrure idéale et les vibrations propices qui font le père d'adoption parfait ! Il faut dire que le rôle de Père me va comme un gant - comme s'il avait été mien de naissance. Je n'essaierai pas de compter le nombre de fois où je suis entré dans un tel rôle vis-à-vis d'une autre personne, dans un accord tacite parfait de part et d'autre. Le plus souvent cette distribution de rôles père-fils ou père-fille est resté dans le non-dit, voire dans l'inconscient, mais il est arrivé aussi qu'il soit formulé de façon plus ou moins claire. Dans certains cas aussi j'ai fait figure de père sans même être entré dans un jeu je crois, dans l'ignorance aussi bien au niveau conscient qu'inconscient de ce qui se tramait.

Je me suis aperçu pour la première fois d'un rôle de père d'adoption en 1972, à l'époque de "Survivre et Vivre", quand je me suis vu confronté soudain à une attitude de rejet violent chez un jeune ami. (Coïncidence intéressante, c'était un étudiant de maths en rupture de ban !) Quelque chose dans mon comportement vis-à-vis de tierces personnes l'avait déçu. J'aurais été prêt sans difficulté, je crois, à reconnaître que sa déception était fondée, que j'avais manqué en l'occurrence de générosité - mais la violence de la réaction m'avait alors littéralement soufflé. C'était comme une soudaine flambée de haine véhémente, qui est d'ailleurs retombée presque aussitôt, quand il était devenu clair qu'il n'avait pas réussi vraiment à me désarçonner. (Il s'en est fallu de peu, mais ça je l'ai gardé pour moi...). Je ne sais comment j'ai eu l'intuition alors qu'il projetait sur ma personne, dûment idéalisée, des conflits non résolus avec son père. Cette intuition subite, tombée dans l'oubli, n'a pas empêché que pendant des années encore, j'ai continué à entrer dans le rôle de père avec toujours la même conviction, sans me méfier le moins du monde. Avec bien sûr toujours le même étonnement douloureux, n'en croyant pas mes yeux ni le reste, quand par la suite je me voyais confronté aux signes de conflit, insidieux ou violents.

C'est après un travail solitaire intense de six ou sept mois sur la vie de mes parents, me faisant voir leur personne dans une lumière insoupçonnée, que j'ai compris ce qu'il y a d'illusoire dans ce rôle de parent d'adoption qui remplacerait (en mieux, c'est entendu d'avance !) un parent véritable qui existe bel et bien, et qui serait déclaré (ne fût-ce que par accord tacite) "défaillant". C’est aider autrui à éluder le conflit là où il se trouve, dans sa relation à son père disons, pour le projeter sur une tierce personne (moi-même en l'occurrence) qui y est entièrement étrangère. Depuis cette méditation, qui a eu lieu d'août 1979 à mars 1980, je suis vigilant vis-à-vis de moi-même, pour ne plus me laisser aller les yeux fermés à ma malencontreuse vocation paternelle. Cela n'a pas empêché que la situation fausse se reproduise (comme dans ma relation à cet élève avec qui j'ai dû cesser le travail) - mais maintenant, je crois, sans connivence de ma part.

Si je mets à part le cas de l'élève frustré dans ces légitimes expectatives, il ne fait aucun doute pour moi que dans tous les autres cas où j'ai été confronté à un antagonisme chez un élève ou ex-élève, ça a été la reproduction du même archétype du conflit au père : le Père à la fois admiré et craint, aimé et détesté - l'Homme qu'il s'agit d'affronter, de vaincre, de supplanter, d'humilier peut-être... mais Celui aussi que secrètement on voudrait être, Le dépouiller d'une force pour la faire sienne - un autre Soi-même, craint, haï et fui...



