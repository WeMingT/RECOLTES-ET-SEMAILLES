\section{(16) Marais et premiers rangs}

Mais je ne suis pas arrivé encore au bout de cette réflexion, sur la part qui a été mienne dans l'apparition du mépris et dans sa progression, dans ce monde auquel je continuais allègrement à référer par le nom de "communauté mathématique". C'est cette réflexion, je le sens maintenant, qui est-ce que j'ai de mieux à offrir à ceux que j'ai aimés dans ce monde, au moment où je m'apprête, non certes d'y retourner, mais à m'y exprimer à nouveau.

Il me reste surtout, je crois, à examiner quel genre de relations j'ai entretenu avec les uns et les autres qui faisaient partie de ce monde-là, aux temps où j'en faisais encore partie comme eux.

En y pensant maintenant, je suis frappé par ce fait qu'il y avait dans ce monde toute une partie que je côtoyais pourtant régulièrement, et qui échappait à mon attention comme si elle n'avait pas existé. Je devais la percevoir en ce temps comme une sorte de "marais" sans fonction bien définie dans mon esprit, pas même celle de "caisse de résonance" je suppose - comme une sorte de masse grise, anonyme, de ceux qui dans les séminaires et les colloques s'asseyaient invariablement aux derniers rangs, comme s'ils y avaient été assignés de naissance, ceux qui n'ouvraient jamais la bouche pendant un exposé pour hasarder une question, certains qu'ils devaient être d'avance sûrement que leur question ne pourrait être qu'à côté de la plaque. S'ils posaient une question aux gens comme moi, réputés "dans le coup", c'était dans les couloirs, quand il était visible que "les compétences" ne faisaient pas mine de vouloir parler entre eux - ils posaient leur question alors vite et comme sur la pointe des pieds, comme honteux d'abuser du temps précieux de gens importants comme nous. Parfois la question paraissait à côté de la plaque en effet et j'essayais alors (j'imagine) de dire en quelques mots pourquoi ; souvent aussi elle était pertinente et j'y répondais également de mon mieux, je crois. Dans les deux cas il était rare qu'une question posée dans de telles dispositions (ou, devrais-je dire plutôt, dans une telle ambiance) soit suivie d'une seconde question, qui l'aurait précisée ou approfondie. Peut-être nous, les gens des premiers rangs, étions en effet trop pressés dans ces cas-là (alors même que nous nous appliquions sûrement parfois à ne pas le paraître), pour que la crainte en face de nous puisse se dissiper, et pour permettre à un échange de naître. Je sentais bien entendu, tout comme mon interlocuteur de son côté, ce que la situation dans laquelle nous étions impliqués avait de faux, d'artificiel - sans que je me le sois alors jamais formulé, et sans que lui non plus, sans doute, ne se le soit jamais formulé. L'un et l'autre, nous fonctionnions comme d'étranges automates, et une étrange connivence nous liait : celle de faire semblant d'ignorer l'angoisse qui étreignait l'un de nous, obscurément perçue par l'autre - cette parcelle d'angoisse dans l'air vague d'angoisse qui saturait les lieux, que tous sûrement percevaient comme nous, et que tous choisissaient d'ignorer d'un commun accord\footnote{(13)\par Il est clair que la description qui précède n'a pas d'autre prétention que d'essayer de restituer tant bien que mal, par des mots maladroits, ce que me livre ce "brouillard" du souvenir, qui ne s'est condensé en aucun cas à l'époque tant soit peu précise, dont j'aurais pu ici donner une description tant soit peu "réaliste" ou "objective". Ce serait déformer mon propos que de faire dire à ce passage que les collègues qui répugnent à s'asseoir aux premiers rangs, ou qui n'ont pas statut de vedette ou d'éminence, soient nécessairement ivres d'angoisse en parlant à un de ces derniers. Ce n'était visiblement pas le cas pour la plupart des amis que j'ai connus dans ce milieu, même parmi ceux à qui il arrivait de hanter colloques ou séminaires. Ce qui est vrai sans aucune réserve, c'est que le statut d' "éminence" crée une barrière, un fossé vis-à-vis de ceux dépourvus de semblable statut, et qu'il est rare que ce fossé s'amenuise, ne fût-ce que l'espace d'une discussion. J'ajoute que la distinction subjective (qui me semble pourtant bien réelle) entre "premiers rangs" et "marais" ne peut nullement se réduire à des critères sociologiques (de position sociale, postes, titres, etc...) ni même de "statut", de renom, mais qu'elle relève aussi des particularités psychiques de tempérament ou de dispositions plus délicates à cerner. Quand j'ai débarqué à Paris à l'âge de vingt ans, je savais que j'étais un Mathématicien, que j'avais fait des maths, et malgré le dépassement dont j'ai eu l'occasion de parler, je me sentais au fond "un des leurs", tout en étant seul à le savoir, et sans même être sûr d'abord que je continuerais à faire des mathématiques. Aujourd'hui je serais plutôt porté à m'asseoir aux derniers rangs (en les rares occasions où la question se pose).}(13).

Cette perception confuse de l'angoisse n'est devenue consciente chez moi qu'aux lendemains du premier "réveil", en 1970, au moment où ce "marais" est sorti de la pénombre dans laquelle il m'avait plu jusque là de le maintenir en mon esprit. Sans que la chose se fasse par quelque décision délibérée, sans que j'en prenne conscience sur le champ, j'ai alors quitté un milieu pour entrer dans un autre - le milieu des gens "des premiers rangs" pour le "marais" - soudain, la plupart de mes nouveaux amis étaient de ceux justement qu'un an avant encore j'aurais facilement situés dans cette contrée sans nom et sans contours. Le soi-disant marais soudain s'animait et prenait vie par les visages d'amis liés à moi par une aventure commune - une autre aventure !
