\section{(18) Vingt ans de fatuité, ou : l'ami infatigable}

Ma relation à Terry n'a pas été dénaturée, à aucun moment je crois, par la différence de nos statuts dans le monde mathématique, ou par un sentiment de supériorité que j'en aurais retiré. Cette amitié, et une ou deux autres encore dont la vie m'a fait don en ces temps-là (sans se soucier si je le "méritais"!) était sûrement un des rares antidotes alors contre une fatuité secrète, alimentée par un statut social et, plus encore, par la conscience que j'avais prise de ma puissance mathématique et la valeur que moi-même lui accordais. Il n'en est pas allé de même dans ma relation avec le troisième ami. Celui-ci, et plus tard sa femme (dont il avait fait connaissance vers le moment où on s'était connus à Nancy) m'ont témoigné au cours de toutes ces années une amitié chaleureuse, empreinte de délicatesse et de simplicité, en toutes les occasions où nous nous sommes rencontrés, dans leur maison ou dans la mienne. Dans cette amitié il n'y avait visiblement aucune arrière-pensée, liée à un statut ou à des capacités cérébrales. Pourtant, ma relation à eux est restée empreinte pendant plus de vingt ans de cette ambiguïté profonde en moi, de cette division dont j'ai parlé, qui a marquée ma vie de mathématicien. En leur présence, chaque fois à nouveau, je ne pouvais m'empêcher de sentir leur amitié affectueuse et d'y répondre, presque à mon corps défendant ! En même temps, pendant plus de vingt ans j'ai réussi ce tour de force de regarder mon ami avec dédain, du haut de ma grandeur. Cela a dû s'enclencher ainsi dès les premières années à Nancy, et pendant longtemps aussi ma prévention s'est étendue à sa femme, comme s'il ne pouvait être qu'entendu d'avance que sa femme ne pourrait être qu'aussi "insignifiante" que lui. Entre ma mère et moi, nous affections de ne le désigner que par un sobriquet moqueur, qui a dû rester gravé en moi bien longtemps encore après la mort de ma mère, qui a eu lieu en 1957. Il m'apparaît maintenant qu'une des forces tout au moins derrière mon attitude était l'ascendant que la forte personnalité de ma mère a exercé sur moi pendant toute sa vie, et pendant près de vingt ans encore après sa mort, pendant lesquels j'ai continué à être imprégné des valeurs qui avaient dominé sa propre vie. Le naturel doux, affable, nullement combatif de mon ami était tacitement classé comme "insignifiance", et devenait l'objet d'un dédain railleur. Ce n'est que maintenant même, prenant la peine pour la première fois d'examiner ce qu'a été cette relation, que je découvre toute l'étendue de cet isolement forcené devant la sympathie chaleureuse d'autrui, qui l'a marquée pendant si longtemps. Mon ami Terry, pas plus combatif ni percutant que cet autre ami, avait eu l'heur, lui, d'être agréé par ma mère et n'a pas été l'objet de sa raillerie - et je soupçonne que c'est pourquoi ma relation à Terry a pu s'épanouir sans résistance intérieure en moi. Son investissement dans les mathématiques n'était pas plus fervent, ni ses "dons" plus prominents, sans que pour autant j'en tire prétexte pour me couper de lui et de sa femme par cette carapace de dédain et de suffisance !

Ce qui reste encore incompréhensible pour moi dans cette autre relation, c'est que l'amitié affectueuse de mon ami ne se soit jamais découragée devant la réticence qu'il ne pouvait manquer de sentir en moi, à chaque nouvelle rencontre. Pourtant, aujourd'hui je sais bien que j'étais autre chose aussi que cette carapace et ce dédain, autre chose qu'un muscle cérébral et une fatuité qui en tirait vanité. Comme en eux, il y avait l'enfant en moi - l'enfant que j'affectais d'ignorer, objet de dédain. Je m'étais coupé de lui, et pourtant il vivait quelque part en moi, sain et vigoureux comme en le jour de ma naissance. C'est à l'enfant sûrement qu'allait l'affection de mes amis, moins coupés que moi de leurs racines. Et c'est lui aussi, sûrement, qui y répondait en secret, à la sauvette, quand le Grand Chef avait le dos tourné.

