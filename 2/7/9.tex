\section{(24) Mes adieux, ou : les étrangers}

Cette rétrospective de ma vie de mathématicien prend un tout autre chemin que je n'avais prévu. A vrai dire, je ne songeais pas même à une rétrospective, mais seulement à dire en quelques lignes, voire en une page ou deux, quelle était aujourd'hui ma relation à ce monde que j'avais quitté, et peut-être aussi, inversement, quelle était la relation à moi de mes anciens amis, d'après les échos qui me parviennent de loin en loin. J'avais eu l'intention, par contre, d'examiner d'un peu plus près les vicissitudes parfois étranges de certaines des idées et notions que j'avais introduites en ces années de travail mathématique intense - je devrais dire plutôt : les nouveaux types d'objets et de structures que j'ai eu le privilège d'entrevoir et de tirer de la nuit de l'inconnu total vers la pénombre, et parfois même jusqu'à la plus claire lumière du jour! Ce propos maintenant semble détonner dans ce qui est devenu une méditation sur un passé, dans un effort pour mieux comprendre et assumer un certain présent, parfois déroutant. Décidément, la réflexion prévue sur une certaine "école" de géométrie, qui s'était formée sous mon impulsion, et qui s'est volatilisée sans (quasiment) laisser de traces, attendra une occasion plus propice \footnote{Cette "occasion plus propice" est apparue plus tôt que prévue, et la réflexion en question fait l'objet de la deuxième partie, "L'Enterrement", de Récoltes et Semailles.}. Dans l'immédiat donc, mon souci sera de mener à son terme cette rétrospective sur ma vie de mathématicien dans le monde des mathématiciens, non d'épiloguer sur une œuvre et le sort qui fut le sien.

Pendant les cinq jours qui viennent de s'écouler, accaparés par d'autres tâches que ces notes de réflexion, un souvenir m'est revenu avec une certaine insistance. Il me servira d'épilogue au De Profundis sur lequel je m'étais arrêté.

Ça se passe vers la fin de 1977. Quelques semaines auparavant, j'avais été cité au Tribunal Correctionnel de Montpellier pour le délit d'avoir "gratuitement hébergé et nourri un étranger en situation irrégulière" (c'est-à-dire, un étranger dont les papiers de séjour en France ne sont pas en règle). C'est à l'occasion de cette citation que j'apprenais l'existence de ce paragraphe incroyable de l'ordonnance de 1945 régissant le statut des étrangers en France, un paragraphe qui interdit à tout français de porter assistance sous quelque forme que ce soit à un étranger "en situation irrégulière". Cette loi, qui n'avait pas son analogue même en Allemagne hitlérienne à l'égard des juifs, n'avait apparemment jamais été appliquée dans son sens littéral. Par un "hasard" très étrange, j'ai eu l'honneur d'être pris comme le premier cobaye pour une première mise en vigueur de ce paragraphe unique en son genre.

Pendant quelques jours j'étais resté sidéré, comme frappé de paralysie, d'un découragement profond. Soudain je m'étais vu revenu de trente-cinq ans en arrière, aux temps où la vie ne pesait pas lourd, surtout celle des étrangers... Puis j'ai réagi, je me suis secoué. Pendant quelques mois j'ai investi la totalité de mon énergie pour essayer de mobiliser l'opinion publique, d'abord dans mon Université et dans Montpellier, et ensuite au niveau national. C'est à cette époque d'activité intense, pour une cause qui par la suite s'est avérée perdue d'avance, que se place l'épisode que je pourrais aujourd'hui appeler celui de mes adieux.

En vue d'une action sur le plan national, j'avais écrit à cinq "personnalités" du monde scientifique, particulièrement connues (dont un mathématicien), pour les mettre au courant de cette loi, qui aujourd'hui encore me paraît toujours aussi incroyable qu'au jour où je fus cité. Dans ma lettre je proposais une action commune pour manifester notre opposition à une loi scélérate, qui équivalait à mettre hors la loi des centaines de milliers d'étrangers résidant en France, et de désigner à la méfiance de la population, tels des lépreux, des millions d'autres étrangers, qui du coup devenaient des suspects, susceptibles d'attirer les pires ennuis aux français qui ne se tiendraient pas sur leurs gardes.

Chose étonnante, complètement inattendue pour moi, je n'ai reçu de réponse de la part d'aucune de ces cinq "personnalités". Décidément, j'avais des choses à apprendre...

C'est alors que je me suis décidé d'aller à Paris, à l'occasion du Séminaire Bourbaki où je ne manquerais pas de rencontrer de nombreux anciens amis, pour mobiliser tout d'abord l'opinion dans le milieu mathématique, qui m'était le plus familier. Ce milieu, il me semblait, serait particulièrement sensible à la cause des étrangers, alors que tous mes collègues mathématiciens, tout comme moi-même, ont à côtoyer quotidiennement des collègues, des élèves et des étudiants étrangers, dont la plupart sinon tous ont eu des moments de difficulté avec leurs papiers de séjour, et ont eu à affronter l'arbitraire et souvent le mépris dans les couloirs et les bureaux des préfectures de police. Laurent Schwartz, que j'avais mis au courant de mon projet, m'avait dit qu'on me laisserait la parole, à la fin des exposés du premier jour du Séminaire, pour soumettre la situation aux collègues présents.

C'est ainsi que j'ai débarqué ce jour-là, un volumineux paquet de tracts dans ma valisette, à l'intention de mes collègues. Alain Lascoux m'a secondé pour les distribuer dans le couloir de l'Institut Henri Poincaré, avant la première séance et à "l'entr'acte" entre les deux exposés. Si je me rappelle bien, il avait même fait un petit tract de son côté - il fait partie des quelques deux ou trois collègues qui, ayant eu écho de l'affaire, s'étaient émus et m'avaient contacté dès avant mon voyage à Paris, pour me proposer leur aide \footnote{(17)\par C'est surtout en dehors du milieu scientifique que j'ai rencontré des échos chaleureux à l'action dans laquelle je m'étais engagé, et une aide agissante. A part l'appui amical d'Alain Lascoux et de Roger Godement, il me faut encore noter ici surtout celui de Jean Dieudonné, qui s'est déplacé à Montpellier à l'audience en Correctionnelle, pour y ajouter son chaleureux témoignage à d'autres témoignages en faveur d'une cause perdue.}(17). Roger Godement fait partie aussi du nombre, il a même fait un tract qui titrait "Un Prix Nobel en Prison ?". C'était chic à lui, mais décidément on n'était pas branchés sur la même longueur d'onde : comme si le scandale était de s'en prendre à un "Prix Nobel", plutôt qu'au premier lampiste venu!

Il y avait foule en effet en ce premier jour de Séminaire Bourbaki, et énormément de gens que j'avais connus de plus ou moins près, y compris les amis et compagnons d'antan de Bourbaki ; je crois que la plupart devaient bien y être. Plusieurs de mes anciens élèves aussi. Ça devait bien faire dix ans bientôt que je n'avais pas vus tous ces gens, et j'étais content en venant de cette occasion de les revoir, même que ça en fasse beaucoup à la fois ! Mais on finirait bien par se retrouver en plus petit nombre...

Les retrouvailles pourtant "n'étaient pas ça", c'était assez clair dès le début. De nombreuses mains tendues et serrées, c'est sûr, et de nombreuses questions "tiens, toi ici, quel vent t'amène ?", oui - mais il y avait comme un air de gêne indéfinissable derrière les tons enjoués. Était-ce parce que la cause qui m'amenait ne les intéressait pas au fond, alors qu'ils étaient venus pour une certaine cérémonie mathématique tri-annuelle, qui demandait toute leur attention? Ou indépendamment de ce qui m'amenait, est-ce ma personne elle-même qui inspirait cette gêne-là, un peu comme la gêne qu'inspirerait un curé défroqué parmi des séminaristes bon teint? Je ne saurais le dire - peut-être y avait-il des deux. De mon côté, je ne pouvais m'empêcher de constater la transformation qui s'était opérée dans certains visages qui avaient été familiers, voire amis. Ils s'étaient figés, aurait-on dit, ou affaissés. Une mobilité que j'y avais connue semblait disparue, comme si elle n'avait jamais été. Je me trouvais comme devant des étrangers, comme si rien jamais ne m'avait lié à eux. Obscurément, je sentais que nous ne vivions pas dans le même monde. J'avais cru retrouver des frères en cette occasion exceptionnelle qui m'amenait, et je me trouvais devant des étrangers. Bien élevés, il faut le reconnaître, je ne me rappelle pas de commentaire aigre-doux, ni de tracts qui auraient traîné par terre. En fait, tous les tracts distribués (ou presque) ont dû être lus, la curiosité aidant.

Ce n'est pas pour autant que la loi scélérate s'est vue mise en péril ! J'ai eu mes cinq minutes, peut-être en ai-je pris même dix, pour parler de la situation de ceux qui pour moi étaient des frères, appelés "étrangers". Il y avait là un amphithéâtre bondé de collègues, plus silencieux que si j'avais fait un exposé mathématique. Peut-être la conviction pour leur parler déjà n'y était plus. Il n'y avait plus, comme jadis, courant de sympathie et d'intérêt. Il doit y avoir des gens pressés dans le nombre, j'ai dû me dire, j'ai écourté, proposant de nous retrouver sur le champ, avec les collègues qui se sentaient concernés, pour se concerter de façon plus circonstanciée sur ce qui pourrait être fait...

Quand la séance a été déclarée levée, ça a été une ruée générale vers les sorties - visiblement, tout le monde avait un train ou un métro sur le point de partir, qu'il ne fallait louper à aucun prix ! En l'espace d'une minute ou deux, l'amphithéâtre Hermite s'est retrouvé vide, cela tenait du prodige! On s'est retrouvé à trois dans le grand amphithéâtre désert, sous les lumières crues. Trois, y inclus Alain et moi. Je ne connaissais pas le troisième, un de ces inavouables étrangers encore je parie, en compagnie douteuse et en situation irrégulière par dessus le marché ! On n'a pas pris le temps d'épiloguer longuement sur la scène bien assez éloquente qui venait de se dérouler devant nous. Peut-être aussi étais-je le seul à ne pas en croire mes yeux, et mes deux amis ont eu la délicatesse alors de s'abstenir de commentaires à ce sujet. Visiblement, je débarquais...

La soirée s'est terminée chez Alain et son ex-épouse Jacqueline, à faire le point de la situation et passer en revue ce qui pourrait être fait ; à faire un peu plus connaissance, aussi. Ni ce jour, ni plus tard, je n'ai pris le loisir de situer par rapport à un passé l'épisode que je venais de vivre. C'est ce jour-là pourtant que j'ai dû comprendre sans paroles qu'un certain milieu, un certain monde que j'avais connu et aimé n'était plus, qu'une chaleur vivante que j'avais pensé retrouver s'était dissipée, depuis longtemps sans doute.

Ça n'a pas empêché que les échos qui me parvenaient encore, an par an, de ce monde-là dont la chaleur a fui, m'ont bien des fois déconcerté, touché douloureusement. Je doute que cette réflexion y change quelque chose pour l'avenir - si ce n'est, peut-être, que je me rebifferai moins d'être ainsi touché...







