\section{(17) Terry Mirkil}

À vrai dire, dès avant ce tournant crucial, j'avais été lié d'amitié avec des camarades (devenus "collègues" par la suite) que j'aurais sans doute situés dans le "marais", si la question s'était posée à moi (et si tel avaient été mes amis...). Il a fallu cette réflexion, et que je fouille mes souvenirs, pour me rappeler et pour que des souvenirs éparpillés s'assemblent. J'ai fait la connaissance de ces trois amis dans les tout premiers temps, quand j'apprenais le métier à Nancy comme eux - à un moment donc où nous étions encore dans le même panier, où l'on ne me désignait comme une "éminence". Ce n'est sûrement pas là un hasard, et qu'il n'y ait pas eu d'autres telles amitiés pendant les vingt ans qui ont suivi. Nous étions étrangers tous les quatre, c'était là sûrement un lien non négligeable - mes relations avec les jeunes "normaliens", parachutés à Nancy comme moi, étaient bien moins personnelles, on ne se voyait guère qu'à la Fac. Un de mes trois amis a émigré en Amérique du Sud un ou deux ans plus tard. Il était comme moi attiré par les recherches au CNRS, et j'avais comme une impression qu'il ne savait pas trop lui-même ce qu'il "cherchait", sa situation au CNRS devenait un peu scabreuse, à force. On a continué à se voir ou s'écrire de loin en loin, et on a fini par perdre contact. Ma relation aux deux autres amis a été de plus longue durée, et aussi plus forte, bien moins superficielle. Nos intérêts mathématiques n'y jouaient d'ailleurs qu'un rôle des plus effacés, voire nul.

Avec Terry Mirkil et sa femme Presocia, menue et fragile comme lui était râblé, avec un air de douceur dans l'un et dans l'autre, nous passions souvent à Nancy des soirées, et parfois des nuits, à chanter, à jouer du piano (c'était Terry qui jouait alors), à parler musique qui était leur passion, et de choses et d'autres importantes dans nos vies. Pas des plus importantes il est vrai - pas de celles qui toujours sont tues si soigneusement... Cette amitié m'a beaucoup apporté pourtant. Terry avait une finesse, un discernement qui me faisaient défaut, alors que la plus grande partie de mon énergie était déjà polarisée sur les mathématiques. Bien plus que moi, il avait gardé le sens des choses simples et essentielles - le soleil, la pluie, la terre, le vent, le chant, l'amitié...

Après que Terry ait trouvé un poste à son goût à Dartmouth Collège, pas tellement loin de Harvard où je faisais des séjours fréquents (à partir de la fin des années cinquante), on continuait à se rencontrer et à s'écrire. Entre-temps, j'ai su qu'il était sujet à des dépressions, qui lui valaient de longs séjours dans les "maisons de fous", comme il les a appelées dans la seule et laconique lettre où il m'en ait parlé, à la suite d'un de ces "séjours horribles". Quand on se rencontrait, il n'en était jamais question - sauf une ou deux fois très incidemment, pour répondre à mon étonnement que lui et Presocia n'adoptaient pas d'enfant. Je ne crois pas que l'idée me soit jamais venue que nous puissions parler du fond du problème, lui et moi, ou seulement l'effleurer - sans doute pas même celle qu'il y avait peut-être des problèmes à regarder, dans la vie de mon ami ou dans la mienne... Il y avait sur ces choses un tabou, inexprimé et infranchissable.

Progressivement, les rencontres et lettres se sont espacées. Il est vrai que je devenais de plus en plus le prisonnier de tâches et d'un rôle, et de cette volonté surtout, devenue comme une idée fixe, un échappatoire peut-être à autre chose, de me surpasser sans cesse dans l'accumulation des œuvres - alors que ma vie familiale se dégradait mystérieusement, inexorablement...

Quand j'ai appris un jour, par une lettre d'un collègue de Terry à Dartmouth, que mon ami s'était suicidé (ça a été longtemps après qu'il soit déjà mort et enterré...), cette nouvelle m'est venue comme à travers un brouillard, comme un écho d'un monde très lointain et que j'aurais quitté, Dieu sait quand. Un monde en moi, peut-être, qui était mort bien avant que Terry ne mette fin à sa vie, dévastée par la violence d'une angoisse qu'il n'avait pas su ou voulu résoudre, et que je n'avais pas su ou voulu deviner...