\section{(45) Le Guru-pas-Guru - ou le cheval à trois pattes}

Cela me ramène aussitôt à cette méditation qui s'était poursuivie de juillet à décembre 1981, après une période de quatre mois que je venais de passer dans une sorte de frénésie mathématique. Cette période un peu démentielle (très féconde d'ailleurs au point de vue maths \footnote{(39) \par C'est la période, entre autres, de la ``Longue Marche à travers la théorie de Galois'', dont il est question dans ``Esquisse d'un Programme'' (par. 3 : ``Corps de nombres associés à un dessin d'enfant'').}(39)) avait pris fin, du jour au lendemain, à la suite d'un rêve. C'était un rêve qui décrivait, par une parabole d'une force sauvage irrésistible, ce qui était en train de se passer dans ma vie - une parabole de cette frénésie. Le message était d'une clarté fulgurante, il m'a fallu pourtant deux jours d'un travail intense pour accepter son sens évident \footnote{(40) La visiblement

Le travail sur ce rêve est l'objet d'une longue lettre en anglais, à un ami et collègue qui avait passé chez moi en coup de vent la veille. Certains des matériaux utilisés par le Rêveur, pour faire surgir d'un apparent néant ce rêve d'un réalisme saisissant, étaient visiblement empruntés à ce court épisode de la visite d'un ami cher que je n'avais plus revu depuis près de dix ans. Aussi, le premier jour de travail et à l'encontre de mon expérience passée, j'ai cru pouvoir en conclure que le rêve qui m'était venu concernait mon ami, plus qu'il ne me concernait - que c'est lui qui aurait dû faire ce rêve et non moi ! C'était une façon d'éluder le message du rêve, qui (j'aurais dû le savoir d'emblée par mon expérience passée) ne concernait nul autre que moi. J'ai fini par m'en rendre compte dans la nuit qui a suivi cette première phase, superficielle, du travail ; que j'ai repris le lendemain dans la même lettre. Je n'ai plus reçu, depuis cette lettre mémorable ; signe de vie de cet ami, un des plus proches que j'ai eus.

Ce travail a été la seule méditation qui ait pris forme de lettre (et en langue anglaise par dessus le marché), et dont de ce fait je n'ai plus de trace écrite. Cet épisode m'a particulièrement frappé, parmi de nombreux autres qui montrent à quel point tout signe d'un travail qui va au-delà d'une certaine façade, et qui amène au jour des faits tout simples, mais qu'on se fait généralement un devoir d'ignorer - à quel point tout tel travail inspire malaise et frayeur en autrui. Je reviens là-dessus plus loin (voir par. 47, ``L'aventure solitaire'').}(40). Cela fait, j'ai su ce que j'avais à faire. Je ne suis plus revenu sur ce rêve au cours de mon travail pendant les six mois qui ont suivi, mais je ne faisais autre chose pourtant que pénétrer plus avant dans son sens et assimiler pleinement son message. Au surlendemain du rêve, ce message était compris à un niveau qui restait superficiel et grossier. Ce qu'il me fallait approfondir, surtout, c'était ``ma'' relation ; celle du patron j'entends, à l'un et l'autre des deux désirs en présence, lesquels m'apparaissaient comme antagonistes.

Tant de choses se sont passées dans ma vie depuis cette méditation, que celle-ci m'apparaît comme dans un passé très lointain. Si j'essaye de formuler ce que j'ai retenu de ce qu'elle m'a enseigné au sujet des motivations du ``patron'', il vient ceci : pendant les douze années qui s'étaient alors écoulées depuis le ``premier réveil'' (de 1970), le patron avait misé sur ce qui visiblement, était ``le mauvais cheval'' : entre la mathématique et la méditation (qu'il se plaisait à opposer l'une à l'autre) il avait opté pour la méditation.

C'est la une façon de parler, puisque la chose et le nom ``méditation'' n'étaient entrés dans ma vie qu'en Octobre 1976, cinq ans auparavant. Mais dans la chère image de moi qui en 1970 s'était vue repeinte à neuf, la méditation venait à point nommé, six ans plus tard, rehausser de son éclat une certaine attitude ou pose, repérée de longue date mais jamais examinée jusqu'en cette méditation de 1981. Je la désignais sous le nom de ``syndrome du maître'', et certains l'ont appelée aussi (à juste titre), ma ``pose de Guru''. Si j'ai adopté la première désignation plutôt que la seconde, c'est sans doute qu'elle favorisait une confusion sur la nature de la chose, dans laquelle il me plaisait de me maintenir. Il y avait bien en moi, depuis ma petite enfance déjà, un plaisir spontané à enseigner, qui ne s'opposait nullement au plaisir spontané à apprendre, et qui n'avait rien d'une pose. C'était cette force-là surtout qui était en jeu en moi dans ma relation à mes élèves ; cette relation était superficielle, mais elle était forte et de bon alloi, par quoi j'entends : sans pose. C'est après ce que j'ai appelé mon ``réveil'' de 1970, alors qu'un univers qui m'avait été familier reculait au point presque de disparaître, et avec lui aussi les élèves et les occasions que j'avais ``d'enseigner'', de faire part de choses que je connaissais et qui pour moi avaient un sens et de la valeur - c'est alors que ``le patron'' a pris sa revanche comme il a pu : au lieu d'enseigner des maths, chose tout juste bonne pour gagner sa vie, mais à part ça indigne de ma nouvelle grandeur je me voyais enseigner par ma vie et l'exemple une certaine ``sagesse''. Je prenais bien garde bien entendu de rien formuler de tel ni à moi-même ni aux autres, et quand je recevais des échos dans ce sens, sûrement je devais me récuser, peiné de tant d'incompréhension de la part de tels amis ou proches. J'avais beau leur expliquer, ils s'obstinaient à ne pas comprendre, élèves désolants s'il en fût !

J'avais lu un livre ou deux de Krishnamurti qui m'avaient fortement impressionnés, et la tête avait assimilé en un tournemain un certain message et certaines valeurs \footnote{(41) Krishnamurti, ou la libération devenue entrave

Il serait inexact de dire que la seule chose que j'aie retiré de cette lecture soit un certain vocabulaire, et une propension à le faire mien et à le substituer finalement, comme de juste, à la réalité. Si la lecture du premier livre de Krishnamurti que j'ai eu entre les mains m'a tellement frappé (et encore n'ai-je eu le loisir d'en lire que quelques chapitres), c'est parce que ce qu'il disait bousculait totalement nombre de choses qui pour moi allaient le soi, et dont je me rendais compte aussitôt que c'étaient des lieux communs qui avaient fait partie depuis toujours de l'air que j'avais respiré. En même temps, cette lecture attirait mon attention, pour la première fois, sur des faits d'une grande portée, et surtout celui de la fuite devant la réalité, comme un des conditionnements de l'esprit les plus puissants et les plus universels. Cela me donnait une clef essentielle pour comprendre des situations qui jusque là avaient été incompréhensibles et par là (sans que je m'en rende compte avant la découverte de la méditation cinq ou six ans plus tard) génératrices d'angoisse. J'ai pu constater immédiatement la réalité de cette fuite partout autour de moi. Cela a dénoué certaines angoisses, sans pourtant changer rien d'essentiel, car je ne voyais cette réalité-là qu'en autrui, tout en me figurant (comme allant de soi) qu'elle n'existait pas en moi-même, que j'étais en somme l'exception qui confirmait la règle (et sans me poser aucune autre question au sujet de cette exception vraiment remarquable). En fait, je n'étais aucunement curieux ni d'autrui ni de moi-même. Cette ``clef'' ne peut ouvrir que dans les mains de celui animé du désir de pénétrer. Dans mes mains elle était devenue exorcisme et pose.

C'est au début de 1974 que pour la première fois je me suis rendu à l'évidence que la destruction dans ma vie, qui me suivait pas à pas, ne pouvait pas venir que des autres, qu'il y avait quelque chose en moi qui l'attirait, l'alimentait, la perpétuait. C'était un moment d'humilité et d'ouverture, propice à un renouvellement. Celui-ci est resté alors périphérique encore et éphémère, faute d'un travail en profondeur. Ce ``quelque chose en moi'' restait encore vague. Je voyais bien que c'était le manque d'amour, mais l'idée même d'un travail qui cernerait de plus près où et comment il y avait eu un manque d'amour en moi, comment il s'est manifesté, quels ont été ses effets concrets, etc... - une telle idée ne pouvait me venir ni d'aucun des milieux ou des personnes que j'avais connus jusqu'à ce jour, ni de Krishnamurti. (Bien au contraire, K. se plaît à insister sur la vanité de tout travail, qu'il assimile automatiquement à la ``fringale de devenir'' du moi.) Ainsi, avec une ``sagesse'' d'emprunt pour toute boussole, je ne voyais rien d'autre à faire que d'attendre patiemment que ``l'amour'' descende en moi comme une grâce du Saint Esprit.

Pourtant, l'humble vérité que je venais d'apprendre au fin creux d'une vague avait suscité la montée d'une puissante vague d'énergie nouvelle, comparable à celle qui devait porter deux ans et demi plus tard ma première lancée dans la méditation. Cette énergie alors n'est pas restée entièrement inemployée. Quelques mois plus tard, alors que j'étais immobilisé par un accident providentiel, elle a porté une réflexion (écrite) où, pour la première fois de ma vie, j'examinais la vision du monde qui avait été la base inexprimée de ma relation à autrui, et qui me venait de mes parents et surtout de ma mère. Je me suis rendu compte alors très clairement que cette vision avait fait faillite, qu'elle était inapte à rendre compte de la réalité des relations entre personnes, et à favoriser un épanouissement de ma personne et de mes relations à autrui. Cette réflexion reste marquée par le ``style Krishnamurti'', et aussi par le tabou krishnamurtien sur tout véritable travail vers une compréhension. Elle a pourtant rendue tangible et irréversible une connaissance née quelques mois auparavant, restée d'abord floue et élusive. Cette connaissance, aucun livre ni aucune autre personne au monde n'aurait pu alors me l'apporter.

Pour avoir qualité de méditation, il manquait surtout à cette réflexion le regard sur ma propre personne et sur ma vision de moi-même, et non seulement sur ma vision du monde, sur un système d'axiomes donc où je ne figurais pas vraiment ``en chair et en os''. Et aussi il y manquait, le regard sur moi-même dans l'instant, au moment même de la réflexion (qui restait en deçà d'un véritable travail) ; regard qui m'aurait fait déceler aussi rien un style d'emprunt, qu'une certaine complaisance dans l'aspect littéraire de ces notes, un manque donc de spontanéité, d'authenticité. Toute insuffisante qu'elle soit, et de portée relativement limitée dans ses effets immédiats sur mes relations à autrui, cette réflexion m'apparaît pourtant comme une étape, probablement nécessaire vu le point de départ, vers le renouvellement plus profond qui devait avoir lieu deux ans plus tard. C'est alors enfin que je découvre la méditation - en découvrant ce premier fait insoupçonné : qu'il y avait des choses à découvrir sur ma propre personne - des choses qui déterminaient de façon quasiment complète le cours de ma vie et la nature de mes relations à autrui...}(41). Il n'en fallait pas plus pour croire que tout était arrivé (tout en prétendant le contraire bien sûr). J'avais pas besoin d'en lire plus, j'étais capable d'improviser du plus pur Krishnamurti par la parole comme par l'écrit, dans un discours d'une cohérence sans failles. Mais le discours avait beau être beau et sans failles, à aucun moment il n'a eu l'air de servir à quoi que ce soit ni à moi ni à autrui. Ça a duré des années sans que je fasse mine d'en prendre de la graine. Avec la découverte de la méditation, le jargon s'est détaché de moi du jour au lendemain, sans laisser de traces. J'ai su alors toute la différence entre un discours et une connaissance.

Le grand chef a rectifié le tir aussitôt : Krishnamurti à la trappe, la méditation en épingle ! Discrètement, il va sans dire, il fallait maintenant qu'il joue avec un tout autre doigté. Les temps avaient changé, avec ce gosse qui maintenant lui courait entre les pattes, et qui avait l'oeil un peu vif parfois. Il faut croire que le gosse était occupé ailleurs. Toujours est-il que c'est cinq ans plus tard seulement, alors qu'une certaine marmite avait explosé et que le gosse était accouru voir ce qui se passait, que le manège du grand chef a été percé à jour.

C'était il n'y a pas si longtemps finalement, ça fait à peine plus de deux ans, que le Guru-sans-en-avoirl'air a été enfin éventé - un déguisement de plus à la trappe ! Le pauvre patron, il allait se retrouver tout nu, quasiment. Ou pour le dire autrement : le cheval ``Méditation'', qui avait pris la place du cheval sans nom (qu'il ne fallait surtout pas appeler ``krishnamurtien'' !) fait des retours de mise vraiment dérisoires, surtout si on les compare aux coquets retours du cheval ``mathématique'' aux temps lointains où le patron misait encore sur lui. S'il a maintenu la mauvaise mise pendant si longtemps, c'était par inertie pure - il avait déjà changé de mise une fois, c'est déjà pas si courant et il avait fallu pour cela tout l'impact d'un événement percutant \footnote{(42) L'arrache salutaire

``L'événement "percutant" en question a été la découverte, à la fin de l'année 1969, du fait que l'institution dont je me sentais faire partie était partiellement financée par des fonds provenant du ministère des armées, chose qui était incompatible avec mes axiomes de base (et l'est d'ailleurs encore aujourd'hui). Cet événement a été le premier dans toute une chaîne d'autres (plus révélateurs les uns que les autres !) qui ont eu pour effet mon départ de l'IHES (Institut des Hautes Etudes Scientifiques), et de fil en aiguille un changement radical de milieu et d'investissements.

Pendant les années héroïques de l'IHES, Dieudonné et moi en avons été les seuls membres, et les seuls aussi à lui donner crédibilité et audience dans le monde scientifique, Dieudonné par l'édition des "Publications Mathématiques" (dont le premier volume est paru dès 1959, l'année qui a suivie celle de la fondation de l'IHES par Léon Motchane), et moi par les "Séminaires de Géométrie Algébrique". Dans ces premières années, l'existence de l'IHES restait des plus précaires, avec un financement incertain (par la générosité de quelques compagnies faisant office de mécènes) et avec pour seul local une salle prêtée (avec une mauvaise humeur visible) par la Fondation Thiers à Paris pour les jours de mon séminaire [Une récente brochure éditée par l'IHES à l'occasion de l'anniversaire des vingt-cinq ans de sa fondation (dont Nico Kuiper a eu la gentillesse de m'envoyer un exemplaire) ne souffle mot de ces débuts difficiles, jugés peut-être indignes de la solennité de l'occasion, fêtée en grande pompe l'an dernier.)]. Je me sentais un peu comme un cofondateur "scientifique", avec Dieudonné, de mon institution d'attache, et je comptais bien y finir mes jours ! J'avais fini par m'identifier fortement à l'IHES, et mon départ (comme conséquence de l'indifférence de mes collègues) a été vécu comme une sorte d'arrachement à un autre "chez moi", avant de se révéler comme une libération.

Avec le recul, je me rends compte qu'il devait déjà y avoir en moi un besoin de renouvellement, je ne saurais dire depuis quand. Ce n'est sûrement pas une simple coïncidence si l'année qui a précédé mon départ de l'IHES, il y a eu un soudain basculement de mon investissement d'énergie, laissant là les tâches qui la veille encore me brûlaient dans les mains, et les questions qui me fascinaient le plus, pour me lancer (sous l'influence d'un ami biologiste, Mircea Dumitrescu) dans la biologie. Je m'y lançais dans les dispositions d'un investissement de longue haleine au sein de l'IHES (ce qui était en accord avec la vocation pluridisciplinaire de cette institution). Sûrement ce n'était là qu'un exutoire au besoin d'un renouvellement beaucoup plus profond, qui n'aurait pu s'accomplir dans l'ambiance d' "étuve scientifique" de l'IHES, et qui s'est fait au cours de cette "cascade de réveils" à laquelle j'ai fait déjà allusion. Il y en a eu sept, dont le dernier a eu lieu en 1982. L'épisode des "fonds militaires" a été providentiel en déclenchant le premier de ces "réveils". Le ministère des armées, tout comme mes ex-collègues de l'IHES, ont finalement eu droit à toute ma reconnaissance !}(42). Les patrons ils aiment pas tellement changer de mise - et là il s'agissait même d'une sorte de retour en arrière, à la mise précédente.

C'est à partir de 1973, quand je me suis retiré à la campagne, que les retours du nouveau cheval ont commencé à se faire vraiment maigres en comparaison avec celui d'antan. L'apparition inopinée de la méditation trois ans plus tard les a un peu relancés. Il y a eu même l'épisode d'une pointe vertigineuse de mars à juillet 1979, sur lequel je ne m'étendrai pas ici, où à nouveau je prenais figure d'apôtre, apôtre cette fois d'une sagesse immémoriale et nouvelle à la fois, chantée dans un ouvrage poétique de ma composition et que je me suis abstenu finalement de confier aux mains d'un éditeur \footnote{``L'ouvrage poétique de ma composition'' contient beaucoup de choses que je connais de première main, et qui aujourd'hui m'apparaissent comme tout aussi importantes dans ma vie, et ``dans la vie'' en général, qu'au moment où il fut écrit, avec l'intention de le publier. Si je m'en suis abstenu, c'est surtout parce que je me suis rendu compte ultérieurement que la forme était affligée par un propos délibéré de ``faire poétique'', de sorte que sa conception d'ensemble trop construite, et de nombreux passages, manquent de spontanéité, au point par moments d'une raideur ou d'une enflure pénibles. Cette forme, ampoulée par moments, était le reflet de mes dispositions, où décidément c'est le ``patron'' souvent qui menait la danse - lourdement il va de soi...}(43). Mais deux ans après, avec le Guru définitivement hors service, c'était un peu comme si le cheval Méditation s'était cassé une jambe (pour ce qui était des retours au patron) - il n'y avait même plus moyen, doigté ou pas doigté, de jouer les Gurus !

Après ça, ça n'a plus beaucoup traîné - le cheval à trois pattes à la trappe, avec l'apôtre-poète, le Guru-pasGuru et Krishnamurti-qui-n'ose-dire-son-nom. Et vive la Mathématique !

On attend avec intérêt la suite des événements...