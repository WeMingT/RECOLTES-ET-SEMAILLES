\section{(44) On re-renverse la vapeur}

Là ça allait faire un an et demi que je n'ai pas médité, à part quelques heures au mois de décembre, pour y voir clair dans une question urgente. Et ça fait un an que j'investis le plus gros de mon énergie à faire des maths. Cette ``vague''-là est venue comme les autres, vagues-maths ou vagues-méditation : elles viennent sans annoncer leur venue. Ou si elles s'annoncent, je ne l'entends jamais! Le patron garde une petite préférence pour la méditation, faut-il croire : à chaque fois la vague-méditation est déjà suivie par une vague-maths ; alors que je la voyais durer à jamais ; et la vague-maths qui (me semblait-il) était une affaire de quelques jours ou tout au plus de semaines, s'attarde et s'étend sur des mois et peut-être même, qui sait, sur des années. Mais le patron a fini par comprendre que ce n'est pas lui qui fait ces rythmes et qu'il n'a rien à gagner à vouloir les régler.

Mais peut-être y a-t-il eu finalement un basculement dans la ``petite préférence'' du patron, puisque ça fait près d'un an que c'est chose entendue et décidée, que je suis parti pour quelques années au moins à ``refaire des maths'', officiellement pour ainsi dire : j'ai même posé ma candidature à un poste au CNRS! Chose plus importante, et entièrement inattendue il y a un an encore, je me remets à publier. Même après la méditation de 1981 dont j'ai parlé tantôt, quand l'envie de faire des maths a cessé d'être traitée en parente pauvre, l'idée ne me serait pas venue que je pourrais me remettre à publier des maths. Autre chose à la rigueur, un livre où je parlerais de la méditation, ou du rêve et du Rêveur - et encore, j'étais bien trop occupé à ce que je faisais pour avoir envie d'écrire un livre dessus ! Et pour quoi faire ?!

Il y a donc eu là une sorte de décision assez importante, qui engage le cours de ma vie pour les années à venir, et qui a été prise un peu par la bande, je ne saurais même trop dire quand et comment. Un jour, quand il a commencé à y avoir un bon paquet de notes dactylographiées (tiens tiens ! jusque là je m'étais borné à écrire à la main mes cogitations mathématiques... \footnote{(38) \par Ces notes étaient en fait la continuation de la longue lettre à..., qui en est devenue le premier chapitre. Elles étaient tapées à la machine pour être lisibles pour cet ami d'antan, et pour deux ou trois autres (dont surtout Ronnie Brown) dont je pensais qu'ils pourraient être intéressés. Cette lettre d'ailleurs n'a jamais reçu de réponse, et elle n'a pas été lue par le destinataire, qui près d'un an après (à ma question s'il l'avait bien reçue) se montrait sincèrement étonné que j'avais pu penser même un moment qu'il pourrait la lire, vu le genre de mathématiques qu'on devait attendre de moi...}(38), sur les champs et les modèles homotopiques, etc..., il s'est trouvé que c'était chose décidée : on publie ça ! Et tant qu'à faire, autant mettre le paquet et démarrer une petite série de réflexions mathématiques, dont le nom était tout trouvé, il suffisait de mettre des majuscules : ``Réflexions Mathématiques''! C’est ça plus ou moins ce que me restitue en ce moment ce fameux ``brouillard'', qui si souvent me tient lieu de souvenir. Souvenir sûrement très raccourci, en l'occurrence. La chose remarquable, en tous cas, c'est que cette chose s'est faite sans même un temps d'arrêt pour regarder où j'allais, ce qui me poussait, ou me portait... C'est ça que j'aurais envie encore de faire, sur la lancée de cette méditation imprévue, pour pouvoir la sentir comme vraiment achevée.

La question qui vient tout de suite à l'esprit : cette ``chose remarquable'' que je viens de constater, est-elle un signe de la (soi-disante ?) ``discrétion'' du patron, qui pour rien au monde ne veut interférer (fut-ce par un regard indiscret...) dans un mouvement spontané si beau qui n'a aucun besoin de lui etc... ; ou est-ce le signe au contraire qu'il a pris partie carrément, et que la soi-disante ``petite préférence'' le fait pousser à fond dans la direction maths?

Il a suffi de mettre la question noir sur blanc pour voir apparaître la réponse! Ce n'est pas le gamin, qui est parti là dans un jeu de plus longue haleine que d'autres, peut-être, qui a décrété pour autant qu'il allait continuer pendant X années sans coup férir, et noircir sagement pendant le temps qu'il fallait le nombre de pages voulu pour faire un nombre raisonnable de volumes d'une belle série à titres majuscules ! C'est le patron qui a tout prévu tout organisé, le gosse il a plus qu'à s'exécuter. Peut-être que le gosse lui il demandera pas mieux, on ne peut pas savoir d'avance - mais c'est une question accessoire. Les envies du gosse dépendent d'ailleurs, dans une certaine mesure au moins, des circonstances, lesquelles dépendent surtout du patron.

Le patron a opté, c'est bien clair. Il vient d'ailleurs de faire preuve d'une certaine souplesse, puisque voilà plus d'un mois qu'une méditation se poursuit sous son oeil bienveillant. Il est vrai aussi que sa bienveillance n'est nullement désintéressée, puisque le produit tangible de la méditation, les notes que je suis en train de rédiger, va être la plus belle pierre angulaire de la tour qu'il se voit déjà construire, avec les pierres gracieusement taillées par l'ouvrier-enfant, apparemment bien disposé. Décidément, il est un peu tôt pour lui faire compliment de ``souplesse''! Quelques heures de méditation il y a trois mois, en tout et pour tout dans un an et demi, ça ferait même plutôt maigre!

Pourtant, je n'ai pas l'impression qu'il y ait eu, pendant tout ce temps un désir de méditation qui aurait été réprimé, frustré. Dans les quelques heures en décembre, j'ai fait le point et vu ce que j'avais à voir ; ça a suffi pour transformer une situation, qui n'avait pas été claire. J'ai repris le fil du travail mathématique interrompu, sans avoir à couper court à autre chose. Il ne me semble pas qu'un conflit soit réapparu en tapinois, j'entends ; celui qui s'était résolu il y a plus de deux ans et qui serait réapparu sous forme cette fois inversée. Que le patron ait des préférences, c'est dans sa nature et c'est bien son droit - ce serait idiot qu'il fasse mine de se l'interdire (encore qu'il arrive des choses plus idiotes que celle là...). Ce n'est pas là le signe d'un conflit, même si souvent ça en est la cause. Au point où en sont les choses, il ne semble vraiment pas qu'il y ait à blâmer pour manque de souplesse!

Ceci bien vu, il me reste à essayer de cerner les ``motivations'' du patron, pour ce renversement de vapeur qui s'est fait le plus discrètement du monde, et qui pourtant, à regarder de près, est assez spectaculaire.