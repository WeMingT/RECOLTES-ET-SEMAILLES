\section{(2) Erreur et découverte}

Pour en revenir au style de mon travail mathématique proprement dit, ou à sa "nature" ou à sa "démarche", ils sont maintenant comme devant ceux que le bon Dieu lui-même nous a enseignés sans paroles à chacun, Dieu sait quand, bien longtemps avant notre naissance peut-être. Je fais comme lui. C'est aussi ce que chacun fait d'instinct, dès que la curiosité le pousse de connaître telle chose entre toutes, une chose investie dès lors par ce désir, cette soif...

Quand je suis curieux d'une chose, mathématique ou autre, je l'interroge. Je l'interroge, sans me soucier si ma question est peut-être stupide ou si elle va paraître telle, sans qu'elle soit à tout prix mûrement pesée. Souvent la question prend la forme d'une affirmation - une affirmation qui, en vérité, est un coup de sonde. J'y crois plus ou moins, à mon affirmation, ça dépend bien sûr du point où j'en suis dans la compréhension des choses que je suis en train de regarder. Souvent, surtout au début d'une recherche, l'affirmation est carrément fausse - encore fallait-il la faire pour pouvoir s'en convaincre. Souvent, il suffisait de l'écrire pour que ça saute aux yeux que c'est faux, alors qu'avant de l'écrire il y avait un flou, comme un malaise, au lieu de cette évidence. Ça permet maintenant de revenir à la charge avec cette ignorance en moins, avec une question-affirmation peut-être un peu moins "à côté de la plaque". Plus souvent encore, l'affirmation prise au pied de la lettre s'avère fausse, mais l'intuition qui, maladroitement encore, a essayé de s'exprimer à travers elle est juste, tout en restant floue. Cette intuition peu à peu va se décanter d'une gangue toute aussi informe d'abord d'idées fausses ou inadéquates, elle va sortir peu à peu des limbes de l'incompris qui ne demande qu'à être compris, de l'inconnu qui ne demande qu'à se laisser connaître, pour prendre une forme qui n'est qu'à elle, affiner et aviver ses contours, au fur et à mesure que les questions que je pose à ces choses devant moi se font plus précises ou plus pertinentes, pour les cerner de plus en plus près.

Mais il arrive aussi que par cette démarche, les coups de sonde répétés convergent vers une certaine image de la situation, sortant des brumes avec des traits assez marqués pour entraîner un début de conviction que cette image-là exprime bien la réalité - alors qu'il n'en est rien pourtant, quand cette image est entachée d'une erreur de taille, de nature à la fausser profondément. Le travail, parfois laborieux, qui conduit au dépistage d'une telle idée fausse, à partir des premiers "décollages" constatés entre l'image obtenue et certains faits patents, ou entre cette image et d'autres qui avaient également notre confiance - ce travail est souvent marqué par une tension croissante, au fur et à mesure qu'on approche du noeud de la contradiction, qui de vague d'abord se fait de plus en plus criante - jusqu'au moment où enfin elle éclate, avec la découverte de l'erreur et l'écroulement d'une certaine vision des choses, survenant comme un soulagement immense, comme une libération. La découverte de l'erreur est un des moments cruciaux, un moment créateur entre tous, dans tout travail de découverte, qu'il s'agisse d'un travail mathématique, ou d'un travail de découverte de soi. C'est un moment où notre connaissance de la chose sondée soudain se renouvelle.

Craindre l'erreur et craindre la vérité est une seule et même chose. Celui qui craint de se tromper est impuissant à découvrir. C'est quand nous craignons de nous tromper que l'erreur qui est en nous se fait immuable comme un roc. Car dans notre peur, nous nous accrochons à ce que nous avons décrété "vrai" un jour, ou à ce qui depuis toujours nous a été présenté comme tel. Quand nous sommes mûs, non par la peur de voir s'évanouir une illusoire sécurité, mais par une soif de connaître, alors l'erreur, comme la souffrance ou la tristesse, nous traverse sans se figer jamais, et la trace de son passage est une connaissance renouvelée.
