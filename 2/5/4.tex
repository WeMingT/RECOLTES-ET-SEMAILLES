\section{(4) Infaillibilité (des autres) et mépris (de soi)}

Décidément c'était un euphémisme, quand tantôt je constatais prudemment que "mon style d'expression" avait changé, laissant même entendre qu'il n'y avait rien là qui puisse surprendre : vous comprenez bien, quand on n'a pas écrit depuis treize ans, c'est plus pareil qu'avant, le "style d'expression" il doit changer, forcément... La différence, c'est qu'avant je "m'exprimais" (sic) comme tout le monde : je faisais le travail, puis je le refaisais à l'envers, en effaçant soigneusement toutes les ratures. Chemin faisant, nouvelles ratures, chamboulant tout le travail parfois pire que lors du premier jet. À refaire donc - parfois trois fois, voire quatre, jusqu'à ce que tout soit impec. Non seulement aucun coin douteux ni balayures poussées subrepticement sous un meuble propice (je n'ai jamais aimé les balayures dans les coins, du moment qu'on prend la peine de balayer); mais surtout, en lisant le texte final, l'impression certes flatteuse qui s'en dégageait (comme de tout autre texte scientifique) c'est que l'auteur (ma modeste personne en l'occurrence) était l'infaillibilité incarnée. Infailliblement, il tombait pile sur "les" bonnes notions, puis sur "les" bons énoncés, s'enchaînant dans un ronron de moteur bien huilé, avec des démonstrations qui "tombaient" avec un bruit mat, chacune exactement à son moment!

Qu'on juge de l'effet produit sur un lecteur qui ne se doute de rien, un élève de lycée disons apprenant le théorème de Pythagore ou les équations du second degré, voire un de mes collègues des institutions de recherche ou d'enseignement dit "supérieur" (à bon entendeur, salut !) s'escrimant (disons) sur la lecture de tel article de tel collègue prestigieux ! Ce genre d'expérience se répétant des centaines, des milliers de fois tout au long d'une vie d'écolier, voire d'étudiant ou de chercheur, amplifié par le concert idoine dans la famille comme dans tous les médias de tous les pays du monde, l'effet est celui qu'on peut prévoir. On le constate en soi comme en les autres, pour peu qu'on se donne la peine d'y être attentif : c'est la conviction intime de sa propre nullité, par contraste avec la compétence et l'importance des gens "qui savent" et des gens "qui font".

Cette conviction intime est compensée parfois, mais nullement résolue ni désamorcée, par le développement d'une capacité à mémoriser des choses incomprises, voire par celui d'une certaine habileté opératoire : multiplier des matrices, "monter" une composition française à coups de "thèse" et "antithèse"...C'est la capacité en somme du perroquet ou du singe savant, plus prisée de nos jours qu'elle ne le fut jamais, sanctionnée par des diplômes convoités, récompensée par des carrières confortables. Mais celui-là même cousu de diplômes et bien casé, couvert d'honneurs peut-être, n'est pas dupe, tout au fond de lui-même, de ces signes factices d'une importance, d'une "valeur". Ni même celui, plus rare, qui a investi son va-tout sur le développement de quelque don véritable, et qui dans sa vie professionnelle a su donner sa mesure et faire œuvre créatrice - il n'est pas convaincu, tout au fond de lui-même, par l'éclat de sa notoriété, par quoi souvent il veut donner le change à lui-même et aux autres. Un même doute jamais examiné habite l'un et l'autre tout comme le premier cancre venu, une même conviction dont jamais peut-être ils n'oseront prendre connaissance.

C'est ce doute, cette intime conviction inexprimée, qui poussent l'un et l'autre à se surpasser sans cesse dans l'accumulation des honneurs ou des œuvres, et à projeter sur autrui (sur ceux avant tout sur qui ils ont quelque pouvoir...) ce mépris d'eux-mêmes qui les ronge en secret - en une impossible tentative de s'en évader, par l'accumulation des "preuves" de leur supériorité sur autrui \footnote{(2)\par (Rajouté en mars 1984) En relisant ces deux derniers alinéas, j’ai eu un certain sentiment de malaise, dû au fait qu'en les écrivant, j'implique autrui et non moi-même. Visiblement, la pensée que ma propre personne pourrait être concernée ne m'a pas effleurée en écrivant. Je n'ai sûrement rien appris, quand je me suis ainsi borné à mettre noir sur blanc (sans doute avec une certaine satisfaction) des choses que depuis des années j'ai perçues en autrui, et vues se confirmer de bien des façons. Dans la suite de la réflexion, je suis conduit à me souvenir que des attitudes de mépris vis-à-vis d'autrui n'ont pas manqué dans ma vie. Il serait étrange que le lien que j'ai saisi entre mépris d'autrui et mépris de soi soit absent dans le cas de ma personne ; la saine raison (et aussi l'expérience de situations similaires de cécité à mon propre égard, dont j'ai fini par me rendre compte) me disent qu'il ne doit sûrement pas en être ainsi ! Ce n'est là pourtant, pour l'instant, qu'une simple déduction, dont la seule utilité possible serait de m'inciter à voir de mes yeux ce qui se passe, et voir et examiner (s'il existe bel et bien, ou a existé) ce mépris de moi-même encore hypothétique, si profondément enfoui qu'il a totalement échappé jusqu'à présent à mon regard. Il est vrai que les choses à regarder n'ont pas manqué ! Celle-ci m'apparaît soudain comme l'une des plus cruciales, du fait justement qu'elle est à tel point cachée... [(Août 1984) Voir cependant à ce sujet la réflexion des deux derniers alinéas de la note "Le massacre", $n^{\circ} 87$.].} (2).

