\section{(3) Les inavouables labeurs}

Ce n'est sûrement pas un hasard que la démarche spontanée de toute vraie recherche n'apparaisse pour ainsi dire jamais dans les textes ou le discours qui sont censés communiquer et transmettre la substance de ce qui a été "trouvé". Textes et discours le plus souvent se bornent à consigner des "résultats", sous une forme qui au commun des mortels doit les faire apparaître comme autant de lois austères et immuables, inscrites de toute éternité dans les tables de granit d'une sorte de bibliothèque géante, et dictée par quelque Dieu omniscient aux initiés-scribes-savants et assimilés ; à ceux qui écrivent les livres savants et les articles non moins savants, ceux qui transmettent un savoir du haut d'une chaire, ou dans le cercle plus restreint d'un séminaire. Y a-t-il un seul livre de classe, un seul manuel à l'usage des écoliers, lycéens, étudiants, voire même de "nos chercheurs", qui puisse donner au malheureux lecteur la moindre idée de ce que c'est que la recherche - si ce n'est justement l'idée universellement reçue que la recherche, c'est quand on est très calé, qu'on a passé plein d'examens et même des concours, les grosses têtes quoi, Pasteur et Curie et les prix Nobel et tout ça. . Nous autres lecteurs ou auditeurs, ingurgitant tant bien que mal le Savoir que ces grands hommes ont bien voulu consigner pour le bien de l'humanité, on est tout juste bons (si on travaille dur) à passer notre examen en fin d'année, et encore...

Combien y en a-t-il, y compris parmi les malheureux "chercheurs" eux-mêmes, en mal de thèses ou d'articles, y compris même parmi les plus "savants", \footnote{Les plus prestigieux parmi nous.} les plus prestigieux parmi nous - qui donc a la simplicité de voir que "chercher", ce n'est ni plus ni moins qu'interroger les choses, passionnément - comme un enfant qui veut savoir comment lui ou sa petite soeur sont venus au monde. Que chercher et trouver, c'est-à-dire : questionner et écouter, est la chose la plus simple, la plus spontanée du monde, dont personne au monde n'a le privilège. C'est un "don" que nous avons tous reçu dès le berceau - fait pour s'exprimer et s'épanouir sous une infinité de visages, d'un moment à l'autre et d'une personne à l'autre...

Quand on se hasarde à faire entendre de telles choses, on récolte chez les uns comme chez les autres, du plus cancre sûr d'être cancre, au plus savant sûr d'être savant et bien au-dessus du commun des mortels, les mêmes sourires mi-gênés, mi-entendus, comme si on venait de faire une plaisanterie un peu grosse sur les bords, comme si on était en train d'afficher une naïveté cousue de fil blanc ; c'est bien beau tout ça, faut cracher sur personne c'est entendu - mais faut pas pousser quand même - un cancre c'est un cancre et c'est pas Einstein ni Picasso!

Devant un accord aussi unanime, j'aurais mauvaise grâce d'insister. Incorrigible décidément, j'ai encore perdu une occasion de me taire...

Non, ce n'est sûrement pas un hasard si, avec un ensemble parfait, livres instructifs ou édifiants et manuels de tout poil présentent "le Savoir" comme s'il était sorti habillé de pied en cap des génials cerveaux qui l'ont consigné pour notre bénéfice. On ne peut pas dire non plus que ce soit de la mauvaise foi, même dans les rares cas où l'auteur est assez "dans le coup" pour savoir que cette image (que ne peut manquer de suggérer son texte) ne correspond en rien à la réalité. Dans un tel cas, il arrive que l'exposé présente plus qu'un recueil de résultats et de recettes, qu'un souffle le traverse, qu'une vision vivante l'anime, qui parfois alors se communique de l'auteur au lecteur attentif. Mais un consensus tacite, d'une force considérable semble-t-il, fait que le texte ne laisse subsister la moindre trace du travail dont il est le produit, même lorsqu'il exprime avec une force lapidaire la vision parfois profonde des choses qui est un des fruits véritables de ce travail.

A vrai dire, à certains moments j'ai moi-même confusément senti le poids de cette force, de ce consensus muet, à l'occasion de mon projet d'écrire et publier ces "Réflexions Mathématiques". Si j'essaye de sonder la forme tacite que prend ce consensus, ou plutôt celle que prend la résistance en moi à mon projet, déclenchée par ce consensus, me vient aussitôt le terme "indécence". Le consensus, intériorisé en moi je ne saurais dire depuis quand, me dit (et c'est la première fois que je prends la peine de tirer à la lumière du jour, dans le champ de mon regard, ce qu'il me marmonne avec une certaine insistance depuis des semaines, sinon des mois) : "Il est indécent d'étaler devant autrui, voire publiquement, les hauts et les bas, les tâtonnements foireux sur les bords, le "linge sale" en somme, d'un travail de découverte. Ça ne fait que perdre le temps du lecteur, qui est précieux. De plus, ça va faire des pages et des pages en plus, qu'il faudra composer, imprimer - quel gâchis, au prix où est le papier imprimé scientifique ! Il faut vraiment être bien vaniteux pour étaler comme ça des choses qui n'ont aucun intérêt pour personne, comme si mes cafouillages même étaient choses remarquables - une occasion de se pavaner, en somme". Et plus secrètement encore : "Il est indécent de publier les notes d'une telle réflexion, telle qu'elle se poursuit vraiment, tout comme il serait indécent de faire l'amour sur une place publique, ou d'exposer ou seulement laisser traîner, les draps tâchés de sang des labeurs d'un accouchement... ".

Le tabou ici prend la forme insidieuse et impérieuse à la fois, du tabou sexuel. C'est au moment d'écrire cette introduction que je commence à entrevoir seulement sa force extraordinaire, et la portée de ce fait luimême extraordinaire, attestant cette force : que la démarche véritable de la découverte, d'une simplicité si déconcertante, une simplicité enfantine, ne transparaisse pratiquement nulle part ; qu'elle est silencieusement escamotée, ignorée, niée. Il en est ainsi même dans le champ relativement anodin de la découverte scientifique, pas celle de son zizi ni rien de tel Dieu merci - une "découverte" en somme bonne à être mise entre toutes les mains, et qui (pourrait-on croire) n'a rien à cacher...

Si je voulais suivre le "fil" qui se présente là, un fil nullement ténu mais tout ce qu'il y a de dru et fort - sûrement il me mènerait bien plus loin que les quelques centaines de pages d'algèbre homologico-homotopique que je finirai bien par terminer et livrer à l'imprimeur.

