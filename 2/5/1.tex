Juin 1983

\section{(1) l'enfant et le Bon Dieu}

Les notes mathématiques sur lesquelles je travaille à présent sont les premières depuis treize ans que je destine à une publication. Le lecteur ne s'étonnera pas qu'après un long silence, mon style d'expression ait changé. Ce changement d'expression n'est pas pourtant le signe d'un changement dans le style ou dans la méthode de travail \footnote{(1)\par(Rajouté en mars 1984) Il est sans doute abusif de dire que mon "style" et ma "méthode" de travail n'aient pas changé, alors que mon style d'expression en mathématique s'est profondément transformé. La plus grande partie du temps consacré depuis une année à "La Poursuite des Champs" a été passé sur ma machine à écrire à taper des réflexions qui sont destinées à être publiées pratiquement telles quelles (à l'adjonction près de notes relativement courtes rajoutées ultérieurement pour faciliter la lecture par des renvois, des corrections d'erreurs, etc...). Pas de ciseaux ni de colle pour préparer laborieusement un manuscript "défi nitif" (qui surtout ne doit rien laisser transparaître de la démarche qui y a abouti) - ça fait quand même des changements de "style" et de "méthode"! A moins de dissocier le travail mathématique proprement dit du travail d'écriture, de présentation des résultats, ce qui est artifi ciel, car cela ne correspond pas à la réalité des choses, le travail mathématique étant indissolublement lié à l'écriture.}(1), et encore moins celui d'une transformation qui se serait faite dans la nature même de mon travail mathématique. Non seulement celle-ci est restée pareille à elle-même - mais j'ai acquis la conviction que la nature du travail de découverte est la même d'une personne qui découvre à l'autre, qu'elle est au-delà des différences que créent des conditionnements et des tempéraments variant à l'infini.

La découverte est le privilège de l'enfant. C'est du petit enfant que je veux parler, l'enfant qui n'a pas peur encore de se tromper, d'avoir l'air idiot, de ne pas faire sérieux, de ne pas faire comme tout le monde. Il n'a pas peur non plus que les choses qu'il regarde aient le mauvais goût d'être différentes de ce qu'il attend d'elles, de ce qu'elles devraient être, ou plutôt : de ce qu'il est bien entendu qu'elles sont. Il ignore les consensus muets et sans failles qui font partie de l'air que nous respirons - celui de tous les gens censés et bien connus comme tels. Dieu sait s'il y en a eu, des gens censés et bien connus comme tels, depuis la nuit des âges !

Nos esprits sont saturés d'un "savoir" hétéroclite, enchevêtrement de peurs et de paresses, de fringales et d'interdits ; d'informations à tout venant et d'explications pousse-bouton - espace clos où viennent s'entasser informations; fringales et peurs sans que jamais ne s'y engouffre le vent du large. Exception faite d'un savoirfaire de routine, il semblerait que le rôle principal de ce "savoir" est d'évacuer une perception vivante, une prise de connaissance des choses de ce monde. Son effet est surtout celui d'une inertie immense, d'un poids souvent écrasant.

Le petit enfant découvre le monde comme il respire - le flux et le reflux de sa respiration lui font accueillir le monde en son être délicat, et le font se projeter dans le monde qui l'accueille. L'adulte aussi découvre, en ces rares instants où il a oublié ses peurs et son savoir, quand il regarde les choses ou lui-même avec des yeux grands ouverts, avides de connaître, des yeux neufs - des yeux d'enfant.

\begin{center}
    * \quad * \\
    *
\end{center}

Dieu a créé le monde au fur et à mesure qu'il le découvrait, ou plutôt il crée le monde éternellement, au fur et à mesure qu'il le découvre - et il le découvre au fur et à mesure qu'il le crée. Il a créé le monde et le crée jour après jour, en s'y reprenant des millions de millions de fois, sans répit, en tâtonnant, se trompant des millions de millions de fois et rectifiant le tir, sans se lasser... À chaque fois, dans ce jeu du coup de sonde en les choses, de la réponse des choses ("c'est pas mal ce coup-là", ou : "là tu déconnes en plein", ou "ça marche comme sur des roulettes, continues comme ça"), et du nouveau coup de sonde rectifiant ou reprenant le coup de sonde précédent, en réponse à la réponse précédente..., à chaque aller-et-retour dans ce dialogue infini entre le Créateur et les Choses, qui a lieu en chaque instant et en tous lieux de la Création, Dieu apprend, découvre, Il prend connaissance des choses de plus en plus intimement, au fur et à mesure qu'elles prennent vie et forme et se transforment entre Ses mains.

Telle est la démarche de la découverte et de la création, telle a-t-elle été de toute éternité semble-t-il (pour autant que nous puissions le connaître). Elle a été telle, sans que l'homme ait eu à faire son entrée en scène tardive, il y a à peine un million d'années ou deux, et qu'il mette la main à la pâte - avec, dernièrement, les conséquences fâcheuses que l'on sait.

Il arrive que l'un ou l'autre de nous découvre telle chose, ou telle autre. Parfois il redécouvre alors dans sa propre vie, avec émerveillement, ce que c'est que découvrir. Chacun a en lui tout ce qu'il faut pour découvrir tout ce qui l'attire dans ce vaste monde, y compris cette capacité merveilleuse qui est en lui - la chose la plus simple, la plus évidente du monde ! (Une chose pourtant que beaucoup ont oubliée, comme nous avons oublié de chanter, ou de respirer comme un enfant respire...)

Chacun peut redécouvrir ce que c'est que découverte et création, et personne ne peut l'inventer. Ils ont été là avant nous, et sont ce qu'ils sont.

