\section{(33) La note - ou la nouvelle éthique}

Certes, une règle de déontologie ne prend son sens que par une attitude intérieure, qui en est l'âme. Elle ne saurait créer l'attitude de respect et d'équité qu'elle s'efforce d'exprimer, tout au plus peut-elle contribuer à la permanence d'une telle attitude, dans un milieu où cette règle jouirait d'un consensus général. En l'absence de l'attitude intérieure ; alors même que la règle serait professée par les lèvres, elle perd tout sens ; toute valeur. Aucune exégèse, si scrupuleuse, si méticuleuse soit-elle, n'y changerait rien.

Tel de mes amis et compagnons d'antan m'a expliqué gentiment dernièrement que par les temps qui courent, hélas, avec l'afflux démesuré qu'on sait de la production mathématique, ``on'' est absolument obligé, qu'on le veuille ou non, de faire un tri sévère dans les papiers qui sont écrits et soumis pour publication, de n'en publier que juste une petite partie. Il le disait d'un air sincèrement désolé, comme s'il était lui-même un peu victime de cette fatalité inéluctable - un peu l'air qu'il avait aussi pour dire qu'il faisait lui-même partie, eh oui c'est malheureux mais c'est comme ça !, des ``six ou sept personnes en France'' qui décident quels articles vont être publiés, et lesquels non. Etant devenu moins loquace avec l'âge, je me suis borné à écouter en silence. Il y avait beaucoup à dire sur ce thème, mais je savais que ce serait peine perdue. Un ou deux mois plus tard j'ai d'ailleurs appris que ce collègue avait refusé il y a quelques années de recommander la publication d'une certaine note aux CR, dont l'auteur aussi bien que le thème (que je lui avais proposé il doit y avoir sept ou huit ans) me tiennent à coeur. L'auteur avait passé deux ans de sa vie à développer ce thème, qui n'est pas à la mode il est vrai (alors qu'il me paraît toujours aussi actuel). Je pense qu'il a fait un excellent travail (présenté comme thèse de 3e cycle). Je n'ai pas été le ``patron'' de ce jeune chercheur, brillamment doué il se trouve (j'ignore s'il continuera à appliquer ses dons dans les mathématiques, vu l'accueil\ldots), et il a fait son travail sans aucun contact avec moi. Mais il est vrai aussi que la provenance du thème développé ne pouvait faire aucun doute ; il était mal barré le pauvre, et sans se douter de rien sûrement ! Ce collègue y a d'ailleurs mis les formes, c'est au moins ça et je n'en aurais pas attendu moins de lui, ``sincèrement désolé mais vous comprenez\ldots''. Deux ans de travail d'un chercheur débutant fortement motivé, contre une note aux CR de trois pages - combien aurait-elle coûté de deniers publics ? Il y a une absurdité qui saute aux yeux, cette disproportion énorme entre l'un et l'autre. Sûrement cette absurdité disparaît, si on prend la peine d'examiner les motivations profondes. Seul ce collègue et ancien ami est en mesure de sonder ses propres motivations, comme je suis seul en mesure de sonder les miennes. Mais sans avoir à aller bien loin, je sais bien que ce n'est pas l'afflux démesuré de la production mathématique vous savez, ni les deniers publics (ou la patience d'un imaginaire ``lecteur inconnu'' des CR) qu'il se serait agi de ménager...

Ce même projet de note aux CR avait eu l'honneur déjà d'être soumis à un autre parmi les ``six ou sept personnes en France\ldots'', qui l'a renvoyé au ``patron'' de l'auteur, car ces mathématiques ``ne l'amusaient pas'' (textuel !). (Le patron, écoeuré mais prudent, lui-même en position plutôt précaire, a préféré les deux fois s'écraser plutôt que de déplaire\ldots) Ayant eu l'occasion de parler de la chose avec ce collègue et ex-élève, j'ai appris qu'il avait pris la peine de lire avec attention la note soumise et d'y réfléchir (elle devait lui rappeler bien des souvenirs\ldots), et qu'il avait trouvé que certains des énoncés auraient pu être présentés de façon plus serviable pour l'utilisateur. Il n'a pas daigné pourtant gaspiller son temps précieux à soumettre ses commentaires à l'intéressé : quinze minutes de l'homme illustre, contre deux ans de travail d'un jeune chercheur inconnu ! Les maths l'ont bien ``amusé'' assez pour saisir cette occasion de reprendre contact avec la situation étudiée dans la note (qui ne pouvait manquer de susciter en lui, tout comme en moi-même, un riche tissu d'associations géométriques diverses), d'assimiler la description donnée, puis, sans mal vu son bagage et ses moyens, détecter les maladresses ou lacunes. Il n'a pas perdu son temps : sa connaissance d'une certaine situation mathématique s'est précisée et enrichie, grâce à deux ans de travail consciencieux d'un chercheur faisant ses premières armes ; travail que le Maître aurait certes été capable de faire (dans les grandes lignes et sans démonstrations) en quelques jours. Cela acquis, on se rappelle qui on est - la cause est jugée, deux ans de travail de Monsieur Personne sont bons pour la poubelle\ldots

Il y en a qui ne sentent rien quand souffle ce vent-là - mais aujourd'hui encore j'en ai le souffle coupé. C'était sûrement un des effets recherchés dans ce cas-là (vue la forme exquise mise au refus), mais sûrement pas le seul. Dans ce même entretien, cet ami d'antan me confiait, avec un air de fierté modeste, qu'il n'acceptait de présenter une note aux CR que lorsque ``les résultats énoncés l'étonnaient, ou qu'il ne saurait comment les démontrer''\footnote{(27) Le ``snobisme des jeunes'', ou les défenseurs de la pureté

Ronnie Brown m'a fait part d'une réflexion de J.H.C. Whitehead (dont il a été élève), parlant du ``snobisme des jeunes, qui croient : qu'un théorème est trivial parce que sa démonstration est triviale''. Beaucoup de mes amis d'antan auraient intérêt à méditer ces paroles. Ce ``snobisme''- là n'est aujourd'hui nullement limité aux jeunes, et je connais plus d'un mathématicien prestigieux qui le pratique couramment. J'y suis tout particulièrement sensible, car ce que j'ai fait de mieux en mathématiques (et ailleurs aussi\ldots), les notions et structures que j'ai introduites qui m'apparaissent les plus fécondes, et les propriétés essentielles que j'ai pu en dégager par un travail patient et obstiné, tombent toutes sous ce qualificatif de ``trivial''. (Aucune de ces choses n'aurait eu de nos jours grande chance à se voir accepter pour une note aux CR, si l'auteur n'était déjà une célébrité !) Mon ambition de mathématicien ma vie durant, ou plutôt ma passion et ma joie ont été constamment de trouver les choses évidentes, et c'est ma seule ambition aussi dans le présent ouvrage (y compris dans le présent chapitre introductif\ldots), La chose décisive souvent, c'est déjà de voir la question qui n'avait pas été vue (quelle qu'en soit la réponse, et que celle-ci soit déjà trouvée ou non) ou de dégager un énoncé (fut-il conjectural) qui résume et contienne une situation qui n'avait pas été vue ou pas été comprise ; s'il est démontré, peu importe que la démonstration soit triviale ou non, chose entièrement accessoire, ou même qu'une démonstration hâtive et provisoire s'avère fausse. Le snobisme dont parle Whitehead est celui du viveur blasé qui ne daigne apprécier un vin qu'après s'être assuré qu'il a coûté très cher. Plus d'une fois en ces dernières années, repris par accès par mon ancienne passion, j'ai offert ce que j'avais de meilleur, pour le voir rejeté par cette suffisance-là. J'en ai ressenti une peine qui reste vivante, une joie s'est trouvée déçue - mais je ne suis pas à la rue pour autant, et je n'essayais pas, heureusement pour moi, de caser un article de ma composition.

Le snobisme dont parle Whitehead est un abus de pouvoir et une malhonnêteté, non seulement une insensibilité ou une fermeture à la beauté des choses, lorsqu'il s'exerce par un homme de pouvoir à l'encontre d'un chercheur à sa merci, dont il a toute latitude d'assimiler et utiliser les idées, tout en bloquant leur publication sous prétexte qu'elles sont ``évidentes'' ou ``triviales'', et donc ``sans intérêt''. Je ne songe pas même ici à la situation extrême du plagiat au sens courant du terme, qui doit être encore très rare en milieu mathématique. Pourtant au point de vue pratique la situation revient au même pour le chercheur qui en fait les frais, et l'attitude intérieure qui la rend possible ne me paraît pas non plus bien différente. Elle est simplement plus confortable, alors, qu'elle s'accompagne du sentiment d'une infinie supériorité sur autrui, et de la bonne conscience et l'intime satisfaction de celui qui se pose en défenseur intransigeant de l'intangible pureté de la mathématique.}(27). C'est sans doute une raison pour laquelle il ne publie que peu. S'il appliquait à lui-même ses propres critères, il ne publierait pas du tout. (Il est vrai que dans la situation où il se trouve, il n'en a nul besoin.) Il est au courant de tout, et il doit être aussi difficile de l'étonner, que de trouver chose démontrable qu'il ne sache démontrer. (L'un ou l'autre ne m'est guère arrivé que deux ou trois fois en l'espace de vingt ans, et encore pas depuis dix ou quinze ans !) Il est visiblement fier de ses critères de ``qualité'', qui le posent en champion de l'exigence poussée à son degré extrême dans l'exercice du métier de mathématicien. J'y ai vu une complaisance à lui-même à toute épreuve, et plus d'une fois un mépris sans retenue pour autrui, derrière les apparences d'une modestie souriante et bon enfant. J'ai pu voir également qu'il y trouve de grandes satisfactions.

Le cas de ce collègue est le plus extrême que j'aie rencontré parmi les représentants de la ``nouvelle éthique''. Il n'en est pas moins typique. Ici encore, tant dans l'incident que j'ai rapporté que dans la profession de foi qui le rationalise, il y a une absurdité ubuesque, en termes de simple bon sens - aux dimensions si énormes que cet ancien ami au cerveau si exceptionnel, et aussi sûrement beaucoup de ses collègues au statut moins prestigieux (qui se contenteront de ne pas s'adresser à lui pour présenter une note aux CR) ne la voient plus. Pour voir en effet, il faut pour le moins regarder. Quand on prend la peine de regarder les motivations (et les siennes propres en tout premier lieu), alors les absurdités apparaissent en pleine lumière, et elles cessent en même temps d'être absurdes, en livrant leur sens humble et évident.

Si en ces dernières années il m'a été souvent à tel point pénible de me voir confronté à certaines attitudes et surtout à certains comportements, c'est sûrement que j'y discernais obscurément comme une caricature poussée à l'extrême, jusqu'au grotesque ou l'odieux, d'attitudes et de comportements qui avaient été miens et qui revenaient sur moi par tel de mes anciens élèves ou amis. Plus d'une fois s'est déclenché en moi le vieux réflexe de dénoncer, de combattre ``le mal'' clairement désigné du doigt - mais s'il m'est arrivé d'y céder, ici et là, c'était avec une conviction divisée. Au fond, je sais bien que me battre, c'est encore continuer à patiner à la surface des choses, c'est éluder. Mon rôle n'est pas de dénoncer, ni même ``d'améliorer'' le monde dans lequel je me trouve, ou ``d'améliorer'' ma propre personne. Ma vocation est d'apprendre, de connaître ce monde à travers moi-même, et me connaître à travers ce monde. Si ma vie peut apporter un quelconque bienfait à moi-même ou à autrui, c'est dans la mesure où je saurai être fidèle à cette vocation, où je saurai être en accord avec moi-même. Il est temps de me le rappeler, pour couper court à ces vieux mécanismes en moi, qui ici me voudraient pousser à plaider une cause (d'une certaine éthique morte disons), ou à convaincre (du caractère soi-disant ``absurde'' de telle éthique qui l'a remplacée, peut-être), plutôt que de sonder pour découvrir et connaître, ou de décrire comme un moyen de sonder. En écrivant les deux ou trois pages qui précèdent, sans propos plus précis que celui de dire quelques mots au sujet des attitudes courantes d'aujourd'hui qui ont remplacé celles de hier, je me suis senti continuellement sur mes gardes vis-à-vis de moi-même, dans les dispositions de celui qui serait préparé d'un moment à l'autre à barrer d'un grand trait tout ce qu'il vient d'écrire pour le jeter à la corbeille ! Je vais conserver pourtant ce que j'ai écrit, qui n'est pas faux mais néanmoins crée une situation fausse, du fait que j'y implique autrui plus que je ne m'y implique. Je sentais au fond que je n'apprenais rien en écrivant, c'est cela sûrement qui a créé ce malaise en moi. Décidément il est temps de revenir à une réflexion plus substantielle, qui m'instruise au lieu de prétendre instruire ou convaincre autrui\footnote{(28) En écrivant les pages précédentes, j'avais d'abord été divisé entre le désir de ``vider mon sac'', et un souci de réserve ou de discrétion. Aussi j'étais resté dans l'à-peu-près, ce qui était sûrement la principale raison de mon malaise, du sentiment que ``je n'apprenais rien''. Depuis que les lignes constatant ce malaise ont été écrites, j'ai réécrit deux fois ces pages qui m'avaient laissé sur un mécontentement intérieur, en m'y impliquant plus clairement et en allant plus au fond des choses. Chemin faisant j'ai bel et bien fini par ``apprendre quelque chose'', et je crois aussi qu'en même temps j'ai réussi à mettre le doigt sur quelque chose d'important, qui dépasse aussi bien le cas d'espèce que ma propre personne.}(28).

