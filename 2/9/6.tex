\section{(38) Pulsion de retour et renouvellement}

Le ravissement qui rayonnait par moments en la personne de Dieudonné a sûrement touché en moi quelque chose de profond et de fort, pour que le souvenir m'en revienne maintenant avec une telle intensité, une telle fraîcheur, comme si je venais d'en être encore témoin à l'instant. (Alors que cela fait prés de quinze ans que je n'ai guère eu l'occasion de rencontrer Dieudonné, sauf une fois ou deux en coup de vent.) Bien sûr, je n'y accordais aucune attention particulière au niveau conscient - c'était tout juste une particularité un peu touchante, par moments presque comique, de la personnalité expansive de mon collègue aine et ami. Ce qui m'importait par contre, c'était d'avoir trouvé en lui le collaborateur parfait, rêvé pourrais-je dire, pour mettre noir sur blanc avec un soin méticuleux, un soin amoureux, ce qui devait servir de fondements pour les vastes perspectives que je voyais s'ouvrir devant moi. C'est en cet instant seulement où j'évoque l'un et l'autre que le lien m'apparaît soudain : ce qui faisait de Dieudonné le serviteur rêvé d'une grande tâche, que ce soit au sein de Bourbaki ou dans la collaboration qui a été la nôtre pour un autre grand travail de fondations, était la générosité, l'absence de toute trace de vanité, dans son travail et dans le choix de ses grands investissements. Constamment je l'ai vu s'effacer derrière les tâches dont il s'est fait le serviteur, leur prodiguant sans compter une énergie inépuisable, sans y chercher aucun retour. Nul doute que sans rien y chercher, il trouvait dans son travail et dans la générosité même qu'il y mettait une plénitude et un épanouissement, que tous ceux qui le connaissent ont dû sentir.

Le ravissement de la découverte que j'ai si souvent senti rayonner de sa personne, s'associe immédiatement en moi à un semblable ravissement, dont il m'est arrivé d'être témoin chez un tout jeune enfant. Il y a deux souvenirs qui se pressent en moi - tous deux me font retrouver ma fille toute petite. Dans la première image, elle doit avoir quelques mois, ça devait être tout juste qu'elle commençait à faire du quatre-pattes. Elle avait dû se traîner du morceau de gazon où on l'avait assise vers une allée de graviers. Elle découvrait les petits graviers, dans une extase muette - et agissante, les empoignant à pleines mains pour les mettre à sa bouche!- Dans l'autre image elle devait avoir un an ou deux, quelqu'un venait de jeter des granulés dans un bocal de poissons rouges. 

Les poissons s'empressaient à qui mieux mieux de nager vers eux, la gueule grande ouverte, pour ingurgiter les minuscules miettes jaunes en suspension qui descendaient lentement dans l'eau du bocal. La petite ne s'était jamais rendue compte avant que les poissons mangeaient comme nous. C'était en elle comme un éblouissement soudain, s'exprimant en un cri de pur ravissement : ``Regarde maman, ils mangent!''. Il y avait de quoi s'émerveiller en effet - elle venait de découvrir en un éclair subit un grand mystère : celui de notre parenté à tous les autres êtres vivants\ldots

Il y a dans le ravissement d'un petit enfant une force communicative qui échappe aux mots, une force qui rayonne de lui et qui agit sur nous, alors que nous faisons de notre mieux, le plus souvent, pour nous y dérober. En les moments de silence intérieur, on sent cette force présente dans l'enfant à tout moment. En certains moments son action est plus forte seulement qu'en d'autres. C'est chez le nouveau-né, dans les premiers jours et mois de la vie, que cette sorte de ``champ de force'' autour de l'enfant est le plus puissant. Le plus souvent, il reste sensible tout au long de l'enfance, en s'effilochant au fil des ans jusqu'à l'adolescence, où souvent déjà il semble ne plus en rester trace. On peut le trouver pourtant rayonner autour de personnes de tout âge, en des moments privilégiés chez certains, ou chez de rares autres comme une sorte de haleine ou de halo qui entoure leur personne à toute heure. J'ai eu la grande chance de connaître une telle personne dans mon enfance, un homme, décédé maintenant.

Je songe aussi à cette autre force, ou puissance, que l'on sent parfois rayonner d'une femme, en les moments surtout où elle est épanouie en son corps, en communion avec lui. Le mot qui me vient souvent est ``beauté'', qui en évoque un aspect. C'est une beauté qui n'a rien à voir avec des canons de beauté ou de soi-disante ``perfection'', elle n'est pas le privilège d'une jeunesse, ou d'une maturité. Elle est le signe plutôt d'un accord profond en la personne. Cet accord reste fragmentaire souvent, et pourtant il se manifeste par ce rayonnement, signe d'une puissance. C'est une force qui nous attire vers le centre dont elle émane - ou plutôt, elle appelle en nous une pulsion profonde de retour dans le corps de la Femme-Mère dont nous sommes sortis, à l'aube de notre vie. Son action est d'une puissance parfois irrésistible, bouleversante quand elle émane de la femme aimée. Mais pour celui qui ne s'y ferme pas délibérément, elle est sensible en toute femme qui laisse s'épanouir en elle cette beauté, cet accord profond.

La force qui rayonne de l'enfant est proche parente de cette force qui émane de la femme qui s'aime en son corps. L'une constamment naît de l'autre, comme l'enfant constamment naît de la Mère. Mais la nature de la force de l'enfance n'est pas celle d'une attirance, pas plus que celle d'une répulsion. L'action humble et discrète que cette force exerce sur celui qui ne se dérobe pas à elle, est une action de renouvellement.

