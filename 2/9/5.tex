\section{(37) L'émerveillement}

Cette rétrospective sur la découverte de la méditation est venue là de façon entièrement imprévue, presque à mon corps défendant - ce n'était pas du tout ce que je me proposais d'examiner en commençant. J'avais envie de parler de l'émerveillement. Cette nuit si riche de tant de choses, a été riche aussi en émerveillement devant ces choses. Au cours du travail déjà, il y avait une sorte d'émerveillement incrédule devant chaque nouveau faux-fuyant mis à jour, comme un costume grossier cousu de gros fil blanc que je m'étais complu, c'était à peine croyable ! à prendre pour du vrai de vrai le plus sérieusement du monde ! Bien des fois encore depuis, dans les années qui ont suivi, j'ai retrouvé ce même émerveillement comme en cette première nuit de méditation, devant l'énormité des faits que je découvrais, et la grossièreté des subterfuges qui me les avaient fait ignorer jusque là. C'était par ses côtés burlesques d'abord que j'ai commencé à découvrir le monde insoupçonné que je porte en moi, un monde qui au fil des jours, des mois et des années s'est révélé d'une richesse prodigieuse. En cette première nuit déjà, pourtant, j'ai eu pour m'émerveiller d'autres sujets que des épisodes de vaudeville. C'est la nuit où pour la première fois j'ai repris contact avec un pouvoir oublié qui dormait en moi, dont la nature encore m'échappait, si ce n'est justement que c'est un pouvoir, et qui est à ma disposition à tout moment.

Et les mois précédents déjà avaient été riches d'un muet émerveillement d'une chose que je portais en moi, depuis toujours sûrement, avec laquelle je venais seulement de retrouver contact. Je ressentais cette chose non comme un pouvoir, mais bien plutôt comme une douceur secrète, comme une beauté à la fois très paisible et troublante. Plus tard, dans l'exultation de la découverte de mon pouvoir si longtemps ignoré, j'ai oublié ces mois de gestation silencieuse, dont témoignaient seulement quelques poèmes épars - des poèmes d'amour, qui peut-être auraient détoné le plus souvent au milieu de mes notes de méditation\ldots

C'est des années plus tard seulement que je me suis souvenu de ces temps d'émerveillement en la beauté du monde et en celle que je sentais reposer en moi. J'ai su alors que cette douceur et cette beauté que j'avais senties en moi, et ce pouvoir que j'ai découvert peu après qui a profondément changé ma vie, étaient deux aspects inséparables d'une seule et même chose.

Et je vois aussi, maintenant, que l'aspect doux, recueilli, silencieux de cette chose multiple qu'est la créati-vité en nous, s'exprime spontanément par l'émerveillement. Et c'est dans l'émerveillement aussi d'une indicible beauté en soi révélée par l'être aimé, que l'homme connaît la femme aimée et qu'elle le connaît. Quand l'émerveillement en la chose explorée ou en l'être aimé est absent, notre étreinte avec le monde est mutilée du meilleur qui est en elle - elle est mutilée de ce qui en fait une bénédiction pour soi et pour le monde. L'étreinte qui n'est un émerveillement est une étreinte sans force, simple reproduction d'un geste de possession. Elle est impuissante à engendrer autre chose que des reproductions encore, en plus grand ou plus gros ou plus épais peut-être, qu'importe, jamais un renouvellement\footnote{(34) L'étreinte impuissante

Le mot ``étreinte'' n'est nullement pour moi une simple métaphore, et la langue courante ici se fait le reflet d'une identité profonde. On pourra dira, non sans raison, qu'il n'est pas vrai alors que l'étreinte sans émerveillement est impuissante - que la terre serait dépeuplée sinon déserte, s'il en était ainsi au sens littéral. Le cas extrême est celui du viol, d'où l'émerveillement est certes absent, alors qu'il arrive qu'un être soit procréé en la femme violée. Sûrement l'enfant qui naît de telles étreintes ne peut, manquer d'en porter la marque, qui fera partie du ``paquet'' qu'il reçoit en partage et qu'il lui appartient d'assumer ; cela n'empêcher qu'un nouvel être est bel et bien conçu et est né. qu'il y a eu création, signe d'une puissance. Et il est vrai aussi qu'il arrive que tel mathématicien que j'ai pu voir empli de suffisance, trouve et prouve de beaux théorèmes, signes d'une étreinte qui n'a pas manqué de force ! Mais il est vrai également que si la vie de tel mathématicien est étouffée par sa suffisance (comme ce fut le cas dans une certaine mesure de ma propre vie, à une certaine époque), les fruits de ces étreintes avec la mathématique ne sont un bienfait pour lui ni pour personne. Et la même chose peut se dire du père comme de la mère de l'enfant issu d'un viol. Si je parle d' ``étreinte sans force'', j'entends avant tout l'impuissance à engendrer un renouvellement en celui qui croit : créer, alors qu'il ne crée qu'un produit, une chose extérieure à lui, sans résonance profonde en lui-même ; un produit qui, loin de le libérer, de créer une harmonie en lui, le lie plus étroitement à la fatuité en lui dont il est prisonnier, qui sans cesse le pousse à produire et reproduire. C'est là une forme d'impuissance à un niveau profond, derrière l'apparence d'une ``créativité'' qui n'est au fond qu'une productivité sans frein.

J'ai eu ample occasion aussi de me rendre compte que la suffisance, l'incapacité d'émerveillement, est dans la nature d'un véritable aveuglement, d'un blocage d'une sensibilité et d'un flair naturels ; blocage sinon total et permanent, du moins manifeste dans certaines situations d'espèce. C'est un état où tel mathématicien prestigieux se révèle parfois, dans les choses même où il excelle, aussi stupide que le plus buté des écoliers ! En d'autres occasions il fera des prodiges de virtuosité technique. Je doute pourtant qu'il soit encore en état de découvrir les choses simples et évidentes qui ont pouvoir de renouveler une discipline ou une science. Elles sont bien trop loin en-dessous de lui pour qu'il daigne encore les voir ! Pour voir ce que personne ne daigne voir, il faut une innocence qu'il a perdue, ou bannie\ldots Ce n'est pas un hasard sûrement, avec l'accroissement prodigieux de la production mathématique en l'espace de ces vingt dernières années, et la profusion déroutante des résultats nouveaux dont se voit submergé le mathématicien qui voudrait simplement ``se tenir au courant'' tant soit peu, qu'il n'y a guère eu pourtant (pour autant que je puisse en juger par les échos qui me parviennent ici et là) de renouvellement véritable, de transformation de vaste envergure (et non seulement par accumulation) d'aucun des grands thèmes de réflexion dont j'ai été tant soit peu familier. Le renouvellement n'est pas chose quantitative, elle est étrangère à une quantité d'investissement, mesurable en un nombre de jours-mathématiciens consacré à tel sujet par tels mathématiciens de tel ``niveau''. Un million de jours-mathématiciens est impuissant à donner naissance à une chose aussi enfantine que le zéro, qui a renouvelé notre perception du nombre. Seule l'innocence a cette puissance, dont un signe visible est l'émerveillement\ldots}(34). C'est quand nous sommes enfants et prêts à nous émerveiller en la beauté des choses du monde et en nous-mêmes, que nous sommes prêts aussi à nous renouveler, et prêts comme instruments souples et dociles entre les mains de l' Ouvrier, pour que par Ses mains et à travers nous des êtres et des choses peut-être se renouvellent.

Je me rappelle bien que dans ce groupe d'amis sans façons qui pour moi représentait le milieu mathématique, à la fin des années quarante et dans les années suivantes, milieu parfois bruyant et sûr de lui, où le ton un peu péremptoire n'était pas si rare (mais sans qu'il s'y glisse pourtant une suffisance) - dans ce milieu il y avait place à tout moment pour l'émerveillement. Celui en qui l'émerveillement était le plus visible était Dieudonné. Que ce soit lui qui fasse un exposé, ou qu'il soit simplement auditeur, quand arrivait le moment crucial où une échappée soudain s'ouvrait, on voyait Dieudonné aux anges, radieux. C'était l'émerveillement à l'état pur, communicatif, irrésistible - où toute trace du ``moi'' avait disparu. Au moment où je l'évoque maintenant, je me rends compte que cet émerveillement par lui-même était une puissance, qu'il exerçait une action immédiate tout autour de sa personne, comme un rayonnement dont il était la source. Si j'ai vu un mathématicien faire usage d'un puissant et élémentaire ``pouvoir d'encouragement'', c'est bien lui ! Je n'y ai jamais re-songé avant cet instant, mais je me souviens maintenant que c'est dans ces dispositions aussi qu'il avait accueilli déjà mes tout premiers résultats à Nancy, résolvant des questions qu'il avait posées avec Schwartz (sur les espaces (\(F\)) et (\(LF\)). C'étaient des résultats tout modestes, rien de génial ni d'extraordinaire certes, on pourrait dire qu'il n'y avait pas de quoi s'émerveiller. J'ai vu depuis des choses de toute autre envergure rejeté par le dédain sans réplique de collègues qui se prennent pour de grands mathématiciens. Dieudonné n'était nullement encombré de semblable prétention, justifiée ou non. Il n'y avait rien de ce genre qui l'empêchait d'être ravi même par les petites choses.

Il y a dans cette capacité de ravissement une générosité, qui est un bienfait pour celui qui veut bien la laisser s'épanouir en lui, comme pour son entourage. Ce bienfait s'exerce sans intention d'être agréable à qui que ce soit. Il est simple comme le parfum d'une fleur, comme la chaleur du soleil.

De tous les mathématiciens que j'ai connus, c'est en Dieudonné que ce ``don'' m'est apparu de la façon la plus éclatante, la plus communicative, la plus agissante aussi peut-être, je ne saurais dire\footnote{(35) \par Ce ``don'' n'est le privilège de personne, nous sommes tous nés avec. Quand il semble absent en moi, c'est que je l'ai moi-même chassé, et il ne tient qu'à moi de l'accueillir à nouveau. Chez moi ou chez un tel, ce ``don'' s'exprime de façon, différente que chez tel autre, de façon moins communicative, moins irrésistible peut-être, mais il n'en est pas moins présent, et je ne saurais dire s'il est moins agissant.}(35). Mais en aucun des amis mathématiciens que j'ai aimé fréquenter, ce don-là n'était absent. Il trouvait occasion à se manifester, de façon peut-être plus retenue, à tout moment. Il se manifestait à chaque fois que je venais vers l'un d'eux pour lui faire partager une chose que je venais de trouver et qui m'avait enchantée.

Si j'ai connu des frustrations et des peines dans ma vie de mathématicien, c'est avant tout de ne pas retrouver, en certains de ceux que j'ai aimés, cette générosité que j'avais connue en eux, cette sensibilité à la beauté des choses, ``petites'' ou ``grandes'' ; comme si ce qui avait fait la vie frémissante de leur être s'était éteint sans laisser de trace, étouffé par la suffisance de celui pour qui le monde n'est plus assez beau pour qu'il daigne s'en réjouir.

Il y a eu aussi, certes, cette autre peine, de voir tel de mes amis d'antan traiter avec condescendance ou avec mépris tel de mes amis d'aujourd'hui. Mais cette peine est infligée par la même fermeture, au fond. Celui qui est ouvert à la beauté d'une chose, si humble soit-elle, quand il a senti cette beauté, ne peut s'empêcher de sentir aussi un respect pour celui qui l'a conçue ou faite. Dans la beauté d'une chose faite par la main de l'homme, nous sentons le reflet d'une beauté en celui qui l'a faite, de l'amour qu'il a mis à la faire. Quand nous sentons cette beauté, cet amour, il ne peut y avoir en nous condescendance ou dédain, pas plus qu'il ne peut y avoir condescendance ou dédain pour une femme, en un moment où nous sentons sa beauté, et la puissance en elle dont cette beauté est le signe.




