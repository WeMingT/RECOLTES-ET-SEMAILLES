\section{(34) Le limon et la source}

Il me semble que pour l'essentiel, j'ai fait le tour de ce qu'ont été mes relations à d'autres mathématiciens de tous âges et de tous rangs, du temps où je faisais partie de leur monde, du monde des mathématiciens ; et en même temps et surtout, de la part que j'ai prise, par mes propres attitudes et comportements, à un certain esprit que j'y constate aujourd'hui, et qui sûrement n'est pas d'hier. Au cours de cette réflexion, ou de ce voyage pour mieux dire, j'ai rencontré à quatre reprises des situations, qui me sont apparues comme typiques de certaines attitudes et ambiguïtés en ma personne, où des dispositions spontanées de bienveillance et de respect vis-à-vis d'autrui ont été perturbées, sinon totalement balayées, par des forces égotiques, et surtout (dans trois de ces cas tout au moins) par une fatuité. Cette fatuité se prévalait surtout de la soi-disante supériorité que m'aurait conféré une certaine puissance cérébrale, et l'investissement démesuré que je faisais en mon activité mathématique. Elle trouvait confirmation et appui dans un consensus général qui valorisait, pratiquement sans réserve aucune, cette puissance cérébrale et cet investissement démesuré.

C'est la dernière des situations examinées, celle du ``jeune malappris qui marchait sur mes plates-bandes'', qui me semble la plus importante des quatre pour mon propos actuel. Les trois premières sont typiques de ma personne, ou de certains aspects de ma personne, à une certaine époque (dans un certain contexte aussi, il est vrai) - mais, comme j'ai eu l'occasion de le dire et répéter, je ne les considère nullement typiques pour le milieu dont je faisais partie. Je ne crois pas non plus qu'ils soient typiques du milieu mathématique actuel en France, disons - il est probable que l'espèce d'égarement chronique qui a caractérisé la relation que j'avais avec ``l'ami infatigable'', par exemple, soit chose peu commune de nos jours comme ça devait l'être alors. Mon attitude et comportement dans le cas du ``jeune malappris'', par contre, est typique de ce qui se passe journellement aujourd'hui même dans le monde mathématique, où qu'on regarde. C'est l'attitude de bienveillance, de respect du mathématicien influent vis-à-vis du jeune inconnu qui devient là rarissime exception, quand ledit inconnu n'a pas l'heur d'être son élève (et encore\ldots), ou élève d'un collègue d'un statut comparable et recommandé par lui. C'est sans doute ce qui me revenait déjà dès les lendemains de mon ``réveil'' de 1970, qui avait délié des langues muettes - mais les témoignages de première main que j'entendais alors restaient pour moi lointains, car ils ne concernaient directement ni ma personne, ni celle des amis qui m'étaient les plus chers dans mon milieu. J'ai été touché plus que superficiellement à partir du moment (vers l'année 1976) où les échos qui me revenaient, ou les faits dont j'étais témoin, avaient pour protagonistes certains de ces amis, voire des ex-élèves devenus importants, et plus encore lorsque ceux qui étaient en butte à une malveillance étaient des personnes que je connaissais bien, des élèves plus d'une fois (élèves d'``après 1970'', il va sans dire !), dont le sort donc me touchait. Dans certains cas, il ne faisait de plus aucun doute que le manque de bienveillance, voire une attitude de mépris ostentatif, étaient renforcés pour le moins, sinon suscités, par le seul fait que tel jeune chercheur était mon élève, ou qu'il prenait le risque (sans être mon élève nécessairement) de faire ce que mes amis d'antan et d'autres collègues aussi appellent volontiers des ``Grothendieckeries''\ldots

Le ``jeune malappris'' m'a encore écrit au début des années 70, pour me demander très courtoisement (alors qu'il n'était nullement tenu de rien me demander du tout !) si je ne voyais pas d'inconvénient qu'il publie une démonstration qu'il avait trouvée pour un théorème dont on lui avait dit que j'étais l'auteur, et qui n'avait jamais été publié. Je me rappelle que je lui ai répondu dans les mêmes dispositions de mauvaise humeur que dans le passé, sans dire oui ni non je crois et en laissant entendre, sans connaître sa démonstration (qu'il était prêt bien sûr à me communiquer mais dont je n'avais cure, occupé que j'étais par mes tâches militantes !), que celle-ci n'apporterait sûrement rien à la mienne (pourtant, elle aurait apporté pour tout le moins d'être écrite noir sur blanc et disponible au public mathématique, ainsi que l'énoncé lui-même !). Cela montre bien à quel point ce fameux ``réveil'' restait encore superficiel, sans aucune incidence sur certains comportements enracinés dans une fatuité et dans des attitudes ``méritocratiques'', que j'étais sûrement en train de dénoncer au même moment dans des articles bien sentis de Survivre et Vivre, dans des interventions en débats publiques, etc\ldots

Cela répond de façon bien concrète à une question que j'avais laissée en suspens précédemment. Autant admettre ici cette humble vérité, que de telles attitudes de fatuité ne sont nullement surmontées ``une fois pour toutes'' dans ma personne, et je doute qu'elles le soient un jour si ce n'est à ma mort. S'il y a eu transformation, ce n'est pas par la disparition d'une vanité, mais par l'apparition (ou la réapparition) d'une curiosité à l'égard de ma propre personne et de la nature véritable de certaines attitudes, comportements etc\ldots chez moi. C'est par cette curiosité que je suis devenu tant soit peu sensible aux manifestations de la vanité en moi. Cela modifie profondément une certaine dynamique intérieure, et modifie par là-même les effets de la ``vanité'' ; c'est-à-dire de cette force qui souvent me pousse à escamoter ou à contrefaire la saine et fine perception que j'ai de la réalité, aux fins d'agrandir ma personne et me mettre au-dessus d'autrui tout en prétendant le contraire.

Peut-être tel lecteur se sentira-t-il dérouté, comme je le fus moi-même un jour, devant la contradiction apparente entre la présence insidieuse et tenace de la vanité dans ma vie de mathématicien (qu'il aura peut-être aussi par moments entrevue dans la sienne), et ce que j'appelle mon amour, ou ma passion, pour la mathématique (qui peut-être éveille également un écho dans sa propre expérience de la mathématique, ou de quelque autre personne ou chose). S'il est dérouté en effet, il a en lui tout ce qu'il faut pour reprendre contact (comme je l'ai fait naguère) avec la réalité des choses elles-mêmes, qu'il peut connaître de première main, plutôt que de tourner comme un écureuil prisonnier dans une cage sans fin de mots et de concepts.

Celui qui voit une eau bourbeuse dira-t-il que l'eau et la boue sont une seule et même chose ? Pour connaître l'eau qui n'est pas boue il suffit de monter à la source et regarder et boire. Pour connaître la boue qui n'est pas eau, il suffit de monter sur la berge séchée par le soleil et le vent, et détacher et égrener dans sa main une boule d'argile grenue. L'ambition, la vanité peuvent régler peu ou prou la part qu'on fait dans sa vie à telle passion, comme la passion mathématique, peuvent la rendre dévorante, si les retours les comblent. Mais l'ambition la plus dévorante est impuissante par elle-même à découvrir ou à connaître la moindre des choses, bien au contraire ! Au moment du travail, quand peu à peu une compréhension s'amorce, prend forme, s'approfondit ; quand dans une confusion peu à peu on voit apparaître un ordre, ou quand ce qui semblait familier soudain prend des aspects insolites, puis troublants, jusqu'à ce qu'une contradiction enfin éclate et bouleverse une vision des choses qui paraissait immuable - dans un tel travail, il n'y a trace d'ambition, ou de vanité. Ce qui mène alors la danse est quelque chose qui vient de beaucoup plus loin que le ``moi'' et sa fringale de s'agrandir sans cesse (fut-ce de ``savoir'' et de ``connaissances'') - de beaucoup plus loin sûrement que notre personne ou même notre espèce.

C'est là la source, qui est en chacun de nous.