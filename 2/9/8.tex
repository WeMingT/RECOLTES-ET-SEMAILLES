\section{(40) La mathématique sportive}

Il est assez clair que l'ouverture à la beauté des choses mathématiques n'a jamais entièrement disparu en moi, même en les années soixante jusqu'en 1970, où la fatuité a pris progressivement une place grandissante dans ma relation à la mathématique et aux autres mathématiciens. Sans un minimum d'ouverture à la beauté des choses, j'aurais été bien incapable de ``fonctionner'' comme mathématicien, même à un régime des plus modestes - et je doute que quiconque puisse faire travail utile en mathématiques, s'il ne reste vivant en lui, tant soit peu, ce sens de la beauté. Ce n'est pas tant, me semble-t-il, une prétendue ``puissance cérébrale'' qui fait la différence entre tel mathématicien et tel autre, ou entre tel travail et tel autre du même mathématicien; mais plutôt la qualité de finesse, de délicatesse plus ou moins grande de cette ouverture ou sensibilité, d'un chercheur à un autre ou d'un moment à l'autre chez le même chercheur. Le travail le plus profond, le plus fécond est celui aussi qui atteste de la sensibilité la plus déliée pour appréhender la beauté cachée des choses\footnote{(36) \par Une telle sensibilité délicate à la beauté me semble intimement liée à une chose dont j'ai eu occasion de parler sous le nom de ``exigence'' (vis-à-vis de soi) ou de ``rigueur'' (au plein sens du terme), que je décrivais comme une ``attention à quelque chose de délicat en nous-mêmes'', une attention à une qualité de compréhension de la chose sondée. Cette qualité de compréhension d'une chose mathématique ne peut être séparée d'une perception plus ou moins intime, plus ou moins parfaite de la ``beauté'' particulière à cette chose.}(36).

S'il en est ainsi, il faut croire que cette sensibilité a dû rester vive en moi jusqu'à la fin, par moments tout au moins, puisque c'est à la fin des années soixante\footnote{(8 août) Vérification faite, il apparaît que les débuts de ma réflexion sur les motifs se placent aux débuts, non à la fin des années soixante.} que j'ai commencé à entrevoir et à dégager tant soit peu la chose mathématique la plus cachée, la plus mystérieuse qu'il m'ait été donné de découvrir - cette chose que j'ai nommée ``motif''. C'est celle aussi qui a exercé la plus grande fascination sur moi dans ma vie de mathématicien (si j'excepte certaines réflexions des toutes dernières années, d'ailleurs intimement liées à la réalité des motifs). Nul doute que si ma vie tout à coup n'avait pris un cours entièrement imprévu, m'entraînant bien loin hors du monde serein des choses mathématiques, j'aurais fini par suivre l'appel de cette fascination puissante, laissant là les ``tâches'' qui m'avaient jusque là maintenues prisonnières!

Peut-être puis-je dire que dans la solitude de ma chambre de travail, le sens de la beauté est resté égal à lui-même jusqu'au moment de mon premier ``réveil'' en 1970, sans être affecté vraiment par la fatuité qui marquait si souvent les relations à mes congénères? Un certain ``flair'' a même dû s'affiner avec les années, au contact journalier et intime avec les choses mathématiques. La connaissance intime que nous pouvons avoir des choses, qui parfois nous permet d'appréhender au-delà de ce que nous connaissons dans l'instant et pénétrer plus avant dans la connaissance - cette connaissance ou cette maturité, et ce ``flair'' qui en est le signe le plus visible, est proche parente de l'ouverture à la beauté et à la vérité des choses. Elle favorise, elle stimule une telle ouverture, et elle est somme et fruit de tous les moments d'ouverture, de tous les ``moments de vérité'' qui ont précédé.

Ce qu'il me reste donc à examiner, c'est dans quelle mesure une sensibilité spontanée à la beauté a été perturbée plus ou moins profondément, aux moments où elle avait eu occasion de se manifester dans ma relation à tel ou tel collègue.

Ce que me livre la mémoire à ce sujet ne se condense pas en un fait tangible et précis, que je pourrais ici rapporter de façon plus ou moins circonstanciée. Le souvenir ici encore se borne à une sorte de brouillard, qui me livre pourtant une impression d'ensemble, qu'il me faut essayer de cerner. C'est l'impression qu'a laissée en moi une certaine attitude intérieure, qui a dû finir par devenir comme une seconde nature, et qui se manifestait chaque fois que je recevais une information mathématique sur quelque chose qui était plus ou moins ``dans mes cordes''. A vrai dire, par un certain aspect relativement anodin, cette attitude a dû être mienne de tout temps, elle fait partie d'un certain tempérament, et j'ai eu l'occasion de l'effleurer en passant. Il s'agit de ce réflexe, de ne consentir d'abord à prendre connaissance que d'un énoncé, jamais de sa démonstration, pour essayer tout d'abord de le situer dans ce qui m'est connu, et de voir si en termes de ce connu l'énoncé devient transparent, évident. Souvent cela m'amène à reformuler l'énoncé de façon plus ou moins profonde, dans le sens d'une plus grande généralité ou d'une plus grande précision, souvent aussi les deux à la fois. C'est seulement lorsque je n'arrive pas à ``caser'' l'énoncé en termes de mon expérience et de mes images, que je suis prêt (presque à mon corps défendant parfois!) à écouter (ou lire\ldots) les tenants et aboutissants qui parfois donnent ``la'' raison de la chose, ou tout au moins une démonstration, comprise ou non.

C'est là une particularité de mon approche de la mathématique, qui me distinguait, il me semble, de tous les autres membres de Bourbaki au temps où je faisais partie du groupe, et qui me rendait pratiquement impossible de m'insérer comme eux dans un travail collectif. Cette particularité a sûrement constitué aussi un handicap dans mon activité d'enseignant, handicap qui a dû être ressenti par tous mes élèves jusqu'à aujourd'hui où (l'âge aidant) elle a fini par s'assouplir quelque peu.

Ce trait en moi est sûrement déjà dans le sens d'un défaut d'ouverture. Elle implique une ouverture partielle seulement, prête à accueillir uniquement ce qui ``vient à point'', ou du moins très réticente dans l'accueil de tout le reste. Dans le choix de mes investissements mathématiques, et du temps que je consens à consacrer à telles informations imprévues ou telles autres, ce propos délibéré de \textbf{``fermeture partielle''} est aujourd'hui plus fort que jamais. Elle est même une nécessité, si je veux pouvoir suivre l'appel de ce qui me fascine le plus, sans pour autant donner encore ``ma vie à dévorer'' à dame mathématique!

Le ``brouillard'' pourtant me restitue plus que cette particularité, dont j'ai fini par me rendre compte depuis quelques années déjà (mieux vaut tard que jamais !). A un certain moment, ce réflexe est devenu comme un point d'honneur ; ce serait bien du diable si je n'arrive à ``avoir'' cet énoncé (à supposer qu'il ne m'était déjà bien familier) en moins de temps qu'il n'en faut pour le dire! Si c'était un illustre inconnu qui était auteur de l'énoncé, il y avait en plus cette nuance : il ne manquerait plus que ça, que moi (qui suis censé être dans le coup, après tout!) n'aie pas déjà tout ça dans mes manches! Et bien souvent en effet je l'avais, et au delà mon attitude alors aurait eu tendance alors d'aller dans le sens : ``Bon, vous pouvez aller vous rhabiller - vous reviendrez quand vous aurez fait un peu mieux !''.

C'était justement là mon attitude dans le cas du ``jeune blanc-bec qui marchait dans mes plates-bandes''. Je ne saurais même pas jurer que dans ce qu'il faisait, il n'y avait pas des détails intéressants qui n'étaient pas couverts par ce que j'avais fait dans mes ``notes secrètes'' - c'est là chose accessoire \footnote{(8 août) Il m'est apparu depuis que cette chose n'est pas si ``accessoire'' que ça, qu'elle constitue la ligne de passage de ``l'attitude sportive'' à un début de malhonnêteté, ligne qu'il m'est peut-être arrivé de franchir\ldots} d'ailleurs. Finalement, cet épisode éclaire également la question que j'examine ici ; celle d'une perturbation profonde de cette ouverture à la beauté des choses mathématiques. On aurait dit qu'à partir du moment où j'avais ``fait'' telle chose, sa beauté était disparue pour moi, et qu'il ne restait qu'une vanité qui en réclamait crédit et bénéfice. (Sans que je daigne pourtant prendre le temps de le publier - il est vrai qu'il y en aurait eu trop.) C'était une attitude typique de possession, analogue à celle d'un homme qui, ayant connu une femme, ne sent plus sa beauté et court cent autres sans souffrir pour autant qu'un autre la connaisse. C'était là une attitude que je réprouvais dans la vie amoureuse, me croyant loin au-dessus d'une telle vanité, tout en me gardant bien de constater ce fait évident, que c'était bel et bien là mon attitude vis à vis de la mathématique!

J'ai comme une impression que ces dispositions grossières de compétition, des dispositions ``sportives'' si on peut dire, sur lesquelles je viens de mettre le doigt dans ma personne, devaient commencer à devenir courantes dans ``mon'' milieu mathématique, vers le moment où elles étaient courantes en moi. Je serais bien en peine de situer dans le temps le moment de leur apparition, ou celui où elles sont devenues comme une partie intime de l'air qu'on respirait dans ce milieu, ou celui que mes élèves respiraient au contact de ma personne. La seule chose que je crois pouvoir dire, c'est que cela doit se placer dans les années soixante, peut être dès les débuts des années soixante, ou la fin des années cinquante. (S'il en est ainsi, tous mes élèves y ont eu droit - c'était pour eux à prendre ou à laisser !) Pour pouvoir le situer, il me faudrait d'autres cas précis, qui en ce moment échappent totalement à mon souvenir.

Cette humble réalité était bien entendu en contraste complet avec la noble image que je me faisais de ma relation aux mathématiques, et aux jeune chercheurs en général. Le subterfuge grossier qui m'a servi à me berner moi-même, était d'inspiration méritocratique : pour cette image, tout ce que je retenais, c'était la relation à mes élèves (lesquels contribuaient à mon prestige, dont ils étaient les plus nobles fleurons !), et aux jeunes mathématiciens particulièrement brillants, dont j'avais su reconnaître les mérites et que je traitais sur un pied d'égalité tout comme mes élèves, sans attendre que leur tête soit couronnée de lauriers (ce qui bien sûr n'a pas tardé - on a le ``flair'' ou on ne l'a pas !). Quand aux jeunes qui n'avaient l'heur ni d'être parmi mes élèves, ou parmi ceux d'un de mes amis, ni d'être de jeunes génies, je ne me préoccupais nullement quelle était ma relation à eux. Ils ne comptaient pas.

Je crois que cette réalité-là était le plus souvent assouplie, tempérée, quand je me trouvais mis en relation personnelle avec le jeune chercheur, soit que je le rencontrais à mon séminaire, soit qu'il s'était adressé à moi par lettre. Il se peut que le cas du ``jeune blanc-bec'' soit de ce point de vue un cas un peu à part, exceptionnel. Il me semble que pour les chercheurs dont je viens de parler, je devais les considérer un peu comme s'étant mis ``sous ma protection'', et cela devait réveiller en moi une attitude plus bienveillante. Dans ce cas aussi, mon désir de me mettre en avant pouvait trouver un exutoire, en faisant mes commentaires à l'intéressé et en lui faisant des suggestions pour reprendre son travail dans une optique peut-être plus vaste, ou en allant plus au fond des choses. Dans un tel cas, il y a des chances que le jeune chercheur, qui pour un temps limité prenait un peu figure d'élève, y trouvait lui aussi son compte, et qu'il gardait un bon souvenir de sa relation à moi. (Tout écho dans un sens ou dans l'autre qui me parviendrait à ce sujet serait bienvenu.)

J'ai pensé ici surtout au cas de chercheurs plus jeunes, alors que l'attitude ``sportive'' n'était nullement limitée à ma relation à ceux-ci, il va sans dire. Mais c'est dans la relation aux jeunes chercheurs, sûrement, que l'impact aussi bien psychologique que pratique d'un mathématicien en vue a tendance à être le plus fort, le plus chargé de conséquence pour leur future vie professionnelle.
