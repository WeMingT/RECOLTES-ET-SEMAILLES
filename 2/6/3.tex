\section{(7) L'héritage de Galois}

Il semblerait que parmi toutes les sciences naturelles, ce n'est qu'en mathématiques que ce que j'ai appelé "le rêve", ou "le rêve éveillé", est frappé d'un interdit apparemment absolu, plus que deux fois millénaire. Dans les autres sciences, y compris des sciences réputées "exactes" comme la physique, le rêve est pour le moins toléré, voire encouragé (selon les époques), sous des noms il est vrai plus "sortables" comme : "spéculations", "hypothèses" (telle la fameuse "hypothèse atomique", issue d'un rêve, pardon d'une spéculation de Démocrite), "théories"...Le passage du statut du rêve-qui-n'ose-dire-son-nom à celui de "vérité scientifique" se fait par degrés insensibles, par un consensus qui s'élargit progressivement. En mathématiques par contre, il s'agit presque toujours (de nos jours du moins) d'une transformation subite, par la vertu du coup de baguette magique d'une démonstration \footnote{(4)\par Même de nos jours d'ailleurs, on rencontre des "démonstrations" au statut incertain. Il en a été ainsi pendant des années de la démonstration par Grauert du théorème de finitude qui porte son nom, que personne (et les bonnes volontés n'ont pas manqué !) n'arrivait à lire. Cette perplexité a été résolue par d'autres démonstrations plus transparentes, et dont certaines allaient plus loin, qui ont pris la succession de la démonstration initiale. Une situation similaire, plus extrême, est la "solution" du problème dit "des quatre couleurs", dont la partie calculatoire a été réglée à coups d'ordinateur (et de quelques millions de dollars). Il s'agit donc là d'une "démonstration" qui ne se trouve plus fondée dans l'intime conviction provenant de la compréhension d'une situation mathématique, mais dans le crédit qu'on fait à une machine dénuée de la faculté de comprendre, et dont l'utilisateur mathématicien ignore la structure et le fonctionnement. A supposer même que le calcul soit confirmé par d'autres ordinateurs, suivant d'autres programmes de calcul, je ne considère pas pour autant que le problème des quatre couleurs soit clos. Il aura seulement changé de visage, en ce sens qu'il ne s'agit plus guère de chercher un contre-exemple, mais seulement une démonstration (lisible, il va de soi !).} (4). Aux temps où la notion de définition mathématique et de démonstration n'était pas, comme aujourd'hui, claire et objet d'un consensus (plus ou moins) général, il y avait pourtant des notions visiblement importantes qui avaient une existence ambiguë - comme celle de nombre "négatif" (rejetée par Pascal) ou celle de nombre "imaginaire". Cette ambiguïté se reflète dans le langage en usage encore aujourd'hui.

La clarification progressive des notions de définition, d'énoncé, de démonstration, de théorie mathématique, a été à cet égard très salutaire. Elle nous a fait prendre conscience de toute la puissance des outils, d'une simplicité enfantine pourtant, dont nous disposons pour formuler avec une précision parfaite cela même qui pouvait sembler informulable - par la seule vertu d'un usage suffisamment rigoureux du langage courant, à peu de choses près. S'il y a une chose qui m'a fasciné dans les mathématiques depuis mon enfance, c'est justement cette puissance à cerner par des mots, et à exprimer de façon parfaite, l'essence de telles choses mathématiques qui au premier abord se présentent sous une forme si élusive, ou si mystérieuse, qu'elles paraissent au-delà des mots...

Un contrecoup psychologique fâcheux pourtant de cette puissance, des ressources qu'offre la précision parfaite et la démonstration, c'est qu'elles ont accentué encore la tabou traditionnel à l'égard du "rêve mathématique" ; c'est-à-dire à l'égard de tout ce qui ne se présenterait pas sous les aspects conventionnels de précision (fût-ce aux dépens d'une vision plus vaste), garantie "bon teint" par des démonstrations en forme, ou sinon (et de plus en plus par les temps qui courent...) par des esquisses de démonstration, censées pouvoir se mettre en forme. Des conjectures occasionnelles sont tolérées à la rigueur, à condition qu'elles satisfassent aux conditions de précision d'un questionnaire, où les seules réponses admises seraient "oui" ou "non". (Et à condition de plus, est-il besoin de le dire, que celui qui se permet de la faire ait pignon sur rue dans le monde mathématique.) A ma connaissance, il n'y a pas eu d'exemple du développement, à titre "expérimental", d'une théorie mathématique qui serait explicitement conjecturale dans ses parties essentielles. Il est vrai que suivant les canons modernes, tout le calcul des "infiniment petits" développé à partir du dix-septième siècle, devenu depuis le calcul différentiel et intégral, prendrait figure de rêve éveillé, qui se serait transformé finalement en mathématiques sérieuses deux siècles plus tard seulement, par le coup de baguette magique de Cauchy. Et cela me remet en mémoire forcément le rêve éveillé d'Evariste Galois, lequel n'a pas eu de chance avec ce même Cauchy ; mais il a suffi cette fois de moins de cent ans pour qu'un autre coup de baguette, de Jordan cette fois (si je me rappelle bien), donne droit de cité à ce rêve, rebaptisé pour la circonstance "théorie de Galois".

La constatation qui se dégage de tout cela, et qui n'est pas à l'avantage des "mathématiques 1984", c'est qu'il est heureux que des gens comme Newton, Leibnitz, Galois (et j'en passe sûrement beaucoup, n'étant pas calé en histoire...) n'aient pas été encombrés de nos canons actuels, en un temps où ils se contentaient de découvrir sans prendre le loisir de canonifier!

L'exemple de Galois, venu là sans que je l'appelle, touche en moi une corde sensible. Il me semble me rappeler qu'un sentiment de sympathie fraternelle à son égard s'est éveillé dès la première fois où j'ai entendu parler de lui et de son étrange destin, aux temps où j'étais encore lycéen ou étudiant, je crois. Comme lui, je sentais en moi une passion pour la mathématique - et comme lui je me sentais un marginal, un étranger dans le "beau monde" qui (me semblait-il) l'avait rejeté. J'ai fini pourtant moi-même par faire partie de ce beau monde, pour le quitter un jour, sans regret... Cette affinité un peu oubliée m'est réapparue tout dernièrement et sous un jour tout nouveau, alors que j'écrivais l' "Esquisse d'un Programme" (à l'occasion de ma demande d'admission comme chercheur au Centre National de la Recherche Scientifique). Ce rapport est consacré principalement à une esquisse de mes principaux thèmes de réflexion depuis une dizaine d'années. De tous ces thèmes, celui qui me fascine le plus, et que je compte développer surtout dans les prochaines années, est le type même d'un rêve mathématique, qui rejoint d'ailleurs le "rêve des motifs", dont il fournit une approche nouvelle. En écrivant cette Esquisse, je me suis souvenu de la réflexion mathématique la plus longue que j'aie poursuivie d'une traite en ces dernières quatorze années. Elle s'est poursuivie de janvier à juin 1981, et je l'ai nommée "La longue Marche à travers la théorie de Galois". De fil en aiguille, j'ai pris conscience que le rêve éveillé que je poursuivais sporadiquement depuis quelques années, qui avait fini par prendre le nom de "géométrie algébrique anabélienne", n'était autre qu'une continuation, "un aboutissement ultime de la théorie de Galois, et dans l'esprit sans doute de Galois".

Quand m'est apparu cette continuité, au moment d'écrire le passage dont est extraite la ligne citée, une joie m'a traversé, qui ne s'est pas dissipée. Elle a été une des récompenses d'un travail poursuivi dans une solitude complète. Son apparition a été aussi inattendue que l'accueil plus que frais reçu naguère auprès de deux ou trois collègues et anciens amis pourtant bien "dans le coup", dont l'un d'ailleurs fut mon élève, auxquels j'avais eu l'occasion de parler, "à chaud" encore et dans la joie de mon cœur, de ces choses que j'étais en train de découvrir...

Cela me rappelle que reprendre aujourd'hui l'héritage de Galois, c'est sûrement aussi accepter le risque de la solitude qui a été sienne en son temps. Peut-être les temps changent-ils moins que nous ne le pensons, souvent ce "risque" pourtant ne prend pas pour moi figure de menace. S'il m'arrive d'être peiné et frustré par l'affectation d'indifférence ou de dédain de ceux que j'ai aimés, jamais par contre depuis de longues années la solitude, mathématique ou autre, ne m'a-t-elle pesé. S'il est une amie fidèle que sans cesse j'aspire à retrouver quand je viens à la quitter, c'est elle !
