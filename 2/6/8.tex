\section{(12) Le mérite et le mépris}

Je voudrais examiner de plus près, à la lumière de ma propre expérience limitée, quand et comment le mépris s'est installé dans le monde des mathématiciens, et plus particulièrement dans ce "microcosme" de collègues, amis et élèves qui était devenu comme ma seconde patrie. Et en même temps, voir quelle a été ma part dans cette transformation.

Il me semble pouvoir dire, sans réserve aucune, que je n'ai pas rencontré en 1948-49, dans le cercle de mathématiciens dont j'ai parlé précédemment (dont le centre pour moi était le groupe Bourbaki initial), la moindre trace de mépris, ou simplement de dédain, de condescendance, vis à vis de moi-même ou d'aucun autre des jeunes gens, français ou étrangers, venus là pour apprendre le métier de mathématicien. Les hommes qui y jouaient un rôle de figure de proue, par leur position ou leur prestige, tels Leray, Cartan, Weil, n'étaient pas craints par moi, ni je crois par aucun de mes camarades. Mis à part Leray et Cartan, qui faisaient très "messieurs distingués", il m'a fallu même un bon moment avant de réaliser que chacun de ces lurons qui débarquaient là sans façons en tutoyant Cartan comme un copain et visiblement "dans le coup", était professeur d’Université tout comme Cartan lui-même, ne vivait nullement comme moi de la main à la bouche mais touchait des émoluments pour moi astronomiques, et était de surcroît un mathématicien d'envergure et d'audience internationale.

Suivant une suggestion de Weil, j'ai passé les trois années suivantes à Nancy, qui à ce moment était un peu le quartier général de Bourbaki, avec Delsarte, Dieudonné, Schwartz, Godement (et un peu plus tard aussi Serre) y enseignant à l'Université. Il n'y avait là avec moi qu'une poignée de quatre ou cinq jeunes gens (parmi lesquels je me rappelle de Lions, Malgrange, Bruhat, Berger, sauf confusion), donc on y était nettement moins "noyé dans le tas" qu'à Paris. L'ambiance était d'autant plus familière, tout le monde se connaissait personnellement, et on se tutoyait tous je crois. Quand je fouille mon souvenir, c'est là pourtant que se situe le premier et seul cas où j'ai vu devant moi un mathématicien traiter un élève avec un mépris non déguisé. Le malheureux était venu pour la journée, d'une autre ville, pour travailler avec son patron. (Il devait préparer une thèse de doctorat, qu'il a d'ailleurs fini par passer honorablement, et il a acquis depuis une certaine notoriété, je crois.) J'étais assez soufflé de la scène. Si quelqu'un s'était permis un tel ton avec moi ne fut-ce qu'une seconde, je lui aurais claqué la porte au nez aussi sec ! En l'occurrence, je connaissais bien le "patron", j'étais même à tu et à toi avec lui, non l'élève que je connaissais de vue seulement. Mon aîné avait, en plus d'une culture étendue (non seulement mathématique) et d'un esprit incisif, une sorte d'autorité péremptoire qui à ce moment (et pendant assez longtemps après encore, jusqu'dans les débuts des années 70) m'impressionnait. Il exerçait un certain ascendant sur moi. Je ne me rappelle pas si je lui ai posé une question au sujet de son attitude, seulement la conclusion que je retirais de la scène : c'est que vraiment ce malheureux élève devait être bien nul, pour mériter d'être traité de cette façon - quelque chose comme ça. Je ne me suis pas dit alors que si l'élève était nul en effet, c'était une raison pour lui conseiller de faire autre chose, et pour cesser de travailler avec lui, mais en aucun cas pour le traiter avec mépris. Je m'étais identifié aux "forts en maths" tels que cet aîné prestigieux, aux dépens des "nullités" qu'il serait licite de mépriser. J'ai suivi alors la voie toute tracée de la connivence avec le mépris, qui m'arrangeait, en mettant en relief ce fait que moi, j'étais accepté dans la confrérie des gens méritoires, des forts en maths ! \footnote{(7)\par L'alinéa qui précède est le premier de toute l'introduction qui soit fortement raturé sur mon manuscript initial, et muni de surcharges nombreuses. La description de l'incident, le choix des mots sont venus d'abord à rebrousse-poil, à contre-courant - une force visiblement poussait pour passer sur l'incident vite fait, comme par acquit de conscience, pour "passer aux choses sérieuses". Ce sont là les signes familiers d'une résistance, ici contre l'élucidation de cet épisode, et de sa portée comme révélateur d'une attitude intérieure. La situation est toute similaire à celle décrite au début de cette introduction (par. 2), celle du moment "crucial" de la découverte d'une contradiction et de son sens, dans un travail mathématique : c'est alors l'inertie de l'esprit, sa répugnance à se séparer d'une vision erronée ou insuffisante (mais où notre personne n'est nullement engagée), qui joue le rôle de la "résistance". Celle-ci est de nature active, inventive au besoin pour arriver à noyer un poisson même sans eau, alors que l'inertie dont j'ai parlé est une force simplement passive. Dans le cas présent, bien plus encore que dans le cas d'un travail mathématique, la découverte qui vient d'apparaître dans toute sa simplicité, dans toute son évidence, est suivie dans l'instant par un sentiment de soulagement d'un poids, un sentiment de libération. Ce n'est pas seulement un sentiment - c'est plutôt une perception aiguë et reconnaissante de ce qui vient de se passer, qui est une libération.}(7) 

Bien sûr, pas plus que quiconque, je ne me serais dit en termes clairs : les gens qui s'essayent à faire des maths sans y arriver sont bons à mépriser ! J'aurais entendu quelqu'un dire quelque chose de cette eau, vers cette époque ou à toute autre, je l'aurais repris de belle façon, sincèrement désolé d'une ignorance spirituelle aussi phénoménale. Le fait est que je baignais dans l'ambiguïté, je jouais sur deux tableaux qui ne communiquaient pas : d'une part les beaux principes et sentiments, de l'autre : pauvre gars, faut vraiment être nul pour se faire traiter comme ça (sous-entendu : c'est pas à moi que ce genre de mésaventure pourrait arriver, c'est sûr !).

Il me semble finalement que l'incident que j'ai rapporté, et surtout le rôle (en apparence anodin) que j'y ai joué, est en fait typique d'une ambiguïté en moi, qui m'a suivie tout au long de ma vie de mathématicien dans les vingt années qui ont suivi, et qui ne s'est dissipée qu'aux lendemains du "réveil" de 1970 \footnote{(8)\par Comme il deviendra clair dans la suite, cette ambiguïté ne s'est nullement "dissipée aux lendemains du réveil de 1970". Il y a là un mouvement de retraite stratégique typique du "moi", qui abandonne aux profits et pertes la période "avant le réveil", lequel devient aussitôt la ligne de démarcation pour un "après" irréprochable!} (8), sans que je la détecte clairement avant aujourd'hui même, en écrivant ces lignes. C'est bien dommage d'ailleurs que je ne m'en sois pas aperçu à ce moment. Peut-être le temps n'était-il pas mûr pour moi. Toujours est-il que les témoignages qui me parvenaient alors sur le règne du mépris, sur lequel j'avais choisi de fermer les yeux, ne me mettaient pas en cause personnellement, ni d'ailleurs aucun des collègues et amis dans la partie la plus proche de moi de mon cher microcosme \footnote{(9)\par Ce n'est pas entièrement exact, il y a au moins une exception parmi mes collègues les plus proches, comme il apparaîtra plus loin. Il y a eu là une "paresse" typique de la mémoire, qui a souvent tendance à "passer à l'as" les faits qui ne "collent" pas avec une vision des choses familière et enracinée de longue date.}(9). C'était plutôt sur l'air de : ah ! que c'est triste d'avoir à apprendre (ou : à vous apprendre) de telles choses, qui l'eût cru, faut vraiment être salaud (j'allais dire : nul, pardon !) pour traiter des êtres vivants de cette façon-là ! Pas si différent de l'autre air finalement, il suffit de remplacer "nul" par "salaud" et "se faire traiter" par "traiter" et le tour est joué ! Et l'honneur, bien sûr, est sauf, pour le champion des bonnes causes !

La chose qui ressort clairement de ceci, c'est ma connivence avec des attitudes de mépris. Elle remonte pour le moins aux tout débuts des années cinquante, dès les années donc qui ont suivi l'accueil bienveillant reçu auprès de Cartan et de ses amis. Si je ne "voyais rien" plus tard, alors que le mépris devenait monnaie courante un peu partout, c'est que je n'avais pas envie de voir - pas plus que dans ce cas isolé, et particulièrement flagrant, où il fallait vraiment mettre le paquet pour faire semblant de ne rien voir ni sentir !

Cette connivence était en étroite symbiose avec ma nouvelle identité, celle de membre respecté d'un groupe, le groupe des gens méritoires, des forts en maths. Je me rappelle que j'étais particulièrement satisfait, fier même, que dans ce monde que je m'étais choisi, qui m'avait coopté, ce n'était pas la position sociale ni même (mais non !) la seule réputation qui comptait, encore fallait-il qu'elle soit méritée - on avait beau être professeur d' Université ou académicien ou n'importe, si on n'était qu'un mathématicien médiocre (pauvre gars !) on n'était rien, ce qui comptait c'était uniquement le mérite, les idées profondes, originales, la virtuosité technique, les vastes visions et tout ça !

Cette idéologie du mérite, à laquelle je m'étais identifié sans réserve (alors qu'elle restait bien entendu implicite, inexprimée), a quand même pris un fier coup chez moi aux lendemains, comme je disais, du fameux réveil de 1970. Je ne suis pas sûr d'ailleurs qu'elle ait disparu dès ce moment sans laisser de traces. Il aurait sans doute fallu pour cela que je la détecte en moi-même clairement, alors que je la dénonçais surtout chez les autres, il me semble. C'est d'ailleurs Chevalley qui a été un des premiers, avec Denis Guedj que j'ai aussi connu par Survivre, à attirer mon attention sur cette idéologie-là (ils l'appelaient la "méritocratie", ou un nom comme ça), et ce qu'il y avait en elle de violence, de mépris. C'est à cause de ça, m'a dit Chevalley (ça devait être au moment de notre première rencontre chez lui, à propos de Survivre), qu'il ne supportait plus l'ambiance dans Bourbaki et avait cessé d'y mettre les pieds. Je suis persuadé, en y repensant, qu'il devait bien s'être aperçu que j'avais bien été partie prenante de cette idéologie-là, et peut-être même qu'il en restait encore des traces dans quelques recoins. Mais je ne me rappelle pas qu'il l'ait jamais laissé entendre. Peut-être que là encore, il avait préféré me laisser le soin de mettre des points sur les i qu'il me traçait, et j'ai attendu jusqu'à aujourd'hui pour les mettre. Mieux vaut tard que jamais !



