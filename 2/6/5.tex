\section{(9) L'étranger bienvenu}

Le moment me semble venu de m'exprimer au sujet de ma relation au monde des mathématiciens. C'est là une chose toute différente de ma relation aux mathématiques. Celle-ci a existé et a été forte dès mon jeune âge, bien avant même que je me doute qu'il existait un monde et un milieu de mathématiciens. Tout un monde complexe, avec ses sociétés savantes, ses périodiques, ses rencontres, colloques, congrès, ses primas-donnes et ses tâcherons, sa structure de pouvoir, ses éminences grises, et la masse non moins grise des taillables et corvéables, en mal de thèse ou d'articles et de ceux aussi, plus rares, qui sont riches en moyens et en idées et se heurtent aux portes closes, désespérant de trouver l'appui d'un de ces hommes puissants, pressés et craints qui disposent de ce pouvoir magique : faire publier un article...

J'ai découvert l'existence d'un monde mathématique en débarquant à Paris en 1948, à l'âge de vingt ans, avec dans ma maigre valise une Licence ès Sciences de l'Université de Montpellier, et un manuscrit aux lignes serrées, écrit recto-verso, sans marges (le papier était cher !), représentant trois ans de réflexions solitaires sur ce qui (je l'ai appris après) était alors bien connu sous le nom de "théorie de la mesure" ou de "l'intégrale de Lebesgue". Faute d'en avoir jamais rencontré d'autre, je croyais bien, jusqu'au jour où je suis arrivé dans la capitale, que j'étais seul au monde à "faire des maths", le seul mathématicien donc. (C'était pour moi la même chose, et l'est un peu resté jusqu'à aujourd'hui encore.) J'avais jonglé avec les ensembles que j'appelais mesurables (sans avoir rencontré d'ailleurs d'ensemble qui ne le soit...) et avec la convergence presque partout, mais ignorais ce qu'est un espace topologique. Je restais un peu paumé dans une douzaine de notions non équivalentes "d'espace abstrait" et de compacité, pêchés dans un petit fascicule (d'un dénommé Appert je crois, dans les Actualités Scientifiques et Industrielles), sur lequel j'étais tombé : Dieu sait comment. Je n'avais pas entendu prononcer encore, dans un contexte mathématique du moins, des mots étranges ou barbares comme groupe, corps, anneau, module, complexe, homologie (et j'en passe !), qui soudain, sans crier gare, déferlaient sur moi tous en même temps. Le choc fut rude!

Si j'ai "survécu" à ce choc, et ai continué à faire des maths et à en faire même mon métier, c'est qu'en ces temps reculés, le monde mathématique ne ressemblait guère encore à ce qu'il est devenu depuis. Il est possible aussi que j'avais eu la chance d'atterrir dans un coin plus accueillant qu'un autre de ce monde insoupçonné. J'avais une vague recommandation d'un de mes professeurs à la Faculté de Montpellier, Monsieur Soula (pas plus que ses collègues il ne m'avait vu souvent à ses cours !), qui avait été un élève de Cartan (père ou fils, je ne saurais plus trop dire). Comme Elie Cartan était alors déjà "hors jeu", son fils Henri Cartan fut le premier "congénère" que j'aie eu l'heur de rencontrer. Je ne me doutais pas alors à quel point c'était d'heureux augure ! Je fus accueilli par lui avec cette courtoisie empreinte de bienveillance qui le distingue, bien connue des générations de normaliens qui ont eu cette chance de faire leurs toutes premières armes avec lui. Il ne devait pas se rendre compte d'ailleurs de toute l'étendue de mon ignorance, à en juger par les conseils qu'il m'a donnés alors pour orienter mes études. Quoi qu'il en soit, sa bienveillance visiblement s'adressait à la personne, non au bagage ou aux dons éventuels, ni (plus tard) à une réputation ou à une notoriété...

Dans l'année qui a suivi, j'ai été l'hôte d'un cours de Cartan à "l'Ecole" (sur le formalisme différentiel sur les variétés), auquel je m'accrochais ferme; celui aussi du "Séminaire Cartan", en témoin ébahi des discussions entre lui et Serre, à grands coups de "Suites Spectrales" (brr !) et de dessins (appelés "diagrammes") pleins de flèches recouvrant tout le tableau. C'était l'époque héroïque de la théorie des "faisceaux", "carapaces" et de tout un arsenal dont le sens m'échappait totalement, alors que je me contraignais pourtant tant bien que mal à ingurgiter définitions et énoncés et à vérifier les démonstrations. Au Séminaire Cartan il y avait aussi des apparitions périodiques de Chevalley, de Weil, et les jours des Séminaires Bourbaki (réunissant une petite vingtaine ou trentaine à tout casser, de participants et auditeurs), on y voyait débarquer, tel un groupe de copains un peu bruyants, les autres membres de ce fameux gang Bourbaki : Dieudonné, Schwartz, Godement, Delsarte. Ils se tutoyaient tous, parlaient un même langage qui m'échappait à peu près totalement, fumaient beaucoup et riaient volontiers, il ne manquait que les caisses de bière pour compléter l'ambiance - c'était remplacé par la craie et l'éponge. Une ambiance toute autre qu'aux cours de Leray au Collège de France (sur la théorie de Schauder du degré topologique dans les espaces de dimension infinie, pauvre de moi !), que j'allais écouter sur les conseils de Cartan. J'avais été voir Monsieur Leray au Collège de France pour lui demander (si je me rappelle bien) de quoi traiterait son cours. Je ne me rappelle ni des explications qu'il a pu me donner, ni si j'y ai compris quoi que ce soit - seulement, que là aussi je sentais un accueil bienveillant, s'adressant au premier étranger venu. C'est cela et rien d'autre, sûrement, qui a fait que je suis allé à ce cours et m'y suis accroché bravement, comme au Séminaire Cartan, alors que le sens de ce que Leray y exposait m'échappait alors presque totalement.

La chose étrange, c'est que dans ce monde où j'étais nouveau venu et dont je ne comprenais guère le langage et le parlais encore moins, je ne me sentais pas un étranger. Alors que je n'avais guère l'occasion de parler (et pour cause !) avec un de ces joyeux lurons comme Weil ou Dieudonné, ou avec un de ces Messieurs aux allures plus distinguées comme Cartan, Leray, ou Chevalley, je me sentais pourtant accepté, je dirais presque : un des leurs. Je ne me rappelle pas une seule occasion où j'aie été traité avec condescendance par un de ces hommes, ni d'occasion où ma soif de connaître, et plus tard, à nouveau, ma joie de découvrir, se soit trouvé rejetée par une suffisance ou par un dédain \footnote{(5)\par Ce fait est d'autant plus remarquable que jusqu'vers 1957, j'étais considéré avec une certaine réserve par plus d'un membre du groupe Bourbaki, qui avait fini par me coopter, je crois, avec une certaine réticence. Une boutade bon-enfant me rangeait au nombre des "dangereux spécialistes" (en Analyse Fonctionnelle). J'ai senti parfois en Cartan une réserve inexprimée plus sérieuse - pendant quelques années, j'ai dû lui donner l'impression de quelqu'un porté vers la généralisation gratuite et superficielle. Il a été tout surpris de trouver dans la première (et seule) rédaction un peu longue que j'ai faite pour Bourbaki (sur le formalisme différentiel sur les variétés) une réflexion tant soit peu substantielle - il n'avait pas été bien chaud quand j'avais proposé de m'en charger. (Cette réflexion m'a été à nouveau utile des années plus tard, en développant le formalisme des résidus du point de vue de la dualité cohérente.) J'étais d'ailleurs le plus souvent largué pendant les congrès Bourbaki, surtout pendant les lectures en commun des rédactions, étant bien incapable de suivre lectures et discussions au rythme où elles se poursuivaient. Il est possible que je ne suis pas fait vraiment pour un travail collectif. Toujours est-il que cette difficulté que j'avais à m'insérer dans le travail commun, ou les réserves que j'ai pu susciter pour d'autres raisons encore à Cartan et à d'autres, ne m'ont à aucun moment attiré sarcasme ou rebuffade, ou seulement une ombre de condescendance, à part tout au plus une ou deux fois chez Weil (décidément un cas à part !). À aucun moment, Cartan ne s'est départi d'une égale gentillesse à mon égard, empreinte de cordialité et aussi de cette pointe d'humour bien à lui qui pour moi reste inséparable de sa personne.} (5). S'il n'en avait été ainsi, je ne serais pas "devenu mathématicien" comme on dit - j'aurais choisi un autre métier, où je pouvais donner ma mesure sans avoir à affronter le mépris...

Alors qu' "objectivement" j'étais étranger à ce monde, tout comme j'étais un étranger en France, un lien pourtant m'unissait à ces hommes d'un autre milieu, d'une autre culture, d'un autre destin : une passion commune. Je doute qu'en cette année cruciale où je découvrais le monde des mathématiciens, un d'eux, pas même Cartan dont j'étais un peu élève mais qui en avait beaucoup d'autres (et des moins largués !), percevait en moi cette même passion qui les habitait. Pour eux, je devais être un parmi une masse d'auditeurs de cours et de séminaires, prenant des notes et visiblement pas bien dans le coup. Si peut-être je me distinguais en quelque façon des autres auditeurs, c'est que je n'avais pas peur de poser des questions, qui le plus souvent devaient dénoter surtout mon ignorance phénoménale aussi bien du langage que des choses mathématiques. Les réponses pouvaient être brèves, voire étonnées, jamais l'hurluberlu ébahi que j'étais alors ne s'est heurté à une rebuffade, à une "remise à ma place", ni dans le milieu sans façons du groupe Bourbaki, ni dans le cadre plus austère du cours Leray au Collège de France. En ces années, depuis que j'avais débarqué à Paris avec une lettre pour Elie Cartan dans ma poche, jamais je n'ai eu l'impression de me trouver en face d'un clan, d'un monde fermé, voire hostile. Si j'ai connu, bien connu cette contraction intérieure en face du mépris, ce n'est pas dans ce monde-là ; pas en ce temps-là, tout au moins. Le respect de la personne faisait partie de l'air que j'y respirais. Il n'y avait pas à mériter le respect, faire ses preuves avant d'être accepté, et traité avec quelque aménité. Chose étrange peut-être, il suffisait d'être une personne, d'avoir visage humain.


