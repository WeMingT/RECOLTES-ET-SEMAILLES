\section{(10) La "Communauté mathématique" : fiction et réalité}

Rien d'étonnant donc si, dès cette année peut-être en mon for intérieur, et de plus en plus clairement en tous cas au cours des années qui ont suivi, je me suis senti membre de ce monde, auquel j'avais plaisir à référer sous ce nom, chargé pour moi de sens, de "communauté mathématique". Avant d'écrire ces lignes, il ne s'est jamais présenté l'occasion d'examiner quel était le sens que je donnais à ce nom, alors pourtant que je m'identifiais dans une large mesure à cette "communauté". Il est clair maintenant que celle-ci représentait pour moi ni plus ni moins qu'une sorte de prolongement idéal, dans l'espace et dans le temps, de ce monde bienveillant qui m'avait accueilli, et m'avait accepté comme un des leurs; un monde, de plus, auquel j'étais lié par une des grandes passions qui ont dominé ma vie.

Cette "communauté", à laquelle je m’identifiais progressivement, n'était pas une extrapolation entièrement fictive de ce milieu mathématique qui m'avait d'abord accueilli. Le milieu initial s'est élargi peu à peu, je veux dire : le cercle des mathématiciens que j'ai été amené à fréquenter régulièrement, mû par des thèmes d'intérêt communs et par des affinités de personnes, est allé s'élargissant dans les dix ou vingt ans qui ont suivi ce premier contact. En termes concrets, c'est le cercle de collègues et amis, ou plutôt cette structure concentrique allant des collègues auxquels j'étais lié le plus près (d'abord Dieudonné, Schwartz, Godement, plus tard surtout Serre, plus tard encore des gens comme Andreotti, Lang, Tate, Zariski, Hironaka, Mumford, Bott, Mike Artin, sans compter les gens du groupe Bourbaki qui lui aussi allait s'élargissant peu à peu, et des élèves qui venaient vers moi à partir des années soixante...), à d'autres collègues que j'avais eu l'occasion de rencontrer ici et là et auxquels j'étais lié de façon plus ou moins étroite par des affinités plus ou moins fortes - c'est ce microcosme donc, constitué au hasard des rencontres et des affinités, qui représentait le contenu concret de ce nom chargé pour moi de chaleur et de résonance : la communauté mathématique. Quand je m'identifiais à celle-ci comme à une entité vivante, chaleureuse, c'était en fait à ce microcosme que je m'identifiais.

Ce n'est qu'après le "grand tournant" de 1970, le premier réveil devrais-je dire, que je me suis rendu compte que ce microcosme douillet et sympathique ne représentait qu'une toute petite portion du "monde mathématique", et que les traits qu'il me plaisait de prêter à ce monde, que je continuais à ignorer, auquel je n'avais jamais songé à m'intéresser, étaient des traits fictifs.

Au cours de ces vingt et deux ans, ce microcosme lui-même avait d'ailleurs changé de visage, dans un monde environnant qui lui aussi changeait. Moi aussi assurément, au fil des ans et sans m'en douter, j'avais changé, comme le monde autour de moi. Je ne sais si mes amis et collègues s'apercevaient plus que moi de ce changement, dans le monde environnant, dans leur microcosme à eux, et dans eux-mêmes. Je ne saurais dire non plus quand et comment c'est fait ce changement étrange - c'est venu sans doute insidieusement, à pas-de-loups : l'homme de notoriété était craint. Moi-même étais craint - sinon par mes élèves ni par mes amis, ou par ceux qui me connaissaient personnellement, du moins par ceux qui ne me connaissaient que par une notoriété, et qui ne se sentaient eux-mêmes protégés par une notoriété comparable.

Je n'ai pris conscience de la crainte qui sévit dans le monde mathématique (et tout autant, sinon plus encore, dans les autres milieux scientifiques) qu'aux lendemains de mon "réveil" d'il y a bientôt quinze ans. Pendant les quinze ans qui avaient précédé, progressivement et sans m'en douter, j'étais entré dans le rôle du "grand patron", dans le monde du Who is Who mathématique. Sans m'en douter aussi, j'étais prisonnier de ce rôle, qui m’isolait de tous sauf de quelques "pairs" et de quelques élèves (et encore...) qui décidément "en voulaient". C'est une fois seulement que je suis sorti de ce rôle, qu'une partie au moins de la crainte qui l'entoure est tombée. Les langues se sont déliées, qui avaient été muettes devant moi pendant des années.

Le témoignage qu'elles m'apportaient n'était pas seulement celui de la crainte. C'était aussi celui du mé-pris. Le mépris surtout des gens en place vis à vis des autres, un mépris qui suscite et alimente la crainte.

Je n'avais guère l'expérience de la crainte, mais bien celle du mépris, en des temps où la personne et la vie d'une personne ne pesaient pas lourd. Il m'avait plu d'oublier le temps du mépris, et voilà qu'il se rappelait à mon bon souvenir! Peut-être n'avait-il jamais cessé, alors que je m'étais contenté simplement de changer de monde (comme il m'avait semblé), de regarder ailleurs, ou simplement : de faire semblant de ne rien voir, rien entendre, en dehors des passionnantes et interminables discussions mathématiques? En ces jours, enfin j'acceptais d'apprendre que le mépris sévissait partout autour de moi, dans ce monde que j'avais choisi comme mien, auquel je m'étais identifié, qui avait eu ma caution et qui m'avait choyé.
