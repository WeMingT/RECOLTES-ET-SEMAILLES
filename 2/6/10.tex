\section{(14) Naissance de la crainte}

C'est vers ce moment je suppose, quand (sans l'avoir cherché) j'ai commencé à être vu comme une vedette dans le monde mathématique, qu'une certaine crainte a dû commencer aussi à entourer ma personne, pour bon nombre de collègues inconnus ou moins connus. Je le suppose, sans pouvoir le situer par un souvenir précis, par une image qui m'aurait frappé et se serait fixée dans ma mémoire, comme cet incident rapporté précédemment (qui a sans doute marqué ma première rencontre avec le mépris dans mon milieu d'adoption). La chose a dû se faire insensiblement, sans attirer mon attention, sans se manifester par quelque incident particulier, typique, que la mémoire aurait retenu, avec un éclairage peut-être tout aussi délibérément anodin que pour cet autre incident. Ce que me restitue "en bloc" mon souvenir de ces années de transition, c'est qu'il n'était pas rare que les gens qui m'abordaient, que ce soit après mon séminaire, ou pendant une rencontre telle que le séminaire Bourbaki ou quelque colloque ou congrès, avaient à surmonter une sorte de trac, qui restait plus ou moins apparent pendant notre discussion, si discussion il y avait. Quand celle-ci durait plus que quelques minutes, cette gêne le plus souvent disparaissait progressivement pendant que nous parlions et que la conversation s'animait. Parfois aussi, rarement, il a dû arriver que la gêne se maintenait, au point de devenir un obstacle réel à la communication même au niveau impersonnel d'une discussion mathématique, et que j'aie senti alors confusément en face de moi une souffrance impuissante, exaspérée d'elle-même. Je parle de tout ceci sans vraiment "me souvenir", comme à travers un brouillard qui, néanmoins, me restitue des impressions qui ont dû être enregistrées, et évacuées sans doute au fur et à mesure. Je serais bien incapable de situer dans le temps, autrement que par une supposition, l'apparition de cette gêne, expression d'une crainte.

Je ne crois pas que cette crainte émanait de ma personne et qu'elle était limitée à une attitude, à des comportements qui m'auraient distingué de mes collègues. S'il en avait été ainsi, il me semble que j'aurais fini par en recevoir des échos au début des années soixante-dix, quand je suis sorti d'un rôle auquel je m'étais prêté jusque là, le rôle justement de vedette, de "grand patron". C’est ce rôle je crois, et non ma personne, qui était entouré de crainte. Et ce rôle, il me semble, avec cet halo de crainte qui n'a rien de commun avec le respect, n'existait pas, pas encore, au début des années cinquante, tout au moins pas dans le milieu mathématique qui m'avait accueilli à partir du moment même où j'ai fait sa rencontre, en 1948.

Avant ce "réveil" de 1970, je n'aurais pas songé d'ailleurs à qualifier de "crainte" ce trac, cette gêne auxquels j'étais confrontés parfois, en des collègues qui ne faisaient pas partie du milieu le plus familier. J'en étais gêné moi-même quand elle se manifestait, et faisait alors mon possible pour la dissiper. Une chose remarquable, typique du peu d'attention accordé à ce genre de choses dans mon cher microcosme : je ne me rappelle pas d'une seule fois, pendant les vingt ans où j'ai fait partie de ce milieu, où la question ait été abordée entre un collègue et moi, ou par d'autres devant moi ! \footnote{(11) Aldo Andreotti, Ionel Bucur \par Bien sûr. il n'est pas impossible qu'il y ait oubli de ma part - sans compter que mes dispositions particulièrement "polar" en ce temps ne devaient guère encourager à parler avec moi de ce genre de choses, ni me porter à me souvenir d'une conversation dans ce sens qui pourrait bien avoir eu lieu. Ce qui est sûr, c'est qu'il devait être très exceptionnel pour le moins que la question de la crainte soit abordée (sans même l'appeler par ce nom...), et ça doit l'être tout autant aujourd'hui, surtout dans le "beau monde".

Parmi mes nombreux amis dans ce monde-là, à part Chevalley, qui a dû prendre conscience de cette ambiance de crainte tout au moins au cours des années soixante, le seul autre dont il me semble qu'il a bien dû la percevoir clairement est Aldo Andreotti. J'avais fait sa connaissance, ainsi que celle de sa femme Barbara et de leurs enfants jumeaux (encore tout petits), en 1955 (à une soirée chez Weil à Chicago, je crois). Nous sommes restés très liés jusqu'au moment du "grand tournant" de 1970, quand j'ai quitté le milieu qui avait été le nôtre et les ai un peu perdus de vue. Aldo avait une très vive sensibilité, qui ne s'était nullement émoussée par le commerce avec la mathématique et avec des "polars" comme moi. Il y avait en lui un don de sympathie spontanée pour ceux qu'il approchait. Cela le mettait à part de tous les autres amis que j'ai connus dans le milieu mathématique, ou même en dehors. Chez lui toujours l'amitié prenait le pas sur les intérêts mathématiques communs (qui ne manquaient pas), et c'est un des rares mathématiciens avec qui j'aie tant soit peu parlé de ma vie, et lui de la sienne. Son père, comme le mien, était juif, et il avait eu à en pâtir dans l'Italie mussolinienne, comme moi dans l'Allemagne hitlérienne. Je l'ai vu toujours disponible pour encourager et appuyer les jeunes chercheurs, dans un climat où il devenait difficile de se faire accepter par l'establishment. Son intérêt spontané toujours le portait d'abord vers la personne, non vers un "potentiel" mathématique ou vers un renom. Il a été l'une des personnes les plus attachantes que j'aie eu la chance de rencontrer.

Cette évocation de Aldo fait surgir le souvenir de Ionel Bucur, lui aussi emporté inopinément et avant l'âge, et comme Aldo, regretté plus encore (je crois) comme l'ami qu'on aime à retrouver, que comme le partenaire de discussions mathématiques. On sentait en lui une bonté, à côté d'une modestie peu commune, une propension à constamment s'effacer. C'est un mystère comment un homme aussi peu porté à se prendre pour important ou à impressionner quiconque, ait fini par se retrouver doyen de la Faculté des Sciences à Bucarest ; sans doute parce que l'idée ne lui venait pas de récuser des charges qu'il était loin de convoiter, mais que ses collègues ou l'autorité politique posaient sur ses épaules, robustes il faut le dire. Il était fils de paysans (chose qui a dû jouer dans un pays où le "critère de classe" est important), et en avait le bon sens et la simplicité. Sûrement il devait se rendre compte de la crainte qui entoure l'homme de notoriété, mais sûrement aussi la chose devait lui paraître comme allant de soi, comme l'attribut naturel d'une position de pouvoir. Je ne pense pas pourtant que lui-même ait jamais inspiré de crainte à quiconque, ni certes à sa femme Florica ou à leur fille Alexandra, ni à ses collègues ou à ses étudiants - et les échos que j'ai pu avoir vont bien dans ce sens.}(11) Ce "brouillard" qui me tient lieu de souvenir ne me restitue pas non plus quelque impression de gratification consciente ou inconsciente que de telles situations auraient suscitée en moi. Je ne pense pas qu'il y en ait eu au niveau conscient, mais ne me hasarderais pas à affirmer que je n'en ai pas été effleuré occasionnellement au niveau inconscient, dans les premières années. Si oui, cela a dû être fugitif, sans se répercuter dans un comportement qui aurait agi comme fixateur d'une gêne. Ce n'est certes pas que ma fatuité n'était engagée dans le rôle que je jouais! Mais si j'investissais dans ce rôle sans compter, ce qui motivait alors mon ego n'était pas l'ambition d'impressionner le "collègue du rang", mais de me surpasser sans cesse pour forcer l'estime sans cesse renouvelée de mes "pairs" - et avant tous autres, peut-être, des aînés qui m'avaient fait crédit et m'avaient accepté comme un des leurs dès avant que j'aie pu donner ma mesure. Il me semble que l'attitude intérieure qui a été la mienne vis-à-vis de la crainte dont j'étais l'objet, que j'essayais de mon mieux d'ignorer tout en la dissipant tant bien que mal là où elle se manifestait - que cette attitude peut être considérée comme typique tout au long des années soixante dans le milieu (le "microcosme") dont je faisais partie.

La situation s'est considérablement dégradée encore, dans les dix ou quinze ans qui se sont écoulés depuis, à en juger tout au moins par les signes qui me parviennent de temps en temps de ce monde, et les situations dont j'ai pu être le proche témoin, voire même parfois un coacteur. Plus d'une fois, parmi ceux-là même de mes anciens amis ou élèves qui m'avaient été les plus chers, j'ai été confronté aux signes familiers, irrécusables du mépris; à la volonté ("gratuite" en apparence) de décourager, d'humilier, d'écraser. Un vent du mépris s'est levé je ne saurais dire quand, et souffle dans ce monde qui m'avait été cher. Il souffle, sans se soucier du "mérite" ou "démérite", brûlant par son haleine les humbles vocations comme les plus belles passions. En est-il un seul parmi mes compagnons d'antan, protégés chacun, avec "les siens", par de solides murailles, installé (comme je le fus naguère) dans la crainte feutrée qui entoure sa personne - en est-il un seul qui sente ce souffle-là ? J'en connais bien un et un seul, parmi mes anciens amis, qui l'ait senti et m'en ait parlé, sans l'appeler par son nom. Et tel autre aussi qui l'a perçu un jour comme à son corps défendant, pour s'empresser de l'oublier le lendemain même \footnote{(12)\par Le mot "lendemain" est ici à prendre au sens littéral, non comme une métaphore .}(12). Car sentir ce souffle et l'assumer, pour un de mes amis d'antan tout comme pour moi-même, c'est aussi accepter de porter un regard sur soi-même.






