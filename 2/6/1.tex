Février 1984

\section{(5) Le rêve interdit}

Je prends l'occasion d'une interruption de trois mois dans l'écriture de la Poursuite des Champs, pour reprendre l'Introduction au point où je l'avais laissée au mois de Juin dernier. Je viens de la relire attentivement, à plus de six mois de distance, et d'y ajouter quelques sous-titres.

En écrivant cette Introduction, j'étais bien conscient que ce type de réflexions ne pourrait manquer de susciter de nombreux "malentendus" - et il serait vain d'essayer d'en prendre les devants, ce qui reviendrait simplement à en accumuler d'autres par dessus les premiers ! La seule chose que j'ajouterais à ce propos, c'est qu'il n'est nullement dans mes intentions de partir en guerre contre le style d'écriture scientifique consacré par un usage millénaire, que j'ai moi-même pratiqué avec assiduité pendant plus de vingt ans de ma vie, et enseigné à mes élèves comme une part essentielle du métier de mathématicien. À tort ou à raison, aujourd'hui encore je le considère comme tel et continue à l'enseigner. Sûrement même je ferais plutôt vieux jeu, avec mon insistance sur un travail fait jusqu'au bout, cousu main du début à la fin, et sans faire grâce à aucun coin un peu sombre. Si j'ai dû mettre de l'eau dans mon vin depuis une dizaine d'années, c'est bien par la force des choses! La "rédaction en forme" reste pour moi une étape importante du travail mathématique, tant comme un instrument de découverte, pour tester et approfondir une compréhension des choses qui sans elle reste approximative et fragmentaire, que comme un moyen pour communiquer une telle compréhension. Au point de vue didactique, le mode d'exposition de rigueur, le mode déductif donc, qui n'exclut nullement la possibilité de brosser de vastes tableaux, offre des avantages évidents, de concision et de commodité des références. Ce sont bien là des avantages réels, et de poids, quand il s'agit d'exposés qui s'adressent à des mathématiciens disons, et plus particulièrement, à des mathématiciens qui sont suffisamment familiers déjà avec certains tenants et aboutissants du sujet traité, ou d'autres tout proches.

Ces avantages par contre deviennent entièrement illusoires pour un exposé qui s'adresse à des enfants, à des jeunes gens ou à des adultes qui ne sont absolument pas "dans le coup" d'avance, dont l'intérêt n'est déjà en éveil, et qui d'ailleurs, le plus souvent, sont (et resteront, et pour cause...) dans une ignorance totale de ce qu'est la démarche véritable d'un travail de découverte. Des lecteurs, pour mieux dire, qui ignorent l'existence même d'un tel travail, à la portée de chacun doué de curiosité et de bon sens.- ce travail dont naît et renaît sans cesse notre connaissance intellectuelle des choses de l'Univers, y compris celle qui s'exprime dans d'imposants ordonnancements comme les "Eléments" d'Euclide, ou "L'Origine des Espèces" de Darwin. L'ignorance complète de l'existence et de la nature d'un tel travail est chose quasiment universelle, y compris parmi les enseignants à tous les niveaux d'enseignement, de l'instituteur au professeur d'université. C'est là un fait extraordinaire, qui m'est apparu en pleine lumière à l'occasion d'abord de la réflexion commencée l'an dernier avec la première partie de cette Introduction, en même temps que j'entrevoyais alors les racines profondes de ce fait déroutant...

Alors même qu'il s'adresserait à des lecteurs parfaitement "dans le coup" à tous points de vue, il reste une chose importante pourtant que le mode d'exposition "de rigueur" s'interdit de communiquer. C'est aussi une chose tout à fait mal vue dans les milieux de gens sérieux, comme nous autres scientifiques notamment! Je veux parler du rêve. Du rêve, et des visions qu'il nous souffle - impalpables comme lui d'abord, et réticentes souvent à prendre forme. De longues années, voire une vie entière de travail intense ne suffiront pas peut-être pour voir se manifester pleinement telle vision de rêve, la voir se condenser et se polir jusqu'à la dureté et l'éclat du diamant. C'est là notre travail, ouvriers par la main ou par l'esprit. Quand le travail est achevé, ou telle partie du travail, nous en présentons le résultat tangible sous la lumière la plus vive que nous pouvons trouver, nous nous en réjouissons, et souvent en tirons fierté. Ce n'est pas en ce diamant pourtant, que nous avons longuement taillé, que se trouve ce qui nous a inspirés en le taillant. Peut-être avons-nous façonné un outil de grande précision, un outil efficace - mais l'outil même est limité, comme toute chose faite par la main de l'homme, même quand elle nous paraît grande. Une vision, sans nom et sans contours d'abord, ténue comme un lambeau de brumes, a guidé notre main et nous a maintenus penchés sur l'ouvrage, sans sentir passer les heures ni peut-être les années. Un lambeau qui s'est détaché sans bruit d'une Mer sans fond de brume et de pénombre... Ce qui est sans limites en nous c'est Elle, cette Mer prête à concevoir et à enfanter sans cesse, quand notre soif La féconde. De ces épousailles-là sourd le Rêve, tel l'embryon niché dans la matrice nourricière, attendant les obscurs labeurs qui le mèneront vers une seconde naissance, à la lumière du jour.

Malheur à un monde où le rêve est méprisé - c'est un monde aussi où ce qui est profond en nous est méprisé. Je ne sais si d'autres cultures avant la nôtre - celle de la télévision, des ordinateurs et des fusées transcontinentales - ont professé ce mépris-là. Ça doit être un des nombreux points par lesquels nous nous distinguons de nos prédécesseurs, que nous avons si radicalement supplantés, éliminés autant dire de la surface de la planète. Je n'ai pas eu connaissance d'une autre culture, où le rêve ne soit respecté, où ses racines profondes ne soient ressenties par tous et reconnues. Et y a-t-il œuvre d'envergure dans la vie d'une personne ou d'un peuple, qui ne soit née du rêve et ne fût nourrie par le rêve avant d'éclore au grand jour? Chez nous pourtant (faut-il même dire déjà : partout?) le respect du rêve s'appelle "superstition", et il est bien connu que nos psychologues et psychiatres ont pris la mesure du rêve en long en large et en travers - à peine de quoi encombrer la mémoire d'un petit ordinateur, sûrement. Il est vrai aussi que plus personne "chez nous" ne sait allumer un feu, ni ose dans sa maison voir naître son enfant, ou mourir sa mère ou son père - il y a des cliniques et des hôpitaux qui sont là pour ça. Dieu merci... Notre monde, si fier de sa puissance en mégatonnes atomiques et en quantité d'information stockée dans ses bibliothèques et dans ses ordinateurs, est sans doute celui aussi où l'impuissance de chacun, cette peur et ce mépris devant les choses simples et essentielles de la vie a atteint son point culminant.

Heureusement le rêve, tout comme la pulsion originelle du sexe dans la société même la plus répressive, a la vie dure ! Superstition ou pas, il continue à la dérobée à nous souffler obstinément une connaissance que notre esprit éveillé est trop lourd, ou trop pusillanime pour appréhender, et à donner vie et à prêter des ailes aux projets qu'il nous a inspirés.

Si j'ai laissé entendre tantôt que le rêve était souvent réticent à prendre forme, il s'agit là d'une apparence, qui ne touche pas vraiment au fond des choses. La "réticence" viendrait plutôt de notre esprit à l'état de veille, dans son "assiette" ordinaire - et encore le terme "réticence" est-il un euphémisme ! Il s'agirait plutôt d'une méfiance profonde, qui recouvre une peur ancestrale - la peur de connaître. Parlant du rêve au sens propre du terme, cette peur est d'autant plus agissante, elle fait un écran d'autant plus efficace, que le message du rêve nous touche de plus près, qu'il est lourd de la menace d'une transformation profonde de notre personne, si d'aventure il venait à être entendu. Mais il faut croire que cette méfiance est présente et efficace même dans le cas relativement anodin du "rêve" mathématique; au point que tout rêve semble banni non seulement des textes (je n'en connais aucun en tous cas où il y en ait trace); mais également des discussions entre collègues, en petit comité, voire en tête à tête.

S'il en est ainsi, ce n'est certes pas que le rêve mathématique n'existerait pas ou n'existerait plus - notre science alors serait devenue stérile, ce qui n'est nullement le cas, sûrement la raison de cette absence apparente, de cette conspiration du silence, est liée de très près à cet autre consensus - celui d'effacer soigneusement toute trace et toute mention du travail par quoi se fait la découverte et se renouvelle notre connaissance du monde. Ou plutôt, c'est un seul et même silence qui entoure et le rêve, et le travail qu'il suscite, inspire et nourrit. Au point que le terme même de "rêve mathématique" paraîtra un non-sens à beaucoup, mus que nous sommes si souvent par des clichés pousse-bouton, plutôt que par l'expérience directe que nous pouvons avoir d'une réalité toute simple, quotidienne, importante.

