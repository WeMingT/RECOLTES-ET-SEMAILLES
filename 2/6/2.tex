\section{(6) Le Rêveur}

En fait, je sais bien par expérience que lorsque l'esprit est avide de le connaître, au lieu de le fuir (ou de l'aborder avec une grille brevetée à la main, ce qui revient au même), le rêve n'est nullement réticent "à prendre forme" - à se laisser décrire avec délicatesse et à livrer son message, toujours simple, jamais sot, et parfois bouleversant. Bien au contraire, le Rêveur en nous est un maître incomparable pour trouver, ou créer de toutes pièces, d'une occasion à l'autre, le langage le plus propre à circonvenir nos peurs, à secouer nos torpeurs, avec des moyens scéniques variant à l'infini, depuis l'absence de tout élément visuel ou sensoriel quel qu'il soit, aux mises en scène les plus époustouflantes. Quand Il se manifeste, ce n'est nullement pour se dérober, mais pour nous encourager (en pure perte presque toujours, sans que ne se lasse Sa bienveillance...) à sortir de nous-mêmes, de la lourdeur où il nous voit engoncés, et qu'il s'amuse parfois, mine de rien, de parodier en des couleurs cocasses. Prêter oreille au Rêveur en nous, c'est communiquer avec nous-mêmes, à l'encontre des barrages puissants qui voudraient à tout prix nous l'interdire.

Mais qui peut le plus, peut le moins. Si nous pouvons communiquer avec nous-mêmes par le truchement du rêve, nous révélant à nous-mêmes, sûrement il doit être possible de façon toute aussi simple de communiquer à autrui le message nullement intime du rêve mathématique, disons, qui ne met pas en jeu des forces de résistance d'une puissance comparable. Et à vrai dire, qu'ai-je fait d'autre dans mon passé de mathématicien, si ce n'est suivre, "rêver" jusqu'au bout, jusqu'à leur manifestation la plus manifeste, la plus solide : irrécusable, des lambeaux de rêve se détachant un à un d'un lourd et dense tissu de brumes? Et combien de fois ai-je trépigné d'impatience devant ma propre obstination à polir jalousement jusqu'à sa dernière facette chaque pierre précieuse ou précieuse à demi en quoi se condensaient mes rêves - plutôt que de suivre une impulsion plus profonde : celle de suivre les arcanes multiformes du tissu-mère - aux confins indécis du rêve et de son incarnation patente, "publiable" en somme, suivant les canons en vigueur ! J'étais d'ailleurs sur le point de suivre cette impulsion-là, de me lancer dans un travail de "science-fiction mathématique", "une sorte de rêve éveillé" sur une théorie des "motifs" qui restait à ce moment purement hypothétique - et qui l'est resté jusqu'à aujourd'hui encore et pour cause, faute à un autre "rêveur éveillé" de se lancer dans cette aventure. C'était vers la fin des années soixante, alors que ma vie {Sans que je m'en doute le moins du monde.} s'apprêtait à prendre un tout autre tournant, qui pendant une dizaine d'années allait reléguer ma passion mathématique à une place marginale, voire reniée.

Mais à tout bien prendre, "A la Poursuite des Champs", cette première publication après quatorze ans de silence, est bien dans l'esprit de ce "rêve éveillé" qui ne fût jamais écrit, et dont il semble avoir pris la suite provisoire. Certes, les thèmes de ces deux rêves-là sont aussi dissemblables, à première vue tout au moins, qu'il est possible pour deux thèmes mathématiques ; sans compter que le premier, celui des motifs, semblerait se situer à l'horizon plutôt de ce qui pourrait être "faisable" avec les moyens du bord, alors que le deuxième, les fameux "champs" et consorts, paraissent tout à fait à portée de la main. Ce sont là des dissemblances qu'on pourrait appeler fortuites ou accidentelles, et qui peut-être s'évanouiront bien plus tôt qu'on ne s'y attend \footnote{(3)\par Je pense ici notamment aux feues conjectures de Mordell, de Tate, de Chafarévitch, qui se sont trouvées démontrées toutes trois l'an dernier dans un manuscript de quarante pages de Faltings, à un moment où le consensus bien établi des gens "dans le coup" statuait que ces conjectures étaient "hors de portée"! Il se trouve que "la" conjecture fondamentale qui sert de clef de voûte au programme de "géométrie algébrique anabélienne" qui m'est cher, est proche justement de la conjecture de Mordell. (Il paraîtrait même que celle-ci serait une conséquence de celle-là, ce qui montrait bien que ce programme n'était pas une histoire pour gens sérieux...)} (3). Elles n'ont que relativement peu d'incidence, me semble-t-il, sur le genre de travail auquel l'un et l'autre thème peuvent donner lieu, dès lors qu'il s'agit justement de "rêve éveillé", ou, pour le dire en termes moins provocateurs : de poursuivre le travail de dégrossissage conceptuel jusqu'à une vision d'ensemble d'une cohérence et d'une précision suffisante, pour entraîner la conviction plus ou moins complète que la vision correspond bien, pour l'essentiel, à la réalité des choses. Dans le cas du thème développé dans le présent ouvrage, cela devrait signifier, plus ou moins, que la vérification circonstanciée de la validité de cette vision devient une question de pur métier. Cela peut certes demander un travail considérable, avec sa part d'astuce et d'imagination, et sans doute aussi des rebondissements et des perspectives inattendus, qui en feront autre chose, heureusement, qu'un travail de pure routine (un "long exercice", comme dirait André Weil).

C'est le genre de travail, en somme, que j'ai fait et refait à satiété dans le passé, que j'ai au bout des doigts et qu'il est donc inutile que je refasse dans les années qui restent encore devant moi. Dans la mesure où je m'investis à nouveau dans un travail mathématique, c'est aux confins du "rêve éveillé" que mon énergie sûrement sera la mieux employée. Dans ce choix, ce n'est pas d'ailleurs un souci de rentabilité qui m'inspire (à supposer qu'un tel souci puisse inspirer quiconque), mais un rêve justement, ou des rêves. Si ce nouvel élan en moi doit se révéler porteur de force, c'est dans le rêve qu'il l'aura puisée !

