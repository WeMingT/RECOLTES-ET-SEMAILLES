\section{(8) Rêve et démonstration}

Mais revenons au rêve, et à l'interdit qui le frappe en mathématiques depuis des millénaires. C'est là le plus invétéré peut-être parmi tous les a-prioris, implicites souvent et enracinés dans les habitudes, décrétant que telle chose "c'est des maths" et telle autre, non. Il a fallu des millénaires avant que des choses aussi enfantines et omniprésentes que les groupes de symétries de certaines figures géométriques, les formes topologiques de certaines autres, le nombre zéro, les ensembles trouvent admission dans le sanctuaire ! Quand je parle à des étudiants de la topologie d'une sphère, et des formes qui se déduisent d'une sphère en ajoutant des anses choses qui ne surprennent pas les jeunes enfants, mais qui les déroutent parce qu'ils croient savoir ce que c'est que "des maths" - le premier écho spontané que je reçois est : mais c'est pas des maths ça ! Les maths bien sûr, c'est le théorème de Pythagore, les hauteurs d'un triangle et les polynômes du second degré... Ces étudiants ne sont pas plus stupides que vous ni moi, ils réagissent comme ont réagi de tous temps jusqu'à aujourd'hui même tous les mathématiciens du monde, sauf des gens comme Pythagore ou Riemann et peut-être cinq ou six autres. Poincaré même, qui n'était pas le premier venu, arrivait à prouver par un A plus B philosophique bien senti que les ensembles infinis, c'étaient pas des maths ! Sûrement il a dû y avoir un temps où les triangles et les carrés c'étaient pas des maths - c'étaient des dessins que les gosses ou les artisans potiers traçaient sur le sable ou dans l'argile des vases, pas confondre...

Cette inertie foncière de l'esprit, étouffé par son "savoir", n'est pas propre certes aux mathématiciens. Je suis en train de m'éloigner quelque peu de mon propos : l'interdit qui frappe le rêve mathématique, et à travers lui, tout ce qui ne se présente pas sous les aspects habituels du produit fini, prêt à la consommation. Le peu que j'ai appris sur les autres sciences naturelles suffit à me faire mesurer qu'un interdit d'une semblable rigueur les aurait condamnées à la stérilité, ou à une progression de tortue, un peu comme au Moyen Age où il n'était pas question d'écornifler la lettre des Saintes Ecritures. Mais je sais bien aussi que la source profonde de la découverte, tout comme la démarche de la découverte dans tous ses aspects essentiels, est la même en mathématique qu'en toute autre région ou chose de l'univers que notre corps et notre esprit peuvent connaître. Bannir le rêve, c'est bannir la source - la condamner à une existence occulte.

Et je sais bien aussi, par une expérience qui ne s'est pas démentie depuis mes premières et juvéniles amours avec la mathématique, ceci : dans le déployement d'une vision vaste ou profonde des choses mathématiques, c'est ce déployement d'une vision et d'une compréhension, cette pénétration progressive, qui constamment précède la démonstration, qui la rend possible et lui donne son sens. Quand une situation, de la plus humble à la plus vaste, a été comprise dans ses aspects essentiels, la démonstration de ce qui est compris (et du reste) tombe comme un fruit mûr à point. Alors que la démonstration arrachée comme un fruit encore vert à l'arbre de la connaissance laisse un arrière-goût d'insatisfaction, une frustration de notre soif, nullement apaisée. Deux ou trois fois dans ma vie de mathématicien ai-je dû me résoudre, faute de mieux, à arracher le fruit plutôt que le cueillir. Je ne dis pas que j'aie mal fait, ou que je le regrette. Mais ce que j'ai su faire de meilleur et ce que j'ai le mieux aimé, je l'ai pris de gré et non de force. Si la mathématique m'a donnée joies à profusion et continue à me fasciner dans mon âge mûr, ce n'est pas par les démonstrations que j'aurais su lui arracher, mais par l'inépuisable mystère et l'harmonie parfaite que je sens en elle, toujours prête à se révéler à une main et un regard aimants.
