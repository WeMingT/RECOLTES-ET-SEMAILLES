\section{(11) Rencontre avec Claude Chevalley, ou : liberté et bons sentiments}

Peut-être les lignes qui précèdent peuvent-elles donner l'impression que j'étais bouleversé par les témoignages qui, presque du jour au lendemain, se mirent à affluer vers moi. Il n'en est rien pourtant. Ces témoignages étaient enregistrées à un niveau qui restait superficiel. Ils s'ajoutaient simplement à d'autres faits que je venais d'apprendre, ou que je connaissais tout en évitant jusque là d'y prêter attention. Aujourd'hui, j'exprimerais la leçon que j'ai apprise alors ainsi : "les scientifiques", des plus illustres aux plus obscurs, sont des gens exactement comme tous les autres! Je m'étais complu à m'imaginer que "nous" étions quelque chose de mieux, que nous avions quelque chose en sus - il m'a fallu bien un an ou deux pour me débarrasser de cette illusion-là, décidément tenace !

Parmi les amis qui m'y ont aidé, un seul faisait partie du milieu que je venais de quitter sans esprit de retour \footnote{(6)Mes amis de Survivre et Vivre \par
Parmi ces amis, je devrais sans doute compter aussi Pierre Samuel, que j'avais connu précédemment surtout dans Bourbaki, tout comme Chevalley, et qui a (comme lui) joué un rôle important au sein du groupe Survivre et Vivre. Il ne me semble pas que Samuel ait été tellement porté sur cette illusion d'une supériorité du scientifique. Il a surtout beaucoup apporté, je sens, par le bon sens et la bonne humeur souriante qu'il mettait dans le travail en commun, les discussions, les relations à autrui, et également pour porter avec grâce le rôle de "l'affreux réformiste" dans un groupe porté vers les analyses et les options radicales. Il est resté dans Survivre et Vivre encore quelque temps après que je m'en sois retiré, faisant office de directeur du bulletin de même nom, et il est parti avec bonne grâce (pour rejoindre les Amis de la Terre) quand il a senti que sa présence dans ce groupe avait cessé d'être utile.

Samuel faisait partie du même milieu restreint que moi, ce qui n'a pas empêché qu'il fasse partie des amis de ces années bouillonnantes dont je crois avoir appris quelque chose (tout mauvais élève que j'aie été...). Ces façons d'être, tout comme celles de Chevalley alors qu'ils ne se ressemblent guère, était un meilleur antidote pour mes penchants "méritocratiques", que l'analyse la plus percutante!

Il m'apparaît maintenant que pour tous les amis de cette période dont j'ai appris quelque chose, c'est plus par leurs façons d'être et leur sensibilité différente de la mienne, et dont "quelque chose" a fini par se communiquer, que par des explications, des discussions, etc... Je me rappelle surtout, à ce propos, en plus de Chevalley et de Samuel, de Denis Guedj (qui avait un grand ascendant sur le groupe Survivre et Vivre), de Daniel Sibony (qui s'est maintenu à l'écart de ce groupe, tout en poursuivant son évolution du coin d'un œil mi-dédaigneux, mi-narquois), Gordon Edwards (qui a été coacteur de la naissance du "mouvement" en juin 1970 à Montréal, et qui pendant des années a fait des prodiges d'énergie pour maintenir une "édition américaine" du bulletin Survivre et Vivre, en langue anglaise), Jean Delord (un physicien à peu près de mon âge, homme fin et chaleureux, qui m'avait pris en affection ainsi que le microcosme survivrien), Fred Snell (un autre physicien établi aux États-Unis, de Buffalo, dont j'ai été l'hôte dans sa maison de campagne pendant un séjour de quelques mois en 1972).

Parmi tous ces amis, cinq sont mathématiciens, deux sont physiciens, et tous sont des scientifiques - ce qui semble montrer que le milieu le plus proche de moi dans ces années est resté un milieu de scientifiques, et surtout de mathématiciens.} (6). C'est Claude Chevalley. Alors qu'il ne faisait pas de discours et n'était pas intéressé par les miens, je crois pouvoir dire que j'ai appris de lui des choses plus importantes et plus cachées que celle que je viens de dire. Aux temps où je le fréquentais assez régulièrement (les temps du groupe "Survivre", auquel il s'était joint avec une conviction mitigée), souvent il me déroutait. Je ne saurais dire comment, mais je sentais qu'il détenait une connaissance qui m'échappait, une compréhension de certaines choses essentielles et toutes simples sûrement, qui peuvent s'exprimer par des mots simples certes, mais sans que pour autant la compréhension "passe" de l'un à l'autre. Je me rends compte maintenant qu'il y avait une différence de maturité entre lui et moi, qui faisait que souvent je me sentais en porte-à-faux vis à vis de lui, dans une sorte de dialogue de sourds qui n'était pas le fait d'un manque de sympathie mutuelle ou d'estime. Sans qu'il se soit exprimé en ces termes (pour autant que je me souvienne), il devait être clair pour lui que les "remises en question" (sur le "rôle social du scientifique", de la science, etc...) auxquelles j'arrivais alors, soit seul, soit par la logique d'une réflexion et d'une activité communes au sein du groupe "Survivre" (devenu par la suite "Survivre et Vivre")-que ces remises en question restaient au fond superficielles. Elles concernaient le monde dans lequel je vivais, certes, et le rôle que j'y jouais même - mais elles ne m'impliquaient pas vraiment de façon profonde. Ma vision de ma propre personne, pendant ces années bouillonnantes, n'a pas changé d'un poil. Ce n'est pas alors que j'ai commencé à faire connaissance avec moi-même. C'est six ans plus tard seulement que pour la première fois de ma vie je me suis débarrassé d'une illusion tenace, non pas sur les autres ou sur le monde environnant mais sur moi-même. Ça a été un autre réveil, d'une portée plus grande que le premier qui l'avait préparé. C'était un des premiers dans toute une "cascade" de réveils successifs, qui, je l'espère, va se poursuivre encore dans les années qui me restent dévolues.

Je ne me rappelle pas que Chevalley ait fait allusion en quelque occasion à la connaissance de soi, ou la "découverte de soi", pour mieux dire. Rétrospectivement, il est clair pourtant qu'il devait avoir commencé à faire connaissance avec lui-même depuis belle lurette. Il lui arrivait parfois de parler de lui-même, juste quelques mots à l'occasion de ceci ou cela, avec une simplicité déconcertante. Il est une des deux ou trois personnes que je n'ai pas entendues sortir de cliché. Il parlait peu, et ce qu'il disait exprimait, non des idées qu'il aurait adoptées et faites siennes, mais une perception et une compréhension personnelle des choses. C'est pourquoi sûrement il me déconcertait souvent, déjà aux temps où nous nous rencontrions encore au sein du groupe Bourbaki. Ce qu'il disait bousculait souvent des façons de voir qui m'étaient chères, et que pour cette raison je considérais comme "vraies". Il y avait en lui une autonomie intérieure qui me faisait défaut, et que j'ai commencé à percevoir obscurément aux temps de "Survivre et Vivre". Cette autonomie n'est pas de l'ordre de l'intellect, du discours. Ce n'est pas une chose qu'on peut "adopter", comme des idées, des points de vue, etc... L'idée ne me serait jamais venue, heureusement, de vouloir "faire mienne" cette autonomie perçue dans une autre personne. Il fallait que je trouve ma propre autonomie. Cela signifie aussi : que j'apprenne (ou réapprenne) à être moi-même. Mais en ces années, je ne me doutais nullement de mon manque de maturité, d'autonomie intérieure. Si j'ai fini par le découvrir, sûrement la rencontre avec Chevalley a été parmi les ferments qui ont travaillé en moi en silence, alors que j'étais embarqué dans de grands projets. Ce ne sont pas des discours ni des mots qui ont semé ce ferment-là. Pour le semer, il a suffi que telle personne rencontrée au hasard de ma route se passe de discours, et se contente d'être elle-même.

Il me semble qu'en ces débuts des années soixante-dix, quand nous nous rencontrions régulièrement à l'occasion de la publication du bulletin "Survivre et Vivre", Chevalley essayait, sans insistance, de me communiquer un message que j'étais alors trop pataud pour saisir, ou trop enfermé dans mes tâches militantes. Je me rendais compte obscurément qu'il avait quelque chose à m'apprendre sur la liberté - sur la liberté intérieure. Alors que j'avais tendance à fonctionner à coups de grands principes moraux et avais commencé à entonner cette trompette-là dès les premiers numéros de Survivre, comme chose allant de soi, il avait une aversion particulière pour le discours moralisateur. C'était je crois la chose qui me déroutait le plus en lui, aux débuts de Survivre. Pour lui, un tel discours était juste une tentative de contrainte, se superposant à une multitude d'autres contraintes extérieures étouffant la personne. On peut passer sa vie bien sûr à discuter une telle façon de voir, le pour et le contre. Elle bousculait totalement la mienne, animée (on s'en doute) par les plus nobles et généreux sentiments. J'étais peiné, il était incompréhensible pour moi que Chevalley, pour qui j'avais la plus grande estime et avec qui je me retrouvais un peu comme un compagnon d'armes, prenne un malin plaisir à ne pas partager ces sentiments ! Je ne comprenais pas que la vérité, la réalité des choses, n'est une question ni de bons sentiments, ni de points de vue ou de préférences. Chevalley voyait une chose, tout ce qu'il y a de simple et réelle, et je ne la voyais pas. Ce n'est pas qu'il l'avait lue quelque part ; il n'y a rien de commun entre voir une chose, et lire quelque chose à son sujet. On peut lire un texte à la rigueur avec ses mains (en écriture Braille) ou avec ses oreilles (si quelqu'un vous en fait la lecture), mais on ne peut voir la chose elle-même qu'avec ses propres yeux. Je ne crois pas que Chevalley avait de meilleurs yeux que moi. Mais il les utilisait, et moi non. J'étais trop pris par mes bons sentiments et le reste pour avoir le loisir de regarder l'effet de mes bons sentiments et principes sur ma propre personne et sur celle d'autrui, à commencer par mes propres enfants.

Il devait bien voir que souvent je ne me servais pas de mes yeux, que je n'en avais pas la moindre envie même. C'est étrange qu'il ne me l'ait jamais laissé entendre. Ou l'a-t-il fait, sans que j'entende ? Ou s'est-il abstenu, jugeant que c'était peine perdue? Ou peut-être l'idée même ne lui serait pas venue - c'était mon affaire après tout et non la sienne, si je me servais de mes yeux ou non !

