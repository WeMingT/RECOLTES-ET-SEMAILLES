\section{(46) Le fruit défendu}

J'ai dû m'interrompre deux jours dans les notes. Après relecture attentive, il me semble bien que le scénario qui précède est bien, grosso modo, une description de la réalité, description qu'il faudrait maintenant fouiller un peu plus. Il me faudrait surtout cerner de plus près les mérites respectifs des deux "chevaux" méditation et mathématique; et aussi essayer de comprendre quels événements ou conjonctures ont fini par déclencher le "basculement" dans la mise du patron, à l'encontre des forces d'inertie qui le pousseraient plutôt à conserver indéfiniment une mise même perdante.

Il faudrait peut-être aussi sonder les préférences du même. C'est une chose maintenant entendue, il a envie de changer de jeu de temps en temps, et le patron apparemment a un minimum de souplesse pour pas le forcer coûte que coûte à jouer toujours à ceci et jamais à cela. Depuis quelques années il a appris à tenir compte du môme, à composer avec lui, sans attendre que des marmites explosent. Ce n'est pas l'harmonie complète, mais ce n'est plus la guerre, une sorte d'entente cordiale plutôt, que les tensions occasionnelles auraient tendance à assouplir, non à durcir.

Quand il n'est pas contré trop durement, le môme est de nature assez souple dans ses préférences. (C'est pas comme le patron, qui a fini par apprendre un minimum de souplesse à son corps défendant seulement et sur ses vieux jours...) Mais que le même soit souple ne signifie pas qu’il n’ait de préférence, lui aussi, qu’il ne soit attiré plus fortement par une chose, que par une autre.

Ce n'est pas du tout évident souvent d'y voir clair, de distinguer entre les désirs du même et les préférences du patron, ou même ce que le patron a décidé une bonne fois pour toutes. Quand je me suis dit naguère : la méditation est meilleure, plus importante, plus sérieuse et tout et tout que la mathématique, pour telles et telles raisons (des plus pertinentes, on s'en doute), c'était le patron qui se donnait de bonnes raisons après coup pour se convaincre que la mise qu'il faisait était bel et bien "la bonne". Le même il dit pas que telle chose est "meilleure","plus importante" que telle autre. Il n'est pas porté sur le discours. Quand il a envie de faire quelque chose il y va si personne ne l'empêche, sans se poser de question si cette chose est "importante" ou "meilleure". Ses envies sont plus ou moins fortes d'une chose à l'autre et d'un moment à l'autre. Pour déceler ses préférences, rien ne sert d'écouter les discours explicatifs du patron, quand il prétend parler au nom du môme alors qu'il ne peut parler que de lui-même. C'est seulement en observant le môme dans ses jeux qu'on peut peut-être déceler ses prédilections. Et même alors c'est pas si évident : quand il joue à ceci avec entrain, ça ne signifie pas toujours qu’il ne jouerait à autre chose avec ravissement, si le patron n’y mettait son coup de pouce à lui.

Visiblement, ce qui avant toute autre chose l'attire, c'est l'inconnu - c'est poursuivre dans les nébuleux replis de la nuit et amener au grand jour, ce qui est inconnu et de lui, et de tous. Et j'ai l'impression que quand j'ai ajouté "et de tous", il s'agit bien là du désir de l'enfant, et non d'une vanité du patron, qui veut épater la galerie et lui-même. C'est une chose entendue aussi que ce que le même ramène à chaque coup de la pénombre de greniers et de caves inépuisables, c'est des choses "évidentes", enfantines. Plus elles apparaissent évidentes, plus même il est content. Si elles ne le sont, c'est qu'il n'a pas fait son boulot jusqu'au bout, qu'il s'est arrêté à mi-chemin entre l'obscurité et le jour.

En maths, les choses "évidentes", ce sont celles aussi sur lesquelles tôt ou tard quelqu'un doit tomber. Ce ne sont pas des "inventions" qu'on peut faire ou ne pas faire. Ce sont des choses qui sont déjà là depuis toujours, que tout le monde côtoie sans y faire attention, quitte à faire un grand détour autour, ou à passer par dessus en trébuchant à tous les coups. Au bout d'un an ou de mille, infailliblement, quelqu'un finit par faire attention à la chose, à creuser autour, la déterrer, la regarder de tous côtés, la nettoyer, et enfin lui donner un nom. Ce genre de travail, mon travail de prédilection, un autre chaque fois pouvait le faire, et ce qui plus est, un autre ne pouvait manquer de le faire un jour ou l’autre \footnote{Il va sans dire que je fais ici abstraction de l’hypothèse, nullement improbable à dire le moins, de l’irruption inopinée d’une guerre atomique ou d'une autre réjouissance du même genre, de nature à mettre fi n brutalement et une fois pour toutes au jeu collectif appelé "Mathématiques", et à bien autre chose avec...} (44).

Ce n'est pas du tout pareil pour la découverte de moi, dans le jeu nullement collectif "méditation". Ce que je découvre, nulle autre personne au monde, aujourd'hui ni à aucun autre moment, ne peut le découvrir à ma place. C'est à moi seul qu'il appartient de le découvrir, c'est-à-dire aussi : l'assumer. Cet inconnu-là n'est pas promis à être connu, par la force des choses presque, que je prenne ou non la peine de m'y intéresser. S'il attend dans le silence le moment où il sera connu, et si parfois, quand le temps est mûr, je l'entends qui appelle, il n'y a que moi seul, l'enfant en moi, qui est appelé à le connaître. Ce n'est pas un inconnu en sursis. Bien sûr, je suis libre de suivre son appel, ou de m'y dérober, de dire "demain" ou "un jour". Mais c'est à moi et à nul autre que s'adresse l'appel, et nul autre que moi ne peut l'entendre, nul autre ne peut le suivre.

Chaque fois que j'ai suivi cet appel, quelque chose a changé dans "l'entreprise", peu ou prou. L'effet a été immédiat, et ressenti sur le champ comme un bienfait - parfois, comme une libération soudaine, un soulagement immense, d'un poids que je portais sans même m'en rendre compte souvent, et dont la réalité se manifeste par ce soulagement, par cette libération. Sur un diapason de moindre amplitude, de telles expé- riences sont courantes dans tout travail de découverte, et j’ai eu l’occasion d’en parler. La chose cependant qui distingue le travail de découverte de soi (qu'il se fasse au grand jour ou qu'il reste souterrain) de tout autre travail de découverte, c'est justement qu'il change vraiment quelque chose dans "l'entreprise" elle-même. Il ne s'agit pas d'un changement quantitatif, une augmentation dans le rendement, ou une différence dans la taille ni même dans la qualité des produits sortant de l'atelier. Il s'agit d'un changement dans la relation entre le patron et l'ouvrier-enfant. Peut-être même y a-t-il un changement dans le patron lui-même, si ça peut avoir un sens autre que pour sa relation à l'ouvrier, au même. Par exemple il regardera peut-être moins à la production - mais c'est aussi un aspect de sa relation à l'ouvrier, par l'apparition d'un souci ou d'un respect peut-être qui auparavant lui étaient étrangers. Dans tous les cas où j’ai médité, le changement était dans le sens d'une clarification et d'un apaisement dans les relations entre patron et ouvrier. Sauf dans certains cas où la méditation est restée superficielle, des méditations "de circonstance" sous la seule pression d'un besoin immédiat et limité, la clarification a duré jusqu'à aujourd'hui, et l'apaisement aussi.

Cela donne au travail de découverte de soi un sens différent de tout autre travail de découverte, alors que beaucoup d'aspects essentiels sont communs. Il y a une dimension dans la connaissance de soi, et dans le travail de découverte de soi, qui les distingue de toute autre connaissance et de tout autre travail. Peut- être est-ce là le "fruit défendu" de l' Arbre de Connaissance. Peut-être la fascination qu'a exercé sur moi la méditation, ou plutôt celle des mystères dont elle m'a révélé l'existence, est-elle la fascination du fruit défendu. J'ai franchi un seuil, où la peur a disparu. Le seul obstacle à la connaissance est une inertie, une inertie parfois considérable, mais finie, nullement insurmontable. Cette inertie, je l'ai sentie presque à chaque pas, insidieuse, omniprésente. Elle m'a exaspérée parfois, mais jamais découragée. (Pas plus que dans le travail mathématique, où c'est elle aussi qui est le principal obstacle, mais d'un poids incomparablement moindre.) Cette inertie devient un des ingrédients essentiels du jeu ; un des protagonistes pour mieux dire, dans ce jeu délicat et nullement symétrique qui en comporte deux - ou trois pour mieux dire : d'un côté l'enfant qui s'élance, et le patron (fait inertie) qui freine tout ce qu’il peut (tout en prétendant ne pas y être), et de l’autre la forme entrevue de la belle inconnue, riche de mystère, à la fois proche et lointaine, qui à la fois se dérobe et appelle...