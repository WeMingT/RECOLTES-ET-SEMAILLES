\section{(49) Constat d'une division}

Cette courte réflexion sur le sens du présent travail, et sur le don et sur l'accueil, vient comme une digression dans le fil de la réflexion ; ou comme une illustration plutôt de certains aspects qui distinguent la "méditation" de tout autre travail de découverte, et notamment du travail mathématique. Je me suis rendu compte, hier, que ces aspects-là ont un double effet, savoir deux effets en sens opposé : une fascination unique sur "le môme", et un total désintérêt pour le "patron". Il semble bien que ce double effet est dans la nature des choses, qu'il ne peut absolument pas être atténué, par quelque compromis ou aménagement. Quoi qu'on fasse, quand le môme suit sa vraie prédilection, le patron n'y trouve pas son compte, mais pas du tout !

Nul doute que c'est là le sens du basculement qui a eu lieu, qui pourrait bien faire table rase de la méditation dans ma vie dans les années qui viennent (à l'exception des "méditations de circonstance", comme il y a trois mois). Je ne pense pas que celles-ci doivent être des années entièrement stériles pour cela, pas plus que l’année passée n'a été stérile. Mais il est vrai aussi que ce que j'y ai appris (en dehors des maths) est minime, si je le compare à ce que j'ai appris dans une quelconque des quatre années qui ont précédé. La chose étrange, c'est que chacune des quatre longues périodes de méditation que j’ai vécues étaient des temps de grande plénitude, sans rien qui puisse laisser soupçonner que quelque chose en moi restait frustré. Pourtant, si des marmites ont explosé, c'est que quelque part il y avait une pression, et cette pression ne devait pas être du jour même ; elle a dû être présente, quelque part hors de ma vue, pendant des semaines ou des mois, alors que j’étais intensément et totalement absorbé par la méditation.

Mais là je me laisse emporter par l’élan de la plume (ou plutôt, de la machine à écrire). La réalité, c’est que (sauf dans la dernière période de méditation, qui a été coupée en plein élan par un concours d’événements et de circonstances), l'intensité de la méditation décroissait progressivement à partir d'un moment, comme une vague justement qui allait être suivie par une autre s'apprêtant à prendre sa place... Le sentiment de plénitude, à vrai dire, suivait ce même mouvement, avec cette différence qu’il n’était présent qu’aux temps des vagues-méditation, et pas des vagues-"mathématique".

La situation que j’essaye de cerner n’est plus, il me semble, une situation de conflit, mais il devient apparent qu'elle renferme encore le germe, la potentialité du conflit. Elle est à présent pour moi le signe peut-être le plus visible, par son impact sur le cours de ma vie, d'une division en moi. Cette division n'est autre que la division patron-enfant.

Je ne puis y mettre fin. Tout ce que je peux faire, maintenant qu'elle est bien décelée, dans cette manifestation-là, c'est y être attentif, en poursuivre les signes et l'évolution au cours des mois et années qui sont devant moi. Peut-être cette passion pour les maths, un peu malencontreuse il faut bien dire, va-t-elle se consumer à force de brûler (comme s’est déjà consumée une autre passion en moi…), pour laisser place à la seule passion de la découverte de moi et de mon destin.

Cette passion est assez vaste, je l'ai dit, pour emplir ma vie - et sûrement ma vie entière ne suffira pas à l'épuiser.