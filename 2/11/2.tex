\section{(47) L’aventure solitaire}

Cette fascination sur moi de la "méditation" a été d'une puissance considérable - aussi puissante que naguère l'attirance de "la femme", dont elle semble avoir pris la place. Si je viens d'écrire "a été", cela ne signifie pas que cette fascination soit aujourd'hui éteinte. Depuis un an que je m'investis dans les mathématiques, elle a passé seulement à l'arrière-plan. L'expérience me dit que cette situation peut se renverser du jour au lendemain, tout comme cette situation est elle-même l'effet d'un renversement entièrement imprévu. En fait, au cours de chacune des quatre longues périodes de méditation par lesquelles j’ai passé (dont l’une s’est étendue sur près d'un an et demi), c'était une chose qui pour moi allait de soi que j'allais continuer sur ma lancée jusqu'à mon dernier soupir, pour sonder aussi loin que je pourrai aller les mystères de la vie et celles de l'existence humaine. Quand les notes se sont accumulées en piles impressionnantes au point de menacer de submerger ma chambre de travail, j'ai même fini par faire faire un meuble sur mesure pour les caser, en prévoyant large (par un rapide calcul de progression arithmétique) pour y caser aussi celles qui ne tarderaient pas à s'y rajouter au fil des années ; j'avais prévu une marge d'une quinzaine d'années si je me rappelle bien (ce qui commençait déjà à faire !). Là le patron avait bien fait les choses, pour de l’intendance c’était de la belle intendance ! Ça, et un rangement de grande envergure de tous les papiers personnels liés de près ou de loin au travail de méditation, a été d'ailleurs sa dernière tâche entreprise et menée (presque) à bonne fin, juste avant le basculement de préférence et de mises. C'est à se demander s'il n'avait pas une arrière-pensée en tête, et s'il ne voyait pas déjà des tomes de "Réflexions Mathématiques" remplir les rayons vides soi-disant destinés aux "Notes" à venir.

Certes, la passion de la méditation, de la découverte de moi est assez vaste pour emplir ma vie jusqu'à la fin de mes jours. Il est vrai aussi que la passion mathématique n'est pas consumée, mais peut-être cette faim-là va-t-elle finir par se rassasier dans les années qui viennent. Quelque chose en moi le souhaite, et ressent la mathématique comme une entrave à suivre une aventure solitaire que je suis seul à pouvoir poursuivre. Et il me semble que ce "quelque chose" en moi n'est pas le patron, ni une des velléités du patron (lequel, par nature, est divisé). Il me semble que la passion mathématique porte encore la marque du patron, et en tous cas. que de la suivre fait mouvoir ma vie dans un cercle fermé ; dans le cercle d'une facilité, et dans un mouvement qui est celui d'une inertie, sûrement pas d'un renouvellement.

Je me suis interrogé sur le sens de cette persistance opiniâtre de la passion mathématique dans ma vie. Quand je la suis, elle n'emplit pas vraiment ma vie. Elle donne des joies, et elle donne des satisfactions, mais elle n'est pas de nature par elle-même à donner un véritable épanouissement, une plénitude. Comme toute activité purement intellectuelle, l'activité mathématique intense et de longue haleine a un effet plutôt abrutissant. Je le constate chez autrui, et surtout chez moi-même chaque fois que je m’y adonne à nouveau. Cette activité est si fragmentaire, elle ne met en oeuvre qu'une partie si infime de nos facultés d'intuition, de sensibilité, que celles-ci s'émoussent à force de ne pas servir. Pendant longtemps je ne m'en étais pas rendu compte, et visiblement la plupart de mes collègues ne s'en rendent pas plus compte que moi dans le temps. C'est depuis que je médite seulement, il me semble, que je suis devenu attentif à cette chose-là. Pour peu qu'on y prête attention, elle crève les yeux - les maths à grosses doses épaissit. Même après la méditation d'il y a deux ans et demi, où la passion mathématique a été reconnue comme une passion en effet, comme une chose importante dans ma vie - quand maintenant je me donne à cette passion, il reste une réserve, une réticence, ce n'est pas un don total. Je sais qu'un soi-disant "don total" serait en fait une sorte d'abdication, ce serait suivre une inertie, ce serait une fuite, non un don.

Il n'y a aucune telle réserve en moi pour la méditation. Quand je m'y donne, je m'y donne totalement, il n'y a trace de division dans ce don. Je sais qu'en me donnant, je suis en accord complet avec moi-même et avec le monde - je suis fidele à ma nature, "je suis le Tao". Ce don-là est bienfaisant à moi-même et à tous. Il m'ouvre à moi-même comme à autrui, en dénouant avec amour ce qui en moi reste noué.

La méditation m’ouvre sur autrui, elle a pouvoir de dénouer ma relation à lui, alors même que l’autre reste-rait noué. Mais il est très rare que se présente l'occasion de communiquer avec autrui si peu que ce soit au sujet du travail de méditation, de telle ou telle chose que ce travail m’a fait connaître. Ce n’est nullement parce qu’il s'agirait de choses "trop personnelles". Pour prendre une image imparfaite, je ne peux communiquer sur des maths qui m’intéressent à un moment donné, qu’avec un mathématicien qui dispose du bagage indispensable, et qui au même moment est disposé à s'y intéresser également. Il arrive que pendant des années je sois fas-ciné par telles choses mathématiques, sans rencontrer (ni même chercher à rencontrer) d’autre mathématicien avec qui communiquer à leur sujet. Mais je sais bien que si j'en cherchais, j'en trouverais, et que même si je n'en trouvais pas, ce serait simple question de chance ou de conjoncture ; que les choses qui m'intéressent ne pourront manquer d'intéresser quelqu'un et même quelques-uns, que ce soit dans dix ans ou dans cent ans peu importe au fond. C'est cela qui donne un sens à mon travail, même si celui-ci se fait dans la solitude. S'il n'y avait d'autres mathématiciens au monde et qu'il ne doive plus y en avoir, je ne crois pas que faire des maths garderait un sens pour moi - et je soupçonne qu’il n’en va pas autrement pour tout autre mathématicien, ou tout autre "chercheur" en quoi que ce soit. Cela rejoint la constatation faite précédemment, que pour moi "l'inconnu mathématique" est ce que personne encore ne connaît - c'est une chose qui ne dépend pas de ma seule personne, mais d'une réalité collective. La mathématique est une aventure collective, se poursuivant depuis des millénaires.

Dans le cas de la méditation, pour communiquer à son sujet, la question d'un "bagage" ne se pose pas ; pas au point où j'en suis tout au moins, et je doute qu'elle se posera jamais. La seule question est celle d'un intérêt en autrui, qui réponde à l'intérêt qui est en moi. Il s'agit donc d'une curiosité vis-à-vis de ce qui ce passe réellement en soi-même et en autrui, au-delà des façades de rigueur, qui ne cachent pas grand-chose du moment qu'on est vraiment intéressé à voir ce qu'elles recouvrent. Mais j'ai appris que les moments où dans une personne apparaît un tel intérêt, les "moments de vérité", sont rares et fugitifs. Il n'est pas rare, bien sûr, de rencontrer des personnes qui "s'intéressent à la psychologie", comme on dit, qui ont lu du Freud et du Jung et bien d'autres, et qui ne demandent pas mieux que d'avoir des "discussions intéressantes". Ils ont ce bagage qu'ils transportent avec eux, plus ou moins lourd ou léger, ce qu'on appelle une "culture". Il fait partie de l'image qu'ils ont d'eux même, et renforce cette image, qu'ils se gardent d'examiner jamais, exactement comme tel autre qui s'intéresse aux maths, aux soucoupes volantes ou à la pêche à la ligne. Ce n'est pas de ce genre de "bagage", ni de ce genre "d'intérêt", que j'ai voulu parler tantôt - alors que les mêmes mots ici désignent des choses de nature différente.

Pour le dire autrement : la méditation est une aventure solitaire. Sa nature est d'être solitaire. Non seule-ment le travail de la méditation est un travail solitaire - je pense que cela est vrai de tout travail de découverte, même quand il s'insère dans un travail collectif. Mais la connaissance qui naît du travail de méditation est une connaissance "solitaire", une connaissance qui ne peut être partagée et encore moins "communiquée"; ou si elle peut être partagée, c'est seulement en de rares instants. C'est un travail, une connaissance qui vont à contre-courant des consensus les plus invétérés, ils inquiètent tous et chacun. Cette connaissance certes s'exprime simplement, par des mots simples et limpides. Quand je me l'exprime, j'apprends en l'exprimant, car l'expression même fait partie d'un travail, porté par un intérêt intense. Mais ces mêmes mots simples et limpides sont impuissants à communiquer un sens à autrui, quand ils se heurtent aux portes closes de l'indif-férence ou de la peur. Même le langage du rêve, d'une toute autre force et aux ressources infinies, renouvelé sans cesse par un Rêveur infatigable et bienveillant, n'arrive à franchir ces portes-là . .

Il n'y a de méditation qui ne soit solitaire. S'il y a l'ombre d'un souci d'une approbation par quiconque, d'une confirmation, d'un encouragement, il n'y a travail de méditation ni découverte de soi. La même chose est vraie, dira-t-on, de tout véritable travail de découverte, au moment même du travail. Certes. Mais en dehors du travail proprement dit, l'approbation par autrui, que ce soit un proche, ou un collègue, ou tout un milieu dont on fait partie, cette approbation est importante pour le sens de ce travail dans la vie de celui gui s'y donne. Cette approbation, cet encouragement sont parmi les plus puissants incentifs, qui font que le "patron" (pour reprendre cette image) donne le feu vert sans réserve pour que le même s'en donne à coeur joie. Ce sont eux surtout qui déterminent l'investissement du patron. Il n'en a pas été autrement dans mon propre investissement dans la mathématique, encouragé par la bienveillance, la chaleur et la confiance de personnes comme Cartan, Schwartz, Dieudonné, Godement, et d'autres après eux. Pour le travail de méditation par contre, il n'y a nul tel incentif. C'est une passion du même-ouvrier que le patron est au fond gentil de tolérer peu ou prou, car elle ne "rapporte" rien. Elle porte des fruits, certes, mais ce ne sont pas ceux auxquels un patron aspire. Quand il ne se berne pas lui-même à ce sujet, il est clair que ce n'est pas dans la méditation qu'il va investir, le patron est de nature grégaire !

Seul l'enfant par nature est solitaire.