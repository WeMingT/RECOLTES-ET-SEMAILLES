\section{(48) Don et accueil}

En parlant hier de l'essence solitaire de la méditation, j'ai été effleuré par la pensée que les notes que j’écris depuis bientôt six semaines, qui ont fini par devenir une sorte de méditation, sont pourtant destinées à la publication. Cela a d'ailleurs, forcément, influé sur la forme de la méditation de bien des façons, notamment par le souci d'une concision, et aussi celui d'une discrétion. Un des aspects essentiels de la méditation, savoir une attention constante à ce qui se passe en moi au moment même du travail, ne s'est manifesté que très occasionnellement, et de façon superficielle. Sûrement tout cela a dû influer sur le cours du travail et sur sa qualité. Je sens pourtant qu'il a qualité de méditation, avant tout par la nature de ses fruits, par l'apparition d'une connaissance de moi-même (en l'occurrence, celle d'un certain passé surtout) que j'avais jusqu'à pré- sent éludée. Un autre aspect est la spontanéité, qui a fait que pour aucune des bientôt cinquante "sections" ou

"paragraphes" en quoi spontanément la réflexion s'est groupée, je n'aurais su dire en la commençant quelle en serait la substance ; à chaque fois celle-ci se révélait en cours de route seulement, et à chaque fois le travail amenait au jour des faits nouveaux, ou éclairait sous un jour nouveau des faits jusque là négligés.

Le sens le plus immédiat de ce travail a été celui d'un dialogue avec moi-même, d'une méditation donc. Pourtant, le fait que cette méditation-là est destinée à être publiée, et de plus, à servir comme une "ouverture" aux "Réflexions Mathématiques" qui doivent suivre, n'est nullement une circonstance accessoire, qui aurait été lettre morte au cours du travail. Elle fait pour moi partie essentielle du sens de ce travail. Si j’ai laissé entendre hier que le patron sûrement y trouve son compte (lui qui est passé maître pour "trouver son compte" en tout, ou peu s'en faut !), cela ne signifie nullement que son sens se réduise à cela - à un "retour" tardif, posthume presque, du fameux cheval à trois pattes ! Plus d'une fois aussi j'ai senti que le sens profond d'un acte dépasse parfois les motivations (apparentes ou cachées) qui l’inspirent. Et dans ce "retour à la mathématique" je devine un autre sens encore que d'être le résultat-somme de certaines forces psychiques qui se sont trouvées en présence dans ma personne à tel moment et pour telles raisons.

Cette "méditation" que je suis en train de poursuivre pour l'offrir à ceux que j'ai connus et aimés dans le monde mathématique - si je sens qu'elle est une part importante de ce sens entrevu, ce n'est pas dans l'expectative que le don sera accueilli. S'il est accueilli ou non ne dépend pas de moi, mais de celui seulement à qui il s'adresse. Qu'il soit accueilli ne m'est nullement indifférent, certes. Mais ce n'est pas là ma responsabilité. Ma seule responsabilité est d’être vrai dans le don que je fais, c’est-à-dire aussi, d’être moi-même.

Ce que me fait connaître la méditation sont les choses humbles et évidentes, des choses qui ne payent pas de mine. Ce sont celles aussi que je ne trouverai dans aucun livre ou traité, si savant, profond, génial soit-il - celles que nul autre ne peut trouver pour moi. J'ai interrogé un "brouillard", j'ai pris la peine de l'écouter, j'ai appris une humble vérité sur une "attitude sportive" et son sens évident, dans ma relation à la mathématique comme dans ma relation à autrui. J'aurais lu "dans le texte" les Saintes Ecritures, le Coran, les Upanishads, et encore Platon, Nietzsche, Freud et Jung par dessus le marché, je serais un prodige d'érudition vaste et profonde - que tout cela n'aurait fait que m'éloigner de cette vérité-là, une vérité enfantine, évidente. Et j'aurais répété cent fois les paroles du Christ "heureux sont ceux qui sont comme les petits enfants, car le Royaume des Cieux leur appartient", et les aurai commentées finement, que cela n'aurait servi encore qu'à me tenir éloigné de l'enfant en moi, et des humbles vérités qui m'incommodent et que l'enfant seul voit. Ce sont ces choses-là, le meilleur que j’aie à offrir.

Et je sais bien que quand de telles choses sont dites et offertes, en des mots simples et limpides, elles ne sont pas accueillies pour autant. Accueillir, ce n'est pas simplement recevoir une information, avec gêne ou même avec intérêt : "Ça alors, qui se serait douté… !", ou : "Ce n'est pas tellement étonnant après tout...". Accueillir, souvent, c'est se reconnaître dans celui qui offre. C'est faire connaissance avec soi-même à travers la personne d'autrui.