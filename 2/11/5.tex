\section{(50) Le poids d’un passé}

Cela fait quelques jours que j’ai terminé de mettre la dernière main à "Récoltes et Semailles" - après avoir cru, pendant plus d'un mois, que j'étais sur le point de terminer dans les jours prochains. Même cette fois-ci, après avoir mis "la dernière main", je n'étais pas entièrement sûr pourtant si j’avais bel et bien terminé - il restait une question en effet que j’avais laissée en suspens. C’était de "comprendre quels événements ou conjonctures ont fini par déclencher le "basculement" dans la mise "du patron"", en faveur de la mathématique en lieu et place de la méditation, à l'encontre de forces d'inertie considérables. Sans propos délibéré mes pensées sont revenues avec une certaine insistance à cette question, en ces derniers jours où pourtant j'avais commencé déjà à embrancher sur d'autres de tout autre ordre, y compris des questions mathématiques (de géométrie conforme). Autant profiter encore de cette "fin de lancée" méditante, pour creuser tant soit peu et laisser place nette.

Plusieurs associations se présentent, quand j'essaye de répondre "au pif" pourquoi "je me remets aux maths" (dans le sens d'un investissement important et prévu pour être de longue haleine, de l'ordre tout au moins de quelques années). La plus forte peut-être de toutes se rapporte au sentiment de frustration chronique que j’ai fini par ressentir dans mon activité enseignante depuis six ou sept ans. Il y a ce sentiment de plus en plus fort d'être "sous-employé", et même, bien souvent, de m'investir et de donner du meilleur de moi-même pour des élèves moroses qui n’ont que faire de ce que j’ai à donner.

Je vois partout des choses magnifiques à faire et qui ne demandent qu'à être faites. Souvent, il suffit d'un bagage dérisoire pour les aborder, ce sont ces choses elles-mêmes qui nous soufflent quel langage développer pour les cerner, et quel outillage acquérir pour les creuser. Je ne peux m'empêcher de les voir, du seul fait d'un contact régulier avec les maths (à un niveau si modeste soit-il) provenant d'une activité enseignante, même en les périodes de ma vie où mon intérêt pour les maths est des plus marginaux. Derrière chaque chose entrevue, si peu qu'on fouille, d'autres belles choses encore, qui en recouvrent et en révèlent d'autres à leur tour... Que ce soit en maths où ailleurs, où qu’on pose les yeux avec un véritable intérêt, on voit se révéler une richesse, s'ouvrir une profondeur qu'on devine inépuisables. La frustration dont je parle, c'est celle de ne pas arriver si peu que ce soit à communiquer à mes élèves ce sentiment de richesse - de profondeur - ne serait ce qu’une étincelle d'envie de faire le tour au moins de ce qui est juste à portée de leur main, de s'en donner à coeur joie pendant les quelques mois ou années qu'ils sont de toutes façons décidés à investir dans une activité dite "de recherche", aux fins de préparer tel ou tel diplôme. Sauf pour deux ou trois parmi les élèves que j’ai eus depuis dix ans, on dirait que l'idée même de "s'en donner à coeur joie" les effraye, qu'ils préfèrent pendant des mois et des années rester bras ballants à piétiner, ou à faire péniblement un travail de taupe dont ils ne connaissent ni les tenants ni les aboutissants, du moment qu’il y a le diplôme au bout. Il y aurait beaucoup à dire sur cette sorte de paralysie de la créativité, qui n'a rien à voir avec l'existence ou la non-existence de "dons" ou de "facultés" - et cela rejoint les tout débuts de ma réflexion, où j'avais effleuré en passant la cause profonde de tels blocages. Mais ce n'est pas là mon propos ici, qui est plutôt de constater l'état de frustration chronique que ces situations, constamment répétées tout au long de ces dernières sept années d'activité enseignante, ont fini par créer en moi.

La façon évidente de "résoudre" une telle frustration, dans la mesure au moins où c'est celle du "mathéma-ticien" en moi et non celle de l'enseignant, c'est de faire par moi-même au moins une partie de ces choses que je désespérais de voir empoigner à la fin des fins par l'un ou l'autre de mes élèves. C'est d'ailleurs ce que j'ai fait tant soit peu ici et là, que ce soit par une réflexion occasionnelle de quelques heures, voire de quelques jours, en marge et à l'occasion de mon activité enseignante, ou pendant des périodes de grosse frin-gale mathématique (qui survenaient parfois comme de véritables explosions...), pouvant durer des semaines ou des mois. Un tel travail occasionnel et par à-coups ne pouvait donner lieu le plus souvent qu’à un tout premier dégrossissage d'une question, et à une vision des plus fragmentaires - c'était plutôt une vision plus claire du travail en perspective, alors que ce travail lui-même reste toujours à faire et, pour être mieux vu, n'en paraît que plus brûlant. J'ai donné il y a deux mois une esquisse d'ensemble sur les principaux thèmes dont j'ai commencé tant soit peu à prendre la mesure. C'est l' "Esquisse d'un Programme", auquel j'ai déjà eu l'occasion de faire allusion, et qui sera joint finalement à la présente réflexion, pour constituer ensemble le volume 1 des "Réflexions Mathématiques".

Il est assez clair que ce seul travail de prospection ("privé" pour ainsi dire) ne pouvait suffire à résoudre ma frustration. Ce sentiment "d'être sous-employé" traduisait sûrement le désir (d'origine égotique, je crois, c'est-à-dire désir "du patron") d'exercer une action. Il s'agit ici moins de l'action sur autrui (sur mes élèves disons, les mettre én mouvement, leur "communiquer quelque chose", ou les aider à avoir tel diplôme qui pourrait leur permettre de postuler tels postes, etc...) que de l'action "de mathématicien" : contribuer à la découverte de tels faits insoupçonnés, à l'éclosion de telle théorie, etc... Cela s'associe immédiatement à la constatation faite précédemment, de ce fait que la mathématique est une "aventure collective". Si je m'in-terroge sur mes dispositions quand j'ai fait des maths au cours de ces dernières dix années, en une période de ma vie où l'idée ne me serait pas venue que je pourrais me remettre un jour à publier, et quand il était plus ou moins clair également qu'aucun de mes élèves présents ou futurs n'aurait que faire de mon travail de prospection - il m'apparaît aussitôt que ce n'étaient nullement pourtant des dispositions de quelqu'un qui ferait quelque chose pour son seul plaisir personnel, ou poussé par un besoin intérieur qui ne concernerait que lui-même, sans relation à autrui. Quand je fais des maths, je crois que quelque part en moi il est bien entendu que ces maths sont faites pour être communiquées à autrui, pour être part d'une chose plus vaste à laquelle je concours, une chose qui n'est nullement de nature individuelle. Cette "chose", je pourrais l'appeler "la mathé- matique", ou mieux "notre connaissance des choses mathématiques". Le terme "notre" ici réfère sans doute, en premier lieu, concrètement, au groupe surtout des mathématiciens que je connais et avec lesquels j’ai des intérêts en commun ; mais il est hors de doute aussi qu'il dépasse ce groupe restreint tout autant qu'il dépasse ma personne. Ce "notre" réfère à notre espèce, en tant que celle-ci, par certains de ses membres à travers les âges, s'est intéressée et s'intéresse aux réalités du monde des objets mathématiques. Je n'ai jamais, avant ce moment même où j’écris ces lignes, songé à l’existence de cette "chose" dans ma vie, et encore moins à m'interroger sur sa nature et sur son rôle dans ma vie de mathématicien et d'enseignant.

Le désir d'exercer une action auquel j'ai fait allusion, me semble prendre chez moi, dans ma vie de mathé- maticien, la forme suivante : faire sortir de l'ombre ce qui est inconnu de tous, non seulement de moi (comme je l'ai vu précédemment), et ceci, de plus, aux fins d'être mis à la disposition de tous, d'enrichir donc un "pa-trimoine" commun. En d'autres termes, c'est le désir de contribuer à l'agrandissement, à l'enrichissement de cette "chose", ou "patrimoine", qui dépasse ma personne.

Dans ce désir, certes, le désir d'agrandir ma personne à travers mes oeuvres n'est pas absent. Par cet aspect, je retrouve la fringale de "croissance", d'agrandissement, qui est une des caractéristiques du moi, du "patron" ; c’est là son aspect envahissant et,à la limite, destructeur (cf note \({44}^{\prime }§{13.1.1}\) , p. 260). Pourtant, je me rends compte aussi que le désir d'augmenter le nombre de choses qui (pour un temps court ou long) porteront plus ou moins mon nom, est loin d'épuiser, de recouvrir ce désir ou cette force plus vaste, qui me pousse à vouloir contribuer à agrandir un patrimoine commun. Il me semble qu'un tel désir pourrait trouver satisfaction (sinon "dans mon entreprise", où le patron reste assez envahissant, du moins chez un mathématicien d'une plus grande maturité) alors que le rôle de sa propre personne resterait anonyme. Ce serait peut-être là une forme "sublimée" de la tendance à l'agrandissement du moi, par identification avec une chose qui le dépasse. A moins que ce genre de force ne soit pas de nature égotique par elle-même, mais de nature plus délicate et plus profonde, qu'elle exprime un besoin profond, indépendant de tout conditionnement, qui atteste du lien profond entre la vie d'une personne et celle de l'espèce entière, un lien qui fait partie du sens de notre existence individuelle. Je ne sais, et ce n'est pas mon propos ici de sonder de telles questions, de portée aussi vaste.

Mon propos plutôt est d'examiner (dans une optique plus modeste) une situation concrète concernant ma personne : une situation de frustration donc, avec un exutoire partiel et provisoire par une activité mathéma-tique sporadique. La logique de la situation, dès lors, devait m'amener tôt ou tard à communiquer ce que je trouvais. Comme jusqu'à l'an dernier je n'étais nullement disposé à consentir pour ma passion mathématique l'investissement de grande envergure et de longue haleine qui aurait été nécessaire pour "exploiter" aux fins de publication, par un "travail sur pièces" circonstancié, les mines que je mettais à jour, il me restait l'alternative de communiquer à certains amis mathématiciens suffisamment "dans le coup" les choses au moins qui me tenaient le plus à coeur.

Je pense que si j'avais trouvé au cours des dernières dix années un ami mathématicien qui joue vis-à-vis de moi un rôle d'interlocuteur et de source d'information (comme cela avait été le cas de Serre dans une très large mesure, pendant de longues années dans les années 50 et 60), en même temps que de relais pour transmettre des "informations" que je pouvais lui transmettre (rôle que Serre n'avait pas eu à jouer jadis, car je m'en chargeais moi-même !), mon désir "d'exercer une action en maths" aurait trouvé une satisfaction suffi-sante pour résoudre ma frustration, tout en me contentant d'un investissement d'énergie épisodique et modéré dans les mathématiques, en laissant la plus large part à ma nouvelle passion. La première fois que je me suis adressé à un ami mathématicien avec une telle expectative (au moins implicite en moi) a été en 1975, et la dernière fois en 1982, il y a un an et demi. Coïncidence amusante, les deux fois c'était pour essayer de "placer" (aux fins qu'il soit répercuté et, qui sait, développé à la fin des fins !) un même "programme" d'algèbre homo-logique et homotopique, dont les premiers germes remontent aux années cinquante, et qui était parfaitement "mûr" (suivant l'intime conviction que j'en avais) dès avant la fin des années soixante ; programme dont un développement préliminaire et dans les grandes lignes est le thème justement de cette Poursuite des Champs dont je suis censé en ce moment écrire l' Introduction ! Toujours est-il que pour des raisons sans doute assez différentes d'un cas à l'autre, mes tentatives pour retrouver une relation "d'interlocuteur privilégié", comme il y en avait eu (avant 1970) avec Serre, et puis avec Deligne, ont tourné court. Une circonstance commune pourtant est la disponibilité relativement limitée que j'étais disposé à accorder aux maths. Cela a sûrement contribué, dans les deux occasions dont j’ai parlé (en 1975 et en 1982), à rendre la communication boiteuse. En fait, je cherchais surtout à "placer" quelque chose, sans trop me soucier de faire l'effort nécessaire de "(re)mise au courant" pour être de mon côté un interlocuteur satisfaisant pour mon correspondant, beaucoup plus "dans le coup" que moi (à dire le moins !) pour les techniques courantes en homotopie.

Je pourrais considérer la "Lettre à ..." qui sert de premier chapitre à la Poursuite des Champs (lettre de février l'an dernier, il y a à peine plus d'un an) comme ma dernière tentative pour trouver un écho, auprès d'un de mes amis d'antan, à certaines de mes idées et préoccupations de maintenant. La continuation de la réflexion commencée (ou plutôt, reprise) dans cette lettre allait devenir (sans que je m’en doute encore pendant des semaines) le premier texte mathématique depuis 1970 promis à une publication. C'est près d'une année plus tard seulement que j’ai reçu une réaction indirecte à cette substantielle lettre (comparer note \footnote{Ces notes étaient en fait la continuation de la longue lettre à. ..., qui en est devenue le premier chapitre. Elles étaient tapées à la machine pour être lisibles pour cet ami d'antan, et pour deux ou trois autres (dont surtout Ronnie Brown) dont je pensais qu'ils pourraient être intéressés. Cette lettre d'ailleurs n'a jamais réçu de réponse, et elle n'a pas été lue par le destinataire, qui près d'un an après (à ma question s'il l'avait bien reçue) se montrait sincèrement étonné que j'avais pu penser même un moment qu'il pourrait la lire, vu le genre de mathématiques qu'on devait attendre de moi…} (38)). Celle-ci a été plus éloquente qu'aucune autre lettre reçue à ce jour d'un collègue mathématicien, pour me faire sentir certaines dispositions vis-à-vis de ma modeste personne, devenues courantes parmi mes amis mathématiciens depuis que j’ai quitté le milieu dont je faisais partie avec eux. Il y a dans cette lettre, provenant de quelqu’un à qui je m'étais adressé comme à un ami, dans des dispositions de sympathie chaleureuse, un propos délibéré de dérision, qui m'a rappelé de façon particulièrement violente une chose dont j'avais fini par me rendre compte de plus en plus clairement au cours des dernières années. Précédemment, j'avais eu l'occasion surtout de remarquer une prise de distances à l'égard de ma personne elle-même, dans le "grand monde" mathématique, et avant tous autres, parmi ceux qui avaient été mes amis plus ou moins proches (45). Là il s'agit non plus de prise de distances au niveau des personnes, mais plutôt d'un consensus, dans la nature d'une mode et comme elle se présentant comme chose allant de soi, entre gens "dans le coup" tant soit peu : que le genre de maths par paquets de mille pages, et les notions avec lesquelles j'ai rabattu les oreilles des gens pendant une décennie ou deux (46,47), ne sont pas très sérieux à tout bien prendre ; qu'il y a là beaucoup de bombinage pour pas grand chose qui vaille, et qu'à part des tartines de "général non-sense" autour de la notion de schéma et de cohomologie étale (qui ont bien leur utilité parfois ; hélas, on veut bien le reconnaître), il est plus charitable d'oublier au moins le reste ; que ceux qui feraient mine néanmoins d'entonner encore ce genre de trompette grothendieckienne, en dépit du bon goût et des canons évidents de sérieux, sont à mettre dans le même sac que leur Maître, avoué ou non, et qu'ils n'ont qu'à s'en prendre à eux-mêmes s'ils sont traités comme ils le méritent...

Sûrement, les nombreux échos dans ce sens (que je viens de transcrire "en clair") qui me sont parvenus depuis 1976 (50), et surtout depuis deux ou trois ans, ont fini par réveiller en moi une fibre de combativité qui s'était quelque peu assoupie au cours des dernières dix années. Ils ont suscité, comme un réflexe, l'envie de me lancer dans la mêlée, de clore le bec à ces blancs-becs qui n'ont rien compris à rien - un réflexe complètement idiot en somme, celui du taureau à qui il suffit de montrer un bout d'étoffe rouge et l'agiter devant son nez, pour qu'aussitôt il se mette en frais et en branle, en oubliant le chemin qu'il était en train de suivre tranquille et qui était le sien ! Je crois quand même que ce réflexe est assez épidermique, et qu’il n’aurait pas suffi à lui seul à me faire m'ébranler. D'ailleurs et heureusement, faire des maths a nettement plus de charme que de foncer sur un bout d'étoffe en se faisant larder de tous côtés. Mais faire des maths, en poursuivant envers et contre tout un style de travail\} une approche des choses qui sont les miens, c'est aussi un peu "se jeter dans la mêlée"; c'est m'affirmer en face des signes d'un dédain, d'un rejet - qui me viennent, à n'en pas douter, en réponse au dédain que mes anciens amis ont senti ou crû sentir en moi, sinon à leur égard, du moins à l'égard d'un milieu auquel ils continuent à s'identifier sans réserve. C'est donc aussi, tant soit peu, suivre le bout d'étoffe rouge, au lieu de suivre mon chemin.

Cette idée-là s'était présentée à moi à plusieurs reprises, au cours de ces dernières semaines, et c'est peut- être vers un examen de cet aspect surtout que s'est acheminée la réflexion d'aujourd'hui. Chemin faisant, un autre aspect est apparu, où les forces du moi ont sûrement une large part également, mais qui ne s'apparente pas à un simple réflexe de combativité. Plutôt, à un désir qui est en moi, et dont en ce moment je ne discerne pas encore clairement la nature, de donner un sens au travail mathématique que j’ai fait en ces dernières dix ou douze années, ou de lui voir prendre tout son sens ; lequel sens (j'en ai l'intime conviction) ne peut se réduire à celui d'un plaisir privé ou d'une aventure personnelle. Mais même si la nature de ce désir reste incompris, alors que je n'ai pas pris le loisir de l'examiner de plus près, cette réflexion suffit à me montrer que c'est bien là, dans ce désir-là, que se trouve véritablement la force qui pèse sur moi et me force la main, pour ainsi dire, en faveur d'un investissement mathématique - la force de "basculement". Elle agirait tout autant. étoffe rouge ou pas. Si elle est signe d'un attachement à un passé, c'est le passé de ces dernières dix années, le passé "d'après 1970" donc, et non le passé des choses déjà écrites noir sur blanc, des choses faites, celles d'avant 1970.

Au fond, il n'y a en moi aucune inquiétude au sujet de ces choses, sur le sort que l'avenir, "la postérité" leur réservera (alors qu'il est douteux qu'il y ait même une postérité…). Ce qui m’intéresse dans ce passé, ce n'est nullement ce que j'y ai fait (et la fortune qui est ou sera la sienne), mais bien plutôt ce qui n'a pas été fait, dans le vaste programme que j'avais alors devant les yeux, et dont une toute petite partie seulement s'est trouvé réalisée par mes efforts et ceux des amis et élèves qui parfois ont bien voulu se joindre à moi. Sans l'avoir prévu ni cherché, ce programme lui-même s'est renouvelé, en même temps que ma vision et mon approche des choses mathématiques. Au fil des années, l'accent s'est déplacé tant pour les thèmes, que pour mon propos même : au lieu que ce soit l'accomplissement de grandes tâches de fondements méticuleux, mon tout premier propos maintenant est de sonder les mystères qui m'ont le plus fasciné, tel celui des "motifs", ou celui de la description "géométrique" du groupe de Galois de Q sur Q. Chemin faisant, certes, je ne pourrai m'empêcher tout au moins d'esquisser des fondements ça et là, comme j'ai commencé à le faire (entre autres) dans "La longue Marche à travers la théorie de Galois", ou comme je suis en train de le faire dans la Poursuite des Champs. Le propos pourtant a changé, et le style qui l'exprime.

Pour le dire autrement : j'ai entrevu en ces dernières dix années des choses mystérieuses et d'une grande beauté, dans le monde des choses mathématiques. Ces choses ne me sont pas personnelles, elles sont faites pour être communiquées - le sens même de les avoir entrevues, ainsi je le sens, c'est de les communiquer, pour être reprises, comprises, assimilées... Mais les communiquer, ne serait-ce qu'à soi-même, c'est aussi les approfondir, les développer tant soit peu - c'est un travail. Je sais bien, certes, qu'il n'est pas question que je mène au bout ce travail, même s’il me restait cent ans à y consacrer. Mais cela n’a pas à être mon souci aujourd'hui, combien d'années ou de mois je vais consacrer à ce travail-là sur le temps qui me reste à vivre et à découvrir le monde, alors qu’un autre travail m’attend que je suis seul à pouvoir faire. Il n’est pas en mon pouvoir, et ce n'est pas mon rôle, de régler les saisons de ma vie.