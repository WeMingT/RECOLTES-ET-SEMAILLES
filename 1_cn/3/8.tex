\section{“我的亲近者”——或共谋}

我无意在此信中逐一回顾《收获与播种》写作中的所有“高光时刻”(或“敏感时刻”),或其某阶段的具体细节\footnote{你可在“夜果”「Les fruits du soir;The Fruits of the Evening」(第179-182号)与“发现过往”「Découverte d'un passé;Discovery of a Past」(第183-186号)两组笔记中,找到对《收获与播种》前三部分的简短回顾总结。}。只需说,这项劳作有四段鲜明阶段或四次“气息”——如呼吸的起伏,或波浪的连续,如同从那广大、无声、静动交织、无边无名的未知深海中涌起的一列浪潮,我不知其何以生,这深海便是“自我”,抑或比承载与滋养这“自我”的我更浩瀚深邃。这些“气息”或“波浪”化为现已写就的《收获与播种》四部分。每波浪来时,我未曾呼唤,亦未预料,途中我无从知晓它将携我何往、何时止息。每当一波结束,新波接续,我仍觉自己处于某次冲势的尾声(最终也将是《收获与播种》的终点!),而实则已被另一浩大运动的新气息托举前行。仅在回望时,这一切才清晰显现,于那被体验为行动与流动之中,显露无误的\textbf{结构}。

当然,这运动未随《收获与播种》的(暂时的!)句点而终,亦不会止于我写给你的这封信的句号——此信乃这运动的一“节拍”。它并非始于1983年6月或1984年2月某日,当我坐于打字机前,着手(或重拾)某数学著作的某引言。它诞生(或更恰当地,重生……)于沉思进入我生命之日……

但我又离题了,任凭当下的意象与联想牵引(甚至席卷……),而未谨守“意图”的线索,循规蹈矩。我今日之意,本欲接续叙述——即便简略——去年4月“埋葬「Enterrement;Burial」之发现”,当时我以为《收获与播种》已完稿两周,却在短短三四周内,一连串发现如瀑布般砸下,一桩比一桩更重大、更不可思议——如此重大而离奇,数月后我仍难以“相信自己健全理智的见证”,难以摆脱对显见的隐秘\textbf{怀疑}\footnote{我试图通过寓言“中国皇帝之袍”「La robe de l'Empereur de Chine;The Robe of the Emperor of China」表达此难,于同名笔记(第77'号)中述及,并在“职责已尽——或真相一刻”「Le devoir accompli - ou l'instant de vérité;The Duty Fulfilled - or the Moment of Truth」(第163号)中重提。}。这顽强的隐秘怀疑,至去年10月(“埋葬全盛”之发现六个月后)始消散,因我那隐秘的前学生好友皮埃尔·德利涅「Pierre Deligne;Pierre Deligne」来访\footnote{我于前述脚注引用的笔记中述此访。}。我首次直面埋葬,不再仅凭文本 посред,以(诚然雄辩的!)言辞论及贬损、掠夺与屠戮某事业,及(以不在场之师为象征)某种数学风格与进路的埋葬——而是以直接而可触的方式,借熟悉的面容与熟稔的嗓音,亲切而天真地呈现。埋葬终于在我面前“血肉现身”,以我已熟识却首次以新目光、新专注审视的忙碌而平淡面貌展露。于是,那在前数月反思中渐显为我隆重葬礼之大主持、“披祭袍之祭司”、空前“行动”之首要匠人与主要“受益者”、被交付嘲讽与掠夺之事业的隐秘继承者,在我眼前铺展开来……

此次相遇处于《收获与播种》“第三波”开端,当时我刚投入对阴阳的漫长沉思,追逐一隐秘而顽强的意念联想。当时,这短暂插曲仅留数行回响,匆匆而过。然它标志一重要时刻,其果实数月后始明晰显现。

另有第二次“埋葬血肉现身”的对峙。仅十日前,它再次“最后关头”重启一场永无止境的探究。这次,仅是对让-皮埃尔·塞尔「Jean-Pierre Serre;Jean-Pierre Serre」的一通电话\footnote{此大致引自笔记“掘墓人——或全体会众”「Le Fossoyeur - ou la Congrégation toute entière;The Gravedigger - or the Entire Congregation」(第97号,第417页)。}。这番“断续交谈”,以惊人方式——超乎预期——证实了我数日前刚勉力自解之事\footnote{于同笔记(第173号)部分 c. “至尊者——或默许”「Celui entre tous - ou l'acquiescement;He Among All - or the Acquiescence」。},关于塞尔在我的埋葬中之角色,及他对其“眼皮底下”之事“隐秘默许”,未作视闻之态。

一如既往,这交谈“轻松”而友好,塞尔对我的友善态度显然真挚无伪。然这次,我真切“看见”——或欲书“触及”——我刚勉强承认的“默许”:无疑“隐秘”(如我先前所写),尤以当时无可疑见的急切。此急切而无保留的默许,欲埋葬该埋葬之物,无论何地、何手段,只要必要,塞尔亲知且不欲见的真实父权,皆被替换为虚构而受欢迎的父权……\footnote{此大致引自笔记“掘墓人——或全体会众”「Le Fossoyeur - ou la Congrégation toute entière;The Gravedigger - or the Entire Congregation」(第97号,第417页)。}。这惊人证实了我一年前直觉,当时我写道\footnote{此引自同笔记(见前脚注),同第417页。}:

\begin{quote}
    “在此光芒下\footnote{“在”那有意抹除一切“不可欲之父权”(见引文所用“难消化”一词)的意图“光芒”下。},首席主持德利涅不再是依内心深层力量——决定其人生与行为——塑成事业之人,而更像是(因其“合法继承者”角色\footnote{德利涅的“继承者”角色既隐秘(其发表文字无一行令人疑其曾受我口传),又为众人感知并承认。这是德利涅及其独特“风格”的典型双重游戏,他巧妙操弄此暧昧,既收受隐秘继承者之利,又否定亡师,主导大规模埋葬行动。})那无懈一致的\textbf{集体意志}的\textbf{工具},执着于不可能的任务——抹去我的名字与个人风格于当代数学。”
\end{quote}

若德利涅当时现为那“无懈一致的集体意志”的“工具”(兼首要“受益者”),塞尔如今则为这意志的\textbf{化身},为其无保留默许之担保——对无数诡计与欺诈,乃至大规模集体蒙蔽与肆意侵占的“行动”,只要关乎对我这卑微亡魂或某敢于自称我、逆众做“格罗滕迪克续者”者的“不可能任务”\footnote{我此处想到佐格曼·梅布胡特「Zoghman Mebkhout;Zoghman Mebkhout」,首见于引言(“埋葬”「L'Enterrement;The Burial」),继于笔记“我的孤儿”「Mes orphelins;My Orphans」(第46号),及(埋葬发现后所写)“教学之败(2)——或创造与虚荣”「Échec d'un enseignement (2) - ou création et fatuité;Failure of a Teaching (2) - or Creation and Vanity」与“无正义无能之感”「Un sentiment d'injustice et d'impuissance;A Feeling of Injustice and Impotence」(第44'、44''号)。我于埋葬第七队列“研讨会——或梅布胡特之层与变态”「Le Colloque - ou faisceaux de Mebkhout et Perversité;The Colloquium - or Mebkhout’s Sheaves and Perversity」(第75-80号)十一笔记中,渐识梅布胡特先驱事业被隐匿与侵占的不公。对此(第四暨末)“行动”的详尽探究与叙述,成“四种运算”「Les quatre opérations;The Four Operations」探究最厚重部分,名正言顺曰“\textbf{极致}”「L'Apothéose;The Apotheosis」(第171 (i) 至 171号)。}。

这乃埋葬诸多悖谬与困惑面向之一——它首要,乃至几近全然,出自我曾为友或学生之人手,在那我从未识敌的世界中尤甚。我相信,正因此,《收获与播种》及我正书写的此信较他人更关乎你,欲成一种\textbf{呼唤}。若你是数学家,曾为我学生或友人,你或不陌生于这埋葬——或以行,或以共谋,即便仅以对我的沉默,面对你门前之事。若(奇迹般)你接纳我卑微之言及其见证,而非锁于闭门,拒斥这不受迎的信息,你或将得知,你与你参与(主动或默许)埋葬的,不仅是他人的事业——我与数学之爱的果实与活证;更于那从不自名的埋葬之下,在更隐秘、更深层,是你自身存有之鲜活本质部分,你原初认知、爱与创造之力,由你亲手埋于他人之身。

在我众学生中,德利涅地位独特,我于反思中多有述及\footnote{尤见《收获与播种 II》“吾友皮埃尔”「Mon ami Pierre;My Friend Pierre」(第60-71号)十七笔记组。}。他远为最“亲近”,亦是唯一(无论学生与否)深纳并自化我愿景者\footnote{德利涅确“深纳并自化”的“宏大愿景”,曾对他有强大吸引,至今仍不由自主地迷住他,然一股强力同时驱他毁之,蓄意破其根基统一,攫取散片。他对被否弃“亡师”的隐秘对立,映其存有之分裂,深烙于我离去后其事业——远逊我所知其惊才。}——这愿景在我与他相遇前,已在我心中孕育成长。于我众共数学热忱之友中,塞尔——略似长者——最亲近(亦远超他者),尤在某十年,他于我劳作中独担“引爆者”角色,为五十至六十年代直至我离去前,我数学思想诸多核心理念之灵感源。他俩与我之特殊关系,确与其各自非凡才具相关,使其对同代及后代数学家有卓绝影响力。除此共性,他俩性情与行事在我看来迥异,几成对极。

无论如何,若有数学家以某身份与我及我事业“亲近”(且为人所知),必是塞尔与德利涅:一为长者与孕育愿景关键期的灵感源;一为我最才俊的学生,我为其主要(且隐秘……)灵感源,埋葬与否皆然\footnote{见前脚注。}。若我离去后(正式成“殞地”)埋葬启动,化为服务同一目的的大小“行动”之漫长队列,必赖他俩协力共谋——前长者与前学生(或前“弟子”):一悄然高效领行动,召集我部分学生\footnote{此处明确指另五学生,如德利涅般以“多样性上同调”「cohomologie des variétés;cohomology of varieties」为主题。},欲屠父(以臃肿可笑的\textbf{超级女霸}为丑像);一予(四项)行动——贬毁、屠戮、肢解与分享无尽遗骸——无保留、无条件、无限度的“绿灯”。