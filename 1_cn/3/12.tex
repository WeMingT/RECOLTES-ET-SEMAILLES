\section{自发性与严谨性}

自发性「Spontanéité;Spontaneity」与严谨性「Rigueur;Rigour」是同一不可分割品质的“阴影”与“光明”两面。只有它们的结合,才能孕育出文本或个体特有的品质,我们或可用“真理的品质”这样的表达来尝试勾勒它。在我过去的出版物中,自发性即便不是完全缺席,也仅占微不足道的一席之地,但我并不认为它在我身上迟来的绽放削弱了严谨性。相反,这位阴性伴侣的充分 присутствие赋予了严谨性新的维度与丰饶。

这种严谨性自我约束,确保其在意识领域中对众多流过的信息进行细腻“筛选”时,始终提炼出重要的或本质的,剔除偶然的或次要的,不让这筛选过程僵化为自我审查或自满的自动机制。唯有我们内心的好奇与求知渴望,才能唤醒并激发这种轻盈的警觉、这种活泼的敏锐,去对抗那无处不在、巨大的惯性——那些所谓“自然的倾向”,由成见雕琢而成,承载着我们的恐惧与条件反射。

同样的严谨性,这份警醒的关注,也投向自发性及其表象,同样在此分辨出那些“自然的倾向”——它们确乎自然——与真正从存在深层喷涌而出的事物,从原始的认知与行动冲动中迸发,推动我们与世界相遇。

在写作层面,严谨性体现为一种恒久的关注:借助语言尽可能精确、忠实地勾勒思想、情感、感知、意象、直觉……凡需表达之处,皆不容模糊或近似的词语代替轮廓分明的事物,也不以虚假的精确(同样扭曲)描述尚在预感迷雾中模糊的事物。只有当我们试图捕捉它当下的真实面貌,且仅在此时,那未知之物才会揭示其真性,或许直至白昼的明光之下——若它注定属于光明,而我们的渴望促其褪去阴影与迷雾的面纱。我们的角色并非假装描述并固定我们未知且无法把握之物,而是谦卑且热忱地认知那包围我们的未知与奥秘。

这意味着写作的角色并非记录研究的成果,而是研究的过程本身——我们与世界之母、那未知者的爱之劳作与爱的果实,她不懈地呼唤我们深入她的无限之躯,在欲望的神秘路径引领下,探知她每一处角落。

为呈现这过程,回溯不可或缺——它们细化、 уточ化、深化,有时修正初稿,甚至二稿、三稿,构成发现过程的核心部分。它们是文本的本质组成部分,赋予其全部意义。因此,《虚荣与更新》「Fatuité et Renouvellement;Vanity and Renewal」末尾的“注释”(或“注解”),在构成文本“初稿”的五十“节”中此处彼处提及,是文本不可分割且至关重要的一部分。我强烈建议你在阅读时逐一查阅,至少在每节包含一或多个此类“注释”引用的末尾查阅。《丰收与播种》「Récoltes et Semailles;Harvests and Sowings」其他部分的脚注,或某“注释”(此处为主文)中提及的后续注释——作为对此的“回溯”或注解——亦是如此。这与我建议你在阅读时不离目录并列,是我能给你的主要阅读建议。

最后一个实用问题,以略显平淡的方式结束这封该收尾的信。为准备《丰收与播种》各分册在大学复印服务处印刷,曾有些许“慌乱”,希望赶在大假期前完成(若可能)。匆忙中,《分册2》(“葬礼(1)——或中国皇帝的袍子”「L'Enterrement (1) - ou La robe de l'Empereur de Chine;The Burial (1) - or The Robe of the Emperor of China」)遗漏了一整页最后一刻添加的脚注。这些脚注主要是修正写作《四种运算》「Les Quatre Opérations;The Four Operations」时近期发现的某些实质性错误。其中一则脚注比其他更重要,我想在此特别指出。它是注释“受害者——或双重沉默”「La victime - ou les deux silences;The Victim - or the Two Silences」(第78'号,第304页)的注解。在该注释中,我尽力(虽完全主观)描绘我对好友佐格曼·梅布胡特「Zoghman Mebkhout;Zoghman Mebkhout」当时如何“内化”他所遭受的独特掠夺的印象,他却认为这对他不公,觉得我几乎将他与其掠夺者“混为一谈”。显然,这则注释仅旨在呈现特定“瞬间”的印象,未述及(或许视为理所当然的)其他同样真实(且或更无可争议)的视角。无论如何,对此微妙话题的反思在一年后于注释“根与孤独”「Racines et Solitude;Roots and Solitude」(第171号)中大幅深化,未引起佐格曼的异议。同一话题的其他反思见于“三个里程碑——或纯真”「Trois jalons - ou l'innocence;Three Milestones - or Innocence」及“死页”「Les pages mortes;The Dead Pages」(第171(x)及(xii)号注释)。这三则注释属《四种运算》中“顶峰”「L'Apothéose;The Apotheosis」部分,聚焦佐格曼·梅布胡特作品的挪用与扭曲。

只剩祝你阅读愉快——并期待读到你的回信!

\hfill 亚历山大·格罗滕迪克「Alexandre Grothendieck;Alexandre Grothendieck」