\phantomsection
\section*{尾声与附言——或一场辩论的背景与前提}
\addcontentsline{toc}{section}{尾声与附言——或一场辩论的背景与前提}

\hfill 1986年2月

\section{瓶子光谱仪}

这封信写成已有整整七个月,寄出连同那“厚砖”一起也近四个月了。每本上都有我亲笔的题词 \footnote{少数例外,主要是不认识的同事,他们只收到临时印刷的第0和第4分册,作为他们积极参与我葬礼「Enterrement;Burial」的额外奖励。}。如同一只“漂流瓶”,或者更像一群漂泊的瓶子,我的讯息落入并流传至那曾熟悉的数学小宇宙最偏远的角落。日子、星期、月复一月,通过直接与间接的回响,我意外地仿佛面对一幅数学界的巨大X光片,这幅图景由一台触角遍布的光谱仪绘制,我的那些无辜“瓶子”恰似其游走的触角。于是(高尚的责任使然!),尽管我并不缺事做,却被赋予了解读这幅“射线图”并尽力述说所见的新任务。这将成为《丰收与播种》「Récoltes et Semailles;Harvests and Sowings」的第六(且最后,我保证!)部分。若天假以年,这将为“我晚年的伟大社会学著作”加冕。目前,先谈几点初步评论。

迎接我这支手工打造的简朴小舰队时,最显著且遥遥领先的,是嘲弄的语气,半带愠怒,像是“瞧,格罗滕迪克晚年变得偏执了”,或“这家伙还真把自己当回事”——一语蔽之!不过,我只收到一封这类风格的信 \footnote{此信来自一位曾是我学生的同事,且是我葬礼的共同参与者之一。},另有两封带着掩饰的、自我满足的揶揄 \footnote{来自两位曾与我在布尔巴基学派「Bourbaki;Bourbaki」共事的旧同事,其中一位是早年以温暖善意接纳我的前辈。}。大多数收信的数学家,包括我昔日的学生,多以沉默回应 \footnote{寄给131位数学家的信中,至今53位有所回应,哪怕只是确认收到。其中6位是我旧生,另外8位毫无音讯。}——这沉默意味深长。

尽管如此,我已收到大量书信。大多数信件带着礼貌的尴尬,常常试图显得友好,仿佛出于礼节考量。有两三次,我在这种尴尬背后,透过它的过滤,感受到依然鲜活的温情。多半时候,若尴尬不以自我或他人的善意声明表达,便化作赞美——我从未一生收到如此之多!诸如“伟大数学家”、“精彩篇章”(论创造力“等等”)、“无可争议的作家”之类,不胜枚举。为锦上添花,我甚至收到一句真挚(绝非讽刺)的赞美,称颂我内在生活的丰富。毋庸多言,这些信中无人触及任何问题的核心,更遑论个人介入;语气更像是“被请求发表意见”(借用一封信的措辞),针对一件略显棘手、或许虚构,甚至全然想象的事务,且无论如何,与己无关。若偶尔提及某问题,也仅指尖轻触,尽力远之——或以对我谆谆劝导,或用谨慎的条件句,或套用不知如何应对时的陈词滥调,或其他方式。少数人暗示或许发生过不太正常的事——却小心翼翼地模糊何事何人……

我也收到十五六位新旧友人的坦诚热烈回响。有些表达了情感,不愿掩饰或压抑。这些回响,及数学界外的同样温暖回应,是我长久孤独工作的回报,这工作不仅为己,也为众。

在收到我信的约130位同事中,有三位真正回应,以全身心投入,而非仅对世纪事件作遥远评论。我还收到一位非数学家的通信者同样真挚的回音。这些是对我讯息的真实回应,亦是我最好的回报。