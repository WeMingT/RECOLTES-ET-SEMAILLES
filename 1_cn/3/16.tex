\section{诚挚致歉——或时代精神 (2)}

在《信》中,我认为自己已充分阐明了撰写《收获与播种》「Récoltes et Semailles;Harvests and Sowings」时的精神,使之清晰表明,我绝无意在此扮演历史学家的角色。这是一份真诚的见证,关乎我亲身经历,以及对此经历的反思。见证与反思供所有人使用,包括历史学家,他可将其作为诸多材料之一。届时,由他依其学科严谨性的准则,对此材料进行批判性分析。

当然,需区分狭义的事实(即“原始事实”或“物质事实”)与对这些事实的“评价”或“阐释”,后者赋予事实意义,而此意义因观察者(或共为者)不同而异。大致而言,可说《收获与播种》的“见证”面向事实,而其“反思”面向对事实的阐释,即我为赋予其意义所做的工作。在构成见证的“事实”中,我亦纳入“心理事实”,特别是那些情感、联想及种种意象,它们映照于我的见证中,无论发生于或远或近的过去,抑或写作当下。

对于我在《收获与播种》中描述或提及的事实,我区分了三类来源。其一,记忆所重现的事实,或多或少精确,或多或少模糊,因时而异,有时甚至有所扭曲。对此,我可担保写作时的真诚态度,但无法保证毫无错误。相反,我曾有机会指出若干细节错误,并在相应处以后续脚注标明。其二,书面文件,尤其是信件及尤为重要的正规科学出版物,我在必要时尽可能精确地加以引用。其三,第三方的见证。有时它补充我的记忆,使之复苏、精确,有时甚至纠正之。在某些罕见场合(我稍后将回溯),此见证为我带来相较已知信息完全新的内容。当我转述此类见证时,并非意味着我能逐一核实其准确性与合理性,而仅表明它以足够可信的方式融入我亲知事实的丰富织体中,使我确信(无论对错……)此见证大体符合真相。

对于细心的读者,我认为他们绝不会在任何时候难以区分事实陈述与对此的阐释,以及(在前者中)辨别我所述三种来源中何者适用。

\begin{center}
    * \quad * \\
    *
\end{center}

当我刚才提及第三方的见证,且我未能在“逐一核实其合理性”的情况下加以转述时,我指的是佐格曼·梅布胡特「Zoghman Mebkhout;Zoghman Mebkhout」关于其作品被大规模掩盖的见证。在《收获与播种》中我所述的“物质事实”中,目前具争议性或依我现今判断需修正的,仅有部分仅由梅布胡特见证证实的事实。为结束此后记,我愿在此对《收获与播种》初稿中呈现的“梅布胡特事件”版本提出批判性评论。更详尽的评论与修正将分别在适当处纳入印刷版(即《收获与播种》的定稿)。

我试图代言的“梅布胡特版本”,其核心似包含以下两大论点:

\begin{enumerate}
    \item 在1972至1979年间,梅布胡特似为唯一一人\footnote{除1975年柏原正树「Kashiwara Masaki;Masaki Kashiwara」的构造定理外,其在理论中的重要性无人质疑。但依梅布胡特版本,这乃柏原对此新兴理论的唯一贡献。此(不准确的)版本因柏原未发表其他至少提及关键思想的出版物而得到佐证。},在普遍冷漠中,受我作品启发,发展了“$\mathscr{D}$-模哲学”「philosophie des $\mathscr{D}$-Modules;philosophy of $\mathscr{D}$-Modules」”,作为我意义上的“上同调系数”新理论。
    \item 一旦此理论的重要性开始被认可,法国乃至国际上似存在一致共识,抹去他的名字及其在此新理论中的角色。
\end{enumerate}

此版本有坚实文献支持:一方面,梅布胡特的出版物极具说服力;另一方面,其他作者的诸多出版物(尤其是1981年6月卢米尼会议「Colloque de Luminy;Luminy Conference」文集),其故意掩盖之意显而易见。此外,梅布胡特后来提供的更详尽细节(我在“葬礼 (3)——或四项行动”部分转述),虽无法直接验证,却与我确信无疑的某种普遍氛围完全吻合。

我刚获悉若干新事实\footnote{我感谢皮埃尔·沙皮拉「Pierre Schapira;Pierre Schapira」与克里斯蒂安·乌泽尔「Christian Houzel;Christian Houzel」好意提醒我注意这些事实,及我对梅布胡特-柏原争议呈现的偏颇性。},表明需大幅修正上述第$1^{\circ}$点。梅布胡特所处的孤立\footnote{此孤立主要源自我昔日学生对梅布胡特思想与工作的冷漠,他固执地以一位被一致遗忘的“先辈”为灵感……}确属真实,但属相对孤立。法国有让-皮埃尔·拉米斯「J.P. Ramis;J.P. Ramis」在同一领域的成果(梅布胡特未曾提及),更重要的是,梅布胡特发展并完成的一些重要思想,他自认原创,却可能归于柏原「Kashiwara;Kashiwara」\footnote{其中最重要者为“对应”思想(借用新式术语),即$\mathscr{D}$-模的“黎曼-希尔伯特对应”「correspondance de Riemann-Hilbert;Riemann-Hilbert correspondence」。相关猜想由梅布胡特证明,据沙皮拉「Schapira;Schapira」称亦由柏原证明(而梅布胡特坚称其证明为唯一发表者)。证明优先级对我仍模糊,我无意余生探究此谜……

至于$\mathscr{D}^{\infty}$模的姊妹命题,似无疑问,其思想与证明的原创归于梅布胡特。}。

由此,梅布胡特版本中所述柏原-梅布胡特争议的某些情节变得不可信或存疑,我对其(过于)忠实代言。

无疑,在“文献工作”层面及某些成功完成的思想构想上,梅布胡特是$\mathscr{D}$-模新理论的主要先驱之一,或许是最主要的;无论如何,他是唯一全身心投入此任务者,其真正意义对他及所有人仍未明。而围绕其作品的掩盖行动——以卢米尼会议为高潮——对我而言,仍是本世纪数学界的一大耻辱。但若声称(如我真诚所为)梅布胡特独自承担此任务,则不实。相反,他是唯一坦诚且勇敢承认我的思想与作品在其研究及新理论萌发中重要性的人。

在此后记中,非详述此事的场合——我将在适当处为之,包括阐明“梅布胡特版本”心理背景的评论。对我而言,“梅布胡特-柏原争议”之所以有趣,仅在于其映照一个时代的普遍氛围。对我来说,即便在其扭曲中,及促成此扭曲的力量中,“梅布胡特版本”亦如我为“时代档案”提供的其他较少争议材料一样,是一个雄辩的“时代标志”。

我须为自己的轻率致歉,因我在呈现梅布胡特-柏原争议时,仅依梅布胡特提供的见证与文献,似其版本不容置疑。此版本将第三方置于可笑乃至可憎的光景,更应谨慎。对此轻率及缺乏健全谨慎,我在此向柏原先生「M. Kashiwara;Mr. Kashiwara」致以最诚挚的歉意。