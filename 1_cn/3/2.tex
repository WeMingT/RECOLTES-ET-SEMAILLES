\section{《收获与播种》的诞生(一次快速回顾)}

在这篇前言中,我想尽可能用几页篇幅,更详尽地告诉你《收获与播种》探讨的内容,而不只是局限于副标题“一位数学家过往的反思与见证”(你肯定猜到了,这是我自己的过往……)。《收获与播种》内容丰富,不同的人阅读时,无疑会看到不同的东西:一场探寻过往的旅程;一次对存在的“沉思”(méditation;Meditation);一幅特定环境与时代的“风俗画”(tableau de moeurs;Portrait of customs)(或者说是一个时代向另一个时代悄然无情滑落的写照……);一次“调查”(enquête;Investigation)(有时近乎侦探调查,有时又类似数学大都会底层的惊险小说……);一场广泛的“数学漫谈”(divagation mathématique;Mathematical digression)(会播下不少种子……);一本实用的应用精神分析论著(或者,也可以说是一本“精神分析小说”);一篇对“自我认知”(connaissance de soi;Self-knowledge)的颂歌;“我的忏悔录”;一本私人日记;一部关于“发现”(découverte;Discovery)与“创造”(création;Creation)的心理学;一份“控诉书”(réquisitoire;Indictment)(理应毫不留情……),甚至是在“数学上流社会”的一场“清算”(règlement de comptes;Settlement of accounts)(绝不留情面……)。可以肯定的是,我在写作过程中从未感到无聊,而且还见识了各种各样的事情。如果你工作之余有闲暇阅读此书,我想你应该也不会觉得无聊——当然,除非你是硬着头皮读。

显然,这本书并非只面向数学家。诚然,在某些部分,它对数学家的针对性更强。在这篇给《收获与播种》的前言中,我想特别总结并突出那些对你作为数学家可能尤为重要的内容。最自然的做法,就是简单地跟你讲述我是如何一步步写出这四五本“书”的。

如你所知,1970年,由于我所在的机构(高等科学研究所「IHES;Institute for Advanced Scientific Studies」)涉及军事资金问题,我离开了“数学界”。在经历了几年类似“文化大革命”风格的反军事主义和生态主义活动后(你可能在某些地方听说过),我几乎从公众视野中消失,躲在某个不知名的外省大学。传言说我整天放羊、打井。事实上,除了许多其他事务,我也像其他人一样,勇敢地去大学授课(这是我不太有新意的谋生手段,至今仍是如此)。偶尔,我也会花上几天、几周甚至几个月的时间,重拾数学研究——我有满满几箱自己的草稿,可能只有我自己能看懂。但这些研究与我过去的工作截然不同,至少乍一看是这样。1955年至1970年间,我主要研究的是上同调(cohomologie;Cohomology),尤其是各类(特别是代数)簇的上同调。我觉得在这个领域我已经做得够多了,其他人可以在没有我的情况下继续探索,既然要研究数学,是时候换个方向了……

1976年,我生命中出现了一种新的热情,其强烈程度不亚于我曾经对数学的热爱,而且与数学密切相关。这就是我所说的“沉思”(méditation;Meditation)的热情(毕竟事物总得有个名字)。这个名字,就像其他任何名字一样,难免会引起无数的误解。就像在数学中一样,这是一项探索性的工作。我在《收获与播种》中不时会谈到这个话题。显然,这足以让我忙碌一生。事实上,有好几次,我都以为数学已经成为过去,从今往后,我只会专注于更严肃的事情——我要去“沉思”。

然而,四年前我终于意识到,我对数学的热情并未熄灭。甚至,连我自己都感到惊讶,我这个近十五年来从未想过再发表一行数学文字的人,突然开始撰写一本似乎永无止境的数学著作,而且会有很多很多卷;既然如此,我打算把我认为在数学方面有话要说的内容,写成一系列(也许是无穷多本)名为《数学反思》(Réflexions Mathématiques;Mathematical Reflections)的书,这里就不多说了。

那是在1983年春天,也就是两年前。当时我正忙着写《追寻场》(A la Poursuite des Champs;In Pursuit of Fields)的第一卷,它同时也是《数学反思》的第一卷,根本无暇思考自己身上发生了什么。九个月后,这本第一卷按计划完成了——也就是说,只剩下写引言、通读全文、添加注释,然后付梓印刷……

但这本书至今仍未完成——一年半以来毫无进展。剩下要写的引言已经超过了一千二百页(打印稿),真正完成时,恐怕会有一千四百页。你肯定猜到了,这篇“引言”就是《收获与播种》。最新消息是,它将构成那套著名“丛书”计划中的第一卷、第二卷以及第三卷的一部分。这套丛书因此改名为《反思》(Réflexions;Reflections)(不再特别强调“数学”)。第三卷的其余部分主要是数学文本,对我来说,这些内容如今比《追寻场》更精彩。《追寻场》要等到明年再做注释、编索引,当然,还要写一篇引言……

第一幕结束!