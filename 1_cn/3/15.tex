\section{坏疽——或时代精神 (1)}

这场“动机行动”不过是四场同类“大型行动”中的一场,同时也是众多规模较小、精神相似的行动中的一员。它绝非我那幅时代“风俗画”中最“重大”的集体神秘化事件,更不用说最不公的了。它仅仅是在富人不在场(或去世……)时,趁机掠夺其羊群,而非在众目睽睽之下(在普遍的冷漠中),为了取乐而在穷人眼前扼杀其羔羊。甚至在现今已进入日常使用的数学语言中,那些看似无害的书名、概念或命题名称,被随时引用,却本身已是神秘化或冒充的象征\footnote{我这里尤其想到那个奇特的缩写“SGA $4 \frac{1}{2}$”(分数真是实用!),它本身就是双重冒充(也是当代数学文献中最常被引用的缩写之一),以及“韦迪耶对偶性”「dualité de Verdier;Verdier duality」或“韦迪耶对偶”「dual de Verdier;Verdier dual」、「德利涅-格罗滕迪克猜想」「conjecture de Deligne-Grothendieck;Deligne-Grothendieck conjecture」,乃至“塔纳卡范畴”「catégories tannakiennes;tannakian categories」(在这里,塔纳卡「Tannaka;Tannaka」倒无辜,因为他从未被咨询过……)。这些将在适当的地方作更详尽的讨论。},以它们的方式见证了一个时代的耻辱。

若我相信自己曾为“数学共同体”做过有益的工作,那便是将一些不光彩的事实——那些在暗中腐烂的事实——带到了光天化日之下。这类事实,想必每个人每日或多或少、或近或远地与之擦肩而过。然而,有多少人曾停下片刻,去嗅一嗅空气,去凝视片刻?

那个曾亲身遭遇某些人的傲慢与另一些人(或同一批人)的欺诈之人,或许曾自诩这是一种专属于他的特殊不幸。将其经历与我的见证对照,他或许会感到,这种“不幸”也是他为时代精神所取的一个名字,这精神压在他身上,如同压在所有人身上一样。而(谁知道呢!)这或许会激励他参与一场辩论,这场辩论与他相关,正如与我相关。

但若我在“公共广场”上抖搂的这堆“脏衣物”,除了引来一些人无乐的冷笑与另一些人礼貌的尴尬,以及所有人的冷漠之外,别无他果,那么一个原本混沌的局面将变得极为清晰。(至少对于那些仍关心使用自己眼睛的人而言。)那些关于诚信与体面的传统共识\footnote{当我提及这些“诚信与体面的共识”时,我并非说它们从未被违反。但即便被违反,那也确实是“违反”,而共识本身依然被普遍接受。},无论是数学家之间的关系,还是数学家与其艺术的关系,从此将成为过去,被“超越”。无需某国际数学家协会郑重宣示,这却已是既定且近乎官方的事实:如今,在数学世界中握有权力的“通过共举的兄弟会”,一切手段皆被允许,不再有任何保留或限制。一切操控思想的伎俩,用以牵着那只求相信的麻木读者的鼻子走;一切关于作者身份的交易,同伙间的虚假引用,对被指定沉默者的沉默,朋党勾结,各种伪造,甚至最粗鄙的剽窃,尽人皆知——是的,对这一切说“阿门”,带着所有“大师”与数学公共广场上所有大小老板的祝福,或通过言辞,或通过沉默(若非积极而热切的参与)。“新风尚”在此大行其道,风靡一时!曾经的艺术,如今在(几乎)一致的同意下,沦为混乱与争斗的集市,在头领们慈父般的注视下。

曾几何时,在数学家世界中行使权力,受到一致且不可动摇的共识限制,这些共识表达了一种集体的体面感。如今,这些共识与这种情感已成过时之物,被超越,无疑不配得上这计算机、太空舱与中子弹的辉煌时代。

这已成为既定且封存的事实:对于握有权力的兄弟会而言,权力是一种全权。