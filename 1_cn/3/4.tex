\section{一场埋葬之风……}

然而,尽管多年来我乐于探究一种弥散的感知——那场规模浩大的埋葬「Enterrement;Burial」——它却从未停止以各种面貌固执地唤起我的回忆,那些面貌远非仅仅是对某项事业的淡忘所能比拟,且绝不琐碎。我逐渐察觉——虽不知究竟如何察觉——那被遗忘的愿景中的若干概念,不仅已然废弃,甚至在某些优雅圈子中,沦为一种高傲轻蔑的对象。尤其是那至关重要的统一概念——拓扑斯「topos;topoi」——新几何学的核心所在,它为拓扑学、代数几何「géométrie algébrique;algebraic geometry」和算术「arithmétique;arithmetic」提供了共通的几何直觉;正是它,使我得以提炼出层上同调「cohomologie étale;étale cohomology」和$\ell$-adic工具,以及晶体上同调「cohomologie cristalline;crystalline cohomology」的核心理念(诚然,这些理念此后或多或少被淡忘……)。实话说,随着岁月流逝,我的名字本身,竟也悄然神秘地成了嘲讽的对象——仿佛成了无穷无尽的晦涩书堆的代名词(恰如那些著名的``拓扑斯'',或是他喋喋不休的``模体''「motifs;motives」,无人见过……),成了千页篇幅中吹毛求疵的象征,成了冗长而庞大的空话,谈及的不过是众人早已熟知、未待他言便已了然于心之事……大抵如此语调,低声呢喃,暗藏弦外之音,带着“高雅人士之间”应有的那份精妙矜持。

在《收获与播种》的反思历程中,我相信自己触及了那些深层力量——它们在人们心中运作,藏于对一项事业的嘲弄与轻慢背后,而这事业的深远意义、生命力与气息,却为他们所不及。我还发现了(撇开我个人特质对事业与命运的烙印不谈)那隐秘的``催化剂'',它激发这些力量,以轻佻的鄙夷形式显现,面对我那依然鲜活的创造力的明证:简言之,那便是葬礼上的大主持,在这场以嘲讽与轻蔑悄然掩饰的埋葬中。奇特的是,这人竟也是与我最为亲近之人:唯一曾将某种充满生命与强烈力量的愿景内化并据为己有之人。但我似已言过其实……

实话说,那些此起彼伏的``隐秘嘲讽气息'',并未过多触动我。直到三四年前,它们某种程度上仍无名无姓。我固然从中窥见时代的一种不祥征兆,却并未真正感到自身受其牵连,亦未因此生出焦虑或不安。然而,有一事却更直接地刺痛我——那是从数学界中诸多昔日友人那里传来的疏远迹象。尽管我已离开我们曾共处的那个世界,我仍觉与他们因同好、因共同的过往而维系着情谊的纽带。每逢如此,我心生怅然,却未曾深究。据我记忆,也从未想过将这三组迹象联结起来:废弃的工地(及被遗忘的愿景)、这``嘲讽之风'',以及那些曾为友人的疏远。我曾逐一写信给他们,却无一人回音。如今,我写给旧友或旧生的信,谈及我心之所系,常无回音,已非稀罕。新时新俗,我又能如何?我只得不再寄信。然而(若你是其中之一),我此刻书写的这封信,将是个例外——再次向你敞开的言辞——这次是迎纳,还是再度关闭,全在你心……

某些旧友对我的疏远,初现端倪,若我未记错,当在1976年。那年,亦是另一``系列''迹象初显之时,待我述及后,再回《收获与播种》。更确切地说,这后两组迹象乃同时浮现。此刻笔下,我蓦然发觉,它们实则密不可分,不过同一现实的两个面向或``面貌'',于那年闯入我亲历的场域。至于我正欲述及的面向,那是一种系统性的``置之不理'',无声而无懈可击,由一种``无隙共识''\footnote{这``无隙共识''在《前言与更新》中偶被提及,最终于下一部分《埋葬(1)》「L'Enterrement (1);The Burial (1)」中,成为详尽见证与反思的对象,伴以``X队列''「Cortège X;Cortege X」或``殡车''「Le Fourgon Funèbre;The Hearse」,形诸``棺椁笔记''(第93-96号)及笔记“掘墓人——或全体会众”「Le Fossoyeur - ou la Congrégation toute entière;The Gravedigger - or the Entire Congregation」。后者为《收获与播种》此部分的收束,同时是这``反思第二气息''的初成。}施予1970年后少数学生及同道,他们的作品、工作风格与灵感,鲜明烙印着我的影响。或许也正是在此,我首次感知那“嘲讽之息”,透过他们,指向某种数学的风格与\textbf{进路}——一种风格与愿景,依当时数学体制中已然普世的共识,竟是\textbf{不应存在}的。

这证据,在潜意识中清晰可辨。那年,它甚至强行闯入我的意识——在一幕荒诞剧反复上演五次后,带着马戏团笑话般的执拗(彰显一篇明显出色的论文无法发表的困境)。此刻回想,我察觉某种现实当时正以善意的固执“向我示意”,而我却佯装未解(或充耳不闻:“喂,看看吧,大傻瓜,留意一下就在你眼皮底下的事,这可是关乎你啊……”)。我稍稍振作,瞥了一眼(仅一瞬),半是茫然半是分神:“哦,是啊,挺怪,好像有人被针对了,事情定是出了岔子,还如此齐整,简直不敢相信!” 

这荒谬得如此离谱,我急忙忘却那笑话与马戏。诚然,我不乏其他有趣事务。这并未阻止那马戏在随后几年再度唤醒我的记忆——不再是笑话的调子,而是隐秘的羞辱快感,或迎面一拳的力道;不过,我们身处优雅之士间,这拳法更为隐晦,却同样有效,留给那些优雅之人的创意发挥……

我感知为“迎面一拳”(施于他人)的插曲,发生在1981年10月\footnote{此插曲述于笔记“棺椁3——或过于相对的雅可比”「Cercueil 3 - ou les jacobiniennes un peu trop relatives;Coffin 3 - or the Overly Relative Jacobians」(第95号),尤见第404-406页。}。那次,自新风气迹象频现以来,我首度深受触动——若那拳落于我身,或许反不及它击中我所钟爱之人那般沉重。他略似我的学生,更是一位天赋卓绝的数学家,方成就斐然——不过,这只是细枝末节。真正非细微者,是我“之前”的三位学生直接参与了一桩被当事人(并非无由)视为羞辱与冒犯之举。另两位旧生已然以安逸者的傲慢,将他视若落魄者般打发\footnote{此事于前述脚注引用的笔记中顺带提及。}。还有一位学生三年后亦步亦趋(仍是“迎面一拳”风格)——当然,这我当时未知。当时触动我的,已然足够。仿佛我那从未真逝的数学过往,骤以狰狞面目,通过五位曾为我学生的要人——如今显赫而傲慢——烙印于我……

那本是绝佳时机,去探究这突如其来的剧烈触动之义。但我内心某处已决意(无需明言……),那“之前”的过往与我无涉,无须驻足;那如今以我太过熟悉的声音——轻蔑时代的回音——呼唤我的,定有误会。然而,我却焦虑难解,持续数日乃至数周,却未曾正视。(直到去年,借《收获与播种》的书写重返此幕,我才察觉那被立即抑制的焦虑。)我未作审视与探究,反倒躁动不安,四处写下``不得不写''的信。对方甚至费心回信,自是精心措辞,却不触及实质。波澜终平,一切如常。去年之前,我似未再多想。然而这次,却留下一道伤痕,或更像一根刺痛的木屑,避之不及;这刺维系着那欲愈合的创口……

这无疑是我数学人生中最痛苦、最难堪的经历——当我得以目睹(却未愿真正\textbf{认知}眼中所见):“我曾深爱的某位旧生或同伴,竟乐于悄然碾压另一位我所爱且在他眼中映出我之人”。这经历对我的触动,定然深于去年那些颇为惊奇的发现——后者在浅观者眼中,或显得更为不可思议……诚然,这经历唤醒了若干同调却较轻的共鸣,当初未曾深察。

这也令我想起,1981年亦是我与唯一一位离去后仍定期联系的旧生关系的剧变之年——他近十五年来,一直是我数学上的``特选对话者''。那年,“轻蔑姿态的迹象”——几年前已现端倪\footnote{此插曲述于笔记“双重转折”「Deux tournants;Two Turning Points」(第66号)。}——“骤然变得如此粗暴”,我遂与他断绝一切数学交流。那是前述“拳击插曲”前的数月。回望,这巧合惊人,但当时我似未作任何联想。我被归入“隔离的格子”;某人更宣称,这些格子无关紧要——结论已定!

令我震动的还有,1981年6月,一场耀眼的研讨会已然举行,堪称多重意义上的难忘——此会当之无愧载入史册(或其残迹……),以“变态研讨会”「Colloque Pervers;Perverse Colloquium」之名永存。我于去年5月2日与之相识(或更像被它砸中!),距4月19日发现血肉之躯的埋葬仅两周,我即刻明白,自己撞上了“\textbf{极致}”。一场埋葬的极致,诚然,亦是对两千余年来数学家伦理基石——不将他人之思与果据为己有的基本准则——轻蔑的\textbf{极致}。此刻记下这时间上的惊人巧合——两件看似性质与影响迥异的事件——我震撼于此显露的深邃而显见的联系:\textbf{对人的尊重}与对艺术或科学基本伦理规则的尊重,使其践行不沦为``群殴集市'',使其翘楚与风向标不堕为``肆无忌惮的黑帮''。但我又一次言过其实……