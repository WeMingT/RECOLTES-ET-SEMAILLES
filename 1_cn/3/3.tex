\section{领班之殒——废弃的工地}

现在,我感到是时候给出一些解释了:为何我如此骤然地离开了一个世界——那个我显然在其中自如生活了二十余年的世界;为何在人们似乎已很好地适应了十五年没有我的日子后,我却萌生了奇特的念头,要``归来''(仿佛幽魂重现……);以及为何一本数学著作的引言,原本预计六七百页,最终却膨胀至一千二(乃至一千四)百页。在这切入正题之际,我恐怕会让你感到些许伤感(抱歉!),甚至可能惹你不悦。因为毫无疑问,如同我从前,你也乐于以``玫瑰色''的目光看待你所属的那个环境——那个你拥有位置、名声以及一切的地方。我深知那是何种滋味……而此刻,些许刺耳的摩擦声将难免响起……

我在《收获与播种》中零星提及我离去的片段,未曾过多驻足。这``离去''在其中更多呈现为我数学人生中的一道重要分界——一切事件皆围绕这``节点''定位,分为``之前''与``之后''。一股巨大的冲击力才足以将我从那个根深蒂固的环境,以及一条清晰划定的``轨迹''中撕扯出来。这冲击源于我在一个与我身份紧密相连的环境中,遭遇了一种腐败形式\footnote{此处指的是全体科学家——无论国籍,以“体制”为首——毫无保留地与军事机构合作,将其视为资金、声望与权力的便利来源。此问题在《收获与播种》中仅偶尔提及一两次,例如去年4月2日的笔记“尊重”「Le respect;Respect」(第179号,第1221-1223页)。},而此前我选择对其视而不见(仅以不参与其中为底线)。回望之时,我意识到,超越这一事件本身,有一股更深层的力量在我内心悄然运作。那是一种强烈的\textbf{内在更新之需}。这样的更新无法在当时那高声誉科学机构的既有框架内实现与延续。在我身后,是二十年数学创造的炽热激情与无尽投入——同时,也是二十年精神停滞的漫长岁月,宛如困于``封闭之瓮''……我未曾察觉,自己正逐渐窒息——我渴求的是广袤天地的清新之风!我那``天赐的离去''骤然终结了这漫长的停滞,成为我迈向阐明内心深处力量的第一步——那些力量,在极度失衡与僵化的状态下被压抑与锁闭……这离去,真正而言,是一个\textbf{崭新的启程}——一次新旅途的初踏……

如我所述,我的数学热情并未因此熄灭。它在零星的反思中找到表达,那些路径与我``之前''所规划的截然不同。至于我留下的\textbf{事业}——那些``之前''的成果,无论是已发表的,还是留在白纸上的,甚至或许更本质的、尚未落笔或公之于众的部分——在我看来,也似乎的确如此,它们已然与我剥离。直到去年《收获与播种》之前,我从未想过要稍稍``停驻'',回望那些零散回响于我耳畔的片段。我深知,自己在数学中所做的一切,尤其是在1955至1970年的``几何''时期,都是\textbf{必须}完成的——我所见或瞥见的,都是\textbf{必须}显现的,\textbf{必须}被带入光明的存在。我也知道,我所完成的工作,以及我促成他人完成的工作,都是精雕细琢之作,是我全身心投入的结晶。我倾注了所有的力量与热爱,而在我看来,它们如今已独立——如同一件鲜活而强劲的生命体——不再需要我如慈母般呵护。从这方面而言,我离去时心境澄澈,无一丝疑虑。我毫不怀疑,那些我留下的已写与未写之物,已托付于可靠之手,它们将知晓如何让这些成果舒展、生长、繁盛,依其作为鲜活而有力的生命体的本性。

在这十五年紧张的数学劳作中,一种宏大的\textbf{统一愿景}在我内心孕育、成熟并壮大,化作几条简洁有力的核心理念。这愿景是“算术几何”「géométrie arithmétique;arithmetic geometry」的图景,融合了拓扑学、几何学(代数与解析)以及算术。我在韦伊猜想「conjectures de Weil;Weil conjectures」中觅得其初胚。这愿景是我那些年主要的灵感源泉,在我看来,那是我提炼出这一新几何学核心理念、锻造其主要工具的岁月。这愿景与这些核心理念对我而言已如第二天性。(而在近十五年与之断绝联系后,我今日惊觉,这“第二天性”在我心中依然鲜活!)它们对我如此简明、如此显而易见,我自然以为“所有人”都已随我一同将其内化并据为己有。直到最近数月,我才蓦然察觉,这愿景与支撑它的几条“核心理念”,既未成为我恒久的指引,也未在任何已发表的文本中被完整书写,至多隐于字里行间。更重要的是,我曾以为已传递的这愿景,以及承载它的那些核心理念,在其臻于成熟后的二十年间,至今仍无人知晓。作为发现这些事物的特权者,我——这工匠与仆人——竟也是唯一让它们存活之人。

我所锻造的某些工具,此处彼处被用来“破解”某公认棘手的问题,如同撬开保险箱。工具看似坚实。然而,我深知它们拥有超越撬棍之力的另一种“力量”。它们属于一个整体,如同肢体属于身体——这整体孕育了它们,赋予其意义,并为其注入力量。你可以取一根断骨去敲碎头颅,这是显而易见。但那并非其真正功能,其存在之由。我看到这些工具散落各处,被人取用,宛如从一具鲜活之躯上小心剥离、清理干净的骨头——而他们却佯装忽视这具活体……

我以纯粹过去的语气述说这些,是长时间反思的结果。而多年来,我必已在某种弥散的感知中逐渐体察到这些,未曾诉诸言语,未曾在意识中成形为思想与意象,也未以清晰的言辞表达。我曾决意,这过去与我再无瓜葛。那些偶尔传来的遥远回音,虽经层层过滤,却依然意味深长,只要我稍作停留。我曾自视为众多工匠中的一员,忙碌于五六个“工地”\footnote{我将在后续笔记“荒凉的工地”「Les chantiers désolés;Desolate Workyards」(第176至178号)中谈及这些废弃的“工地”,并逐一检视。那是三月前的记录。我首次重拾与我事业及其命运的联系,是在笔记“我的孤儿”「Mes orphelins;My Orphans」(第46号)。},热火朝天——或许是经验更丰富的工匠,是那个曾独自在这些地方耕耘多年的长者,在后援到来前默默劳作;是的,长者,但本质上与其他工匠无异。然而,当我离去,一切仿佛建筑公司因领班猝然去世而宣告破产;几乎一夜之间,工地荒废。那些“工匠”散去,各人携着自认为家中可用的零碎工具离去。资金已空,再无理由继续辛劳……

这仍是经过一年多反思与探究沉淀出的表述。但无疑,这感知早在离去后的最初几年便已“某处”萌生。撇开德利涅「Deligne;Deligne」关于Frobenius特征值的绝对值研究(即我最近理解的“声望问题”……),当我偶尔遇见昔日同工于这些工地的老学生,问及……“?”,总见他们做出同样意味深长的动作,双臂高举似求饶……显然,他们正忙于比我心之所系更重要的事;同样显然,尽管人人看似忙碌而重要,却鲜有实质进展。那本质已然消逝——赋予局部任务意义的\textbf{统一性},以及一种\textbf{温暖},我相信,皆不复存。余下的,是脱离整体的零散任务,各人守着自己的小小宝藏,或勉力使其增值。

尽管我曾试图抗拒,这一切的骤停仍令我隐隐悲伤:不再闻及模体「motifs;motives」、拓扑斯「topos;topoi」、六种运算「six opérations;six operations」、德拉姆系数「coefficients de De Rham;de Rham coefficients」、霍奇系数「coefficients de Hodge;Hodge coefficients」、那应将所有素数的$\ell$-adic系数与德拉姆系数联结于同一扇面的“神秘函子”「foncteur mystérieux;mysterious functor」,乃至晶体「cristaux;crystals」(只闻其仍在原地踏步),也无“标准猜想”「conjectures standard;standard conjectures」及其他我曾提出的、显然至关重要的问题之声。即便《代数几何基础》「Éléments de Géométrie Algébrique;Elements of Algebraic Geometry」(得益于迪厄多内「Dieudonné;Dieudonné」不懈协助)奠定的宏大基础工作,只需顺势推进,却也被弃置一旁:众人满足于入住他人耐心搭建、组装、擦亮的屋舍与家具。工匠离去,竟无人想到卷起袖子,拿起泥刀,为自己与众人建造尚待兴建的众多居所——那些宜居的家园……

我仍不禁再度以全然清醒的意象串联叙述,这些意象在反思中逐渐清晰并浮现。但我毫不怀疑,这些意象早已以某种形式存于我存在的深层。我必已感知到我事业与我自身一道被\textbf{埋葬}的隐秘现实,这现实于去年4月19日,以无可辩驳之力、以“埋葬”「L'Enterrement;The Burial」之名,骤然向我袭来。然而在意识层面,我未曾深感愤慨,甚至未觉悲哀。毕竟,无论昔日“亲近”与否,那仅关乎个人如何选择消磨时光。若昔日激励或启迪他的事物不再激发灵感,那是他的事,与我无关。若这情形完美无瑕地发生在每一位旧生身上,无一例外,那仍是各人自己的事,我自有他务缠身,无暇探究其意义,仅此而已!至于我留下的那些事物——那些与我仍存深邃而未觉联系的事物——尽管它们显然被弃于荒凉的工地,我深知,它们并非畏惧“时间之伤”或时尚变迁之物。若它们尚未融入共同遗产(如同我昔日所想),它们迟早必将扎根,或十年,或百年,终究无甚差别……