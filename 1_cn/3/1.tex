\section{千页长信}

1985年5月

我在此寄给你的这份文本,由我的大学帮忙打印并限量发行,但它既不是单行本,也不是预印本。其标题《收获与播种》(Récoltes et Semailles)已清晰表明了这一点。我把它当作一封长信寄给你——而且是一封极为私人的信。我之所以直接寄给你,而不是等你哪天(如果你有兴趣的话)在书店里看到某本书(如果有足够大胆的出版商愿意冒险出版……)时再去了解它,是因为我在信里更多是在和你交流。写这封信的时候,我多次想到你——说起来,我全身心投入写这封信已经一年多了。这是我送给你的礼物,写作时我格外用心,尽我所能(在每一刻)奉上我最好的内容。我不知道这份礼物是否会被接受——你的回应(或者没有回应……)会告诉我答案……

我在给你寄《收获与播种》的同时,也把它寄给了数学界里与我在某个时刻有过密切联系,或以某种方式出现在我思考中的所有同事、朋友或(前)学生,无论是否提及他们的名字。你很可能也在其中,如果你用心去读,而不只是用眼睛和头脑,即便没有提到你的名字,你肯定也能在文中找到自己的影子。我还把《收获与播种》寄给了其他一些朋友,无论他们是否从事科学工作。

你现在正在读的这篇“介绍信”,既向你预告又向你介绍了这封“千页长信”(先这么说……),它也将充当序言。写这些话的时候,真正的序言还未完成。此外,《收获与播种》由五个部分组成(不算“抽屉式”引言)。我在这里寄给你的是第一部分(《愚昧与更新》「Fatuité et Renouvellement;Folly and Renewal」)、第二部分(《葬礼(一)——或中国皇帝的新衣》「L'Enterrement (1) - ou la Robe de l'Empereur de Chine;The Burial (1) - or the Emperor of China's New Clothes」)和第四部分(《葬礼(三)——或四则运算》「L'Enterrement (3) - ou les Quatre Opérations;The Burial (3) - or the Four Operations」)\footnote{我把那些以某种方式出现在我的思考中,但我并不认识的同事排除在外。我只给他们寄《四则运算》(这部分与他们特别相关),同时附上包含这封信的“零号分册”,以及《收获与播种》的引言(还有前四个部分的详细目录)。}。我觉得这些部分可能与你尤其相关。第三部分(《葬礼(二)——或阴阳之钥》「L'Enterrement (2) - ou la Clef du Yin et du Yang;The Burial (2) - or the Key of Yin and Yang」)无疑是我这份见证中最私人的部分,同时,与其他部分相比,它似乎更具有一种超越其诞生背景的“普世”价值。我在第四部分(《四则运算》)中不时提及这一部分,不过,第四部分可以独立阅读,甚至(在很大程度上)可以独立于前两部分阅读\footnote{一般来说,你会发现《愚昧与更新》中的每个“章节”,或《收获与播种》接下来三部分中的每个“注释”都有其自身的完整性和独立性。它可以独立于其他部分阅读,就像我们可以饶有兴致地观察一只手、一只脚、一根手指、一只眼睛,或身体的任何大小部位,同时也不会忘记这是整体的一部分,正是这个整体(虽然未被言说)赋予了一切意义。}。

如果你读了我寄给你的这些内容后愿意给我回信(这正是我的期望),并且你还想读缺失的部分,请告诉我。只要你的回复让我觉得你的兴趣不只是表面的好奇,我会很乐意把缺失的部分寄给你。