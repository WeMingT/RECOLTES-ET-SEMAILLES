\section{运动与结构}

我想我已经说完了关于《丰收与播种》「Récoltes et Semailles;Harvests and Sowings」我想对你说的最重要的事情,至少让你大致明白“这是怎么回事”。当然,我说的已经足够多,足以让你判断接下来这封超过千页的信是否“与你有关”,以及你是否会继续阅读。如果答案是“是”,我觉得再附上一些关于《丰收与播种》形式的说明(特别是实用性的)会很有帮助。

这种形式是某种精神的反映与表达,我在前面几页中试图传递这种精神。相比我过去的出版物,如果说《丰收与播种》以及其源头《追寻场》「A la Poursuite des Champs;In Pursuit of Fields」中出现了一种新的特质,那无疑是自发性「spontanéité;spontaneity」。诚然,有贯穿始终的主线和大问题赋予整个反思以连贯性和统一性。然而,这反思却是日复一日展开的,没有预设的“计划”或“规划”,从不预先设定“必须证明什么”。我的目的不是证明,而是发现,深入探索未知的实质,让那些仅被预感、怀疑、瞥见的事物逐渐凝结。我可以毫不夸张地说,在这项工作中,没有一天或一夜的反思是在“预期”的领域中展开的——无论是当时浮现的观念、意象还是联想,当我坐在白纸前,执着地追寻一根顽强的“线索”,或拾起另一根刚浮现的线索时都是如此。每次,反思中浮现的东西都与我能预测的不同,如果我冒险试图提前描述我以为眼前的景象,也无法做到。更多时候,反思从一开始就走上完全未预料的路径,通向同样未预料的新风景。即使它遵循一条或多或少预定的路线,旅途中数小时的揭示也与我出发时的想象大相径庭——正如真实风景与明信片之别:真实风景有清凉阴影与温暖光线的交织,行者步伐下细腻变幻的视角,无数声响与无名芬芳随微风起舞,草木摇曳,林间低吟……这活生生、捉摸不定的风景,与一张明信片,哪怕再美、再精致、再“真实”,也截然不同。

由一天或一夜连续进行的反思构成不可分割的单位,仿佛整个反思(在此即《丰收与播种》)中的活细胞与个体。这反思的每一单位(或“注释”\footnote{最初在写《虚荣与更新》「Fatuité et Renouvellement;Vanity and Renewal」时,“注释”对我而言等同于“注解”,如同脚注。为排版方便,我选择将这些注解置于文本末尾(第1至44号注释,第141至171页)。这么做的一个原因是,某些“注释”或“注解”长达一页或多页,甚至超过它们所注释的正文。至于反思“初稿”的不可分单位,因找不到更贴切的名称,我当时称之为“节”(比“段落”更顺耳!)。

这种情形及文本结构在下一部分有所改变,该部分最初名为“葬礼”「L'Enterrement;The Burial」,后成为“葬礼(1)”「L'Enterrement (1);The Burial (1)」(或“中国皇帝的袍子”「La robe de l'Empereur de Chine;The Robe of the Emperor of China」)。此反思接续于双重注释“我的孤儿”「Mes orphelins;My Orphans」与“拒绝遗产——或矛盾的代价”「Refus d'un héritage - ou le prix d'une contradiction;Refusal of an Inheritance - or the Price of a Contradiction」(第46、47号注释,第177、192页),作为《丰收与播种》(或其第一部分《虚荣与更新》)最终“节”“过去的重量”「Le poids d'un passé;The Weight of a Past」(第50号,第131页)的注解。随后,又为此节添加了其他注解(第44'及50号注释),还有为“我的孤儿”添加的注解,这些注解又催生新的注解;此外,这次还有真正的脚注,当预定注解(写成后仍保持)篇幅适中时使用。于是,理论上,《丰收与播种》的这一部分(当时应为第二且最终部分)表现为“过去的重量”一节的“注释”集合。由于惯性,这种“注释”(而非“节”)的划分延续至后三部分,我同时使用脚注(若篇幅允许)和文中提及的后续注释,作为反思“初稿”的注解方式。

在排版上,“注释”与《丰收与播种》第一部分用作“初稿”基本单位的“节”区别在于标记,如(1)、(2)等(括号内数字上标,遵循注解引用的普遍惯例),置于注释开头或正文引用的适当位置。节则以阿拉伯数字1至50标记(避免繁琐的索引与上标,如注释因实用需求所用)。可以说,《丰收与播种》第一部分的“节”与后续部分的“注释”在功能上无本质差异。我在此信部分(“自发性与结构”)关于此功能的评论,同样适用于第一部分的“节”,尽管我常用“注释”一词。

关于其他细节与惯例,尤其是阅读《葬礼(1)》目录的说明,见引言第7节(葬礼的安排)「L'Ordonnancement des Obsèques;The Ordering of the Funeral」,特别第xiv-xv页。},构成“旋律……”)自此各有其名,获得独特身份与自主性。然而,有时因偶然原因中断的反思,会在次日或隔日自然延续;或连续数日的反思,回顾时仿佛一气呵成,似乎仅因睡眠需要被迫插入某种(生理上的)停顿,仅以简短日期标示(或多个日期)于连续段落间,该“注释”因单一名称而自成一体。

因此,《丰收与播种》的每则注释都有其独特性,有其面貌与功能,与其他注释迥异。对每则注释,我试图以其名称表达其独特本质,旨在还原或唤起其核心,或至少某些本质性的东西,即它“要说的”。每则注释,我首先通过其名辨识它,日后需要引用时也以此名呼唤。

名称常自发浮现,甚至在我想到命名之前。它意外出现时提醒我,正在写的这则注释即将完成——它已说完该说的,只待写完当前段落……名称也常在重读前日或前两日的注释时自发出现,继续反思前。有时,名称在注释诞生后的数日或数周略有调整,或增添我最初未想到的第二名称。许多注释有双重名称,表达信息的不同视角,有时互补。《虚荣与更新》初现的双重名称是“与克洛德·舍瓦利会面——或自由与善意”「Rencontre avec Claude Chevalley - ou liberté et bons sentiments;Meeting with Claude Chevalley - or Freedom and Good Sentiments」(第11号)。

仅两次我在开始注释前已有名称在脑海——两次都被后续事件颠覆!

仅在回顾时,隔数周乃至数月,才显现出整体运动与结构,在日复一日接续的注释中。我尝试通过注释的各种分组与子分组捕捉这两者,每组有其名称,赋予其独特存在与功能或信息;如同同一身体的器官与肢体(借用前述意象),及其部分。在《丰收与播种》“整体”中,有我已提及的五“部分”,各具独特结构:《虚荣与更新》分为八“章”I至VIII \footnote{在《虚荣与更新》中,我偶尔称这些章为《丰收与播种》的“部分”,当然不可与前述逐渐浮现的五部分混淆。},构成“葬礼”「L'Enterrement;The Burial」的三部分(也在数月中逐渐清晰)由十二“队列”「Cortèges;Corteges」I至XII的庄严长列组成。最后一队列,或更确切地说“葬礼仪式”「Cérémonie Funèbre;Funeral Ceremony」(其名如此),此前十一队列(多半未觉端倪……)朝其行进,其规模宏大,与其所悼念的伟业相称:几乎涵盖《丰收与播种》第三部分(葬礼(2))全部及第四部分(葬礼(3))全部,近八百页,约一百五十则注释(最初此仪式仅计划两则!)。由首席主持者亲自(以其著名的谦逊……)细腻引导,仪式分九“阶段”或礼拜行为,从《悼词》「Eloge Funèbre;Funeral Oration」(可想而知)开始,至最终《深渊呼声》「De Profundis;De Profundis」(理应如此)结束。其中两“阶段”,“阴与阳之钥”「La Clef du Yin et du Yang;The Key to Yin and Yang」与“四种运算”「Les Quatre Opérations;The Four Operations”,各占其所在部分(第三或第四)绝大部分,并以此命名。

贯穿《丰收与播种》,我如珍视眼珠般呵护目录,不断调整以适应未预料注释的持续涌入 \footnote{这些未预料注释包括“由脚注因篇幅过长而衍生”的注释。通常,我将其置于相关注释后,编号加'或",必要时加"'——避免繁琐地重新编号后续已写注释!这些由脚注衍生的注释,在《葬礼(1)》目录中以!标记。},尽力细腻反映反思的整体运动与逐渐浮现的精妙结构。在第三部分尤其是第四部分(刚提及),“钥匙”与“四种运算”,此结构最为复杂与交织。

为保留文本的自发性及反思的未预料性,如其真实展开与体验,我未让注释前置其名,因名称总在事后才浮现。故建议你在读完每则注释后查阅目录,了解其名称;偶尔也可一瞥其如何嵌入已展开的反思,甚至未来反思。否则,你可能在看似杂乱无章、编号有时怪异甚至繁琐的注释群中迷失 \footnote{关于这些看似怪异的编号缘由,见此冗长信的前一脚注。};如旅行者迷失于陌生城市(随世代与世纪的 capricieux 奇异生长……),无指南或地图指引。在准备印刷的手稿中,我计划在正文中加入“章”及其他注释与节的组名,唯独注释(或节)本身除外。但即便如此,偶尔查阅目录仍不可或缺,以免迷失于数百注释连绵千余页的混乱中……