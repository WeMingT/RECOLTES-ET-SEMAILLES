\section{三足鼎立于一盘}

许多我的数学家同事和朋友表达了希望,《收获与播种》「Récoltes et Semailles;Harvests and Sowings」能在数学界开启一场广泛的辩论,讨论数学界风气现状、数学家的伦理,以及其工作的意义与目的。然而,至少目前看来,事情并未朝这个方向发展。从现在起(为了玩一个必须的文字游戏),关于“葬礼”的辩论似乎完全被一场辩论的“葬礼”所取代!

尽管如此,无论人们愿不愿意,也不管大多数人的沉默与冷漠如何,一场辩论确实已经开启。它不太可能发展成一场真正的公开辩论,甚至(愿上帝保佑不会如此!)带有“官方”辩论的浮夸与僵硬。然而,许多人早已迅速采取行动,在尚未了解情况之前,就在内心深处将这场辩论关闭,凭借那永恒不变的共识——“一切在最好的世界(此处指数学世界)中皆为最佳”。不过,或许质疑最终会从外部逐渐浮现,通过那些“见证者”——他们并非来自同一圈子,不受其群体共识的束缚,因此也不会(即便在内心深处)感到自己被质疑。

在我收到的几乎所有反馈中,我注意到关于两个前提问题的相同困惑:由《收获与播种》所(至少是隐晦地)提出的“辩论”究竟针对什么;以及谁有资格了解并对此发表意见,或者说,谁能在充分知情的情况下形成自己的看法。对此,我想在此明确标出三个“参照点”。当然,这无法阻止那些执意保持困惑的人继续如此。但至少,对于那些想弄清楚事情真相的人,这或许能帮助他们不被四面八方的喧嚣(甚至包括那些最善意的……)所干扰。

a) 一些真诚的朋友向我保证,“一切终将好转”(我想,“一切”指的是那些不幸受损的“事物”……);我只需回归,“通过新的工作强行确立地位”,发表演讲等等——其他人会处理剩下的事。他们会慷慨地说:“我们对这个该死的格罗滕迪克「Grothendieck;Grothendieck」确实有些不公”,然后悄无声息地、或多或少带着信念地纠正错误 \footnote{我已多次注意到这类隐秘的迹象,表明人们已清楚意识到这头狮子苏醒了……};甚至,以一种父辈的姿态拍拍他的肩膀,称他为“伟大的数学家”,以此安抚这位总体上还算体面、却不幸表现出愤怒并掀起不受欢迎波澜的家伙。

正如这些朋友所暗示的,这绝非“松懈”或“让步”的问题。就我而言,我既不需要赞美,也不需要真诚的仰慕者,更不需要为了“我的”事业或任何事业而来的“盟友”。这不是关于我——我过得很好;也不是关于我的作品——它们自己会说话,哪怕是对聋子。如果这场辩论也涉及我和我的作品,那只是作为揭示其他事物的媒介,通过一场“葬礼”(确实是最具揭示性的事件之一)的现实。

如果有“某人”让我觉得应当引起警觉、忧虑和紧迫感,那绝不是我本人,甚至也不是我的任何“共同被埋葬者”。而是一个既难以捉摸又极为具体的集体存在,人们常常提及它,却从不愿仔细审视,它的名字叫“数学共同体”。

在过去几周里,我终于将它视为一个有血有肉的人,其身体正遭受深重的坏疽之苦。最好的食物、最精致的菜肴,在它体内却化为毒药,使病痛进一步扩散和加深。然而,它却有着一种无法抑制的贪食欲望,不断地填塞更多,仿佛这是掩饰病痛的方式——一种它无论如何也不愿正视的病痛。无论对它说些什么都是徒劳——最简单的词语也失去了意义。它们不再承载信息,而只触发恐惧与拒绝的开关……

b) 我的大多数同事或昔日好友,即便态度友善,在冒险发表意见时,也会谨慎地使用条件语气,如“如果真如所述……那确实不可接受”——以此心安理得地继续高枕无忧。我原以为自己已经表达得很清楚了……

在七个月的回顾后,我现在可以明确指出,《收获与播种》中报告并评论的事实几乎全部,其真实性无可争议。我将在后文提及少数例外情况,并会在适当位置逐一标明。对于其他所有事实,在《收获与播种》初稿完成后,我与几位主要相关者(即皮埃尔·德利涅「Pierre Deligne;Pierre Deligne」、让-皮埃尔·塞尔「Jean-Pierre Serre;Jean-Pierre Serre」和吕克·伊吕西「Luc Illusie;Luc Illusie」)进行了仔细对质,消除了细节上的错误,并就事实本身达成了明确无误的共识 \footnote{我很高兴在此向这三位表达我的感激,感谢他们在这一场合表现出的善意,并确认他们在涉及事实问题上的完全诚意。}。

因此,这场辩论绝非关乎事实的真实性——这一点并无争议,而是关于这些事实所描述的实践与态度是否应被视为理所当然的“正常”行为。

这里涉及的是一些实践,我在证言中(或许错误地……)称之为丑闻;如滥用信任或权力,以及显而易见的欺诈行为,不止一次达到了独一无二且厚颜无耻的程度。我在了解这些事实(十五年前还难以想象)后仍需学习的一件相当不可思议的事是,我的数学家同事中绝大多数,甚至包括我的学生或朋友,如今认为这些实践是正常的,且完全正当。

c) 我的许多同事和昔日朋友还有第二种保持困惑的方式。他们会说:“抱歉,我们不是这方面的专家——别要求我们了解那些(天意如此……)超出我们理解的事实……”

我却断言,恰恰相反,要了解主要事实,无需成为“专家”(我也感到抱歉!),甚至无需知道乘法表或毕达哥拉斯定理「Théorème de Pythagore;Pythagorean Theorem」。甚至不必读过《熙德》「Le Cid;Le Cid」或拉封丹寓言「Fables de la Fontaine;Fables of La Fontaine」。一个正常发育的十岁孩子完全有能力做到这一点,甚至比最知名的专家做得更好…… \footnote{当然,我写《收获与播种》并非针对十岁孩子,若要对他说话,我会选择他熟悉的语言。}

请允许我通过一个例子——“葬礼”中的“第一个例子”——来说明这一点 \footnote{这是我发现的首次“大型葬礼行动”,时间是1984年4月19日,那天“葬礼”这个名字也强加于我。参见同日所写的两篇笔记:“梦的回忆——或动机的诞生”和“葬礼——或新父亲”(Res III, n${}^{\circ}$s 51, 52)。其中还包括即将提及的书的完整参考。}。要了解以下几个事实并对此作出判断,无需了解“动机”「Motif;Motive」这一多面且极为微妙的数学概念的来龙去脉,也无需拥有小学毕业证书。

$1^{\circ}$) 在1963至1969年间,我引入了“动机”「Motif;Motive」的概念;并围绕这一概念发展了一种“哲学”和“理论”,部分仍属猜想性质。不管对错(此处无关紧要),我认为动机理论是我对当代数学最深刻的贡献。今日无人再否认“动机瑜伽”的重要性与深度(在我离开数学舞台后,关于它的近乎完全沉默持续了十年)。

$2^{\circ}$) 在第一本也是唯一一本(1981年出版)主要致力于动机理论的书中(其标题中包含我引入的“动机”一词),唯一可能让读者怀疑我这微不足道之人与书中详尽阐述的某种理论有任何关联的段落,出现在第261页。这段文字(两行半)向读者解释,书中发展的理论与某个名叫格罗滕迪克「Grothendieck;Grothendieck」的人的理论毫无关系(该理论在此首次也是最后一次被提及,无任何其他参考或说明)。

$3^{\circ}$) 有一个著名的猜想,称为“霍奇猜想”「Conjecture de Hodge;Hodge Conjecture」(具体内容在此无关紧要),其成立意味着该书中所谓“另一”动机理论,与我在近二十年前公开发展并为众人所知的理论(的一个非常特殊情况)完全相同。

我还可以补充 $4^{\circ}$) 该书的四位合著者中最负盛名者曾是我的学生,他那些在此书中作为新发现呈现的出色想法,无一不是多年来从我这里学来的 \footnote{我并非说该书中没有这位作者或其他合著者的原创且出色的想法。但书的全部问题框架、赋予其意义的观念背景,甚至包括技术上构成核心的$X$-范畴理论(被错误称为“塔纳卡式的”「Tannakien;Tannakian」),皆为我的作品。};以及 $5^{\circ}$) 这两点在消息灵通人士中是众所周知的,但你在文献中徒劳地寻找任何书面证据,证明这位杰出作者可能从我口中有所获益 \footnote{不过有一个例外,即塞尔「Serre;Serre」1977年的一份报告中的一行文字,将在适当位置提及。};以及 $6^{\circ}$) 该书的主要作者亲口向我解释,其核心问题是一个精妙的算术问题(未提及我的名字),而这个问题是我在六十年代通过“动机瑜伽”提出的,且他也是通过我得知的;我还可以继续列出 $7^{\circ}$ 和 $8^{\circ}$ 等等(我在适当位置确实未曾遗漏)。

以上所述足以阐明我的观点,即:了解此类事实并作出判断,无需任何特殊“能力”——问题不在这个层面上。此处涉及的能力,除了每个人理应具备的健全理性外,我称之为“体面感”。

这本书现已成为数学文献中最常被引用的著作之一,其“主要作者”也是当代最负盛名的数学家之一。话虽如此,在我看来,这件事中最引人注目的,如今是无数读者——包括那些亲知内情的人、我的学生或朋友——无人对此书感到任何异常。至少截至我写下这些文字的今日,无一人向我表达过对这本声名显赫的书的丝毫保留意见 \footnote{总共只有两位同事(包括佐格曼·梅布胡特「Zoghman Mebkhout;Zoghman Mebkhout」)向我表达过此类“保留”。他们二人都不能算是这本书的“读者”。他们只是出于好奇翻看了它,想了解一下……}。

至于那些从未拿过这本书的同事和昔日朋友,并以此为由声称无能为力,我对他们说:无需成为“专家”,只需在最近的数学图书馆借阅此书,翻阅一番,你便可亲自确认无人否认的事实……