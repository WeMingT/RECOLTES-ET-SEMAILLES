\section{尊重与坚韧}

然而,我又一次离题,提前述及反思教我的东西。我起初怀着双重意图,这在我尚未开始反思前已清晰存于心中:一是“意图的沉思”,二是(与之密切相关,如方才所示)表达对创造劳作本质的看法。然而,还有第三种意图,意识层面或不甚明朗,却回应更深层、更本质的需求。这需求源于那些时而令人困惑的“呼唤”,从我数学家过往传来,借由曾为我学生或友人(至少其中许多人)之口。在表层,这需求化为“倾囊而出”的渴望,吐露几许“刺耳真相”。但更深处,定是那\textbf{认知}某个过往的需要——我此前选择回避的过往。《收获与播种》正源于此需。这漫长反思是我对内心认知冲动与外部世界——我曾决意永别的那“数学界”——不断更新的呼唤的日日“回应”。除《虚荣与更新》「Fatuité et Renouvellement;Vanity and Renewal」最初几页,即构成其前两章的“劳作与发现”「Travail et découverte;Work and Discovery」与“梦与梦者”「Le rêve et le Rêveur;The Dream and the Dreamer」,自“恐惧之生”「Naissance de la crainte;Birth of Fear」(第18页)一章起,伴随未列入计划的“见证”,正是这认知过往并全然承担的需要(我相信),成为《收获与播种》书写中的主要驱力。

数学家世界的呼唤向我袭来,在《收获与播种》全程(尤在第二与第四部分的“探究”中)以新力回荡于我身。这呼唤初现便披上自满的面具,若非(“精心调配的”)轻蔑、嘲讽或鄙夷——或针对我(有时),或(尤甚)针对那些敢于受我启发之人(他们自未料及后果),被某种隐秘而无情的裁决“归类”为与我相关。于此,我再次见那“显而易见且深邃”的联系:在对他人的\textbf{尊重}(或缺尊重)、对创造行为及其最精微、最本质果实的尊重,以及对科学伦理最显明规则——植根于对己对人基本尊重的规则,我愿称之为我们技艺践行中的“体面规则”——的尊重之间。这些,定是“自尊”那基本且本质的诸多面向。若以一句简练之语总结《收获与播种》对那曾属我、逾二十年我与之同化的世界的教诲,我会说:这是一个\textbf{丧失尊重}的世界\footnote{此表述不仅适用于我近观其详的特定圈子,它似也概括了当代世界的某种普遍退化。(对比第1页第19行脚注。)在《收获与播种》“探究”的较窄框架内,此表述现于去年4月2日笔记“尊重”「Le respect;Respect」(第179号)。}。

这已是此前数年强烈感知、虽未明言之事。在《收获与播种》全程,它不断被证实与 уточ化,总出乎意料,有时令人震惊。自那“哲学”与通论性质的反思骤变为个人见证一刻——“受欢迎的异乡人”「L'étranger bienvenu;The Welcome Stranger」(第9号,第18页)开启前述“恐惧之生”章——它已清晰可见。

然这感知并非以尖刻或苦涩的责难之调呈现,而是(因书写的内在逻辑与它激发的不同态度)以一种\textbf{探问}之声:在这退化、在这我今日所见的尊重丧失中,我的责任何在?这是贯穿并支撑《收获与播种》第一部分的主问,直至其终以清晰无误的结论消解\footnote{于“竞技数学”「La mathématique sportive;Sporting Mathematics」与“边缘终结”「Fin la marge;End of the Margin」(第40、41号)节。}。此前,这退化似“从天而降”,莫名其妙,愈发令人愤慨难忍。反思中,我发现它早已悄然蔓延,五六十年代无人察觉——无论身周或自身,\textbf{包括我本人}。

这谦卑事实的确认——显而易见却无甚外观——标志见证的首个关键转折,及即刻的质变\footnote{次日,见证深化为自我沉思,在随后数周保持此特质,直至《收获与播种》“第一气息”终结(“过往之重”「Le poids d'un passé;The Weight of a Past」,第50号)。}。这是我需从数学家过往及自我中学到的首件要事。这对普遍退化中我所负\textbf{责任之份}的认知(反思中时而敏锐时而模糊),如基调与提醒,贯穿《收获与播种》。尤在反思转为探究时代失德与不公之时,伴随理解的渴望——那驱动一切真发现劳作的好奇——这谦卑认知(途中屡被遗忘,却总在最意料外处重现……)使我的见证(我相信)未堕为对世人之忘恩负义的徒劳责怨,或与某些曾为学生或友人(或兼具)者的“清算”。这对己无纵容,亦赋予我内心平静,或曰坚韧,使我免于对他人的纵容陷阱,甚至虚假“谨慎”之诱。我信自己需言之事,无论关于我、某同事、旧生或友人,或某圈子、某时代,皆在反思某刻道尽,未曾需强迫己之不情愿。每逢踌躇,只需细察,它们便消散无踪。