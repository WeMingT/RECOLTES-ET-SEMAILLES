\section{剥离}

正如我先前暗示,我须克服内心巨大的阻力——或更恰当地,通过耐心、细致而顽强的劳作化解之——方能摆脱某些根深蒂固的熟悉意象。这些意象惯性沉重,数十年来在我心中(如同众人,包括你,定然亦然)取代了对现实直接而细腻的感知——在此,即某个数学世界的现实,我因过往与事业与之相连。其中最深植的意象,或成见之一,是似乎天然排除一位国际知名的学者,乃至被视为伟大数学家之人,可能(即便偶尔,更遑论习以为常……)沾染大小欺诈;或若他(仍依旧习)未亲手为之,却能欣然接纳他人操办的此类行动“(有时公然悖逆体面之感)”,并因某种缘由从中获益。

我精神上的惰性如此之重,仅在不到两月前,历经整整一年的漫长反思,我才怯然瞥见塞尔「Jean-Pierre Serre;Jean-Pierre Serre」或亦与这埋葬「Enterrement;Burial」有关——如今这于我显而易见,即便不依仗我近日与他那雄辩的交谈。在我初入“布尔巴基学派「Bourbaki;Bourbaki」圈子”——彼时它以善意接纳我——的所有成员中,尤在他身上,我对其人存有一种隐秘的“禁忌”。他俨然某种“优雅”的化身——不仅限于形式,更涵括严谨与一丝不苟的正直。

去年4月19日发现埋葬前,我连梦中也未想过,我的学生中竟有人在其职业操守上不诚——无论对我或他人;对其中最杰出、与我最亲近者,此猜想尤显荒谬!然而,自离去之时起,乃至随后多年直至今日,我有充分机会察觉他与我的关系何其分裂。我亦多次见他(似仅为取乐)运用权力,挫败与羞辱他人,时机若恰。每逢如此,我深受触动(或超乎我愿承认的程度……)。这些迹象已足够雄辩,揭示一种深层失调,我有充分机会确认,这失调绝非仅限于他,即便在我最亲近的学生小圈中亦然。此失调,因丧失对他人的尊重,不比所谓“职业不诚”所显者更浅显或浅薄。即便如此,发现此不诚对我仍是全然意外与震撼。

此惊人启示后的数周,伴随一连串同类“瀑布”,我逐渐察觉,某些学生间的某种诡计\footnote{见前脚注。},早在离去前的数年已然萌芽。尤在最杰出者身上——我离去后,他定调并(如我先前所述)“悄然高效地领行动”——此尤为明显。近二十年回望,这诡计如今“刺眼”显明。若我当时选择闭目,追逐那“白鲸”,沉浸于“一切皆秩序与美”的世界(我乐于如此想象),我今察觉,我未担起对学生的责任——他们在与我共处中习得我热爱的职业;这职业不仅关乎技艺,或某种“嗅觉”的培养。因对杰出学生的纵容,我(以隐秘裁决)视其为“超然之存在”,无可疑之,我当时贡献了我的份量\footnote{此“贡献”尤见于笔记“超然之存在”「L'être à part;The Exceptional Being」(第67'号),及“攀登”「L'ascension;The Ascent」与“暧昧”「L'ambiguïté;The Ambiguity」(第63'、63''号),又于(略异视角)笔记“驱逐”「L'éviction;The Eviction」(第169号)末现。另一类“贡献”见《虚荣与更新》「Fatuité et Renouvellement;Vanity and Renewal」,对天赋稍逊的年轻数学家持虚荣态度。此对普遍退化中责任之份的觉醒,于“竞技数学”「La mathématique sportive;Sporting Mathematics」(第40号)节达顶峰。},助长了我今日所见那(似前所未有的)腐败,蔓延于曾珍视的世界与人之中。

诚然,鉴于其巨大惰性,我需剧烈而持续的劳作,方能摆脱俗称的“幻觉”(不无遗憾之调……),我宁称其为成见;关于我自己、曾认同的圈子、我曾爱且或仍爱之人——“摆脱”这些成见,或更恰当地,任其从我剥离。这确是劳作,然非挣扎——这劳作带给我诸多珍贵之物,时有悲怆,却无一刻遗憾或苦涩。苦涩乃逃避认知、逃避生命讯息之途;以对自身的顽固幻觉为代价,换取对世界与他人的另一“幻觉”(某种负向)。

我无苦涩无遗憾,见这些曾“珍贵”的成见——因旧习与“自古如此”——如累赘乃至压迫之重,一一从我剥离。它们确已成“第二本性”。然此“第二本性”非“我”。逐片剥离,非撕裂,亦非失落之挫败——如被夺珍宝者。我所述“剥离”,如劳作之报偿与果实。其标志是即刻而舒心的释然,一种受欢迎的解放。