\section{旅途}

我想,此刻我大致已勾勒出我“重返数学”的背景,以及由此一环扣一环诞生的《收获与播种》。去年三月末,在《虚荣与更新》的最后一节“过往之重”「Le poids d'un passé;The Weight of a Past」(第50号)中,我终于开始思索这意外归来的缘由与意义。至于“缘由”,最强烈的,定是那既弥散又迫切的印象:那些我曾以为托付于关爱之手的强劲而鲜活之物,在我十五年未曾顾及的时光里,竟被囚于墓穴,与风、雨、阳光的恩泽隔绝,黯然腐朽\footnote{参见“过往之重”「Le poids d'un passé;The Weight of a Past」(第50节),尤见第137页,(**)。}。我必是逐渐领悟——虽直到今日才明言——唯有我,方能劈开那朽烂的木板,释放这些活物,它们不应在封闭的棺椁中腐烂,而应在广阔天地间绽放。而那些围绕这堆衬垫过剩的棺椁(无疑映照那逝去的后裔……)的虚伪肃穆与隐秘嘲讽之风,也“终在我心中唤醒了近十年来稍显沉睡的斗志”,激起我投身混战的渴望……\footnote{引自笔记“墓中旋律——或自满”「La mélodie au tombeau - ou la suffisance;The Melody at the Tomb - or Complacency」(第167号),第826页。}

如此,两年前,一场原计划仅为数日或数周的匆匆探访——针对某个被弃的“工地”——演变为一部多卷的数学长篇,嵌入那赫赫有名的“反思”「Réflexions;Reflections」新系列(暂称“数学”,待剔除这多余限定)。况且,自知在撰写一部拟出版的数学著作那一刻起,我也明白,除了一篇大致遵循惯例的“数学”引言外,我还将附上另一篇更具个人色彩的“引言”。我感到有必要阐明我的“归来”——这绝非重返某个\textbf{环境},而仅是回归炽烈的数学投入与我亲笔数学文本的发表,且时限未定。同时,我想说明我如今书写数学的心态,与离去前的著述精神在某些方面迥异——如同一场发现之旅的“航海日志”。此外,我心头还有其他思绪,或许与此相关,却更觉本质。于我而言,自不待言,我将从容述说心中所想。这些尚显模糊的事物,与我即将书写的卷帙及嵌入其中的“反思”之意义密不可分。我无意草草提及,如同为占用匆忙读者的宝贵时间而致歉。若《追寻域》「À la Poursuite des Champs;In Pursuit of Fields」中有内容值得他及众人知晓,正是在这引言中我欲保留的述说。若二三十页不足以言尽,我便写上四十,甚至五十页,无妨——况且我并未强求谁来读我……

《收获与播种》由此诞生。1983年6月,我在《追寻域》第一卷写作的间隙,写下引言的最初几页。随后推迟至去年2月,当时此卷已近数月基本完稿\footnote{[其间]我花了整整一月,思索一个伪直线系统的“结构表面”「surface structurale;structural surface」,以所有伪直线相对于该系统的“可能相对位置”集合表达。我还写了“纲领草图”「L'Esquisse d'un Programme;Sketch of a Program」,将收录于《反思》第3卷。}。我原指望这引言能让我厘清心中一二尚存模糊之事。但我未曾料到,它将如刚写就的那卷一般,成为一场\textbf{发现之旅};这旅途通往一个远比我计划探查的——已写之卷及后续诸卷——更丰饶、更广袤的世界。在日复一日、周复一周、月复一月的书写中,我未太觉察 проис之事,这新旅程悄然延续,探寻某个过往(三十余年来固执回避……)、自我及与那过往的联结;也探寻数学界中曾与我亲近、却知之甚少的某些人;甚至趁势更展开一场数学发现之旅——十五或二十年来首次\footnote{在五六十年代,我常压抑追逐这些诱人而炽热问题的冲动,被无尽的基础工作占据,无人能或愿代我前行,而我离去后,亦无人真心接续……},我得以悠然重拾离去时留下的炽热问题。可以说,在《收获与播种》的篇页中,我追寻着\textbf{三重}交织的发现之旅。直至第十二百余页的句点,无一旅程告终。我见证的回响(乃至沉默的回声……)将成为旅途的“续篇”。至于其终点,这旅程或属永无完结之类——甚至,或许至死未竟……

我终于绕回起点:若可能,预先告知你《收获与播种》“何以为题”。但诚然,前述篇页未刻意为之,已多少道尽。或许更引人入胜的,是顺势\textbf{讲述},而非“预告”。

\begin{flushright}
1985年6月
\end{flushright}