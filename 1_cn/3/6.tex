\section{阴影之坡——或创造与轻蔑}

前述篇页写于上月短暂的“空隙时刻”。其间,我终于为“四种运算”「Quatre Opérations;Four Operations」(《收获与播种》的第四部分)作了最后润色——如今只剩完成这封信或“前信”(它竟也悄然膨胀至令人咋舌的篇幅……),一切便可付诸打字与复制。近一年半来,我始终“即将完成”这些著名笔记,已几不抱希望!去年2月(乃至前年6月),我着手这部数学著作那有些异类的“引言”时,我相信自己主要想表达三类心声。首先,我想阐明重拾数学活动的意图,述说我以何种精神写就《追寻域》「À la Poursuite des Champs;In Pursuit of Fields」首卷(刚宣布完稿),以及我将以何种精神继续更广阔的数学探查与发现之旅,融入“反思”「Réflexions;Reflections」。今后,我不再为孕育中的新数学宇宙呈现一丝不苟、四平八稳的基础。那更像是“航海日志”,记录日复一日的劳作,不加掩饰,真切如其所是——带着失误与错误,固执的回溯,亦有突如其来的跃进——这劳作日复一日被无形之线牵引向前(尽管意外与波折无数),仿佛追逐某种隐秘、顽强而确凿的愿景。这劳作常是摸索,尤其在“敏感时刻”,当某种无名无貌的直觉隐约浮现,微不可察;或在某新旅程启始,追随最初的几抹灵感与直觉——它们常隐晦,抗拒被语言之网捕获,而恰恰是那能轻柔捕捉它们的恰当语言,尚付阙如。于是,首先需从那看似虚无的朦胧迷雾中凝炼出这语言。那仅存预感的,尚未瞥见,更遑论“看见”或触及之物,逐渐从无形中澄清,褪去阴影与迷雾的外衣,成形、具肉、赋重……

这部分劳作,外观寒碜,甚至(往往)不免狼狈,却是其中最精微、最本质的部分——真正有新物诞生的时刻,源于强烈的专注、深切的关怀、对那脆弱而纤细、即将降生之物的尊重。这是创造之巅的部分——孕育与漫长孕育的部分,在滋养母体的温暖幽暗中,从不可见的双重原初配子,化为无形胚胎,经日月暗中剧烈的劳作,无形无貌,蜕变为血肉之躯的新生命。

这也是“幽暗”部分,发现劳作的“阴”或“\textbf{女性}”面向。其互补面,“明亮”部分,或“阳”或“\textbf{男性}”,更似以锤或大锤敲击,施于锋利凿子或坚韧钢楔。(这些工具已备好待用,其效用久经考验……)两者各有其存在之理与功用,相依共生,密不可分——或更恰当地说,它们是阴与阳,是原初两股宇宙力量那不可解之侣的配偶,其永续更新的拥抱,不断唤起孕育、孕育与诞生的幽暗创造劳作——\textbf{新生儿},新事物之诞生。

第二件我感到需在个人且“哲学”的“引言”中表达之事,正关乎创造劳作的本质。我已察觉多年,这本质常被忽视,被陈词滥调掩盖,被古老的压抑与恐惧遮蔽。其深重程度,我仅在随后逐渐发现,日复一日、月复一月,贯穿《收获与播种》的反思与“探究”。自这反思的“起航一刻”,在1983年6月的几页中,我首次被这看似琐碎却惊人事实的深意攫住——只要稍作停留便可察觉:我刚述及的发现劳作中“创造之巅”的部分,在理应呈现此劳作(或至少其最 tangible 果实)的文本或话语中,几乎\textbf{无迹可寻};无论教科书及其他教学文本、原创论文与专著、口述课程与研讨会报告,皆然。仿佛自数学及其他艺术与科学肇始,千年以降,围绕这些“\textbf{不可告人之劳}”——一切新观念(无论大小)萌发前奏,更新我们对这永恒创生之世界的认知——存有一种“沉默共谋”。

坦白说,对这最关键面向或阶段——一切发现劳作(及一般创造劳作)之核心——认知的压抑,似已如此彻底,且被亲历此劳作之人内化至深,以至常令人疑心,即便这些人,也已从意识记忆中抹去其痕迹。恰如极端清教社会中,一位母亲对其每个须她擦鼻涕、拭臀的孩子,抹去了受孕时(不情愿承受的)拥抱记忆、妊娠的漫长月日(视为失礼),以及分娩的漫长时辰(忍受为纯粹恶心的苦难,终以解脱告终)。

这比喻或显夸张,若我将其用于二十年前我所熟知的数学界精神——我曾身处其中——或许的确如此。但在《收获与播种》的反思中,我逐渐察觉,尤其在近数月(撰写“四种运算”时)触目惊心,自我退出数学舞台后,我所熟知的圈子——乃至(至少在很大程度上)整个数学界——其主导精神发生了惊人的\textbf{退化}\footnote{此退化绝不仅限于“数学界”。它亦见于整个科学生活,乃至当代全球世界。相关观察与反思初探,见于继阴阳反思后的笔记“尊重与坚韧”「Le respect et la fortitude;Respect and Fortitude」(第106号)。}。或许因我独特的数学个性及离去时的境况,我的退出如催化剂,加速了一场已在进行中的演变\footnote{此演变于前述脚注引用的笔记中审视。其与(我个人及事业的)埋葬「Enterrement;Burial」之关联,在“被埋葬的阴阳(4)”「le yin et le yang enterrés yin (4);the buried yin and yang (4)」、「天赐之机——或极致」「La circonstance providentielle - ou l'Apothéose;The Providential Circumstance - or the Apotheosis」、「否定(1)——或召回」「Le désaveu (1) - ou le rappel;The Disavowal (1) - or the Recall」、「否定(2)——或蜕变」「Le désaveu (2) - ou la métamorphose;The Disavowal (2) - or the Metamorphosis」(第124、151、152、153号)反思中渐显清晰。另见《收获与播种 IV》较新笔记“无用细节”「Les détails inutiles;Unnecessary Details」(第171号 (v))及注 (c) “看似无物之物——或枯竭”「Des choses qui ressemblent à rien - ou le dessèchement;Things That Resemble Nothing - or the Drying Up」及“家族相册”「L'album de famille;Family Album」(第173号,部分 c. “至尊者——或默许”「Celui entre tous - ou l'acquiescement;He Among All - or the Acquiescence」)。}——这场演变,我当时毫无察觉(除克洛德·舍瓦利耶「Claude Chevalley;Claude Chevalley」或为唯一例外,我的同事与友人皆然)。我在此尤思及的退化面向(仅为诸多面向之一\footnote{《收获与播种》中最常聚焦,且尤于两“探究”部分(《收获与播种 II》或“中国皇帝之袍”「La robe de l'Empereur de Chine;The Robe of the Emperor of China」及《收获与播种 IV》或“四种运算”)最令我震动的面向,或为职业伦理的退化,表现为某些当世最显赫、最杰出的数学家,肆无忌惮地掠夺、贬低、粉饰,且(很大程度上)众目睽睽。其他更微妙且与之直接相关的面向,见前述引注(第106号部分 c.)“看似无物之物——或枯竭”。}),是那\textbf{隐秘轻蔑}——若非毫不掩饰的嘲讽——针对(以数学为例)非纯然锤砧或凿击之劳作的一切;是对最精微(常外观平平)创造过程的轻蔑;是对\textbf{灵感}、\textbf{梦想}、\textbf{愿景}(无论多么强大与丰饶)的轻蔑,甚至(极端而言)对任何观念——无论构思与表述多么清晰——的轻蔑:凡未以黑白分明的严苛命题形式书写与发表、可编目与已编目、继而存入我们巨型计算机无尽内存“数据库”者,皆受此蔑视。

正如C.L.席格尔「C.L. Siegel;C.L. Siegel」所言\footnote{此表述引用于前述脚注所引笔记,并有评述。},发生了一场惊人的“\textbf{扁平化}”,数学思想的“\textbf{缩减}”,被剥夺了本质维度,失去了全部“阴影之坡”,那“阴”之面向。诚然,依古老传统,这发现劳作之一面多被遮蔽,几无人\textbf{言及}——但与滋养伟大愿景与宏图的梦想深源的活联系,(据我所知)从未断绝。如今,我们似已步入一\textbf{枯竭时代},此源未干涸,然通往它的路径却被普遍轻蔑的无情裁决与嘲讽的报复封锁。

我们似正趋近一刻,不仅每个人的\textbf{记忆}中,与源头亲近的、“女性”的劳作(被讥为“晦涩”、“软弱”、“空洞”——或相反为“琐碎”、“儿戏”、“闲逛”……)将被抹除,连这劳作本身及其果实——孕育、构思与诞生新概念与愿景之处——亦将被连根拔起。那也将是我们技艺沦为干枯而徒劳的脑力“举重与杠铃”展示的时代,竞相以技艺“破解”竞赛难题(“谚语般的难度”)——一时代,充斥狂热而贫瘠的“超级雄性”膨胀,接替逾三世纪的创造更新。