\section{运动中的四波}

如其所然,这封信完全不像我开始写时所预想的那样。我原以为主要会写一个小小的“概要”关于葬礼「Enterrement;Burial」:事情大致是这样发生的,你信不信由你(我自己都难以置信……),但事实确实如此,无可辩驳,不管你喜欢与否,黑纸白字的出版物,某某期刊或某某书籍,某年某月某页,只要一看便知——况且这一切在《丰收与播种》「Récoltes et Semailles;Harvests and Sowings」中都已详细拆解;参见“四种运算”「Quatre Opérations;Four Operations」那样的注释——接不接受随你!如果你宁愿不读我写的东西,自然会有别人代你承担……

结果这些都没写——然而这封信已然写到了三十页,而我原计划总共不过五六页。甚至没刻意为之,我却在字里行间被引导着对你说了些本质性的东西。而那个我曾迫不及待想清空的“袋子”(一开始就明摆在那儿,在头几页里!),至今仍未打开!我的手指甚至不再为此蠢蠢欲动,那股冲动已在途中消散。我明白了,这里并不是合适的地方……

说实话,《丰收与播种》的第四部分(也是最长的一部分),名为“葬礼(3)”「L'Enterrement (3);The Burial (3)」或“四种运算”「Les Quatre Opérations;The Four Operations」,源于一个最初打算作为“小概要”的“注释”,用来大致总结去年那场令人意外的调查(匆匆一瞥)所揭示的内容,那调查在第二部分(“葬礼(1)”「L'Enterrement (1);The Burial (1)」,或“中国皇帝的袍子”「La robe de l'Empereur de Chine;The Robe of the Emperor of China」)中得以延续。我原以为不过写个五到十页的“注释”,仅此而已。结果却一发不可收拾,调查重新启动,最终写出了近四百页——几乎是我原打算总结或平衡的那部分的两倍!这意味着那个小概要始终缺席,尽管《丰收与播种》的六百页都致力于葬礼的调查。这有点蠢,的确。但总还有时间在引言的第三部分(反正也不差这十页二十页)中补上,然后将我的笔记交给印刷商。

《丰收与播种》的五个部分(最后一部分尚未完成,估计还要几个月才能完稿)呈现出(三波)“冥想”与(两波)“调查”的交替。这仿佛是我过去九年生活的缩影,那也是一场“波浪”的交替,源自我如今生活中占主导的两种激情:冥想的激情与数学的激情。实话说,我方才草草称为“调查”的《丰收与播种》的两个部分(或“波浪”),正是直接源于我作为数学家过去的根基,由我内心的数学激情以及随之植根的自我依附所裸露而出。

第一波,“虚荣与更新”「Fatuité et Renouvellement;Vanity and Renewal」,是我与作为数学家的过去的初次相遇,引向对当下的冥想,我刚发现当下植根于这过去。虽非丝毫预谋,但这一部分为《丰收与播种》的后续奠定了“基调”,如同一场内在的准备,天意使然且不可或缺,为迎接紧随其后的第二波中“葬礼的全部壮丽”「Enterrement dans toute sa splendeur;Burial in all its splendor」的发现做好铺垫,即“葬礼(1)——或中国皇帝的袍子”「L'Enterrement (1) - ou la robe de l'Empereur de Chine;The Burial (1) - or the Robe of the Emperor of China」。与其说是“调查”,不如说这是那发现日复一日的故事,它对我的存在产生的影响,我努力面对这突如其来砸向我的东西,试图以我的经历、那些最终变得熟悉的事物来定位这不可思议的存在,尽力让它变得可理解。这运动导向一个初步的暂时结果,在注释“掘墓人——或整个会众”「Le Fossoyeur - ou la Congrégation toute entière;The Gravedigger - or the Entire Congregation」(第97号)中,这是首次尝试为某件多年来、如今比以往任何时候都更尖锐地呈现为对常识的可怕挑战的事物,辨明解释与意义!

这第二波运动还导向一个“疾病插曲”\footnote{此插曲涉及两则注释“事件——或身体与精神”「L'incident - ou le corps et l'esprit;The Incident - or Body and Mind」及“陷阱——或轻松与疲惫”「Le piège - ou facilité et épuisement;The Trap - or Ease and Exhaustion」(第98、99号),开启了名为“逝者(尚未死去)”「Le défunt (toujours pas décédé);The Deceased (Still Not Dead)」的“队列十一”「Cortège XI;Cortege XI」。},迫使我完全休息,超过三个月停止一切智力活动。那时我以为自己又一次即将完成《丰收与播种》(只剩些最后的“杂务”……)。去年九月末恢复正常活动,准备为那些搁置的笔记画上最后句点时,我仍以为只需再添两三则终结性注释,包括一则关于我刚经历的“健康事件”。事实上,接连数周、数月,又写了一千页——比已写部分的两倍还多——这次我清楚地知道还没完稿\footnote{“尚未完稿”——哪怕仅因第五部分尚未完成,写下这些时仍未结束。}!实际上,这长长的中断期间,我几乎与那尚炽热(甚至灼热)的素材失去联系,离开时它正热火朝天,这几乎迫使我以全新目光重审这素材,若不想只是愚蠢地“草草收尾”一个已失去活联系的“计划”。

由此诞生了《丰收与播种》这广大运动中的第三波——一波关于阴与阳「yin et yang;yin and yang」的漫长“冥想波”,探讨事物动态与人类存在中的“阴影”与“光明”两面。这波冥想源于对葬礼「Enterrement;Burial」中深层力量更深理解的渴望,却从一开始就获得自身的自主性与统一性,直指最普遍与最私密个人的东西。在这冥想中我发现(若稍加思索便显而易见),我在探索事物——无论数学或其他——时的自发方式,其“基调”是“阴”的、“女性”的;且令人惊讶的是,相较于常情,我始终忠于内在的原始本性\footnote{这“忠于原始本性”并非完全彻底。长期以来,它仅限于我的数学工作,而在其他方面,尤其是与他人的关系中,我随大流,推崇并优先展现我认为“阳刚”的特质,压抑“阴柔”特质。这在注释组“人生故事:三乐章循环”「Histoire d'une vie : un cycle en trois mouvements;Story of a Life: A Cycle in Three Movements」(第107-110号)中有详细探讨,开启了《阴与阳之钥》「Clef du Yin et du Yang;Key to Yin and Yang」。},从未屈从或修正以适应惯常做法,我始终忠于这内在本性,未加扭曲或调整以迎合周围环境中盛行的主导价值。这发现起初仅是好奇,逐渐却显露为理解葬礼的关键。更重要的是——这在我看来意义更深远——我如今无比清晰、无一丝疑虑地看到:若非拥有超常的智力天赋,我却能在数学工作中持续发挥全力,孕育出广阔、强大且丰饶的作品与视野,全归功于这忠实,归功于我不屑于遵循规范,因而能全然信赖原始的认知冲动,不修剪、不削弱其力量、细腻与完整本性。

然而,在这“葬礼(2)——或阴与阳之钥”「L'Enterrement (2) - ou la Clef du Yin et du Yang;The Burial (2) - or the Key to Yin and Yang」的冥想中,关注的中心并非创造力及其源泉,而是“冲突”,创造力的阻塞状态,或因内心对立(往往隐秘的)力量对抗而导致的创造能量的分散。暴力、看似“无端”、为“乐趣”的暴力特质,在葬礼中多次令我困惑,唤起无数类似经历。这暴力体验在我生命中如“冲突经验中坚韧的核心”。我从未直面暴力存在及其在人类普遍存在、尤其在我个人存在中的可怕谜团。这谜团贯穿阴与阳冥想的后半部分(“阴”或“衰退”面),成为关注核心。在此部分中,逐渐浮现对葬礼意义及其表达力量的更深刻洞察。这也是《丰收与播种》中自我认知最丰饶的部分,我感到,它将我与关键问题和情境相连,让我正视那直到去年仍被回避的“关键性”。

完成这漫长的“阴与阳”离题后,我仍需写“两三则注释”(至多再加一两则,其中一则已命名“四种运算”……),以结束《丰收与播种》。后续已知:这“最后几则注释”成了《丰收与播种》中最长部分,近五百页。这是运动的“第四波”,也是葬礼的第三且最后部分,我命名为“四种运算”「Les Quatre Opérations;The Four Operations」,其核心是注释组“四种运算(在一具遗体上)”「Les quatre opérations (sur une dépouille);The Four Operations (on a Corpse)」。这是《丰收与播种》中最严格意义上的“调查”部分——然而带点调味,因这调查不限于纯“技术”或“侦探”层面,而是如书中其他部分一样,主要由认知与理解的渴望驱动。语气比葬礼第一部分更“有力”,那时我还在揉眼睛,怀疑是否在梦中!即便如此,页面揭示的事实常恰到好处,以生动实例阐释此前仅略提及的内容。这是数学离题占据重要地位的部分,因调查需要重新接触十五年未触及的素材。光谱另一端,有我友佐格曼·梅布胡特「Zoghman Mebkhout;Zoghman Mebkhout」(此部分献给他)的即时叙述,他落入无耻高阶“黑帮”之手,始料未及,只因涉足(看似无害却引人入胜的)各类流形的上同调「cohomologie;cohomology」主题。为理清这部分“调查”的繁复笔记、子笔记、子子笔记迷宫,我指向目录(第167'至$176_{7}$号注释),及首则注释“侦探——或玫瑰人生”「Le détective - ou la vie en rose;The Detective - or Life in Pink」(第$167^{\prime}$号)。但需指出,此注释(4月22日)后被“事件超车”,因调查在随后两月仍跌宕起伏,我原以为已近尾声。

这第四波持续逾四月,从二月中至六月末。尤其在此部分,通过对“证据”的细致顽强工作,日复一日、页复一页,与葬礼现实建立具体、可触的联系;我逐渐“熟悉”它,尽管它引发的本能拒绝(且持续如此)阻碍真正认知。这长反思始于对德利涅「Deligne;Deligne」拜访的回顾(信中已提),止于对与塞尔「Serre;Serre」关系及塞尔在葬礼中角色的“最后一刻”反思\footnote{在注释“家族相册”「L'album de famille;Family Album」(第$173$号)的c、d、e部分,最后一节日期为6月18日(恰十日前)。仅一则注释或片段日期更晚,即“五论屠杀——或孝道”「Cinq thèses pour un massacre - ou la piété filiale;Five Theses for a Massacre - or Filial Piety」(第$176_{7}$号,次日6月19日)。你会注意到,此第四部分或“调查部分”,与其他部分不同,注释常依逻辑而非时间顺序排列。如葬礼最后两则注释(构成最终“深渊呼声”「De Profundis;De Profundis」)日期为4月7日,比前述注释早两月半。需说明,除葬礼(3)的“调查”核心(第$167^{\prime}-176_{7}$号注释,葬礼仪式的“第五阶段”,阴与阳之钥为第二阶段)外,注释多按写作顺序排列,少有例外。}。此前我因“禁忌”(已提及)默认塞尔“无责”,这或是我理解葬礼的最大缺憾,至上月仍存——这“最后一刻”反思遂成第四波带给我最重要的东西,使我对葬礼及其力量的把握更扎实、更丰满。