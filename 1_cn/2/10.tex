\section{“新几何”——或数与量的联姻}

我又跑题了——我本想谈谈那些主导主题,如何如同众河归海般汇聚于一个共同的母愿景……

这一宏大的统一愿景可被描述为一种新几何。据说,这是上世纪克罗内克(Kronecker,Kronecker)所梦想的几何 \footnote{我对“克罗内克之梦”的了解仅来自传闻,有人(或许是约翰·塔特(John Tate,John Tate))告诉我,我正在实现这个梦想。在我从前辈那里接受的教育中,历史参考极为罕见。我的滋养并非来自阅读古今作者,而是主要通过与其他数学家的直接交流——口头或书信——尤其是我的前辈。1958 年概形理论(théorie des schémas,theory of schemes)突然而有力的起步,其主要(或许唯一)的外部灵感,来自塞尔(Serre,Serre)那篇广为人知的文章,简称 FAC(《相干代数层》(Faisceaux algébriques cohérents,Coherent Algebraic Sheaves)),发表于几年前。除此之外,我在后续发展中的主要灵感源自理论本身,并在多年中通过追求内在简洁性与一致性的需求不断更新,以在新背景下解释代数几何(géométrie algébrique,algebraic geometry)中“众所周知”的内容(这些内容在我手中逐渐转化),并由这些“已知”引导我预感更深层次的东西。}。然而,现实(有时大胆的梦想能预示或瞥见,并激励我们去发现……)总是以其丰富性与共鸣超越最勇敢或最深刻的梦想。无疑,对于新几何的许多方面(若非全部),在它出现前夕,无人曾想到——连工匠自己也不例外。

可以说,“数”(nombre,number)擅于捕捉“离散”或“不连续”的聚合结构:那些通常有限的系统,由彼此“孤立”的“元素”或“对象”组成,缺乏从一到另一的“连续过渡”原则。而“量”(grandeur,magnitude)则是最适于“连续变化”的品质,因此擅于捕捉连续的结构与现象:运动、空间、各类“簇”(variétés,varieties)、力场等。于是,算术(arithmétique,arithmetic)大致是离散结构的科学,而分析(analyse,analysis)是连续结构的科学。

至于几何(géométrie,geometry),自两千多年前作为现代意义上的科学存在以来,它一直“跨立”于这两种结构——“离散”与“连续”之间 \footnote{实际上,传统上几何学家关注的焦点是“连续”面向,而“离散”性质,尤其是数值与组合特性,常被忽略或轻视。十年前,我惊叹于发现二十面体(icosaèdre,icosahedron)的组合理论之丰富,而这一主题在克莱因(Klein,Klein)关于二十面体的经典著作中甚至未被触及(很可能也未被察觉)。几何学家两千年来忽视自然融入几何的离散结构的另一个显著例证是:群(groupe,group)(尤其是对称群)的概念直到上世纪才出现,且最初由伽罗瓦(Évariste Galois,Évariste Galois)引入时,背景并不被视为“几何”。即便今日,许多代数学家仍未认识到伽罗瓦理论(théorie de Galois,Galois theory)本质上是一种“几何”愿景,革新了我们对所谓“算术”现象的理解……}。长期以来,并未真正出现两种几何的“分裂”——一种离散,另一种连续。更确切地说,对同一几何图形的探究存在两种不同视角:一种强调“离散”性质(特别是数值与组合特性),另一种关注“连续”性质(如在周围空间中的位置,或以点间距离测量的“量”等)。

直到上世纪末,随着所谓“抽象(代数)几何”(géométrie (algébrique) abstraite,abstract (algebraic) geometry)的出现与发展,这种分裂才显现。大致而言,它为每个素数 $p$ 引入了一种“特征 $p$ 的(代数)几何”(géométrie (algébrique) de caractéristique $p$,(algebraic) geometry of characteristic $p$),模仿前几个世纪继承的(连续)代数几何模型,但置于一个看似无可避免的“离散”与“不连续”背景中。这些新几何对象自本世纪初以来日益重要,尤其是因其与算术——离散结构科学的密切关联。安德烈·韦伊(André Weil,André Weil)的作品似乎以此为指导思想之一 \footnote{安德烈·韦伊,移居美国的法国数学家,是“布尔巴基学派”(Bourbaki,Bourbaki)的创始成员之一,将在《收获与播种》(Récoltes et Semailles,Harvests and Sowings)第一部分中多有提及(偶尔也会提到韦伊本人)。},或许是最主要的潜在推动力(在其书面作品中或多或少未明言,如常理),即“代数几何”(géométrie algébrique,algebraic geometry),特别是与不同素数相关的“离散”几何,应为算术的大规模革新提供钥匙。正是在此精神下,他于 1949 年提出了著名的“韦伊猜想”(conjectures de Weil,Weil conjectures)。这些猜想着实令人叹为观止,它们为这些离散性质的新“簇”(variétés,varieties)或“空间”(espaces,spaces)揭示了某些构造与论证的可能性 \footnote{(致数学读者。)这里指的是与可微或复数簇的上同调理论相关的“构造与论证”,特别是涉及勒夫谢茨不动点公式(formule des points fixes de Lefschetz,Lefschetz fixed-point formula)和霍奇理论(théorie de Hodge,Hodge theory)的那些。},此前这些仅在分析家眼中“名副其实”的“空间”——即所谓“拓扑空间”(espaces topologiques,topological spaces,连续变化概念适用的空间)——框架内才看似可行。

可以说,新几何首先是对这两个世界——迄今邻近且紧密相连却又分隔的世界——的综合:一是“算术”世界,居住着无连续性原则的(所谓的)“空间”;二是连续量的世界,居住着分析家眼中“真正”的“空间”,因其可被分析工具触及而被接纳为数学城邦的合法居民。在新愿景中,这两个曾经分离的世界合而为一。

这一“算术几何”(géométrie arithmétique,arithmetic geometry,我提议如此称呼新几何)愿景的最初萌芽见于韦伊猜想。在我若干主要主题的发展中 \footnote{指“中间四主题”(第 5 至 8 号),即拓扑斯(topos,topos)、埃塔尔与 $\ell$-进上同调(cohomologie étale et $\ell$-adique,étale and $\ell$-adic cohomology)、模体(motifs,motives),以及(较次要的)晶体(cristaux,crystals)。这些主题是我在 1958 至 1966 年间陆续提出的。},这些猜想在 1958 至 1969 年间始终是我主要的灵感源泉。早在我之前,奥斯卡·扎里斯基(Oscar Zariski,Oscar Zariski)和随后的让-皮埃尔·塞尔(Jean-Pierre Serre,Jean-Pierre Serre)已为抽象代数几何中那些桀骜不驯的空间发展出某些“拓扑”方法,灵感来自此前适用于众人认可的“正统空间”的技术 \footnote{(致数学读者。)扎里斯基在此方向的主要贡献在我看来是引入“扎里斯基拓扑”(topologie de Zariski,Zariski topology,后成为塞尔在 FAC 中的关键工具)、“连通性原理”(principe de connexité,connectedness principle)及他所谓的“全纯函数理论”(théorie des fonctions holomorphes,theory of holomorphic functions)——在他手中演变为形式概形(schémas formels,formal schemes)理论,以及形式与代数间的“比较定理”(théorèmes de comparaison,comparison theorems),其第二灵感源自塞尔的基础性文章 GAGA。至于我文中提到的塞尔贡献,首当其冲的是他在抽象代数几何中引入层(faisceaux,sheaves)的观点(该概念由让·勒雷(Jean Leray,Jean Leray)约十二年前在截然不同的背景下提出),见于前述基础性文章 FAC(《相干代数层》)。

结合这些“回顾”,若要为新几何愿景命名直接“先祖”,奥斯卡·扎里斯基、安德烈·韦伊、让·勒雷和让-皮埃尔·塞尔的名字立即浮现。其中塞尔角色尤为特殊,因我主要通过他不仅接触到他自身的思想,还了解到扎里斯基、韦伊和勒雷的思想,这些在几何新愿景的萌发与发展中都起到作用。}。

他们的思想在我构建算术几何的最初步伐中自然扮演了重要角色;不过,与其说是持续滋养我梦想与计划的灵感源泉,不如说是出发点与工具(为适应更广阔的背景,我不得不或多或少从头重塑这些工具)。无论如何,一开始就很清楚,即便重塑,这些工具仍远不足以迈向那些奇幻猜想的第一步。