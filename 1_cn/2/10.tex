\section{新几何学——或数与量的联姻}

然而,我又一次偏离了主题——我本打算谈论那些汇聚于同一母愿景中的主题,就像众多河流最终回归它们所源自的海洋……

这种广阔的统合性愿景可以被描述为一种新几何学。据说,这是克朗内克(Kronecker)在上个世纪所梦想的\footnote{我仅通过传闻了解到这个“克朗内克的梦想”,当时有人(可能是约翰·泰特(John Tate))告诉我,我正在实现这个梦想。在我从前辈那里接受的教育中,历史参考极为罕见,我的知识来源并非阅读古代或当代作者的著作,而主要是通过与其他数学家的口头或书信交流,尤其是我的前辈们。1958年,概形理论的突然而迅猛的启动,主要的外部灵感或许是唯一的灵感,来自塞尔(Serre)那篇以FAC(“代数凝聚层”)为名的著名文章,该文章发表于几年前。除此之外,我在理论后续发展中的主要灵感源于其自身,并随着时间推移不断更新,仅仅是为了满足内在的简洁性和一致性要求,努力在这一新背景下解释代数几何中“众所周知”的内容(这些内容在我手中逐渐转化),以及这些“已知”内容让我隐约感受到的东西。}。但现实(一个大胆的梦想有时会预示或隐约揭示它,并鼓励我们去发现……)每一次都在丰富性和共鸣上超越了最大胆或最深刻的梦想。当然,对于这种新几何学的许多方面(如果不是全部),在其出现的前一天,没有人会想到——包括工匠本人。

可以说,“数”能够捕捉“离散”或“不连续”聚合体的结构:这些系统通常由孤立的“元素”或“对象”组成,彼此之间没有某种“连续过渡”的原则。而“量”则是一种能够“连续变化”的特质;因此,它能够捕捉连续的结构和现象:运动、空间、各种“流形”、力场等。因此,算术(大致上)表现为离散结构的科学,而分析则表现为连续结构的科学。

至于几何学,可以说,自从它作为一门现代意义上的科学存在两千多年以来,它一直“横跨”这两种结构类型:“离散”和“连续”\footnote{事实上,传统上几何学家的注意力集中在“连续”方面,而“离散”性质的特征,尤其是数值和组合性质,则被忽视或草率处理。大约十年前,我惊奇地发现了二十面体的组合理论的丰富性,而这一主题在克莱因(Klein)关于二十面体的经典著作中甚至未被提及(甚至可能未被注意到)。我认为,几何学家对几何中自发引入的离散结构的忽视(长达两千年)的另一个显著标志是,群(尤其是对称群)的概念直到上个世纪才出现,而且它最初(由埃瓦里斯特·伽罗瓦(Évariste Galois))是在一个当时不被视为“几何”领域的背景下引入的。事实上,即使在今天,仍有许多代数学家尚未理解,伽罗瓦理论本质上是一种“几何”视角,它革新了我们对所谓“算术”现象的理解……}。长期以来,这两种几何学之间并没有真正的“分离”,它们并非属于不同种类,一种是离散的,另一种是连续的。相反,它们是对同一几何图形的两种不同研究视角:一种强调“离散”性质(尤其是数值和组合性质),另一种则强调“连续”性质(例如在周围空间中的位置,或通过点之间距离测量的“量”等)。

直到上个世纪末,随着所谓的“抽象(代数)几何”的出现和发展,一种分离才开始显现。大致上,这种几何学为每个素数$p$引入了一种“特征$p$”的(代数)几何,它模仿了从前几个世纪继承下来的(连续)几何模型,但其背景却显得不可还原地“离散”。这些新的几何对象自本世纪初以来变得越来越重要,尤其是因为它们与算术——离散结构的科学——之间的紧密联系。这似乎是安德烈·韦伊(André Weil)作品中的主导思想之一\footnote{安德烈·韦伊,法国数学家,移民美国,是“布尔巴基学派”的“创始成员”之一,在《收获与播种》的第一部分中将多次提及该学派(以及韦伊本人)。},甚至可能是主要的思想动力(在其著作中或多或少保持隐晦,正如应有的那样),即“代数几何”,尤其是与不同素数相关的“离散”几何,应该为算术的广泛革新提供关键。正是在这种精神下,他在1949年提出了著名的“韦伊猜想”。这些猜想绝对令人惊叹,它们为这些新的“离散”性质的“流形”(或“空间”)预示了某些类型的构造和论证的可能性\footnote{(针对数学读者。)这里指的是与微分流形或复流形的上同调理论相关的“构造和论证”,尤其是涉及莱夫谢茨(Lefschetz)不动点公式和霍奇(Hodge)理论的那些。},而这些构造和论证在此之前似乎只能在分析学家认为值得称为“空间”的框架内进行——即所谓的“拓扑”空间(其中连续变化的概念成立)。

可以说,新几何学首先是这两个世界之间的综合:一个是“算术”世界,其中生活着没有连续性原则的(所谓的)“空间”;另一个是连续量的世界,其中生活着分析学家认可的“空间”,这些空间因其可访问性而被视为值得存在于数学领域。在新愿景中,这两个曾经分离的世界合二为一。

这种“算术几何”(我提议这样称呼这种新几何学)愿景的雏形可以在韦伊猜想中找到。在我某些主要主题的发展中\footnote{这里指的是四个“中间”主题(编号5至8),即平展上同调和$\ell$-进上同调的拓扑、动机理论,以及(在较小程度上)晶体理论。我在1958年至1966年间逐步提炼出这些主题。},这些猜想一直是我在1958年至1969年间的主要灵感来源。事实上,在我之前,奥斯卡·扎里斯基(Oscar Zariski)和让-皮埃尔·塞尔(Jean-Pierre Serre)已经为“抽象”代数几何中的无规则空间开发了一些“拓扑”方法,这些方法受到了此前用于“正统”空间的方法的启发\footnote{(针对数学读者。)在我看来,扎里斯基在这方面的主要贡献是引入了“扎里斯基拓扑”(后来成为塞尔在FAC中的重要工具),以及他的“连通性原理”和他所谓的“全纯函数理论”——这些在他手中发展为形式概形理论和形式与代数之间的“比较定理”(塞尔的另一篇基础文章GAGA是第二个灵感来源)。至于我在文中提到的塞尔的贡献,当然主要是他在抽象代数几何中引入了层观点(由让·勒雷(Jean Leray)在十几年前在一个完全不同的背景下引入),这体现在他另一篇已提及的基础文章FAC(“代数凝聚层”)中。

根据这些“回顾”,如果要列举新几何学愿景的直接“祖先”,我立刻想到的名字是奥斯卡·扎里斯基、安德烈·韦伊、让·勒雷和让-皮埃尔·塞尔。其中,塞尔扮演了一个特殊的角色,因为正是通过他,我不仅了解了他自己的思想,还了解了扎里斯基、韦伊和勒雷的思想,这些思想在新几何学的萌芽和发展中发挥了重要作用。}。

当然,他们的思想在我构建算术几何的初期发挥了重要作用;然而,它们更多是作为起点和工具(我需要根据更广泛的需求对其进行彻底改造),而非一种持续滋养我梦想和计划的灵感来源。无论如何,从一开始就很清楚,即使经过改造,这些工具也远远不足以迈出朝向那些惊人猜想的第一步。