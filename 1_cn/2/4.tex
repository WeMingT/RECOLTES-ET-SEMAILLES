\section{风俗画卷}

在述及我作为数学家的过往时,以及其后(仿佛身不由己地)发现我作品那场宏大“葬礼”的曲折与隐秘时,我无意间被引向描绘某个环境、某个时代的图景——一个见证某些赋予人类工作意义之价值分解的时代。这是“风俗画卷”的面向,围绕着一桩在“科学”编年史上或许独一无二的“杂闻”展开。我在前文所述已足够清晰,我想,你不会在《丰收与播种》(Récoltes et Semailles,Harvests and Sowings)中找到一份关于某桩“非凡事件”的“卷宗”,让你匆匆了解概况。某位朋友一心寻觅这卷宗,却闭目不视,几乎错过了构成《丰收与播种》实质与血肉的一切。

正如我在“信件”(Lettre,Letter)中更为详尽的解释,这场“调查”(或曰“风俗画卷”)主要在第二与第四部分展开,即“葬礼(一)——或中国皇帝的新衣”(L'Enterrement (1) - ou la robe de l'Empereur de Chine,The Burial (1) - or the Emperor of China's Robe)和“葬礼(三)——或四种运算”(L'Enterrement (3) - ou les Quatre Opérations,The Burial (3) - or the Four Operations)。在这些页面中,我执着地将一桩桩鲜活的事实(至少可如此形容)逐一挖掘,时而艰难地试图将其安放妥当。这些事实渐渐聚合成一幅整体画卷,从迷雾中浮现,色彩愈发鲜明,轮廓愈加清晰。在这日复一日的笔记中,新现的“原始事实”与个人回忆、心理与哲学的评论与反思,甚至(偶尔)数学的思索,交织难分。这便是其样貌,我无能为力!

基于我耗费一年多心血完成的工作,若要以“调查结论”的形式整理出一份卷宗,对感兴趣的读者而言,依其好奇心与严谨度,或许仅需数小时或数日额外努力。我曾一度尝试整理这所谓的卷宗。那是在我开始撰写一篇原定名为“四种运算”的笔记时\footnote{该笔记最终分化为《丰收与播种》第四部分(同名“四种运算”),包含约70篇笔记,延展至四百余页。}。然而,不行,毫无办法。我做不到!这显然非我的表达方式,尤其在晚年愈发如此。如今我认为,《丰收与播种》已为“数学共同体”贡献足够,我可无憾地将整理“必要卷宗”的任务留给他人(若我同事中有谁觉此事与己相关)。
