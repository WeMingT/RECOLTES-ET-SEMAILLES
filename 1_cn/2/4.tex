
\section{风俗画卷}

谈及我作为数学家的过往,随后又(仿佛身不由己地)发现了我作品巨大葬礼中的曲折与奥秘,我无意中被引导去描绘某个特定环境与时代的图景——一个以某些赋予人类工作意义的价值解体为标志的时代。这便是围绕“科学”年鉴中无疑独一无二的“杂闻”所勾勒出的“风俗画卷”面貌。我想,我之前所言已足够清晰地表明,你不会在《收获与播种》中找到关于某件不寻常“事件”的“档案”,以便快速了解情况。然而,某位寻找档案的朋友,却闭目而过,几乎错过了构成《收获与播种》实质与血肉的一切。

正如我在信中更为详尽地解释的那样,“调查”(或“风俗画卷”)主要在第二部分和第四部分继续展开,即“葬礼(1)——或中国皇帝的袍子”与“葬礼(3)——或四则运算”。随着页面的推进,我坚持不懈地逐一揭示出众多(至少可以说是)引人入胜的事实,并尽力将它们适时“安置”。渐渐地,这些事实汇聚成一幅整体图景,逐渐从迷雾中显现,色彩愈发鲜明,轮廓愈加清晰。在这些日常笔记中,刚浮现的“原始事实”与个人回忆、心理、哲学乃至(偶尔)数学性质的评论和思考错综复杂地交织在一起。事情就是这样,我无能为力!

基于我耗时一年多的紧张工作,以“调查结论”风格整理出一份档案,对于感兴趣的读者而言,根据其好奇心与要求,可能只需额外花费几小时或几天的时间。我曾一度尝试整理那份著名的档案。那是在我开始撰写一篇本应名为“四则运算”的笔记时\footnote{该笔记最终扩展为《收获与播种》第四部分(同名“四则运算”),包含约70条笔记,跨越近四百页。}。但后来,我放弃了。我做不到!这显然不是我的表达风格,尤其是在晚年,更是如此。如今,我认为通过《收获与播种》,我已为“数学界”的利益做了足够多,可以毫无愧疚地将整理必要“档案”的任务留给他人(如果我的同事中有感到关切者)。