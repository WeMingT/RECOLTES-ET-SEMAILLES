\section{“孩子与母亲”}

当这“前言”(avant-propos,preface)开始转为一场漫步,穿越我作为数学家的毕生之作,伴着我对“继承者”(héritiers,heirs,正统者)与“建造者”(bâtisseurs,builders,不改初衷者)的简短议论,一个未成形的前言命名也随之浮现:它应是“孩子与建造者”(L'enfant et le bâtisseur,The Child and the Builder)。随后几日,越发清晰的是,“孩子”(l'enfant,the child)与“建造者”(le bâtisseur,the builder)乃同一人。于是,这名字简化为“建造之子”(L'enfant bâtisseur,The Builder Child)。这名字,凭心而论,不失气度,正合我意!

然而,反思揭示,那高傲的“建造者”,或(更谦逊些)那嬉戏建屋的孩子,仅是那玩耍之子(enfant-qui-joue,child-who-plays)的一面,而此子有双面。还有那爱探物之子(enfant-qui-aime-à-explorer-les-choses,child-who-loves-to-explore-things),钻入沙中,埋进无名泥泞,寻访最不可思议、最荒诞之地……或许为掩饰(哪怕仅对自己),我先以“先锋”(pionnier,pioneer)的耀名引之,后以更朴实却仍带光环的“探险者”(explorateur,explorer)随之。细想,“建造者”与“先锋-探险者”(pionnier-explorateur,pioneer-explorer),孰更雄性、更诱人?如掷币难决?

再细察,这无畏“先锋”竟是女孩(我曾乐于装扮她为男孩)——沼泽、雨水、细雨与夜之姊妹,沉默且几近隐于影中——那常被遗忘者(若非遭人佯笑……)。我也连日忘却她——可说双重遗忘:起初只愿见那男孩(嬉戏建屋者)——即便后来不得不认出另一面,仍将她视为男孩……

如此,那漫步的美名(beau nom,beautiful name)顿不成立。它全“阳”(yang,yang)、尽“刚”(macho,macho),一瘸之名。要平稳,须并列另一面。奇的是,“另一面”(l'autre,the other)无真名。唯一稍贴者是“探险者”,却仍是男孩之称,奈何。语言在此狡黠,暗设陷阱,与古老偏见串通,令人无觉。

或可改为“建屋之子与探秘之子”(L'enfant-qui-bâtit et l'enfant-qui-explore,The Child Who Builds and the Child Who Explores)。隐去一为“男孩”、一为“女孩”,实乃同一子,雌雄兼具,建中探,探中建……但昨日,除那观探之“阴-阳”(yin-yang,yin-yang)双面与命名建造之别外,另一面向复现。

“宇宙”(Univers,Universe)、“世界”(Monde,World)、乃至“Kosmos”(Cosmos,Cosmos),本质疏远且遥远,与我们无真切关联。非它们引动我们深处的认知冲动(pulsion de connaissance,impulse of knowledge)。吸引我们的是其有形即时之化身(Incarnation tangible et immédiate,tangible and immediate incarnation),最亲近、最“血肉”(charnelle,carnal)的,饱含深邃回响与神秘——那与我肉身及种族起源交融者,亦自古静待我于“路之彼端”(l'autre bout du chemin,the other end of the path),沉默而迎。她是“母亲”(la Mère,the Mother),生我如生世界者,冲动自她涌出,欲念之路向她延伸,引我与之相会,奔她而去,循环归返,沉没于她。

如此,在一场未料的“漫步”(promenade,stroll)转角,我猝不及防重拾一则熟悉却稍忘的寓言——“孩子与母亲”之喻(parabole de l'enfant et la Mère,parable of the child and the Mother)。可视为“生命追寻自身”之喻(La Vie, à la quête d'elle-même,Life, in quest of itself)。或于个体存有的谦卑层面,为“存在探物”之喻(l'être, à la quête des choses,being, in quest of things)。

这是寓言,亦是植根心魂(psyché,psyche)深处的古老体验表达——滋养深层创造的原初象征中最有力者。我信其以亘古意象语言,述说了人之创造力气息(souffle,breath),赋予肉身与精神生命,自最卑微短暂至最辉煌恒久的显现。

此“气息”,如其血肉化身,是世上最谦卑之物。亦最脆弱,最被忽视、最遭轻蔑……

此气息在你一生中的际遇,便是你之历险,你生命中的“认知冒险”(aventure de connaissance,adventure of knowledge)。无声表达此喻的,是“孩子与母亲”之寓言。

你是孩子,自“母亲”而出,庇于她中,受她大力滋养。孩子自“母亲”——那极近、熟知者——跃出,奔向“母亲”——那无垠、永未知且神秘者……

\hfill “穿越毕生之作的漫步”终
