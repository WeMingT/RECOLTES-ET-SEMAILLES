
\section{孩子与母亲}

当这篇“前言”开始在我作为数学家的作品中漫步,伴随着我对“继承者”(纯正血统)和“建设者”(不可救药)的小小论述时,一个名字也开始浮现于这篇未竟的前言之中:“孩子与建设者”。随后的日子里,越来越清晰地显现出,“孩子”与“建设者”实则是同一个人物。于是,这个名字简化为“建设中的孩子”。一个,说实话,颇具风采的名字,完全合我心意!

然而,深思之下,这位高傲的“建设者”,或更谦逊地说,那个喜欢搭建房屋的孩子,不过是著名“玩耍的孩子”的一面,而后者拥有两面。还有那个喜欢探索事物、潜入沙地或无名泥潭,深入最不可能、最奇异之地的小孩……为了混淆视听(至少对自己而言……),我起初以“先锋”这一耀眼之名引入,随后是更为朴实却仍带光环的“探索者”。在“建设者”与“先锋-探索者”之间,真让人疑惑,哪一个更具男子气概,更吸引人!正面还是反面?

再细看之下,我们勇敢的“先锋”原来是个女孩(我曾乐于将她装扮成男孩)——她是池塘、雨水、薄雾与夜晚的姐妹,静默且几乎隐形,因总隐于阴影之中——那个总被遗忘的人(当人们不假装嘲笑她时……)。而我,日复一日,也找到了双重遗忘她的方式:起初我只愿看见那个男孩(那个搭建房屋的……)——即便最终不得不看见另一个,我仍视其为男孩……

至于我那漫步的美丽名字,此刻已不再适用。这是一个全阳、全“男子气概”的名字,一个跛脚的名字。要让它站稳,必须同时包含另一方。但奇怪的是,“另一方”并无确切名字。唯一勉强贴切的是“探索者”,但这仍是个男孩的名字,无可奈何。语言在此成了陷阱,不知不觉中让我们陷入,显然与古老的偏见同谋。

或许我们可以用“建设中的孩子与探索中的孩子”来解围。不明确指出一个是“男孩”,另一个是“女孩”,而是同一个男孩女孩,在建设中探索,在探索中建设……但昨日,除了这阴阳两面的观照与探索、命名与构建之外,事物的另一面也显现出来。

宇宙、世界,乃至宇宙,本质上是陌生且遥远的。它们并不真正关乎我们。内心深处,求知欲并非指向它们。吸引我们的,是它们具体而直接的化身,最亲近、最“肉感”,充满深刻共鸣与丰富神秘——那与我们的血肉之躯及物种起源相混淆的存在,也是自古以来默默等待、准备迎接我们的“道路尽头”。正是她,母亲,那个孕育了我们如同孕育了世界的她,涌动着欲望的冲动,开辟着欲望之路——它们引领我们与她相遇,朝她奔去,不断回归并消融于她之中。

于是,在一次意外的“漫步”途中,我意外地重拾了一个曾经熟悉却稍被遗忘的寓言——孩子与母亲的寓言。这可以被视为“生命,追寻自我”的寓言。或者,在个体存在更为谦卑的层面上,是“存在,追寻事物”的寓言。

这是一个寓言,也是深深植根于心灵深处的祖先经验的表达——滋养深层创造力的最强大原始象征之一。我相信,在其中,我认出了以原型图像那古老语言表达的,正是人类创造力的气息,它激活了人的肉体与精神,在其最谦卑、最短暂的表现中,如同在最辉煌、最持久的显现里。

这“气息”,正如体现它的肉体形象一样,是世间最谦卑之物。它也是最脆弱的,最被众人忽视与轻视的……

而这一气息在你存在历程中的兴衰史,正是你的冒险,你生命中的“认知冒险”。表达这一点的无言寓言,便是孩子与母亲的故事。

你是那孩子,源自母亲,受她庇护,汲取她的力量。孩子从母亲——那最亲近、最熟悉的怀抱中跃出,去迎接那无限、永远未知且充满神秘的母性……

\hfill “穿越一部作品的漫步”结束