\section{动机——或核心中的核心}

拓扑斯(topos)主题源于概形(schemes)主题,甚至与概形的出现同年——但其范围远远超出了母主题。正是拓扑斯主题,而非概形主题,成为了那个“床”或“深河”,几何与代数、拓扑与算术、数理逻辑与范畴论、连续世界与“离散”或“不连续”结构的世界在此联姻。如果说概形主题是新几何学的核心,那么拓扑斯主题则是其外壳或居所。它是我构思的最广阔的主题,旨在通过一种富有几何共鸣的语言,精细地捕捉数学世界中来自不同领域情境的“本质”。

然而,拓扑斯主题远未像概形主题那样广为人知。我在《收获与播种》中多次谈及这一点,这里不再赘述这一概念所经历的奇特波折。尽管如此,新几何学的两个主要主题——两个互补的上同调理论——都源于拓扑斯主题,它们旨在为韦伊猜想提供一种途径:平展(或“$\ell$-进”)主题和晶体主题。前者在我手中具体化为$\ell$-进上同调工具,如今已成为本世纪最强大的数学工具之一。而晶体主题在我离开后几乎被埋没,最终在1981年6月因需求压力而被重新发掘,以一种借用的名义登上舞台,其背景甚至比拓扑斯更为奇特。

$\ell$-进上同调工具如预期般成为证明韦伊猜想的关键工具。我自己证明了其中的大部分,而最后一步则由我最杰出的“上同调学家”学生皮埃尔·德利涅(Pierre Deligne)在我离开三年后以大师级的手法完成。

此外,我在1968年左右提出了韦伊猜想的一个更强且更“几何”的版本。这些猜想仍然“沾染”(如果可以这么说!)着一种看似不可还原的“算术”特征,尽管这些猜想的精神正是通过“几何”(或“连续”)来表达和捕捉“算术”(或“离散”)\footnote{(针对数学读者)韦伊猜想依赖于“算术”性质的假设,尤其是所考虑的流形必须定义在有限域上。从上同调形式主义的角度来看,这导致与这种情境相关的弗罗贝尼乌斯自同态(Frobenius endomorphism)被赋予特殊地位。在我的方法中,关键性质(如“广义指标定理”类型)涉及任意代数对应,并且不对预先给定的基域做任何算术性质的假设。}。从这个意义上说,我提出的猜想版本比韦伊本人的版本更“忠实”于“韦伊哲学”——这种未书写且极少被提及的哲学,或许是过去四十年几何学非凡发展的主要潜在动机\footnote{然而,在我1970年离开后,出现了一种明显的反动运动,导致了相对的停滞,我在《收获与播种》中多次提及这一点。}。我的重述主要是提炼出一种“精华”,即在所谓的“抽象”代数流形框架内,保留经典“霍奇理论”中对“普通”代数流形有效的部分\footnote{“普通”在这里意味着“定义在复数域上”。霍奇理论(即“调和积分理论”)是复数域上代数流形背景下最强大的已知上同调理论。}。我将这一全新的、完全几何化的猜想版本称为“标准猜想”(针对代数循环)。

在我看来,这是在发展$\ell$-进上同调工具之后,迈向这些猜想的又一步。但与此同时,更重要的是,这也是通向我认为自己在数学中引入的最深刻主题的可能途径之一\footnote{这是最深刻的主题,至少在我作为数学家的“公开”活动期间(1950年至1969年,即我离开数学舞台之前)是如此。我认为从1977年开始发展的阿贝尔几何和伽罗瓦-泰希米勒理论的深度与之相当。}:动机主题(它本身源于“$\ell$-进上同调主题”)。这一主题如同概形主题的核心或灵魂,是概形主题中最隐秘、最难以窥见的部分,而概形主题本身又是新愿景的核心。标准猜想中提炼出的几个关键现象\footnote{(针对代数几何读者)这些猜想可能需要重新表述。更详细的评论请参见《工地巡礼》(ReS IV 注释第178号,第1215-1216页)和《信念与知识》(ReS III,注释第162号)中的脚注。}可以被视为动机主题的终极精华,如同这一最微妙主题的“生命气息”,是新几何学“核心中的核心”。

简而言之,事情是这样的。我们已经看到,对于给定的素数$p$,构建“特征$p$”的“(代数)流形”的“上同调理论”的重要性(尤其是为了韦伊猜想)。而著名的“$\ell$-进上同调工具”正好提供了这样一种理论,甚至提供了无限多种不同的上同调理论,即每一种都与不同于$p$的素数$\ell$相关联。然而,这里显然还缺少一种理论,即对应于$\ell$等于$p$的情况。为了解决这个问题,我特意构思了另一种上同调理论(即前面提到的“晶体上同调”)。此外,在$p$为无穷的重要情况下,我们还有另外三种上同调理论\footnote{(针对数学读者)这些理论分别对应于贝蒂上同调(通过基域嵌入复数域的超越方法定义)、霍奇上同调(由塞尔定义)和德·拉姆上同调(由我定义),后两种理论早在五十年代就已出现(而贝蒂上同调则出现在上个世纪)。}——并且没有证据表明我们不会在不久的将来引入更多具有类似形式性质的新上同调理论。与普通拓扑学不同,我们在这里面对的是令人困惑的多种上同调理论。我们有一种非常清晰的感觉,即这些理论在某种起初相当模糊的意义上应该“殊途同归”,它们“给出了相同的结果”\footnote{(针对数学读者)例如,如果$f$是代数流形$X$的自同态,诱导出上同调空间$H^{i}(X)$的自同态,那么后者的“特征多项式”应该是整系数的,且不依赖于所选的特定上同调理论(例如:$\ell$-进,$\ell$可变)。对于一般的代数对应,当$X$被假设为紧致且光滑时,情况也是如此。可悲的事实是(这让人对$p>0$特征的代数流形上同调理论在我离开后的悲惨状态有所了解),即使在$X$是光滑射影曲面且$i=2$的特殊情况下,这一点至今仍未得到证明。事实上,据我所知,在我离开后,还没有人愿意关注这一关键问题,它是标准猜想的典型子问题。时尚的法令是,唯一值得关注的自同态是弗罗贝尼乌斯自同态(德利涅用现有方法单独处理了它……)。}。

正是为了表达这种不同上同调理论之间“亲缘关系”的直觉,我提出了与代数流形相关的“动机”概念。通过这一术语,我试图暗示它是潜在于流形所有可能上同调理论中不同上同调不变量的“共同动机”(或“共同理由”)。这些不同的上同调理论就像是对同一“基本动机”(称为“上同调动机理论”)的不同主题发展,每一种都有其独特的“节奏”、“调性”和“模式”(“大调”或“小调”),而这一基本动机同时也是所有这些不同“主题化身”(即所有可能的上同调理论)中最基本或最“精细”的。因此,与代数流形相关的动机将构成“终极”上同调不变量,所有其他不变量(与不同上同调理论相关)都将从中推导出来,就像不同的“音乐化身”或“实现”一样。流形“上同调”的所有基本性质都已经“绑定”(或“听到”)在相应的动机上,因此特定上同调不变量(例如$\ell$-进或晶体)上的熟悉性质和结构,仅仅是动机内部性质和结构的忠实反映\footnote{(针对数学读者)另一种看待域$k$上动机范畴的方式是将其视为$k$上有限型分离概形范畴的一种“阿贝尔包络范畴”。与这样一个概形$X$相关的动机(或“$X$的动机上同调”,我记为$H_{\text {mot}}^{*}(X)$)因此表现为$X$的一种“阿贝尔化化身”。这里的关键在于,正如代数流形$X$可能“连续变化”(其同构类因此依赖于“连续参数”或“模”),与$X$相关的动机,或更一般地,一个“可变”动机,也可能连续变化。这是动机上同调的一个方面,与所有经典上同调不变量(包括$\ell$-进不变量)形成鲜明对比,唯一的例外是复代数流形的霍奇上同调。

这让我们对“动机上同调”作为一种更精细的不变量有了一个概念,它比纯粹的拓扑不变量更紧密地捕捉了流形$X$的“算术形式”(尽管这一表达可能有些冒险)。
在我的动机愿景中,动机构成了一种非常隐秘且微妙的“纽带”,将流形$X$的代数几何性质与由其动机体现的“算术”性质联系起来。后者可以被视为本质上具有“几何”性质的对象,但其中从属于几何的“算术”性质可以说被“赤裸裸地”展现出来。

因此,动机在我看来是迄今为止与代数流形相关的最深刻的“形式不变量”,除了其“动机基本群”$\pi_1$和最近被我视为“动机同伦类型”的“阴影”的另一个不变量(我在《工地巡礼——或工具与愿景》(ReS IV,第178号,参见第5章动机)中顺便提到了一些,尤其是第1214页)。在我看来,后者应该是任意代数流形的“算术形式”(或“动机”)这一难以捉摸的直觉的最完美体现。}。

这是用一种非技术的音乐隐喻语言表达的,一种既简单又微妙且大胆的想法的精华。我在1963年至1969年间,在完成我认为更紧迫的基础任务之余,以“动机理论”或“动机哲学(或‘瑜伽’)”的名义发展了这一思想。这是一个结构丰富性令人着迷的理论,其中大部分内容仍然是猜想性的\footnote{我在这些年里向任何愿意倾听的人解释了我的动机愿景,却没有费心将其以书面形式发表(因为还有其他为所有人服务的任务)。这后来让我的某些学生能够更轻松地剽窃,而我的所有老朋友都对此视而不见,他们对情况了如指掌。(见下文脚注。)}。

我在《收获与播种》中多次谈到这种“动机瑜伽”,它对我尤为重要。这里不再赘述我在其他地方对此的论述。我只想说,“标准猜想”最自然地源于动机瑜伽的世界。同时,它们为动机概念的某种可能构造提供了一种途径。

这些猜想在我看来,至今仍然是代数几何中最根本的两个问题之一。无论是这个问题,还是另一个同样关键的问题(即所谓的“奇点解消”问题),目前都尚未解决。然而,尽管第二个问题在今天和一百年前一样,被视为一个 prestigious 且令人生畏的问题,我有幸提出的这个问题却被时尚的专断法令归类为格罗滕迪克式的可爱胡闹(在我离开数学舞台后的几年里,动机主题本身也是如此\footnote{事实上,这一主题在1982年(晶体主题被发掘一年后)以其原名被重新发掘(尽管形式狭隘,仅限于基域特征为零的情况),而工匠的名字却未被提及。这是众多例子中的一个,说明在我离开后,某些概念或主题被埋葬为格罗滕迪克的幻想,却在接下来的十到十五年间被我的某些学生一一发掘,带着谦逊的骄傲(无需赘言)且未提及工匠……})。但又一次,我跑题了……