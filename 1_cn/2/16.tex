\section{动机——或心中的心}

拓扑斯(topos)的主题源于概形(schémas)的主题,就在概形出现的那一年——但在广度上,它远远超越了母题。正是拓扑斯的主题,而非概形的主题,成为了那个“床”或“深河”,几何与代数、拓扑与算术、数理逻辑与范畴论、连续世界与“离散”或“断续”结构的世界在此交融。如果说概形的主题是新几何的心脏,那么拓扑斯的主题则是其外壳或居所。它是我所构想的最广阔的主题,旨在通过一种富有几何共鸣的语言,精细地捕捉数学宇宙中来自不同区域的、相距最远的情境之间的共同“本质”。

然而,拓扑斯主题的命运远不及概形主题。我在《收获与播种》(Récoltes et Semailles)中多次谈及此事,此处不宜赘述这一概念所经历的奇异波折。尽管如此,新几何的两大主题却源自拓扑斯主题,即两个互补的“上同调理论”,它们都是为了逼近韦伊猜想(conjectures de Weil)而设计的:平展(étale)主题(或“$\ell$-进”主题)和晶体(cristallin)主题。前者在我手中具体化为$\ell$-进上同调工具,如今已成为本世纪最强大的数学工具之一。而晶体主题,在我离开后几乎被埋没,最终在1981年6月因需求压力而被重新发掘,登上舞台,并以一个假名出现,其背后的故事甚至比拓扑斯主题更加离奇。

正如预期,$\ell$-进上同调工具成为证明韦伊猜想的关键工具。我自己证明了其中的大部分,而最后一步则由我最杰出的“上同调学家”学生皮埃尔·德利涅(Pierre Deligne)在我离开三年后以大师般的手法完成。

此外,我在1968年左右提出了韦伊猜想的一个更强且更“几何”的版本。这些猜想仍然“沾染”(如果可以这么说!)着一种看似不可约化的“算术”特征,尽管其精神正是通过“几何”(或“连续”)来表达和捕捉“算术”(或“离散”)\footnote{(针对数学家)韦伊猜想依赖于“算术”性质的假设,特别是所考虑的簇必须定义在有限域上。从上同调形式主义的角度来看,这导致与这种情境相关的弗罗贝尼乌斯自同态(Frobenius endomorphism)被赋予特殊地位。在我的方法中,关键性质(如“广义指标定理”类型)涉及任意代数对应,并且不对预先给定的基域做任何算术性质的假设。}。从这个意义上说,我所提出的猜想版本比韦伊本人的版本更“忠实”于“韦伊哲学”——这种未书写且鲜少言明的哲学,或许是过去四十年几何学非凡发展的主要潜在动力\footnote{然而,在我1970年离开后,出现了一种明显的反动趋势,导致了相对的停滞状态,我在《收获与播种》中多次提及此事。}。我的重述本质上是从经典的霍奇理论(Hodge theory)中提取出一种“精华”,使其在“抽象”代数簇的框架中仍然有效,而霍奇理论原本适用于“普通”代数簇\footnote{“普通”在这里意味着“定义在复数域上”。霍奇理论(又称“调和积分理论”)是复数代数簇背景下最强大的上同调理论。}。我将这一全新的、完全几何化的版本称为“标准猜想”(对于代数循环)。

在我看来,这是在发展$\ell$-进上同调工具之后,迈向这些猜想的新一步。但与此同时,更重要的是,这也是通向我认为自己在数学中引入的最深刻主题的可能途径之一\footnote{这是最深刻的主题,至少在我作为数学家的“公开”活动期间(1950年至1969年,即我离开数学舞台之前)是如此。我认为从1977年开始发展的阿贝尔几何(anabélienne)和伽罗瓦-泰希米勒理论(Galois-Teichmüller theory)主题具有相当的深度。}:动机(motifs)主题(其本身源自“$\ell$-进上同调主题”)。这一主题如同心脏或灵魂,是概形主题中最隐秘、最难以窥见的部分,而概形主题本身又是新视野的核心。标准猜想中提取的几个关键现象\footnote{(针对代数几何学家读者)这些猜想可能需要重新表述。更详细的评论参见《工地巡礼》(ReS IV 注释第178号,第1215-1216页)以及《信念与知识》(ReS III,注释第162号)中的脚注。}可以被视为动机主题的终极精华,如同这一微妙主题的“生命气息”,是新几何“心中的心”。

以下是大致的内容。我们已经看到,对于一个给定的素数$p$,构建“特征$p$代数簇”的“上同调理论”的重要性(特别是对于韦伊猜想)。而著名的“$\ell$-进上同调工具”正好提供了这样一种理论,甚至提供了无限多种不同的上同调理论,即每一种都与不同于$p$的素数$\ell$相关联。然而,这里显然还缺少一种理论,即对应于$\ell$等于$p$的情况。为了填补这一空白,我特意构想了一种新的上同调理论(即前面提到的“晶体上同调”)。此外,在$p$为无穷大的重要情况下,我们还拥有另外三种上同调理论\footnote{(针对数学家读者)这些理论分别对应于贝蒂上同调(Betti cohomology,通过将基域嵌入复数域而定义的超越方法)、霍奇上同调(Hodge cohomology,由塞尔(Serre)定义)和德·拉姆上同调(De Rham cohomology,由我定义),后两者早在五十年代就已出现(而贝蒂上同调则源自上个世纪)。}——并且没有任何证据表明我们不会在未来引入更多具有类似形式性质的新上同调理论。与普通拓扑学不同,我们在这里面对的是令人困惑的多种上同调理论。我们有一种非常清晰的直觉,即在某种起初相当模糊的意义上,所有这些理论都“殊途同归”,它们“给出了相同的结果”\footnote{(针对数学家读者)例如,如果$f$是代数簇$X$的一个自同态,诱导出上同调空间$H^{i}(X)$的一个自同态,那么后者的“特征多项式”应该是整系数的,且不依赖于所选的特定上同调理论(例如:$\ell$-进理论,$\ell$可变)。对于一般的代数对应,当$X$被假设为紧致光滑时,情况也是如此。可悲的是(这也反映了自$p>0$特征的代数簇上同调理论在我离开后的可悲状态),这一事实至今仍未得到证明,甚至在$X$是光滑射影曲面且$i=2$的特殊情况下也是如此。事实上,据我所知,在我离开后,还没有人愿意关注这一关键问题,它典型地属于那些从属于标准猜想的问题。时尚的裁决是,唯一值得关注的自同态是弗罗贝尼乌斯自同态(Deligne通过现有手段单独处理了它……)。}。正是为了表达这种不同上同调理论之间“亲缘关系”的直觉,我提出了与代数簇相关联的“动机”(motif)概念。通过这一术语,我试图暗示它是所有这些不同上同调不变量背后的“共同动机”(或“共同理由”),这些不变量是通过所有可能的上同调理论得到的。这些不同的上同调理论就像是对同一“基本动机”(称为“上同调动机理论”)的不同主题展开,每一种都以自己的“节奏”、“调式”和“模式”(“大调”或“小调”)进行。因此,与代数簇相关联的动机将构成“终极”上同调不变量,所有其他不变量(与不同上同调理论相关联)都将从中推导出来,就像不同的“音乐化身”或“实现”。所有关于簇“上同调”的本质性质都已经“绑定”(或“听到”)在相应的动机上,因此特定上同调不变量(如$\ell$-进或晶体不变量)上的熟悉性质和结构,仅仅是动机内部性质和结构的忠实反映\footnote{(针对数学家读者)另一种看待动机范畴的方式是将其视为一种“阿贝尔范畴包络”,包络了在域$k$上有限型分离概形的范畴。与这样一个概形$X$相关联的动机(或“$X$的动机上同调”,我记为$H_{\text {mot}}^{*}(X)$)因此表现为$X$的一种“阿贝尔化化身”。这里的关键在于,正如代数簇$X$可以“连续变化”(其同构类依赖于“连续参数”或“模”),与$X$相关联的动机,或更一般地,一个“可变”动机,也可以连续变化。这是动机上同调的一个方面,与所有经典上同调不变量(包括$\ell$-进不变量)形成鲜明对比,唯一的例外是复数代数簇的霍奇上同调。

这让我们对“动机上同调”作为更精细的不变量有了一个概念,它更紧密地捕捉了簇$X$的“算术形式”(尽管这一表达有些冒险),而非纯粹的拓扑不变量。在我的动机观中,动机构成了一种非常隐秘且微妙的“纽带”,将簇$X$的代数几何性质与其动机所体现的“算术”性质联系起来。后者可以被视为一种“几何”性质的对象,但其精神中的“算术”性质却几乎被“赤裸裸地”揭示出来。

因此,动机在我看来是迄今为止与代数簇相关联的最深刻的“形式不变量”,除了其“动机基本群”$\pi_1$和另一个最近被我视为“动机同伦型”的“阴影”的不变量(我在《工地巡礼——或工具与视野》(ReS IV,第178号,参见第5章动机)中顺便提到,特别是第1214页)。在我看来,后者应该是“算术形式”(或“动机形式”)这一难以捉摸的直觉的最完美体现。}。

这就是用非技术的音乐隐喻表达的、一个既简单又微妙且大胆的想法的精髓。我在1963年至1969年间,以“动机理论”或“动机哲学(或‘瑜伽’)”的名义,在完成更紧迫的基础任务之余,发展了这一理论。这是一个结构丰富性令人着迷的理论,其中大部分内容仍然是猜想性的\footnote{我在这些年里向任何愿意倾听的人解释了我的动机观,但没有费心将其以书面形式发表(因为还有其他任务需要为所有人服务)。这后来使得某些学生能够更轻松地剽窃,而我的所有老朋友都心知肚明,且对此视而不见。(参见随后的脚注。)}。

我在《收获与播种》中多次谈及这一“动机瑜伽”,它对我尤为重要。此处不宜重复我在其他地方对此的论述。我只想说,“标准猜想”最自然地源自这一动机瑜伽。同时,它们也为动机概念的某种形式化构建提供了一种途径。

这些猜想在我看来,至今仍然是代数几何中最根本的两个问题之一。无论是这个问题,还是另一个同样关键的问题(即所谓的“奇点解消”问题),至今仍未解决。然而,尽管后一个问题在今天和一百年前一样,被视为一个声望卓著且令人敬畏的问题,我所提出的问题却被时尚的专断裁决(在我离开数学舞台后的几年里,动机主题本身也是如此\footnote{事实上,这一主题在1982年(晶体主题被发掘一年后)以原名(且在基域特征为零的狭窄情况下)被重新发掘,而工人的名字却未被提及。这只是众多例子中的一个,这些概念或主题在我离开后被埋葬为格罗滕迪克的幻影,随后在十到十五年间被我的某些学生逐一发掘,带着谦虚的骄傲(且无需再提工人的名字……)})归类为格罗滕迪克的善意胡闹。但再一次,我提前了……