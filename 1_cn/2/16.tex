\section{“模体”——或心中的核心}

“拓扑斯”(topos,topos)主题源于“概形”(schémas,schemes)主题,二者同年问世,但其广度远远超越母题。若说“概形”主题是新几何(géométrie nouvelle,new geometry)的核心,那么“拓扑斯”主题则是其外壳,或居所。它是我构想出的最广阔之物,以一种富含几何共鸣的语言,精妙捕捉来自数学浩瀚宇宙不同区域、看似遥远的情境所共有的“本质”(essence,essence)。在这深邃的“床”或“河流”中,几何与代数、拓扑学(topologie,topology)与算术(arithmétique,arithmetic)、数学逻辑(logique mathématique,mathematical logic)与范畴论(théorie des catégories,category theory)、连续世界与“离散”或“不连续”结构世界得以联姻。

然而,“拓扑斯”主题远未如“概形”主题那般广受青睐。我在《收获与播种》(Récoltes et Semailles,Harvests and Sowings)中多次提及此事,此处不宜细述这一概念遭遇的奇特波折。即便如此,新几何的两大主导主题——互补的“上同调理论”(théories cohomologiques,cohomological theories),皆为韦伊(Weil,Weil)猜想而设计——仍源自“拓扑斯”:即“埃塔尔”(étale,或“$\ell$-进”,$\ell$-adic)主题与“晶体”(cristallin,crystalline)主题。前者在我手中化为“$\ell$-进上同调工具”(outil cohomologique $\ell$-adique,$\ell$-adic cohomological tool),现已成为本世纪最有力的数学工具之一。至于“晶体”主题,在我离去后近乎隐秘存在,最终于 1981 年 6 月,在需求压力下被挖掘,重现于聚光灯下,却以借名登场,其周遭情境较“拓扑斯”更显诡异。

“$\ell$-进上同调工具”如预期,成为确立韦伊猜想的关键。我亲手证明了其中不少,最后一步由我最杰出的“上同调”学生皮埃尔·德利涅(Pierre Deligne,Pierre Deligne)在我离去三年后,以卓越技艺完成。

约在 1968 年,我提出了韦伊猜想的一个更强、更具“几何”意味的版本。原猜想似仍带有无法消减的“算术”痕迹(若可如此说!),而其精神恰在于通过“几何”(或“连续”)中介表达并捕捉“算术”(或“离散”)\footnote{(致数学读者)韦伊猜想受限于“算术”性质假设,因所涉簇需定义于有限域上。在上同调形式体系中,这使弗罗贝尼乌斯(Frobenius,Frobenius)自同态占据特殊地位。我的进路中,关键性质(如“广义指数定理”类型)涉及任意代数对应,无需预设基域的算术假设。}。在此意义上,我提炼的版本较韦伊本人更忠于“韦伊哲学”——那未成文且罕被言明的哲学,或许是过去四十年几何惊人发展的主要隐秘动力 \footnote{然而,我 1970 年离去后,出现明显反弹,导致相对停滞,我在《收获与播种》中多次提及此况。}。我的重述主要在于,从经典“霍奇理论”(théorie de Hodge,Hodge theory)——适用于“普通”代数簇 \footnote{“普通”指“定义于复数域上”。霍奇理论(即“调和积分理论”)是复代数簇背景下最有力的上同调理论。}——中提炼出适用于“抽象”代数簇的“精髓”(quintessence,quintessence)。我称此全然几何的新版本为“标准猜想”(conjectures standard,standard conjectures,针对代数循环)。

在我看来,这是在发展“$\ell$-进上同调工具”后,向猜想迈出的新步。但更重要的是,它也是通向我认为最深刻数学主题——“模体”(motifs,motives,源于“$\ell$-进上同调”主题)——的可能进路之一 \footnote{此为我 1950 至 1969 年“公开”数学活动中——即至退出数学界前——最深刻主题。我视 1977 年起发展的阿贝尔代数几何(géométrie algébrique anabélienne,anabelian algebraic geometry)与伽罗瓦-泰希米勒理论(théorie de Galois-Teichmüller,Galois-Teichmüller theory)为同等深度。}。若“概形”主题是新愿景的核心,“模体”则是其心或魂,最隐秘、最难窥见的部分。“标准猜想”中提炼的几个关键现象 \footnote{(致代数几何读者)或需重述这些猜想。详见《工地巡览》(Le tour des chantiers,The Tour of the Workyards)(《收获与播种》第四卷,注释 178,页 1215-1216)及《信念与知识》(Conviction et connaissance,Conviction and Knowledge)(《收获与播种》第三卷,注释 162,页 769)的脚注。},可视为“模体”主题的终极精髓,是这至微主题的“生命气息”,新几何“心中的核心”。

大体而言,情况如下。为给定素数 $p$,构造“特征 $p$ 代数簇”(variétés algébriques de caractéristique $p$,algebraic varieties of characteristic $p$)的“上同调理论”(théories cohomologiques,cohomological theories)至关重要(尤其针对韦伊猜想)。著名的“$\ell$-进上同调工具”恰提供此类理论,甚至无穷多种——即对每个不同于 $p$ 的素数 $\ell$ 各一。显然,仍缺一种理论,对应 $\ell = p$ 的情形。为此,我特意设想了另一“晶体上同调”(cohomologie cristalline,crystalline cohomology)理论(前文已提及)。此外,当 $p$ 为无穷时,还有三种上同调理论可用 \footnote{(致数学读者)分别对应贝蒂上同调(cohomologie de Betti,Betti cohomology,通过基域嵌入复数域以超限方式定义)、霍奇上同调(cohomologie de Hodge,Hodge cohomology,由塞尔(Serre,Serre)定义)及德拉姆上同调(cohomologie de De Rham,De Rham cohomology,由我定义),后两者始于五十年代(贝蒂上同调则始于上世纪)。}——且不排除迟早需引入具类似形式性质的新理论。与普通拓扑学不同,此处面对的是令人困惑的多种上同调理论。直觉强烈暗示,这些理论在某种尚模糊的意义上“殊途同归”,“结果一致” \footnote{(致数学读者)例如,若 $f$ 为代数簇 $X$ 的自同态,诱导上同调空间 $H^{i}(X)$ 的自同态,其“特征多项式”应具整数系数,不依赖特定上同调理论(如 $\ell$-进,$\ell$ 可变)。对一般代数对应亦然,当 $X$ 为真且光滑时。可悲的是(这也反映了我离去后特征 $p>0$ 代数簇上同调理论的荒废),至今未获证明,即便在 $X$ 为射影光滑曲面且 $i=2$ 的特例中。据我所知,我离去后无人关注这一标准猜想相关的关键问题。时尚裁定,唯一值得关注的自同态是弗罗贝尼乌斯自同态(德利涅(Deligne,Deligne)曾以有限手段单独处理)。}。为表达不同上同调理论间的“亲缘”直觉,我提炼出与代数簇关联的“模体”(motif,motive)概念。此术语意在暗示,它是借助一切可能上同调理论所得多种上同调不变量的“共同动机”(motif commun,common motif)或“共同理由”(raison commune,common reason)。这些上同调理论如同基于同一“基础模体”(即“模体上同调理论”(théorie cohomologique motivique,motivic cohomology theory))的不同主题演绎,各具“节奏”、“调性”与“调式”(“大调”或“小调”),而此基础模体同时是所有这些“主题化身”(即可能的上同调理论)中最基本、最精微者。于是,代数簇关联的模体构成其“终极”“卓越”的上同调不变量,其他所有不变量(依不同上同调理论而定)皆由此衍生,如同多种“音乐化身”或“实现”。簇上同调的一切本质属性,皆可于对应模体上“读取”(或“聆听”),故特定上同调不变量(如 $\ell$-进或晶体)的熟悉属性与结构,仅是模体内在属性与结构的忠实映照 \footnote{(致数学读者)另一视角是将域 $k$ 上的模体范畴视为 $k$ 上有限型分离概形范畴的“包络阿贝尔范畴”。关联于此类概形 $X$ 的模体(或“模体上同调”(cohomologie motivique,motivic cohomology),记为 $H_{\text{mot}}^{*}(X)$)如同 $X$ 的“阿贝尔化身”。关键在于,如代数簇 $X$ 可“连续变化”(其同构类依连续“参数”或“模”而变),其关联模体(或更广义的“可变模体”)亦然。这与除复代数簇霍奇上同调外的所有经典上同调不变量形成鲜明对比。

这表明“模体上同调”捕捉 $X$ 的“算术形式”(forme arithmétique,arithmetic form,此表达虽冒险)远比纯拓扑不变量精妙。在我对模体的愿景中,它们是连接 $X$ 代数-几何属性与“算术”属性的隐秘而精巧“纽带”,后者由模体体现。模体本质为“几何”对象,却将依附于几何的“算术”属性“剥露无遗”。

故模体是我迄今为代数簇关联的最深刻“形式不变量”,除其“模体基本群” $\pi_1$ 及我近期视为“模体同伦类型”(type d’homotopie motivique,motivic homotopy type)“影子”的另一不变量外(后者尚待描述,我在《工地巡览——或工具与愿景》(Le tour des chantiers - ou outils et vision,The Tour of the Workyards - or Tools and Vision)(《收获与播种》第四卷,注释 178,第五章“模体”)中提及,尤见页 1214)。后者似为代数簇“算术形式”(或“模体形式”)飘忽直觉的最完美化身。}。

这以非技术性的音乐隐喻表达了一念的精髓——仍具童稚单纯,却精妙而大胆。我在 1963-1969 年间,在更紧迫的基础任务之余,发展此念,称其为“模体理论”(théorie des motifs,theory of motives)或“模体哲学(或瑜伽)”(philosophie (ou yoga) des motifs,philosophy (or yoga) of motives)。此理论结构丰饶迷人,大部仍属猜想 \footnote{那些年,我向愿闻者阐述模体愿景,未费笔墨出版(忙于他务服务众人)。这后使某些学生在我老友熟知且温柔注视下更自在地“掠取”(见后脚注)。}。

我在《收获与播种》中多次谈及这“模体瑜伽”,它尤为我心所系。此处不赘述他处之言。只需说,“标准猜想”自此瑜伽自然流出,同时为模体概念的一种可能构造提供进路。

这些猜想在我看来——至今如此——是代数几何两大最根本问题之一。另一同样关键问题(“奇点消解”(résolution des singularités,resolution of singularities))与此均未解决。若后者百年如一日被视为崇高而艰巨,前者——我有幸提炼者——却在我退出数学界后,被时尚武断裁定(连同“模体”主题 \footnote{实则此主题于 1982 年(晶体主题后一年)以原名复出(仅限特征零基域的狭义形式),未提工匠之名。此为我离去后多主题被视为“格罗滕迪克奇想”埋葬、后由学生于十至十五年间逐一挖掘的例证,他们谦逊自豪,未提工匠(无需赘言)。})为“可亲的格罗滕迪克骗局”(fumisterie grothendieckienne,Grothendieckian hoax)。然我又提前言及了……