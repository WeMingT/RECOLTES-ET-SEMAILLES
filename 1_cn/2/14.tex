\section{“空间概念的蜕变”——或气息与信念}

“概形”(schéma,scheme)概念是对“代数簇”(variété algébrique,algebraic variety)概念的极大拓展,因此彻底革新了我前辈传承的代数几何(géométrie algébrique,algebraic geometry)。“拓扑斯”(topos,topos)概念则构成了“空间”(espace,space)概念意想不到的延伸,更确切地说,是一场蜕变。由此,它带来了对拓扑学(topologie,topology)乃至更广义几何(géométrie,geometry)类似革新的希望。事实上,它已在新几何(géométrie nouvelle,new geometry)的兴盛中扮演了关键角色(尤其通过其衍生的 $\ell$-进($\ell$-adique,$\ell$-adic)与晶体(cristallin,crystalline)上同调主题,并通过这些主题,助力韦伊(Weil,Weil)猜想的证明)。如同其年长且近乎孪生的姊妹,它具备一切丰饶推广所需的两种互补特质,具体如下。

首先,新概念不过于宽泛。这意味着,在这些新“空间”(更愿称之为“拓扑斯”,以免刺痛敏感的耳朵 \footnote{“拓扑斯”(topos,topos)之名(与“拓扑学”(topologie,topology)或“拓扑”(topologique,topological)关联)被选中,以暗示其为拓扑直觉适用的“卓越对象”。通过此名唤起的丰富意象云团,应视其或多或少等同于“空间”(拓扑空间)(espace topologique,topological space),只是更强调概念的“拓扑”特异性。(因此,有“向量空间”(espaces vectoriels,vector spaces),但迄今无“向量拓扑斯”!)保留这两个术语并用,各具特质,实属必要。})中,那些对昔日经典空间熟稔的最核心“几何”直觉与构造 \footnote{这些“构造”中,尤包括所有熟悉的“拓扑不变量”(invariants topologiques,topological invariants),如上同调不变量(invariants cohomologiques,cohomological invariants)。为赋予后者适用于任何“拓扑斯”的意义,我已在早前提及的文章(《东北》(Tohoku),1955)中做了充分准备。},可或多或少自然地移植。换言之,新对象继承了旧式对象独有的丰富意象与联想、概念及至少部分技术。

其次,新概念又足够宽广,能包容大量此前未被视为具有“拓扑-几何”直觉的情境——正是过去专属于普通拓扑空间(且有其理由)的直觉。

在此,对于韦伊猜想的关键在于,新概念确实足够宽广,使我们能为每个“概形”关联一个这样的“广义空间”或“拓扑斯”(称为该概形的“埃塔尔拓扑斯”(topos étale,étale topos))。此拓扑斯的某些“上同调不变量”(invariants cohomologiques,cohomological invariants,极为“简单”!)似乎颇有望提供“所需之物”,以完整阐释这些猜想,甚至(谁知道呢!)提供证明的手段。

在我撰写的这些篇章中,作为数学家生涯中首次,我得以悠然回溯(哪怕仅对自己)我数学作品中的主导主题与伟大指导思想。这让我更清晰地评估每个主题及其体现的“观点”在统一它们的宏大几何愿景中的位置与意义。正是在此工作中,新几何那初萌却有力的兴盛中两大神经中枢般的创新思想——“概形”思想与“拓扑斯”思想——得以光芒毕现。

如今在我眼中,这两者中,“拓扑斯”思想更显深刻。若五十年代末,我未曾卷起袖子,日复一日顽强钻研,历经十二载春秋,打造出精妙而强大的“概形工具”——我仍觉几乎不可思议的是,在随后十至二十年间,竟无他人最终按捺不住引入这显然势在必行的概念(即便违背其意愿),并至少草草搭建几座老旧的“预制棚屋”,若非我亲手逐石垒砌的宽敞舒适居所。然而,在过去三十年的数学舞台上,我看不到他人具备那份天真或纯真,能代我迈出这至关重要的一步,引入如此童稚的“拓扑斯”思想(或哪怕仅“位点”(sites,sites)概念)。即便此思想已慷慨奉上,携其看似羞涩的希望——我亦看不到昔日朋友或学生中,有谁具备那份气息,更遑论信念,去将这谦卑思想 \footnote{(致数学读者)我所谓“将这谦卑思想推向终点”,指的是以埃塔尔上同调(cohomologie étale,étale cohomology)作为韦伊猜想的进路。受此启发,我于 1958 年发现“位点”(site,site)概念,而此概念(或极相近的“拓扑斯”概念)及埃塔尔上同调形式体系,在 1962 至 1966 年间,在我的推动下(与几位将在后文提及的合作者协助下)得以发展。

我提及的“气息”与“信念”,是“非技术性”的品质,在此却显得至关重要。在另一层面,我或可补充所谓“上同调嗅觉”(flair cohomologique,cohomological flair),即我在构建上同调理论时培养的那种直觉。我曾以为已将其传授给我的上同调学生。十七年后,退出数学界回望,我发现无人保留此嗅觉。}(看似微不足道,而目标遥不可及……)从最初的蹒跚起步,带向“埃塔尔上同调掌握”的成熟境界——这思想最终在我手中,在随后岁月里得以化身。