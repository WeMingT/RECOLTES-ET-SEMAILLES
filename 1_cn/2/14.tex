
\section{空间概念的嬗变——或气息与信仰}

图式的概念极大地扩展了“代数簇”的概念,并因此彻底革新了由我前辈传承下来的代数几何。而拓扑斯的概念则构成了一个意想不到的扩展,更确切地说,是空间概念的一次蜕变。由此,它预示着拓扑学乃至更广泛的几何学将迎来类似的革新。事实上,它已经在新型几何(特别是通过由此衍生的$\ell$-adic和晶体上同调主题,以及通过这些主题在Weil猜想的证明中)的兴起中发挥了关键作用。如同其长姐(近乎双胞胎)一般,它具备所有富有成效的推广所必需的两个互补特性,如下所述。

首先,新概念并不过于宽泛,这意味着在新的“空间”(更确切地称为“拓扑斯”,以免冒犯敏感的耳朵\footnote{“拓扑斯”这一名称的选择(与“拓扑学”或“拓扑的”相关联)旨在暗示它是拓扑直觉应用的“卓越对象”。通过这一名称唤起的丰富心理意象,应将其视为与“空间”(拓扑的)一词大致相当,只是更加强调了概念的“拓扑”特性。(因此,存在“向量空间”,但迄今为止没有“向量拓扑斯”!)有必要同时保留这两个表达,各自保持其独特性。})中,那些对于昔日经典空间最为基本的“几何”直觉和构造\footnote{在这些“构造”中,尤其包括所有熟悉的“拓扑不变量”,包括上同调不变量。对于后者,我在已引用的文章(“Tohoku”1955)中已做了必要的工作,以便能够为任何“拓扑斯”赋予其意义。},可以或多或少明显地移植过来。换言之,新对象拥有了全部丰富的心理意象与联想、概念及至少部分技术,这些以往仅限于旧式对象。

其次,新概念同时足够广泛,能够涵盖大量此前不被认为能引发“拓扑-几何”直觉的情境——正是那些过去仅保留给普通拓扑空间的直觉(原因显而易见……)。

从Weil猜想的角度来看,关键在于新概念确实足够广泛,使我们能够为每个“图式”关联一个这样的“广义空间”或“拓扑斯”(称为该图式的“étale拓扑斯”)。该拓扑斯的某些“上同调不变量”(尽管看似简单!)似乎极有可能提供“所需之物”,以充分理解这些猜想,并(谁知道呢!)或许提供证明它们的手段。

正是在我撰写这些文字的过程中,作为数学家生涯中的首次,我得以悠闲地回顾(哪怕只是对自己)我数学作品中的全部主导主题和重大指导思想。这使我更深刻地认识到每个主题及其所体现的“观点”在统一并孕育它们的宏大几何视野中的地位与影响。正是通过这项工作,两个在新型几何强劲崛起中至关重要的创新理念——图式与拓扑斯——得以清晰显现。

这是两个想法中的第二个,即拓扑斯(topos)的概念,现在在我看来是两者中更为深刻的一个。如果在五十年代末期,我没有卷起袖子,日复一日地坚持开发一个精致而强大的“图解工具”——尽管这似乎几乎难以想象,但在随后的十年或二十年里,其他人最终可能也会不可避免地引入这一显然必要的概念(即使他们自己并不情愿),并至少搭建起一些简陋的“预制”建筑,而不是像我那样,用心一块石头一块石头地搭建起宽敞舒适的住所。然而,在过去的三十年里,我在数学界没有看到其他人能有这种天真或纯真,去(代替我)迈出这关键的一步,引入如此孩子气的拓扑斯概念(或者仅仅是“站点”的概念)。即使假设这个想法已经被慷慨地提供,并且伴随着它似乎蕴含的微弱希望——我也看不到其他人,无论是我的老朋友还是学生,有那份气魄,尤其是信念,去完成这个谦卑的想法\footnote{(针对数学读者。)当我提到“完成这个谦卑的想法”时,指的是将étale上同调作为通向Weil猜想的方法。正是受到这一想法的启发,我在1958年发现了站点的概念,随后这一概念(或与之非常接近的拓扑斯概念)以及étale上同调形式体系,在1962年至1966年间在我的推动下得以发展(并得到了一些合作者的协助,这些将在适当的地方提及)。

当我提到“气魄”和“信念”时,指的是那些“非技术性”的品质,在我看来,这些正是至关重要的品质。在另一个层面上,我还可以加上我称之为“上同调逃避”的东西,即在我心中发展起来的那种逃避,用于构建上同调理论。我曾以为我已经将其传达给我的上同调学生。然而,在我离开数学界十七年后,我发现它并未在他们中的任何一人身上保留下来。}(表面上看似微不足道,而目标却显得遥不可及……):从最初的蹒跚起步,到“掌握étale上同调”的完全成熟,它最终在我手中得以体现,在随后的岁月中。