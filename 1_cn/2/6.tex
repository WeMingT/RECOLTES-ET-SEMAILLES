\section{观点与视野}

此刻,我回过头来谈谈我自身及我的作品。

若我在数学家的艺术中有所卓越,与其说是因娴熟与坚韧,解决了前人遗留的问题,不如说是因我内在的一种自然倾向,驱使我看见无人察觉却显然关键的问题,或挖掘出缺失的“恰当概念”(往往在这些新概念出现前,无人意识到其缺失),以及无人想到过的“恰当命题”。通常,这些概念与命题契合得如此完美,我心中毫无疑问它们是正确的(至多需稍作调整)——若非为发表而进行的“逐件工作”,我常就此止步,不再费时完善证明。因为一旦命题及其语境被清晰洞察,证明往往仅剩“技艺”之事,甚至近乎例行公事。吸引注意的事物无穷无尽,不可能逐一穷尽其召唤!即便如此,在我已撰写并发表的作品中,经严谨证明的命题与定理数以千计。我相信,除极少数例外,它们皆已融入数学界共有遗产,成为普遍认可的“已知”并被广泛运用。

然而,比起发现新问题、新概念、新命题,我的独特天赋更倾向于探寻丰饶的观点,这些观点不断引领我引入并或多或少发展全新的主题。这,我认为,是我对当代数学最根本的贡献。实话说,我刚提及的无数问题、概念、命题,唯有在这种“观点”的光芒下才具意义——更准确地说,它们从中自发诞生,带着显而易见的力度;恰如黑夜中乍现的光(即便微弱),似乎从虚空中唤出它所揭示的轮廓,或模糊或清晰。若无这道将它们聚为一束的光,十个、百个、千个问题、概念、命题不过是一堆杂乱无形的“心智小玩意”,彼此孤立——而非某整体的部分。这整体或许仍隐不可见,藏于夜的褶皱中,却已被清晰预感。

丰饶的观点揭示出这些无人感知的炽热问题,仿佛它们是包容并赋予其意义的同一整体的活的部分;它还揭示出(或许作为对这些问题的回应)那些如此自然却无人挖掘的概念,以及那些看似水到渠成的命题——只要激发它们的疑问和表述它们的概念尚未浮现,无人会冒险提出它们。比起数学中所谓的“关键定理”,丰饶的观点才是我们艺术中最为有力的发现工具\footnote{不仅在“我们的艺术”中如此,我认为在一切发现工作中皆然,至少在智识认知层面上如此。}——或者更确切地说,它们并非工具,而是研究者渴求探知数学事物本质的双眼。

因此,丰饶的观点即那“眼”,它既让我们发现,又让我们在多样性中辨识统一。这统一正是生命本身,是联结并赋予多物生机的气息。

然而,正如其名所示,“观点”本身始终是片面的。它揭示了风景或全景的一个面向,在众多同样有效、同样“真实”的面向中仅占其一。唯有多个互补的观点交汇于同一现实,我们的“眼”倍增,目光才能更深入事物认知。欲了解的现实愈丰富复杂,拥有多重视角便愈重要\footnote{每一观点皆催生表达其特性的独特语言。拥有多重“眼”或“观点”来理解情境,在数学中至少意味着掌握多种不同语言去围捕它。},以全面且精微地把握其全貌。

有时,多重视角汇聚于同一广阔风景,凭借我们体内那能透过多样性抓住“一”的能力,会孕育出一件新生事物;这事物超越每一局部视角,宛如生命体超越其肢体与器官。这新生事物,可称之为视野。视野统合已知的观点,赋予其形体,并揭示此前未见的其他视角,正如丰饶的观点使多样的问题、概念、命题显现并被理解为同一整体的部分。

换言之,视野之于其源起并统合的观点,恰如白昼明亮温暖的光之于太阳光谱的各色成分。广阔深邃的视野如源泉无尽,注定启发并照亮不仅那位初生其心并为之仆者,也照亮世代之人——他们或许(如他当年)为视野隐约示现的遥远边界所迷。
