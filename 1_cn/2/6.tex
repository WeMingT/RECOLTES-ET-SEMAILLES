
\section{观点与视野}

然而,我回归到自身及我的作品。

若我在数学家的艺术上有所卓越,与其说是凭借解决前人遗留问题的技巧与毅力,不如说是源于我内在的一种自然倾向,驱使我发现那些显然关键却无人察觉的问题,或是提炼出缺失的“正确概念”(往往在全新概念显现之前,无人意识到其缺失),以及构思出无人想到的“恰当表述”。很多时候,这些概念与表述搭配得如此完美,以至于我心中几乎确信它们是正确的(至多需稍作修饰)——因此,当面对的是为发表而进行的“片段工作”时,我常省去进一步深入,不再花时间完善证明,因为一旦表述及其背景清晰,证明往往不过是“技艺”问题,甚至可以说是例行公事。吸引注意力的事物无穷无尽,不可能逐一响应到底!尽管如此,在我撰写并发表的著作中,以正规形式证明的命题与定理数以千计,我相信可以说,除极少数例外,它们均已融入数学界公认的“已知”知识库,并在数学的各个角落广泛使用。

但比起发现新问题、新概念和新表述,我更倾向于寻找富有成效的视角,这些视角不断引导我引入并或多或少地发展全新的主题。这,在我看来,是我为当代数学贡献的最核心部分。事实上,上述无数的问题、概念、表述,唯有在这样一个“视角”的照耀下,对我而言才具有意义——更准确地说,它们由此视角自发而生,带着不言自明的力量;就如同黑夜中突然出现的一束光(即便是散射的),似乎从虚无中勾勒出那些或模糊或清晰的轮廓,瞬间展现在我们眼前。没有这束将它们汇聚于共同光束中的光,那十、百、千个问题、概念、表述,只会呈现为一堆杂乱无章、形态各异的“思维小玩意”,彼此孤立——而非作为一个整体的组成部分,这个整体虽可能仍隐匿于夜的褶皱之中,却已清晰可感。

富有成效的视角,是那种向我们揭示,如同同一整体中鲜活部分般,那些无人感知的紧迫问题,以及(或许作为对这些问题的回应)那些自然得令人惊讶却无人想到要提炼的概念,还有那些似乎水到渠成、在相关问题与概念尚未显现之前无人敢于提出的表述。在数学中,比所谓的“关键定理”更为重要的,是这些富有成效的视角,它们是我们这门艺术中最强大的发现工具——或者说,它们并非工具,而是研究者渴望了解数学事物本质的双眼,充满激情地探索着。

因此,富有成效的观点无异于那只“眼睛”,它既让我们发现,又让我们在发现的多重性中认识到统一。而这种统一正是生命本身,是连接并激活这些多样事物的气息。

然而,正如其名所示,一个“观点”本身仍是片面的。它向我们揭示了一个景观或全景的一个方面,在众多同样有效、同样“真实”的其他方面之中。正是在同一现实的不同互补观点相结合,我们的“眼睛”增多之时,目光才能更深入地洞察事物的本质。我们渴望了解的现实越是丰富复杂,拥有多个“眼睛”以全面而细致地把握它就显得愈发重要\footnote{任何观点都会催生一种表达它且独具特色的语言。拥有多个“眼睛”或“观点”来理解一个情境,也相当于(至少在数学上)掌握了多种不同的语言来界定它。}。

有时,一系列针对同一广阔景观的汇聚观点,凭借其内在力量,使我们能够透过多样把握统一,赋予新事物以形体;这新事物超越了每一个局部视角,正如一个生命体超越其每一个肢体和器官。这新事物,我们可以称之为一种愿景。愿景将已知的、体现它的观点统一起来,并揭示出我们之前未曾注意到的其他观点,正如富有成效的观点发现并理解众多问题、概念和新陈述为同一整体的一部分。

换言之,愿景与其看似源自并统一的各种观点之间的关系,犹如清澈温暖的日光与太阳光谱的不同组成部分之间的关系。一个广阔而深邃的愿景如同不竭的源泉,不仅旨在激励和启迪其诞生之日的主人——那位成为其仆人的个体——的工作,也旨在照亮被其揭示的遥远界限所吸引的世代的工作,或许他们(如同他本人一样)会被这些界限所深深吸引……