\section{“伟大思想”——或树木与森林}

我所谓的“多产”时期的数学活动,即以正式出版物为证的时期,从 1950 年到 1969 年,持续了二十年。而在从 1945 年(当时我 17 岁)到 1969 年(我快 42 岁)的二十五年间,我几乎将全部精力投入到数学研究中。毫无疑问,这是一种过度的投入。我为此付出了长期的精神停滞和逐渐“变厚”的代价,这在《收获与播种》(Récoltes et Semailles,Harvests and Sowings)的篇章中我将不止一次地提及。然而,在纯粹智力活动的有限领域内,通过对仅限于数学事物的世界的视野的绽放和成熟,这些年是极富创造力的。

在这段漫长的人生阶段中,我几乎将所有的时间和精力都奉献给了所谓的“具体工作”:细致入微的塑造、组装和磨合工作,这些工作是为了从头开始建造那些由内心的声音(或魔鬼……)命令我建造的房屋,根据它在我工作进行时逐渐向我透露的总设计师的计划。我乐于在这些工作中倾注爱意和细心,从事着石匠、泥瓦匠、木匠,有时甚至是管道工、细木工和木匠的各种“职业”任务。我很少有闲暇将那对所有人(正如后来所显示的……)都不可见、但对我而言却在日复一日、月复一月、年复一年中以梦游者般的确定性引导着我的手的总计划轮廓记录下来,哪怕只是粗略地记录在纸上 \footnote{“梦游者”这一形象的灵感来自库斯勒(Koestler,Koestler)那本杰出的书《梦游者》(Les Somnambules,The Sleepwalkers,Calmann-Lévy 出版社),该书呈现了“关于宇宙观念史的尝试”,从科学思想的起源到牛顿(Newton,Newton)时代。库斯勒在这一历史中强调的一个方面是,在我们对世界的认识中,从某一点到逻辑上(事后看来)似乎非常接近的另一点的道路,常常要经过最荒诞的曲折,这些曲折似乎在挑战健全的理性;然而,尽管有这些似乎会永远误导他们的无数曲折,寻求宇宙“钥匙”的人们却以“梦游者般的确定性”,几乎是无意中且常常没有意识到,偶然发现了他们远未预见的其他“钥匙”,而这些钥匙却被证明是“正确的”。

根据我周围的观察,在数学发现的层面上,这些惊人的曲折是某些杰出研究者的特征,但并非所有人都如此。这可能是因为在过去的两三个世纪里,自然科学的研究,尤其是数学研究,已经摆脱了与特定文化和时代相关的宗教或形而上学预设的束缚,这些预设曾是“科学”理解宇宙(无论好坏)发展的特别强大的障碍。然而,确实有一些最基本、最明显的数学思想和概念(如位移、群(groupe,group)、零(zéro,zero)、文字计算、空间中点的坐标、集合(ensemble,set)的概念或拓扑“形式”(forme topologique,topological form)的概念,更不用说负数和复数了),在出现之前花了数千年时间。这些都是根深蒂固的“障碍”的雄辩迹象,这些障碍深深植根于心灵之中,阻碍着全新思想的构想,即使在这些思想极其简单、似乎以证据的力量自然而然地强加于人时,情况也是如此,这种情况持续了几代人,甚至数千年……

回到我自己的工作,我的感觉是,在我的工作中,“失误”(可能比大多数同事都多)仅限于细节问题,通常很快就被我自己发现了。这些只是纯粹“局部”性质的“路途事故”,对关于所考察情况的基本直觉的有效性没有严重影响。相反,在思想和指导性的大直觉层面上,我的工作似乎没有出现任何“失误”,尽管这听起来难以置信。正是这种在每时每刻都能准确把握的确定性,即使不能把握一种方法的最终结果(这些结果通常隐藏在视线之外),至少也能把握最富有成效的方向,引导我直接走向本质事物——正是这种确定性让我想起了库斯勒的“梦游者”形象。}。必须说,我喜欢在工作中倾注爱意和细心,这项工作绝不是令我不悦的。此外,我的长辈们所教授和实践的数学表达方式(至少可以说)优先考虑了工作的技术方面,并不鼓励那些停留在“动机”上的“离题”,甚至不鼓励那些试图从迷雾中浮现出某种可能启发性的图像或愿景的尝试,这些图像或愿景,由于尚未体现在木材、石头或坚硬的水泥等有形的建筑中,更像是梦的碎片,而不是工匠专注而认真的工作。

在数量层面上,我在这些高产年份的工作主要体现在大约一万二千页的出版物中,这些出版物以文章、专著或研讨会的形式出现 \footnote{从 1960 年代起,其中一部分出版物是在同事(尤其是迪厄多内(J. Dieudonné,J. Dieudonné))和学生的合作下撰写的。},以及成百上千(如果不是数千)个新概念,这些概念已经进入共同的遗产,保留着我最初发现它们时赋予它们的名称 \footnote{这些概念中最重要的在《主题草图》(Esquisse Thématique,Thematic Sketch)及其附带的《历史评论》(Commentaire Histoire,Historical Commentary)中进行了回顾,这些内容将包含在《反思》(Réflexions,Reflections)的第四卷中。其中一些名称是由朋友或学生建议的,例如“光滑态射”(morphisme lisse,smooth morphism,由迪厄多内(J. Dieudonné,J. Dieudonné)提出)或在吉罗(Jean Giraud,Jean Giraud)的论文中发展的“位点(site,site)、层(champ,sheaf)、胚(gerbe,gerbe)、联系(lien,connection)”等术语。}。在数学史上,我相信我是将最多的新概念引入我们科学的人,同时也是因此而不得不为这些概念发明最多新名称的人,以便以尽可能微妙和启发性的方式表达它们。

当然,这些“数量”上的指示只能提供对我的作品的一种极其粗略的理解,忽略了真正构成其灵魂、生命和活力的东西。正如我刚才所写,我在数学中带来的最好的东西,是我首先能够瞥见,然后耐心地发掘并或多或少地发展的新“观点”。就像我刚才提到的概念一样,这些新观点,引入到多种多样的不同情境中,本身几乎是无数的。

然而,有些观点比其他观点更广阔,单凭它们就能在多种不同特定情境中激发和包含大量局部观点。这样的观点也可以恰当地称为“伟大思想”。凭借其固有的丰富性,这样的思想会催生出大量后代,这些后代都继承了它的丰富性,但其中大多数(如果不是全部)的影响范围都比母思想要小。

至于表达一个伟大思想,“说出”它,这通常几乎和它的构思以及在构思者心中缓慢孕育一样微妙——或者更准确地说,这种孕育和形成的艰苦工作,正是“表达”思想的工作:耐心地、日复一日地,从环绕它诞生的迷雾中解脱出来,逐渐赋予它有形的形式,在一个随着周、月、年的流逝而丰富、巩固和细化的画面中。简单地命名这个思想,用一些引人注目的公式或多或少技术性的关键词,可能只需要几行,甚至几页——但很少有人能够在不已经很好地了解它的情况下,听到这个“名字”并从中认出一个面孔。而当思想达到完全成熟时,也许一百页就足以表达它,令那个在其中诞生的工人完全满意——也可能一万页经过深思熟虑和权衡的文字也不足以 \footnote{在 1970 年离开数学舞台时,我关于概形(schéma,scheme)这一中心主题的全部出版物(其中许多是合作完成的)应该有大约一万页。然而,这只是我眼前看到的广泛计划中的一小部分,涉及概形。这个计划在我离开后被无限期地放弃了,尽管事实上,几乎所有已经发展和发表的内容都立即进入了共同的遗产,成为“众所周知”的概念和结果。

我在离开时完成的关于概形主题及其扩展和分支的计划部分,本身就代表了数学史上最庞大的基础工作之一,当然也是科学史上最庞大的之一。}。

无论在哪种情况下,在那些为了使之成为自己的而了解了最终呈现思想在全盛时期的著作的人中——就像在一片荒凉的土地上突然长出的一片宽广的林地——很可能有很多人会看到所有这些茁壮而苗条的树木,并利用它们(有人攀爬,有人从中获取梁和板,还有人用它们在壁炉中生火……),但很少有人会看到森林……