\section{“伟大的思想”——或树木与森林}

我数学活动的所谓“高产”时期,即以正式出版物为标志的时期,从1950年持续到1969年,共二十年。而从1945年(我十七岁时)到1969年(我即将四十二岁时)的二十五年间,我几乎将全部精力投入了数学研究。这种投入无疑是过度的。我为此付出了精神长期停滞的代价,经历了一种逐渐的“僵化”,我将在《收获与播种》的篇章中多次提及这一点。然而,在纯粹智力活动的有限领域内,通过一种仅限于数学世界的视野的萌发与成熟,那是一段充满强烈创造力的岁月。

在这段漫长的人生阶段中,我几乎将所有时间和精力都投入到了所谓的“零件工作”中:即那些精细的塑造、组装和调试工作,这些工作是为了建造一座座房屋,而一个内在的声音(或恶魔……)命令我按照一位工头的指示去建造,这位工头随着工作的进展不断向我传达指令。我被“手艺”任务所占据:时而充当石匠,时而充当泥瓦匠、木匠,甚至管道工、细木工和家具匠——我很少有时间将那个对所有人(正如后来显现的那样……)都不可见的总体规划记录下来,哪怕只是粗略地勾勒出来,而这个规划在日复一日、月复一月、年复一年中,以梦游者般的确定性指引着我的手\footnote{“梦游者”这一形象受到了科斯特勒(Koestler)的杰出著作《梦游者》(Calman Lévy出版社)的启发,该书呈现了“关于宇宙观念的历史随笔”,从科学思想的起源一直到牛顿。科斯特勒强调的一个历史现象是,我们对于世界的认识从某一点到另一点(逻辑上且回顾起来似乎很近)的路径,常常会经过一些令人震惊的迂回,这些迂回似乎违背了理性;然而,通过这些看似会永远迷失的千回百转,那些寻找宇宙“钥匙”的人们却以“梦游者般的确定性”,仿佛不经意间,甚至常常毫无察觉地,找到了他们远未预见的其他“钥匙”,而这些钥匙却最终被证明是“正确的”。

根据我在数学发现领域的观察,这些发现路径中的巨大迂回是一些杰出研究者的特征,但绝非所有人都是如此。这可能是因为过去两三个世纪以来,自然科学的研究,尤其是数学研究,已经摆脱了与特定文化和时代相关的宗教或形而上学的强制性预设,这些预设曾是阻碍对宇宙“科学”理解(无论好坏)发展的强大力量。然而,确实有一些最基本、最显而易见的数学思想和概念(如位移、群、零、文字计算、空间中点的坐标、集合的概念,或拓扑“形式”的概念,更不用说负数和复数)花费了数千年才得以出现。这些都是根深蒂固的心理“障碍”的显著标志,这种障碍阻碍了全新概念的诞生,即使这些概念简单如孩童般,且似乎以不言自明的力量自我显现,却仍然需要数代甚至数千年的时间才能被接受……

回到我自己的工作,我感觉其中的“失误”(可能比大多数同事更多)仅限于一些细节问题,通常很快被我自行发现。这些只是“局部”性质的“行进中的意外”,对所研究情境的基本直觉的有效性没有严重影响。然而,在思想和重大指导性直觉的层面上,我的作品似乎没有任何“失败”,尽管这听起来难以置信。正是这种从未出错的确定性,使我在每一刻都能把握住那些最富成果的方向,带领我直奔事物的本质——正是这种确定性让我想起了科斯特勒的“梦游者”形象。}。必须承认,这种我乐于投入的“零件工作”并未让我感到不快。此外,我的前辈们所教授和实践的数学表达方式,至少可以说,更注重工作的技术层面,而不鼓励那些可能停留在“动机”上的“离题”;甚至,那些试图从迷雾中召唤出某种或许具有启发性的形象或愿景的尝试,由于尚未具体化为木材、石头或纯粹坚硬水泥的构造,更像是梦的碎片,而非工匠专注而细致的工作。

从数量上看,我在这段高产时期的成果主要体现在约一万两千页的出版物中,包括文章、专著和研讨会记录\footnote{从20世纪60年代起,这些出版物中的一部分是与同事(尤其是J. Dieudonné)和学生合作完成的。},以及数百甚至数千个新概念,这些概念连同我为它们命名的名称一起进入了数学的共同遗产\footnote{其中最重要的概念在《主题纲要》及其附带的《历史评论》中进行了回顾,这些内容将被收录在《反思》第四卷中。一些名称是由朋友或学生建议的,例如“光滑态射”(J. Dieudonné)或“位、场、层、链”这一系列术语,这些术语在Jean Giraud的论文中得到了发展。}。在数学史上,我相信我是为这门科学引入最多新概念的人,同时也是为此创造了最多新名称的人,以便以尽可能细腻且富有启发性的方式表达这些概念。

这些“数量”上的指标当然只能对我的作品提供一个极为粗略的把握,而忽略了其真正的灵魂、生命与活力。正如我之前所写,我为数学带来的最宝贵的东西,是我首先预见并随后耐心提炼和发展的“新观点”。正如我提到的那些概念一样,这些新观点在众多不同情境中引入,其数量几乎不可计数。

然而,有些观点比其他观点更为广阔,它们本身就能激发并涵盖众多局部观点,适用于多种不同的特定情境。这样的观点也可以恰当地被称为“伟大的思想”。凭借其自身的丰富性,这样的思想会孕育出一群活跃的后代,这些后代继承了它的丰富性,但大多数(如果不是全部)的适用范围都比母思想更为有限。

至于表达一个伟大的思想,“说出”它,这通常几乎与其构思本身以及在构思者内心的缓慢孕育过程一样微妙——或者更准确地说,这种孕育和形成的艰辛过程正是“表达”思想的过程:即耐心地将其从诞生时的迷雾中剥离出来,逐渐赋予其具体形态,形成一幅随着周、月、年的推移而不断丰富、巩固和精细化的图景。简单地用一个引人注目的公式或一些或多或少技术性的关键词来命名这个思想,可能只需要几行甚至几页——但很少有人能在尚未深入了解它的情况下,听懂这个“名字”并从中辨认出一个面孔。而当这个思想完全成熟时,也许一百页就足以表达它,足以让孕育它的工匠感到满意——但也可能需要一万页经过精心打磨的文字,却仍然无法完全表达\footnote{在1970年离开数学舞台时,我关于“概形”这一核心主题的出版物(其中许多是与他人合作的)总计约一万页。然而,这仅代表了我所看到的关于概形的宏大计划中的一小部分。这一计划在我离开后被无限期搁置,尽管事实上,几乎所有已经开发和发布的内容都立即进入了数学共同遗产,成为被广泛使用的“众所周知”的概念和结果。
我在离开前完成的关于概形主题及其延伸和分支的部分,是数学史上最广泛的基础工作之一,也无疑是科学史上最广泛的工作之一。}。

无论是哪种情况,在那些为了掌握这一思想而阅读了最终展现其蓬勃发展的作品的人中——就像在一片荒地上生长出的一片茂密森林——很可能会有许多人看到这些挺拔而强壮的树木并加以利用(有人攀爬,有人取材做梁木和木板,还有人用来点燃壁炉中的火焰……),但很少有人能真正看到整片森林……