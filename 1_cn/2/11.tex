
\section{神奇的扇子——或纯真}

在新型几何学的启动与发展中,两个核心思想力量分别是概形(schéma)与拓扑斯(topos)的概念。它们几乎同时出现,并紧密共生\footnote{关于这一始于1958年的启动,参见b. de p. 第23页的注释。格罗滕迪克拓扑(Grothendieck topology)或“站点”的概念,作为拓扑斯的初步版本,紧随概形概念之后出现。正是这一概念提供了“局部化”或“下降”的新语言,在主题与概形工具的发展过程中每一步都得到应用。更为内在且几何化的拓扑斯概念,在随后的几年里起初隐而不显,直到1963年随着étale上同调的发展才逐渐显现,并逐渐被我视为最基础的概念。},自诞生之日起,它们就如同同一根动力神经,推动着新几何学的惊人发展。在总结我的工作之际,至少有必要对这两个概念稍作阐述。

概形的概念是最自然、最“显而易见”的,旨在将之前使用的无限系列“(代数)簇”概念统一为一个单一概念(每个素数对应一个这样的概念\footnote{此系列还应包括$p=\infty$的情况,对应于“零特征”的代数簇。}...)。此外,同一个“概形”(或新式“簇”)为每个素数$p$生成一个确定特征的“(代数)簇”。这些不同特征的簇的集合,可以被视作一种“(无限)簇的扇子”(每种特征对应一个)。概形就是这把神奇的扇子,它将所有可能特征的“化身”或“体现”作为不同的“分支”相互连接。由此,它提供了一个有效的“过渡原则”,将之前看似孤立、互不相连的几何学中的“簇”联系起来。如今,它们被纳入一个共同的“几何学”并通过它相互关联。我们可以称之为概形几何学,这是后来几年中它将绽放为“算术几何学”的初步形态。

概形这一概念本身简单得近乎幼稚——如此简单,如此谦逊,以至于在我之前无人想到要俯身至此。甚至可以说,它“傻”到在多年后,尽管显而易见,对我许多博学的同事而言,它仍显得“不够严肃”!我不得不独自埋头苦干数月,才说服自己这确实“行得通”——那个我固执地坚持测试的、看似愚蠢的新语言,确实足以在新的光芒与细腻中,在一个共同的框架下,捕捉到与之前“$p$特征几何学”相关的一些最初几何直觉。这是一种被所有“知情者”预先判定为愚蠢且无望的练习,而我,或许是所有同事和朋友中唯一一个,不仅想到了要尝试,甚至(受某种隐秘的驱动力驱使...)不顾一切地将其完成!

与其让我被周围那些关于何为“严肃”、何为不严肃的共识所分心,我选择像过去一样,简单地信任事物那谦卑的声音,以及我内心懂得倾听的那一部分。回报是立竿见影的,且超乎所有预期。在这短短几个月内,甚至无需刻意“努力”,我便触及了一些强大而意想不到的工具。这些工具不仅让我以更深刻的视角重新发现(仿佛游戏般)那些曾被认为艰深的旧有成果,并超越了它们,还最终让我能够着手解决“特征$p$几何”中的问题,这些问题在此之前似乎用尽所有已知方法都遥不可及\footnote{关于这一理论“强力启动”的记述,见于我在1958年爱丁堡国际数学家大会上的报告。该报告文本在我看来是介绍概形观点的最佳入门之一,或许能激励几何学读者勉力熟悉那部宏大的(后续)著作《代数几何基础》,它详尽地(且不遗漏任何技术细节)阐述了代数几何的新基础与新技巧。}。

在我们对宇宙万物(无论是数学还是其他领域)的认知中,那股革新之力无非是我们内心的纯真。那是我们出生时共同分享的原始纯真,它深藏于每个人心中,常遭我们轻视,也是我们最隐秘恐惧的对象。唯有这纯真,能将谦逊与勇敢融为一体,引领我们深入事物的核心,也让事物得以深入我们,与我们融为一体。

这种力量绝非非凡“天赋”的特权——那种(可以说)超乎寻常的脑力,用以熟练而自如地吸收和驾驭大量已知的事实、思想和技术。这样的天赋固然珍贵,对于像我这样出生时并未“超乎寻常”地得到如此恩赐的人来说,无疑令人羡慕。

然而,并非这些天赋,甚至不是最炽热的野心,辅以坚定不移的意志,能够突破那些环绕我们宇宙的“无形而严苛的界限”。唯有纯真,在不知不觉中,在我们独处倾听万物、全神贯注于孩童般游戏的瞬间,才能跨越这些界限……