\section{“魔法扇面”——或纯真}

在新几何的起步与发展中,两大关键推动力是概形(schéma,scheme)和拓扑斯(topos,topos)的思想。这两者几乎同时出现,且彼此紧密共生 \footnote{这一起步发生在 1958 年,见第 23 页脚注。位点(site,site)或“格罗滕迪克拓扑”(topologie de Grothendieck,Grothendieck topology,作为拓扑斯概念的初版)紧随概形概念之后出现。它反过来为概形主题与工具的发展提供了“局部化”(localisation,localization)或“下降”(descente,descent)的新语言,在每一步中都被使用。更具内在性与几何性的拓扑斯概念在随后几年中起初隐而不显,主要自 1963 年起随着埃塔尔上同调(cohomologie étale,étale cohomology)的发展逐渐明晰,并逐渐成为我眼中最基本的概念。},如同同一动力神经,推动了新几何自诞生之年起便惊艳崛起。为结束对我作品的概览,我至少得谈谈这两个思想。

概形的概念是最自然、最“显而易见”的想象,用以将此前扱手的无穷“簇”(variété,variety)(代数)概念系列统一为单一概念(每个素数对应一种概念 \footnote{这一系列还应包括 $p=\infty$ 的情形,对应“特征零的代数簇”(variétés algébriques de caractéristique nulle,algebraic varieties of characteristic zero)。}……)。此外,单一的“概形”(或新式“簇”)为每个素数 $p$ 孕育出一个明确定义的“特征 $p$ 的代数簇”(variété (algébrique) de caractéristique $p$,(algebraic) variety of characteristic $p$)。这些不同特征的簇集合可被想象为一种“(无穷)簇扇面”(éventail (infini) de variétés,(infinite) fan of varieties,每一特征对应一支)。这个“概形”便是那魔法扇面,将其在所有可能特征下的“化身”或“体现”如不同“分支”般联结起来。由此,它提供了一个有效的“过渡原则”(principe de passage,principle of passage),连接起此前看似或多或少孤立、彼此割裂的“簇”,隶属于不同的几何。如今,它们被包容于一个共同的“几何”之中,并由其联结。可以称之为概形几何(géométrie schématique,schematic geometry),这是“算术几何”(géométrie arithmétique,arithmetic geometry)最初的雏形,在随后几年中得以绽放。

概形这一思想本身简单得如童稚——如此朴素、谦卑,以至于在我之前无人想到俯身如此之低。甚至可以说,它“笨拙”得过分,尽管显而易见,许多博学的同事多年来仍觉其“不够严肃”!我独自紧锣密鼓地工作数月,才在角落里说服自己“这行得通”——这个新语言,如此“笨拙”,却因我无可救药的天真固执而坚持尝试,确实足以在新的光芒与精妙中,在一个如今共通的框架内,捕捉那些附着于先前“特征 $p$ 几何”(géométries de caractéristique $p$,geometries of characteristic $p$)的最早几何直觉。这类练习,在任何“消息灵通”者看来事先便愚蠢而无望,我恐怕是所有同事与朋友中唯一会突发奇想、甚至(受某种隐秘魔力驱使……)不顾一切坚持到底的人!

我未被周围关于“何为严肃、何非严肃”的共识牵绊,只是如以往般单纯信赖事物的低语,以及我内心那懂得倾听的部分。回报即刻到来,超乎所有期待。在这短短数月中,甚至无需“刻意为之”,我便触及了强大而未曾预料的工具。它们不仅让我如游戏般重现那些古老、艰深的成果,以更透彻的光芒超越之,还让我得以着手解决此前所有已知手段都无法触及的“特征 $p$ 几何”问题 \footnote{概形理论这一“强势起步”的记录,见于我 1958 年在爱丁堡国际数学家大会上的报告。该报告文本在我看来是概形观点的最佳入门之一,或许能激励几何读者勉力熟悉那部(后来的)宏大著作《代数几何基础》(Éléments de Géométrie Algébrique,Elements of Algebraic Geometry),其详尽阐述(不放过任何技术细节)了代数几何的新基础与新技术。}。

在我们对宇宙事物(数学或其他)的认知中,那革新之力无他,正是纯真(innocence,innocence)。这是我们出生时共有的原始纯真,栖于每个人内心,却常被我们轻视,成为最隐秘恐惧的对象。唯有它将谦卑与大胆合一,让我们深入事物核心,也让事物渗入我们、浸润我们。

这种力量绝非“非凡天赋”的特权——如超常的脑力,能轻松自如地吸收与操控海量已知事实、思想和技术。此类天赋固然珍贵,对于未被如此慷慨赋予之人(如我),无疑令人艳羡,“超乎一切尺度”。

然而,跨越那些“无形而强制”的圈环——它们围困我们的宇宙——靠的不是这些天赋,也不是哪怕最炽烈的雄心辅以无懈意志。唯有纯真能穿越其间,不自知也不在意,在我们独处聆听事物、沉浸于童稚游戏的瞬间……