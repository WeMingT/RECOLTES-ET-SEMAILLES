\section*{继承者与建设者(Les héritiers et le bâtisseur)}

现在是我该谈谈我的数学工作的时候了,它在我的生活中占据了重要位置,并且(令我自己惊讶的是)至今仍保持着这一地位。在《收获与播种》(Récoltes et Semailles)中,我多次回顾这一工作——有时以一种对每个人都清晰易懂的方式,有时则以略显技术性的术语\footnote{在书中各处,除了对我过去工作的数学概述外,还有一些包含全新数学发展的段落。其中最长的是《收获与播种》第四卷第171(ix)号注释中的“五张照片(晶体与$\mathscr{D}$-模)”。}。这些技术性较强的段落,很大程度上会“超乎”不仅是非专业人士的理解,甚至对那些不太“熟悉”所讨论数学的数学家同事来说也是如此。当然,你可以跳过那些对你来说过于“专业”的段落。但你也可以浏览它们,或许在阅读过程中捕捉到数学世界那“神秘之美”的一丝光芒(正如一位非数学家朋友写给我的那样),这些数学事物如同“奇异的、难以接近的岛屿”,在反思的广阔而流动的海洋中浮现……

我刚才说过,大多数数学家倾向于将自己局限在一个概念框架内,一个一劳永逸地固定的“宇宙”中——本质上,就是他们在学习时发现的“现成”宇宙。他们就像一座宏伟而美丽、设施齐全的房子的继承者,房子里有客厅、厨房、工坊,还有各种厨具和工具,足够用来烹饪和修补。至于这座房子是如何在世代更迭中逐渐建成的,某些工具(而不是其他工具)是如何被构思和制作的,房间为什么在这里这样布置,在那里又那样安排——这些问题,这些继承者从未想过要问。这就是“宇宙”,是“既定的”生活环境,仅此而已!它看似宏大(而且通常情况下,人们远未探索完所有的房间),但同时又熟悉,最重要的是:不变。当他们忙碌时,是为了维护和美化这份遗产:修理一件摇摇欲坠的家具,粉刷一面墙壁,磨砺一件工具,甚至有时,对于最有进取心的人来说,会在工坊里从头开始制作一件新家具。而且,当他们全身心投入时,新家具可能会非常美丽,整个房子也因此显得更加华丽。

然而,更罕见的是,有人会想到修改储藏室中的某件工具,甚至在反复且迫切的需求压力下,设想并制作一件新工具。在这样做时,他几乎会为自己感到某种歉意,因为他觉得自己仿佛违反了对家族传统的虔敬,这种创新似乎扰乱了那份传统。

在这座房子的许多房间里,窗户和百叶窗都被小心地关上——或许是害怕外来的风吹进来。而当新制的美丽家具——这儿一件那儿一件——加上后代子孙,开始让房间变得狭窄,甚至挤满走廊时,这些继承者中没有一个愿意承认,他们那熟悉而舒适的宇宙已经有些局促了。与其正视这一事实,他们宁愿艰难地挤来挤去:有人在路易十五风格的橱柜和藤制摇椅间钻来钻去,有人夹在流鼻涕的小孩和埃及石棺之间,还有人走投无路,只好尽力攀爬一堆摇摇欲坠、杂乱无章的椅子和长凳……

我刚刚描绘的这幅小小图景并非数学家世界的特有现象。它反映了一种根深蒂固、由来已久的制约,这种制约在所有环境和人类活动领域中都能见到,而且(据我所知)在所有社会和时代皆是如此。我已经有过机会提及这一点,并且我并不声称自己完全免于这种制约。恰恰相反,正如我的证言将要显示的。只是,在智力创造活动的相对有限层面上,这种制约对我的影响较小\footnote{我认为主要原因在于我童年直到五岁时所处的某种有利氛围。参见相关注释“纯真”(《收获与播种》第三卷,第107号)。},这种制约可以称为“文化盲视”——即无法看到(也无法活动于)周围文化所固定的“宇宙”之外。

至于我自己,我感到自己属于这样一类数学家的谱系:他们的天生使命和乐趣在于不断建造新的“房屋”\footnote{“房屋”这一原型意象在此浮现并首次被表述,见注释“仆人尹与新主人”(《收获与播种》第三卷,第135号)。}。在这一过程中,他们不禁要发明和逐步塑造所有必需的工具、器具、家具和仪器,既为了从地基到屋顶建造房屋,也为了丰富未来的厨房和工坊,并布置房屋以便居住和舒适。然而,一旦一切就绪,从最后一根排水槽到最后一张凳子,工人很少会长时间逗留在这些地方——每一个石头和每一根横梁都留下了他亲手劳作和安放的痕迹。他的位置不在那些现成的、宁静的宇宙中,无论它们多么宜人、多么和谐——无论是由他自己的手还是由前人的手所布置。其他任务已经在召唤他前往新的工地,受到他或许是唯一能清晰感受到的迫切需求的推动,或者(更常见的是)预见到他唯一能预感到的需求。他的位置在广阔的天地中。他是风的朋友,不惧怕独自劳作数月、数年,甚至如果必要的话,终其一生——除非有欢迎的接班人前来援助。诚然,他和所有人一样只有两只手——但这两只手在每一刻都知道自己该做什么,既不厌恶最粗重的活计,也不厌恶最精细的工作,而且从不厌倦于一次次认识那些无数的事物——这些事物不断召唤着它们去了解。两只手或许微不足道,因为世界是无限的。它们永远无法穷尽世界!然而,两只手,也已经很多了……

我对历史并不精通,但如果要列举属于这一谱系的数学家,我会自然而然地想到上个世纪的伽罗瓦(Galois,Galois)和黎曼(Riemann,Riemann),以及本世纪初的希尔伯特(Hilbert,Hilbert)。如果要从那些在我初入数学界时接待过我的前辈中寻找一位代表\footnote{我曾在“受欢迎的异乡人”一节(《收获与播种》第一卷,第9号)中谈及这些初体验。},首先浮现在我脑海的是让·勒雷(Jean Leray,Jean Leray)的名字,尽管我与他的接触一直非常有限\footnote{这并不妨碍我(继亨利·嘉当(Henri Cartan)和让-皮埃尔·塞尔(Jean-Pierre Serre)之后)成为勒雷引入的一个伟大创新概念——“层”(faisceau,sheaf)——的主要使用者和推广者之一。这一概念贯穿我作为几何学家的全部工作,也是我将“空间”(espace,space)(拓扑学意义上的)概念扩展为“拓扑斯”(topos,topos)的关键所在,后文将对此展开讨论。

不过,在我看来,让·勒雷与我所描绘的“建设者”肖像有所不同,他似乎并不倾向于“从地基到屋顶建造房屋”。相反,他忍不住在无人想到的地方奠定广阔的基础,同时将完成这些基础并在其上建造的任务留给他人,并且在房屋建成后,让他人入住(哪怕只是暂时的)。}。

我刚刚粗略地勾勒了两种肖像:一种是“安于现状”的数学家,满足于维护和美化遗产;另一种是“建设者-开拓者”\footnote{我悄悄地、侧面地为这一形象贴上了两个带有雄性共鸣的形容词(“建设者”和“开拓者”),它们表达了发现冲动的不同面向,其性质比这些词语所能唤起的更为微妙。这将在后续的漫步-反思中显现,见“发现母亲——或两个侧面”(第17号)。},他们不禁要不断跨越那些“无形而强制的圆环”——这些圆环划定了一个宇宙的边界\footnote{与此同时,他无意中为这一旧宇宙(即便不是为自己,至少为那些不如他灵活的同辈)设定了新的界限,这些新界限形成更大的圆环,虽然同样无形且同样强制,却取代了先前的界限。}。我们也可以用一些略显生硬但颇具启发性的名称来称呼他们,即“保守者”和“创新者”。两者都有其存在的理由和角色,在世代相传、跨越世纪和千年的共同冒险中发挥作用。在科学或艺术的繁荣时期,这两种气质之间既无对立也无敌意\footnote{例如在数学界,1948至1969年间——我作为直接见证者并身处其中时——便是如此。在我1970年离开后,似乎出现了一种大规模的反应,一种对“观念”——尤其是我所引入的重大创新观念——普遍的“轻视共识”。}。它们是不同的,彼此互补,就像面团和酵母一样。

在这两种极端类型(但本质上并不对立)之间,当然存在着一系列中间气质。某些“安于现状”的人,虽然从不考虑离开熟悉的居所,更不用说去承担在某个天知道的地方建造新居的辛劳,但当空间确实变得狭窄时,他们也会毫不犹豫地拿起泥刀,布置一个地下室或阁楼,加高一层楼,甚至在必要时,在墙上增建一些规模适中的附属建筑\footnote{我的一些“前辈”(例如在引言“受欢迎的债务”(第10节)中提到的)大多属于这种中间气质。我想到的人包括亨利·嘉当(Henri Cartan)、克洛德·舍瓦莱(Claude Chevalley)、安德烈·韦伊(André Weil)、让-皮埃尔·塞尔(Jean-Pierre Serre)、洛朗·施瓦茨(Laurent Schwartz)。除了韦伊或许例外,他们都以“同情的目光”——没有“暗中的担忧或责备”——看待我独自踏上的冒险旅程。}。虽然他们灵魂深处并非建设者,但他们常常以同情的目光——至少没有暗中的担忧或责备——看待那些曾与他们共处一室、如今却在某个偏远的乡村辛勤收集梁木和石料、仿佛已经看到一座宫殿矗立在那里的同伴……