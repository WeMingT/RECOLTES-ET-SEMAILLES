
\section{继承者与建设者}

是时候在这里谈谈我的数学工作了,它在我生命中占据了重要位置,并且(令我惊讶地)持续保持着这一地位。在《收获与播种》中,我不止一次提及这项工作——有时以每个人都能清晰理解的方式,有时则用略带技术性的语言\footnote{除了对我过去工作的数学见解外,书中还散布着包含新数学发展的段落。最长的一段是《五张照片(晶体与$\mathscr{D}$-模)》在ReS IV中的注释n$^{\circ}$ 171 (ix)。}。这些段落大多会“超出”不仅“外行”的理解范围,甚至对于那些不再或不太“涉足”相关数学领域的数学同行也是如此。你当然可以跳过那些对你来说显得过于“专业”的部分。同样,你也可以浏览它们,或许能在途中捕捉到数学世界“神秘之美”的一瞥(正如一位非数学界的朋友写信告诉我的那样),它如同在广阔流动的思维海洋中涌现的“奇异而不可及的岛屿”……

正如我之前所说,大多数数学家倾向于将自己局限在一个概念框架内,一个一旦确定便不再改变的“宇宙”——基本上,这是他们在求学时期就已发现的现成世界。他们就像继承了一座宏伟且布置妥当的大宅,里面有客厅、厨房、工作室,还有一应俱全的厨具和工具,确实,有了这些,烹饪和修理都不在话下。这座房子是如何历经数代逐渐建成的,这些工具为何及如何被设计并制造出来(而非其他……),为何房间在这里这样布置,那里又是另一种安排——这些都是这些继承者们从未想过要问的问题。这就是“宇宙”,是必须接受的“给定”,仅此而已!它看似宏大(而且往往远未探索完所有角落),同时又熟悉,最重要的是:永恒不变。当他们忙碌时,是为了维护和美化这份遗产:修理摇晃的家具,粉刷外墙,磨利工具,甚至有时,对于最具进取心的人来说,在工作室里从头打造一件新家具。当他们全身心投入时,那件家具可能美轮美奂,整座房子也因此增色不少。

更为罕见的是,其中一人会考虑对储备中的某件工具进行些许修改,或是在反复且迫切的需求压力下,构想并制造一件全新的工具。这样做时,他几乎会不自觉地道歉,因为他感到自己似乎触犯了应尊重的家族传统,觉得自己通过一项不寻常的创新冒犯了它。

在房屋的大多数房间里,窗户和百叶窗都小心翼翼地紧闭着——大概是害怕从别处吹来的风会灌进来。当新添置的漂亮家具,一件在这里,另一件在那里,更不用说那些后代们,开始让本已狭窄的房间变得拥挤,甚至侵占到走廊时,这些继承者们谁也不愿意承认,他们那熟悉而温馨的小天地开始显得有些局促了。与其接受这一现实,他们宁愿在路易十五风格的餐柜和藤制摇椅之间,或是在一个流着鼻涕的小孩和一座埃及石棺之间,勉强挤过身去;而另一些人,在绝望中,则会尽力攀爬上一堆摇摇欲坠、杂乱无章的椅子和长凳……

我刚才描绘的这幅小图景并非数学家世界所独有。它描绘了根深蒂固、由来已久的条件反射,这些条件反射在人类活动的各个领域和各个层面都能遇到,而且(据我所知)在所有社会和所有时代都是如此。我之前已经提到过这一点,并且我绝不声称自己能够免俗。正如我的证言所示,事实恰恰相反。只不过,在相对有限的创造性智力活动层面上,我受这种条件反射的影响较小\footnote{我认为主要原因在于我五岁之前所处的某种有利环境。关于这一点,请参阅“纯真”注释(ReS III, n$^{\circ}$ 107)。},这种条件反射可以被称为“文化盲视”——即无法看到(也无法行动)于周围文化所设定的“宇宙”之外。

就我个人而言,我感觉自己属于那些自发地以建造新房子为乐和使命的数学家之列\footnote{这一“房子”建造的原型意象首次出现并表述于“Yin le Serviteur, et les nouveaux maîtres”注释中(ReS III, n$^{\circ}$ 135)。}。在建造过程中,他们不可避免地也会发明并逐步打造出所有必要的工具、器皿、家具和仪器,既用于从地基到屋顶的房屋建造,也用于为未来的厨房和工作室提供充足的物资,并布置好房屋以便居住和舒适生活。然而,一旦所有东西,直到最后一块檐槽和最后一张凳子都安置妥当,工匠很少会在这些地方久留,因为每一块石头和每一根椽子都留有他亲手加工和安置的痕迹。他的位置不在那些现成的、无论多么温馨和谐的宇宙中——无论这些宇宙是由他自己的双手还是由他的前辈们所布置。其他任务已经在召唤他前往新的工地,在那些他可能是唯一能清晰感受到的迫切需求推动下,或者(更常见的是)在他预感到的需求之前。他的位置是在广阔天地中。他是风的朋友,毫不畏惧独自承担任务,无论是数月、数年,甚至是一生,除非有援手及时到来。他只有两只手,和所有人一样,这是肯定的——但这两只手每时每刻都能猜出它们该做什么,既不拒绝最繁重的工作,也不回避最精细的任务,而且永远不厌倦去认识和重新认识那些不断召唤它们去了解的无数事物。两只手或许不多,因为世界是无限的。它们永远无法穷尽世界!然而,两只手,已经足够……

我对历史并不精通,若要列举这一脉络中的数学家,我脑海中自然浮现的是伽罗瓦和黎曼(上个世纪)以及希尔伯特(本世纪初)的名字。若要在那些在我初入数学界时接纳我的前辈中寻找一位代表\footnote{我在“受欢迎的异乡人”一节(ReS I, n$^{\circ}$ 9)中谈到了这些初入时的经历。},那么首先想到的便是让·勒雷的名字,尽管我与他的接触极为有限\footnote{这并不妨碍我(继H.嘉当和J.P.塞尔之后)成为勒雷引入的一个重要创新概念——层的主要使用者和推广者之一,这一概念贯穿了我整个几何学工作的始终。它同样为我提供了扩展(拓扑)空间概念至拓扑斯的关键,下文将对此进行讨论。
此外,勒雷似乎与我描绘的“建设者”形象有所不同,在我看来,他似乎并不倾向于“从地基到屋顶完整地建造房屋”。相反,他不由自主地在无人想到的地方打下广阔的基础,而将完成建筑和在其上建造的任务留给他人,一旦房屋建成,便搬入其中(哪怕只是暂时)...}。

我刚刚勾勒了两幅肖像:一幅是满足于维护和美化遗产的“居家型”数学家,另一幅则是不断跨越“无形而威严的界限”界定宇宙的建设者-先驱者\footnote{我刚刚,不经意间且“旁敲侧击”地,将两个具有男性共鸣的形容词(“建设者”和“先驱者”)并列,它们表达了探索冲动中截然不同且更为微妙的方面,这些方面是这些名称所无法完全传达的。这将在接下来的反思漫步中,在“发现母亲——或两面”阶段($\mathrm{n}^{\circ} 17$)中显现。}。我们也可以用一些略显随意但富有暗示性的名称来称呼他们,即“保守派”和“革新派”。两者都有其存在的理由和扮演的角色,在跨越世代、世纪乃至千年的集体冒险中共同前行。在一门科学或艺术蓬勃发展的时期,这两种气质之间并无对立或对抗\footnote{特别是在数学界,在我亲身经历的1948年至1969年期间便是如此。1970年我离开后,似乎出现了一种大规模的反动,一种对“思想”尤其是对我引入的重大创新思想的“普遍蔑视共识”。}。他们彼此不同却又相辅相成,正如面团与酵母的关系。

在这两种极端(但本质上并不对立)的类型之间,自然存在着各式各样的中间性格。比如,某个“恋家者”可能从未想过离开熟悉的居所,更不用说去某个未知之地另建新居了,然而一旦空间确实变得局促,他便会毫不犹豫地拿起泥铲,改造地下室或阁楼,加高楼层,甚至在必要时,为墙壁添建一处小巧的附属建筑\footnote{我的许多“前辈”(例如在《一笔受欢迎的债务》引言第10页提及的那些)便属于这种中间性格。我特别想到了亨利·嘉当、克劳德·谢瓦莱、安德烈·韦伊、让-皮埃尔·塞尔、洛朗·施瓦茨。除了韦伊可能例外,他们都以“同情的目光”,毫无“担忧或暗自责备”,看待我独自踏上的那些冒险旅程。}。尽管他们骨子里并非建筑家,却常常以同情的眼光,至少不带忧虑或暗自责备,看待曾与他们共居一室,如今却在偏远之地辛苦收集梁木与石块,仿佛已预见一座宫殿的同伴……