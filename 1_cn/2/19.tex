
\phantomsection
\section*{尾声:无形的圈子}
\addcontentsline{toc}{chapter}{尾声:无形的圈子}
\section{死亡是我的摇篮(或三个顽童为一个垂死者)}

直到五十年代末拓扑观点的出现,空间概念的演变在我看来本质上是一种“连续”的演变。它似乎从欧几里得对我们周围空间的理论化开始,以及希腊人遗留下来的几何学,专注于研究存在于该空间中的某些“图形”(直线、平面、圆、三角形等),平稳而无跳跃地延续着。诚然,数学家或“自然哲学家”对“空间”的理解方式发生了深刻的变化\footnote{我最初撰写《跋》的意图,是包含对这些“深刻变化”的简要概述,并展示我所看到的这种“本质连续性”。我放弃了这一打算,以免使这次漫游过于冗长,它已经比预期长了许多!我打算在《反思》第四卷的历史评论中再谈这个问题,这次是面向数学读者(这完全改变了阐述的任务)。}。但这些变化在我看来都属于一种本质上的“连续性”——它们从未让数学家(像所有人一样)面对突然的陌生感。这些变化就像我们从小就认识的一个人的变化,可能深刻但渐进,从蹒跚学步到成年直至完全成熟。在某些漫长的平静时期,变化几乎察觉不到,而在其他时期则可能动荡不安。但即使在最强烈的成长或成熟期,即使我们数月甚至数年未见,也绝不会有丝毫怀疑或犹豫:我们再次见到的,依然是他,一个熟悉而亲切的存在,尽管面貌可能有所改变。

此外,我认为可以说,到了本世纪中叶,这个熟悉的存在已经相当老迈——就像一个最终筋疲力尽、被新任务压垮的人,这些任务他完全没有准备。也许他甚至已经安详离世,却无人留意并记录。“所有人”仍在忙碌于一个活人的家中,仿佛他确实还活着。

然而,想象一下,当习惯于这所房子的人们,看到原本僵直坐在椅子上的尊贵老人突然被一个活力四射、不过三颗苹果高的小男孩取代,他漫不经心、一本正经地声称自己就是空间先生(现在你甚至可以随意省略“先生”二字),这该是多么令人不快的景象!如果他至少看起来有家族特征,或许是个私生子也说不定……但完全不是!乍一看,没有任何地方让人想起我们曾经熟知(或自以为熟知……)的老空间父亲,我们当然确信(这是最起码的……)他是永恒的……

这就是那著名的“空间概念的转变”。这就是我早在六十年代初期至少就已经“看到”的、显而易见的事情,直到此刻写下这些文字之前,我从未有机会将其表述出来。而通过这一形象的唤起及其立刻引发的一连串联想,我突然以全新的清晰度看到:传统的“空间”概念,以及与之紧密相关的“多样性”(各种类型,尤其是“代数多样性”)概念,在我涉足这一领域时,已经显得如此陈旧,仿佛它们已经死去……\footnote{这一断言(对某些人来说可能显得武断)应持保留态度。它的有效性并不比我在下文重申的观点更强或更弱,即在本世纪初爱因斯坦介入之前,“牛顿模型”的(地球或天体)力学已经“垂死”。事实上,即使在今天,在物理学的大多数“常规”情况下,牛顿模型仍然完全适用,考虑到测量中允许的误差范围,寻求相对论模型将是疯狂的。同样,在数学的许多情况下,熟悉的旧“空间”和“多样性”概念仍然完全适用,无需引入幂零元素、拓扑或“适度结构”。但在两种情况下,对于尖端研究中越来越多的情境,旧的概念框架已无法表达即使是最“常规”的情况。} 我可以说,随着概型观点(及其后代\footnote{(针对数学家)在这“后代”中,我特别包括形式概型、各种类型的“多重性”(尤其是概型或形式多重性),以及所谓的“刚性解析空间”(由Tate引入,遵循我提供的“总设计师”,受到新拓扑概念和形式概型概念的启发)。这份列表远非详尽无遗……},外加一万页的基础理论)的接连出现,以及拓扑斯的观点,一个未命名的危机局面最终得以解开。

在刚才的比喻中,不应说是由一个孩子,而是由两个孩子作为突变的结果。这两个孩子之间有着不可否认的“家族相似性”,尽管他们与已故的老者并不相似。进一步细看,可以说概型这个小孩像是已故的空间之父(又名各种类型的多样性)与拓扑斯小孩之间的“亲缘纽带”\footnote{此外,对于这两个小孩,还应加上第三个更年幼的,出现在不那么宽容的时代:那就是适度空间的小孩。正如我在别处提到的,它没有得到出生证明,我是在完全非法的情况下将其列入我有幸在数学中引入的十二个“主题”之一的。}。