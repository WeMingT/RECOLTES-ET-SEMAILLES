\phantomsection
\section*{后记:隐秘的圆环}
\addcontentsline{toc}{chapter}{后记:隐秘的圆环}

\section{死亡是我的摇篮(或三个小鬼为一垂死者)}

直到五十年代末拓扑斯(topos,topos)观点出现之前,空间概念的演变在我看来本质上是一种“连续的”演进。它似乎从欧几里得(Euclide,Euclid)对我们周围空间的理论化,以及希腊人传承下来的几何学开始,平稳地、没有跳跃地延续着。那时的几何学专注于研究居住于此空间中的某些“图形”(直线、平面、圆形、三角形等)。诚然,数学家或“自然哲学家”对“空间”的构想方式发生了深刻的变化\footnote{我在撰写这篇《后记》时的初衷,是想粗略勾勒出这些“深刻变化”中的若干,并凸显我所看到的这种“本质上的连续性”。但为了不使这场漫步过于冗长——它已远超我预期——我放弃了这一打算!我考虑在《反思》第4卷的《历史评论》中再行探讨,届时面向的是数学家读者(这完全改变了阐述的任务)。}。然而,这些变化在我看来都属于一种“本质上的连续性”——它们从未让数学家,那些依恋于(如同所有人一般的)熟悉心智图像的人们,骤然面对一种突如其来的陌生感。这些变化仿佛是一个我们自幼便熟识的人,随着岁月流转,从蹒跚学步到成年乃至完全成熟,其间经历的转变——或许深刻,却渐进。在某些平静无波的漫长时段里,这些变化细微难察;在另一些时期,或许显得汹涌激荡。然而,即便在成长或成熟最为剧烈的阶段,即便我们数月乃至数年未曾谋面,也从未有过一丝疑惑或迟疑:这依然是他,那个我们熟知且亲近的存在,即便他的面容已然改变。

我甚至可以说,到本世纪中叶,这个熟悉的存在已然苍老不堪——宛如一个终于疲惫衰竭的人,被一波接一波他毫无准备的新任务所压倒。或许,他早已悄然迎来了自己的安然辞世,只是无人留意,无人记录。“所有人”依然在一位活人的家中忙碌着,仿佛他确实仍然活着,栩栩如生。

然而,试想那些习惯了这屋子的人们会有多么愕然:当他们期待看到那位端坐于扶手椅中、僵直而肃穆的老者时,却突然冒出一个活蹦乱跳的小家伙,身高不过三尺,还一本正经地、不容置疑地宣称,空间先生(Monsieur Espace,Mr. Space)——哦,现在你们甚至可以省去“先生”这称呼,随意吧——就是他!若他至少看起来还有些家族特征,或许还能算个私生子,谁知道呢……可完全不是这样!乍一看,他与我们熟知(或自以为熟知)的老空间之父(Père Espace,Father Space)毫无相似之处。我们曾确信——至少这是最起码的——他永恒不朽……

这就是那著名的“空间概念的突变”。这就是我早在六十年代初便“看到”的东西,显而易见,却直到此刻书写这些文字时,才首次有机会明确表达出来。借助这幅形象的描绘及其瞬间激发的联想之云,我突然以全新的清晰度看到:传统的“空间”概念,以及与之密切相关的“流形”(variété,variety)概念(各类流形,尤其是“代数流形”(variété algébrique,algebraic variety)),在我进入这一领域时,已然老态龙钟,仿佛早已死去……\footnote{这一断言(某些人或觉武断)需略带保留地理解。它并不比我下文认可的说法——“牛顿力学模型”(modèle newtonien,Newtonian model)在本世纪初爱因斯坦(Einstein,Einstein)出手相助时已“行将就木”——更真或更假。事实是,即便今日,在物理学中大多数“常见”情境下,牛顿模型依然完全适用,若考虑到测量中的误差范围,去追求相对论模型反倒是愚蠢之举。同样,在数学的诸多情境中,传统的“空间”与“流形”概念依然充分胜任,无需引入幂零元素(éléments nilpotents,nilpotent elements)、拓扑斯或“适度结构”(structures modérées,moderate structures)。然而,在这两者中,对于前沿研究中日益增多的语境,旧的概念框架已无法表达哪怕是最“常见”的情境。}我可以说,正是伴随着概形(schéma,scheme)观点及其后代\footnote{(致数学家读者)在这一“后代”中,我尤其包括形式概形(schémas formels,formal schemes)、各类“多重性”(multiplicités,multiplicities)(特别是概形多重性或形式多重性),以及所谓的“刚性解析空间”(espaces rigide-analytiques,rigid-analytic spaces)(由泰特(Tate,Tate)引入,其灵感源于我提供的“蓝图”,既受拓扑斯新概念启发,也源自形式概形)。此列表远非穷尽……}接连涌现,外加万余页的基础奠定,随后拓扑斯观点的出现,才最终解开了一个不言自明的危机局面。

在刚才的比喻中,与其说是一个小鬼作为突变的产物,倒不如说是两个。而且,这两个小鬼之间有着无可否认的“家族相似性”,尽管他们与那位已故的老者并无太多相像之处。甚至,若仔细观察,概形(Schémas,Schemes)这个小家伙似乎充当了已故空间之父(即各类流形)与拓扑斯(Topos,Topos)小家伙之间的“亲缘纽带”\footnote{实际上,在这两个小家伙之外,还应再添一个更年轻的小鬼,出生于不太友好的时代:适度空间(Espace modéré,Moderate Space)。正如我在别处提到的,它未获出生证明,我却仍在完全“非法”的情况下,将其列入我有幸引入数学的十二个“主导主题”之中。}。