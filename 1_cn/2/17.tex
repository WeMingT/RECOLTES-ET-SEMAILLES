
\section{探索母体——或两面性}

说实话,我对于Weil猜想本身的思考,旨在证明它们,一直是零星的。开始展现在我面前并努力审视和捕捉的全景,其广度和深度远远超出了证明所需的假设需求,甚至超出了这些著名猜想最初所能暗示的一切。随着概型主题和拓扑斯主题的出现,一个全新且意想不到的世界突然敞开。“猜想”在其中占据了核心位置,确实,就像一个大帝国或大陆的首都,拥有无数省份,但其中大多数与这个光辉而显赫的地方只有最遥远的联系。我从未明言,但我知道自己从此成为一项伟大任务的仆人:探索这个广阔而未知的世界,把握其轮廓直至最远的边界;同时,全方位地遍历并系统而细致地盘点那些最接近、最易达的省份,绘制出忠实且精确的地图,连最小的村庄和最简陋的茅屋都有其位置……

尤其是这最后一项工作,占据了我大部分精力——这是一项耐心而广泛的基础工作,只有我清楚地看到,尤其是“深切感受到”。从1958年(概型主题和拓扑斯主题接连出现的那一年)到1970年(我退出数学舞台的那一年),这项工作远远占据了我最多的时间。

很多时候,我因被这些无尽的任务所束缚而感到焦躁,这些任务(一旦看到本质)对我来说更像是“后勤工作”,而非向未知的冲刺。我不得不不断抑制那种向前冲的冲动——那种先驱者或探索者的冲动,他们出发去发现和探索无名未知的世界,不断呼唤我去认识和命名它们。这种冲动,以及我投入其中的精力(几乎是偷偷摸摸地!),总是处于极度匮乏的状态。

然而,我深知,正是这种从“任务”中“偷来”的精力,才是最稀有、最自由的精髓——在我作为数学家的工作中,“创造”首先就体现在这里:在那强烈的关注中,去把握温暖而无穷尽的母体那阴暗、无形且潮湿的褶皱中,尚未诞生之物的最初形态和轮廓的痕迹,它们似乎在呼唤我,以成形、具象并诞生……在探索工作中,这种强烈的关注,这种炽热的关怀,是一种本质的力量,就像太阳的热量对于埋藏在肥沃土壤中的种子的黑暗孕育,以及它们卑微而奇迹般地破土而出,迎接日光。

在我的数学工作中,我主要看到这两种力量或冲动在起作用,它们同样深刻,但性质(在我看来)不同。为了描述这两者,我使用了建设者的形象,以及先驱者或探索者的形象。将它们并列,我突然意识到它们都非常“阳刚”,非常“男性化”,甚至“大男子主义”!它们带有神话的高傲回响,或是“重大时刻”的共鸣。无疑,它们受到我内心残留的“英雄式”创作观——超级阳刚的视角——的启发。如此这般,它们呈现了一种强烈着色、甚至僵化、“立正”的现实观,而现实本应更加流动、谦逊、“简单”——一种活生生的现实。

在这股“建设者”的强烈冲动中,它似乎不断驱使我投身于新的工地,然而,我同时也能辨识出居家者的那份情怀:深深依恋于“那个”家的人。首要的,是“他的”家,属于“亲近之人”的所在——一个他感到自己是其中一部分的亲密生命体的栖息地。随后,随着“亲近”之圈的扩展,它才逐渐成为“众人的家”。在这“建造家园”的冲动中(如同“进行”爱抚一般……),首要且核心的,是一份柔情。那是对待每一块亲手雕琢、倾注爱心、唯有通过这种深情的触碰才能真正了解的材料所怀有的接触冲动。当墙壁竖起,梁柱与屋顶安置妥当,一间间房间被布置起来,目睹这些厅堂、卧室及角落中逐渐建立起和谐有序的生活之家——美丽、温馨、宜居——内心便涌起深深的满足。因为家,在我们每个人内心深处,首先且隐秘地,也是母亲——环绕并庇护我们的存在,既是避风港也是慰藉;或许(更深一层,即便我们正亲手一砖一瓦地构建它),它也是我们自身的起源,那个在我们出生前已被永远遗忘的时代里,曾庇护并滋养我们的所在……它同样是怀抱。

而先前自发浮现的形象,为了超越“先驱者”这一显赫称号,揭示其背后更为隐秘的现实,同样剥离了所有“英雄主义”的色彩。再次,显现的是母性的原型意象——滋养的“子宫”及其无形且晦暗的劳作……

这两股在我看来“性质各异”的冲动,最终比我预想的更为接近。两者皆属于“接触冲动”的本质,引领我们与“母亲”相遇:她既体现亲近、“已知”,也包含“未知”。屈服于其中任何一种冲动,便是“重归母亲怀抱”。这是对亲近、“或多或少已知”以及“遥远”、“未知”但同时又预感即将揭示的事物的接触的更新。

这里的区别在于调性、比例,而非本质。当我“建造房屋”时,是“已知”主导;而当我“探索”时,则是未知。这两种“模式”的发现,或更准确地说,同一过程或同一工作的这两个方面,是密不可分的。它们各自至关重要,且互为补充。在我的数学工作中,我察觉到一种在这两种探索方式之间,或者说,在其中一种占主导的时刻(或时期)与另一种占主导的时刻之间不断往复的运动\footnote{我在此关于数学工作的论述同样适用于“冥想”工作(这在《收获与播种》中多处有所提及)。我几乎可以肯定,这是所有发现工作中都会出现的现象,包括艺术家的创作(比如作家或诗人)。我这里描述的两种“面向”也可以被视为,一方面是表达及其“技术”要求,另一方面是接收(各种感知和印象),通过高度集中的注意力转化为灵感。两者在工作的每一刻都同时存在,且在这两种“时间”之间——一种占主导,另一种占主导——存在着这种持续的“往复”运动。}。但同样明显的是,在每一刻,这两种模式都是并存的。当我构建、布置,或清理、整理、排序时,是工作的“阳”或“男性”面向在定调。当我摸索着探索难以捉摸、无形、无名之物时,我则处于“阴”或“女性”的面向。

对我而言,不存在贬低或否认我本性中任何一面的问题,两者都至关重要——“男性”面构建并孕育,“女性”面构思并孕育缓慢而隐秘的成长。我“是”两者——“阳”与“阴”,“男”与“女”。但我也知道,在创造过程中,最微妙、最轻盈的本质往往存在于“阴”、“女性”的一面——谦逊、幽暗,且常常显得微不足道的一面。

正是工作的这一面,我相信,自古以来对我施加了最强烈的吸引力。然而,盛行的共识却鼓励我将大部分精力投入到另一面,即体现在有形“产品”中的那一面,更不用说那些轮廓分明、以雕琢石头的明显性证明其真实性的成品……

回首往事,我清楚地看到这些共识如何影响了我,以及我是如何“承受其重”的——以柔克刚!直到我离开的那一刻,我工作中“构思”或“探索”的部分仍被压缩至最小限度。然而,在回顾我作为数学家的成就时,一个惊人的事实显而易见:构成我作品精髓与力量的,正是如今被忽视、甚至沦为嘲笑或居高临下轻蔑对象的那一面——“思想”,乃至“梦想”,而绝非“成果”。在这些篇章中,我试图通过俯瞰森林而非细究树木的视角,界定出我为当代数学贡献的最核心内容——我所见的,不是一系列“伟大定理”的榜单,而是一幅生机勃勃、孕育丰硕思想的画卷\footnote{并非说我的作品中缺少所谓的“伟大定理”,包括那些解决了他人提出、此前无人能解的问题的定理。(我在第554页的脚注b中回顾了其中一些,见“涨潮之海...”笔记(ReS III, n ${ }^{\circ}$ 122)。)但正如我在这段“漫步”之初就已强调的(在“观点与视野”阶段,$\mathrm{n}^{\circ} 6$),这些定理对我来说,唯有置于由某个“丰硕思想”开启的重大主题的滋养背景下,才具有完整的意义。它们的证明因此如同泉水般自然流畅地涌现,源自承载它们的主题本身的“深度”——就像河流的波浪似乎温柔地从其水深处诞生,无断裂,无费力。我在已引用的“涨潮之海...”笔记中,以类似但不同的意象表达了这一点。},所有这一切共同汇聚成一个宏大而统一的视野。