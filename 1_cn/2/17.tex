\section{“发现母亲”——或双重面向}

实话说,我对韦伊(Weil,Weil)猜想本身的思考及其证明,始终断续。眼前逐渐展开的图景,我尽力探察与捕捉,其广度与深度远超任何证明的假想需求,甚至超越这些著名猜想最初揭示的一切。随着“概形”(schémas,schemes)主题与“拓扑斯”(topos,topos)主题的浮现,一个崭新而未曾预料的世界骤然开启。“猜想”在其中居核心地位,诚然,宛如广袤帝国或大陆的首都,周围环绕无数省份,但大多与这耀眼尊贵之地仅存遥远关联。我从未自言,却深知自己已成为一项伟大使命的仆人:探索这浩瀚未知的世界,领会其轮廓直至最远边界;同时,遍历并以顽强而有条理的细心清点最邻近、最易触及的区域,绘制忠实精准的地图,连最小村落与茅舍皆有其位……

尤是这后一工作,耗费我最多精力——一项耐心而宏大的奠基工程,唯我清晰洞见,更由“肺腑之感”领会。它占据了我 1958 年(“概形”与“拓扑斯”主题接连问世之年)至 1970 年(我退出数学界之年)间绝大部分时光。

我常为此受缚而暗自焦躁,仿佛被顽强黏滞的重担牵制,这些无尽任务,在洞悉本质后,于我更似“杂务”,而非投身未知的豪情。我须时时抑住前冲的冲动——那开拓者或探险家的渴望,奔向未知无名的世界,它们不断召唤我去认识与命名。这冲动及其投入的精力(几近偷取!),始终仅得微薄份额。

然而,我心底深知,这偷来的能量,才是稀有而精妙之本质——我数学工作的“创造”,首要在此:在那炽烈专注中,探寻温暖丰饶的孕育母体,其幽暗、无形、湿润的褶层里,初现尚未诞生的形迹与轮廓,它们似在呼唤我,欲成形、化身、诞生……在发现之旅中,这炽烈专注与热切关怀是根本之力,恰如太阳之暖,催动深埋滋土的种子暗中孕育,谦卑而奇迹般绽放于日光之下。

在我数学工作中,我尤见两种深邃之力或冲动并存,其性似异。为喻此二者,我用了“建造者”与“开拓者”或“探险家”的意象。并置观之,二者骤显甚“阳”、甚“雄性”,乃至“刚霸”!它们带有神话的傲然回响,或“盛大场合”的共鸣。无疑,它们受我昔日“英雄式”创造观的遗迹启发,那极“阳”之见。此态呈现的图景浓烈偏颇,甚至僵硬,“立正肃立”,远不及真实那般流畅、谦卑、简朴——不及那活泼的真实。

在“建造者”这雄性冲动中——似无休止催我开启新工地——我却也辨出“居者”之情:那深系“家”之人。首先,那是“他的”家,亲近之所——他自感隶属的亲密活体之地。其次,随“亲近”圈扩展,才成为“众人的家”。在这“造屋”冲动中(如同“做爱”……),首要还有柔情。有与材料逐一接触的冲动,以爱意塑形,唯此爱之触方真知其性。墙立、梁置、屋顶盖就后,有深沉满足于逐室安顿,眼见厅堂、卧房、小间渐成和谐的活屋秩序——美观、迎人、宜居。因家,于我们每人内心深处,皆为母亲——环绕与庇护我们,既是避所又是慰藉;或许(更深层,即便我们正全力建造),它也是我们的源头,曾在出生前那永忘之时庇护与滋养我们……它亦是“怀抱”(Giron,Womb)。

先前自发浮现的意象,欲超越“开拓者”的显赫称谓,触及它掩藏的真实,亦无一丝“英雄”气息。那仍是母性之原型——“母体”(matrice,matrix)的滋养及其幽暗无形的辛劳……

这两股看似“性异”的冲动,终比我所想更近。二者皆为“接触冲动”(pulsion de contact,impulse of contact),引我们会“母亲”(la Mère,the Mother):那既体现“亲近”“已知”,又体现“未知”之存在。任由任一冲动牵引,皆是“重会母亲”。既更新与“亲近”“略知”之物的联系,又触及“遥远”“未知”却隐约预感、即将显露之物。

此间差异在色调与比重,而非本质。“建屋”时,“已知”主调;“探索”时,“未知”领衔。这两种发现“模式”,或更恰当说,同一过程或工作的双面,相辅相成,缺一不可。在我数学工作中,我察觉这两种进路间——或其主导的时段间——有恒常往复 \footnote{我论数学工作的此言,亦适用于“冥想”(méditation,meditation)工作(《收获与播种》中多处提及)。我无疑认为,这在一切发现工作中皆现,包括艺术家(例如作家或诗人)之作。我述之“双面向”(versants,slopes),亦可视为一为“表达”及其“技术”需求,一为“接收”(种种感知与印象),因炽烈专注化为灵感。二者每刻皆存,其间有恒常“往复”,在某者主导之时与彼者主导之时间。}。但显然,每刻二者皆在。建构、布置,或清扫、整理、归序时,“阳”或“雄性”面向定调;摸索探寻那不可捉、未成形、无名之物时,我为“阴”或“雌性”一面。

我不欲贬抑或否认我本质之任一面,二者皆不可或缺——“雄性”建构与孕生,“雌性”孕育与庇护幽暗缓慢的孕育。我兼具二者——“阳”与“阴”,“男”与“女”。但我也知,创造过程最精妙、最轻灵之本质,在“阴”“雌性”面向——那谦卑、幽暗、常貌不惊人的一面。

自始,我信此面向对我诱力最强。然通行共识却催我将大半精力投于另一面,那化身并彰显于有形“产物”——乃至完竣、轮廓分明的产物,以雕石之明证宣示其真实……

回望,我清楚见这些共识如何压我,我也如何“顺受其重”——柔韧地!至我离去,“构思”或“探索”部分确受限微薄份额。然回顾我数学家之作,最本质与力量赫然源自如今被忽视——甚至遭嘲笑或傲慢轻蔑——的此面:即“理念”(idées,ideas),乃至“梦”(rêve,dream),绝非“结果”(résultats,results)。于此页试图圈定我为时代数学献上的最要之物,以览森林而非驻足树木的目光——我未见“伟大定理”功绩簿,而是一扇活泼的丰饶理念之谱 \footnote{我作中不乏所谓“伟大定理”,包括解决前人未解之问(《海涨……》(La mer qui monte…,The Rising Sea)注释(《收获与播种》第三卷,注释 122,页 554)中回顾若干)。然如我于此“漫步”初(《观点与愿景》(Points de vue et vision,Viewpoints and Vision),注释 6)所述,这些定理唯在丰饶主题——由“丰饶理念”启始——的滋养语境中,方具全义。其证明遂如泉涌,无阻无碍,自主题之性与“深度”流出——如河浪柔然生于水深,无断无劳。我于前述《海涨……》中,以他喻表达同义。},皆共赴同一宏大愿景。