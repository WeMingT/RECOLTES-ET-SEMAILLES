
\section{内心之旅——或神话与见证}

首先,《收获与播种》是对我自己及我生活的反思。因此,它也是一种见证,且以两种方式呈现。它是对我过去的见证,这是反思的主要承载。但同时,它也是对最直接当下的见证——即我写作之时,以及《收获与播种》一页页诞生的时刻,不分昼夜。这些页面忠实记录了我对生活的长期冥想,正如它真实地持续着(并且此刻仍在继续……)。

这些页面并无文学野心。它们构成了关于我自身的一份文献。我只在非常有限的范围内允许自己对其进行修改(主要是偶尔的文体修饰)\footnote{因此,对于可能存在的错误(实质性的或视角上的等)的修正,并非初稿的修改机会,而是通过脚注或后续对考察情境的“回顾”来完成。}。若说有何追求,那便是力求真实。而这已足够重要。

此外,这份文献并非“自传”。你从中不会得知我的出生日期(这对绘制星盘或许有些许意义),也不会了解我父母的名字及其生平,不会提及曾为我妻及在我生命中占据重要地位的其他女性的名字,也不会提及这些爱情结晶的子女及其各自的生活轨迹。并非这些事物在我生命中无足轻重,或至今仍不重要。而是随着自我反思的展开与延续,我从未感到有必要深入描述这些我偶尔触及的事物,更不用说刻意罗列姓名与数字。我始终认为,这些内容无法为我当时所追求的主题增添任何价值。(尽管在前几页中,我或许不由自主地包含了比后续千页更多的个人生活细节……)

若你问我,这千页篇幅所追求的“主题”究竟是什么,我会回答:是讲述并由此发现我生命中的内心冒险。这一冒险的叙述-见证同时在我刚才提到的两个层面上展开。一方面,是对过去冒险的探索,追溯其根源直至我的童年。另一方面,则是这一“同一”冒险在《收获与播种》写作过程中的延续与更新,作为对外界强烈呼唤的自然回应\footnote{关于这一“强烈呼唤”的具体说明,参见“信件”,特别是第3至8节。}。

外部事件仅在激发内心冒险的反弹或有助于其明晰时,才为反思提供养分。而我的数学作品被埋葬与掠夺,这一事件便是如此强烈的刺激。它在我内心激起了强烈的自我反应,同时揭示了我与源自我的作品之间深藏不露的紧密联系。

确实,我属于“数学高手”这一事实,并不必然成为你对我这段“特殊”冒险感兴趣的理由(更谈不上是个好理由)——同样,我在改变环境和生活方式后与同事间产生的摩擦,也不足以引起你的兴趣。况且,不乏同事乃至朋友,他们认为公开表露(如他们所言)自己的“内心状态”极为可笑。重要的是“成果”。而“灵魂”,即我们体内体验这些“成果”产生过程的部分,或是其带来的各种影响(无论是在“生产者”的生活中,还是在其同类的生活中),却遭到轻视,甚至公开嘲笑。这种态度自诩为“谦逊”的表现,而我视之为逃避的迹象,一种由我们呼吸的空气所助长的奇异失调。显然,我并非为那些深受这种自我潜在蔑视之苦的人写作,这种蔑视使他们对我所能提供的最宝贵之物不屑一顾。这是一种对真正构成其自身生活,以及构成我生活的要素的蔑视:那些驱动心灵的表层与深层、粗糙或细腻的运动,正是这“灵魂”体验并回应经历,时而凝固,时而绽放,时而退缩,时而学习……

唯有亲身经历者,才能讲述内心的冒险,他人无法代劳。然而,即便这叙述仅为自己而作,也鲜有不滑入构建神话的窠臼者,其中叙述者自封为英雄。此类神话非源于民族与文化的创造性想象,而是源自不敢直面卑微现实、乐于以心灵构建取而代之者的虚荣。但真实的叙述(若有之),如实记录一段真正经历的冒险,实属珍贵。其价值不在于叙述者可能(无论对错)享有的声望,而仅在于其存在本身,及其真实性的品质。这样的见证弥足珍贵,无论它出自一位声名显赫乃至杰出的人物,还是一位前途渺茫、肩负家庭重担的小职员,或是一名普通法罪犯之口。

若此类叙述对他人有所裨益,首要之处在于通过他人未经粉饰的经验见证,使其重新面对自我。或者换种说法,或许能在他心中(哪怕仅在阅读的片刻)消除那种对自己冒险经历的轻视,以及对那既是乘客又是船长的“灵魂”的轻蔑……