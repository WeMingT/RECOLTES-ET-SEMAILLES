\section{内在的冒险——或神话与见证}

在一切之前,《丰收与播种》(Récoltes et Semailles,Harvests and Sowings)是对我自身及我生命的反思。由此,它也是一种见证,以两种方式呈现。它是对我过去的见证,反思的重心落在那里。但与此同时,它也是对最当下之此刻的见证——即我写作的瞬间,《丰收与播种》的页面在时辰、在昼夜交替中诞生的时刻。这些页面忠实地见证了我对生命的漫长冥想,这冥想真实地延续着(甚至在此刻仍在继续……)。

这些页面并无文学上的野心。它们构成了关于我自身的文献。我仅在极狭窄的限度内允许对其稍作修饰(尤其是偶尔进行文体上的润色)\footnote{因此,对可能的错误(无论是实质性还是视角上的等等)的修正,并未成为修改初稿的契机,而是通过脚注,或在后续对所审视情境的“回溯”中完成。}。若说它有何企图,那仅仅是力求真实。而这已然意义重大。

然而,这份文献绝非“自传”。你不会从中得知我的出生日期(这或许仅对绘制星盘有些许用处),也不会知道我父母的名字或他们的职业,亦无从了解我曾娶之妻及其他在我生命中重要的女性的姓名,或那些因爱而生的孩子们的名字,以及他们各自如何度过人生。这并非说这些事物在我生命中无足轻重,或如今已不再重要。只是,在这场关于自身的反思开始并延续的过程中,我从未感到有任何冲动,去稍稍描述那些我偶尔触及的事物,更不用说一丝不苟地罗列名字与数字。在任何时候,我都不觉得这能为我当时追寻的旨意增添什么。(然而,在前几页中,我似乎不由自主地提及了比随后千页更多的关于我生活的具体细节……)

若你问我,这贯穿千页的“旨意”究竟为何,我会答:它是对我生命这一内在冒险的叙述,同时也是对它的发现。这冒险的叙述与见证,在我刚述及的两个层面上同时展开。其一是探索过去的冒险,追溯其根源与起源,直至我的童年。其二是这“同一”冒险的延续与更新,在我书写《丰收与播种》的瞬间与日子里,作为对外界突如其来的强烈质询的自发回应而展开\footnote{关于这“强烈质询”的详情,见“信件”(Lettre,Letter),尤其是第3至8节。}。

外部事实仅在激发或推动内在冒险的转折,或有助于阐明它时,才滋养这场反思。而对我数学作品的埋葬与掠夺——这将是后文长篇讨论的主题——便是这样一种挑衅。它在我内心激起了强烈的自我反应,同时揭示了我与自身作品之间那些深邃而未曾察觉的联系,至今仍将我与之相连。

诚然,我属于“数学强者”之列,这未必是让你关注我这场“特定冒险”的理由(更遑论充分理由)——我与同事间的纠葛,或因生活环境与方式的转变而生的麻烦,亦是如此。况且,不乏同事乃至朋友认为,公开袒露(他们如是说)“内心状态”是极为荒谬的。他们眼中重要的唯有“结果”。至于“灵魂”——即我们内在那体验“结果生产”及其种种后果(无论对“生产者”自身,还是对同类)的部分——却被轻视,甚至公开遭到嘲弄。这种态度自诩为“谦逊”的表达,我却从中看到逃避的痕迹,以及一种奇异的失调,由我们呼吸的空气所助长。可以肯定,我并非为那些对自己怀有隐秘轻蔑之人而写,这种轻蔑让他们鄙弃我所能给予的最珍贵之物。那是对真正构成其自身生命,以及我的生命的事物的轻蔑:那些驱动心灵的表层与深层、粗糙或微妙的波动,那正是体验并回应经验的“灵魂”,它或僵滞或绽放,或退缩或学习……

内在冒险的叙述只能由亲历者述说,别无他人。然而,即便这叙述仅为自己而写,也罕能避免滑入构建神话的窠臼,使叙述者成为其中的英雄。这类神话并非源于民族与文化的创造想象,而是出自那不敢直面朴素现实者的虚荣,他们乐于以精神构造取而代之。但一个真实叙述(若真有其事),述说一场真切经历的冒险,却是无价之物。这价值并非来自围绕叙述者的是非功名(无论对错),而仅因其存在及其真实的特质。这样的见证弥足珍贵,无论它出自声名显赫之人,还是无望的小职员与一家之主,或是普通的罪犯。

若此叙述对他人生有何裨益,那首先是通过另一个人的坦诚见证,让读者重新面对自身。或者换言之,它或许能(哪怕仅在阅读的片刻间)抹去他对自身冒险及那身为旅人与舵手的“灵魂”所持的轻蔑……
