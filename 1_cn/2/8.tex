\section{愿景——或十二个主题的和声}

或许可以说,"伟大思想"是这样一种观点:它不仅展现出新颖且富有成果的特质,更在科学中引入了一个全新而广阔的主题,使之得以体现。而当我们不将科学视为权力与支配的工具,而是将其视为人类世代相传的认知冒险时,任何科学都不过是这种和声——在不同时代或广阔或狭小,或丰富或贫瘠——它通过各个主题的微妙对位法,在世代与世纪中展开,如同从虚无中被召唤出来,彼此交织融合。

在我所揭示的众多数学新观点中,有十二个,回顾起来,我愿称之为"伟大思想"\footnote{为满足好奇的数学读者,以下是这十二个主导思想或我作品的"主旋律"(按出现时间顺序排列):

1. 拓扑张量积与核空间
2. "连续"与"离散"对偶(导出范畴,"六运算")
3. 黎曼-罗赫-格罗滕迪克瑜伽($K$-理论,与相交理论的关系)
4. 概形
5. 拓扑斯
6. 平展上同调与$\ell$-进上同调
7. 动机与动机伽罗瓦群(格罗滕迪克的$\otimes$-范畴)
8. 晶体与晶体上同调,德拉姆系数、霍奇系数瑜伽...
9. "拓扑代数":$\infty$-叠、导子;拓扑斯的上同调形式体系,作为新同伦代数的灵感
10. 温和拓扑
11. 代数几何阿贝尔瑜伽,伽罗瓦-泰希米勒理论
12. 正多面体与各类正则构型的"概形"或"算术"观点

除第一个主题的重要部分出现在我的博士论文(1953年)并在1950-1955年间的泛函分析时期得到发展外,其余十一个主题均在我1955年后的几何学家时期逐渐形成。}。

要"看见"我的数学作品,就是要"感受"其中至少某些思想,以及它们所引入的这些宏大主题——它们构成了作品的经纬与灵魂。

由于事物本质,某些思想比其他思想"更伟大"(相应地,其他思想就"更渺小"!)。换言之,在这些新主题中,有些比其他主题更广阔,有些则更深地触及数学事物的奥秘核心\footnote{在这些主题中,我认为最广阔的是拓扑斯,它提供了代数几何、拓扑与算术的综合思想。就目前发展程度而言,最广阔的是概形主题。(参见第20页脚注(*))它为其他八个主题(即除第1、5、10主题外的所有主题)提供了"卓越"的框架,同时为代数几何及其语言的彻底革新提供了核心概念。

相反,十二个主题中的第一个和最后一个在我看来比其他主题规模更小。然而,就最后一个主题而言,它为正多面体与正则构型这一古老主题引入了新视角,我怀疑即使一位数学家倾其一生也难以穷尽。至于第一个主题——拓扑张量积,它更多扮演了一个即用型新工具的角色,而非后续发展的灵感源泉。尽管如此,直到近年我仍不时收到零散的研究成果,这些成果在二三十年后解决了我当年遗留的某些问题。

在我看来,这十二个主题中最深刻的是动机主题,以及与之紧密相关的代数几何阿贝尔瑜伽与伽罗瓦-泰希米勒瑜伽。

从工具完善程度与使用广度来看,过去二十年间在多个"前沿领域"研究中广泛使用的"概形"与"平展上同调及$\ell$-进上同调"部分最为显著。对于一位消息灵通的数学家而言,我认为现在已毋庸置疑,概形工具及其衍生的$\ell$-进上同调工具属于本世纪几项重大成就之列,它们滋养并革新了近几代人的数学科学。}。

其中有三个主题(在我看来并非最不重要的)在我退出数学舞台后才出现,至今仍处于萌芽状态;"正式"而言它们甚至不存在,因为没有正式出版物为其提供出生证明\footnote{唯一"半官方"提及这三个主题的文本是1984年1月为申请法国国家科学研究中心(CNRS)借调而撰写的《纲领草案》。该文本(在引言3"指南针与行囊"中也有提及)原则上将收录于《反思》第四卷。}。

在我退出前出现的九个主题中,最后三个主题当时正处于蓬勃发展期,如今却仍处于幼年期,因为在我离开后缺乏关爱之手来照料这些"孤儿",它们被遗弃在一个充满敌意的世界中\footnote{在我离开后不久,这三个孤儿就被无声无息地埋葬了,其中两个在1981年和次年(鉴于操作的成功)被大张旗鼓地重新发掘,却未提及原作者的贡献。}。

至于其他六个主题,它们在我离开前的二十年里已臻成熟,可以说(除一两个保留意见外\footnote{"几乎"主要涉及格罗滕迪克对偶瑜伽(导出范畴与六运算)和拓扑斯瑜伽。这将在《收获与播种》第二部分和第四部分(《葬礼(1)》和《(3)》)中详细讨论。})它们在当时已成为共同遗产:尤其在几何学家群体中,如今"每个人"都在不知不觉中(如同莫里哀笔下的茹尔丹先生写散文般)日复一日地运用它们。它们已成为"做几何"时的空气,或在做算术、代数或多少带点"几何"的分析时不可或缺。

我作品中的这十二个宏大主题绝非彼此孤立。在我看来,它们构成了精神与意图的统一体,如同贯穿我所有"书面"与"非书面"作品的共同而持久的基调。在写下这些文字时,我仿佛又听到了同样的音符——如同一种召唤!——穿越那些无偿、执着而孤独的工作岁月,那时我甚至不曾关心世上是否还有其他数学家存在,只因我被那召唤我的事物深深吸引...

这种统一性不仅源于同一工匠在其作品上留下的印记。这些主题通过无数微妙而明显的联系相互关联,如同在一首宏大对位法中展开并交织的各个清晰可辨的主题——它们被一种和声所集合、推动,赋予每个主题以意义、运动与完满,而这一切都与其他主题息息相关。每个局部主题似乎都诞生于这种更广阔的和声,并随着时间流逝从中重生,远胜于和声作为"总和"或"结果"由预先存在的构成主题所呈现。说实话,我无法摆脱这样一种感觉(或许有些荒谬...):在某种意义上,正是这种尚未显现但肯定已然"存在"的和声——它潜藏在尚未诞生之物的幽暗怀抱中——依次唤起了这些主题,而这些主题只有通过它才能获得完整意义;也正是它在那些炽热孤独的青春岁月里,以低沉而迫切的声音召唤着我...

无论如何,我作品中的这十二个主旋律都仿佛被一种隐秘的预定所引导,共同谱写出一首交响曲——或者换一个比喻,它们体现了不同的"观点",共同指向同一个宏大的愿景。

这一愿景直到1957、58年左右才开始从迷雾中显现,呈现出可辨识的轮廓——那是孕育的岁月\footnote{1957年是我提出"黎曼-罗赫"主题(格罗滕迪克版本)的一年——一夜之间,我成为了"大明星"。这也是我母亲去世的年份,标志着我的生命中一个重要转折点。这是我一生中创造力最旺盛的年份之一,不仅在数学层面。那时,我全部的精力已投入数学工作整整十二年。那一年,我开始感到自己已经"走遍"了数学工作的方方面面,也许是时候投身其他事物了。那显然是一种内在更新的需求,在我生命中首次浮现。当时我考虑成为一名作家,并停止了所有数学活动数月之久。最终,我决定至少要把手头已有的数学工作付诸文字,估计需要几个月,最多一年...

或许时机尚未成熟,无法实现这一重大跨越。无论如何,一旦重新开始数学工作,它就再次占据了我。在接下来的十二年里,它再也没有放开我!

这次间歇后的第二年(1958年)或许是我数学生涯中最富成果的一年。这一年见证了新几何两个核心主题的诞生:概形理论的强势启动(同年夏天我在爱丁堡国际数学家大会上的报告主题),以及"位点"概念的出现——这是关键概念拓扑斯的临时技术版本。回顾近三十年后的今天,我可以说是这一年真正诞生了新几何的愿景,伴随着这一几何的两大工具:概形(代表了旧有"代数簇"概念的蜕变)与拓扑斯(代表了更深层次的"空间"概念的蜕变)。}。奇怪的是,这一愿景对我来说如此接近,如此"显而易见",以至于直到一年前\footnote{我第一次想到为这一愿景命名是在1984年12月4日的思考中,在"阴之仆(2)——或慷慨"一节的子注($\mathrm{n}^{\circ} 136_{1}$)(《收获与播种》第三卷,第637页)。},我都没有想过要给它一个名字。(尽管我的一大爱好就是不断为发现的事物命名,作为理解它们的第一步...)确实,我无法指出某个特定时刻,将其视为这一愿景出现的时刻,或在回顾时能将其认定为这样的时刻。一个新的愿景是如此广阔,以至于它的出现不可能定位于某个特定时刻,而必须经过漫长岁月(如果不是几代人的话)逐渐渗透并占据那些观察与沉思者的身心;仿佛新的眼睛必须在熟悉的双眼背后艰难地形成,逐渐取代它们。这一愿景也过于广阔,以至于无法像抓住路旁突然出现的第一个概念那样"抓住"它。因此,最终不必惊讶,为如此广阔、如此接近又如此弥散的事物命名的想法,只有在回顾时,当它完全成熟后才出现。

说实话,直到两年前,我与数学的关系还仅限于(除了教学任务外)实践数学——追随一种不断推动我前进的冲动,进入那不断吸引我的"未知"。我从未想过要停下脚步,哪怕片刻,回头看看或许已经走过的道路,甚至定位一段已完成的作品。(无论是将其定位在我的生命中,作为一件与我保持着深刻而长期被忽视的联系的事物;还是将其定位在"数学"这一集体冒险中。)

更奇怪的是,要让我最终"停下"脚步,重新认识这件半被遗忘的作品,或仅仅想到为作为其灵魂的愿景命名,我必须突然面对一场规模空前的葬礼的现实:通过沉默与嘲弄,埋葬了愿景,也埋葬了孕育它的工匠...
