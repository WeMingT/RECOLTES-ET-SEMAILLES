\section{“愿景”——或十二个主题的和谐}

或许可以说,“伟大思想”是一种观点,它不仅新颖且富有成果,还在科学中引入了一个崭新而广阔的主题来体现它。而任何一门科学,当我们不将其视为权力和统治的工具,而是作为我们这个物种穿越岁月认知冒险时,无非是这种和谐。这种和谐在不同时代或广或狭,或丰饶或贫瘠,通过一代又一代、一个世纪又一个世纪的展开,由所有依次出现的主题以精妙的对位法构成,仿佛从虚空中被召唤而来,加入其中并彼此交织。

在我发掘的众多数学新观点中,回过头来看,有十二个我称之为“伟大思想”的主题 \footnote{以下是为好奇的数学读者列出的这十二个主导思想,或我作品中的“主导主题”(按出现的时间顺序排列):

1. 拓扑张量积与核空间(Produits tensoriels topologiques et espaces nucléaires,Topological tensor products and nuclear spaces)。  
2. “连续”与“离散”的对偶性(Dualité "continue" et "discrète","Continuous" and "discrete" duality)(导出范畴(catégories dérivées,derived categories)、“六运算”)。  
3. 黎曼-罗赫-格罗滕迪克瑜伽(Yoga Riemann-Roch-Grothendieck,Riemann-Roch-Grothendieck yoga)($K$-理论($K$-théorie,$K$-theory),与交理论的关系)。  
4. 概形(Schémas,Schemes)。  
5. 拓扑斯(Topos,Topos)。  
6. 埃塔尔上同调与 $\ell$-进上同调(Cohomologie étale et $\ell$-adique,Étale cohomology and $\ell$-adic cohomology)。  
7. 模体与模体伽罗瓦群(Motifs et groupe de Galois motivique,Motives and motivic Galois group)(格罗滕迪克的 $\otimes$-范畴($\otimes$-catégories de Grothendieck,Grothendieck’s $\otimes$-categories))。  
8. 晶体与晶体上同调(Cristaux et cohomologie cristalline,Crystals and crystalline cohomology),德拉姆系数瑜伽(Yoga "coeffi cients de De Rham",Yoga of "de Rham coefficients")、霍奇系数("coeffi cient de Hodge","Hodge coefficients")……  
9. “拓扑代数”("Algèbre topologique","Topological algebra"):$\infty$-场($\infty$-champs,$\infty$-fields)、导出器(dérivateurs,derivators);拓扑斯的上同调形式,作为一种新同伦代数的灵感。  
10. 适度拓扑(Topologie modérée,Moderate topology)。  
11. 阿纳贝利代数几何瑜伽(Yoga de géométrie algébrique anabélienne,Yoga of anabelian algebraic geometry),伽罗瓦-泰希穆勒理论(Théorie de Galois-Teichmüller,Galois-Teichmüller theory)。  
12. 正则多面体及各类正则构形的“概形”或“算术”观点(Point de vue "schématique" ou "arithmétique","Schematic" or "arithmetic" viewpoint)。

除了第一个主题——其重要部分属于我的博士论文(1953 年)并在 1950 至 1955 年间的泛函分析时期得到发展——其余十一个主题是在我作为几何学家的时期,从 1955 年起逐渐浮现的。}。  
理解我的数学家生涯,感受它,至少需要在一定程度上看到并“感知”这些思想,以及它们引入的构成作品脉络与灵魂的伟大主题。

不可避免地,其中一些思想比其他思想“更伟大”(因而其他思想相对“较小”!)。换句话说,在这些新主题中,有些比其他主题更广阔,有些则更深入数学事物奥秘的核心 \footnote{在这些主题中,就其影响范围而言,最广阔的似乎是拓扑斯(topos,topos)主题,它提供了代数几何(géométrie algébrique,algebraic geometry)、拓扑学(topologie,topology)和算术(arithmétique,arithmetic)的综合思想。目前就其引发的扩展广度而言,最广阔的是概形(schémas,schemes)主题。(参见第 20 页 (*) 脚注的相关说明。)它为其他八个主题(即除第 1、5、10 外的所有主题)提供了“卓越”的框架,同时为中心概念提供了彻底革新代数几何及其代数-几何语言的基础。

在另一端,十二个主题中的第一个和最后一个在我看来比其他主题的规模更 скром скром。然而,对于最后一个主题,它为正则多面体和正则构形这一古老主题引入了新视角,我怀疑即使一个数学家全身心投入一生也未必能穷尽其可能性。至于第一个主题——拓扑张量积(produits tensoriels topologiques,topological tensor products),它更多扮演了一个现成新工具的角色,而非后续发展的灵感来源。尽管如此,直到近几年,我仍偶尔听到一些或多或少近期工作的回声,这些工作解决了我二十或三十年前留下的悬而未决的问题。

在我看来,这十二个主题中最深刻的,是模体(motifs,motives)主题,以及与之密切相关的阿纳贝利代数几何(géométrie algébrique anabélienne,anabelian algebraic geometry)和伽罗瓦-泰希穆勒瑜伽(yoga de Galois-Teichmüller,Galois-Teichmüller yoga)。

从工具的完善程度、由我亲自调试并在过去二十年研究中多个“前沿领域”广泛使用的角度看,概形(schémas,schemes)和埃塔尔及 $\ell$-进上同调(cohomologie étale et $\ell$-adique,étale and $\ell$-adic cohomology)这两个方面最为突出。对于一个消息灵通的数学家来说,现在几乎无疑,概形工具及其衍生的 $\ell$-进上同调工具,是本世纪少数重大成就之一,在近几代人中滋养并更新了我们的科学。}。  

其中有三个主题(在我眼中绝非次要)在我离开数学舞台后才出现,仍处于萌芽状态;“正式”来说,它们甚至不存在,因为没有任何正式出版物为其颁发出生证明 \footnote{唯一提到这三个主题的“半正式”文本是《计划草图》(Esquisse d’un Programme,Sketch of a Program),该文本于 1984 年 1 月为申请 CNRS 调动而撰写。该文本(也在《引言 3:罗盘与行囊》(Introduction 3, "Boussole et Bagages")中提及)原则上将收录于《反思》(Réflexions,Reflections)第四卷中。}。  

在我离开前出现的九个主题中,最后三个在我离开时正处于蓬勃发展状态,但由于我走后缺乏“慈爱之手”照料这些“孤儿”的必需,它们至今仍处于幼年状态,在一个敌对的世界中被遗弃 \footnote{在我离开后的第二天,这三个孤儿被悄无声息地埋葬。然而,其中两个在 1981 年和次年被大张旗鼓地挖掘出来,未提及原作者,且操作毫无瑕疵。}。  

至于另外六个在我离开前的二十年中达到完全成熟的主题,可以说(除了一两个例外 \footnote{“几乎完全”主要涉及格罗滕迪克的对偶性瑜伽(yoga grothendieckien de dualité,Grothendieck duality yoga)(导出范畴与六运算)以及拓扑斯(topos,topos)。这些将在《收获与播种》(Récoltes et Semailles,Harvests and Sowings)的第二部分和第四部分(《葬礼(1)》和《葬礼(3)》)中详细讨论。}),它们当时已进入共同遗产:尤其在几何学家群体中,如今“每个人”整天随时随地吟唱它们,甚至不自知(就像约丹先生(Monsieur Jourdain,Monsieur Jourdain)无意中创作散文一样)。它们已成为人们“做几何”、或做带几何色彩的算术、代数或分析时呼吸的空气的一部分。

我作品中的这十二个主导主题绝非彼此孤立。在我看来,它们属于一种精神与意图的统一体,如同贯穿我所有“已写”和“未写”作品的一道持久的基调。在写下这些文字时,我似乎再次听到了这道音符——如一声召唤!——在三年的“无偿”、顽强而孤独的工作中回响,那时我还未在意世上是否还有其他数学家,只因我被那召唤我的东西深深吸引……

这种统一不仅仅是一个工匠在其作品上留下的印记。这些主题之间通过无数微妙而显见的联系相互连接,就像在一场宏大的对位中,彼此清晰可辨的主题展开并交织——在一种和谐中汇聚,推动它们向前,并赋予每个主题意义、动态和充实,所有其他主题都参与其中。每个局部主题似乎从这更广阔的和谐中诞生,并在每个瞬间不断重生,而这种和谐远非这些主题的“总和”或“结果”,这些主题并非先于它而存在。说实话,我无法摆脱这样一种感觉(或许有些荒诞……),在某种意义上,正是这种尚未显现但确实“已存在”的和谐,隐藏于尚未诞生的事物幽暗深处——正是它依次唤起了这些主题,这些主题只有通过它才获得完整意义,也正是它在我炽热的孤独岁月、刚脱离青春期时,以低沉而急切的声音召唤着我……

无论如何,我作品中的这十二个主导主题,像是受某种隐秘的宿命驱使,共同谱写了一场交响乐——或者换个比喻,它们体现为多个不同的“观点”,共同汇聚成一个宏大而统一的愿景。

这个愿景直到 1957、1958 年——那些孕育激烈的岁月——才开始从迷雾中浮现,显露出可辨的轮廓 \footnote{1957 年是我提出“黎曼-罗赫”(Riemann-Roch,Riemann-Roch)主题(格罗滕迪克版)的一年,这一主题一夜之间让我成为“耀眼明星”。这一年也是我母亲去世的一年,因此是我生命中一个重要转折点。这是我生命中最具创造力的一年,不仅在数学层面。十二年来,我的全部精力都投入到数学工作中。那一年,我开始感到自己大致“穷尽”了数学工作的内涵,或许是时候投入其他事物了。这显然是我生命中第一次浮现的内在更新的需求。当时我考虑成为作家,数月间停止了所有数学活动。最终,我决定至少将已着手进行的数学工作记录下来,或许只需几个月,最多一年……

显然,当时时机尚未成熟,无法迈出那大步。总之,一旦我重拾数学工作,它便重新占据了我。此后十二年,它再未放手!

接下来的 1958 年,或许是我数学家生涯中最丰饶的一年。这一年,新几何的两个核心主题绽放:概形理论(théorie des schémas,theory of schemes)强势起步(成为我当年夏天在爱丁堡国际数学家大会上的报告主题),以及“位点”(site,site)概念的出现,这是拓扑斯(topos,topos)这一关键概念的技术性初版。回顾近三十年后,我可以说,这一年是新几何愿景真正诞生的一年,伴随着新几何的两个主导工具:概形(schémas,schemes,旧“代数簇”(variété algébrique,algebraic variety)概念的蜕变)和拓扑斯(topos,topos,对“空间”(espace,space)概念更深刻的蜕变)。}。  
奇怪的是,这个愿景对我而言如此贴近、如此“显而易见”,以至于直到一年前 \footnote{我第一次考虑为这个愿景命名是在 1984 年 12 月 4 日的反思中,在注释“阴之仆人(2)——或慷慨”(Yin le Serviteur (2) - ou la générosité,Yin the Servant (2) - or Generosity)的子注释($\mathrm{n}^{\circ} 136_{1}$,《收获与播种》第三部分,第 637 页)。},我从未想过为它命名。(尽管我一直热衷于为展现在我面前的事物命名,作为理解它们的第一步……)确实,我无法指出一个具体时刻,作为这个愿景出现的瞬间,或回顾时能辨识的时刻。新愿景是如此宏大,其出现恐怕无法定位于某一刻,而需在漫长岁月,甚至数代人中,逐渐渗透并占据那些凝视与沉思者的内心;仿佛新的眼睛必须在熟悉的旧眼中艰难形成,逐渐取而代之。而且,这愿景太过广阔,无法像抓住路边乍现的普通概念那样“把握”它。因此,毫不奇怪,直到它完全成熟、有了距离回顾时,我才想到为如此宏大、贴近又弥散的事物命名。

说实话,直到两年前,我与数学的关系(除了教学任务外)仅限于“做”数学——跟随一股不断推我向前的冲动,奔向那吸引我的“未知”。我从未想过停下这股冲动,哪怕一刻,去回望走过的路,或定位一个已完成的作品。(无论是将其置于我生命中,作为仍与我有深刻而长久未察联系的事物;还是将其置于“数学”这一集体冒险中。)

更奇怪的是,让我最终“停下”并重新认识这半被遗忘的作品,或仅是考虑为赋予其灵魂的愿景命名,竟需面对一场规模巨大的“葬礼”现实:通过沉默与嘲讽,对愿景及其孕育者的埋葬……