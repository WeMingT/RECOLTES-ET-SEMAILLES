\section{拓扑学——或迷雾的测绘}

我们刚刚看到,"概形"这一创新思想使得与不同素数(或不同"特征")相关联的各种"几何"得以相互联系。然而,这些几何本质上仍然是"离散"或"不连续"的,与过去几个世纪(可追溯至欧几里得)传承下来的传统几何形成对比。扎里斯基(Zariski)和塞尔(Serre)引入的新思想在某种程度上为这些几何恢复了"连续性"维度,这一维度随即被刚刚出现的"概形几何"所继承,以实现它们的统一。但就"非凡猜想"(韦伊猜想)而言,这还远远不够。从这一角度来看,这些"扎里斯基拓扑"是如此粗糙,以至于几乎仍停留在"离散聚合"阶段。显然,缺少的是某种新原则,能够将这些几何对象(或"簇",或"概形")与通常的"空间"(拓扑空间)或"优质"空间联系起来;也就是说,那些"点"明显彼此分离的空间,而在扎里斯基引入的无法无天的空间中,点往往倾向于相互粘附...

正是这样一个"新原则"的出现,而非其他,才能实现这些"数与量的联姻"或"离散几何"与"连续几何"的结合,韦伊猜想中首次预示了这一点。

"空间"的概念无疑是数学中最古老的概念之一。它在我们对世界的"几何"理解中如此基础,以至于在两千多年里一直或多或少地保持沉默。直到上个世纪,这一概念才逐渐摆脱了直接感知(我们周围唯一且相同的"空间")及其传统理论化("欧几里得")的专制束缚,获得了自身的自主性与动力。如今,它已成为数学中最普遍、最常用的概念之一,无疑为所有数学家所熟悉。此外,它还是一个多面概念,根据所纳入的结构类型呈现出千百种面貌,从最丰富的结构(如古老的"欧几里得"结构,或"仿射"与"射影"结构,或同名"簇"的"代数"结构,它们推广并柔化了这些结构)到最简化的结构:那些所有"定量"信息似乎都已消失,仅保留"邻近"或"极限"概念的精髓\footnote{谈到"极限"概念,我主要指的是"极限过程",而非(非数学家更熟悉的)"边界"概念。},以及形式直觉(称为"拓扑")的最难以捉摸的版本。在这些概念中,最简化的——在过去半个世纪里一直充当涵盖所有其他概念的广阔概念母体——是拓扑空间的概念。对这些空间的研究构成了几何学中最迷人、最活跃的分支之一:拓扑学。

尽管这种由"空间"(称为"拓扑"空间)体现的"纯粹质量"结构乍看之下似乎难以捉摸,缺乏任何定量数据(如两点之间的距离)让我们能够依附于某种熟悉的"大小"直觉,但在上个世纪,我们仍然成功地将这些空间精细地捕捉在精心"量身定制"的语言的紧密而灵活的网中。更有甚者,我们发明并制造了各种"米尺"或"测杆",尽管困难重重,仍用于为这些似乎逃避任何测量尝试的触手可及的空间(如同难以捉摸的迷雾)附加某种"度量"(称为"拓扑不变量")。确实,这些不变量中的大多数,尤其是最本质的,其性质比简单的"数字"或"量"更为微妙——它们本身就是或多或少精巧的数学结构,通过或多或少复杂的构造附加在所考虑的空间上。其中最古老且最关键的一个不变量,由意大利数学家贝蒂(Betti)在上个世纪引入,由与空间相关联的不同"群"(或"空间")组成,称为"上同调"群\footnote{实际上,贝蒂引入的不变量是同调不变量。上同调是其或多或少等价的"对偶"版本,引入时间要晚得多。这一方面在初始的"同调"方面获得了优势,尤其是(无疑)在让·勒雷(Jean Leray)引入层观点之后,下文将讨论这一点。从技术角度来看,可以说我的几何学家工作的很大一部分在于揭示并或多或少地发展所缺少的上同调理论,适用于各种空间和簇,尤其是"代数簇"和概形。在此过程中,我还重新诠释了传统的同调不变量,将其转化为上同调术语,从而使其呈现出全新的面貌。

拓扑学家还引入了许多其他"拓扑不变量",以捕捉拓扑空间的这种或那种性质。除了空间的"维度"和(上)同调不变量外,第一批其他不变量是"同伦群"。我在1957年引入了另一个不变量,即(称为"格罗滕迪克"群的)$K(X)$群,它立即获得了巨大成功,其重要性(无论在拓扑学还是算术中)不断得到证实。

在我的"温和拓扑"计划中,预见到了一大批比目前已知和使用的不变量更为微妙的新不变量,但我认为它们是基础的。(该计划的简要概述见《纲领草案》,将收录于《反思》第四卷。)该计划基于"温和理论"或"温和空间"的概念,它有点像拓扑斯概念,是"空间"概念的(第二次)"蜕变"。在我看来,它比后者更为明显且不那么深刻。我预计它对"纯粹"拓扑学的直接影响将更为显著,并将通过深刻改变几何拓扑学家工作的概念背景,彻底改变他们的"职业"。(正如在代数几何中引入概形观点一样。)我已将我的《草案》寄给了几位老朋友和著名拓扑学家,但似乎没有引起任何人的兴趣。}。正是这些不变量(主要在"字里行间")介入韦伊猜想,赋予其深刻的存在理由,并(至少对我而言,在塞尔的解释下"浸入其中")赋予其全部意义。但将这些不变量与这些猜想中涉及的"抽象"代数簇相关联的可能性,以满足这一事业所需的非常精确的要求——这只是一个希望。我怀疑除了塞尔和我之外,没有人真正相信这一点(甚至,尤其是安德烈·韦伊本人!\footnote{矛盾的是,韦伊对上同调形式体系有一种顽固的、似乎是本能的"抵触"——而正是他的著名猜想在很大程度上激发了从1955年开始的代数几何中伟大上同调理论的发展(塞尔以其基础文章FAC打响了第一枪,前文脚注中已提及)。

在我看来,这种"抵触"是韦伊对所有"大杂烩"、对任何类似于形式体系(当它无法用几页纸概括时)或任何稍微复杂的"构造"的普遍厌恶的一部分。他当然不是"建造者",显然是在不情愿的情况下,在三十年代被迫发展了"抽象"代数几何的第一个基础,这些基础(鉴于这种倾向)对用户来说成为了真正的"普罗克拉斯提斯之床"。

我不知道他是否因为我走得更远,并致力于建造广阔的殿堂,使得克罗内克(Kronecker)和他的梦想得以体现在精致而有效的语言和工具中,而对我有所不满。无论如何,他从未对我所从事的工作或已完成的工作发表过任何评论。我也没有收到他对《收获与播种》的任何回应,我在三个多月前寄给了他,并附上了我亲笔的热烈题词。})...

不久之前,我们对这些上同调不变量的理解因让·勒雷(Jean Leray)的工作(在战争期间,四十年代上半叶,他在德国被囚禁期间继续进行)而得到了极大的丰富和更新。关键的创新思想是空间上的(阿贝尔)层概念,勒雷将其与一系列相应的"上同调群"(称为"以该层为系数")相关联。这就像我们迄今为止用于"测绘"空间的标准"上同调米尺"突然倍增为数量难以想象的新"米尺",具有所有可想象的大小、形状和物质,每一种都紧密适应于所考虑的空间,每一种都为我们提供了关于该空间的完美精确信息,且只有它才能提供这些信息。这是我们对各种空间方法的深刻转变中的核心思想,无疑是本世纪出现的最关键思想之一。主要得益于让-皮埃尔·塞尔(Jean-Pierre Serre)的后续工作,勒雷的思想在出现后的十年内,首先在拓扑空间理论(尤其是与其密切相关的"同伦"不变量)中取得了令人印象深刻的重新启动,并在所谓的"抽象"代数几何中取得了同样重要的重新启动(塞尔的基础文章"FAC"于1955年发表)。我从1955年开始的几何工作与塞尔的这些工作一脉相承,因此也与勒雷的创新思想紧密相连。
