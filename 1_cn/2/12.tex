\section{“拓扑学”——或迷雾中的丈量}

我们刚刚看到,“概形”(schéma,scheme)的创新思想在于,它连接了与不同素数(或不同“特征”)相关的各种“几何”。然而,这些几何各自本质上仍是“离散”或“不连续”的,与过去数世纪传承下来的传统几何(追溯至欧几里得(Euclide,Euclid))形成鲜明对比。扎里斯基(Zariski,Zariski)和塞尔(Serre,Serre)引入的新思想,在某种程度上为这些几何恢复了“连续性”的维度,这一特性随即被新出现的“概形几何”(géométrie schématique,schematic geometry)继承,以统一它们。然而,对于韦伊(Weil,Weil)的“奇幻猜想”(conjectures fantastiques,fantastic conjectures),这仍远远不足。从这个角度看,“扎里斯基拓扑”(topologies de Zariski,Zariski topologies)过于粗糙,几乎像是仍停留在“离散聚合”的阶段。显然,缺少的是某种新原则,能将这些几何对象(或“簇”(variétés,varieties)、“概形”)与常规的“拓扑空间”(espaces topologiques,topological spaces)——即“正统”空间——联系起来。在这些正统空间中,“点”彼此清晰分离,而在扎里斯基引入的桀骜不驯的空间中,点却有令人困扰的黏连倾向……

显然,正是这样一个“新原则”的出现——且非次要——才能真正促成“数”(nombre,number)与“量”(grandeur,magnitude)、“离散几何”与“连续几何”的“联姻”,其最初预感已从韦伊猜想中浮现。

“空间”(espace,space)的概念无疑是数学中最古老的概念之一。它在我们“几何”认知世界中如此根本,以至于两千多年来一直或多或少隐而不显。仅在过去一个世纪中,这一概念才逐渐挣脱即时感知(单一包围我们的“空间”)的专制束缚及其传统(“欧几里得”)理论化,获得自身的独立性与动态。如今,它是数学中最普遍、最常用的几个概念之一,恐怕没有哪位数学家不熟悉。它还是一个多形概念,根据附着于空间的结构类型,呈现千姿百态:从最丰富的结构(如古老的“欧几里得结构”(structures euclidiennes,Euclidean structures)、“仿射结构”(structures affines,affine structures)和“射影结构”(structures projectives,projective structures),或其推广与柔化的“代数结构”中的“簇”),到最简朴的结构——一切“量的”信息似乎无迹可寻,仅剩“邻近”(proximité,proximity)或“极限”(limite,limit)概念的质性精髓 \footnote{说到“极限”,我这里主要指“趋向极限”(passage à la limite,passing to the limit),而非非数学家更熟悉的“边界”(frontière,boundary)。},以及形式直觉的最飘忽版本(即“拓扑”形式)。在这些概念中,最简朴的——过去半个世纪中作为包容所有其他概念的广阔概念母体——是“拓扑空间”(espace topologique,topological space)。研究这些空间构成了几何学中最迷人、最活跃的分支之一:拓扑学(topologie,topology)。

尽管乍看之下,这种由“拓扑空间”体现的“纯质”结构看似飘忽,因缺乏任何量的信息(如两点间距离)而无法依托我们熟悉的“大小”直觉,但在过去一个世纪中,人们已通过精心“量身定制”的严密而灵活语言,精妙地捕捉这些空间。更妙的是,人们从无到有发明并打造出种种“尺”或“丈”,不顾一切地为这些看似如迷雾般不可捉摸的庞大“空间”附上某种“度量”(称为“拓扑不变量”(invariants topologiques,topological invariants))。诚然,这些不变量中大多数,尤其是最核心的,其本质远比单纯的“数”或“量”微妙——它们本身是或多或少精致的数学结构,通过或多或少复杂的构造附着于所考察的空间。其中最古老、最关键的不变量之一,由意大利数学家贝蒂(Betti,Betti)于上世纪引入,是与空间关联的多个“上同调群”(groupes de cohomologie,cohomology groups)或“空间” \footnote{实际上,贝蒂引入的是同调(homologie,homology)不变量。上同调是其大致等价的“对偶”版本,引入时间晚得多。这一面向之所以后来居上(尤其在让·勒雷(Jean Leray,Jean Leray)引入层(faisceaux,sheaves)观点后),可能因其技术优势。从技术角度看,我作为几何学家的大部分工作在于发掘并深入发展各类空间与簇——尤其是“代数簇”(variétés algébriques,algebraic varieties)与概形——所需的上同调理论。在此过程中,我也用上同调术语重新诠释传统同调不变量,使其焕然一新。

拓扑学家还引入了许多其他“拓扑不变量”,以捕捉拓扑空间的各类特性。除“维数”(dimension,dimension)与(上)同调不变量外,首批其他不变量是“同伦群”(groupes d’homotopie,homotopy groups)。我在 1957 年引入另一个不变量——“格罗滕迪克群”(groupe de Grothendieck,Grothendieck group)$K(X)$,它立即大获成功,其在拓扑学与算术中的重要性持续得到确认。

我还在“适度拓扑”(topologie modérée,moderate topology)计划中预见了一批新不变量,比现有已知不变量更微妙但在我看来更根本。其粗略草图见《计划草图》(Esquisse d’un Programme,Sketch of a Program),将收录于《反思》(Réflexions,Reflections)第四卷。该计划基于“适度理论”(théorie modérée,moderate theory)或“适度空间”(espace modéré,moderate space)概念,类似于拓扑斯(topos,topos),是“空间”概念的(第二次)蜕变。它比拓扑斯更显而易见(我认为),但不如后者深刻。我预见其对“狭义拓扑学”的直接影响将更为显著,将通过深刻转变几何拓扑学家工作的概念框架,彻底革新其“技艺”。(如同概形观点引入代数几何时的情况。)我曾将《计划草图》寄给几位老友及著名拓扑学家,但似乎未能引起任何兴趣。}。正是这些不变量(多隐于字里行间,诚然如此)构成了韦伊猜想的深层“存在理由”,赋予其完整意义(至少对我而言,在塞尔解释的“浸润”下)。但能否将此类不变量关联至猜想中的“抽象代数簇”,以满足其苛刻需求,这仅是希望。我怀疑除塞尔与我之外,无人真正相信(甚至韦伊本人尤其不信!\footnote{矛盾的是,韦伊对上同调形式主义有种顽固、近乎本能的“障碍”——尽管他的著名猜想在很大程度上启发了 1955 年起代数几何中宏大上同调理论的发展(以塞尔 1955 年基础性文章 FAC(《相干代数层》(Faisceaux algébriques cohérents,Coherent Algebraic Sheaves))为开端,已在前注提及)。

我认为,这一“障碍”是韦伊对一切“繁琐杂物”的普遍厌恶的一部分,反感任何无法浓缩于几页的形式主义或稍显复杂的“构造”。他显然不是“建造者”,在三十年代发展“抽象”代数几何初步基础时,显然违背其意愿,这些基础对他而言(鉴于此性情)成了名副其实的“普洛克路斯忒斯之床”(lit de Procruste,Procrustean bed)。

我不知他是否因我超越其框架、投身建造宏大居所而心生怨意——这些居所让克罗内克(Kronecker,Kronecker)与他的梦想化为精妙有效的语言与工具。无论如何,他从未对我从事或完成的工作发表只言片语。我三个多月前寄给他《收获与播种》(Récoltes et Semailles,Harvests and Sowings),附上我手写的热情献词,也未收到任何回音。})……

不久前,让·勒雷在战时德国 captivity 中(四十年代前半期)继续的研究,深刻丰富并更新了我们对上同调不变量的理解。其核心创新思想是空间上的(阿贝尔)层(faisceau abélien,abelian sheaf),勒雷为之关联了一系列对应的“上同调群”(groupes de cohomologie,cohomology groups,称“以此层为系数”)。这仿佛将我们此前丈量空间的单一“标准上同调尺”骤然化为无数新“尺”,大小、形态、材质各异,每一把都与空间亲密贴合,各以独有方式传递精确信息。这是本世纪最关键的思想之一,深刻转变了我们对各类空间的理解。尤其通过让-皮埃尔·塞尔后续工作,勒雷思想在问世后的十年内初结硕果:一是拓扑空间理论(特别是其“同伦”不变量,与上同调密切相关)的惊人重启;二是“抽象”代数几何的同样重要的重启(以塞尔 1955 年基础性文章 FAC 为标志)。我自 1955 年起的几何工作延续了塞尔的研究,从而也承接了勒雷的创新思想。