\section{对岸一瞥}

在我看来,这种情况与本世纪初爱因斯坦相对论理论出现时的情况非常相似。当时存在一个更为明显的概念死胡同,表现为一个突然的矛盾,似乎无法解决。理所当然地,将秩序带回混乱的新思想是一个极其简单的想法。值得注意的是(并且符合一个非常重复的剧本...),在所有那些突然紧张起来试图"挽救局面"的杰出、卓越、有声望的人中,没有人想到这个想法。必须是一个不知名的年轻人,刚从学生讲堂的座位上走出来(也许对自己的大胆感到有些尴尬...),向他的杰出前辈们解释如何"挽救现象":只需要不再将空间与时间分开\footnote{当然,作为对爱因斯坦思想的描述,这有点简短。在技术层面上,必须明确在新时空中放置什么结构(尽管麦克斯韦理论和洛伦兹的思想已经"在空气中")。这里的关键步骤不是技术性的,而是"哲学性的":意识到远距离事件的同时性概念没有任何实验现实。这就是"幼稚的观察","但皇帝是赤裸的!",它突破了那个著名的"限制宇宙的无形而专制的圆圈"...}。从技术上讲,当时一切条件都已具备,让这个想法得以开花结果并被接受。爱因斯坦的前辈们能够接受这个新想法,而没有过多责备,这是他们的荣誉。这表明那仍然是一个伟大的时代...

从数学的角度来看,爱因斯坦的新想法是平凡的。但从我们对物理空间的概念来看,这是一个深刻的突变和突然的"异化"。这是自2400年前欧几里得提出的物理空间数学模型以来的第一次此类突变,自古代以来(包括牛顿),所有物理学家和天文学家都为了力学的需要而原封不动地采用了这个模型,以描述地球和恒星的力学现象。

爱因斯坦的这个初始想法后来得到了极大的深化,体现在一个更微妙、更丰富、更灵活的数学模型中,借助了已有的丰富数学概念\footnote{主要是"黎曼流形"的概念,以及在这种流形上的张量计算。}。随着"广义相对论",这个想法扩展为一个广阔的物理世界视野,将无限小的亚原子世界、太阳系、银河系和遥远的星系,以及电磁波在每一点都被存在的物质弯曲的时空中的传播,都纳入同一个视野\footnote{这个模型与欧几里得(或牛顿)的时空模型,以及爱因斯坦的最初模型("狭义相对论")最显著的区别之一是,时空的整体拓扑形式仍然是不确定的,而不是由模型本身的性质强制规定的。作为一个数学家,我认为确定这种整体形式的问题是宇宙学中最迷人的问题之一。}。这是宇宙学和物理学史上第二次(也是最后一次)出现一个广阔的统合视野,用数学模型的语言描述宇宙中所有物理现象,继三个世纪前牛顿的第一次伟大综合之后。

爱因斯坦的物理宇宙视野后来也被事件所超越。自本世纪初以来,"所有物理现象"的范围已经大大扩展!出现了大量的物理理论,以或多或少成功地解释所有"观察到的现象"的巨大混乱中的有限事实。我们仍在等待那个大胆的孩子,他会在玩耍中找到新的钥匙(如果有的话...),那个梦想中的"蛋糕模型",能够"运行"以同时挽救所有现象...\footnote{有人将这种假设的理论称为"统一理论",它能够"统一"并调和前面提到的众多部分理论。我认为,等待进行的基本反思必须在两个不同的层面上展开。

$1^{\circ}$)一种"哲学性"的反思,关于"数学模型"概念本身,用于描述现实的一部分。自牛顿理论的成功以来,物理学家们已经默认存在一个数学模型(甚至是一个唯一的模型,或"那个"模型)来完美表达物理现实,没有"脱节"或瑕疵。这个共识,已经存在了两个多世纪,就像毕达哥拉斯"万物皆数"的生动视野的化石遗迹。也许这就是新的"无形圆圈",取代了旧的形而上学圆圈,限制了物理学家的宇宙(而"自然哲学家"的种族似乎已经彻底灭绝,被计算机轻松取代...)。只要稍作停留,就会清楚地看到这个共识的有效性并不明显。甚至有一些非常严肃的哲学理由,导致人们先验地怀疑它,或者至少预见到它的有效性有非常严格的限制。现在是时候对这个公理进行严格的批判,甚至可能"证明"它没有基础:不存在一个严格的唯一数学模型,能够解释迄今为止记录的所有"物理"现象。

一旦"数学模型"的概念及其"有效性"(在允许的"误差范围"内)得到满意的界定,"统一理论"或至少一个"最优模型"(在某种意义下)的问题将最终明确提出来。同时,我们可能会更清楚地了解选择这样一个模型所附带的(也许是必要的)任意程度。

$2^{\circ}$)只有在这样的反思之后,我认为,提出一个比其前辈更令人满意的明确模型的"技术"问题才会真正有意义。那时,也许是时候摆脱物理学家的第二个默认公理了,这个公理可以追溯到古代,深深植根于我们对空间的感知方式:即空间和时间(或时空)的连续性,也就是"物理现象"发生的"场所"。

大约十五或二十年前,我在翻阅黎曼的完整作品集时,被他的一句"顺便一提"的话所震撼。他指出,空间的最终结构可能是"离散的",而我们对它的"连续"表示可能是一种(也许长期来看是过度的...)简化,是对更复杂现实的简化;对人类思维来说,"连续"比"离散"更容易把握,因此它作为一种"近似"帮助我们理解离散。这是一个令人惊讶的深刻见解,尤其是出自一位数学家之口,当时欧几里得的物理空间模型还从未受到质疑;从严格的逻辑意义上讲,传统上"离散"是技术性地接近"连续"的方式。

过去几十年的数学发展也显示了连续和离散结构之间比本世纪前半叶所想象的更为密切的共生关系。无论如何,找到一个"令人满意"的模型(或者,如果需要,一组这样的模型,尽可能"无缝"地连接...),无论它是"连续"的、"离散"的还是"混合"性质的——这样的工作肯定会涉及大量的概念想象力,以及把握和揭示新型数学结构的敏锐直觉。在我看来,这种想象力或"直觉"在物理学家中非常罕见(爱因斯坦和薛定谔似乎是少数例外),甚至在数学家中也是如此(我在这里说的是有充分根据的)。

总之,我预计期待中的更新(如果它还会到来...)更可能来自一个具有数学灵魂、对物理学的重大问题有深入了解的人,而不是一个物理学家。但最重要的是,这将需要一个具有"哲学开放性"的人来把握问题的核心。这个问题绝不是技术性的,而是一个基本的"自然哲学"问题。}。

将我对当代数学的贡献与爱因斯坦对物理学的贡献进行比较,有两个原因:两者都是在"空间"概念(一个是数学意义上的,另一个是物理意义上的)发生突变的背景下完成的;两者都采取了一种统合的视野,涵盖了在此之前似乎彼此分离的大量现象和情况。我在他的作品\footnote{我绝不声称熟悉爱因斯坦的作品。事实上,我从未读过他的任何著作,对他的思想只有道听途说和非常粗略的了解。然而,我似乎能辨别出"森林",即使我从未费心去仔细观察任何一棵树...}和我的作品之间看到了明显的精神亲缘关系。

这种亲缘关系似乎并没有被明显的"实质"差异所否定。正如我之前所暗示的,爱因斯坦的突变涉及物理空间的概念,而他借鉴了已有的数学概念,从未需要扩展甚至颠覆它们。他的贡献在于从当时已知的数学结构中,找出最适合\footnote{关于"垂死"这个形容词的评论,请参见前面的脚注(第55页的注释)。}作为物理现象世界"模型"的结构,取代他的前辈们遗留下来的垂死模型。从这个意义上说,他的作品确实是物理学家的作品,更进一步说,是牛顿及其同时代人所理解的"自然哲学家"的作品。这种"哲学"维度在我的数学作品中是缺失的,我从未被引导去思考"理想"概念构造(在数学事物的宇宙中进行)与物理宇宙中发生的现象(甚至是在心灵中展开的体验事件)之间的可能关系。我的作品是一个数学家的作品,他刻意回避了"应用"(到其他科学)的问题,或工作的"动机"和心理根源。此外,这是一个具有特殊天赋的数学家,他不断扩展其艺术基础的概念库。就这样,我不知不觉地、仿佛在玩耍中,颠覆了几何学家最基本的概念:空间(和"簇")的概念,即我们对几何生物所居住的"场所"的理解。

新的空间概念(作为一种"广义空间",但构成"空间"的点或多或少消失了)在实质上与爱因斯坦在物理学中引入的概念毫无相似之处(后者对数学家来说并不令人困惑)。然而,与薛定谔发现的量子力学进行比较是恰当的\footnote{我相信(通过各种渠道传回的回声)人们普遍认为本世纪物理学中有三次"革命"或重大变革:爱因斯坦的理论,居里夫妇发现的放射性,以及薛定谔引入的量子力学。}。在这种新力学中,传统的"质点"消失了,取而代之的是一种"概率云",根据点出现在该区域的"概率",在周围空间的不同区域中密度不同。在这种新视角下,我们感受到一种比爱因斯坦模型所体现的更深层次的突变,它不仅仅是用一个稍微宽松或更合身的类似模型取代一个有点狭窄的数学模型。这一次,新模型与传统的老模型如此不同,以至于即使是力学领域的顶尖数学家也突然感到陌生,甚至迷失(或愤怒...)。从牛顿力学过渡到爱因斯坦力学,对数学家来说,有点像从古老而亲切的普罗旺斯方言过渡到最新潮的巴黎俚语。然而,过渡到量子力学,我想,就像从法语过渡到中文。

而这些取代了昔日令人安心的物质粒子的"概率云",让我奇怪地联想到拓扑斯中那些难以捉摸的"开邻域",它们像幽灵般围绕着"虚构"的点,尽管想象力顽固地抗拒,仍然紧紧依附于它们...
