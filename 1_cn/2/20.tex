\section{对岸一瞥}

此刻的情形与我而言,恰似本世纪初爱因斯坦(Einstein)相对论诞生时的光景。那时存在着一个更为显著的概念死胡同,具体表现为突如其来的矛盾,看似无解。理所当然地,将为混沌重赋秩序的新思想,竟有着孩童般的单纯。值得玩味的是(也符合那反复上演的剧本……),在所有那些突然焦头烂额、试图"挽回残局"的杰出、卓越、声名显赫的人物中,竟无人曾萌生此念。偏偏得是个初出茅庐的无名青年(或许刚离开大学阶梯教室的长椅),带着几分自惭于胆大妄为的窘迫……来向那些 illustrious aînés( illustrious elders,显赫前辈)阐明该如何"拯救现象":只需不再将空间与时间割裂\footnote{当然,如此描述爱因斯坦的思想未免过于简略。技术层面上,关键在于揭示如何为新的时空赋予结构(其实麦克斯韦理论和洛伦兹思想已为此埋下伏笔)。这里的核心突破并非技术性质,而是"哲学性"的——意识到远距事件的同时性概念根本不具备实验现实性。正是这个"孩童般的发现",这个"皇帝其实赤身裸体!"的顿悟,让人突破了那个"限制宇宙的无形威严之圈"}。从技术角度看,当时一切条件都已成熟,静待这个思想破茧而出并被接纳。而爱因斯坦的前辈们确实不负众望地接纳了这个新思想,未加过多苛责,这为他们赢得了荣誉。此乃时代依然伟大的明证……

从数学视角观之,爱因斯坦的新思想平淡无奇。但就我们对物理空间的认知而言,这却是深刻的突变与突如其来的"陌生化"。这是自2400年前欧几里得(Euclide,Euclid)构建物理空间的数学模型以来,首次发生的范式转变——此模型被自古希腊至牛顿(Newton)的所有物理学家和天文学家原封不动地沿用,用以描述地球与星体的力学现象。

爱因斯坦的原始思想后来不断深化,借助既有数学概念的丰富武器库\footnote{主要指"黎曼流形(variété riemanienne,Riemannian manifold)"概念及相关的张量运算},具体化为更精妙、丰富且灵活的数学模型。"广义相对论"将此思想拓展为包罗万象的物理图景,将亚原子级的微观世界、太阳系、银河系与遥远星系,以及电磁波在时空中的传播轨迹尽收眼底——这个时空的每一点都因其中物质的存在而弯曲\footnote{此模型与欧几里得(或牛顿)时空模型及爱因斯坦最初模型("狭义相对论")最显著的区别在于:时空整体拓扑形态不再由模型本质强制规定,而是保持开放。作为数学家,我认为探究这种整体形态正是宇宙学最引人入胜的问题之一}。这是宇宙学与物理学史上(继三百年前牛顿的首次大综合之后)第二次也是最后一次出现这样的盛景——以数学模型语言统一描述宇宙中所有物理现象的宏大愿景。

爱因斯坦的宇宙图景终究也被新进展所超越。需要解释的"所有物理现象"自本世纪初以来已极大扩充!在浩如烟海的"观测事实"中,涌现出大量物理理论,各自以不同程度的成功解释着有限的事实集合。人们仍在等待那个胆大的孩子,在嬉戏间找到新钥匙(倘若存在……),那个梦寐以求的"完美模型(modèlegâteau,cake-model)",能一次性拯救所有现象\footnote{这种假想中的理论被称为"统一场论(théorie unitaire,unified theory)",旨在"统一"协调前文所述的各种局部理论。我认为待开展的基础性思考需在两个层面进行:

$1^{\circ}$)关于"数学模型"概念本身之本质的"哲学性"思考。自牛顿理论成功以来,"存在完美表达物理现实的数学模型(甚至唯一模型)"已成为物理学家的默认公理,这个延续两百余年的共识,犹如毕达哥拉斯"万物皆数"鲜活愿景的化石残骸。或许正是这个新的"无形之圈",取代了旧有的形而上学界限,禁锢着物理学家的宇宙(而"自然哲学家"一族似乎已彻底灭绝,被计算机轻易取代……)。只要稍作停留便会发现,这个共识的有效性绝非不证自明。甚至有严肃的哲学理由让我们先验地质疑它,或至少预见其严格局限性。现在正是对此公理进行严格批判的时机,或许还能"证明"它根本站不住脚:不存在能解释迄今所有"物理现象"的严格唯一数学模型。

只有先充分厘清"数学模型"概念及其"有效性"定义(在允许测量误差范围内),"统一场论"或至少"最优模型"问题才能被清晰提出。同时,我们或许也能更清楚地认识到此类模型选择中必然存在的任意性程度。

$2^{\circ}$)唯有完成上述思考后,在我看来,构建比前人更满意的显式模型这一"技术性"问题才获得完整意义。届时或许该摆脱物理学家的第二个默认公理——这个可追溯至古代、深深植根于我们空间感知模式的公理:时空(或时空)具有连续性,是"物理现象"发生的"场所"。

约十五二十年前,我翻阅黎曼(Riemann)那本薄薄的全集时,曾被他随口的一句评论震撼。他指出空间的终极结构可能是"离散的",我们持有的"连续"表象或许是对更复杂现实的(长期来看可能过度的……)简化;对人类心智而言,"连续"比"离散"更易把握,因此被用作理解后者的"近似"。在那个欧几里得物理空间模型尚未受质疑的时代,这位数学家竟有如此深邃的洞见。严格说来,传统上反倒是"离散"常被作为技术手段来逼近"连续"。

近几十年数学发展已表明,连续与离散结构的共生关系远比本世纪前半叶所想象的更为密切。无论如何,要找到"令人满意"的模型(或必要时一组能最优"衔接"的模型),无论其属"连续"、"离散"还是"混合"性质——这项工作必将需要超凡的概念想象力,以及把握并揭示新型数学结构的精湛直觉。此类想象力或"直觉"在我看来极为罕见,不仅存在于物理学家中(爱因斯坦与薛定谔似属凤毛麟角),即使在数学家中亦然(对此我有充分发言权)。

总之,我预见期待中的革新(倘若还会来临……)更可能来自深谙物理学重大问题的数学家,而非物理学家。但最关键的是,此人须具备把握问题核心的"哲学开放性"。这绝非技术性问题,而是关乎"自然哲学"的根本命题}。

将我对当代数学的贡献与爱因斯坦对物理学的贡献相比较,源于两点:二者都通过改变我们对"空间"(数学意义或物理意义上)的理解而实现;二者都呈现为统一的宏大视野,涵盖此前看似互不关联的大量现象与情境。我从中看到他的工作\footnote{我并非自称熟悉爱因斯坦的著作。事实上我从未读过其任何论文,对其思想仅有道听途说的模糊了解。但即便未曾细察过任何具体"树木",我仍自信能辨识出整片"森林"……}与我的工作间存在明显的精神亲缘性。

这种亲缘性丝毫不因二者"实质"的明显差异而削弱。如前所述,爱因斯坦的变革关乎物理空间概念,而他只需运用已知数学概念武器库,无需扩展或颠覆它。他的贡献在于:从当时已知数学结构中,遴选出最适合替代前辈垂死遗产的"模型"\footnote{关于"垂死"这个修饰词的评论,参见前文脚注(第55页注)}。就此而言,他的工作确属物理学家之作,更准确说是牛顿及其同代人理解的"自然哲学家(philosophe de la nature,philosopher of nature)"之作。这种"哲学维度"在我的数学工作中是缺失的——我从未需要思考数学宇宙中"理想"概念构造与物理宇宙现象(乃至心灵中发生的事件)间的可能关联。我的工作是数学家之作,刻意回避"应用"(于其他科学)问题或工作动机与心理根源问题。更准确说,这是一个不断拓展数学基础概念疆域的数学家之作。我就这样在不知不觉的嬉戏间,颠覆了几何学家最基础的概念:空间(及"流形")概念,即我们对几何存在所居"场所"的理解。

这种新空间概念(作为某种"广义空间",其中构成"空间"的点已或多或少消失)在实质上与爱因斯坦引入物理学的概念毫无相似之处(后者对数学家而言根本不足为奇)。但倒是与薛定谔(Schrödinger)发现的量子力学\footnote{据各方反馈,一般认为本世纪物理学有三次"革命"或重大突破:爱因斯坦理论、居里夫妇(Curie)发现的放射性,以及薛定谔引入的量子力学}可相提并论。在这新力学中,传统"质点"消失了,代之以某种"概率云"——其在周围空间不同区域的密度分布,反映粒子出现在该区域的"概率"。这种新视角带来的机械现象认知"突变",比爱因斯坦模型所体现的更为深刻——这不是简单地用更宽松或更合身的类似模型替换狭窄旧模型。这次的新模型与传统老模型如此迥异,即便专精力学的数学家也会突然感到陌生,甚至迷失(或愤慨……)。对数学家而言,从牛顿力学转向爱因斯坦力学,大概像从亲切的普罗旺斯方言转为时髦巴黎俚语;而转向量子力学,我想,无异于从法语改说中文。

这些取代昔日可靠物质粒子的"概率云",奇异般令我想起拓扑斯(topos,topos)中那些难以捉摸的"开邻域"——它们如缥缈幽灵般游荡,环绕着那些"虚点";而顽固的想象力仍不顾一切地紧抓着这些虚点不放……