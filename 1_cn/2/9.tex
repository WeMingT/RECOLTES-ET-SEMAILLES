
\section{形式与结构——或事物之道}

未曾预料,这篇“前言”逐渐演变成对我作品的一种正式介绍,主要面向非数学专业的读者。既然已经深入其中无法回头,我只好继续完成这些“介绍”!我试图尽力至少对前面几页中提到的那些令人惊叹的“伟大思想”(或“主题”)的实质,以及这些主导思想汇聚而成的著名“愿景”的本质,说上几句。由于无法使用任何技术性语言,我恐怕只能传递一个极其模糊的印象(如果真有什么能“传递”的话……\footnote{这幅图像必须保持“模糊”,但这并不妨碍它忠实,并且确实反映了所观察对象(即我的作品)的本质。相反,即使图像清晰,也可能被扭曲,并且只包含次要内容,完全遗漏了本质。因此,如果你能理解我对作品的看法(那么我心中的某些印象必定会“传递”给你),你可能会自豪地认为自己比我的任何学者同事都更好地把握了我作品的核心!})。

传统上,我们将宇宙中事物的“品质”或“方面”分为三类,作为数学思考的对象:它们是数\footnote{这里指的是所谓的“自然数”$0,1,2,3$等,或者(严格来说)那些通过基本运算用这些数表示的数(如分数)。这些数不像“实数”那样适合测量可能连续变化的量,例如直线上、平面内或空间中两点之间的距离。}、大小和形状。我们也可以称之为事物的“算术”方面、“度量”(或“分析”)方面和“几何”方面。在数学研究的大多数情况下,这三个方面同时存在并紧密互动。然而,通常其中一个方面会明显占主导地位。在我看来,对于大多数数学家来说,他们的基本气质是“算术家”、“分析家”还是“几何学家”是相当清楚的(对于那些了解他们或熟悉他们作品的人而言)——即使他们多才多艺,涉猎了所有可能的领域和范围。

我最初独自对测度论和积分论的思考,无疑属于“大小”或“分析”的范畴。同样,我在数学中引入的第一个新主题(在我看来,其规模不及其他十一个主题)也是如此。我通过“分析”的途径进入数学,似乎并非由于我的特殊气质,而是由于一种可以称为“偶然情况”的原因:在我中学和大学所接受的教育中,对于我追求普遍性和严谨性的思维来说,最大的空白恰恰在于事物的“度量”或“分析”方面。

1955年标志着我数学工作的一个关键转折点:从“分析”转向“几何”。我至今仍记得那种震撼的感觉(尽管完全是主观的),仿佛我离开了贫瘠而崎岖的草原,突然置身于一个“应许之地”,那里繁茂的财富无限繁衍,随处可摘、可探……而这种超乎一切度量的、压倒性的丰富感\footnote{我使用了“压倒性的,超乎一切度量”这一词组,试图尽可能传达德语“überwältigend”及其英语对应词“overwhelming”的含义。在前一句中,“震撼的感觉”这一(不恰当的)表达也应理解为带有这种细微差别:当我们面对非凡的辉煌、伟大或美丽时,内心的感受和情感突然淹没我们,以至于任何表达我们感受的企图似乎都预先被摧毁了。},在随后的岁月里不断得到证实和深化,直至今日。

这就是说,如果数学中有一件事(或许一直以来)比其他任何事物都更让我着迷,那既不是“数”,也不是“量”,而是形式。而在形式向我们展现的无数面貌中,最令我着迷且持续吸引我的,是隐藏在数学事物中的结构。

事物的结构绝不是我们可以“发明”的东西。我们只能耐心地揭示它,谦逊地认识它,“发现”它。如果在这项工作中存在创造性,如果我们有时像铁匠或不知疲倦的建筑师那样工作,那绝不是为了“塑造”或“建造”“结构”。这些结构根本不需要我们等待它们存在,它们早已存在,并且正是它们本来的样子!但为了尽可能忠实地表达我们正在发现和探索的事物,以及那些不愿轻易展现的结构,我们摸索着,或许用一种尚不成熟的语言,试图去界定它们。因此,我们不断“发明”能够越来越精细地表达数学事物内在结构的语言,并借助这种语言,逐步且从头开始“构建”那些旨在解释所把握和所见内容的“理论”。这里存在着一种持续不断的、不间断的往复运动,在事物的把握与通过不断精炼和再创造的语言表达所把握内容之间,这种运动在工作的过程中,在即时需求的持续压力下进行。

正如读者可能已经猜到的那样,这些“理论”,“凭空构建”的,也不过是之前提到的那些“美丽房屋”:我们从先辈那里继承下来的,以及我们在事物的召唤和倾听中亲手建造的。如果说我之前谈到了建造者或铁匠的“创造力”(或想象力),那么我必须补充的是,其灵魂和隐秘力量绝非来自那些傲慢地说:“我要这个,不要那个!”并乐于随心所欲做决定的人;就像一个糟糕的建筑师,在未观察和感受场地、未探明其可能性和要求之前,就已经在脑海中准备好了设计图。研究者创造力和想象力的质量,取决于其倾听事物声音的专注程度。因为宇宙中的事物从不厌倦自我表达和揭示,只要有人愿意倾听。而最美的房屋,“其中显现出工匠之爱的,并非比其它房屋更大或更高。美丽的房屋是那些忠实反映事物隐藏结构和美感的房屋。