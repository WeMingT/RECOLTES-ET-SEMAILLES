\section{独一无二——或孤独的天赋}

这次对“对面邻居”——物理学家们的短暂探访,或许能为一位对数学家世界一无所知(如同大多数人)的读者提供一个参照点。这样的读者想必听闻过爱因斯坦(Einstein,Einstein)及其著名的“第四维”(quatrième dimension,fourth dimension),甚至可能了解量子力学(mécanique quantique,quantum mechanics)。毕竟,尽管发明者未曾预料他们的发现会具体化为广岛(Hiroshima,Hiroshima)的灾难,乃至后来的军事及所谓“和平”核竞赛,物理学发现对人类世界的影响却是切实且几乎即刻的。相比之下,数学发现,尤其是所谓“纯数学”(mathématiques pures,pure mathematics)——即不以“应用”为动机的数学——的影响则不那么直接,且无疑更难捉摸。例如,我从未听说我的数学贡献“服务”于任何具体事物,比如制造某件器具。当然,我对此毫无功劳可言,但这并不妨碍我因此感到安心。一旦涉及应用,可以肯定的是,军队(以及随后的警方)总是最先将其攫取;而至于工业(即便号称“和平”),情况也未必好多少……

诚然,为了我自己的梳理,或为一位数学家读者着想,更恰当的做法是尝试通过数学史本身的“参照点”来定位我的作品,而非在别处寻找类比。这些天我一直在思考这一点,尽管我对那段历史的了解颇为模糊\footnote{自我幼年起,我就从未对历史(乃至地理)产生过多兴趣。(在《收获与播种》(Récoltes et Semailles,Harvests and Sowings)的第五部分——目前仅部分完成——中,我曾“顺带”探究了我认为自己对历史这一“部分阻塞”的深层原因。这一阻塞似乎在近几年正逐渐消解。)我在“布尔巴基圈”(cercle bourbachique,Bourbaki circle)中从前辈那里接受的数学教育,也并未改善这一状况——其中对历史的偶尔提及极为罕见。}。在之前的“漫步”中,我已提及一位数学家的“谱系”,他们的气质与我相通:伽罗瓦(Galois,Galois)、里曼(Riemann,Riemann)、希尔伯特(Hilbert,Hilbert)。若我对自身艺术的历史了解更深,或许能将这一谱系追溯得更远,或在其中插入一些我仅从传闻略知的名字。令我印象深刻的是,我不记得曾从比我更精通历史的朋友或同事的只言片语中得知,除我之外还有哪位数学家提出了众多创新想法——这些想法并非彼此零散无关,而是构成一个宏大统一视野的一部分(如同牛顿(Newton,Newton)和爱因斯坦在物理学与宇宙学中,或达尔文(Darwin,Darwin)和巴斯德(Pasteur,Pasteur)在生物学中所为)。我仅知道数学史上两个“时刻”诞生了具有广阔视野的新洞见。其一是2500年前古希腊数学作为一门科学的诞生,按我们今日理解的意义而言。其二是十七世纪微积分(calcul infinitésimal et intégral,infinitesimal and integral calculus)的诞生,以牛顿、莱布尼茨(Leibnitz,Leibniz)、笛卡尔(Descartes,Descartes)等人的名字为标志。据我所知,这两个时刻诞生的视野并非一人之功,而是那个时代的集体成果。

当然,从毕达哥拉斯(Pythagore,Pythagoras)与欧几里得的时代到十七世纪初,数学的面貌已发生巨变;同样,从十七世纪数学家创立的“微积分”到十九世纪中叶亦然。但据我所知,这两个时期——一个超过两千年,另一个三世纪——发生的深刻变化从未凝结或体现为某一部作品中表达的新视野\footnote{在写下这些文字数小时后,我突然意识到自己未曾提及布尔巴基(Bourbaki,Bourbaki)先生那部集体编写的巨著,它试图呈现当代数学的宏大综合。(在《收获与播种》第一部分中还将多次提及布尔巴基团体。)这在我看来有两重原因。

其一,这一综合仅限于对已知的大量观念与成果进行某种“整理”,并未引入其原创的新想法。若真有新意,或许在于对“结构”(structure,structure)概念给出了精确的数学定义,这一定义贯穿全书,成为宝贵的指引线索。但此想法在我看来更像一位聪颖而富想象力的词典编纂者的贡献,而非一种语言的革新要素,无法带来对现实(此处为数学事物)的全新理解。

其二,自五十年代起,随着“范畴方法”(méthodes catégoriques,categorical methods)突然涌入数学最具活力的领域——如拓扑学(topologie,topology)或代数几何(géométrie algébrique,algebraic geometry)——结构的概念已被事件超越。(因此,“拓扑斯”(topos,topos)这一概念拒绝被装进布尔巴基那显然过于狭窄的“结构之袋”!)布尔巴基在充分知情的情况下选择不涉足这一“泥潭”,由此放弃了其最初雄心:为当代数学整体提供基础与基本语言。

然而,它固定了一种语言,同时也确立了一种数学写作与研究方式的风格。这一风格最初反映(虽颇为片面)了某种精神——希尔伯特活泼而直接的遗产。在五六十年代,这一风格逐渐占据主导——既有其益处,更带来(尤其)弊端。近二十年来,它已僵化为一种纯粹表面“严谨”的刻板“准则”,昔日赋予其生命的精神似乎已无迹可寻。},然而这与物理学和宇宙学中牛顿、继而爱因斯坦在两个关键时刻实现的伟大综合颇为相似。

看来,作为一位孕育于我内心的宏大统一视野的仆人,我在从起源至今的数学史上似乎“独一无二”。抱歉若这显得过于标榜自我,超出了允许的范围!不过让我宽慰的是,我似乎辨认出一位潜在(且天赐!)的“兄弟”。我先前已提及他,作为我“气质兄弟”谱系中的首位:那便是埃瓦里斯特·伽罗瓦(Évariste Galois,Évariste Galois)。在他短暂而耀眼的一生中,我仿佛看到了一幅伟大视野的开端——正是“数与量的联姻”,在一个全新的几何视野中。我在《收获与播种》中另有提及\footnote{参见“伽罗瓦的遗产”(L'héritage de Galois,The Legacy of Galois)(《收获与播种》第一部分,第7节)。},两年前,我心中突然浮现这一直觉:在当时对我最具吸引力的数学工作中,我正在“接续伽罗瓦的遗产”。此后这一直觉虽少被提及,却在沉默中逐渐成熟。近三周来我对自己作品的回顾反思,想必也对此有所助益。我如今认为与过去某位数学家的最直接关联,正是与埃瓦里斯特·伽罗瓦的联系。不论对错,我感到我在生命中十五年里发展出的这一视野——在我离开数学舞台后的十六年间仍在我内心成熟与丰富——也是伽罗瓦若身处我位,定会不由自主发展的视野\footnote{我确信,若有伽罗瓦在,他定会走得远超于我。一则因他那无与伦比的天赋(我并未有幸分享);二则他或不会如我般,将大部分精力分散于无休止的、逐一细致整理已大致明了事物的任务……},若非早逝在一场决斗中残酷中断其壮丽激情\footnote{埃瓦里斯特·伽罗瓦(1811-1832)在二十一岁时死于决斗。据我所知,他有数部传记。我年轻时读过物理学家因费尔德(Infeld,Infeld)撰写的一部浪漫化传记,当时深受触动。}。

还有另一原因,必定也促成了我对这一“本质亲缘”的感受——这种亲缘不仅限于“数学气质”,也不仅限于作品的显著特征。在他与我的生命之间,我亦感到一种命运的亲近。当然,伽罗瓦在二十一岁时愚蠢地死去,而我已近六十,仍决意长寿。但这并不妨碍埃瓦里斯特·伽罗瓦在其生前,如同半个多世纪后的我,在官方数学世界中始终是个“边缘人”。对伽罗瓦而言,乍看之下,这一边缘性似属“偶然”,仿佛他仅是未及凭借创新观念与工作“崭露头角”。对我而言,在我数学生涯最初三年,这一边缘性源于我(或许刻意的)无知,未察觉需面对的数学家世界的存在;而自十六年前离开数学舞台后,则是自觉选择的结果。这一选择,定然引发了一种“集体无懈可击的意愿”,欲将我的名字及我所服务的视野从数学中抹去。

但超越这些偶然差异,我相信这一“边缘性”有其共同且本质的原因。这原因我并未归于历史环境,亦非“气质”或“性格”的特质(他与我的差异无疑如同人与人之间的差异一般大),更不用说“天赋”(伽罗瓦的天才显而易见,而我则相对平庸)。若真有“本质亲缘”,我认为其存在于更谦卑、更基本的层面。

我一生中仅在少数场合感受到如此亲缘。这也使我感到与另一位数学家——我的前辈克洛德·舍瓦利耶(Claude Chevalley,Claude Chevalley)——的“亲近”\footnote{我在《收获与播种》中多处提及克洛德·舍瓦利耶,尤其在“与克洛德·舍瓦利耶的相遇——或自由与善意”(Rencontre avec Claude Chevalley - ou liberté et bons sentiments,Meeting with Claude Chevalley - or Freedom and Goodwill)(《收获与播种》第一部分,第11节)及“告别克洛德·舍瓦利耶”(Un adieu à Claude Chevalley,A Farewell to Claude Chevalley)(《收获与播种》第三部分,注释第100号)中。}。我所指的联系,是一种“天真”或“纯真”,我曾有机会谈及。它表现为一种倾向(常不为周围人所喜),即以自己的双眼观察事物,而非透过某些或大或小的人类群体——因某种理由被赋予权威——慷慨提供的专利眼镜。

这种“倾向”或内在态度,并非成熟的特权,而是童年的恩赐。这是诞生时与生命一同获赠的天赋——谦卑却令人敬畏。一种常被深埋的天赋,某些人得以稍稍保留,或或许重新找回……

也可称之为孤独的天赋。