
\section{独特——或孤独的馈赠}

这次短暂地造访“对面的邻居”——物理学家们,或许能为一位读者(如同大多数人一样)对数学家的世界一无所知,但肯定听说过爱因斯坦及其著名的“第四维度”,甚至量子力学,提供一个参考点。毕竟,即便发明者们未曾预料到他们的发现会具体化为广岛事件,以及后来军事和(所谓的)“和平”原子竞赛,事实是物理学发现对人类世界有着直接且几乎立竿见影的影响。数学发现,尤其是所谓的“纯”数学(即没有“应用”动机的数学)的影响则不那么直接,且无疑更难以界定。例如,我未曾听闻我的数学贡献“服务于”任何事物,比如建造任何设备。我对此并无功劳可言,这是肯定的,但这并不妨碍我感到安心。一旦有应用出现,可以肯定的是,军方(紧随其后的是警方)会首先将其据为己有——至于工业(即便是所谓的“和平”工业),情况也不总是好得多……

对我自己而言,或者对一位数学家读者来说,更恰当的做法是通过数学史本身的“地标”来定位我的作品,而非在其他领域寻找类比。最近几天,在我对这段历史相当模糊的了解范围内,我思考了这一点\footnote{从小时候起,我就对历史(以及地理)不太感兴趣。(在《收获与播种》第五部分(仅部分完成)中,我有机会“顺便”揭示了我对历史部分“抵触”的深层原因——我相信,这种抵触在最近几年正在逐渐消退。)我在“布尔巴基圈”中接受的长辈们的数学教育,并未改善这一状况——那里偶尔的历史参考极为罕见。}。在《漫步》中,我已经有机会提到一系列与我气质相投的数学家:伽罗瓦、黎曼、希尔伯特。如果我对我的艺术史了解更多,我可能会将这一谱系追溯得更远,或在其间插入一些我仅耳闻其名的其他人物。令我印象深刻的是,我不记得有听说过,即便是通过比我更精通历史的朋友或同事的暗示,除了我自己之外,还有哪位数学家带来了众多创新思想,这些思想并非或多或少相互独立,而是作为一个统一广阔视野的一部分(如牛顿和爱因斯坦在物理学和宇宙学中,达尔文和巴斯德在生物学中所做的那样)。我只知道数学史上的两个“时刻”,其中诞生了全新的广阔视野。其中一个时刻是数学作为我们今天所理解的科学的诞生,2500年前在古希腊。另一个则是微积分和积分学的诞生,主要发生在17世纪,以牛顿、莱布尼茨、笛卡尔等人为代表。据我所知,这两个时刻诞生的视野并非一人之功,而是一个时代的集体成果。

当然,从毕达哥拉斯和欧几里得的时代到十七世纪初,数学已有足够的时间改头换面,同样地,从十七世纪数学家创立的“微积分”到十九世纪中叶,数学也经历了显著变化。但据我所知,在这两个时期——一个跨越两千多年,另一个长达三个世纪——发生的深刻变革,从未具体化或凝聚成体现在某一特定作品中的新视野\footnote{写完这几行后不久,我猛然意识到,我在此处竟未提及M. Bourbaki(集体)著作中竭力呈现的当代数学宏大综合。(在《收获与播种》的第一部分中,Bourbaki小组还将被频繁提及。)这似乎源于两个原因。

一方面,这一综合仅限于对大量已知思想和成果进行某种“整理”,并未引入自身独创的新颖理念。若说有新意,那便是对“结构”概念给出了精确的数学定义,这一概念贯穿全书,成为宝贵的指导线索。然而,在我看来,这一理念更类似于一位聪明且富有想象力的词典编纂者的工作,而非语言革新的一部分,后者能赋予现实(此处指数学事物)以全新的理解。

另一方面,自五十年代起,随着“范畴论”方法在数学某些最具活力的领域(如拓扑学或代数几何)中的突然涌现,结构的概念已被事件所超越。(例如,“拓扑斯”的概念拒绝被纳入Bourbaki的“结构之袋”,显然,这个袋子在接缝处显得过于狭窄!)Bourbaki在充分知情的情况下,决定不涉足这一“纷争”,从而放弃了其最初的目标,即为整个当代数学提供基础和基本语言。

相反,它确立了一种语言,同时,也确立了一种特定的数学写作和探索风格。这一风格最初是某种精神(希尔伯特生动而直接的遗产)的(非常片面的)反映。在五六十年代,这一风格最终确立——既有其优点,也(尤其)有其弊端。近二十年来,它已演变为一种僵化的“规范”,其“严谨”仅流于表面,昔日赋予其生命的精神似乎已一去不复返。},然而,其方式与物理学和宇宙学中牛顿及爱因斯坦在其历史关键时刻所实现的伟大综合颇为相似。

似乎,作为诞生于我内心的广阔统一愿景的仆人,我在数学史上从起源至今都是“独一无二”的。抱歉,我可能显得过于自我标榜了!但令我欣慰的是,我相信自己发现了一个潜在的(且天赐的!)兄弟。我之前曾提到过他,作为我“性情兄弟”中的第一位:那就是埃瓦里斯特·伽罗瓦。在他短暂而辉煌的一生中,我察觉到了一种伟大愿景的萌芽——即“数与量的联姻”,在一个全新的几何视角中。我在《收获与播种》\footnote{参见“伽罗瓦的遗产”(ReS I,第7节)。}中其他地方提到,两年前,我心中突然涌现出这种直觉:在我当时最为着迷的数学工作中,我正在“继承伽罗瓦的遗产”。这一直觉虽鲜少提及,却已在静默中成熟。过去三周里,我对自身作品的回顾无疑也促进了这一点。我现在认为,与过去数学家中最直接的传承关系,正是我与埃瓦里斯特·伽罗瓦之间的联系。无论对错,我觉得自己生命中十五年发展起来的这一愿景,以及在我离开数学舞台后十六年间继续成熟和丰富的这一愿景,正是伽罗瓦若在世也必定会发展的愿景\footnote{我深信,伽罗瓦若在世,其成就将远超于我。一方面,因其非凡的天赋(这是我未曾分享到的)。另一方面,他可能不会像我一样,将大部分精力分散于无尽的、精细的整理工作中,去逐步完善那些已或多或少掌握的内容...},如果他处于我的位置,且未因早逝而骤然中断那壮丽的飞跃。

还有另一个原因,无疑加深了我这种“本质亲缘”的感觉——这种亲缘不仅限于“数学性情”,也不仅限于作品的显著特征。在他的生命与我的生命之间,我也感受到了一种命运的亲缘。诚然,伽罗瓦愚蠢地死于二十一岁,而我即将步入六十岁,决心长寿。但这并不妨碍埃瓦里斯特·伽罗瓦在世时,如同我一个半世纪后一样,是官方数学世界中的“边缘人”。对于伽罗瓦,表面看来,这种边缘性似乎是“偶然的”,他只是还没来得及通过其创新思想和作品“确立自己的地位”。而在我这里,作为数学家的头三年,我的边缘性源于我对数学家世界存在的无知(或许是故意的...),我本应与之交锋;而自从十六年前离开数学舞台后,这种边缘性则是我有意选择的结果。正是这一选择,无疑引发了“无懈可击的集体意志”的报复,欲从数学中抹去我名字的所有痕迹,连同我为之服务的愿景。

然而,在这些偶然的差异之外,我认为我察觉到了这种“边缘性”的一个共同原因,我感到这是本质性的。这个原因,我并未在历史环境中看到,也不在于“性情”或“性格”的特殊性(这些无疑在他与我之间,如同人与人之间一样,存在差异),更不用说是在“天赋”的层面上(伽罗瓦的天赋显然惊人,而相比之下,我的则显得谦逊)。如果确实存在一种“本质上的亲缘关系”,我是在一个更为谦卑、更为基础的层面上看到的。

在我的一生中,我曾在极少数场合感受到这样的亲缘关系。正是通过这种亲缘,我也感到与另一位数学家——我的前辈克劳德·谢瓦莱(Claude Chevalley)——的“亲近”\footnote{我在《收获与播种》中多处提及克劳德·谢瓦莱,特别是在“与克劳德·谢瓦莱的相遇——或自由与善意”一节(ReS I 第11节),以及“告别克劳德·谢瓦莱”笔记中(ReS III,笔记第100号)。}。我想说的联系是一种特定的“天真”或“纯真”,我曾有机会谈及这一点。它表现为一种倾向(往往不为周围人所欣赏),即用自己的眼睛去看待事物,而不是通过由某个或多或少广泛的人类群体慷慨提供的、因某种原因被赋予权威的专利眼镜。

这种“倾向”,或者说这种内心的态度,并非成熟的专利,而是童年的特权。它是与生命一同降临的礼物——一份谦逊而又令人敬畏的礼物。这份礼物常常深埋心底,有些人能够或多或少地保留它,或者或许能够重新发现它……

我们也可以称之为孤独的礼物。