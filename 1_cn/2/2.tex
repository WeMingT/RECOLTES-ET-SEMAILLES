
\section{独处的重要性}

当我最终在巴黎与数学界取得联系时,已是两年之后的事了。在那里,我学到了许多东西,其中就包括我独自一人用有限资源完成的工作,几乎就是众所周知的“勒贝格测度与积分理论”。对于我向两三位明智之人提及甚至展示过手稿的这项工作,在他们看来,我似乎只是在浪费时间,重复“已知之事”。不过,我并未因此感到失望。那时,我心中尚未萌生通过所做工作获取“认可”、哪怕只是他人赞同或兴趣的念头。更何况,我的精力全被用于适应一个截然不同的环境,尤其是学习在巴黎被视为数学家基础的知识\footnote{我在《收获与播种》(ReS I)的第一部分“受欢迎的异乡人”(第9节)中简要叙述了那段略显艰难的过渡时期。}。

然而,如今回想那三年时光,我意识到它们绝非虚度。在不知不觉中,我在孤独中领悟了数学家职业的精髓——这是任何导师都无法真正传授的。无需他人告知,也无需遇见能分享我求知渴望的伙伴,我内心深处便已明白,自己是一名数学家:一个真正“做”数学的人,如同“做”爱一般。数学于我,已成为一位永远欢迎我渴望的情人。那些独处的岁月奠定了一种从未动摇的信心——无论是二十岁初到巴黎时发现自己无知之广、需学之多的震撼;还是二十多年后,我毅然决然离开数学界的动荡经历;亦或是近年来,由昔日亲密伙伴精心策划的,对我个人及作品的某种“葬礼”(预演且无瑕疵)的种种离奇桥段……

换句话说:在这些关键的年头里,我学会了独处\footnote{这一表述有些不准确。我从未需要“学会独处”,原因很简单,我童年时期从未丧失过这种与生俱来的能力,它存在于我出生之时,如同存在于每个人身上。但独自工作的这三年,让我能够按照自己自发的高标准来衡量自己,确认并重新确立了我与数学工作的关系中的自信与从容,这种自信与从容不依赖于任何共识或潮流。我将在“根源与孤独”笔记中再次提及这一点(ReS IV, n$^{\circ}$ 171$_3$,特别是第1080页)。}。我的意思是:凭借自己的理解去探索我想了解的事物,而不是依赖来自某个群体——无论我是否自认为是其中一员,或是出于其他原因赋予其权威——的明示或默示的观念与共识。在中学和大学时期,无声的共识告诉我,对于“体积”这一概念本身无需质疑,它被视为“众所周知”、“显而易见”、“没有问题”。我像对待理所当然的事情一样忽略了这一点——正如几十年前勒贝格不得不忽略的那样。正是在这种“忽略”的行为中,在本质上做自己而非仅仅是法律共识的表达,不局限于它们为我们划定的强制性圈子里——正是在这种孤独的行为中,存在着“创造”。其余的一切都是额外的。

后来,在这个接纳我的数学世界里,我有机会遇到许多人,既有前辈也有与我年龄相仿的年轻人,他们显然比我聪明得多,天赋异禀。我钦佩他们轻松学习新概念的能力,仿佛在玩耍中就能掌握,并像从摇篮时期就熟知一般运用自如——而我却感到笨拙,像只鼹鼠一样艰难地在需要学习(据说是重要的)的庞杂事物中开辟道路,感到无法把握其中的来龙去脉。实际上,我并非那种轻松通过著名竞赛、迅速掌握繁重课程的天才学生。

我那些更为出色的同学大多成为了有能力的知名数学家。然而,回顾过去三十或三十五年,我发现他们并未在我们时代的数学上留下深刻的印记。他们在既定的框架内做了一些事情,有时是美好的事情,但未曾想过触碰那个框架本身。他们在不知不觉中成为了那些无形而强制性的圈子的囚徒,这些圈子在特定环境和时代中界定了一个宇宙。要跨越这些圈子,他们需要重新找回出生时拥有的那种能力,正如我一样:独处的能力。

小孩子则毫无困难地独处。他们天生孤独,尽管偶尔的陪伴并不令他们反感,也知道在需要喝奶时向妈妈索要奶瓶。他们无需言语便知道奶瓶是为他们准备的,也知道如何喝奶。但往往,我们与内心那个孩子失去了联系。我们不断错过最好的东西,却不愿看见……

在《收获与播种》中,如果我向除了自己之外的某人倾诉,那对象并非“公众”。我向你——阅读此书的个体,且是独自一人的你——诉说。我愿与深藏于你内心、懂得孤独的那个孩子对话,仅此而已。那孩子常居远方,我深知此点。他历经沧桑,久远以来目睹了世间百态。他隐匿于无人知晓的角落,往往难以触及。人们或许会断言他早已逝去,甚至从未存在过,然而我坚信,他仍在某处,生机勃勃。

我也知晓自己被倾听的迹象。那便是,当跨越文化与命运的种种差异,我所讲述的关于自身与生活的点滴,能在你心中激起回响与共鸣;当你在其中亦寻觅到自己的生活轨迹,自我体验的片段,或许是以一种你此前未曾留意的方式。这并非对远离你的事物的“认同”。而是,或许在某种程度上,你通过我逐页在《收获与播种》中,乃至此刻我正在书写的篇章里,对自我生活的重新发现,再次邂逅了最贴近你自身的生活本质。