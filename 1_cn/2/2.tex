\section{独处的重要性}

当我最终在巴黎接触到数学界时,大约一两年后,我在众多事物中了解到,我独自在角落里用手头资源所做的工作,(大致上)正是“人人皆知”的勒贝格(Lebesgue,Lebesgue)测度与积分理论(théorie de la mesure et de l'intégrale de Lebesgue,Lebesgue's theory of measure and integration)的内容。在我向两三位审稿人提及这项工作(甚至展示手稿)时,他们的反应仿佛我只是在浪费时间,重做“已知之事”。我并不记得曾感到失望。当时,寻求“认可”、赞同或仅仅是他人对我工作的兴趣,这种想法对我而言尚属陌生。更何况,我的精力已全被适应一个截然不同的环境所占据,尤其是学习在巴黎被视为数学家必备基础的知识 \footnote{我在《丰收与播种》(Récoltes et Semailles,Harvests and Sowings)第一部分第9节“受欢迎的陌生人”(L'étranger bienvenu,The Welcome Stranger)中简述了那段略显艰难的过渡时期。}。

然而,如今回想这三年,我意识到它们绝非虚度。甚至在无意识中,我在孤独中学会了数学家职业的精髓——这是任何导师都无法真正传授的。我从未明言,也从未遇到过能分享我求知渴望的人,但我“从内心深处”知道,我是一名数学家:一个真正“做”数学的人——正如人们“做”爱一样。数学对我而言,已成为一位永远欢迎我欲望的情人。这几年的孤独奠定了一种从未动摇的信任基础——无论是在二十岁抵达巴黎时发现自己无知的广袤和需要学习的浩瀚时,还是二十多年后我毅然离开数学界时的动荡经历,抑或近几年某些“葬礼”(预先安排且毫无瑕疵)——由我昔日最亲密的同伴策划的,对我个人及其作品的“葬礼”——的疯狂插曲中。

换言之,我在那些关键岁月中学会了独处 \footnote{这种表述略显不妥。我从未需要“学习独处”,因为在童年时期,我从未失去这种与生俱来的能力——每个人出生时都具备的能力。但这三年的孤独工作,让我得以按自己的标准衡量自己,遵循我内在的自发要求,巩固了我在数学工作中的信任和宁静的自信,这种自信与主流共识和时尚无关。我在《丰收与播种》第四部分第$171_3$节“根源与孤独”(Racines et solitude,Roots and Solitude)中再次提及(尤其在第1080页)。}。我所说的独处,是指凭自己的洞察力探索我想了解的事物,而不是依赖于某个我所属或因其他原因被赋予权威的团体的明确或隐含的观念和共识。在高中和大学,沉默的共识告诉我,“体积”这一概念“众所周知”、“显而易见”、“毫无问题”,无需质疑。我置之不理,认为这是理所当然——正如几十年前勒贝格必定也曾置之不理。正是这种“置之不理”的行为——即做自己,而非仅仅是主流共识的表达,不被其划定的无形而强制的界限所束缚——构成了“创造”的核心。其余皆为附赠。

此后,在接纳我的数学界中,我遇到了许多人——前辈和同龄人——他们显然比我更聪明、更“有天赋”。我钦佩他们学习新概念的轻松自如,仿佛与生俱来,而我则感到笨拙而迟缓,像鼹鼠般艰难地在无形的知识山中穿行,面对那些据说重要的、但我无法把握来龙去脉的事物。实际上,我绝非那种轻松通过声望考试、瞬间掌握艰深课程的杰出学生。

事实上,我大多数更聪明的同学都成为了有能力和声誉的数学家。然而,三十或三十五年后回首,我发现他们并未在当代数学中留下真正深刻的印记。他们在既定的框架内完成了工作,有时是优美的工作,但从未想过触及框架本身。他们在无意识中被那些无形而强制的界限所囚禁,这些界限在特定环境和时代中划定了宇宙的范围。要跨越这些界限,他们需要重新发掘自己出生时就拥有的能力——正如我曾拥有的:独处的能力。

而幼儿则毫无困难地独处。他们天生孤独,即便偶尔享受陪伴,也会在需要时向母亲索要奶瓶。他们深知,奶瓶是为自己准备的,自己会喝。但我们常常与内心的孩子失去联系,不断错过最美好的事物,甚至不屑一顾……

如果在《丰收与播种》中,我还向其他人——而非仅仅我自己——倾诉,那并非面向“公众”。我是在向你——正在阅读的你——作为个体、作为独一无二的人倾诉。我想与你内心的那个懂得独处的人——那个孩子——对话,仅此而已。我深知,那个孩子往往遥不可及。他历经沧桑,早已藏匿于某个角落,难以触及。人们会发誓他早已死去,甚至从未存在过;然而,我确信他就在某处,鲜活如初。

我也知道,当我被倾听时,会有何种迹象。那时,超越文化和命运的差异,我对自己和生活的叙述会在你心中激起回响和共鸣;你会在其中重新发现自己的生活、自己的体验,或许是以一种你之前未曾注意的角度。这并非“认同”于某个遥远的事物或人。也许,你会通过我对自身生活的重新发现——在《丰收与播种》的页页篇章中,甚至在今天我正在书写的这些文字中——重新发现你自己的生活,那最贴近你的事物。