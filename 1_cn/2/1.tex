\section{事物的魔力}

当我还是个孩子时,我喜欢去学校。我们有一位老师教我们阅读、写作、计算、唱歌(他会拉小提琴伴奏),还讲史前人类和火的发现。我不记得那时在学校里曾经感到无聊。他拥有数字的魔力,文字、符号和声音的魔力。还有韵律的魔力,在歌曲或小诗中。韵律中似乎蕴藏着一种超越文字的神秘感。这种感觉一直持续,直到有一天,有人向我揭示了一个极其简单的“奥秘”:所谓韵律,不过是让两个连续的口语片段以相同的音节结尾,于是它们仿佛被施了魔法,化作了诗句。这对我而言是一场启示!在家时,周围的人都乐于回应我的热情,我连续数周、数月地沉浸于创作诗句的乐趣中。有一段时间,我甚至只用韵语说话。幸好,这种习惯最终消退了。然而,即便到了今日,我偶尔仍会提笔写诗——不过不再刻意追求韵脚,除非它自然流淌而来。

另一段时间,一个已上高中的年长朋友教会了我负数的概念。这又是一种有趣的游戏,但乐趣耗尽得更快。还有填字游戏——我曾连续数日、数周地制作它们,设计得越来越错综复杂。在这游戏中,形式的魔力与符号、文字的魔力交织融合。然而,这股热情最终离我而去,似乎未留下任何痕迹。

在高中时,我先在德国度过了第一年,随后到了法国。我是个好学生,但并非那种“出类拔萃的学生”。对于最吸引我的事物,我倾注无限热情;而对兴趣不大的东西,我往往置之不理,也不太在乎相关老师的评价。1940年,我在法国上高中的第一年,与母亲一同被拘禁在集中营中,地点在门德(Mende,Mende)附近的里约克罗(Rieucros,Rieucros)。那是战争岁月,我们是外国人——用当时的话说,是“不受欢迎的人”。然而,集中营的管理对营里的孩子们略为宽容,尽管他们同样被视为“不受欢迎”。我们得以相对自由地进出。我是其中年龄最大的,也是唯一前往高中就读的人,学校距营地四五公里。无论风雪交加,我都穿着凑合的鞋子步行前往,那些鞋子总是渗水。

我至今记得第一次数学“作文”的经历。老师因我在证明“三角形全等的三个条件”之一时未遵循课本的方法,给了我一个差评。他虔诚地依循教科书的范式。而我却清楚,我的证明与书上的并无优劣之分,我只是秉承其精神,运用了那些永恒的传统手法——“将某个图形以某种方式滑动至另一图形”。显然,这位教我的老师无法凭自己的洞察力判断(在这里,是一个推理的有效性)。他必须仰仗权威——在此即教科书的权威。这种态度一定深深震撼了我,以至于这个小插曲至今历历在目。此后,直至今日,我无数次见证,这种依赖权威的倾向绝非例外,而是近乎普遍的法则。关于这一点,可说的实在太多——我在《丰收与播种》(Récoltes et Semailles,Harvests and Sowings)中多次以不同形式触及这一主题。然而,即便到了现在,每当再次面对这种情形时,无论我愿不愿意,仍会感到一阵茫然失措……

战争的最后几年,母亲仍被拘于集中营,我则住在利尼翁河畔尚邦(Chambon sur Lignon,Chambon sur Lignon)的“瑞士救援”(Secours Suisse,Swiss Relief)儿童之家,那是为难民儿童设立的庇护所。我们大多是犹太人。每当当地警察警告我们盖世太保(Gestapo,Gestapo)即将搜捕,我们便两三人一组,藏进树林,度过一两夜。当时我们并未完全意识到,这实实在在关乎生死。塞文地区(pays cévenol,Cévennes region)藏匿着无数犹太人,多亏当地居民的团结互助,许多人得以幸存。

在“塞文学院”(Collège Cévenol,Cévennes College)——我读书的地方——最令我震惊的,是同学们对所学内容何其漠不关心。而我则在学年初如饥似渴地啃读课本,期待这次终于能学到真正有趣的东西;余下的学年,我尽力自谋出路,而预定课程则如流水线般无情推进,贯穿整个学期。我们倒是有几位极为友善的老师。自然历史老师弗里德尔先生(Monsieur Friedel,Mr. Friedel)的人性和智慧品质令人叹服。然而,他无法“严加管教”,课堂上被学生闹得天翻地覆。到学年末,他的无力之声完全淹没在喧嚣中,课程已无法继续。或许正因如此,我未成为生物学家!

我花了许多时间,甚至在课堂上(嘘……),钻研数学问题。很快,书上的习题已无法满足我。或许因它们久而久之过于雷同;但更重要的,我认为,是它们太过突兀地接连出现,既不说从何而来,也不指明去往何处。那是书本的问题,而非我的问题。然而,真正自然的问题从不匮乏。例如,当一个三角形三边长 \( a \)、\( b \)、\( c \) 已知时,此三角形便已确定(不计其位置),因此必存在一个明确的“公式”,如以 \( a \)、\( b \)、\( c \) 表示其面积。同理,对于已知六条边长的四面体,其体积如何求解?这次我费尽心思,但最终坚持了下来,找到了答案。无论如何,当某件事“攫住”我时,我从不计较投入的时日,甚至忘却一切!(如今依然如此……)

数学书中令我最不满意的,是对长度(曲线)、面积(曲面)、体积(立体)概念缺乏严肃定义。我暗自承诺,一旦有暇,必填补这一空白。1945年至1948年间,我在蒙彼利埃大学(Université de Montpellier,University of Montpellier)求学时,将大部分精力倾注于此。大学的课程无法令我满意。我从未明言,但内心一定觉得,教授们不过在重复课本,就像我在门德高中时的第一位数学老师。因此,我甚少踏足校园,仅偶尔了解那永恒的“课程计划”。课本足以应付这计划,但显然,它们完全无视我的疑问。实话说,它们甚至看不到这些问题,正如我的高中课本同样视而不见。只要它们为众人提供计算长度、面积、体积的现成配方——用单重、双重、三重积分(谨慎避开三维以上的维度……),内在定义的问题似乎从未浮现,无论对我的教授,还是教科书作者,皆是如此。

以我当时有限的经验,似乎我是世上唯一对数学问题怀有好奇的人。至少,这是我在那几年完全的智力孤寂中未曾言明的信念,而这孤寂并未让我感到沉重。\footnote{1945年至1948年间,我与母亲住在蒙彼利埃(Montpellier,Montpellier)约十公里外的小村庄莫拉尔格(Maurargues,Maurargues),通过旺达尔格(Vendargues,Vendargues),隐于葡萄园中。(我父亲于1942年在奥斯维辛集中营(Auschwitz,Auschwitz)失踪。)我们靠我微薄的奖学金过着俭朴生活。为维持生计,我每年参与葡萄采摘,之后设法酿酒(据说违反了当时法律……)。另有一个花园,我从未耕作,却为我们提供了丰富的无花果、菠菜,甚至(临近结束时)西红柿,皆由邻居在美丽的罂粟花海中种植。那是美好生活——有时却捉襟见肘,如需更换眼镜框或一双磨穿至绳的鞋子时。幸好,母亲因长期拘禁于集中营而体弱多病,我们享有免费医疗。否则,我们永远付不起医生费用……} 实话说,我从未想过深入探究,是否真我是世上唯一对此感兴趣的人。我的精力全被自己设下的挑战所吞噬:

我毫不怀疑,只要我肯费心探究,将它们逐一诉诸笔端,便定能成功,揭开事物的终极答案。例如,对体积的直觉无可辩驳。它必是某种暂时难以捉摸却真实可触的现实之映照。我要做的,仅是抓住这现实——或许有些像那“韵律”的魔力现实,曾在某日被我抓住、“理解”。

17岁刚从高中毕业,我着手此事,以为几周即可完成。结果我为之耗费了三年。我甚至在大学二年末的考试中失手——球面三角学考试(在“深入天文学”选项中,原文如此),因一个愚蠢的数值计算错误。(须承认,离开高中后,我的计算能力从未出色……)因此,我不得不在蒙彼利埃再留一年,完成学士学位,而未即刻前往巴黎——据说那里是唯一能遇见深谙数学要义之人的地方。我的告知者苏拉先生(Monsieur Soula,Mr. Soula)还向我保证,数学中最后的问题已在二三十年前由勒贝格(Lebesgue,Lebesgue)解决。他恰巧(何其巧合!)发展了测度与积分理论(théorie de la mesure et de l'intégration,theory of measure and integration),为数学画上句点。

苏拉先生,我的“微积分”教授,对我友善且充满善意。但我不认为他说服了我。我内心一定早已感受到,数学在广度与深度上皆无止境。海洋有“终点”吗?无论如何,我从未萌生念头,去寻觅苏拉先生提及的那本勒贝格之书——他自己也未必曾亲手翻阅。在我看来,书中的内容与我以自己的方式探索、满足好奇心的工作毫无共通之处,那些事物曾深深吸引着我。