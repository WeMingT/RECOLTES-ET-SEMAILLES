
\section{事物的魔力}

小时候,我很喜欢上学。我们有一位老师教我们读写、算术、唱歌(他会用小提琴为我们伴奏),还有史前人类和火的发现。我记得那时候在学校从未感到无聊。数字有魔力,文字、符号和声音也是。歌曲或小诗中的韵律同样如此。韵律似乎蕴含着超越文字的神秘。直到有一天,有人向我解释了一个简单的“窍门”:韵律不过是让连续的两个口语动作以相同的音节结尾,这样一来,仿佛被施了魔法,它们就变成了诗句。这真是一个启示!在家里,我找到了共鸣,连续几周甚至几个月,我都乐此不疲地创作诗句。有一段时间,我说话都带着韵律。幸好,这种状态过去了。但即使到了今天,偶尔我还会写诗——只是不再刻意追求韵律,除非它自然而来。

还有一次,一个已经上高中的大朋友教我负数。那是另一个有趣的游戏,但很快就玩腻了。还有填字游戏——我花了几天甚至几周时间制作,越来越复杂。这个游戏结合了形状的魔力,以及符号和文字的魔力。但这种热情已经离我而去,似乎没有留下任何痕迹。

在高中,先在德国读了一年,后来在法国,我是个不错的学生,但算不上“优秀学生”。我对最感兴趣的东西投入无限热情,而对不太感兴趣的东西则倾向于忽视,不太在意相关“老师”的评价。1940年,我在法国的第一年高中,和母亲一起被关在里厄克罗斯集中营,靠近芒德。那时是战争时期,我们是外国人——所谓的“不受欢迎者”。但集中营的管理者对营里的孩子们睁一只眼闭一只眼,尽管他们也是“不受欢迎者”。我们可以随意进出。我是年纪最大的,也是唯一一个上高中的,离集中营四五公里,无论下雪还是刮风,我都穿着总是进水的破鞋去上学。

我还记得第一次“数学作文”,老师因为我对“三角形全等的三种情况”之一的证明给了我很低的分数。我的证明不是书上的那种,而他严格遵循书本。然而,我清楚地知道,我的证明并不比书上的逊色,我遵循的是同样的思路,用那些传统的“将某个图形以某种方式滑动到另一个图形上”的方法。显然,这位教我的老师觉得自己没有能力独立判断(在这里,是一个推理的有效性)。他必须依赖权威,在这里就是书本。这种态度一定给我留下了深刻印象,以至于我至今还记得这个小插曲。后来直到今天,我有很多机会看到,这种态度绝非例外,而是几乎普遍的规则。关于这一点,有很多可以说的——我在《收获与播种》中多次以不同形式触及这个话题。但即使到了今天,无论我愿不愿意,每当我再次遇到这种情况时,我仍然感到困惑……
战争最后几年,当我母亲仍被关在集中营时,我在“瑞士救助”组织为逃难儿童设立的儿童之家,位于尚邦-叙尔-利尼翁。我们大多是犹太人,每当接到(当地警方的)通知,说盖世太保要来搜捕,我们就会分成两三个人的小组,躲进树林里过上一两夜,并未真正意识到这关乎我们的生死。这个地区隐藏着许多犹太人,在塞文山区,许多人因当地居民的团结互助而得以幸存。

在“塞文学院”(我在此就读)最让我印象深刻的是,我的同学们对所学内容兴趣寥寥。而我则在新学年伊始便如饥似渴地阅读课本,以为这次终于能学到真正有趣的东西;而一年中的其余时间,我尽力利用时间,而预定的课程却无情地按部就班,贯穿整个学期。尽管如此,我们有些老师非常和蔼可亲。自然历史老师弗里德尔先生,其人文与智力素养非凡。但他无法“严厉”,结果被学生们闹得不可开交,到了学年末,他的课几乎无法继续,他那无助的声音被一片喧嚣淹没。或许正因为如此,我才没有成为生物学家!

我花了不少时间,甚至在课堂上(嘘……),做数学题。很快,书上的题目已不能满足我。也许是因为它们逐渐变得过于相似:但更重要的是,我认为,它们出现得太过突兀,一个接一个,没有说明来龙去脉。这些是书上的问题,不是我的问题。然而,真正自然的问题并不缺乏。例如,当三角形三边长度$a$、$b$、$c$已知时,这个三角形就确定了(忽略其位置),因此应该有一个明确的“公式”来表达,比如,三角形的面积作为$a$、$b$、$c$的函数。同样,对于一个已知六条棱长的四面体:体积是多少?那次我可能费了不少劲,但最终还是解决了。无论如何,一旦某件事“抓住”了我,我会不计时间地投入,甚至忘记其他一切!(现在依然如此……)

在数学课本中,最让我不满的是缺乏对长度(曲线的)、面积(表面的)、体积(固体的)概念的严格定义。我决心一旦有机会就要填补这一空白。1945年至1948年间,我在蒙彼利埃大学求学时,大部分精力都投入于此。大学的课程并不能满足我。虽然从未明说,我可能觉得教授们只是在重复他们的课本,就像我在芒德高中的第一位数学老师一样。因此,我只是偶尔去大学,了解那永恒的“课程”进展。书本足以应付所谓的课程,但显然它们无法回答我心中的疑问。实际上,它们甚至没有触及这些问题,就像我的高中课本一样。既然它们提供了计算长度、面积和体积的通用方法,通过单重、双重、三重积分(谨慎地避开了三维以上的维度……),似乎没有必要给出一个内在的定义,无论是我的教授还是教材作者都这么认为。

根据我当时有限的经验,似乎我是世界上唯一一个对数学问题充满好奇的人。无论如何,那是我在完全孤独的智力生活中未言明的信念,而这份孤独并未让我感到沉重\footnote{
    1945年至1948年间,我与母亲住在离蒙彼利埃约十公里的小村庄Maurargues(靠近Vendargues),四周环绕着葡萄园。(我父亲于1942年在奥斯威辛失踪。)我们靠我微薄的学生奖学金勉强维持生计。为了收支平衡,我每年都参与葡萄收获,之后还会制作一些残次葡萄酒,尽管过程磕磕绊绊(据说还违反了当时的法规...)。此外,我们有一个花园,无需劳作,便为我们提供了丰富的无花果、菠菜,甚至(接近尾声时)还有西红柿,这些是由一位好心的邻居在一片绚丽的罂粟花海中种植的。那是美好的生活——但有时也会捉襟见肘,比如需要更换眼镜框或一双穿到破旧的鞋子时。幸运的是,对于因长期集中营生活而饱受折磨、病痛缠身的母亲,我们有资格享受免费医疗援助。否则,我们根本无力支付医生的费用...
}说实话,我相信在那段时间里,我从未深入思考过自己是否真的是世界上唯一一个可能对我所做的事情感兴趣的人。我的精力完全被自己设定的挑战所占据:

我内心毫不怀疑,只要我肯下功夫去探究,将它们的启示一一记录在案,我必定能够揭开事物的终极奥秘。比如说,对体积的直觉是不可否认的。它只能是暂时难以捉摸但完全可触的现实的反映。正是这个现实需要被把握,简单来说——或许有点像“韵律”这一神奇现实曾被某一天“捕捉”、“理解”那样。

十七岁,刚从高中毕业,我开始着手此事时,以为几周就能完成。结果我坚持了三年。甚至在大学第二年末,由于一个愚蠢的数字计算错误,我错过了球面三角学的考试(在“深入天文学”选修课中,没错)。(必须承认,自从高中毕业后,我在计算方面一直不太在行...)正因如此,我不得不在蒙彼利埃多待一年完成我的学位,而不是直接前往巴黎——据说是唯一一个能让我遇到那些了解数学界重要事务的人的地方。我的消息来源,苏拉先生,还向我保证,数学中最后剩下的问题已在二三十年前被一个叫勒贝格的人解决了。他恰好(真是奇妙的巧合!)发展了一套测度与积分理论,为数学画上了句号。


苏拉先生,我的“微分计算”老师,是一位对我既仁慈又友善的人。尽管如此,我并不认为他说服了我。我内心早已存在一种认知,即数学在广度和深度上都是无限的。大海有“终点”吗?无论如何,我从未有过一丝念头去翻阅苏拉先生提及的那本勒贝格的书,而且他本人大概也从未真正拥有过那本书。在我心中,书中所能包含的内容与我为了满足自己对某些引发好奇事物的探索而进行的工作之间,毫无共同之处。