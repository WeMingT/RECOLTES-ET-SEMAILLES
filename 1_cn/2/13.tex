
\section{拓扑斯——双人床}

Leray引入的层观点和语言使我们以全新的视角审视各种“空间”和“流形”。然而,它们并未触及空间概念本身,只是让我们以新的眼光更细致地理解这些传统的“空间”,这些空间对所有人来说早已熟悉。但事实证明,这种空间概念不足以解释表达“抽象”代数流形(如适用于Weil猜想的那些)乃至一般“概形”(推广了旧有的流形)的“形状”的最基本“拓扑不变量”。对于期待中的“结合”,“在数量和规模上”,这就像一张显然狭窄的床,其中只有未来的伴侣之一(即新娘)勉强可以找到栖身之所,但两人同时却绝无可能!为了兑现由吉祥仙女承诺的结合,尚待发现的“新原则”,正是未来夫妇所缺少的那张宽敞的“床”,而此前竟无人察觉……

这张“双人床”随着拓扑斯的概念出现(仿佛魔杖一挥……)。这一概念在一个共同的拓扑直觉中,既包含了体现连续规模世界的传统(拓扑)空间,也包含了那些执迷不悟的抽象代数几何学家所谓的“空间”(或“流形”),以及无数其他类型的结构,这些结构此前似乎无可救药地固着于“离散”或“不连续”聚集的“算术世界”。

正是层的观点,作为无声而可靠的向导,有效的钥匙(绝非秘密),引领我毫不犹豫、直截了当地走向拥有宽阔婚床的新房。一张如此宽阔的床(宛如一条宽阔而平静的深邃河流……),以至于
\begin{quote}
    ``tous les chevaux du roi\\
    y pourraient boire ensemble...''
    \end{quote}-正如一首老歌所唱,想必你也曾唱过,或至少听过。而那位最早唱出这首歌的人,比我的任何一位昔日博学的学生和朋友,都更深刻地感受到了topos隐秘之美与宁静之力……

无论是初步且临时的探索(通过“site”这一非常便利但非本质的概念),还是对topos的探索,关键始终如一。现在,我想尝试描述topos这一概念。

考虑由给定(拓扑)空间上所有层构成的集合,或者,如果你愿意,可以想象成由所有用于丈量的“尺子”组成的庞大武库\footnote{(致数学家)实际上,这里指的是集合层,而非由Leray引入作为构成“上同调群”的最一般系数的阿贝尔层。此外,我相信我是第一个系统性地使用集合层的人(始于1955年,在我于堪萨斯大学发表的论文《A general theory of fibre spaces with structure sheaf》中)。}。我们将这个“集合”或“武库”视为配备了其最明显的结构,这种结构可以说是一目了然的;即所谓的“范畴”结构。(非数学专业的读者不必因不了解这一术语的技术含义而感到困惑,后续内容无需此知识。)正是这种被称为“层范畴”(在所考虑的空间上)的“超级丈量结构”,从今以后将被视为“体现”空间最本质的东西。这在“数学常识”中是合理的,因为事实证明,我们可以完全通过与之关联的“层范畴”(或丈量武库)来“重建”一个拓扑空间\footnote{(致数学家)严格来说,这仅适用于所谓的“sober”空间。然而,这类空间涵盖了几乎所有常见的空间,特别是分析学家钟爱的所有“分离”空间。}。(验证这一点是一个简单的练习——一旦问题被提出,当然……)这足以让我们确信,如果出于某种原因我们愿意,我们现在可以“忘记”初始空间,仅保留并使用关联的“范畴”(或“武库”),它将被视为表达“拓扑结构”(或“空间结构”)最恰当的体现。

正如数学中常见的那样,我们在此(得益于“层”或“上同调尺”这一关键思想)成功地将某一概念(此处为“空间”)用另一概念(“范畴”)来表达。每一次,将一种概念(表达某种类型的情境)翻译成另一种概念(对应另一种类型的情境)的发现,都通过特定直觉的意外汇合,丰富了我们对于这两种概念的理解。因此,一个“拓扑”性质的情境(由给定空间体现)在此被翻译为一个“代数”性质的情境(由“范畴”体现);或者说,空间所体现的“连续”被“翻译”或“表达”为具有“代数”性质的范畴结构(而此前,这种结构主要被视为“离散”或“不连续”性质的)。

但这里还有更多内容。第一个概念,即空间的概念,对我们来说似乎是一种“最大”的概念——它已经如此普遍,以至于我们难以想象如何再找到一种仍然“合理”的扩展。然而,在镜子的另一侧\footnote{这里提到的“镜子”,如同《爱丽丝梦游仙境》中的那样,是指将放置在它前面的空间的“图像”呈现为与之相关联的“范畴”,被视为空间的一种“镜像”,即“镜子的另一侧”...},从拓扑空间出发所遇到的这些“范畴”(或“武器库”)具有非常特殊的性质。事实上,它们拥有一系列高度类型化的属性\footnote{(针对数学家)这里主要涉及的是我在范畴论中引入的“精确性属性”(同时引入了现代范畴论中“归纳极限”和“投射极限”的一般概念)。参见《关于同调代数的几点》,东北数学杂志,1957年(第119-221页)。},这使得它们类似于最简单可想象的范畴的“拼贴”——即从一个仅包含一个点的空间出发所得到的范畴。也就是说,一种“新风格的空间”(或拓扑斯),它推广了传统的拓扑空间,将被简单地描述为一个“范畴”,它不一定来自普通空间,但拥有所有这些良好属性(当然,这些属性已被明确指定)的“层范畴”。

\begin{center}
    * \quad * \\
    *
    \end{center}

这就是新思想。它的出现可以被视为这一观察的结果,实际上几乎是幼稚的,即在拓扑空间中真正重要的根本不是它的“点”或点的子集\footnote{因此,可以构建非常“大”的拓扑斯,它们只有一个“点”,甚至根本没有“点”!},以及它们之间的邻近关系等,而是该空间上的层及其形成的范畴。总之,我只是将Leray的初始思想推向其最终结论——并在此过程中迈出了关键一步。

正如层的思想(归功于Leray)或概形的思想,如同任何“伟大思想”一样,它们颠覆了根深蒂固的观念,拓扑斯的思想也因其自然性、“显而易见”的特质及其简单性(可以说,近乎天真或简单,甚至“幼稚”)而令人困惑。这种特质常常让我们惊呼:“哦,原来只是这样!”语气中带着半失望半羡慕;或许还带有一种“古怪”、“不严肃”的暗示,这种暗示常常被赋予那些因出乎意料的简单性而令人困惑的事物。它或许让我们回想起那些早已被埋藏和否认的童年时光...