\section{“拓扑斯”——或双人床}

让·勒雷(Jean Leray,Jean Leray)引入的层(faisceaux,sheaves)观点与语言,使我们得以用新的光芒审视各类“空间”(espaces,spaces)和“簇”(variétés,varieties)。然而,它们并未触及“空间”概念本身,仅让我们以新颖的目光更精妙地理解那些早已为众人熟知的传统“空间”。然而事实证明,这一空间概念不足以表达“抽象代数簇”(variétés algébriques abstraites,abstract algebraic varieties,如韦伊(Weil,Weil)猜想所涉及者)乃至更广义“概形”(schémas,schemes,概化了旧式簇)的“形式”(forme,form)所依赖的最核心“拓扑不变量”(invariants topologiques,topological invariants)。对于期待中的“数”(nombre,number)与“量”(grandeur,magnitude)的“联姻”,这就像一张过于狭窄的床,或许能勉强容纳一位未来配偶(即新娘),却绝无可能同时容纳两人!那尚待发现的“新原则”,要实现吉祥仙子预言的联姻,恰恰是这对未来夫妇所缺的“宽敞之床”——而此前,竟无人察觉这一缺失……

这张“双人床”随着拓扑斯(topos,topos)思想的出现而现身(仿佛魔杖一挥……)。这一思想以共同的拓扑直觉,包容了传统“拓扑空间”(espaces topologiques,topological spaces)——体现连续量世界的代表,以及那些冥顽不化的抽象代数几何学家眼中的(所谓)“空间”或“簇”,乃至无数其他结构类型——这些结构此前似乎被牢牢钉死在“算术世界”(monde arithmétique,arithmetic world)的“离散”或“不连续”聚合之中。

层(faisceaux,sheaves)的观点是那沉默而可靠的向导,是那有效且毫不神秘的钥匙,引领我毫不迟疑、无需绕道,直抵那宽敞双人床所在的婚房。这张床如此宽广(宛如一条深邃宁静的大河……),以至于
\begin{quote}
    “国王的所有骏马\\
    都能齐聚饮水……”
\end{quote}
——正如一首古老歌谣所唱,你必定也曾唱过,或至少听过。而那最早唱响此曲之人,比我昔日的任何博学学生与朋友,都更深切地感受到拓扑斯隐秘的美感与宁静的力量……

这把钥匙在最初的临时方法(通过极为便利却非内在的“位点”(site,site)概念)与拓扑斯方法中始终如一。现在,我想试着描述拓扑斯的思想。

设想一个给定“拓扑空间”(espace topologique,topological space)上所有层(faisceaux,sheaves)的集合,或者,若你愿意,这支由所有丈量此空间的“尺”组成的惊人军械库 \footnote{(致数学读者)严格来说,此处指的是集合层(faisceaux d’ensembles,sheaves of sets),而非勒雷引入的阿贝尔层(faisceaux abéliens,abelian sheaves),后者作为最广义系数用于构造“上同调群”(groupes de cohomologie,cohomology groups)。我相信自己是首个系统研究集合层的人(自 1955 年起,见我在堪萨斯大学发表的文章《带结构层的纤维空间通论》(A général theory of fibre spaces with structure sheaf))。}。我们将这一“集合”或“军械库”视为具备最显而易见的结构,若可用直觉形容,便是“显而易见”的结构,即所谓“范畴”(catégorie,category)的结构。(非数学读者无需因不熟悉此术语的技术含义而不安,后文无需此知识即可理解。)这一“超丈量结构”,称为“层范畴”(catégorie des faisceaux,category of sheaves,基于所考察的空间),从今往后将被视为体现空间最本质特征的存在。这在“数学常识”中是正当的,因为我们发现,可以通过这一“层范畴”(或丈量军械库)完全重构一个拓扑空间 \footnote{(致数学读者)严格来说,这仅对所谓“清醒空间”(espaces sobres,sober spaces)成立。但此类空间几乎涵盖了常见的所有空间,尤其是分析家珍视的“分离空间”(espaces séparés,separated spaces)。}。(验证这一点是个简单练习——当然,前提是问题已被提出……)这足以让我们确信,若出于某种原因需要,我们可“忘却”初始空间,仅保留并使用与之关联的“范畴”(或“军械库”),将其视为表达“拓扑结构”(structure topologique,topological structure)或“空间性”(spatialité,spatiality)的最佳化身。

如数学中常有的情形,我们在此借助“层”(faisceau,sheaf)或“上同调尺”(mètre cohomologique,cohomological meter)的关键思想,成功将某一概念(此处为“空间”)转化为另一概念(即“范畴”)来表达。每当我们发现一种概念(对应某种情境)可被翻译为另一种概念(对应另一情境)时,这种意外汇合便丰富了我们对二者的理解——通过各自独特直觉的交融。于是,此处一种“拓扑”情境(由给定空间体现)被翻译为一种“代数”情境(由“范畴”体现);或者可以说,空间所体现的“连续性”被“范畴结构”——一种“代数”性质(此前被视为本质上“离散”或“不连续”)——所“翻译”或“表达”。

但在此处,更进一步。第一个概念,即“空间”,看似已是某种“极广”概念——如此一般化,难以想象还能合理扩展。而通过镜子的另一侧 \footnote{此处“镜子”如《爱丽丝漫游奇境》(Alice au pays des merveilles,Alice in Wonderland)中的意象,指将空间置于其前,映出关联的“范畴”,作为空间在“镜子另一侧”的“双重”……},我们发现,从拓扑空间出发所得的这些“范畴”(或“军械库”)具有极为特殊的性质。它们拥有一组高度鲜明的特性 \footnote{(致数学读者)此处主要指我在范畴论中引入的“精确性性质”(propriétés d’exactitude,exactness properties),连同现代范畴意义上的“广义归纳与投射极限”(limites inductives et projectives générales,general inductive and projective limits)概念。见《论同调代数的若干要点》(Sur quelques points d’algèbre homologique,On Some Points of Homological Algebra),《东北数学杂志》(Tohoku Math. Journal),1957 年,第 119-221 页。},仿佛是对最简单范畴——由单点空间所得者——的某种“仿作”。由此,新式“空间”(或拓扑斯),作为传统拓扑空间的推广,可简单描述为一个“范畴”,它未必源于普通空间,却具备所有这些优良特性(当然需一次性明确指定)——即“层范畴”的特性。

\begin{center}
    * \quad * \\
    *
\end{center}

这就是新思想。其出现可视为一种近乎童稚观察的自然结果:拓扑空间中真正重要的,不是其“点”或点的子集 \footnote{因此,可构造出“极大”的拓扑斯,仅有一个“点”,甚至完全无“点”!},以及它们间的邻近关系等,而是其上的层(faisceaux,sheaves)及其形成的范畴。我不过将勒雷的原始思想推向极致——随后迈出这一步。

如勒雷的层(faisceaux,sheaves)思想、概形(schémas,schemes)思想一样,作为颠覆固有视野的“伟大思想”,拓扑斯(topos,topos)以其自然、“显而易见”的特质令人困惑。其简朴(近乎天真或简单,甚至因那独特品质显得“笨拙”),常使我们惊呼:“哦,不过如此!”——语调半是失望,半是羡慕;或许还隐含一丝“异想天开”或“不严肃”的意味,常用于评判那些因出乎意料的简朴而令人迷惘的事物。它们或许唤起我们早已埋藏并否认的童年时光……