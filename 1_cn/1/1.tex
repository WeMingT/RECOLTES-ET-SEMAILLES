
1986年1月30日

万事俱备,只欠一篇序言,便可将《收获与播种》(Récoltes et Semailles)交付印刷。我发誓,我确实怀着最大的诚意,想要写出一篇合适的序言。这一次,要写点\textbf{合理}的东西。三四页足矣,但要言之有物,用以介绍这部超过千页的“巨著”。要写点能吸引那些麻木读者的东西,让他们隐约感到,在这“超过千页”的厚重篇幅中,或许有些内容会让他们感兴趣(甚至与他们息息相关,谁知道呢?)。说实话,吸引读者并非我的强项,但这次我打算破例一次!毕竟,总得让那位“敢于冒险的出版商”(愿意出版这部显然难以出版的“怪物”)尽可能地收回成本。

然而,事与愿违。我尽力了,真的尽力了。而且不止一个下午,正如我原本计划的那样,匆匆了事。明天就是整整三周了,稿纸堆积如山。写出来的东西,显然不能被称为一篇“序言”。又一次失败了,真是命中注定!到了我这个年纪,已经无法改变——我也不适合推销或让别人推销。即使是为了取悦自己或朋友们。

写出来的,更像是一段漫长的“漫步”,伴随着对我数学作品的评述。这段漫步主要是为“外行”准备的——那些“从未理解过数学”的人。同时也是为我自己准备的,因为我从未有过这样的闲情逸致。渐渐地,我发现自己开始揭示并说出一些一直未曾言明的东西。巧合的是,这些东西也是我感觉最为本质的,无论是在我的工作还是作品中。它们与技术无关。至于我是否成功地完成了这项天真的“传递”任务——这项或许同样有些疯狂的任务——就由你来评判了。我的满足与快乐,在于能否让你感受到这些。这些东西,我的许多博学同事早已无法感受。或许他们变得太过博学、太过显赫。这常常会让人失去与简单而本质的事物的联系。

在这段“漫步”中,我也谈到了我的生活。偶尔,也会提到《收获与播种》中的一些内容。在随后的“信件”(日期为去年五月)中,我会更详细地再次谈及这些内容。这封信原本是写给我的前学生和数学界“昔日好友”的。但它同样没有技术性内容。任何感兴趣的读者都可以毫无障碍地阅读,通过这段“鲜活”的叙述,了解最终促使我写作《收获与播种》的来龙去脉。比“漫步”更进一步,它还会让你提前感受到数学“大世界”中的某种氛围。同时(与“漫步”一样),也能让你感受到我的表达风格——据说有些特别。还有通过这种风格表达的精神——这种精神也并非人人都能欣赏。

在“漫步”以及《收获与播种》的许多地方,我谈到了\textbf{数学工作}。这是我非常熟悉且亲身经历的工作。我所说的许多内容,无疑适用于任何创造性工作,任何发现性工作。至少对于所谓的“智力工作”是如此,那种主要通过“头脑”完成并通过书写表达的工作。这种工作的标志是对我们正在探索的事物的\textbf{理解}的萌发与绽放。但举一个相反的例子,爱情的激情同样是一种发现的冲动。它向我们敞开了一种被称为“肉体”的知识,这种知识同样会更新、绽放、深化。这两种冲动——一种是驱动工作中的数学家的冲动,另一种是驱动爱人的冲动——比人们通常认为的或愿意承认的要接近得多。我希望《收获与播种》的篇章能让你在你的工作和日常生活中感受到这一点。

在“漫步”中,主要讨论的是数学工作本身。然而,我几乎对这项工作的背景以及在工作时间之外的\textit{动机}保持沉默。这可能会给人一种关于我本人,或关于数学家或“科学家”的过于美好但扭曲的形象。像是“伟大而崇高的激情”,没有任何修正。总之,符合“科学神话”(请用大写字母S)的基调。这种英雄式的、“普罗米修斯式”的神话,作家和科学家们(并且仍在继续)争先恐后地陷入其中。或许只有历史学家有时能抵抗这种如此诱人的神话。事实是,在“科学家”的动机中,有时驱使他们不计代价地投入工作的,野心和虚荣心扮演着与任何其他职业同样重要且几乎普遍的角色。它们以或粗糙或微妙的形式表现出来,取决于当事人。我绝不声称自己是个例外。我希望我的证言不会让人对此产生任何怀疑。

同样真实的是,最贪婪的野心也无法发现或证明任何一个数学命题——正如它无法(例如)“让人勃起”(字面意义)。无论是男性还是女性,让人“勃起”的绝不是野心、炫耀的欲望或展示力量的欲望——恰恰相反!而是对某种强烈、真实且微妙的事物的敏锐感知。我们可以称之为“美”,这是这种事物的千面之一。有野心并不一定妨碍我们偶尔感受到一个人或一件事的美,但可以肯定的是,让我们感受到美的绝不是野心……

第一个发现并掌握火的人,正是像你我一样的普通人。绝不是我们想象中的“英雄”或“半神”。当然,像你我一样,他也曾经历过焦虑的刺痛,以及虚荣的安慰,这种安慰让人忘记刺痛。但在他“认识”火的那一刻,既没有恐惧,也没有虚荣。这就是英雄神话中的真相。当神话被用来掩盖事物的另一面——同样真实且同样本质的一面时,它就变得乏味,变成了一种安慰剂。

我在《收获与播种》中的意图是谈论这两方面——知识的冲动,以及恐惧及其虚荣的解毒剂。我相信我“理解”或至少了解这种冲动及其本质。(也许有一天,我会惊讶地发现,我一直在自欺欺人……)但对于恐惧和虚荣,以及由此衍生的创造力的隐秘阻碍,我知道我并未深入探究这一巨大的谜题。我也不知道在我余下的岁月里,是否能够揭开这一谜题的真相……

在写作《收获与播种》的过程中,两幅画面浮现出来,代表了人类冒险的这两方面。它们是\textbf{孩子}(即工人)和\textbf{老板}。在接下来的“漫步”中,几乎完全讨论的是“孩子”。他也是副标题“孩子与母亲”中的主角。我希望这个名字能在“漫步”过程中逐渐清晰。

在其余部分的反思中,老板则占据了舞台的中心。他成为老板并非没有原因!更准确地说,这里讨论的不是一个老板,而是竞争企业的老板们。但所有老板在本质上都是相似的。当我们开始谈论老板时,也意味着会出现一些“坏人”。在反思的第一部分(“疲劳与更新”,紧随这篇介绍性部分之后,即“四乐章前奏”),主要是我,“坏人”。在接下来的三部分中,主要是“其他人”。轮流上场!

这意味着,除了深刻的哲学反思和“忏悔”(绝非悔悟)之外,还会有一些“尖刻的肖像画”(借用我一位同事和朋友的表达,他发现自己被稍微冒犯了……)。更不用说一些大规模的“行动”,绝非儿戏。罗伯特·若兰\footnote{罗伯特·若兰(Robert Jaulin)是我的老朋友。我了解到,他在民族学界的处境(作为“白狼”)与我在数学“美丽世界”中的处境有些相似。}曾半开玩笑地告诉我,在《收获与播种》中,我是在“做数学界的人类学”(或者也许是社会学,我也说不清了)。当然,当得知自己在不知不觉中做了些学术性的事情时,我感到很受恭维!事实上,在反思的“调查”部分(尽管我并不情愿……),我看到自己正在书写的页面上,数学界的大部分机构轮番登场,更不用说许多地位较为普通的同事和朋友了。而最近几个月,自从去年十月我寄出《收获与播种》的临时印刷版以来,这种情况又再次发生。显然,我的证言像一块石头扔进了池塘。反响五花八门(除了无聊……)。几乎每次,都完全出乎我的意料。还有许多沉默,意味深长。显然,我还有很多东西要学习,而且是各种各样的东西,关于我的前学生和其他同事(无论地位高低)脑子里在想什么——抱歉,我是说关于“数学界的社会学”!对于那些已经为我晚年这部伟大社会学作品做出贡献的人,我在此表达我的感激之情。

当然,我对那些热情洋溢的回应特别敏感。也有一些罕见的同事向我表达了他们的情感,或一种(此前未曾表达的)危机感,或对数学界内部退化的感受,他们觉得自己是这个圈子的一部分。

在这个圈子之外,最早对我的证言表示热烈甚至感动的欢迎的人中,我想在此提到西尔维和凯瑟琳·谢瓦莱\footnote{西尔维和凯瑟琳·谢瓦莱(Sylvie et Catherine Chevalley)是克劳德·谢瓦莱(Claude Chevalley)的遗孀和女儿,克劳德是我的同事和朋友,《收获与播种》的核心部分(RES III,“阴阳之钥”)就是献给他的。在反思的多个地方,我谈到了他,以及他在我的历程中所扮演的角色。}、罗伯特·若兰、斯特凡·德利戈尔热、克里斯蒂安·布尔瓜。如果《收获与播种》能够比最初的临时印刷版(面向一个非常有限的圈子)传播得更广,这主要归功于他们。归功于他们那种富有感染力的信念:我努力捕捉和表达的东西,必须被说出来。而且它能够被一个比我的同事(常常阴郁、甚至暴躁,并且千百次地愿意重新审视自己……)更广泛的圈子所理解。正是因此,克里斯蒂安·布尔瓜毫不犹豫地冒险出版了这部难以想象的作品,而斯特凡·德利戈尔热则荣幸地将我这难以消化的证言收录在“认识论”系列中,与牛顿、居维叶和阿拉戈并列(我无法想象更好的伙伴!)。对于每一位在这个特别“敏感”的时刻给予我反复的同情和信任的人,我在此表达我深深的感激。

现在,我们即将开始一段“漫步”,作为穿越一生的旅程的序章。一段漫长的旅程,是的,超过千页,每一页都密密麻麻。我用了一生的时间来完成这段旅程,却仍未穷尽它,又用了一年多的时间重新发现它,一页一页地。有时,词语犹豫不决,难以表达一种仍在逃避犹豫不决的理解的经验——就像堆积在压榨机中的成熟葡萄,有时似乎想要逃避挤压它的力量……但即使在词语似乎争先恐后地涌出的时刻,它们也并非为了“幸运的幸福”而争先恐后地涌出。每一个词都在经过时被称重,或者事后被调整,如果发现它太轻或太重,就会被仔细调整。因此,这部反思-证言-旅程并非为了被匆忙阅读,在一天或一个月内,由一个急于看到结局的读者读完。在《收获与播种》中,没有“结局”,没有“结论”,正如我的生活或你的生活中也没有。这里有一种酒,在我的存在之桶中陈酿了一生。你喝下的最后一杯不会比第一杯或第一百杯更好。它们都是“同一杯”,又各不相同。如果第一杯酒变质了,整个桶也就变质了;那么,不如喝点好水(如果有的话),而不是坏酒。

但好酒不能匆匆喝下,也不能随意饮用。

