\subsection{3. 罗盘与行囊}

我该简单说几句,关于与《收获与播种》「Récoltes et Semailles;Harvests and Sowings」一同构成同名卷册的另外两篇文本。

《纲领初稿》「Esquisse d'un Programme;Sketch of a Program」勾勒了我过去十年中追求的主要数学反思主题。我至少打算在未来几年里,在一系列非正式的反思——我已提及的《数学反思》「Réflexions Mathématiques;Mathematical Reflections」中,稍加展开其中一些主题。这份初稿是去年一月我为申请法国国家科学研究中心「CNRS;CNRS」研究职位所写报告的原文复制。我将其收入本卷,因为显然这个纲领远远超出了我这微不足道之人的能力,即便我还能再活一百年,并选择将这些时间用于尽可能深入探索这些主题。

《主题概要》「Esquisse thématique;Thematic Sketch」则是1972年为另一场申请(法兰西学院「Collège de France;College of France」教授职位)所写。它按主题概述了我当时认为自己对数学的主要贡献。这篇文本带有写作时的心境痕迹,那时我对数学的兴趣可以说微乎其微,至少是如此。因此,这份概要不过是干巴巴且条理分明的列举(幸好它并未试图穷尽一切……)。它似乎缺乏一种视野或欲望的激情驱动——仿佛我在其中审视的那些事物,只是出于良知例行公事(确实如此),从未被鲜活的洞见触及,也未被一种热情点燃,去将它们从迷雾与阴影的面纱后揭示出来,当时它们尚只是隐约可感……

然而,若我还是决定将这份缺乏灵感的报告收入此处,恐怕主要是为了堵住某些高高在上的同事以及某种风气的嘴(假设这真能做到的话)。自从我离开那个我们曾共处的世界后,他们便故作姿态,俯视他们亲切称之为“格罗滕迪克怪论”「grothendieckeries;grothendieckeries」的东西。据说这是对过于琐碎事物的夸夸其谈,不值得一个严肃且品味高雅的数学家浪费宝贵时间。或许这份“难以下咽的摘要”在他们眼中会显得更“严肃”!至于那些由视野与热情驱动的我的文字,它们并非为那些被时尚维系并为其傲慢辩护的人而写,那些人对令我着迷的事物无动于衷。如果我为他人而非自己写作,那是为那些不认为自己时间与人格过于珍贵、不倦追求无人肯见的显而易见之物的人,为那些为每件发现之物的内在美感而欣喜、并将其独特之美与其他已知之美区分开来的人。

若要定位本卷三篇文本彼此间的关系,以及它们在我已投身的这场《数学反思》之旅中的各自角色,我可以说,《收获与播种》这部反思-证言反映并描述了我开启这场旅程的精神状态及其意义所在。《纲领初稿》描绘了我的灵感源泉,它们为这场未知之旅设定了一个方向——未必是终点——仿佛一柄罗盘,或一根坚韧的阿里阿德涅之线「Fil d'Ariane;Ariadne's Thread」。而《主题概要》则快速回顾了我1970年前作为数学家的过往积累的行囊,其中至少一部分将在旅程的某些阶段派上用场并受到欢迎(例如我的代数上同调与拓扑斯「Topos;Topos」直觉,如今在《场的追寻》「Poursuite des Champs;Pursuit of Fields」中已不可或缺)。这三篇文本的排列顺序,以及它们各自的篇幅,也很好地反映了(并非我刻意为之)我在旅程中赋予它们的重要性和分量,而这旅程的第一步已接近尾声。