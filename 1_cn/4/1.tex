\section{(I) 五叶苜蓿}

\subsection*{1. 梦与实现}

再过三个月,就到了七月,那时将是我做了一个不寻常的梦的三周年。如果我说“不寻常”,那只是事后回想醒来时的印象。梦本身对我而言仿佛是最自然、最显而易见的事,没有喧嚣,也没有号角——甚至在醒来时,我差点没在意它,随手将它推入遗忘的角落,去处理“日常事务”。前一天,我开始反思自己与数学的关系。那是我人生中第一次费心去探究这件事——即便如此,当时会着手去做,也实在是迫不得已!在之前的几个月乃至几年里,发生了一些如此奇怪、甚至可以说是激烈的事情,一种数学激情如同爆炸般毫无预警地闯入我的生活,让我实在无法继续视而不见。

我所说的这个梦没有任何情节或动作。它仅由一幅画面构成,静止却又充满生气。那是一个人的头部侧影,从右向左看去。那是一个成熟的男子,无须,头发狂野,围绕头部形成一道力量的光环。尤其引人注目的是这个头部散发出的印象——一种青春洋溢、欢快的活力,似乎从颈部柔韧而有力的弧线中喷涌而出(那弧线更多是让人感知,而非清晰可见)。脸上的表情与其说是成熟男子的,或是那种稳重之人的,不如说更像一个顽皮的孩子,带着恶作剧的喜悦,仿佛正为刚刚完成或正在酝酿的某件妙事而得意。整张脸流露出一种强烈的、克制的生命喜悦,在嬉戏中迸发……

梦中没有第二个人,没有一个“自我”在注视或凝视这个仅见头部的他人。但对这个头部、对它所散发的一切,有一种强烈的感知。也没有人在那里感受这些印象、评论它们、命名它们,或为这个被感知的人贴上“某某”的标签。有的只是这个极具生命力的东西——这个人的头部,以及对它的同样鲜活而强烈的感知。

至于醒来时,我并非刻意回想,而是自然记起了夜间的梦境。这个男子头部的影像在众多梦中并不特别突出,它并未主动跳出来对我喊道或低语:“看我!是我值得你关注!”当这个梦出现在我快速扫视夜间梦境的视野中时,在温暖宁静的床上,我自然而然有了清醒时精神的习惯反应——为所见之物命名。我无需费力寻找,只需提出问题,便立刻知道这个梦中出现的男子头部正是我自己的。

“这可真不赖,”我当时想,“能这样在梦中看见自己,仿佛是另一个人,也真是了不起!”这个梦来得就像我在散步时,纯属偶然发现了一株四叶苜蓿,甚至是五叶苜蓿,让我按理应有的那样惊叹片刻,然后继续前行,仿佛什么也没发生。

至少,事情差点就这样过去了。幸好,正如我在类似情况下常有的习惯,我还是凭着良知,将这个“不赖”的小插曲白纸黑字记录下来,开始了一场本该延续前日思路的反思。然而,那天的反思却逐渐局限于探究这个不起眼的梦、这幅独特的画面,以及它带给我的关于自身的启示。

这里不是展开叙述那天冥想带给我教益与收获的地方。或者更准确地说,是这个梦带给我的教益与收获——一旦我让自己进入专注与倾听的状态,便能接纳它要对我说的话。梦与倾听的第一个直接果实,是一股突如其来的新能量。这股能量支撑了随后数月中漫长的冥想,克服了我内心顽强的阻力,我不得不通过耐心而执着的工作,一一拆解这些阻力。

在过去五年中,我开始留意某些梦境,而这个梦是我遇到的第一个“信使之梦”,它并未以我如今已能辨识的那种信使之梦的模样出现——那些梦通常带有震撼的场景手段和异常强烈的视觉效果,有时甚至令人震撼。这个梦则完全“低调”,没有任何强迫注意力的东西,甚至带着一种谨慎——接纳与否,全无纠缠。

几周前,我曾有过一个旧式风格的信使之梦,带着戏剧性甚至狂野的基调,骤然终结了我一段长时间的数学狂热。这两个梦唯一的表面相似之处在于,两者中都没有观察者。那场梦以一句简练有力的寓言,展示了我生活中正在发生却未被我留意的某事——坦白说,是我刻意忽视的东西。正是那个梦让我意识到反思工作的紧迫性,几周后我投入其中,这场反思持续了近六个月。我在《收获与播种》「Récoltes et Semailles;Harvests and Sowings」这部反思-证言的最后部分略有提及,那部分为本卷开篇并赋予其名 \footnote{参见第43节,“捣乱的老板——或压力锅”。}。

如果我以另一个梦——我对自身的影像-幻象(我在德语笔记中称之为“Traumgesicht meiner selbst”;Dream-vision of myself”)——来开启这篇引言,那是因为在最近几周,当“关于数学家过去的冥想”接近尾声时,这个梦的念头不止一次浮现。坦白说,回顾过去,自那个梦以来的三年仿佛是沉淀与成熟的岁月,朝向它那简单而清晰信息的实现。梦向我展示了“真实的我”。显然,在清醒的生活中,我并未完全成为梦中那个我——来自遥远过去的负担与僵硬常常(至今依然)阻碍我完全且单纯地做自己。在这些年里,尽管我很少想起这个梦,它一定以某种方式在起作用。它并非某种我努力模仿的模型或理想,而是对一种“属于我”的欢快单纯的悄然提醒,这种单纯以多种方式显现,并注定要从压抑它的重担中解放出来,充分绽放。这个梦既脆弱又坚韧,连接着仍被过去诸多重担拖累的现在,与孕育于此刻且近在咫尺的“明天”——那个“明天”即现在的我,且无疑一直在我心中……

无疑,若非最近几周这个极少提及的梦再度鲜明浮现,那是因为在某个非探析与思考的层面,我“知道”自己正在完成的工作——深化三年前那场工作的努力——是朝梦中关于自身信息实现的又一步。

对我而言,这便是《收获与播种》——近两个月紧张工作的主要意义。现在它完成了,我才意识到完成它有多重要。在这过程中,我体验了许多欢乐时光,常带点顽皮、戏谑与奔放的喜悦。也有悲伤的时刻,以及重温近几年深深触动我的挫折与痛苦的时刻——但从未有过一丝苦涩。我带着圆满的满足离开这项工作,如同知晓自己已将一项任务进行到底的人。没有任何“微小”之事被我回避,也没有任何我心之所系却未说出的话,此刻在我心中留下哪怕最微小的不满或遗憾。

在书写这份证言时,我清楚它不会讨所有人喜欢。甚至很可能,我设法让每个人无一例外地感到不满。然而,这绝非我的意图,甚至也不是要让任何人不满。我的意图只是去看那些简单而重要的事物,那些我作为数学家过去的(有时也是现在的)日常事物,去最终发现(迟到总比不到好!)它们是什么、它们是怎样——毫无疑问或保留;并在这一路上,用简单的语言说出我所见。