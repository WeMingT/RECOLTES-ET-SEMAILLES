\subsection{10. 一种尊重的行为}

这场最终命名为“葬礼”「L'Enterrement;The Burial」的反思,始于一种尊重的行为。对我所发现之物的尊重——那些我在虚空中凝结成型的事物,我第一个品尝其滋味与活力,并为之命名,以表达我对它们的认知与敬意。我将最好的自己献给了这些事物。它们汲取我内在的力量而成长,绽放如多枝健壮的枝条,从同一活树干萌发,那树干根深叶茂。这些是鲜活而现存之物,非可随意制造或废弃的发明——它们在一种活的统一性中紧密相连,每一物在其中各得其所、各有其义,拥有起源与归宿。我早已将它们放下,未带丝毫忧虑或遗憾,因我知自己所留之物健康而强壮,无需我便能依其本性继续生长、繁荣与繁衍。我留下的不是一袋可被窃取的金币,也非一堆可能锈蚀或腐朽的工具。

然而,多年过去,当我自以为远离那曾抛下的世界时,从隐退处断续传来隐秘的轻蔑与微妙的嘲弄,指向那些我熟知的强大而美好的事物,它们各有独一无二的位置与功用,无可替代。我感到它们如孤儿,置身于一个充满敌意的世界,一个因轻蔑之病而病态的世界,执意攻击无甲之物。正是在此情境下,这场反思始于对这些事物——进而对我自身——的尊重行为,如同唤醒我与这些事物间深层联结的提醒:那乐于对这些曾受我爱滋养之物表露轻蔑者,实则乐于轻蔑我,及我所生之一切。

同样,那熟知我与某物联结之人——此联结仅由我传授于他——若佯装此联结可忽略,或加以否认,或(哪怕以沉默与省略的方式)为己或他人虚构“原创”身份,我清楚视之为对一事物及其创作者的轻蔑——对那使之诞生的隐秘而精微劳动的轻蔑,对创作者的轻蔑,而首要且更隐秘、更本质地,是对他自身的轻蔑。

若我“重返数学”仅为唤起我对这联结的记忆,并于众人前——在那些故作轻蔑者与冷漠见证者前——激起这尊重的行为,此回归便非无用。

诚然,我确已与自己留下的已写与未写(或至少未发表)的作品失去联系。反思初始,我尚能清晰分辨那些枝条,却不太记得它们同属一树。奇妙的是,需逐渐在我眼前揭露我所留之物被掠夺的图景,我才重拾那被掠夺与分散之物的活统一性之感。有人携走金币,有人取一两工具自夸或利用——但那赋予我所留之物生命与真正力量的统一性,却无人得之、无人尽得。然而,我深知一人深刻感受到这统一性与力量,且今日内心深处仍感之,他却乐于分散自身之力,欲摧毁他通过他人作品所感受到的这统一性。这活统一性中蕴藏着作品的美感与创造力。尽管掠夺,我发现它们完好如初,仿佛刚离我而去,只是我已成熟,以新目光视之。

然而,若有何物被掠夺、残损,失去其原初力量,那是在那些遗忘自身内在力量之人中,他们自以为掠夺一任其摆布之物,却仅是从那对众人开放、却不受任何人摆布或掌控的创造力中自我割离。

于是,这反思,及其带来的意外“回归”,亦使我重拾一被遗忘的美感。正是充分感受到这美感,赋予了这尊重行为的全部意义,它在注释“我的孤儿”「Mes orphelins;My Orphans」\footnote{此注释(编号46)按时间顺序为“葬礼”中所有注释之首。}中笨拙表达,我在此有意识地重申。