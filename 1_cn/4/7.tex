\subsection*{7. 葬礼的编排}

在“葬礼”「L'Enterrement;The Burial」这个名称下,我在目录中汇集了与看似平淡无奇的章节“过去的重量”「Le poids d'un passé;The Weight of a Past」 (第50节) 相关的众多“笔记”,形成了一场壮观的队列,从而充分赋予了这个名称其意义——这个名称在《收获与播种》「Récoltes et Semailles;Harvests and Sowings」“初稿”的最终章节一开始就强加于我。

在这支由多重关联交织而成的长长队列中,过去四周内(笔记 (51) 至 (97))新加入的那些 \footnote{还需加上1984年5月12日的笔记 $\mathrm{n}^{\circ}$ 104。笔记 $\mathrm{n}^{\circ}$ 98 及之后(除前述笔记 $\mathrm{n}^{\circ}$ 104 外)构成反思的“第三口气”,始于1984年9月22日。它们同样带有日期。} 因其带有日期(从4月19日至5月24日)\footnote{在一系列同日连续书写的笔记中,仅首篇标有日期。其余未标日期的笔记包括第44'至50号笔记(构成队列 I、II、III)。第46、47、50号笔记写于3月30或31日,第44'、48、48'、49号笔记写于4月上半月,最后第44''号笔记带有日期(5月10日)。} 而显得与众不同。我认为最自然的方式是按照它们在反思中出现的先后顺序排列 \footnote{我偶尔在此时间顺序上做了小幅调整,以遵循所谓的“逻辑顺序”,只要我觉得这不会扭曲反思整体进程的印象。然而,作为例外,我要指出有十一篇笔记(其编号前带有 ! 符号)源自后续的草稿笔记,因篇幅过长而被置于与其相关的笔记之后(除笔记 $\mathrm{n}^{\circ}$ 98 外,它与第47号笔记相关)。},而非某种所谓的“逻辑”顺序,或这些笔记在先前笔记中被提及的出现顺序。为了便于追溯参与笔记间这种(绝非线性的)关联顺序,我在目录中为每篇笔记的编号后标注了首次提及它的先前笔记的编号 \footnote{当某笔记(如 (45))的提及出现在“过去的重量”章节本身中时,则在其编号后括号内标注该章节的编号 (50),如 46 (50)。},或(若无前者)标注其作为直接延续的笔记的编号 \footnote{在目录中,作为前一笔记直接延续的笔记编号(其编号依次相连)前带有 * 符号。例如 *47, 46 表示第47号笔记是第46号笔记的直接延续(此处第46号笔记并非紧接其前的笔记,后者为第46$_9$号笔记)。}。(这种后者关系在文本中通过置于首篇笔记末尾的参考符号标示,例如 (46) 号笔记末行末尾的 $(\Rightarrow 47)$,指向其延续的 (47) 号笔记。)此外,某些带点技术性的补充说明被集中于相关笔记末尾,以连续编号的子笔记形式呈现——如 (46) 号笔记“我的孤儿”「Mes orphelins;My Orphans」的子笔记 $\left(46_{1}\right)$ 至 $\left(46_{9}\right)$。

为了略微结构化“葬礼”的整体编排,并便于在众多纷繁笔记中辨识方向,我认为适合此情此景的方式是在队列中插入几个严肃而富有暗示性的小标题,每个标题引领一组或长或短、由共同主题连接的连续笔记队列。

于是,我愉快地见证了十个 \footnote{(9月29日) 事实上,最终有十二个队列,若算上“殡车” (X) 和“亡者(尚未死去)” (XI),后者在最后时刻硬是挤进了队列……} 队列逐一集结,组成了一支庄严的长长队伍,为我的葬礼致敬——有些谦卑,有些壮观,有些满怀悔意,有些暗自欢欣,正如这种场合必然如此。它们依次走来:死后的学生(人人皆有义务无视他),孤儿们(为此场合特意新近挖掘出土),时尚及其杰出人物(我当之无愧),动机「Motifs;Motives」(我所有孤儿中最后诞生也最后被挖掘者),我的朋友皮埃尔「Pierre;Pierre」谦逊地引领最重要的队列,紧随其后的是齐声(默然)协奏的笔记所达成的全体一致,以及完整的研讨会(号称“变态”)(通过插队的葬礼队列携带鲜花与花冠,与死后的学生——即未知的学生——区分开来);最后,为了体面地结束这场盛大游行,又有学生(绝非死后,更非未知)即老板登场,随后是我的学生们忙碌的队伍(手持铲子和绳索),以及殡车(展示四具坚实拧紧的橡木棺材,刨工不算在内)……十个队列终于齐全(也该是时候了),缓缓走向葬礼仪式。

仪式的亮点是由我的朋友皮埃尔亲自献上的悼词,他以绝妙的手法回应众人意愿,主持葬礼,令所有人满意。仪式在一曲最终且彻底的《深渊呼声》「De Profundis;De Profundis」(至少人们希望如此)中结束,这曲真诚的感恩之歌由那位令人惋惜的亡者本人唱出——他在无人知晓中挺过了这场盛大的葬礼,并从中汲取了养分,感到完全满足——这份满足构成了最终的音符,也是这场难忘“葬礼”的最后一个和弦。