\subsection{8. 秘密的终结}

在这反思的最终阶段(希望如此),我意识到将两篇数学性质的文本作为“附录”加入《数学反思》「Réflexions Mathématiques;Mathematical Reflections」第一卷颇有意义,除此前提及的三篇外\footnote{此外,我考虑为《主题草图》「Esquisse Thématique;Thematic Sketch」(见“指南与行囊”「Boussole et bagages;Compass and Baggage」,引言,第3节)附一“评论”,就我对其中简要回顾的“主题”的贡献,及我数学作品中主要思想萌发时所受影响,提供若干 уточ化。这六周的回顾已(令我意外地)揭示塞尔「Serre;Serre」作为“引爆者”的角色,推动了这些思想的起步,及我1955至1970年间设定的某些“重大任务”。

另有一数学性质(常规意义)的文本,是唯一在非技术性文本《收获与播种》中偶现者,见子注释编号87,附于注释“屠杀”「Le massacre;The Massacre」(编号87),我以应有的谨慎阐明熟悉的连贯上下文中的黎曼-罗赫-格罗滕迪克定理「théorème de Riemann-Roch-Grothendieck;Riemann-Roch-Grothendieck theorem」的“离散”变体(猜想性)。此猜想(与其他若干)出现在1965/66年SGA 5研讨会的闭幕报告中,该报告(如许多其他)在十一年后出版的SGA 5卷中无迹可寻。此关键研讨会落入我某些学生手中之命运,及其与“行动SGA $4 \frac{1}{2}$”的关联,在注释编号63''、67、67'、68、68'、84、85、85'、86、87、88中逐步揭示。

另有注释提供较充实的数学评论,关于是否应尽可能为已知具备“六运算对偶性”「dualité des six opérations;duality of six operations」形式的案例,提炼共同“拓扑斯”「topossique;toposic」框架,见子注释编号81₂,附于注释“赊账论文与全险”「Thèse à crédit et assurance tous risques;Thesis on Credit and All-Risk Insurance」(编号81)。}。

这一连贯形式的草图对我而言,是迈向“动机之梦的宏大全景”「vaste tableau d'ensemble du rêve des motifs;vast overall picture of the dream of motifs」的第一步,这十五年来一直“等待一位大胆的数学家愿意描绘它”。显然,这位数学家将是我自己。的确,近二十年前在私密中诞生并托付之物,非为一人独享,而是供所有人使用,如今是时候走出秘密之夜,再次诞生于光天化日之下。

诚然,除我之外,仅一人对“动机瑜伽”「yoga des motifs;yoga of motifs」有深入了解,他从我口中日复一日、年复一年地学习,直到我离去。在我有幸发现并揭示的所有数学事物中,动机的现实仍是最迷人、最神秘的——直指“几何”与“算术”深层同一性的核心。这长期被忽视的现实引我进入的“动机瑜伽”,或许是我数学人生第一阶段挖掘出的最有力的发现工具。

但同样真实的是,这现实及其试图贴近它的“瑜伽”,我从未刻意保密。忙于编写基础性文献(如今人人都乐于直接用于日常工作)的迫切任务,我未花数月时间为这动机瑜伽撰写宏大草图,以供众人使用。然而,在我意外离去前的几年,我未曾吝于在偶遇中向愿听者述说,从我的学生开始,除一人外,他们皆忘却,正如所有人忘却。若我提及此事,非为标榜“我的发明”,而是为唤起注意:这一现实在每一步显现,只要关注代数簇的上同调,尤其是其“算术”性质及已知上同调理论间的关联。这现实如昔日的“无穷小”般 tangible,早在严谨语言出现前即被感知,直到完美把握并“确立”它的语言诞生。为把握动机的现实,我们今日不乏灵活适切的语言,也不缺构建数学理论的丰富经验,这是前人所无。

若我昔日高声宣示却落入聋耳,若一人的轻蔑沉默引来所有佯装关注上同调者(他们如我一样有眼有手……)的沉默与倦怠,我不能仅责怪那选择独占我为众人所托之“益处”者。不得不承认,我们的时代——其科学生产力狂热堪比军备或消费品——远不及十七世纪前辈的“大胆活力”。他们为发展无穷小演算“不择手段”,未因其“猜想性”而止步,也不待某权威人物点头,便着手处理亲眼所见、亲手所感之物。