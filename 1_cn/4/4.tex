\subsection{4. 追寻显而易见之物的旅程}

我还需对这趟始于一年前的旅程——《数学反思》「Réflexions Mathématiques;Mathematical Reflections」——多说几句详尽的话。在《收获与播种》「Récoltes et Semailles;Harvests and Sowings」的前八节(即反思的第一、二部分)中,我已相当详细地阐释了我开启此旅程的精神,我认为这在当前首卷中已显而易见,亦将在紧随其后的《追寻场域》「À la Poursuite des Champs;In Pursuit of Fields」第一卷(《模型史》「Histoire de Modèles;History of Models」)中体现,该卷正在完成中。因此,我觉得在此引言中无需赘述此话题。

我当然无法预言此旅程将如何展开,这将随着它的推进逐步揭示。我目前并无哪怕大致规划的路线,也不认为近期会形成一个。正如我此前所述,启发我反思的主要主题已在《纲领草图》「Esquisse d'un Programme;Sketch of a Program」——这一“指南文本”——中或多或少有所勾勒。其中包括《追寻场域》的核心主题,即“场域”「champs;fields」,我希望今年内能将其彻底探究(并就此止步),或用两卷,或许三卷。关于此主题,我在《纲领草图》中写道:“……这有点像一笔债务,我借此清偿过去十五年(1955至1970年间)的科学时光,那时发展上同调工具是我在代数几何基础工作中不变的主旋律。”因此,在预定主题中,这是与我“科学过去”根基最深的一个。它也在过去十五年间始终如遗憾般萦绕,成为我离开数学舞台时留下的工作中最显眼的缺憾之一,而我昔日的学生或朋友无人关心填补。欲了解此在进行工作的详情,感兴趣的读者可参阅《纲领草图》中相关章节,或《追寻场域》第一卷(此次为真正的!)引言,该卷正在完成。

我科学过去的另一重要遗产,尤为我所珍视的,是“动机”「motif;motif」概念,它自问世约十五年来始终被困于暗夜,等待破晓。倘若未来几年无人比我更适合(因更年轻,或因掌握的工具与知识)从事此基础工作,我不排除亲自投入其中。

借此机会,我想指出,“动机”概念的命运(或更恰当地说,不幸……)以及我发掘出的其他最具(潜在)丰饶性的概念,在《收获与播种》中一篇近二十页的回顾性反思中有所探讨,这是最长(且几乎最后)的“注释”之一\footnote{此双重注释(编号46、47)及其子注释已纳入《收获与播种》第二部分“葬礼”「L'Enterrement;The Burial」,为其直接延续。}。我事后将其分为两部分(“我的孤儿”「Mes orphelins;My Orphans」与“拒绝遗产——或矛盾的代价”「Refus d'un héritage - ou le prix d'une contradiction;Refusal of an Inheritance - or the Price of a Contradiction」),另附三条“子注释”\footnote{即子注释编号48、49、50(注释48'为后来添加)。}。这五条连续注释是《收获与播种》中唯一非仅提及即过的数学概念部分。这些概念成为例证,揭示数学家世界内部的某些矛盾,而这些矛盾又映照于人自身的矛盾。我曾考虑将这庞大注释从原文中分离,附于《主题草图》「Esquisse thématique;Thematic Sketch」。这本可为其提供视角,为过于目录化的文本注入生气。然而,我最终未如此做,为保留见证的真实性,这冗长注释,无论我愿否,皆为其一部分。

关于我在《收获与播种》中所述开启《反思》的态度,我在此仅补充一点,我已在某注释(“年轻人的势利——或纯洁的捍卫者”「Le snobisme des jeunes - ou les défenseurs de la pureté;The Snobbery of the Young - or the Defenders of Purity」)中表达:“我一生作为数学家的抱负,或更恰当地说,我的喜悦与激情,始终是发现显而易见之物,这也是我在当前著作(《追寻场域》)中的唯一抱负。”这亦是我自一年前以《反思》继续此新旅程的唯一抱负。在《收获与播种》中(至少对我的读者而言,若有的话),开启此旅程时亦是如此。