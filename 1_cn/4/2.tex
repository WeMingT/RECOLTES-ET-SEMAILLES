\subsection{2. 旅程的精神}

这场最终成为《收获与播种》「Récoltes et Semailles;Harvests and Sowings」的反思,起初是为《追寻场域》「À la Poursuite des Champs;In Pursuit of Fields」第一卷(正在完成中)撰写的“引言”,这是我自1970年以来首部计划出版的数学著作。我在去年六月的一个空闲时刻写下了最初几页,并在不到两个月前,从中断处重拾这一反思。我意识到有许多值得审视与述说之事,因此预期会有一篇较为充实的引言,大约三四十页。然而,在随后近两个月中,直到此刻我为这原本是引言的内容写下新的引言时,我每日都以为当天能完成此工作,或最迟翌日、最多后日。当几周后篇幅接近百页时,这引言被提升为“引论章节”。又过几周,当该“章节”的篇幅远远超过准备中卷本其他章节(写下此行时除最后一章外皆已完成)时,我终于明白,它不适合置于数学书中,这反思与见证显然在其中局促不安。它们的真正归属是一个独立卷本,即我计划未来几年继续推进的《数学反思》「Réflexions Mathématiques;Mathematical Reflections」的第1卷,乘着《追寻场域》的势头。

我不会称《收获与播种》——《数学反思》系列的首卷(后将接续《追寻场域》的两三卷作为开端)——为《反思》的“引论卷”。更确切地说,我视此首卷为后续之基,或更恰当地,为定下基调者,承载我开启这场新旅程的精神。这旅程我将在未来几年持续推进,不知将把我引向何方。

为结束关于本卷主体部分的说明,我提供一些实用指引。读者不会惊讶于在《收获与播种》正文中偶见对“本卷”的提及——意指《追寻场域》第一卷(《模型史》「Histoire de Modèles;History of Models」),我仍以为自己正在为其撰写引言。我未“修正”这些段落,首先因我珍视文本的自然流露,及其作为见证的真实性,不仅关乎遥远过去,也关乎我写作的当下。

出于同样理由,我对初稿的修改仅限于矫正文风笨拙或表述混乱之处,以免妨碍我意图表达的理解。这些修改有时使我比初稿时更清晰或细腻地领悟。任何稍具实质性的改动——为细化、 уточ化、补充或(有时)纠正初稿——则体现在约五十个编号注释中,集中于反思末尾,占文本逾四分之一\footnote{此指《收获与播种》第一部分“虚荣与更新”「Fatuité et Renouvellement;Vanity and Renewal」的文本。第二部分写此行时尚未成稿。}。我以(1)等标记指引至此。其中,我挑出约二十个注释,因其篇幅或实质,与反思自发组织的五十个“节”或“段”重要性相当。这些较长注释被列入目录,置于五十节列表后。如预期,某些长注释需附加一或多条子注释,随即附于其后,采用相同指引方式;除较短者,则以“脚注”形式同页呈现,指引如 或。

我乐于为文本每节及每个重要注释命名——后来证明这对理清头绪不可或缺。不言而喻,这些名称是事后所取,因开始某节或长注释时,我无法预知其核心实质。对我事后将五十节分组为八部分(I至VIII)的命名(如“工作与发现”「Travail et découverte;Work and Discovery」等)亦然。

关于这八部分内容,我仅作简评。前两部分I(工作与发现)和II(梦与梦者)「Le rêve et le Rêveur;The Dream and the Dreamer」含关于数学工作及一般发现工作的反思。我的个人参与在此远较后几部分偶发且间接。后几部分尤具见证与冥想特质。第III至VI部分主要反思与见证我1948至1970年间作为数学家的过去“在数学世界中”。驱动此冥想的主要动机,是渴望理解这段过去,以努力理解并承担当下某些令人失望或困惑的面向。第VII部分(童子嬉戏)「L'Enfant s'amuse;The Child at Play」与VIII(孤独冒险)「L'aventure solitaire;The Solitary Adventure」则更多关乎我自1970至今与数学关系的演变,即自离开“数学家世界”不再返回以来。我特别审视那些动机、力量与环境,它们(令我意外地)促使我恢复“公开”的数学活动(撰写并出版《数学反思》),在中断逾十三年后。