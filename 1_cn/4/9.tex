\subsection{9. 舞台与演员}

凭借其内在结构与独特主题,“葬礼”「L'Enterrement;The Burial」(现已占据《收获与播种》「Récoltes et Semailles;Harvests and Sowings」文本的半数以上)在逻辑上很大程度上独立于前面的长篇反思。然而,这种独立性只是表面的。对我而言,这场围绕“葬礼”的反思——它逐渐从未言明与隐约感知的迷雾中浮现——与之前的反思密不可分,后者是其源头并赋予其全部意义。起初,这不过是对我有些(甚至颇多)疏远的一部作品的命运投以匆匆一瞥,却在未曾预料也未刻意追求的情况下,演变为对我生命中一段重要关系的冥想,进而引向对这部作品在“那些曾是我的学生”手中命运的思考。将这场反思与其自然生发的源头割裂开来,在我看来无异于将其简化为单纯的“风俗画”(甚至是数学“高雅世界”中的一场清算)。

诚然,若有人执意如此,整个《收获与播种》同样可被简化为“风俗画”。当然,一个时代、一个特定圈子中盛行的风气,塑造了其中人们的生活,确实重要且值得描述。然而,对于细心的读者而言,显然我的意图并非描绘风俗,即某个随时间与地点而变的舞台,我们的行动在其上展开。这个舞台在很大程度上定义并限制了我们内在各种力量可用的手段,使之得以表达。尽管舞台及其提供的手段(以及它强加的“游戏规则”)千变万化,但那些深层力量的本质——在集体层面塑造舞台、在个人层面于其上表达——似乎在不同圈子、文化与时代之间并无二致。在我生命中,除数学与对女性的爱之外,若有何物让我(虽是迟至今日)感受到神秘与吸引,那便是这些隐秘力量的本质——它们有能力驱使我们行动,或为“最好”,或为“最坏”,或埋藏,或创造。