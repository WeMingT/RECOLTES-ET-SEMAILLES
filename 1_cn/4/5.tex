\subsection{5. 一份受欢迎的债务}

我想以几句话结束这篇引言,谈谈《收获与播种》「Récoltes et Semailles;Harvests and Sowings」本卷的两则献词。

献词“致那些曾是我的学生,我将最好的自己——也包括最坏的——给予了他们”至少从去年夏天起就萦绕在我心头,尤其是当我撰写原本打算作为数学著作引言的前四节时。这表明我很清楚——实际上几年前就已明白——有一个“最坏的部分”需要审视,而现在正是时候,否则就再无机会!(不过我并未料到,这个“最坏的部分”最终会引领我完成近两百页的冥想。)

相比之下,献词“致那些曾是我的前辈”则是在写作途中才浮现的,就像这部反思的标题(后来也成为本卷的名字)一样。这部反思向我揭示了他们在我的数学家生涯中所扮演的重要角色,其影响至今依然鲜活。这一点在接下来的篇章中或许会相当明显,因此这里无需赘述。这些“前辈”,按他们(大致)在我二十岁时进入我生命中的顺序排列,是亨利·嘉当「Henri Cartan;Henri Cartan」、克洛德·舍瓦利耶「Claude Chevalley;Claude Chevalley」、安德烈·韦伊「André Weil;André Weil」、让-皮埃尔·塞尔「Jean-Pierre Serre;Jean-Pierre Serre」、洛朗·施瓦茨「Laurent Schwartz;Laurent Schwartz」、让·迪厄多内「Jean Dieudonné;Jean Dieudonné」、罗杰·戈德芒「Roger Godement;Roger Godement」、让·德尔萨特「Jean Delsarte;Jean Delsarte」。作为一个无知的新来者,我被他们每一位以善意接纳,随后许多人给予了我持久的友谊与关爱。我还必须在此提及让·勒雷「Jean Leray;Jean Leray」,他在我首次接触“数学家世界”(1948/49年)时给予的友好接待,同样是宝贵的鼓励。我的反思让我意识到对这些“来自另一个世界、拥有另一种命运”的人,我欠下了一份感恩之债。这份债务绝非负担。它的发现带来喜悦,让我感到更加轻松。

\hfill 1984年3月末