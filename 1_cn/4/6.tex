\section{(II) 一种尊重的行为}

(- 5月4日 - … 6月)

\subsection{6. 葬礼}

一个意外事件重启了一场已告终结的反思。它在过去几周内引发了一连串大大小小的发现,逐渐揭开一个原本模糊的局面,并使其轮廓愈发鲜明。这尤其促使我深入且详尽地探讨此前仅略提或暗示的事件与情境。于是,前述(引言,第4节)关于某作品命运的“约十五页回顾性反思”出乎意料地扩展,新增约两百页。

由于事态使然及反思内在逻辑的驱动,我在途中不得不将他人与我自身同样牵涉其中。受牵连最深者(除我之外)是一位与我近二十年友谊相连的人。我曾写道(以委婉之词\footnote{关于此“委婉”之义,见注释“独特的存在”「L'être à part;The Being Apart」,编号57'。}),他在我们这段基于共同热情而深植的友谊最初几年“略似学生”。长久以来,在我内心深处,我视他为我所能带给数学的某种“合法继承人”,超越我已发表的片段性作品。许多人想必已认出他:皮埃尔·德利涅「Pierre Deligne;Pierre Deligne」。

我不为在此公开这些注释——其中包括对一段个人关系的个人反思——而未征询他意见致歉。我认为,将一个长期隐晦且混乱的局面公之于众并加以审视,既重要又有益于所有人。我因此提供了一份见证,虽为主观,且不宣称能穷尽这一微妙复杂的局面,亦不自诩无误。其首要价值(如同我过去的出版物,或我目前致力之作)在于其存在,供感兴趣者取用。我无意说服他人,亦非以所谓“显而易见”之事自保于错误或疑虑之外。我的关切在于真实,在每个瞬间如我所见、所感述说事物——以此深化理解。

“葬礼”「L'Enterrement;The Burial」这一名称,用于指代所有与“过去的重量”「Poids d'un passé;The Weight of a Past」相关的注释,在反思中以愈发强烈的力量浮现\footnote{在反思接近尾声时,另一名称浮现,表达了五周来逐渐展现在我眼前的某画面的另一引人注目的面向。这是一个故事的名称,我将在适当处回溯:“中国皇帝的新衣”「La robe de l'Empereur de Chine;The Robe of the Emperor of China」……}。我扮演了提前殒地者的角色,与几位(远比我年轻)的数学家共处这阴郁的葬礼,他们的作品在我1970年“离去”后成型,带有我的影响印记,体现于某种风格与数学进路。其中最前列的是我的朋友佐格曼·梅布胡特「Zoghman Mebkhout;Zoghman Mebkhout」,他背负着作为“1970年后格罗滕迪克的学生”所面临的一切劣势,却未曾享有与我直接接触、获我鼓励与建议之利,他仅通过我的著作成为我的“学生”。那是一个我(在他所处的世界中)早已被视为“亡者”的时代,以至于长久以来,谋面的念头似乎从未浮现,直到去年,一段持续的(个人与数学的)关系才得以建立。

\footnote{在反思接近尾声时,另一名称浮现,表达了五周来逐渐展现在我眼前的某画面的另一引人注目的面向。这是一个故事的名称,我将在适当处回溯:“中国皇帝的新衣”「La robe de l'Empereur de Chine;The Robe of the Emperor of China」……}

这并未阻止梅布胡特逆专横潮流与前辈(曾为我学生)的轻视,在近乎完全的孤立中,完成原创且深刻的成就。他出人意料地综合了佐藤学派「École de Sato;Sato School」与我的思想,为解析与代数簇的上同调提供了新视角,并预示了对这一上同调理解的大规模革新。无疑,若梅布胡特能在理应支持他的人那里——如他们当年从我处所得——获得热忱接纳与无保留支持,这一革新早已实现且持续多年。至少,自1980年10月起,他的思想与工作为代数簇上同调理论的惊人重启提供了灵感与技术手段,使之(除德利涅围绕威尔猜想「conjectures de Weil;Weil conjectures」的结果外)终结了长期停滞。

令人难以置信却真实的是,近四年来,他的思想与成果被“所有人”使用(如同我的成果一般),而他的名字却被那些熟知其作品并在自身研究中核心利用它的人小心翼翼地忽视与沉默。我不知数学史上是否曾有如此耻辱,当某些最具影响力或声望的从业者在普遍冷漠中,以身作则,蔑视数学职业伦理中最普遍接受的准则。

我看到四位才华横溢的数学家,他们与我一同享有被沉默与轻视葬送的“殊荣”。我在他们每人身上看到轻视之痕,噬咬着曾激励他们的美好热情。

除此之外,我尤见两人,在数学公共广场的聚光灯下主持葬礼,与众多同伴共事,同时(在更隐秘的意义上)以己手埋葬自己,亦埋葬他们蓄意葬送之人。我已提及其中之一。另一位也是我的前学生与旧友,让-路易·韦迪耶「Jean-Louis Verdier;Jean-Louis Verdier」。我1970年“离去”后,除几次匆忙的职业接触外,他与我未保持联系。因此,他在此反思中仅通过其职业生涯的某些行为出现,而这些行为的潜在动机——在其与我的关系层面——未被审视,且我完全无从知晓。

若有一迫切疑问贯穿我多年来的思索,成为《收获与播种》的深层动机,并在此反思中始终伴随,那便是:在一个曾属于我、我在其中作为数学家度过逾二十年并与之认同的世界里,某种精神与风气的兴起使上述耻辱成为可能,而我对此有何责任。反思让我发现,我内心的某些虚荣态度——表现为对能力平庸同事的默然轻视,以及对我自己与某些才华出众数学家的纵容——并非与我今日所见在那些我曾爱过、曾教导我所爱职业之人中蔓延的精神无关;那些我未好好爱护、未好好教导之人,如今在这我曾珍视却已离开的世界中定调(若非立法)。

我感到一阵自满、冷嘲与轻视之风吹过。“它不问‘功’与‘过’,肆意吹拂,以其气息灼烧谦卑的志向与最美的热情……”我明白,这风是我曾盲目且漫不经心播下的种子所结的丰硕果实。若其气息回卷于我,及我曾托付他人之物,及我今日所爱并敢于自称或仅受我启发之人,这乃事物之回返,我无由抱怨,且从中可学甚多。